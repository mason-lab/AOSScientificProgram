\normaltalk{Social Networking of Wintering Chestnut-Collared Longspurs in Oklahoma's Grasslands}{Nuwanthika Perera, John Muller, Jeremy Ross}{Chestnut-collard longspurs (Calcarius ornatus; here after CCLO) is an endemic North American grassland bird that overwinters in Oklahoma. They have shown considerable declines potentially due to habitat loss and/or mortality. Estimates have suggested that CCLO population has significantly declined by more than 87\% since 1960s. According to BBS tend, CCLO are prone to decline in the rate of -4.9\% per year. Objective of this study was to test the hypothesis that CCLO wintering flock dynamics are socially driven, with the prediction that the topology of pairwise association network over space and time would be strongly non-random. On the other hand, null hypothesis is that flocking is resource-driven would hold if individuals in the flock only loosely interacted around clumped resources. The study was carried out in the Wichita Mountains Wildlife Refuge of southwestern Oklahoma in winters of 2018-19 and 2019-20. Birds were captured at four different sites several kilometers apart using mist nets in a novel capture technique. Birds were captured at night when birds were roosting in tall grass and VHF radio-transmitters with unique frequencies were attached. In two field seasons, 44 radio transmitters and 46 radio transmitters, were put out respectively. GPS coordinates were gathered daily for each radio-tagged individual until the tag fell off. Four different areas of aggregation for Chestnut-collared longspurs were observed in the results from 1st field season. These flocking behaviors did not depend on the sex or the age of the birds.}

\normaltalk{Polarization Properties of Solar Panels may Trigger Maladaptive Water-Seeking Behavior in Birds.}{Jackson Barratt Heitmann, Bruce A Robertson}{Birds are attracted to utility-scale solar facilities where they frequently collide with solar panels and die from their injuries.  Solar panels are known to polarize reflected sunlight, much like natural water bodies do. Water is the primary natural source of terrestrial polarized light on earth, and it is possible that solar panels may look like lakes and rivers that birds are attracted to fly toward. Recent and emerging evidence demonstrates that birds are capable of seeing sky polarization patterns and should be pre-adapted to locate water bodies. However, the degree to which solar panels mimic the visual appearance of water may vary as a function of the direction (e.g. afront, behind, left) and the angle at which birds approach them, the weather (e.g. overcast vs. clear), the time of day, and the structural composition of different types of solar panels, themselves. We use imaging polarimetry to examine the visual properties of the most commonly used solar panels types, thin film and polycrystalline, under different lighting conditions and angles of view to map the circumstances under which birds should mistake solar panels for actual water bodies. Polycrystalline panels were stronger polarizers of light across a wider range of lighting conditions and angles than thin-film solar panels in both the visual and ultraviolet ranges. Surprisingly, over the course of each 24 hour day, polarization of panels was maximized at sunrise and sunset for both panel types.Our results suggest that polycrystalline panel solar fields could effectively mimic the appearance of lakes to migrating birds, across a wide range of angles of approach and during time periods throughout each day when sunlight is present.}

\normaltalk{Variation in Crown Feather Reflectance in Two Species of Hybridizing Warblers}{Lesley Bulluck, Valerie Galati, Gunnar Kramer, Henry Streby}{Golden-winged warblers (GWWA) and Blue-winged warblers (BWWA) readily hybridize in regions where their ranges overlap. Hybrids show intermediate plumage and can successfully back-cross with parent species. GWWA populations in the Appalachians have declined significantly in recent decades, while BWWA populations are stable or expanding in this region. Suspected causes of GWWA declines include habitat loss throughout the annual cycle and competition/hybridization with BWWA.  Little is known about the role of ornamentation in avian hybridization dynamics, yet ornaments vary significantly among individuals and can be used to attract mates and/or compete for resources.  The objective of this study was to assess how crown plumage ornamentation (yellow intensity and carotenoid content) varies among males and females of these two species, and whether differences are stronger in sympatry (where species co-occur) or allopatry (species do not co-occur).  We collected crown feathers from 465 individuals at 25 field sites throughout the two species' breeding ranges. We found that GWWA are generally less ornamented than BWWA, and these differences are greatest in GWWA females who have lower carotenoid content and lower yellow intensity than female BWWA.  For males only, we found differences between the two species in allopatric sites, but not in sympatric sites.  A higher degree of ornamentation in BWWA may facilitate hybridization in regions of overlap through social and/or sexual selection.  More ornamented GWWA males may be selected for in sympatry as a result of interspecific competition for mates. More study is needed to test these potential mechanisms.}

\normaltalk{Rufous Hornero (Furnarius Rufus) Actively Anting with a Millipede in Southernmost Brazil}{Amanda P Marcon, João Vitor P Andriola}{Anting is a rarely observed and poorly understood behavior performed by more than 200 species of birds on which individuals actively or passively rub ants or other animals/objects to their feathers, presumably, in order to remove ectoparasites, heal infections or soothe irritated skin during moulting. Millipedes are known to be used for anting, probably due to its secretions, which are filled with chemical substances that may contribute for removing ectoparasites, treating dermal diseases and highlight smelling. Only three records of millipede-anting for two species of Neotropical birds are known, ours being the third species. We observed and described a Rufous Hornero anting with a millipede during 1:40 minutes. It mainly focused on its cloaca while pecking and hitting the millipede on the ground, which was probably a tactic to obtain more fluids. The higher amount of time spent on anting in the cloaca region could be linked to the elimination of bacteria or infections, or to soften dermal irritation, possibly making it more attractive to the opposite sex, as it would seem healthy and clean. Another hypothesis would be the bird was self-stimulating, although most authors do not agree with it. Anting is until this day not well understood by science, and more research as well as detailed observations are encouraged.traduçao pra colocar la tambem: Formicar \'{e} um comportamento raramente observado e pouco compreendido, realizado por mais de 200 esp\'{e}cies de aves em que os indiv\'{i}duos ativa ou passivamente esfregam formigas ou outros animais/objetos em suas penas, presumivelmente, para remover ectoparasitas, curar infeções ou aliviar pele irritada durante a muda. Sabe-se que mil\'{i}pedes são utilizados para formicar, provavelmente devido a suas secreções, repletas de substâncias qu\'{i}micas que podem contribuir para remover ectoparasitas, tratar doenças de pele e acentuar odores. Conhece-se apenas três registros de formicação com mil\'{i}pedes para duas esp\'{e}cies de aves neotropicais, o nosso sendo a terceira esp\'{e}cie. N\'{o}s observamos e descrevemos um joão-de-barro formicando com um mil\'{i}pede durante 1:40 minutos. Ele focou principalmente em sua cloaca enquanto bicava e batia o mil\'{i}pede no solo, provavelmente, como uma t\'{a}tica para obter mais flu\'{i}dos. A maior parte do tempo gasta formicando a região da cloaca pode estar ligada à eliminação de bact\'{e}rias ou infecções, ou para aliviar irritação d\'{e}rmica, possivelmente tornando-o mais atrativo ao sexo oposto, j\'{a} que lhe daria aparência mais saud\'{a}vel e limpa. Outra hip\'{o}tese seria de que a ave estava se auto-estimulando, apesar de a maioria dos autores não concordarem com a mesma. Formicar \'{e}, at\'{e} hoje, pouco compreendido pela ciência, e mais pesquisas, assim como observações detalhadas, são encorajadas.}

\normaltalk{The Lights are On, but is Anyone Home? Assessing Problem-Solving Performance of Wild Hybrid Chickadees in the Field}{Breanna L Bennett, Michael G Rowley, Dovid Y Kozlovsky, Robert L Curry}{In many species, cognitive functions, such as problem-solving, likely influence an individual's ability to survive in the wild. Many captive studies of cognition remove individuals from their natural ecological context, including their immediate social environment, but doing so may impede our ability to extrapolate conclusions to wild populations. This issue applies to investigations of cognitive ecology of hybridizing vertebrates: a recent captive study suggested that interbreeding negatively affects cognitive performance of individual hybrid chickadees--likely caused by poor neural development in birds with admixed genomes--but whether and how wild birds cope with these deficiencies, especially in mixed-species social groups, remains unknown. We used a field-based novel problem-solving test (a lidded cup) baited with high-reward food items (waxworms) to assess individual problem-solving performance in a mixed population of Black-capped, Carolina, and hybrid chickadees in southeast Pennsylvania. We performed problem-solving tests in the field during winter when chickadees gather in mixed-species flocks. We predicted that that hybrids would perform poorly on cognitive tasks in the field, but also hypothesized that hybrids associated with parental-species ``tutors" may use social learning to overcome innate cognitive shortcomings. Preliminary results suggest that hybrid chickadees may not be learning from other chickadees, but may be taking advantage of associations with ``smart" non-chickadee birds in their mixed-species social groups, like Tufted Titmice, by scrounging for food rewards when titmice succeed at a problem-solving task. Further analysis will investigate potential change in individual problem-solving performance of chickadees, with respect to genotype, over the course of the winter season.}

\normaltalk{Predictors of Nest Placement and Location Within the Weaverbirds (Family, Ploceidae) of Awash National Park, Ethiopia}{Noah Burg, Bobby Habig}{The weaverbirds (Family Ploceidae) are a diverse group of small- to medium-sized songbirds distributed throughout sub-Saharan Africa; many are known as virtuosic nest builders. In Ethiopia, several Ploceidae species build nests and mate during the rainy season. Because of various selection pressures, including predation and extreme weather, suitable nest site selection and nest placement are critical for reproductive success. Here, we conduct a comprehensive survey of active nests of weaverbirds in Awash National Park during the rainy season. We identified nests in over 130 trees or bushes across seven weaver species. At each nesting site, we gathered geographical, biotic and abiotic environmental data related to site selection and nest placement. Our goal was to better understand what factors predict the location and placement of weaver nests of different species within the Family Ploceidae.}

\normaltalk{Rain or Shine: The Influences of Temperature and Precipitation on the Frequency and Duration of Avian Parental Care}{Jacob E Morgan, Madeline Sudnick, Kelly A Williams}{As global weather patterns are altered by climate change, many organisms are faced with the task of adapting to greater variation in temperature and precipitation. Individuals may alter their activity patterns to handle changing environmental conditions; however, shifting environmental factors place time budget constraints onto breeding birds, requiring additional or altered parental care effort. For birds raising altricial young, increased variation in weather may impact the time spent performing vital parental care behaviors. Temperature and precipitation limit the time avian parents have available to brood, forage, and provision offspring. We monitored nests of Eastern bluebirds (Sialia sialis) in Athens, Ohio, to determine if precipitation and temperature affect parental activity budgets and offspring care. We quantified provisioning rates and time spent at the nest to test the hypothesis that daily minimum and maximum temperatures and precipitation affected parental care behaviors. Determining how weather variation affects the activity patterns of parent birds and the impact on offspring care allows us to better understand the constraints that climate change places on breeding bird populations.}

\normaltalk{Responses of House Wrens to a Novel Object: Interactions with Fitness-Related Traits and Physiology}{Medhavi Ambardar, Leslie A Watson-Divittore, Chloe M Musgrove, Annie D Hinds}{Individual birds vary in their reactions to their local environment. One way in which birds react to their environment is by being bold. Bold individuals may gain a fitness advantage as they are more likely to explore new territories, seek mates, and defend offspring against predators. Alternatively, boldness may incur costs as bold individuals may engage in more risky behaviors. Boldness occurs along a gradient that varies from very bold individuals to shy (non-bold) individuals. Physiological mechanisms, such as secretion of corticosterone, a metabolic hormone involved with the stress response, may interact with boldness or shyness. Using House Wrens (Troglodytes aedon), we tested the hypothesis that boldness increases fitness. We also assessed the relationship between boldness and corticosterone levels. We used the latency of adult wrens to approach a novel object as a proxy for boldness. When nestlings were 7-9 days old, we placed a novel object on top of the nest box. We then recorded the latency to land on the nest box and the latency to enter the nest box for both males and females over a total trial period of 30 minutes. We ran a control trial during which no object was placed on the nest box. We compared male and female responses to the novel object with nestling mass at fledging, and the number of offspring fledged. Finally, we compared adult responses to the novel object to baseline and stress-induced corticosterone levels. Our findings will be discussed in relation to evolutionary consequences of bold behavior.}

\normaltalk{Nest Site Selection and Nest Predation of Louisiana Waterthrush (Parkesia Motacilla) in Central Tennessee}{Sylvia Powell, Stefan Woltmann}{Predation risk can influence avian nest site selection. In central Tennessee, we noticed Louisiana Waterthrush (Parkesia motacilla) foraging and carrying food or nesting material from 2nd order permanently flowing stream channels (main stems) to smaller 1st-order often-ephemeral streams (smaller tributaries). We asked if smaller tributaries provide nests with a lower predation risk. During the 2019 breeding season, we used an artificial nest experiment with (Coturnix chinensis) eggs to test for differences in predation likelihood in smaller tributaries versus main stems. We also compared habitat characteristics of 24 natural nests in 2018-2019 to paired random sites. Camera traps ("game cams") at half of the artificial nests and 14 of 17 active natural nests helped us to document predation events and identify nest predators. Predation was relatively low at both artificial (11/48 predated) and active natural (3/17 predated and 2/17 abandoned) nests. We found no difference between predation likelihood of artificial nests along main stems versus smaller tributaries. However, natural nest position was not random within territories; nests were on steeper streambanks and were more concealed (MANOVA: F8, 39 = 7.84, P $<$ 0.01). We discuss how within-territory nest site selection may influence nest survival.}

\normaltalk{Assessing the Effects of Individual Traits on Boldness in a Wild Population of Song Sparrows (Melospiza Melodia)}{Theresa Higgins, Rebecca Wilcox, Ryan R Germain, Peter Arcese, Corey Tarwater}{Animal populations are threatened by climate change, habitat loss, and invasive species. Small, isolated populations are particularly vulnerable to changes in the environment. Recent research has highlighted the importance of animal behavior for species conservation, and in particular, the behavioral traits of aggression, neophobia, and boldness. Greater flexibility in these traits within a population is predicted to allow for populations to respond to changing environments. However, it is often unclear what individual traits (i.e. age and sex) are linked to these behaviors. Identifying these traits is an important step to understand how these behaviors scale up to impact demography and species persistence. Here we measured boldness in an isolated population of song sparrows (Melospiza melodia) on Mandarte Island, Canada, to determine which individual traits (i.e., age and sex) explained variation in boldness. To address this question, we performed a series of startle tests with banded individuals in the winter of 2013 and examined latency to return. First, we found that juveniles were bolder than adults, suggesting that wariness may increase with experience. Second, we found that there was no difference in boldness between males and females. This result is surprising given that the males are known to defend territories. This study shows the importance of individual traits on behaviors and future research is essential to fully understand the impact of these traits on the behavioral plasticity of a population.}

\normaltalk{Testing the Challenge Hypothesis in a Tropical Polygynous Bird, the Wire-Tailed Manakin (Pipra Filicauda)}{Camilo Alfonso, Ignacio Moore}{The Challenge Hypothesis predicts seasonal patterns of testosterone in male vertebrates. Circulating levels of testosterone should be elevated as long as fertile females are present and should decrease when males care for young as testosterone suppresses paternal care in favor of mating behavior. Therefore, for males of polygynous species that do not exhibit parental care, the prediction is that plasma testosterone levels increase in the pre-breeding season and remain elevated throughout the breeding season.  To test this prediction, we compared testosterone responsiveness in the wire-tailed manakin (Pipra filicauda), a polygynous species in which the male provides no care for offspring. We compared breeding baseline levels of testosterone with maximum physiological levels obtained by challenging males with an intramuscular injection of gonadotropin-release hormone (GnRH). Previous studies in the wire-tailed manakin have shown that testosterone levels are dependent on social status, with territory-holding males have higher testosterone levels than floater males. We conducted our study on both territory holding and floater males to test the testosterone responsiveness across social classes.}

\normaltalk{Revealing Hummingbirds' Aggressive Behavior in Feeders: a Case Study in Colombia}{Nicolas Tellez, Alejandro Rico, F G Stiles}{There is a plethora of complex interactions among members of a biological community, within the same foraging guild competition is often the most common interaction. In hummingbirds (Trochilidae) agonistic interactions are evident, and it is important to know how factors such as morphology, food quality, and environmental variables, regulate aggressions and territoriality. We performed experiments and high-speed video recordings in artificial feeders at the Finca Colibr\'{i}-Gorriazul Research Station (Fusagasug\'{a}, Colombia), in order to evaluate the incidence of a set of variables on agonistic encounters among hummingbirds. A total of 20145 interactions were recorded in a 198-hour sample. We found differences regarding the size and identity of the individuals involved: smaller species/sexes were confronted more times, while the heavier ones presented higher dominance and received less aggressions. We found that the environmental variables measured did not have a considerable influence on competition, with the exception of rainfall, which was related to the intensity and outcome of the attacks. Time of day was also an important factor: despite the crepuscular higher activity pattern that has been reported in natural conditions, our investigation indicates that the amount of aggressions was not strictly greater at the beginning and at the end of the day. This study is an approach to understand hummingbird behavior around feeders, and it is the first of its kind in Colombia. However, it is necessary to collect more data on further questions generated from this study, due to the simultaneous influence of many variables in the interactions of these birds.}

\normaltalk{Lesser Prairie-Chicken Lek Formation, Lek Persistence, and Female Space Use}{Carly Aulicky, David Haukos}{Lek breeding species are characterized by differences in paternal care and corresponding difference in habitat use by sex. Male lesser prairie-chickens (Tympanuchus pallidicinctus) form leks in areas of short vegetation and females prioritize vegetation structure with cover for nests and broods. Lesser prairie-chicken management relies on lek counts to estimate population size. Conservation strategies assume leks form according to the hotspot hypothesis in areas of female home range overlap. The density of leks on the landscape is dynamic, changing between breeding seasons and even within a breeding season. Short grass suitable for lek formation is also not limiting, so how do males select where to form leks and what determines if a lek persists into subsequent breeding season? We tested the hotspot hypothesis and lek persistence using 53 lek locations across three Kansas field sites and 165 estimated female lesser prairie-chicken home ranges. We examined shifts in female space use from GPS telemetry point locations with an optimized hotspot analysis by study year as a test of the hotspot hypothesis We tested female density, surrounding nesting and brooding habitat, and changes to vegetation at lek sites to determine what factors influence the persistence of a lek from one breeding season to the next. Our results indicate that leks form in locations with high female home concentration consistent with the hotspot hypothesis and that lek persistence is determined by grassland at broad scales, visual obstruction at the lek scale, and female space use.}

\normaltalk{Telomere Dynamics in a Free-Living Population of a Neotropical Lekking Passerine}{Meredith R Kuzel, Emily H DuVal}{The duration and timing of key life events differ between conspecifics, but it remains unclear why variation persists despite consequences for fitness, survival, and reproductive success. A factor which may maintain this variation is longevity: social environment, settlement decisions, and mating strategy may be influenced by an animal's lifespan. Telomeres are repetitive sequences of DNA capping the ends of chromosomes to protect the vital majority of DNA during cell replication, and are hypothesized to relate to variation in key life history traits. This project investigates telomere length (TL) determination in a population of lance-tailed manakins (Chiroxiphia lanceolata) on Isla Boca Brava, Chiriqu\'{i}, Panama. In this species, early-life TL correlate with variation in timing of reproduction, but it is unclear what generates this early-life variation in TL. We investigated underlying causes of early-life TL variation, testing whether offspring TL is predicted by parent TL or by parental age. We quantified average TL of chicks aged 1-2 days and compare them to (1) the age of the mother and father at hatch (n = 45 chicks, from 34 unique mother and father pairs), and (2) the TL of the mother (n = 42) or father (n = 40) at hatch. If TL is heritable, chick TL will be predicted by parental TL and highly correlated with sibling TL. If TL is determined by behavior, parental age will predict initial TL in chicks. Understanding the factors that relate to early-life telomere variation will help identify the mechanisms underlying life history variation.}

\normaltalk{How Long Do Maternal Effects Last? a Study of Maternal Hormonal Investment in Quail Embryos}{Sarah K Winnicki, Thomas J Benson, Mark E Hauber, Ryan T Paitz}{Variation in maternal investment of nutrients and hormones into eggs can result in variation in embryonic and hatchling growth rates and developmental tactics. However, much of the prior work on maternal effects focused on investment levels of single steroid hormones and measured only concentrations at the onset of incubation. We hypothesized that eggs with different initial concentrations of multiple steroid hormones would maintain those relative levels throughout embryonic development, producing different-sized offspring. We incubated eggs from two sources of wildtype Japanese Quail (Coturnix japonica). We sacrificed eggs with embryos every three days from day 0--15 and measured, when possible, a suite of morphological traits of size (N=3--6 embryos/line). At these time points we also assessed the concentrations of 29 yolk steroid hormones (N=5 eggs/line) using liquid chromatography tandem mass spectrometry (LC-MS-MS). Only 9 steroids were present above threshold in quail yolk, including androgens and progestogens but no glucocorticoids. The two lines differed in their egg volume, initial yolk weight, and hormone concentrations but not in the total amount of each steroid deposited into the yolk. Line-specific differences in initial hormone concentrations disappeared by day three and we observed no differences in embryonic growth metrics between the lines. Focusing on initial maternal investment without accounting for changes in hormone concentration over ontogeny in ovo would fail to explain the observed growth patterns; future research should account for the rapid embryonic metabolism of maternally invested factors following the onset of incubation.}

\normaltalk{What Predicts Torpor in Hummingbirds? an Integrative Examination of Torpor Across the Phylogeny, Landscape, and Individual}{Austin R Spence, Spencer B Hudson, Emily E Virgin, Susannah S French, Morgan W Tingley}{Hummingbird torpor is an energy saving mechanism used during the night by lowering metabolic rates and body temperature. While we have known of torpor for decades, previous work has usually been performed on few individuals per species. To better understand torpor in hummingbirds, we investigated this physiological process across three biological levels. First, to understand the extent of torpor use across the phylogeny, we conducted a literature review to find all known instances in torpor use across the Trochilidae family. We found that only 40 species of 360 (11\%) have been tested for torpor. Of these species, larger hummingbirds are less likely to use torpor than their smaller counterparts. Second, to understand how torpor is used across the landscape, we measured nighttime metabolic rates of approximately 150 Anna's and calliope hummingbirds across an elevational gradient. While both species used torpor often ($>$50\% of nights), Anna's hummingbirds used it more often in colder temperatures and later in the season. Finally, to investigate the underlying physiology of torpor in an individual, we measured how a suite of hormones changed during different portions of the torpor bout. We found that corticosterone, testosterone, and immune function did not change through the night, but female hummingbirds that used torpor had higher estradiol than female hummingbirds that remained normothermic. Using an integrative approach allows us to examine torpor use in context of the phylogeny and species traits and help predict under which environmental conditions torpor will be used in natural conditions.}

\normaltalk{Interactions Between Stress Hormones and Blood Parasites along an Elevation Gradient}{Viridiana Martinez, Jacquelyn K Grace}{Parasite infection rates typically decrease along elevation gradients, in what are known as `elevation refugia' from disease. While much is known regarding environmental factors that influence parasite abundance/transmission across elevations, little research has focused on potential intrinsic physiological factors in birds. Glucocorticoids are one of these potential factors.  Circulating glucocorticoid concentration often also vary along elevation gradients in birds, and are known to affect the immune system, but how they influence parasitism rates of birds is unclear. This study investigates the interactions between avian glucocorticoids, parasite prevalence, and habitat elevation in wild birds to improve our understanding of the mechanisms underpinning elevation refugia from disease, with implications for disease transmission and species survival under climate change scenarios. We examined the effects of elevation on glucocorticoid levels (i.e., corticosterone in blood and feathers) and parasite presence in avian communities across various elevations in west Texas during the summer months of 2018 and 2019.}

\normaltalk{A Retrospective Analysis of Methylmercury in the Feathers of Indonesian Avifauna}{Sarah A Dzielski, Dewi M Prawiradilaga, N R Razavi}{Artisanal and small-scale gold mining (ASGM) is the largest anthropogenic source of mercury globally, though few studies have explored how the release of this toxicant into the environment is affecting avian wildlife in Indonesia, an ASGM hotspot. In this study, feather samples from Indonesian museum specimens were used to examine changes in methylmercury (MeHg) and diet through time. The following questions were assessed: 1. Do concentrations of MeHg in Indonesian bird specimens increase proportionally with the estimated doubling of Hg emissions from ASGM during the last two decades as described by the United Nations Environmental Programme? 2. Do long term trends in MeHg through time in Indonesian bird specimens parallel global trends in Hg emissions, or are they more closely tied to local trends in ASGM Hg emissions?  and 3. How does foraging guild and within-species variation in diet influence MeHg concentrations in Indonesian bird specimens? Feather samples from seven species collected between 1860 and 2019 were used to address these questions. This study provides critical data that is currently lacking on avian Hg exposure in a global avian hotspot severely impacted by Hg pollution.}

\normaltalk{Changes in Avian Mercury Concentrations Through Time: A Scoping Review and Meta-Analysis}{Sarah A Dzielski, N R Razavi}{Long-term avian studies worldwide have reported changes through time in mercury in avian tissues. These trends in tissues often mirror trends in global emissions of mercury through time, though many factors such as diet and local emissions trends are known to influence temporal patterns in some avian populations. While numerous studies have examined trends through time in avian tissues, no review exists outlining what patterns emerge through time across taxa with different life histories, diets, and distributions. In this review and meta-analysis, we will assess the following questions: 1. How do mercury trends through time in birds vary spatially (e.g., global versus local emissions trends, hemisphere, ecosystem type)?, 2. How are mercury trends through time influenced by diet (i.e., shifts in stable isotope ratios of carbon and nitrogen)?, 3. How do mercury trends through time vary by age, sex, and life history traits (e.g. molt strategy, reproductive strategy, mercury excretion methods)?, and 4. How do mercury concentrations vary through time within a species or individual? This study will provide a summary of trends through time in avian tissues and identify key factors influencing the accumulation of mercury through time.}

\normaltalk{Examining the Additive Burdens of Pesticide Exposure and Malaria Infection on Spring Departure in a Migratory Songbird}{Allison J Byrd, Daniel Becker, Katie Talbott, Benjamin Higgins, Tara Smiley, Ellen Ketterson}{Bird migrations are shifting globally due to climate change, habitat loss, and fluctuating seasonal cues. The timing of departure for avian spring migration is a complex process mediated by many internal and external factors, including exposure to pesticides. Neurotoxic insecticides (neonicotinoids) are consumed via foraging at pesticide-treated fields or bird feeders supplied with contaminated commercial-grade seed.  Additionally, warmer winters are increasing the abundance of arthropods that are vectors of haemosporidian parasites, which cause avian malaria. Both neonicotinoid exposure and haemosporidian infection can reduce body fat stores and delay migration in songbirds. Dark-eyed juncos (Junco hyemalis) are a model species to study neonicotinoid exposure, as they forage in pesticide-treated agricultural areas and bird feeders. Avian malaria is also common in juncos, with some populations having an infection prevalence $>$ 50\%. We studied the effects of naturally-occurring haemosporidian infection coupled with an ecologically-relevant dose of neonicotinoids on the body mass and timing of migratory departure on male juncos using a fully factorial design. We determined migratory departure using nanotags and controlled for variation in breeding latitude using stable isotope analyses of hydrogen from feather samples. We hypothesize that juncos naturally infected with haemosporidians and exposed to neonicotinoids will show a significant reduction in body mass and delay in migratory departure compared to controls.}

\normaltalk{Oxidative Stress in the Seaside Sparrow (Ammospiza Maritima) Following the Deepwater Horizon Oil Spill}{Aaron N Angel, Sabrina S Taylor, Philip C Stouffer, Andrea Bonisoli-Alquati, Juanita K Jellyman}{The Deepwater Horizon (DWH) oil spill introduced massive amounts of oil into Louisiana saltmarshes. Carbon isotopic and molecular analyses showed Seaside Sparrows (Ammospiza maritima) incorporated oil through their diet. Polycyclic aromatic hydrocarbons (PAHs) are toxic components of oil that generate reactive oxygen species, inducing oxidative stress when the antioxidant capacity of the organism is overwhelmed. Our goal is to determine whether exposure increased oxidative stress in Seaside Sparrows. We measured plasma concentration of reactive oxygen metabolites (ROMs) and plasma non-enzymatic antioxidant capacity in birds from oiled and control sites (2013-2015). Antioxidant capacity was higher in exposed sparrows compared to controls in 2013. Antioxidant capacity in exposed sparrows was also higher in 2015 compared with 2013. Additionally, ROMs concentration was higher in exposed sparrows compared to controls in 2014. Our results suggest that exposed sparrows mobilized antioxidants from storage tissues to combat oxidative damage from chronic exposure to PAHs. This mobilization was stronger in earlier years, and seemingly effective in preventing oxidative damage. However, mobilization of antioxidants may trade off with the ability to provision offspring or fight oxidative stress or inflammation in other tissues. To test this, we are also measuring oxidative damage to proteins in the heart tissue using PAGE-Western blot. Our project will clarify fitness consequences for this vulnerable terrestrial species and will aid conservation planning by providing crucial information on chronic exposure effects. Our findings will also be applicable to other terrestrial bird species exposed to spilled oil.}

\normaltalk{Relationships Between Lead Toxicity and Oxidative Stress in Migrant and Resident Populations of Turkey Vulture (Cathartes Aura)}{Alexandra M Gresham, Miguel D Saggese, Peter H Bloom, Andrea Bonisoli-Alquati}{Migration is energetically costly, often leaving birds in a deficit for fat stores and antioxidants. The Turkey Vulture (Cathartes aura) is an obligate scavenger that ranges from Canada to South America, and has migrant and resident populations. Turkey Vultures are at risk of lead (Pb) toxicosis from the ingestion of Pb ammunition embedded in carcasses left behind by hunters. The State of California recently banned Pb ammunition for hunting purposes. Yet, migratory populations of Turkey Vultures are not necessarily protected due to potential Pb exposure outside California. Additionally, oxidative stress from costly migratory behavior might make migrating birds even more susceptible to Pb toxicosis. To study how Pb exposure and effects vary with migratory behavior, we are capturing vultures from migratory and resident populations in Southern California, and quantifying blood Pb levels and plasma markers of oxidative stress. We expect migratory birds who travel outside of California to have (I) higher blood Pb levels, and (II) higher oxidative stress because of the metabolic cost of migration. Additionally, we expect (III) vultures with high blood Pb levels will have the highest oxidative stress, because limited antioxidant resources may be allocated to combat either Pb toxicosis or the metabolic results of flight, but hardly both. We expect our study using Turkey Vultures as sentinels of Pb toxicity will show that local environmental measures inadequately address the conservation needs of long-distance migrants, and support a need for a flyway approach to conservation.}

\normaltalk{Does Range Expansion or Ongoing Dispersal Explain Apparent Panmixia in Bachman's Sparrow? a Temporal Study using Museum DNA}{Amie E Settlecowski, Brant C Faircloth, Jeremy M Brown, James A Cox, James W Tucker, John P Carpenter, Sabrina S Taylor}{Bachman's Sparrow (Peucaea aestivalis) is uncommon throughout its wide distribution across the southeastern US, primarily persisting in declining populations in remnant fragments of pine-dominated forest. Despite these declines and the extreme loss and fragmentation of its habitat, a previous study indicated Bachman's Sparrow is panmictic across its distribution. Eighteen microsatellite and two mitochondrial loci showed limited genetic differentiation even among distantly separated populations, such as their southern limit in Florida and west of the Mississippi River in Louisiana. However, the extensive contemporary gene flow necessary for connectivity among such distant populations is improbable in Bachman's Sparrow, given the extent of loss and fragmentation of their habitat. Instead, apparent panmixia may be a signature of past gene flow resulting from a recent range expansion and subsequent retraction. To evaluate this hypothesis, we are reassessing Bachman's Sparrow population structure with time-stratified genetic data from five regions and four time periods: (i) before and (ii) during range expansion, (iii) during retraction, and (iv) modern day. If the panmictic signal is the result of gene flow associated with the range expansion, the five sampled regions should show differences in population structure and connectivity before and after the range expansion. Alternatively, if dispersal ability maintains connectivity among Bachman's Sparrow populations, populations should maintain a consistent pattern of structure and connectivity through time. We collected 2,944 single nucleotide polymorphisms from 318 modern blood samples and 282 museum toepads to distinguish between these scenarios.}

\normaltalk{Population Genetic Structure and Genetic Diversity of Toll-Like Receptors Genes in the Endangered Sierra Madre Sparrow}{Jos\'{e} G Ham-Due\~{n}as, Ricardo Canales-del-Castillo, Gary Voelker, Irene Ruvalcaba-Ortega, Carlos E Aguirre-Calder\'{o}n, Jose I Gonz\'{a}lez-Rojas}{Toll-like receptors (TLRs) are innate immune genes related to pathogen resistance, and polymorphisms that may reflect not only levels of functional diversity, but may also be used to assess genetic diversity within and among populations, especially in species that are of conservation concern. Here, we combined four putative adaptive markers (TLRs) with one mitochondrial (COI) marker to evaluate genetic variation in the endangered Sierra Madre Sparrow (Xenospiza baileyi). This species offers an ideal model to investigate population and evolutionary genetic processes that may be occurring in a habitat restricted endangered species with disjunct populations (Mexico City and Durango). TLRs diversity in the Sierra Madre Sparrow was relatively high, which was not expected given its two small, geographically isolated populations. Genetic diversity was different, but not significantly, between the two populations, with less diversity seen in the smaller Durango population. Population genetic structure between areas was due to isolation and different selective forces acting on different TLRs; moreover, population structure was also evident in COI. Our findings offer the first avian-based evidence of population genetic structure in TLRs for a mainland system, which in turn suggests signals of local adaptation. The reduction of genetic diversity in COI was observed over 20 years in the Durango population, a result likely caused by habitat loss. Our results provide information related to the ways in which genetic variation can be altered by demographic changes due to human-mediated habitat alterations. Furthermore, our findings may help to guide conservation schemes for both populations and their habitat.Los receptores tipo Toll (TLRs) son genes inmunes innatos relacionados con la resistencia a pat\'{o}genos y polimorfismos que pueden reflejar no solo los niveles de diversidad funcional, sino que tambi\'{e}n pueden usarse para evaluar la diversidad gen\'{e}tica dentro y entre las poblaciones, especialmente en especies prioritarias para la conservaci\'{o}n. Aqu\'{i}, combinamos cuatro marcadores adaptativos putativos (TLRs) con un marcador mitocondrial (COI) para evaluar la variaci\'{o}n gen\'{e}tica en el Gorri\'{o}n serrano (Xenospiza baileyi) que se encuentra en peligro de extinci\'{o}n. Esta especie ofrece un modelo ideal para investigar los procesos de gen\'{e}tica de poblaciones y evolutivos que pueden estar ocurriendo en un h\'{a}bitat restringido y con poblaciones disjuntas (Ciudad de M\'{e}xico y Durango). La diversidad de TLRs en el Gorri\'{o}n serrano fue relativamente alta, lo cual no se esperaba debido a sus dos peque\~{n}as poblaciones geogr\'{a}ficamente aisladas. La diversidad gen\'{e}tica fue diferente, pero no significativa, entre las dos poblaciones, con una menor diversidad en la poblaci\'{o}n m\'{a}s peque\~{n}a de Durango. La estructura gen\'{e}tica poblacional entre \'{a}reas se debi\'{o} al aislamiento y a las diferentes fuerzas selectivas que actúan sobre diferentes TLR; Adem\'{a}s, la estructura de la poblaci\'{o}n tambi\'{e}n era evidente en COI. Nuestros hallazgos ofrecen la primera evidencia de estructura gen\'{e}tica poblacional en TLRs basada en aves con distribuci\'{o}n continental, lo que a su vez sugiere se\~{n}ales de adaptaci\'{o}n local. Se observa una reducci\'{o}n de la diversidad gen\'{e}tica en COI, en la poblaci\'{o}n de Durango, en un periodo de 20 a\~{n}os, un resultado probablemente causado por la p\'{e}rdida de h\'{a}bitat. Nuestros resultados proporcionan informaci\'{o}n relacionada con la manera en que la variaci\'{o}n gen\'{e}tica puede ser alterada por los cambios demogr\'{a}ficos, debido a las alteraciones del h\'{a}bitat mediadas por el ser humano. Adem\'{a}s, nuestros hallazgos pueden ayudar a guiar esquemas de conservaci\'{o}n, tanto para las poblaciones como para su h\'{a}bitat.}

\normaltalk{Overland Spring Migration of Common Loons (Gavia Immer) in Florida.}{Andrew W Kratter, Paul R Spitzer}{The Gulf of Mexico harbors a significant population of wintering Common Loons in eastern North America, an iconic species of northern lakes in Canada and the northern Unites States.  Although loons usually eschew migratory routes over land, recent studies in Florida have identified two significant overland pathways during spring migration for loons heading north toward their summering grounds: straight north from the Big Bend region of the Florida panhandle toward the Great Lakes region, and a cross-peninsular flight starting near Cedar Key, Florida, and reaching the Atlantic near Jacksonville. Satellite telemetry has revealed that the cross-peninsular route may be particularly important for young birds, before they reach breeding age at ca. 5 years; many of these birds summer off the coast of eastern Canada.  Loons using both routes are concentrated geographically (by coastal geometry) and temporally (by beginning migration at first light).  As a result, surveys of migrating loons at key sites (St. Marks NWR on the Florida Panhandle and Alachua County in the north-central Florida Peninsula) can capture a significant proportion of loons using these two migration routes.  With minimal efforts, previous surveys at these sites have identified annual numbers in excess of 3200 birds at St. Mark's and 850 birds at Alachua County. In 2020, networks of observers, recruited through citizen science projects, will be strategically placed along both routes to better quantify the scope and importance of these migratory routes.}

\normaltalk{Migratory bird directional movements near the southwestern shoreline of Lake Erie}{Michael J Wellik, Eileen M Kirsch}{Night migrating birds in the vicinity of the Great Lakes may fly over or around these large bodies of water. Their flight path is influenced by many things: weather, time of night, body condition, etc. It is important to document bird movement at different altitudes in relation to Great lakes shorelines and discover factors associated with over-lake versus shoreline flight to assist planning of near and offshore development. The shoreline of Lake Erie, with it's east-west orientation, is thought to concentrate birds during spring and fall migration, and a long -term banding dataset collected by the Black Swamp Bird Observatory supports this notion.  We used X-band marine radars with parabolic antennas, operating at a slant range of 1.5km, to sample nightly migration at 2 sites along the southwestern shoreline of Lake Erie and at 2 sites 4.8km inland, for two years during spring and fall. We developed individual flight track profiles of presumed birds between 65 and 720m above ground level using radR (R software). We used circular statistics to compare flight directions at shoreline versus inland sites during spring and fall. We will present these results on flight direction patterns for presumed birds flying at different times of night.}

\normaltalk{Interannual Variability in Energetic Condition and Recruitment of Semipalmated Sandpiper in Northeastern Venezuela}{Gianco Angelozzi-Blanco, Andreina L\'{o}pez-Marcos, Dylan Garc\'{i}a, Gabriel Le\'{o}n}{The study of shorebird body condition at stopover and wintering sites allows to infer about their migratory capacity, habitat quality and to evaluate population level effects of environmental changes on the breeding grounds and other stopover areas used on migration. Semipalmated Sandpiper is the most abundant shorebird on the coastal wetlands in Isla de Margarita, Venezuela, and their arrival during fall migration match peak invertebrate productivity on these wetlands. In this study, carried out at La Acequia coastal lagoon, Isla de Margarita, we mist-netted 187 birds during the nonbreeding season, over two consecutive periods (2017-2018 and 2018-2019). Age, wing length and weight was recorded. Size-adjusted mass (SAM) was derived from measurements and we analyzed the effects of period (2017-2018 and 2018-2019) and non-breeding cycle phase (arrival: August-September, wintering: October-January, and departure: February-April) on the SAM. The juvenile-to-adult ratio differed between periods, being higher in the first compared to the latter (80:8 in 2017-2018 vs. 13:84 in 2018-2019). SAM variation was explained the interaction of period and nonbreeding cycle phases. Juveniles outnumbered adults and had a better condition during the arrival phase in 2017-2018 compared to 2018-2019, when an opposite pattern was observed. Despite the initial differences, SAM increases during wintering and departure phases were consistently similar between periods. Our results suggest some interannual flexibility in stopover decision of Semipalmated Sandpipers, highlighting the role of stopover and wintering sites in limiting possible carry-over effects and showing some recruitment fluctuations that might be linked to extreme climatic events on the breeding grounds.El estudio de la condici\'{o}n corporal de aves playeras migratorias en sitios de parada e hibernada permite inferir sobre su capacidad migratoria, la calidad del h\'{a}bitat y detectar los efectos de cambios ambientales en las \'{a}reas reproductivas y otros sitios de parada usados en la migraci\'{o}n sobre las poblaciones. Calidris pusilla es el playero m\'{a}s abundante en las albuferas de la Isla de Margarita, y su arribo durante la migraci\'{o}n post-reproductiva coincide con el pico de productividad de esos humedales. En este estudio, realizado en la albufera la Acequia, capturamos 187 individuos de C. pusilla y determinamos, edad, longitud del ala y peso. Con estas medidas, derivamos el \'{i}ndice de masa corporal ajustada (IMC) y analizamos los efectos del periodo (2017-2018 y 2018-2019) y la fase del ciclo post-reproductivo (llegada: agosto-septiembre, invernada: octubre-enero, y pre-migratoria: febrero-abril) sobre sus variaciones. La proporci\'{o}n juveniles:adultos difiri\'{o} entre periodos (80:8 en 2017-2018 vs. 13:84 en 2018-2019). Las variaciones del IMC respondieron a la interacci\'{o}n entre el periodo y la fase del ciclo post-reproductivo. En este sentido, los juveniles fueron m\'{a}s abundantes y tuvieron mejor condici\'{o}n en la llegada de 2017-2018; mientras en 2018-2019, el patr\'{o}n fue opuesto. A pesar de las diferencias iniciales, los incrementos del IMC en las fases de invernada y pre-migratoria fueron similares entre periodos. Nuestros resultados sugieren cierta flexibilidad interanual en las decisiones, resaltando la importancia de los sitios tropicales para limitar efectos de arrastre y revelando fluctuaciones en el reclutamiento que podr\'{i}an vincularse a eventos clim\'{a}ticos en el \'{a}rtico.}

\normaltalk{Tracking Tiny Migrants: First Description of the Annual Cycle of Tropeiro Seedeater (Sporophila Beltoni)}{M\'{a}rcio Repenning, Alex E Jahn, Michael T Hallworth, Carla S Fontana}{Neotropical seedeaters are emblematic of South America's threatened grassland species. The Tropeiro Seedeater (Sporophila beltoni) is a newly described long distance migrant species that breeds at south temperate latitudes and overwinters at tropical latitudes of South America. There is a large gap in our knowledge about its life history during winter, and about its annual migratory routes. To address this, we tracked two males throughout a complete migration cycle using miniature light-level geolocators. The first-year and adult males departed from the breeding site in southern Brazil on March 5 and 12, respectively, migrating northwards on fall migration for 14 and 28 days, respectively to overwinter in the Cerrado biome, approximately 1,200 km from their breeding site. The adult and first-year males stayed in the wintering area for 195 and 218 days and used 4 and 5 wintering sites, respectively. Both males started spring migration the following October, which lasted about 40 days. The younger male made three stopovers during the journey while the adult male used no stopovers traveling 985 km from the last winter site until arriving near to the breeding site. Hence, the first-year male arrived at the breeding site almost one month after the adult. The males made a round-trip of 3,200 and 3,585 kilometers, respectively. Our findings provide the first description of long-distance movements of a Neotropical seedeater throughout its full annual cycle and highlight the importance of using modern technologies to understand the full annual cycle of small migratory songbirds.}

\normaltalk{Accipiter Hawks of the Laurentian Upland and the Interior Plains Undertake the Longest Migrations: Insights from Birds Banded or Recovered in Veracruz}{Enya A C\'{o}rdoba-Cuevas, Sara Ibarra-Zavaleta, Ernesto Ruelas Inzunza}{Bird banding has allowed us to understand diverse aspects of the life history of migratory birds. In this study, we describe the migratory movements of Sharp-shinned Hawks (Accipiter striatus) and Cooper's Hawks (Accipiter cooperii) based on banding and recapture records of birds that migrate through Veracruz, Mexico. Our aim was to determine their breeding, migration, and non-breeding localities, estimate their migration distance, and contribute to the understanding of their migration patterns. With a total of 80 records, we calculated migration distances, made Kernel Density Estimation analyses to identify the regions that aggregate recaptures by season, and analyzed the migratory patterns of these hawks. The distribution of recaptures largely coincides with breeding localities in the Laurentian Upland and the Interior Plains physiographic regions. All migration records follow a trajectory that extends from the United States Midwest to the Gulf coastal plain of Mexico. The mean breeding season migration distance to Veracruz for Sharp- shinned Hawks is 3,374 km (a difference of 27 degrees of latitude), whereas that for the Cooper's Hawks is 2,926 km (a difference of 25 degrees of latitude). Our non-breeding records indicate the populations of Accipiter hawks of central-northern North America migrate the longest distances to reach Central America, the southernmost distribution of their migratory populations. Our results support existing research of how continental physiography determines migration routes, migratory behavior, and migratory connectivity of these hawks.}

\normaltalk{Diversity, Abundance and Avian  Health of Migratory Grassland Birds in Estaci\'{o}n Salado, Vanegas, San Luis Potos\'{i}, M\'{e}xico}{Erendira del Rocio Vargas Facundo, Leonardo Chapa Vargas}{Bird populations that depend on grasslands habitats are declining in North America, the alteration of the winter habitat in Mexico for migratory grassland birds is an important component in the change of land uses, specifically the effects of grazing on grassland birds generally they are considered indirect, since these are manifested through the alteration of the vegetation structure. Vegetation changes created by livestock activity influence wildlife species through the quality and quantity of food, food availability, protection from predators. This study aims is estimate the diversity and abundance of migratory grassland birds in the Estaci\'{o}n Salado, Vanegas, San Luis Potos\'{i}, M\'{e}xico, during the winter season. We install mist nest in areas with livestock activity and livestock exclusi\'{o}n to capture birds on this h\'{a}bitat along the winter season, as well as knowing the avian state of health of birds we estimate the heterophile / lymphocyte index and mass scaling index on site with livestock activity and livestock exclusion. Likewise, studies were carried out on the composition and structure of the vegetation at the study site in order to compare the structure of the vegetation with the diversity of bird species at the sites of study. We obtein a total of 300 bird catches during the winter period, of which 150 correspond to migratory grasslnad birds, within the results obtained it can be highlighted that there is greater diversity and abundance of migratory grassland birds in sites with excluding livestock and that at their its related to vegetation structure.}

\normaltalk{Understanding Habitat Quality Drivers for Migratory Songbirds in a Coffee Production Landscape}{Fabiola Rodr\'{i}guez, David I King, Dar\'{i}o Alvarado, David Murillo, Jeffery L Larkin, Caz Taylor}{Preserving winter habitat in Mesoamerica is critical to the conservation of migratory birds and such habitat is increasingly located in working landscapes rather than in protected parks. We studied habitat quality drivers for two Neotropical migratory bird species, Wilson's Warbler (Cardellina pusilla) and Wood Thrush (Hylocichla mustelina), in a coffee production landscape of Honduras during two overwintering seasons. We conducted mist-netting to measure age and body condition and estimate how variation in body condition is driven by date, coffee production (cover) type, and elevation. During our second season, we conducted a mark-resight study for Wilson's Warblers for apparent survival. In total we caught 96 Wilson's Warbler (56 juveniles, 17 adult, 23 unknown age) and 73 Wood Thrush (38 juveniles, 14 adults, 21 unknown age). Wilson's Warbler fat scores increased as the winter progressed but we found no difference in fat scores and body condition across cover types. In contrast, Wood Thrush body condition and fat scores decreased significantly with elevation but we found no difference between cover types and no change over time. Preliminary estimates of apparent survival of Wilson's Warbler was 0.86 (sd = 0.06) with mean probability of detection 0.4 (sd = 0.07). For further analyses, we will relate and discuss apparent survival to coffee cover type and landscape metrics. Migratory songbird populations may be limited in different parts of their annual life cycle- we contribute with our knowledge on overwintering ecology.}

\normaltalk{Cabo De San Antonio, Guanahacabibes pen\'{i}nsula  as Stopover Site for Migrantory Birds in Cuba}{Alejandro Llanes Sosa, Alina P\'{e}rez Hernandez}{Cabo de San Antonio Stopover site for migrant birds in Cuba. The Guanahacabibes peninsula, well known as a very important site for the arrival of Migrant Birds, that come from their breeding  territories and use it like winter ground or like transient  territories in their flyway to Central America and South America. ``The project follow the migration in the gulf of Mexico", leader by Center of Research and Environmental Service (ECOVIDA) from Pinar del R\'{i}o province, have studied the birds fall migration since 2015, using mist net, capture and banding techniques. Using the fat level and amount of muscle in all the bird we have captured mainly belong to the genera:  Empidonax, Vireo, Catharus, Geothlypis, Setophaga, among others we argued their status in the area. We also argued why this area is the most specie rich of rare species belong to the genera: Tyrannus,  Spizella,  Chondestes  y Spiza   in Cuba when we compare with the rest of Cuba where there are not any records or are very rare or accidentals.La pen\'{i}nsula e Guanahacabibes es conocida por su importancia, en el arribo de aves migratorias, que procedentes de sus \'{a}reas de reproducci\'{o}n en Norte Am\'{e}rica, llegan a la misma y  la utilizan como sitio de residencia invernal o de tr\'{a}nsito, en su ruta migratoria hacia Centro y Suram\'{e}rica. Poco se ha publicado  acerca de la  importancia de la mencionada \'{a}rea, como sitio de parada ( Stopover) de aves migratorias. El proyecto ``Seguimiento de la Migraci\'{o}n en el golfo de M\'{e}xico", liderado por el Centro de Investigaciones y Servicios Ambientales ECOVIDA, de Pinar del Rio, estudia la migraci\'{o}n oto\~{n}al en el cabo de San Antonio desde 2015, mediante la captura de aves y anillamiento de estas, utilizando redes ornitol\'{o}gicas. Se dan a conocer las principales especies de los g\'{e}neros: Empidonax, Vireo, Catharus, Geothlypis, Setophaga, entre otros,  que utilizan dicha \'{a}rea de Cuba como sitio de tr\'{a}nsito, al analizar los niveles de grasa corporal y musculo de los individuos capturados. Se discute adem\'{a}s porque esta \'{a}rea de Cuba constituye  una de las m\'{a}s ricas  en especies, que para el resto del territorio nacional son raras, muy raras o  casuales a trav\'{e}s de la presencia de los g\'{e}neros: Tyrannus,  Spizella,  Chondestes  y Spiza.}

\normaltalk{Motus Automated Telemetry Reveals Sanderling and Red Knot Migration Patterns in the North American Midcontinental Flyway}{Kristin M Bianchini, Jessica E Howell, Ann E McKellar, David J Newstead, Christy A Morrissey}{Understanding the migratory connectivity is critical to assessing the factors that limit long-distance migratory bird populations. Previous data suggested that sites along the North American Midcontinental Flyway are important staging areas for two related Arctic-breeding shorebird species, Sanderlings (Calidris alba) and Red Knots (C. canutus). However, little research has examined shorebird migration along this route, and associations among midcontinental staging sites are poorly understood. In this study, we tracked the spring northward and fall southward migration movements of Sanderlings and Red Knots using the Motus Wildlife Tracking System, an international network of automated radio-telemetry receivers. Between 2015 and 2018, we attached individually-coded nanotags to 120 Sanderlings and 40 Red Knots in the Gulf of Mexico and 147 Sanderlings at Chaplin Lake, Saskatchewan. Most radio-tagged Sanderlings (81\%) detected on northward and southward migrations underwent an elliptical migration pattern, travelling north along the Midcontinental Flyway and south along the Atlantic Flyway. Red Knots used the Midcontinental Flyway (56\%) and Atlantic Flyway (44\%) on their northward migration, and southward detections of Red Knots were restricted to Hudson and James Bays, with southward migratory routes unknown. Both species showed near identical migration timing and staging site use; Chaplin Lake and nearby Reed Lake, Saskatchewan, were important northward staging sites, and Hudson Bay, James Bay, Delaware Bay, and the Great Lakes were important southward staging areas. Our results show the importance of the Midcontinental Flyway for shorebird migration and highlight the value of the Motus system for understanding avian migratory connectivity.}

\normaltalk{Assessing Post-Breeding Dispersal Movements of Adult and Juvenile Wood Thrush in Southeastern Pennsylvania}{Amanda Bebel, Lucas DeGroote, Lisa Kiziuk}{Many songbird species are experiencing dramatic declines in population numbers, unfortunately, scientists are using outdated information to predict complete life cycles. Learning about how and when adult and juvenile Wood Thrush (Hylocichla mustelina) disperse after breeding can give crucial information as to what habitat structures are needed for this species and help inform management practices. In this study, Wood Thrush dispersal was tracked using nanotags at Rushton Woods supported by the Willistown Conservation Trust in southeastern Pennsylvania. The birds were tracked using hand-held radio receivers together with local automated receiver stations, which are part of the larger Motus network (www.motus.org). That information was used to determine what habitat structures the birds move to after breeding in a mature deciduous forest. This study looked at post-breeding dispersal of Wood Thrush in the highly developed southeastern Pennsylvania with hopes of informing management practices.}

\normaltalk{Programa Venezolano De Anillamiento De Aves (PVAA): Implementing a Nationwide Bird Banding Network in Venezuela}{Juan C Fernandez}{The Venezuelan Bird Banding Program (PVAA, for its acronym in Spanish: Programa Venezolano de Anillamiento de Aves) was born in 2018, thanks to the joint work of several Venezuelan groups and persons dedicated to scientific bird research. The program is based on three main ideas: ornithological research, coordination among banders, and training for researchers using bird banding. The program plans to coordinate, on a voluntary basis, scientific bird banding activities in Venezuela. The main objectives of the PVAA are to: 1) study the migration, and natural history of resident and migratory birds in Venezuela; 2) evaluate the importance of Venezuela as a stopover and wintering site for migrants; 3) facilitate banding training for researchers, and ornithologists; and, 4) create a public, coordinated database with information generated from scientific banding, and make this information available to researchers. The program will host of database of people and groups that work banding birds in Venezuela, their qualifications, banding projects, species studied, number of banded birds, as well as bands used.  The PVAA will collaborate with public and private institutions, provide field data, and information about banded birds. In addition, it will organize and carry out research projects, theoretical and practical training courses in bird banding techniques, following international standards, such as including the use of capture methods, metal bands and auxiliary marks, biometric measurements and bird studies in hand: molt, age and sex determination, etc. Additionally, The PVAA will organize workshops on taxonomy, taxidermy, banding ethics, knowledge of migratory routes, birdwatching, etc.}

\normaltalk{Ongoing Bird Banding Monitoring Schemes in Cuba: Objectives and Main Results}{Freddy Rodriguez-Santana, Alina P , Ángel Arias-Barreto, Elien Dom\'{i}nguez-Tan, Yasit Segovia-Vage, Alejandro Llanes-Sosa, Edwin Ruiz-Rojas, Liorna Medina-Marcos, Carmen Plasencia-Le\'{o}n, Hiram Gonz\'{a}lez-Alonso, Felix R Rodr\'{i}guez-Le\'{o}n, Nibaldo Francisco-Falc\'{o}n}{There are currently four bird banding monitoring schemes in Cuba located in western (Pinar del R\'{i}o), central (Matanzas and Villa Clara), and eastern (Santiago de Cuba) Cuba. The overarching objectives of these banding efforts are to describe the magnitude of autumn and spring migration, to determine the importance of Cuban habitats for Nearctic-neotropical birds, to study breeding phenology, molt patterns and life expectancy of Cuban resident birds and in a long term to determine the effects of climate change on bird communities. Approximately one-third of the 380 bird species reported to Cuba are being studied at these sites, mostly forest birds. We report important concentrations of migratory birds, which are the most frequent species captured in western and central Cuba, especially during autumn migration and in central Cuba during both autumn and spring migration, which highlights the importance of the Cuban archipelago for Nearctic-neotropical migratory birds. In eastern Cuba, the most frequent species captured were resident birds. We report two new migrant bird records for Cuba on western (Vermilion Flycatcher, Pyrocephalus rubinus) and eastern Cuba (Connecticut Warbler, Oporornis agilis).}

\normaltalk{Using the USGS Bird Banding Lab Database to Assess Dispersal in Northern Cardinals (Cardinalis Cardinalis)}{Daniel P Shustack}{Banding and encounter records contains valuable information about avian movements and dispersal. The sheer volume of data in the Bird Banding Lab (BBL) database means that even low probability events are likely to be detected in large numbers. I reviewed the BBL encounter records of the Northern Cardinal (Cardinalis cardinalis) in order to describe patterns of movements and probable dispersal. More than half of a million cardinals have been banded since 1960 in the United States, with a ~1.1\% encounter rate. In the entire BBL database (from 1928 through March 2019) there are 11,572 encounters of banded Cardinals. Of these, 915 (7.9\%) encounters indicate movement beyond the original block in which the cardinal was banded. Females which changed 10-min banding blocks moved an average of 41.2 km (N=342), males 40.6 km (N=485), and individuals of unknown sex 63.4 km (N=76). The longest movement in the BBL records is a cardinal (unknown sex) which moved 1800 km northward over 23 months. There are six other records of movements greater than 500 km, and an additional 18 records great than 250 km. While these are interesting examples, these long-distance movements appear to be exceptional and cardinals display a strong tendency to remain near their banding locations for their whole life.}

\normaltalk{Barn Swallow (Hirundo Rustica) Fledglings Use Crop Habitat more Frequently in Relation to Its Availability than Pasture and Other Habitat Types}{Chloe K Boynton, Nancy A Mahony, Tony D Williams}{Populations of birds that forage on aerial insects have been declining across North America for several decades, but the main causes of and reasons for geographical variation in these declines remains unclear. We examined the habitat use and survival of post-fledging Barn Swallows (Hirundo rustica) near Vancouver, British Columbia, Canada, using VHF radio telemetry. We predicted that fledgling Barn Swallows hatched in higher-quality natal habitat (pasture) would fledge at higher quality, stay closest to the nest, disproportionately use higher-quality habitat during the post-fledge stage, and have higher survival rates in the region. Contrary to our predictions, we found that natal habitat (crop, pasture, or non-agriculture) had no effect on fledgling quality or movement distance. Barn Swallow fledglings used crop habitat more frequently in relation to its availability than other habitat types, including pasture. Barn Swallows had low post-fledging survival rates (0.44; 95\% CI: 0.35--0.57), which could negatively influence the population trend of the species in this region. While natal habitat had only minor effects, crop habitat appears to be important for fledgling Barn Swallows and, therefore, a decline in this habitat type could have further negative implications for an already declining species.}

\normaltalk{Phylogeography, Population Structure, and Species Delimitation in Rockhopper Penguins (Eudyptes Chrysocome and Eudyptes Moseleyi).}{Herman L Mays, David A Oehler, Kyle W Morrison, Ariadna E Morales, Manuel Marin, Andrea Raya Rey, Antje Steinfurth, Sarah Crofts, Yves Cherel, Maud Poisbleau, Petra Quillfeldt, David R Thompson}{Rockhopper penguins are delimited as 2 species, the northern rockhopper (Eudyptes moseleyi) and the southern rockhopper (Eudyptes chrysocome), with the latter comprising 2 subspecies, the western rockhopper (Eudyptes chrysocome chrysocome) and the eastern rockhopper (Eudyptes chrysocome filholi). We conducted a phylogeographic study using multilocus data from 114 individuals sampled across 12 colonies from the entire range of the northern/southern rockhopper complex to assess potential population structure, gene flow, and species limits. Bayesian and likelihood methods with nuclear and mitochondrial DNA, including model testing and heuristic approaches, support E. moseleyi and E. chrysocome as distinct species lineages with a divergence time of 0.97 Ma. However, these analyses also indicated the presence of gene flow between these species. Among southern rockhopper subspecies, we found evidence of significant gene flow and heuristic approaches to species delimitation based on the genealogical diversity index failed to delimit them as species. The best-supported population models for the southern rockhoppers were those where E. c. chrysocome and E. c. filholi were combined into a single lineage or 2 lineages with bidirectional gene flow. Additionally, we found that E. c. filholi has the highest effective population size while E. c. chrysocome showed similar effective population size to that of the endangered E. moseleyi. We suggest that the current taxonomic definitions within rockhopper penguins be upheld and that E. chrysocome populations, all found south of the subtropical front, should be treated as a single taxon with distinct management units for E. c. chrysocome and E. c. filholi.}

\normaltalk{Reintroduction of Northern Bobwhite to the Gus Engeling Wildlife Management Area (Texas, Usa)}{Ricardo Cagigal Perez, Diego Navarro, Kyle T Hand, Kyle R Brunson, Jeffrey W Gunnels, Therese A Catanach, Michael C Frisbie, Jason B Hardin, Robert M Perez, Brian L Pierce, Nova J Silvy}{Northern Bobwhite (Colinus virginianus) were reintroduced into the Gus Engeling Wildlife Management Area (GEWMA) where they have been extirpated.  Prior to reintroduction, GEWMA was surveyed to ensure no quail were present. Forty-six (26 males and 20 females) bobwhites were trapped from 7 March--5 April 2019 in South Texas, radio-tagged then transported to GEWMA for release.  In addition, 17 (9 males and 8 females) bobwhites were trapped from 13--15 April 2019, radio-tagged, and returned to the source population to identify movement, reproduction, and survival differences between the source and released bobwhites.  Survival for bobwhites released at GEWMA was 37.0\% through 1 July 2019 compared to 70.6\% for birds not translocated.  However, during this time 3 nests were found at GEWMA ) while none were found on the ranch in South Texas. Movement distances between daily locations for males and females did not differ at GEWMA or in South Texas; however there was a significant (P $<$ 0.001) difference in daily movement between quail at GEWMA and South Texas.  Female quail at GEWMA moved 5.4 times the distance of female quail in South Texas and male quail at GEWMA moved 5.9 times the distance of male quail in South Texas.  Quail at GEWMA were located in woody cover only 24.2\% of the time, whereas quail in South Texas were located in woody cover 76.1\% of the time. The greater daily movement and lower use of woody cover by quail at GEWMA probably contributed to their lower survival.}

\normaltalk{Modeling Woodpecker Damage in Wooden Utility Pole Systems Across the Southeast United States}{Hannah C Wright, Carlos Ramirez-Reyes, Scott A Rush}{Woodpecker activity in utility poles and subsequent costs to repair and replace these structures have been an ongoing source of wildlife-human conflict in the U.S. during the last 100 years. In the southeast United States, the Tennessee Valley Authority utility company has accrued an estimated \$5 million dollar/year in costs through mitigation of woodpecker damage. With the goal of reducing the annual cost to the utility providers, we will be using species occurrence records spanning 475,000 kmtextasciicircum 2 of utility pole systems that have been affected by various woodpecker species. We are working to identify features (abiotic and biotic) associated with areas of heightened woodpecker damage through the use of an Ensemble Species Distribution Model. Using information obtained from this model, we will identify areas that are associated with the highest woodpecker activity and thus likely to accrue damage to the utility infrastructure. This ongoing work has the objective of recommending areas of increased surveillance due to landscape characteristics associated with dense clusters of woodpecker damage to allow mitigation tactics to be implemented before the power system is affected.}

\normaltalk{An Integrated Population Model for Spatial Capture-Recapture and Distance Sampling Data}{Narmadha M Mohankumar, Trevor J Hefley, Katy Silber, W A Boyle}{Classical capture-recapture (CR) models are widely used to estimate survival rates and the abundance of species in ecology. However, classical CR models are unable to accommodate the spatial distribution of individuals; thus, they do not account for the spatially explicit processes created by resource availability, dispersal, migration, and human-caused factors such as habitat fragmentation. The development of spatial capture-recapture (SCR) models incorporate an explicit model for the spatial distribution allowing the ecologists to test hypotheses, including spatial processes that are inherent in ecological studies. Recent developments include integrating SCR and other ecological survey data into a unified modeling framework to obtain inferences on demographic parameters with the potential to reduce uncertainty with abundance, survival, and resource selection. Distance sampling (DS) is a common source of ecological data where distances of the objects are surveyed from randomly placed lines or point transects. Our approach introduces a unified modeling framework integrating SCR and DS by incorporating the spatial distribution. Our study leads to leverage the strengths of data sources, SCS, and DS and improves inference on demographic parameters with enhanced statistical power. We evaluate the performance of the proposed approach using both simulated data and real data for Grasshopper Sparrows on Konza Prairie Biological Station, Kansas.}

\normaltalk{Rapid Assessment of Breeding Productivity, Demographics, and Habitat Suitability of the Canada Warbler (Cardellina Canadensis)}{Christian A Burns, Leonard R Reitsma}{How demographics and habitat structure affect breeding success can be critical components to conservation planning, but intensive population studies can be time-intensive. We tested a method of rapidly assessing breeding productivity while obtaining habitat and demographic information for the Canada Warbler (Cardellina canadensis), a Neotropical migrant songbird currently listed as threatened in Canada, and as a species of conservation concern in the United States. We performed this study from 2015 -- 2019 in Canaan, New Hampshire, on five plots. Two plots were part of an ongoing long-term population study. The other three were newly established ``blitz" plots. We aged and banded males on each plot, and capture sites were revisited once fledging occurred on the long-term plots. We determined successful fledging for a given male based upon observations at or near the capture site where habitat characteristics were also recorded. Captures and revisits were completed within two field days per plot by two researchers. Over fiveive years, 55.94\% (45.2 -- 60.380\%; n = 99127 of 177235)  of males surveyed successfully fledged at least one young. Successful males had significantly higher shrub-level density and deciduous stems at capture sites.  This is an efficient method to assess breeding productivity for the Canada Warbler. Deployment across the breeding range could provide details on breeding productivity for the species that could inform a comprehensive management plan.}

\normaltalk{A Novel Method for Captive Rearing and Translocation of Juvenile Common Loons}{Michelle Kneeland, Vincent Spagnuolo, David Evers, James Paruk, Lee Attix, Nina Schoch, Mark Pokras, Virginia Stout, Alex Dalton, Katy Silber}{Common loons (Gavia immer) are diving waterbirds that are particularly challenging to keep in captivity due to their specific behavioral and physiologic needs, special housing requirements, and susceptibility to stress-related disease. We report a novel method for housing and captive rearing common loon chicks that was developed as part of the first-ever loon translocation effort in southeast Massachusetts, USA from 2015-2017. Thirteen loon chicks were reared in aquatic pens in a natural lake environment, utilizing non-invasive feeding and monitoring techniques that avoided human habituation. Chicks were reared in aquatic pens for 16-28 days before being released onto the lake. All chicks remained clinically normal and were monitored on the lake for up to four months following release. At least four of the chicks were subsequently confirmed to have survived to adulthood when they returned to the area in breeding plumage two to three years following release. Two of these confirmed adults displayed prolonged territorial pair behavior together, and this is an encouraging early sign that captive-reared individuals may form successful breeding pairs in the future. Because most immature loons remain on the ocean until at least 3 years of age, we expect additional captive-reared loons to return to the release area in subsequent years. These husbandry techniques could be applied to other loon and diving bird species that are notoriously difficult to house in captivity. The novel feeding techniques described here could also be adapted for loon chicks being reared in pools or other traditional captive settings.}

\normaltalk{Wildlife Drones: Innovative Technology for Efficiently Radio-Tracking Small Birds}{Debbie Saunders, Josh Bobruk}{Every year millions of dollars are invested in managing threatened and invasive bird species, and understanding how these birds move across the landscape is critically important for making effective management decisions.  However, remote tracking technologies such as satellite and GPS tags remain too large and heavy for the majority of birds. Therefore, the bulk of small bird tracking performed globally is undertaken using Very High Frequency (VHF) radio-transmitter tags.  Where precise locations are required for detailed behavioural studies and habitat use assessments, the need to manually radio-track individual birds one at a time often limits both the number of birds that can be tracked and the quantity of rigorous scientific data collected.  This is due to the very labour intensive and time-consuming nature of manual tracking techniques.  However, when using Wildlife Drones aerial radio-tracking technology, up to 40 birds with unique VHF transmitter frequencies are tracked simultaneously from the air, with location coordinates of each bird mapped in real time. We provide examples of projects where Wildlife Drones been used to gain insights into the movements of critically endangered species, and demonstrate how the flexibility of drone radio-tracking has enabled greater quantities of location data to be collected in less time, including within rugged and remote landscapes.  Such innovative technology provides new opportunities for improving and expanding bird research, monitoring and conservation projects globally.}

\normaltalk{Emerging Tools: Raptor Habitat 3D Modeling with an Unmanned Aircraft System: Osprey Nest Case Study}{June Stephens, Joseph Cerreta, Dean Walton}{Studying and researching raptor habitat can be challenging, particularly in boreal forests and along rivers. Raptor nests are generally located high up in tall trees or on top of structures such as communication towers.   Unmanned aircraft systems provide an approach to observing and recording raptor habitats that is less invasive than human intervention and provides information through photography, videography and 3D imaging. The objective of this study is to utilize an unmanned aerial vehicle to capture numerous images and compile them into a 3D model for further study.     The selected site was an abandoned osprey nest located on top of a collocated cell/lighting tower. Utilizing a DJI Phantom quadcopter and Pix4D modeling software, a 3D image was created showing nest construction. The data was also used to 3D print a small model for educational purposes.  The results were that unmanned aircraft systems provide a non-destructive approach to recording images of raptor habitat while migratory raptors are away; high-resolution 3D models can be created through pre-flight programming and post-flight processing; construction of the nest can be determined from 3D modeling; and 3D printed models can be made from the data. Thus, unmanned aircraft systems provide a study tool for raptor habitat analysis, education and behavior.}

\normaltalk{Rapid Prototyping for Ornithology - What can 3D Printers and Laser Cutters Do for the Study of Birds?}{Constance J Woodman, Donald J Brightsmith}{Specialty scientific equipment can be extremely expensive, even when built at an on-campus fabrication shop to save on costs. ``Rapid prototyping" methods, such as 3D Printing and laser cutters can create low cost, immediately useful equipment for ornithological studies. Presented are our laboratory's examples of equipment and the data outcomes related to that equipment, including custom capture equipment; laboratory behavior measurement equipment; caged bird enrichment equipment, and data logging equipment. The presentation introduces the technical jargon needed to utilize technology, as well as caveats special to ornithology. Specific technologies discussed are fused deposition modeling printers, Stereolithography resin printers, carbon dioxide laser cutting, and low-cost in-lab photogrammetry for in-lab 3D color scanning. The software discussed focuses on those with minimum learning curves while also being low cost or likely to already be available through campus licenses.}

\normaltalk{Using Drones to Reduce Disturbance While Monitoring Endangered Chaco Eagle Nests}{Diego Gallego, Jos\'{e} H Sarasola}{In birds, obtaining information related to nest occupancy, offspring status or breeding success is essential for population monitoring. However, traditional nest monitoring (tree climbing) demands a lot of time and effort in order to gather enough data while avoiding risks for both observer and bird. Instead, Unmanned Aerial Vehicles (hereafter, drones) present an alternative to this method, but few studies have been done measuring their influence on birds. We addressed the utility of a drone equipped with an on-board camera in examining nesting status of the endangered Chaco Eagle (Buteogallus coronatus) in central Argentina. We performed 65 drone flights at eagles´ nests during four consecutive reproductive seasons, where adult behaviors towards the drone were recorded. Some adults flew from the nest before drone took off, but mostly remained in the nests or in the surroundings ($<$100 meters away), particularly throughout incubation period. During drone flights, only one adult flew from the nest, the rest remaining vigilant or emitting alarm calls. No attack towards the drone was registered. The usage of drones for monitoring Chaco eagle's nests significantly reduced levels of disturbance when compared with the traditional method, where all adults flew away during nest inspection. In addition, this procedure was almost three times faster in comparison to climbing and had no negative effects on reproductive success of Chaco Eagles. These results encourage the use of drones for monitoring breeding populations of raptor species. However, guidelines should be formulated to avoid any kind of interference between individuals and drones.La obtenci\'{o}n de informaci\'{o}n relacionada con la ocupaci\'{o}n y el estado del nido o el \'{e}xito reproductivo es esencial para monitorear las poblaciones de aves. Sin embargo, el monitoreo tradicional (escalada de \'{a}rboles) demanda mucho tiempo y esfuerzo para poder recabar datos suficientes sin poner en peligro al observador y al ave. Por otro lado, el monitoreo con drones como m\'{e}todo alternativo ha sido poco estudiado en relaci\'{o}n a su impacto en las aves. Se comprob\'{o} la utilidad de un dron equipado con una c\'{a}mara para el seguimiento de nidos del Águila del Chaco (Buteogallus coronatus) en el centro de Argentina. Se realizaron 65 vuelos sobre nidos de \'{a}guila durante cuatro temporadas reproductivas consecutivas, en los cuales se registr\'{o} el comportamiento de los adultos hacia el dron. Algunos de ellos se volaron antes del despegue del mismo, pero en su mayor\'{i}a se quedaron en el nido o en los alrededores ($<$100 metros), particularmente en el periodo de incubaci\'{o}n. Durante los vuelos, solamente un adulto escap\'{o} del nido, mientras que el resto permanecieron vigilantes o emitiendo llamadas de alarma. No se registr\'{o} ningún ataque al dron. El uso de drones para monitorear nidos del Águila del Chaco redujo significativamente los niveles de disturbio en comparaci\'{o}n con el m\'{e}todo tradicional, donde todos los adultos se escapaban durante la inspecci\'{o}n de los nidos. Adem\'{a}s, este m\'{e}todo se realiz\'{o} casi tres veces m\'{a}s r\'{a}pido que el de escalada, y no tuvo efectos negativos en el \'{e}xito reproductivo de los nidos de esta especie amenazada. Estos resultados incentivan el uso de drones para monitorear poblaciones reproductivas de aves rapaces. Sin embargo, se deben formular directrices para evitar cualquier tipo de interferencia entre individuos y drones.}

\normaltalk{A Novel Method for Captive Rearing and Translocation of Juvenile Common Loons}{Michelle Kneeland, Vincent Spagnuolo, David Evers, James Paruk, Lee Attix, Nina Schoch, Mark Pokras, Virginia Stout, Alex Dalton, Katy Silber}{Common loons (Gavia immer) are diving waterbirds that are particularly challenging to keep in captivity due to their specific behavioral and physiologic needs, special housing requirements, and susceptibility to stress-related disease. We report a novel method for housing and captive rearing common loon chicks that was developed as part of the first-ever loon translocation effort in southeast Massachusetts, USA from 2015-2017. Thirteen loon chicks were reared in aquatic pens in a natural lake environment, utilizing non-invasive feeding and monitoring techniques that avoided human habituation. Chicks were reared in aquatic pens for 16-28 days before being released onto the lake. All chicks remained clinically normal and were monitored on the lake for up to four months following release. At least four of the chicks were subsequently confirmed to have survived to adulthood when they returned to the area in breeding plumage two to three years following release. Two of these confirmed adults displayed prolonged territorial pair behavior together, and this is an encouraging early sign that captive-reared individuals may form successful breeding pairs in the future. Because most immature loons remain on the ocean until at least 3 years of age, we expect additional captive-reared loons to return to the release area in subsequent years. These husbandry techniques could be applied to other loon and diving bird species that are notoriously difficult to house in captivity. The novel feeding techniques described here could also be adapted for loon chicks being reared in pools or other traditional captive settings.}

\normaltalk{A Data-Driven Geospatial Workflow to Improve Mapping Species Distributions and Assessing Extinction Risk under the IUCN Red List}{Ruben D Palacio, Pablo J Negret, Jorge Vel\'{a}squez-Tibat\'{a}, Andrew P Jacobson}{Species distribution maps are essential for assessing extinction risk and guiding conservation efforts. Here, we developed a data-driven, reproducible geospatial workflow to refine the distribution of a species starting from its Extent of Occurrence (EOO) to Area of Habitat (AOH) within the species range. The ranges are produced with an Inverse Distance Weighted (IDW) interpolation procedure, using presence and absence points derived from primary biodiversity data. As a case-study, we mapped the distribution of 2,273 bird species in the Americas, 55\% of all terrestrial birds found in the region. Our workflow generated species ranges with fewer errors of omission, commission, and a better overall accuracy than the IUCN range maps. The spatial overlap between both datasets was low (28\%) and the expert-drawn range maps were consistently larger due to errors of commission. Their estimated Area of Habitat (AOH) was also larger for a subset of 741 forest-dependent birds. Additionally, incorporating our derived spatial data increased the number of threatened species by 52\% in comparison to the 2019 IUCN Red List, and 103 species could be placed in threatened categories (VU, EN, CR) pending further assessment. The implementation of our geospatial workflow increases the transparency and reliability of species risk assessments and improves mapping species distributions for conservation planning and decision-making.}

\normaltalk{Observing the Behavioral Thermoregulation of Common and Thick-Billed Murres (Uria Aalge and U. Lomvia) with Thermal Imaging}{Emmylou Kidder, Rick Sherwin}{The interaction between seabirds and their habitat has become increasingly important due to recent evidence of population declines associated with rising atmospheric temperatures and warm water anomalies in the Gulf of Alaska. These environmental factors have resulted in unusual mortality events and reproductive failure of some alcid species occurring in the Gulf of Alaska, which has decreased population stability.  This study utilizes infrared thermal imaging technology to investigate the behavioral thermoregulation of seabirds in the Gulf of Alaska, primarily common murres (Uria aalge) and thick-billed murres (U. lomvia).  Previous research has suggested that these species use the sun-orientation of their contrasting black and white plumage to regulate their thermophysiological needs. Boat-based surveys were conducted during summers 2018 and 2019 to photograph murres in the Chiswell Islands of the Alaska Maritime NWR using a thermal camera (FLUKE Ti50). This thermal data was collected of murres engaged in various activities such as resting, swimming, flying, foraging, and incubating to study energetic costs of these activities and how heat is displaced within the body. Additionally, thermal images were collected of common murres at the Alaska SeaLife Center aviary at a variety of life stages to assess how thermoregulation mechanisms may change and develop over the lifetime of an individual. Our results found that thermal imaging is an innovative and reliable tool to investigate the thermoregulation of seabirds and will allow an improved understanding of the physiological needs of climate-affected species in the Gulf of Alaska to lend to their management and conservation.}

\normaltalk{Fowl Language: Cassin's Sparrow Males Use Different Songs in Different Contexts}{Dylan Allenback, Gus Forrest, Claire W Ramos}{Cassin's Sparrows (Peucaea cassinii), although not the prettiest to look at, may be one of the most interesting of songbirds in prairies of North America. Cassin's Sparrows are understudied and are declining by 3\% per year in Colorado. Cassin's Sparrow males, like many other birds, have displays and songs that they use to defend territories and attract mates, but unlike many songbirds, Cassin's Sparrows have two very distinct songs. Anecdotal evidence suggests that these songs may be used in different social contexts and may have different functions. We used playback experiments to try to determine the function of these two songs and in what context male Cassin's Sparrow are using them at different nesting stages. The data suggest that one song may be used for initial attraction of a mate and the other may be used to defend that mate from neighboring males. Males that are accompanied by females in their territories tend to use one song more than males without a female present in their territories. Further analysis would include testing testosterone levels of males at different nesting stages to see why they might behave this way. This research will increase understanding of the basic breeding behavior of this bird and may assist in future conservation efforts for this species.}

\normaltalk{Comparisons Between Birds Detected by Autonomous Recordings Units (ARU) and Mist Net Captures at a Migration Banding Station}{Yuting Deng, David Bonter, Russell Charif}{Autonomous recording units (ARU) are increasingly used to estimate wildlife abundance based on acoustic monitoring. Numerous studies in recent years compare ARU data to conventional point counts. However, comparisons of acoustic data to mist-net capture data are lacking. In this study, we correlate the daily vocal activity rate recorded from ARUs with capture rates to test the potential of ARUs for monitoring songbirds at a migration stopover site. During spring migration in 2017, we deployed ARUs at Braddock Bay Bird Observatory near Rochester, New York, USA to monitor the acoustic activity of birds within a grid of 36 mist nets operated daily. From the banding data, we estimated bird abundance as the number of birds captured during the first six hours of operation after sunrise each day (including recaptures of previously banded birds). Using BirdNet, a machine learning algorithm detecting and classifying avian sounds, we calculated the number of distinct vocalizations for 20 species. Controlling for date and temperature, the number of vocalizations recorded was positively correlated with total capture rate (both sexes combined) for 12 species (p$<$0.05), not correlated for 3 species (p$>$0.05), and negatively correlated for 5 species (p$<$0.05). When examining sexually dimorphic species, we found that the vocal activity rate is more positively correlated with the male capture rate. Our work demonstrates that the use of ARUs may help monitor certain species, especially dimorphic species, but the application may not be very accurate for other species.}

\normaltalk{Geographic Variation in Song Within the Yellow-Backed Oriole (Icterus Chrysater)}{Amelia Suter, Rachel Sturge, Nandadevi Cortes}{The Yellow-backed oriole (Icterus chrysater) is a New World oriole species native to southern Mexico and parts of Central America, Venezuela, and Colombia. The species is split into two populations: one of which is to the north of Costa Rica while the other is to its south. Previous studies have found evidence of some divergence in their morphology and genetics. However, the vocalizations of these two populations have not yet been closely investigated to look for evidence of separation of this species. The main objective of this study is to examine the vocalizations and determine if there are differences between the northern and southern populations. In order to do this, we collected 18 recordings from the southern part of their range and 59 recordings from the northern part of their range from xenocanto.  Characteristics such as average note length, number of notes, lowest frequency and highest frequency will be examined to see if any of these characteristics differ between the two populations. Of the songs analyzed so far, the northern population appears to have more notes in their songs and sing at higher frequencies than the southern populations; however, there does not yet appear to be differences in the note length.}

\normaltalk{Geographic Variation in Vocalizations of the Acorn Woodpecker (Melanerpes Formicivorus).}{Jos\'{e} dJ Zazueta-Algara, J R Sosa-L\'{o}pez, Mar\'{i}a dC Arizmendi, Adolfo G Navarro-Sigüenza}{Geographic variation in the vocalizations of birds is associated to diverse ecological, phylogenetic, sexual, and cultural factors. If changes in population's vocalizations accumulate, such signals can act as a premating barrier among isolated populations. The Acorn Woodpecker (Melanerpes formicivorus) ranges from the United States to Colombia, showing morphological and genetic divergence among its populations. Here, we analyze the geographic variation in the structure of the Acorn woodpecker's vocalizations. Our main goals were to assess the extent of the acoustic variation within the complex, and to evaluate whether this variation coincides with the previously described genetic and morphological groups. We clearly classified four main call types along the species' distribution range and used two of them to perform our analyses. We found a marked geographic variation in both vocalization types, separating populations north and south of the Isthmus of Tehuantepec, thus classifying populations into two acoustic groups. However, Baja California Sur Cape region populations do not share the vocalization type with the genetically and geographically closest populations, but rather with the more distant southern ones. Overall, the geographic distribution of vocalization types reflects the previously described patterns of genetic and morphological differentiation, in which each of the four genetic groups, that are restricted to the main mountain ranges of the region, belong to one of the two acoustic groups characterized.}

\normaltalk{Geographic Variation in Golden-Cheeked Warbler Song Characteristics}{Drew S Finn, Ashley M Long}{Many taxa exhibit geographic variation in acoustic signals, which can lead to reproductive isolation and divergence among populations. As such, understanding the spatial variation of acoustic signals within a species may reflect its population structure, and can inform management of priority species. The golden-cheeked warbler (Setophaga chrysoparia; hereafter warbler) is an endangered Neotropical songbird that nests exclusively in oak-juniper woodlands in central Texas. Using songs recorded in 2012, 2017, and 2018 at 25 study sites (n = 171 individuals), we examined geographic variation in warbler song characteristics across this species' relatively restricted breeding range. We analyzed two song types (A songs and B songs) and found statistically significant differences in 20 of 23 A song characteristics and 38 of 42 B song characteristics across breeding range regions. Similar to other warbler species, B songs exhibited more regional variation than A songs, and our multivariate analyses suggest that songs from the northern and central portions of the warbler's breeding range are more similar to each other than to songs from the southwestern portion. These results may reflect behavioral aspects of warbler population structure and could be used in conjunction with other data (e.g., genetics) to help inform conservation planning for this species.}

\normaltalk{Brown Creeper Songs are Individually Distinctive and Geographically Variable}{Jordan Winter, Jennifer R Foote}{The songs of oscine songbirds often show variation within and among individuals. Across the range of a species, there can also be considerable geographic variation. However, the extent of the geographic variation is highly variable between different species, with some showing relatively conserved songs across their range and others showing extensive variation. The Brown Creeper (Certhia americana) is a cryptic songbird with several subspecies found throughout North America. Previous studies have shown evidence of differences in song structure between populations in coastal and inland California populations. However, geographic variation hasn't been identified for the widespread americana subspecies in Eastern North America. This study sought to examine and quantify the differences in song structure between two populations of Brown Creeper (subspecies americana) in Ontario. We identified 10 songs from 7 birds from autonomous recordings made in Sault Ste Marie and Northwestern Ontario (900km apart). We compared the songs using two approaches. First, we used spectrogram cross-correlation to calculate spectrogram overlap between all songs. Second, we measured fine-structural temporal and frequency properties of songs to quantify variation between populations. We found evidence of both individually distinctive songs and geographic variation in song using spectrogram cross correlation. We also identified several frequency differences between the songs from the two populations. Quantifying the song differences between these two populations will help to further understanding the structure of this species song, the range of variability in Brown Creeper song, and contributes to our understanding of geographic patterns of variation in bird song.}

\normaltalk{Differences in Song Structure of Three Bird Species along an Altitudinal Gradient in the Tropical Andes of Bolivia}{Rhayza V Cortes-Romay, Adriana Rico-Cernohorska, Flavia Monta\~{n}o-Centellas}{Environmental conditions change rapidly with elevation in tropical mountains. To convey their message, organisms that communicate acoustically must adapt their songs to local conditions across elevations. Elevation, however, can also affect song structure indirectly by affecting aspects of individual morphology (body mass, bill dimensions). We evaluated direct (adaptation hypothesis) and indirect (by-product hypothesis) effects of elevation on the song structure of three sympatric species of birds: Myothlypis luteoviridis, Atlapetes rufinucha and Ochthoeca frontalis. We first quantify sound persistence (reverberation) at different elevations to assess the effect of environmental variables on sound transmissibility. Then, we investigated elevational differences in song structure (maximum, minimum and peak frequencies, and bandwidth) from recordings obtained along the gradient, and the environmental variables that explain such differences. Finally, we examined elevational changes in bill morphology and tarsus length in birds captured with mist nets at different altitudes. We found sound reverberation to be greater in warmer and humid sites, and sites with low vegetation complexity. Song structure varied with elevation, with maximum frequency decreasing in warmer elevations and low structural complexity, and greater values of peak frequency and bandwidth in colder high elevations. Finally, we found bill height and length (culmen) to decrease at high elevations potentially affecting song structure. Altogether our results provide evidence supporting both the adaptation and they by-product hypotheses, suggesting that elevation has both direct and indirect effects on the structure of mountain bird song}

\normaltalk{Characterizing Vocalizations of an Endangered Hawaiian Finch: Cultural Evolution in Palila?}{Lisa LK Mason, Kristina L Paxton, Patrick J Hart}{The best examples of cultural evolution in animals are demonstrated in the spatial and temporal changes of birdsongs. Unique song dialects may arise over subsequent generations between or within populations as a result of environmental or social divergence. In species of oscine songbirds (order Passeriformes), vocal memes are socially learned and transmitted from adult tutors to young. However, cultural diversity may change with new introductions of memes resulting from variations in the learning process or decrease as populations shrink leaving fewer tutors. The critically endangered Palila bird (Loxoides bailleui) is the last remaining endemic finch on Hawai'i Island. This study is the first to characterize the vocalizations of wild Palila and to study the longitudinal effects of their declining population on song complexity and overall vocal diversity at Pu'u La'au, Mauna Kea over a nearly 50 year period (1974 to present). For comparison, vocalizations of a stable, coexisting population of Hawai'i `Amakihi (Chlorodrepanis virens) were analyzed as a control. Preliminary analyses reveal the need for high-precision, automated spectral analyses to tease apart differences in meme types. I hypothesize the current population of Palila will sing simpler songs and have lower overall vocal diversity than earlier years. Comparatively, I hypothesize no differences in song complexity or vocal diversity across years in the stable population of Hawai'i `Amakihi. The results of this research will be used to improve management and conservation efforts. Furthermore, this research lends itself towards future studies on the emerging topic of the cultural evolution of Hawaiian forest birds.}

\normaltalk{Association Between Song and a Complex Habitat Transition in a Neotropical Passerine, the Rufous-Collared Sparrow}{Natal\'{i} Attin\'{a}, Pablo D Lavinia, Pablo L Tubaro, Dar\'{i}o A Lijtmaer}{The acoustic adaptation hypothesis (AAH) proposes that song structure is adapted to minimize its degradation during transmission and predicts that notes should be more separated in closed than in open environments. The Rufous-collared Sparrow (Zonotrichia capensis) has been extensively studied in this regard because of the close association between its song and habitat structure, which generates song dialects that differ in the rate of their trill notes. However, previous studies have been rather simple with respect to the spatial assessment of song variation in this species. We studied the association between habitat and vocalizations in this species across a complex matrix of interdigitated steppe (open habitat) and espinal woodland (closed habitat) in central Argentina. We recorded 629 individuals in an area of 100 km x 170 km and used Raven 1.4 to measure temporal and frequency variables on each song. We found a reversed pattern of association between song and habitat compared to that expected under the AAH in the south of the study area, which could be the consequence of strong winds and a lower density of individuals in the steppe. The pattern was less clear in the north due to an overall reduction in trill note separation. Transitions between dialects were gradual across environments and vocalizations from one habitat were influenced by those from the neighboring one. Our results confirm that the song of Z. capensis is associated with habitat structure, even across intricated environmental transitions, but highlight that this association can be complex and difficult to predict.}

\normaltalk{The Effect of Corticosterone on the Ontogeny of Babbling in a Wild Parrot Species}{Celia R McLean, Karl S Berg}{The stress-axis of the endocrine system allows for animals to respond to environmental stressors in contextually appropriate ways. A stressor prompts short-term, physiological and behavioral changes like sharpened cognition, the mobilization of glucose, and stimulation of immune function. Chronic stress can have deleterious effects on early development, especially in altricial birds. Elevated levels of the stress hormone corticosterone (CORT) in male songbirds can result in compromised song learning ability. Parrots form a sister group to songbirds, but it is unknown whether CORT effects vocal development in parrots. Objectives are to characterize the ontogeny of vocal babbling and to assess the effects of CORT supplements in green rumped parrotlets (Forpus passerinus) of Venezuela.  Parrots are very altricial and both sexes learn new vocalizations throughout life. Daily audio-video recordings of marked nestling parrotlets were made inside specially designed nest tubes. CORT-oil solution was administered to one nestling per nest over a period of seven days, beginning at 13 days post hatching of a 30-day nestling period; controls received pure oil or nothing. Bouts of babbling will be extracted from video and analyzed spectrographically. Understanding how stress influences early learning strategies in parrots may provide insights into why vocal learning manifests itself differently in songbirds.}

\normaltalk{Behavioral and Neural Responses to Far-Range Songs in a Territorial Songbird}{Amy VH Strauss, Luke Remage-Healey, Jeffrey Podos}{Acoustic signals are particularly useful for animals that communicate over distance or out of visual range of each other, as they enable long-distance interactions in real time. However, as sounds propagate over distance between signal senders and signal receivers, they are subject to distance- and habitat-dependent structural degradation. This physical phenomenon results in vocalizations that may acquire an acoustic transmission signature and provide receivers with information about singer location ('ranging hypothesis'). In territorial songbirds, the accuracy of locating singing conspecific rivals may have real fitness consequences, and the precision with which receivers can do so should depend on their perceptual discrimination abilities. In two parallel experiments, we tested acoustic distance assessment of conspecific song by song sparrows, an open habitat species. In field trials, we tested the behavioral responses of territorial males to playback of conspecific songs recorded locally at near to far ranges. In lab trials, we presented these same stimuli to anesthetized birds and measured neurophysiological activity in forebrain nuclei known to function in auditory discrimination. Results from field trials indicate that birds' behavioral responses did not vary with degrees of stimulus degradation, suggesting that song sparrows do not rely solely on acoustic cues to determine signaler distance. Initial analyses from the electrophysiology experiments show no consistent firing differences in auditory processing nuclei in response to songs from near to far ranges, suggesting a potential perceptual limitation. We explore the potential role of habitat as a selective agent shaping neighbor localization strategies in territorial songbirds.}

\normaltalk{Do morphological traits constrain song traits in a family (Icteridae) with diverse feeding adaptations?}{Hannah R Mirando, Natalia C Garc\'{i}a}{Song production and its evolution in birds can be influenced by morphological traits such as beak morphology and body size. Larger beaks have been known to constrain note rate and frequency bandwidth in some species, and internal body structure also affects the properties of a bird's vocalization. The New World Blackbirds (Passeriformes: Icteridae) is a family of birds with a diverse range of beak morphologies and associated jaw musculature, related to the different feeding habits found in this clade. Here, we analyze the relationship between different acoustic variables and beak dimensions in a group of species in the Icteridae family. We found that larger beaks are associated with lower emphasized frequencies as found in previous studies. Interestingly, we did not find the predicted relation of temporal structure of song and beak size. This could be related to the aforementioned adaptations to different diets, as several previous studies have compared species with similar diets.}

\normaltalk{The Production of Vocal Cues in Response to Changing Food Availability in Pine Siskins}{Jessica K Tir, Heather E Watts}{One means by which birds can improve their foraging success is by using social cues. However, less is known about which sensory modalities are used to transmit these cues. For example, some animals may evaluate local resources by watching the behavior of conspecifics, while others may use olfactory or acoustic clues. This research begins to address this gap by experimentally testing whether pine siskins alter vocal production based on local food availability. Pine siskins are highly gregarious birds whose migration is linked to food availability, thus this species is particularly likely to use social information to aid their migratory decision-making. Specifically, we examined whether birds altered their rate of vocalizations, the types of calls used, and/or the acoustic characteristics of their vocalizations in response to food availability by recording the vocalizations of captive pine siskins experiencing reduced or ad libitum food availability. All birds experienced both control and treatment conditions so we could evaluate individual-level changes in vocal behavior. This work helps clarify if acoustic cues are an important modality of transmitting information about local food availability.}

\normaltalk{Using Hawkes Process Models to Assess Behavioral Motivations to Sing, Applied in a Fragmented Forest Landscape}{Eliza M Grames, Piper L Stepule, Benjamin T Ranelli, Chris S Elphick}{Singing plays a critical role in breeding success for male songbirds, which use song to establish and defend territory boundaries from neighboring males and intruders and to attract mates. The different motivations for singing can be broadly grouped into two categories: 1) internally motivated, such as singing to attract a mate, and 2) responsive to other songs, such as countersinging to defend territorial boundaries. We used Hawkes process models, a class of inhomogeneous, self-exciting point process models, to separate patterns of Ovenbird (Seiurus aurocapilla) songs into these underlying behavioral components and measure how the motivations to sing change across the breeding season and with the size of a forest patch. We found that internally motivated singing decreased across the breeding season in medium and large forests, whereas it remained constant in small forests. We also found that the proportion of the overall singing rate attributable to responsive songs decreased throughout the breeding season in small forests, whereas it increased in medium and large forests, indicating higher rates of countersinging in larger forests. Our results suggest that males in small forests remain unpaired late in the breeding season and continue singing to attract a mate, which is consistent with results from field studies that have tracked male pairing success rates in small and large forests. Because our modeling framework allows us to decompose song patterns from acoustic recordings into the underlying motivations to sing, they present an opportunity to remotely study factors affecting breeding bird behavior.    /////    El canto es fundamental en el \'{e}xito reproductivo de la mayor\'{i}a de machos passeriformes. Las motivaciones para cantar se pueden agrupar en dos categor\'{i}as: 1) motivadas internamente, al cantar para atraer a un compa\~{n}ero, y 2) sensibles a otras canciones, como contraataque definiendo l\'{i}mites territoriales. Utilizamos modelos de proceso de Hawkes para separar los patrones de las canciones de Seiurus aurocapilla en estas categor\'{i}as y medir c\'{o}mo cambian las motivaciones para cantar a lo largo de la temporada de reproducci\'{o}n y con el tama\~{n}o de un parche de bosque. Descubrimos que el canto motivado internamente disminuy\'{o} durante la temporada de reproducci\'{o}n en bosques medianos y grandes, pero se mantuvo constante en los bosques peque\~{n}os. Tambi\'{e}n encontramos que la proporci\'{o}n de la tasa general de canto atribuible a las canciones receptivas disminuy\'{o} durante la temporada de reproducci\'{o}n en los bosques peque\~{n}os, mientras que aument\'{o} en los bosques medianos y grandes. Nuestros resultados sugieren que los machos en los bosques peque\~{n}os permanecen sin aparearse al final de la temporada de reproducci\'{o}n y continúan cantando para atraer a una pareja, lo que es consistente con los resultados de estudios de campo que han rastreado las tasas de \'{e}xito del apareamiento de machos en bosques peque\~{n}os y grandes. Debido a que los modelos Hawkes nos permiten descomponer los patrones de canciones de las grabaciones acústicas en las motivaciones subyacentes para cantar, los modelos ofrecen la oportunidad de estudiar de forma remota los factores que afectan el comportamiento de las aves en \'{e}pocas reproductivas.}

\normaltalk{A Repository of the Sounds of Colombia: The Environmental Sound Collection at Humboldt Institute}{Orlando Acevedo-Charry}{Media collections include important behavioral traits and trackable evidence of the presence of species. However, lack of well-structured repositories can result in loss of valuable information. Here, I present the Environmental Sound Collection at the Humboldt Institute (IAvH-CSA), the most important repository of sound archives in Colombia. This collection, previously known as ``Banco de Sonidos Animales--BSA", began on 1997 as a strategy to document the Colombian biological diversity and strengthen the biological inventory of Colombia, interest that is still latent today in the Humboldt Institute. After over 20 years of work, it currently has more than 23,000 recordings, mainly of birds (76\%) from all departments in Colombia. These digital specimens are valuable complementary information in different research dynamics that may include acoustics, such as ecology, taxonomy, evolution, behavior, and outreach. The collection keeps a focus on the audible spectrum (2-20,000 Hz), but in the near future, we intend to extend our understanding of the modes of acoustic communication to infra (below the audible spectrum) and ultrasound (above the audible spectrum). We have some digital copies of our archives in interinstitutional collaborations with Macaulay Library at the Cornell Lab of Ornithology and Xeno-canto; the two biggest acoustic repositories in the world.}

\normaltalk{Does Mesoclimate Drive Singing Phenology in Desert Songbirds?}{Meelyn M Pandit, Eli S Bridge, Jeremy D Ross}{Climate change is increasing aridity across multiple grassland and desert habitats across the southwestern United States, reducing available resources and drastically changing the breeding habitat for many bird species. While much of the research focuses on how aridity will affect the physical habitat, little research has examined how habitat soundscapes will change under these new conditions. As aridity increases it will reduce sound propagation distances, negatively impacting habitat soundscapes and creating new selection pressures for territorial songbirds. As resources such as food and water decline, individual songbirds will need to expand territory sizes to obtain the resources necessary for reproduction and survival. Individuals will need to defend these larger territories, leading to sexual selection favoring elevated singing activity and increased territorial movement. However, with the reduced acoustic transmission properties, reduced and the increased risk of dehydration through evaporative water loss, natural selection may select against territorial singing, leading to a reduction in song activity, effectively altering the soundscape. To determine the threshold of declining singing activity under real-world and future weather conditions, we developed an agent-based model that examined singing activity levels under different temperature, relative humidity, and sound propagation distances. We hypothesized that singing activity would decline earlier under low relative humidity and high temperature conditions due to the elevated cost of singing under low sound propagation levels. Understanding how increasing aridity will affect the soundscape of grassland and desert birds may determine the behavioral mechanisms behind community declines and collapses.}

\normaltalk{Influence of Climate Change on Avian Patterns of Functional and Phylogenetic Structure in a Seasonally Dry Tropical Forest}{Victor Leandro-Silva, Thiago Gonçalves-Souza, Luciano N Naka}{Climate change likely represents one of the greatest threats to biodiversity. Future scenarios predict changes in species distributions and drastic modifications in species assemblages. Neotropical Dry forests are likely to become much harsher environments, with more severe droughts and even hotter climates. Understanding how these changes may influence patterns of avian diversity is a fundamental step towards preparing for such future. Here, was used three carbon concentration scenarios (RCP 45, 60 and 85) to evaluate the effect of climate change on avian species distributions, species richness, and functional and phylogenetic diversity in the Caatinga Dry forest of northeastern Brazil. We modelled the current and future (2050 and 2070) distribution of the entire avifauna (455 species) of the Caatinga. We divided the ~850,000 km2 of the Caatinga into ~2,500 grids (20km x 20km) and calculated the Net relatedness index (NRI) for each grid for current and future communities. Our models predict that a minimum of 120 species (~26\%) will lose more than 25\% of their area under the most optimistic scenarios (RCP 45 for 2050), and almost 200 species (~40\%) under more pessimistic ones (RCP 85 for 2070). These loses are due to climatic change alone, without considering deforestation and desertification. Although we did not find significant changes in functional diversity, possibly due to a high degree of ecological redundancy, our models predict significant changes in species phylogenetic structure under all future scenarios, with the loss of important avian lineages.}

\normaltalk{Impacts of Extreme Disturbances at Wintering Areas on Piping Plover Survival and Migratory Connectivity}{Kristen S Ellis, Michael J Anteau, Francesca J Cuthbert, Cheri L Gratto-Trevor, Joel G Jorgensen, David N Koons, David J Newstead, Larkin A Powell, Megan M Ring, Mark H Sherfy, Rose J Swift, Dustin L Toy}{Effective conservation for listed migratory species requires understanding linkages between breeding and non-breeding areas. Environmental conditions away from breeding areas may have important influences on demography of these species because most of the annual cycle occurs outside breeding areas. Piping Plovers (Charadrius melodus) are a federally-listed migratory species that periodically experience extreme environmental disturbances on wintering areas, including oil spills, toxic algal blooms, and hurricanes. Further, strong site fidelity of Piping Plovers at wintering areas emphasizes potential exposure to these disturbances. To evaluate how these disturbances impact survival and migratory connectivity, we implemented a seasonal multistate model (6 geographic areas representing 3 breeding and 3 wintering areas). We used capture and resighting data collected between 2002 -- 2019 from Northern Great Plains USA, Southern Great Plains USA, and Prairie Canada breeding areas. Individuals from all breeding areas were more likely to migrate to the western Gulf of Mexico (Texas and Mexico) than to the eastern Gulf of Mexico or the Bahamas and Atlantic coast, although individuals from breeding areas mixed across all winter areas. Hurricanes and oil spills appeared to negatively influence wintering-season survival. Despite concerns over increased frequency of these extreme disturbances, we did not detect a negative trend in adult survival throughout our study period. Mixing among individuals at wintering areas may provide a buffering effect against impacts of extreme events on any one breeding sub-population. Our results moreover suggest that understanding migratory connectivity and linking seasonal threats to population dynamics can better inform conservation strategies for Piping Plovers.}

\normaltalk{Practical Help for Avian Victims of Disaster: Are you Ready to Help Fallen Parrots after the Storm?}{Patricia J Latas, Brooke Durham}{Should you be first on the scene of a hurricane or other disaster, how can you best assist the avian victims? We present here practical methods and techniques of wildlife capture, restraint, handling, transport and containment for the ornithologist first-repsonder, using on-hand, non-technical materials and post-disaster situations.}

\normaltalk{Effects of Hurricane Maria on Abundance and Distribution of Pearly-Eyed Thrashers in the Luquillo Experimental Forest, Puerto Rico}{Alberto C Cruz-Mendoza, Laura L Fidalgo-De Souza, Frank F Rivera-Milan, Joseph M Wunderle, Jessica Ilse, Wayne J Arendt}{The Pearly-eyed Thrasher (PETH, Margarops fuscatus) is a species of concern to wildlife managers because it is a major predator of a variety of animals  and a competitor for nest sites used by the endangered Puerto Rican Parrot in the Luquillo Experimental Forest (LEF).  Although previous post-hurricane studies have documented PETH declines in abundance and distribution contraction, little is known how post-hurricane PETH abundance and distribution vary with elevation, forest type, and vegetation damage. Substantial post-hurricane changes in PETH abundance in relation to these factors were expected after Hurricane Maria (Sept. 2017), given that its damage to vegetation was found to be twice the level of damage observed after recent hurricanes on Puerto Rico.   To document effects of Maria on PETHs we compared abundance, occupancy (psi), and detection probability (p) in sites sampled three times with point counts in 1998 and again in 2018 after Maria. Baseline 1998 psi was 0.83 and detections differed among visits (p1=0.65, p2=0.77, p3=0.66). In contrast, post-Maria psi was 0.35 and detections differed between the first and both the second and third visits which were equivalent (p1=0.6, p2=p3=0.5).  These post-Maria declines in PETH abundance and distribution were consistent with previous hurricane studies.  Despite these changes, PETHs maintained their greatest abundance after the hurricane in traditional nesting areas in the Palo Colorado forests.  Post-hurricane PETH abundance and occupancy probability were highest in sites with the greatest vegetation damage, likely due to diet shift from fruit to anoles, which may have been more exposed in damaged sites.}

\normaltalk{Climatic Variability Explains Interannual Variation in Breeding Distributions of Grasshopper Sparrows}{Dylan J Smith, Trevor J Hefley, W A Boyle}{Grassland bird populations are declining, and beyond widespread habitat losses, the causes of site-level  differences in declines are unclear. Understanding the causes of such declines requires demographic  studies accounting for factors affecting birds over their full annual cycles. However, in grassland birds,  this is difficult due to notoriously low site fidelity and high mobility. Thus, understanding the patterns  and drivers of dispersal over large spatial scales can increase understanding of the causes of declines.  Mid-continental grasslands experience high inter-annual climatic variability that varies regionally, and  has been hypothesized to influence avian dispersal and settlement decisions. We determined which  climatic variables most strongly influence interannual variation in breeding distributions of Grasshopper  Sparrows (Ammodramus savannarum) from 2008-2018 in the Great Plains. We hypothesized that  distributions could be influenced by vegetation structure driven by variation in precipitation in  preceding years, and/or by direct physiological consequences of weather during spring migration and  settlement. Data from eBird revealed interannual shifts in the center of the breeding distribution of  89--213 km. Local abundance was positively associated with precipitation the preceding year, and direct  effects of temperature were weaker. We provide a new line of evidence for large scale breeding  dispersal and evidence that such movements are associated with precipitation likely acting indirectly, via  vegetation structure. Our results improve the ability to predict how grassland bird distributions will  change with changing climates; fortunately, their high mobility implies that if habitat can be conserved  in regions with projected favorable climates, the birds will likely find them.}

\normaltalk{The Influence of Climate Nest Temperature on Nestling Survival and Growth in Barn Swallows: Implications of Climate Change}{Kaitlyn M Baker, Joanna Hubbard}{According to the most recent National Climate Change assessment released in 2018, the rising surface temperatures of the Earth are undeniable and will cause dramatic changes in the future, impacting all ecological aspects of life. These impacts are particularly prevalent for barn swallows (Hirundo rustica), who trade-off overall fitness with the rate and length of incubation periods as well as the duration, quantity, and quality of the broods, which leads to a direct effect on nestling growth and success. Therefore, the primary goal of this study is to determine how variation in nest temperature and ambient temperature affects hatching success, growth, and reproductive success in barn swallows, as well as whether females adjust their behavior in response to ambient temperatures near and within the nests. We monitored 111 barn swallow nests across seven sites in Northeastern Missouri, using ambient iButton temperature loggers and dummy egg data loggers. These collected ambient and internal nest temperature. We made comparisons between  average nest temperature, hatching success, and nestling growth and survival rates. Additionally, ambient and internal temperature were compared with the length and duration of female incubation bouts. We expect that as nest temperatures increase, there will be a positive correlation in nestling success and female incubation rates until an optimal threshold value is reached. These results are essential in understanding how barn swallows can adapt to rising temperatures within their nests, and how these changes affect their hatching survival rate overall.}

\normaltalk{Can Nest Type and Climate Select Egg Shape?}{Tatiane Ld Silva, Neander M Heming, Miguel  Marini}{Several hypotheses have been proposed to explain the high variation of egg shapes across avian taxa including clutch size, nest type, allometry, flight ability, and climate conditions. Here we tested if egg shape varies between nest type (open and closed) and among climate zones in the Neotropical region. Climate zones were classified according to Köppen-Geiger maps of date of egg set collection of eggs deposited at 16 museums. Due to the small sample size for each climate zone, we merged climate classification into four broad zones related to two variables: major climate ('equatorial' and `temperate') and precipitation ('fully humid' and `winter dry'). Average elongation index (length/width) was obtained from 157 digital photos of egg sets of two open-cup (Elaenia flavogaster and Myiophobus fasciatus), and two closed (Myiozetetes similis and Pitangus sulphuratus) nesting tyrant flycatchers. Linear mixed models showed that major climate and nest type explained most of the variation in egg shape (ß = 1.38 ± 0.011). Eggs were more elongated at temperate than equatorial zone (F1,48 = 7.06, p = 0.01) and in closed than in open-cup nests (F1,13 = 6.62, p = 0.02). Precipitation, however, had no effect on egg elongation (F1,61 = 0.05, p = 0.82). Our results suggest that temperature and nest type (probably through their microclimates) can play an important role in shaping birds' eggs along species' geographic distributions. This can contribute to better understand life history strategies of birds and how these strategies might be affected by future climatic changes.}

\normaltalk{Effects of Climate on Nest Success of the Chalk-Browed Mockingbird (Mimus Saturninus) in the Woodlands of Buenos Aires}{M R Jacoby, Paola Salio, Juan C Reboreda, Vanina D Fiorini}{There is an increasing concern about the effects of climate change on the animal fitness. In birds, the climate can influence reproduction in several ways. Temperature alters incubation of eggs and thermoregulation of nestlings, while precipitation has consequences on prey availability, among others. Moreover, different phenological responses to climate change may produce a mismatch between life cycles of populations, impacting on their dynamics. To understand and predict these effects, more information about the impact of weather on the populations is required. Our goal was to investigate how temperature and precipitation influence nest success of one Neotropical bird species, the Chalk-browed Mockingbird (Mimus saturninus). The field work was carried out in the Reserve ``El Destino", Province of Buenos Aires, Argentina, during October 2019-January 2020. We evaluated the nest success of the Chalk-browed Mockingbird in relation with the maximum media temperature (Tmax) and the sum of precipitation (Pp) of the last seven days. Results show a positive relationship between the probability of nest success and the Tmax (GLM, Z= 2.630, p= 0.008). The probability of nest success increased in a 60\% for an increment of one degree of the Tmax (OR: 1.60). Therefore, temperature affects positively the reproductive success of this species. No significant correlations were found between nest success and Pp. This is the first step in the understanding of the complex relationships between climate and the reproductive success of a Neotropical bird species that will allow predicting the consequences of climate change on population dynamics.}

\normaltalk{Effects of Drought on Sex Biasing in Cassin's Sparrows (Peucaea Cassinii)}{Alexandria F Sinker, Garrett M Visser, Claire WV Ramos}{Fisherian sex ratio theory predicts that sexual selection leads to an equal investment in male and female offspring, causing most species to have equal amounts of males and females. However, some situations may arise when it benefits a female to produce females whose reproductive success may be less impacted by their condition. The Cassin's Sparrow (Peucaea cassinii) is a migratory, grassland sparrow that breeds on the plains of Colorado. Populations of Cassin's Sparrows in Colorado have been declining by approximately 3\% per year in recent years. The plains of Colorado are subject to years of extreme drought which impact nesting success of Passerine birds and may also become more frequent as the climate changes. Here we investigate the effects of drought on the primary sex ratio in Cassin's Sparrows. We predict that chicks produced during drought years will be more likely female than chicks produced during years with higher rainfall. Hatchlings will be captured from the nests and sexed genetically using W and Z chromosome markers. The results showed no significant difference between the sex ratios of the two seasons. However, there was a significant difference in the overall nest success showing that drought conditions are negatively affecting the Cassin's Sparrow. It may be that female Cassin's Sparrow are unable to manipulate their sex ratio or that conditions in the nest do not influence adult reproductive success.}

\normaltalk{Upslope Shifts in an Afrotropical Bird Community: a 39-Year Resurvey}{Monte HC Neate-Clegg, Simon Stuart, Cagan H Sekercioglu, William D Newmark}{Increasing global temperatures are expected to push highland bird populations upslope towards extinction. In the understudied Afrotropics, montane forest bird populations are restricted to isolated mountain ranges fragmented by agriculture. It is critical to assess whether birds in these ranges are shifting with climate change. In this study we quantify upslope shifts in the bird community of the Usambara Mountains of Tanzania. In 2019 we used mist nets to resurvey seven sites from 300 m to 2100 m in elevation that were originally surveyed around 1980. We surveyed each site for five days and recorded and banded all birds captured. We calculated the differences in mean elevation between the two datasets and used a Monte Carlo technique to correct for changes in elevation expected by chance alone. We then compared these corrected shifts to the rate at which temperatures have shifted upslope over time using temperature data from the Climatic Research Unit and a known local lapse rate. The bird community on average shifted upslope by 98 m. Of 29 species, 20 shifted upslope, seven downslope and two did not change. After correcting for chance shifts, the community on average shifted upslope by 94 m and this was roughly 24\% less than the 123 m predicted by rising temperatures. We also found that community composition shifted towards species found at lower elevations. Montane bird populations face local extinction if they continue to be pushed towards their elevational limits.}

\normaltalk{Climate Change Impacts for Threatened, Range-Restricted Birds of Northwest Peru's Dry Forests.}{Diego Garc\'{i}a Olaechea}{Climatic conditions worldwide have been changing significantly over the last 60 years, and more dramatic and heterogeneous changes are expected to come by the end of this century. In order to persist and survive in these changing environments, species likely will need to adapt or shift their geographical distributions to track their climatic preferences, which will result in range extensions or contractions. Strong empirical evidence points to the importance of increased temperatures in changes in species' distributions across latitudinal and altitudinal gradients. However, there is little consensus that changes in precipitation could exert an equally or even more important impact on species distributions. In this study, I report results from 40 years of data on bird species occurrence in extreme northwestern Peru -- a mix of dry, semi-deciduous and evergreen forests.  In these forests, the number of humid-affiliated bird species new to the region has increased, while, in contrast, some dry-affiliated species have been locally extirpated, and other species that were once common are now rare. The changes in species occurrences are consistent with increased precipitation in the region. Precipitation increase is hypothesized to promote the colonization of humid forest species, which are common and widely distributed, but detrimental to species associated with dry forests, which are mostly range-restricted and, in some cases, threatened. Climate projections for the region predict a further 10\% to 40\% increase in precipitation by the end of the century, which can be expected to have severe conservation implications for Peru's range-restricted dry forest species.}

\normaltalk{Evaluating Conservation Effectiveness Through Trend Analysis of Community Science Data}{Nicole L Michel, Sarah P Saunders, Timothy D Meehan, Chad B Wilsey}{Evaluation of protected area effectiveness is critical for conservation of biodiversity. Protected areas that prioritize biodiversity conservation are, optimally, located and managed in ways that support relatively large and stable or increasing wildlife populations. Yet evaluating conservation effectiveness remains a challenging endeavor. Here, we utilize an extensive community science dataset, eBird, to evaluate effectiveness of protected areas for birds across the Gulf of Mexico and Atlantic coasts of the United States. National Audubon Society identified 13 priority birds that use these coastal areas during one or more stages of the annual cycle. We extracted eBird checklists collected in breeding and wintering periods during 2007-2018. We filtered data following recommended protocols, subsampled data to produce spatiotemporally balanced samples, and used INLA to estimate population trends across the study area, repeating each subsample and analysis 100 times. We separately modelled population trends at two types of protected areas: sites where National Audubon Society conducts stewardship and/or management efforts, and protected areas managed for biodiversity or conservation, and evaluated differences in relative abundance and trend between protected areas and the Gulf and Atlantic coasts as a whole. We found that all 13 priority species showed a positive response (i.e., either higher relative abundance or more positive trend) at sites where Audubon works, versus 7 species at protected areas overall. The sites where Audubon works, combined with the stewardship and management efforts conducted there, provide effective conservation for priority coastal bird species that helps ensure conservation of priority coastal bird species.}

\normaltalk{Waterfowl Migration Phenology, Climate Change, and Implications for Management}{Barbara Frei, Ana Morales, Christian Roy}{Changes in the phenology of migrating birds, or lack-thereof, in consequence of a changing climate have distinct implications for the success, survival, and management of migratory birds worldwide. In the northern hemisphere, where springs are arriving earlier and warmer temperatures spanning later into the fall seasons, breeding birds may remain longer on their breeding grounds prior to fall migration. Over the last decades, there is evidence that many species of waterfowl are conspicuously delaying fall migration and remaining on their breeding grounds into the fall and winter months. This shift in phenology may have extensive implications, including: loss of socioeconomic opportunities by waterfowl hunters and associated tourism, reduced efficacy of monitoring and management efforts by land managers and policy makers, and conservation implications of habitat overuse in key staging areas. Using 45 years of citizen science data from Qu\'{e}bec, Canada, we identified changes in fall migration over time for waterfowl species using hierarchical generalized additive models. We explored the relative strength of regional (temperature, precipitation) and national climate (El Nino Southern Oscillation, North Atlantic Oscillation) as drivers of changes in migration phenology across various ecological groups of waterfowl. Increasing our understanding in how, and why, waterfowl migration is changing, is critical for the effective planning and management of a species group with vast ecological and cultural significance in North America. While the management of waterfowl is a success story, as demonstrated by overall recent population increases, this work highlights the need for adaptive management in a rapidly changing world.}

\normaltalk{Urban Bird Inventory Municipality Chacao Caracas Venezuela and Considerations on Population Density vs Points Monitored}{Rosa Elena E Albornoz M\'{e}ndez}{During six months from October 2019 to May 2020, 25 monitoring points were visited to inventory the species that live in the Municipality of Chacao in Caracas.  This municipality has a representation of 22 families of the 83 families that exist in the country. This represents 26\% of the total universe of bird families in Venezuela.    The families with the greatest representation of species are the Flycatchers, Macaws, Parrots and Parakeets, Kites, Hawks, eagles, and hummingbirds.    The different points monitored reveal species transit, population density and diversity.}

\normaltalk{Pay Attention. be Astonished. Tell About It./Presta atenci\'{o}n. As\'{o}mbrate. Cu\'{e}ntalo.}{Nathaniel T Wheelwright}{To be effective public ambassadors for birds and their conservation, ornithologists need to tell impactful stories about birds' lives that have emotion, clear statements of the issues, and calls to action. Supported by a Fulbright Fellowship, I taught a course in Colombia on how ordinary citizens can make short, engaging backyard natural history videos, even in a major tropical city. The course resulted in a series of videos widely-viewed on YouTube called Momentos Naturales Colombia (https://research.bowdoin.edu/momentos-naturales-colombia/). I describe that project and emphasize the importance of publishing our work in the language of our host country.}

\normaltalk{Improving Resightings of Two-Banded Plovers Flag Codes by using Digital Photo-Identification in Patagonia, Argentina}{Glenda D. Hevia, Carla D Fiorito, Dallas Jordan, No\'{e} Terorde, Frederick D. Jordan, Dar\'{i}o H. Podest\'{a}, Agust\'{i}n Esmoris, Rub\'{e}n F. Villareal, Luis Burgue\~{n}o, Nicol\'{a}s S. Corbal\'{a}n, Mart\'{i}n A. Chaparro, Marcelo Bertellotti, Ver\'{o}nica L D'Amico}{Wildlife photography and digital photo-identification is a growing field that enhances many ecological features of avian population studies. Traditionally, binoculars and telescopes have been used to collect resighting data from marked birds, but the use of a non-invasive and complementary tool, such as a camera with telephoto lenses, has recently become a commonplace practice to identify flag codes on small banded shorebirds. We conducted surveys to resight marked Two-banded Plover (Charadrius falklandicus) adults between September and December 2017-2019 in northern Patagonia, Argentina, in order to explore site fidelity at their breeding grounds. Plovers were banded during the austral spring and summer using an orange flag with alphanumeric code placed on the tibia and a metal ring in the tarsus according to guidelines from the Pan American Shorebird Protocol. Banded plovers' detection and the ability to read flag codes in the field can be low when using binoculars and/or telescopes due to Patagonia's characteristically strong wind conditions, which often limit field work activities. Hence, we used cameras with telephoto zoom lenses to increase accuracy when reading flag codes, allowing for banded plover's individual identification. This fieldwork was possible due to the key participation of wildlife photographers, who contributed significantly with their time, expertise, and knowledge. In addition, several of the photographers belong to the Argentine Association of Nature Photographers (AFONA: www.afona.com.ar). Lastly, we strongly encourage the use of this technique as a complementary tool to improve banded bird's detection and to verify flag codes.  Resumen: La fotograf\'{i}a de vida silvestre y la identificaci\'{o}n a trav\'{e}s de fotos digitales es un campo en crecimiento que mejora muchas caracter\'{i}sticas ecol\'{o}gicas de los estudios sobre las poblaciones de aves. Tradicionalmente, los binoculares y telescopios se han utilizado para recopilar registros observacionales de las aves marcadas, pero el uso de una herramienta no invasiva y complementaria, como una c\'{a}mara con teleobjetivos, se ha convertido recientemente en una pr\'{a}ctica común para identificar c\'{o}digos de banderillas en aves playeras marcadas. Entre septiembre y diciembre de 2017-2019, realizamos relevamientos para reavistar adultos de chorlos doble collar (Charadrius falklandicus) en el norte de Patagonia, Argentina, para explorar la fidelidad a sus sitios de reproducci\'{o}n. Durante la primavera y el verano austral, los chorlos doble collar fueron anillados con un anillo de metal en el tarso y una banderilla naranja con c\'{o}digo alfanum\'{e}rico en la tibia, siguiendo los lineamientos del Protocolo Panamericano de Aves Playeras. La detecci\'{o}n de las aves anilladas y la capacidad de leer los c\'{o}digos de las banderillas en el campo puede ser baja cuando únicamente se usan binoculares y/o telescopio, puesto que las condiciones clim\'{a}ticas, como los fuertes vientos caracter\'{i}sticos de la Patagonia, a menudo limitan las actividades de trabajo de campo. Por lo tanto, utilizamos c\'{a}maras con teleobjetivos para aumentar la precisi\'{o}n al leer los c\'{o}digos de banderilla, lo que permiti\'{o} la identificaci\'{o}n individual de cada chorlo marcado. Este trabajo de campo fue posible debido a la participaci\'{o}n clave de fot\'{o}grafos de vida silvestre, quienes contribuyeron significativamente con su tiempo, experiencia y conocimiento. Adem\'{a}s, varios de ellos pertenecen a la Asociaci\'{o}n Argentina de Fot\'{o}grafos de Naturaleza (AFONA: www.afona.com.ar). Por último, recomendamos fuertemente el uso de esta t\'{e}cnica como una herramienta complementaria para mejorar la detecci\'{o}n de las aves marcadas y verificar los c\'{o}digos de las banderillas.}

\normaltalk{Citizen Science as an Alternative to Study Altitudinal Migration in Neotropical Birds}{Santiago A Herrera, Camila G\'{o}mez, Carlos D Cadena}{Altitudinal migration is a phenomenon present in a large number of bird species that inhabit mountain environments around the world. Only in the Neotropics it is thought that 543 species may be carrying out this type of movement. In Colombia, some of the most complex mountain systems in the Neotropics converge and 43.8\% of the birds apparently move seasonally in the elevation gradients. However, it is surprising how little is known about the mountains where altitudinal migrations occur, the factors that drive them and the community of species that carry them out in Colombia.  One of the main reasons why there are few studies on altitudinal migration in the Neotropics, and more specifically in Colombia, is the great difficulty of following the movements of birds in complex mountain landscapes over relatively long and continuous time scales. However, data from citizen observations throughout the country allow us to multiply the effort and potentially record the simultaneous movements of mountain birds along the mountain systems present in Colombia. This study evaluated the usefulness of participatory science data obtained from the eBird platform to establish a pattern to predict the altitudinal movements of mountain species in different gradients of the Colombian Andes. It was possible to demonstrate the great potential that this participatory science tool represents for studies of altitudinal migration. However, it is necessary to increase the amount of useful data in Colombia in order to have a better resolution of the movements of the species along the gradients.  La migraci\'{o}n altitudinal es un fen\'{o}meno presente en una gran cantidad de especies de aves que habitan ambientes monta\~{n}osos en todo el mundo. Solo en el Neotr\'{o}pico se piensa que 543 especies pueden estar realizando este tipo de movimientos, que dependen de diferentes factores dentro de los cuales est\'{a}n la disponibilidad de alimento, las variaciones en el clima y los eventos de depredaci\'{o}n, entre otros. En Colombia confluyen algunos de los sistemas monta\~{n}osos m\'{a}s complejos del neotr\'{o}pico y 43,8\% de las aves que aparentemente se mueven estacionalmente en los gradientes de elevaci\'{o}n. Sin embargo, es sorprendente lo poco que se conoce sobre las monta\~{n}as en donde ocurren las migraciones altitudinales, los factores que las impulsan y la comunidad de especies que las llevan a cabo en Colombia.  Una de las principales razones por las que existen pocos estudios sobre la migraci\'{o}n altitudinal en el neotr\'{o}pico y m\'{a}s espec\'{i}ficamente en Colombia, es la gran dificultad de seguir los movimientos de las aves en paisajes complejos de monta\~{n}a a lo largo de escalas relativamente largas y continuas de tiempo. Sin embargo, los datos provenientes de observaciones de ciudadanos en todo el pa\'{i}s, permiten multiplicar el esfuerzo y potencialmente, registrar los movimientos simult\'{a}neos de las aves de monta\~{n}a a lo largo de los sistemas de monta\~{n}as presentes en Colombia. En este estudio se evalu\'{o} la utilidad de los datos de ciencia participativa obtenidos de la plataforma eBird para establecer un patr\'{o}n que permita predecir los movimientos altitudinales de las especies de monta\~{n}a en diferentes gradientes de la cordillera de los Andes colombiana. A partir de la observaci\'{o}n de los datos para especies en Costa Rica de las cuales se conoce con claridad los movimientos que realizan a lo largo de un gradiente ya estudiado, se logr\'{o} demostrar el gran potencial que esta herramienta de ciencia participativa representa para estudios de migraci\'{o}n altitudinal. Sin embargo, es necesario aumentar la cantidad de datos útiles en Colombia para de esta manera tener una mejor resoluci\'{o}n de los movimientos de las especies en los gradientes.}

\normaltalk{With Atlantic Forest Conversion to Open Farmland, Cavity-Nesting Birds Increasingly Rely on Woodpecker Excavation in Snags}{Eugenia Bianca Bonaparte, Jos\'{e} Tom\'{a}s Ibarra, Kristina L Cockle}{Understanding nest-site selection is critical to conserving tree-cavity-nesting wildlife, but nest-site use may vary across landscapes. We examine variation in the characteristics of trees and cavities used by cavity-nesting birds from globally-threatened primary Atlantic Forest to open farmland. We used a stratified case-control design and 20 random plots to assess variation in characteristics of trees and cavities (used and available) across gradients of canopy cover and distance to forest edge in Argentina. For secondary cavity nesters, nest cavities were more likely to occur in larger-diameter trees across all stand conditions, but more likely to occur in snags as canopy cover declined (in open farmland; n = 123 nest trees). For primary excavators, nest cavities were more likely to occur in larger snags, regardless of stand conditions (n = 54 nest trees). Available cavities declined from 4/ha in primary forest to 0.4/ha in open farmland. Available cavities and those used by secondary-cavity nesters were increasingly of excavated origin in open farmland, which indicates that avian excavation may partly compensate for the loss of decay-formed on farms. As forest landscapes shift toward a predominance of agroecosystems, snags and primary cavity nesters may take on important roles in conserving cavity-nesting communities and their ecosystem functions. However, nest cavities declined in height and depth, and increased in entrance size toward open farmland, raising the possibility that birds increasingly use suboptimal cavities as forest cover declines. Conservation of cavity-nesting birds should include retention of large trees and recruitment of native young trees across all landscapes but especially on farms.Comprender la selecci\'{o}n de sitios de nidificaci\'{o}n es fundamental para conservar las especies que anidan en cavidades de \'{a}rboles, pero el uso de sitios de nidificaci\'{o}n puede variar según el paisaje. En la selva Atl\'{a}ntica de Argentina, examinamos la variaci\'{o}n en las caracter\'{i}sticas \'{a}rboles y cavidades utilizados por aves que anidan en cavidades, desde selva primaria hasta chacras abiertas. Utilizamos un dise\~{n}o estratificado caso-control y 20 parcelas aleatorias para evaluar la variaci\'{o}n en las caracter\'{i}sticas de \'{a}rboles y cavidades (usadas y disponibles) a trav\'{e}s de dos gradientes, cobertura del dosel y distancia al borde de selva m\'{a}s cercano. Para aves no excavadoras, la ocurrencia de nidos fue mayor en \'{a}rboles de mayor di\'{a}metro en todas las condiciones del sitio, pero su ocurrencia en \'{a}rboles muertos aument\'{o} a medida que la cubierta del dosel disminuy\'{o} (en chacras abiertas; n = 123 \'{a}rboles nidos). Para aves excavadoras, las cavidades nido ocurrieron m\'{a}s en \'{a}rboles muertos grandes, independientemente de las condiciones del sitio (n = 54 \'{a}rboles de nidos). El número de cavidades disponibles disminuy\'{o} de 4/ha en selva primaria a 0.4/ha en chacras abiertas. La proporci\'{o}n de cavidades excavadas aument\'{o} en chacras, tanto para cavidades disponibles como aquellas utilizadas por aves no excavadoras, lo que indica que la producci\'{o}n de cavidades por aves excavadoras podr\'{i}a compensar la p\'{e}rdida de cavidades formadas por degradaci\'{o}n cuando se talan \'{a}rboles grandes. A medida que los bosques cambian hacia un predominio de agroecosistemas, los \'{a}rboles muertos y las aves excavadoras pueden asumir papeles importantes en la conservaci\'{o}n de las comunidades que anidan en las cavidades y sus funciones ecosist\'{e}micas. Sin embargo, en chacras, disminuy\'{o} la altura y profundidad de las cavidades usadas y aument\'{o} su tama\~{n}o de entrada, lo que plantea la posibilidad de que las aves utilicen cavidades sub\'{o}ptimas a medida que disminuye la cobertura de selva. La conservaci\'{o}n de la comunidad de aves que anidan en cavidades debe incluir la retenci\'{o}n de \'{a}rboles grandes y el reclutamiento de \'{a}rboles j\'{o}venes nativos en todos los paisajes, pero especialmente en chacras.}

\normaltalk{Community Training in Bird Monitoring is an Effective Tool in Increasing Public Engagement in Conservation}{Quincy Augustine}{Apart from persons directly affected by the policy that may destroy natural habitats, public advocacy for conservation in Grenada is low. For taxa that do not have a clear economic value, citizen science may improve public engagement in the protection of their habitat. Consequently, we asked: could training in bird monitoring improve attitudes on bird habitat protection? We conducted two 4-day trainings on  bird monitoring with 16 persons in Grenada. We taught participants how to: 1) identify common wetland and terrestrial birds, and 2) administer bird counts. Based on surveys conducted before and after the training, we observed: 1) a 10 \% increase in the number of individuals that felt that bird conservation was very important, and 2) a 20\% increase in people who felt confident leading bird counts. While we recognize that our conservation-bias in participants meant that we did not see large changes in attitudes to bird conservation, we believe that by expanding this training, we may see greater public interest in bird conservation.}

\normaltalk{The Value of Supporting Locally Led Co-Created Participatory Science Projects Integrating Culture, the Arts and Stewardship Through the Americas.}{Karen Purcell, Marilú L\'{o}pez-Fretts, Josmar Marquez}{Celebrate Urban Birds (CUBs), a participatory science project from the Cornell Lab of Ornithology, promotes birdwatching and data collection while integrating connections to local culture, greening and the arts.  For over ten years, children, youth, adults and elders throughout the Americas have participated in and experienced appreciation for nature and conservation through community-led, co-created events. The projects focus on locally selected species across the US, Canada, Puerto Rico, Mexico, Panama, Perú, Chile, Guatemala and Colombia. CUBs strives to co-create inclusive, equity-based citizen science projects that serve communities that have been historically excluded from birding and citizen science. The project has purposely provided greater access to the sciences and birding and to date, more than half a million people and 14,000 organizations have participated. These projects include printed materials that are distributed free of charge by the collaborators and online resources that have been compiled from the participating communities at www.Celebrateurbanbirds.org. Through research-based strategies the project is increasing pride of local birds and the environment. Mini-grants focused on greening, community wellbeing, and the arts are used to broaden participation and strengthen approaches led by local communities throughout the continent. We share project methodologies and impacts.}

\normaltalk{Direct Assessment Shows Little Effect of Science Communication Training}{Margaret A Rubega, Kevin R Burgio, AAndrew M MacDonald, Anne Oeldorf-Hirsch, Robert S Capers, Robert Wyss}{Ornithology, along with other scientific disciplines, has made communicating with the public a priority in the interests of public education, and avian conservation and funding policy. Increasingly, science communication training programs are being widely offered to ornithologists, especially to graduate students. Graduate science communication training programs vary widely, and there is no standard of evidence for whether training actually works, or to compare training approaches. The use of trainee self-evaluation as program assessment is widespread, despite evidence that it is unreliable. In order to develop a rigorous, transferrable approach to science communication training assessment we used a semester-long graduate science communication course as a fully controlled experiment, using a large undergraduate research pool as a test audience. Evaluators provided fully independent scores of the effectiveness of the standardized communications of trainees and their matched, untrained controls, both before and after the training period. Bayesian analysis of the data showed very small gains in communication skills of trainees, and no difference from untrained controls. High variance in scores suggests little agreement among audience members on what constitutes ``good" communication. Our results strongly suggest that, while content knowledge of trainees may increase, even a 15 week long, intensive, best-practices, active-learning training program had little short term effect on science communication skills. We suggest that the approaches, time and resources applied to training ornithologists in science communication need re-examination.}

\normaltalk{Sustainable Bioenergy Production and Integrated Valuation of Ecosystem Services}{Duston R Duffie, Scott A Rush}{As fossil energy use declines, biofuel production has increased to meet novel demands. The production of biofuels is directly linked to land use change as agricultural land use may shift toward growing crops used in biofuel production. While increased production of biofuels is inheritably an ecosystem service to mitigate changes in fuel sources, the conversion of lands to agriculture to meet these needs may lead to ecosystem disservices by decreasing available habitats. To assess potential environmental services, or disservices, of biofuel production, we will evaluate shifts in species distributions and relative abundance under predicted land use and climate change scenarios in the Mississippi River basin. The Mississippi River basin partially covers 32 states in the central United States. We will incorporate North American Breeding Bird Survey species counts from approximately 2,400 routes run at least once from 2008 -- 2018 with landscape level climate and land use variables to predict the current and future species distributions and relative abundance of breeding bird species. These models will be spatially explicit with predictions extrapolated for areas where breeding bird survey routes do not currently occur. As agricultural land use and composition change to meet the demands of biofuel production, we expect more agricultural lands will be converted to monocultures of ethanol producing crops. We predict demoted avian diversity in counties where increase of monoculture crop production occurs to support ethanol needs. Species guilds most likely to be affected by these changes are those breeding in open grasslands.}

\normaltalk{The Role of Songbirds in Agricultural Insect Pest Control}{Samuel J Mayne, David King, Joseph Elkinton, Jeremy Andersen}{Songbird control of agricultural pest populations through predation has the potential to aid small-scale farming operations, representing a rare win-win situation where wildlife conservation directly benefits human productivity.  Agricultural lands account for nearly half of global land use, and chemically intensive, industrialized farming poses a serious threat to many wildlife populations.  Songbird populations on small, diversified farms in New England may play an important role in mitigating pest damage and outbreaks; however, what bird species consume the most agricultural pests and the level of pest suppression is not known.  Using gene metabarcoding of songbird feces, we will analyze the diets of common songbird species to determine their role in insect pest reduction.  We will also evaluate the magnitude of pest reduction services rendered by songbirds through experimental elimination of songbird insect predation in row crops.  We expect that songbird predation significantly decreases pest populations and crop damage, and that crop pests constitute a larger portion of the diets of generalist and insectivore species than other local songbird species.  With these results, farmers and wildlife managers can adjust their land management practices to enhance songbird pest removal services on local and landscape scales.  By quantifying songbird ecosystem services, policy makers and farmers can more accurately compare the costs of intensive, homogenized farming and diversified, wildlife-supporting agriculture.}

\normaltalk{Using Citizen Science to Assess the Ecological Impacts of Urbanization on Birds}{Sebastian Moreno, Paige S Warren, Susannah B Lerman}{Urbanization is a dominant process changing land cover, leading to population declines for many bird species. Studying these populations in cities is essential to understanding these declines. Urban areas pose unique challenges for ecological research, one of which is access to the private yards that make up a large portion of the available green space. Citizen science programs can mitigate this challenge by engaging the public in science while gaining access to private land. Though citizen science is an attractive method for increasing data collection, many programs often attract those who have social and financial capital, excluding underserved communities. We present a plan to simultaneously increase understanding of the ecological impacts supplemental feeding has on birds while increasing participation of underserved communities in citizen science. Our model intends to expose community members of Springfield, Massachusetts, the third largest city in the state with over half the population identifying as LatinX. Members will attend banding demonstrations in local greenspaces and workshops overviewing birding etiquette, identification, and habitat features that benefit wildlife. Participants will make bird feeder observations in private and public greenspace that will be complemented by personal training sessions to improve observational skills. We expect to evaluate what approaches to engage participants work best and quantify impacts supplemental feeding on birds. This framework will create a conduit for more members of underrepresented and underserved communities to participate in well established citizen science programs and provide additional insights into ecological dynamics occurring in urbanized areas.}

\normaltalk{Use of a Spanish Language Poster for Outreach in a Diverse Community (Uso De Un p\'{o}ster en espa\~{n}ol Para divulgaci\'{o}n en Una Comunidad Diversa)}{Lizbeth Gasga, Alina Bhatti, Chemyn Cortez, Arwa Al Aqori, Itab Nagi, Tricia A Van Laar}{A 2016 survey by the US Fish and Wildlife Service described the average birder as older than 45, white, college educated, with an above average income which does not reflect our campus or community diversity. California State University, Fresno (Fresno State) is a minority serving, primarily undergraduate institution located in the Central Valley of California. At Fresno State, 78\% of undergraduates (n=22,189) fit into one or more of the following categories: under-represented minority (URM), first generation, and/or Pell grant recipients. In the Biology department, 52\% (566/1085) of students self-identified as URM with the majority (543) identifying as Hispanic/Latino. To attract a diverse population of birders, the students in enrolled in Field Ornithology at Fresno State generated a poster of the birds found on our campus. We translated the poster into Spanish and have used it with bilingual student presenters during outreach events. Currently, there is a single Spanish-language field guide to the birds of North America, so there is a clear need. The English and Spanish posters were also provided to the Fresno Audubon Society who plan to display them during local beginning birding classes and at booths during their outreach events. Our success with this poster project has generated interest to lead bilingual bird walks for K-12 students on our campus to develop a population of young bird researchers and/or birders. Therefore, by making birding accessible to a diverse population of students, we have made ornithology and the birding hobby accessible to our local diverse community.    En Espanol:  En el 2016, una encuesta realizada por la agencia Estadounidense de Pesca y Vida Silvestre, USFWS por sus siglas en ingl\'{e}s, describi\'{o} al observador(a) de aves promedio como una persona mayor de 45 a\~{n}os, anglosaj\'{o}n, con t\'{i}tulo universitario, y con ingresos mayores al promedio, lo cual no refleja nuestro campus universitario ni la diversidad de nuestra comunidad. La Universidad Estatal de California, Fresno (Fresno State) es una instituci\'{o}n dedicada a atender minor\'{i}as, la cual ofrece primordialmente licenciaturas y esta ubicada en el Valle Central de California. En Fresno State, 78\% de los universitarios (n=22,189) entran en una o m\'{a}s de las siguientes categor\'{i}as: minor\'{i}a subrepresentada, primera generaci\'{o}n y/o beneficiarios de la beca federal Pell. En la facultad de Biolog\'{i}a, 52\% (566/1085) de los estudiantes se autodescriben como minor\'{i}a subrepresentada, y la mayor\'{i}a (543) se identifican como Hispanos/Latinos. Para atraer a una poblaci\'{o}n diversa de observadores de aves, los estudiantes inscritos en el campo de Ornitolog\'{i}a en Fresno State crearon un p\'{o}ster de las aves que se encuentran en nuestro campus.  Tradujimos el p\'{o}ster en espa\~{n}ol y se ha utilizado por estudiantes bilingües para exponer durante eventos de extensi\'{o}n educativa. Actualmente, solo existe una sola gu\'{i}a de campo en espa\~{n}ol sobre las aves de America del Norte, lo cual es una clara necesidad. Los p\'{o}steres en ingl\'{e}s y espa\~{n}ol fueron compartidos con la Sociedad Audobon de Fresno, la cual planea exhibirlos durante clases locales para observadores de aves principiantes y en eventos de extensi\'{o}n educativa. El \'{e}xito con este p\'{o}ster ha generado inter\'{e}s en dirigir caminatas bilingües para observar aves con estudiantes del K-12 nivel acad\'{e}mico en nuestro campus para crear una poblaci\'{o}n joven de ornit\'{o}logos u observadores de aves. Por lo tanto, al hacer la observaci\'{o}n de aves mas accesible para una poblaci\'{o}n diversa de estudiantes, hemos hecho de la ornitolog\'{i}a un hobby accesible para nuestra diversa comunidad.}

\normaltalk{Festival of Migratory Birds and Shorebirds of Coche Island, Nueva Esparta, Venezuela.}{Daniel Serva, Josmar E Marquez}{We encourage the protection of birds and ecosystems of the island through the cultural traditions of the inhabitants; Historically, ``Las Diversiones" (Cultural Festival) is celebrated on Coche island with dances, songs and dramatizations, where the inhabitants celebrate their identity and express the daily life on the island. Based on this premise, we created this festival as an outreach strategy for bird conservation, which was held for the first time in November 2018, and continued with a second edition in December 2019. In these two editions more than 3.000 people (mostly children) have participated. Activities such as workshops, theater presentations, dances, murals, beach cleaning, bird watching, and all sorts of free artistic activities have been carried out. Resulting in an effective strategy for the integration of local communities in conservation.}

\normaltalk{Our Coffee Our Birds: Birds as Indicators of Environmental Sustainability in Costa Rican Coffee Farms}{Alejandro Quesada-Murillo}{Coffee farms that follow environmental sustainability programs promote the integrity of ecosystems, agricultural sustainability, the resilience of food systems, and adaptation to climate change. Efforts like maintaining shade and reduce the use of agrochemicals, have positive impacts on the bird species that use the coffee plantations. Starting in 2019, in San Ram\'{o}n de Alajuela, Costa Rica, we carried out the ``Our Coffee Our Birds" project in coffee plantations that follow the Nespresso AAA environmental sustainability programs. In this project, we monitored the bird species richness using SWIFT autonomous recorders. Besides, we conducted count points using PROALAS protocol. The information generated in these monitoring is also the basis for conducting environmental education and scientific divulgation campaigns with coffee producers, as well as with local schools. Learning about the birds that live on their farms helps raise awareness among producers and families about how bird diversity can be favored by establishing responsible agricultural practices. We are monitoring a total of 40 farms using previous methods. In addition, we conducted workshops with the farm owners and their workers, about birds that inhabit coffee plantations and how to use Ebird and Merlin applications. The monitoring has generated more than 12,000 hours of recording and 190 bird species detected. At the same time, eight local schools were visited, to teach kids between 8 and 10 years old about birds that inhabit their community and coffee plantations, using the ``Friends of coffee" approach.}

\normaltalk{Using MoSI Protocols as a Basis for Overwintering Avian Studies in a Private, Working Forest}{Rebecca D Bracken, Daniel U Greene, Darren A Miller, Scott A Rush}{Approximately 90\% of Mississippi's 8 million hectares of forestland are privately owned. This forested landscape mosaic consists of forest patches of varying stand ages and vegetative species composition.  Although extensive research has been conducted on birds during breeding season within private forests, less information exists about wintering bird communities on these landscapes.  Our study objectives are to evaluate, 1)how maintaining varying stand ages supports bird communities during the non-breeding season by questioning how species segregate among loblolly pine (Pinus taeda) stands of differing stage classes and, 2) if birds respond to food resources or forest structure during winter.  At present, we have conducted three 2-day mist netting sessions during our pilot season using the MoSI (Monitoreo de sobrevivencia invernal -- Monitoring overwinter survival program) protocols as a basis for study design.  We placed mist nets in young, mid-age, and mature stands (0 -- 5, 6 -- 15, and 16 -- 30 years since planting, respectively), banded captured individuals, and collected a blood sample from each bird.  We will use blood samples to evaluate dietary origins of nutrients assimilated by assessing ratios of light to heavy isotopes of nitrogen and carbon in each individual.  To date, we have captured 23 individuals of 7 species, with Ruby-crowned Kinglets (Regulus calendula) comprising approximately half of all captures.  Captures between the mature and mid-age stands were equal, with limited captures in the young stand.  Our data will be useful in determining ecological relationships of birds using private, working forests during winter in the southeastern U.S.}

\normaltalk{The First Charadriiform Bird from the Eocene Green River Formation Clarifies the Pattern and Timing of a Major Neoavian Radiation.}{Grace Musser, Julia A Clarke}{The Paleogene fossil record of Charadriiformes (shorebirds) is scarce and largely consists of fragmentary single elements. The only known Eocene charadriiform represented by a partial skeleton is Scandiavis mikkelseni from the earliest Eocene of Denmark. Here, we describe a new charadriiform species from the early Eocene Green River Formation of North America comprising a well-preserved partial skeleton and feathers. The partial skeleton also lacks a sternum, shoulder girdle and forelimbs; however, it exhibits several key features consistent with those of Charadriiformes. To place this taxon we built upon a previously-published morphological dataset to create a final matrix consisting of over 600 characters and 60 taxa, including S. mikkelseni and the earliest known charadriiform taxa represented by single elements. These more fragmentary records comprise two distal humeri from the early Eocene Naranbulag Formation of Mongolia and Nanjemoy Formation of Virginia. Our phylogenetic analyses recover the new taxon most consistently as a stem charadriiform with S. mikkelseni. Placement of the new species as a stem charadriiform is sensitive to relationships among major avian subclades recovered by recent molecular-based phylogenies. When using constraints that do not recover Charadriiformes and Gruiformes (rails, cranes and allies) as sister-taxa, the new taxon and S. mikkelseni are recovered within basal Gruiformes.  Both Paleogene fossil humeri are recovered within crown Charadriiformes across all analyses. If placement of both the stem and crown charadriiform taxa is correct, this may indicate that recent divergence time analyses have underestimated the crown age of another major crown avian subclade.}

\normaltalk{Systematics of a Radiation of Neotropical Suboscines (Aves: Thamnophilidae: Epinecrophylla)}{Oscar Johnson, Jeffrey T Howard, Robb T Brumfield}{The ``stipple-throated" antwrens of the genus Epinecrophylla (Aves: Thamnophilidae) are represented by eight species primarily found in the lowlands of the Amazon Basin and the Guiana Shield. The genus has a long and convoluted taxonomic history, and many attempts have been made to address the taxonomy and systematics of the group. Here we employ DNA sequences from ultraconserved elements (UCE) to provide both the most comprehensive subspecies-level phylogeny of Epinecrophylla antwrens and the first population-level genetic analyses for most species in the genus. Most of our analyses are robust to a wide selection of phylogenetic and population genetic methods, but we found that even with thousands of loci we were unable to confidently place the western Amazonian taxon pyrrhonota in our phylogeny. We uncovered phylogenetic relationships between taxa and patterns of population structure that are discordant with both morphology and current taxonomy. In our phylogenetic analyses we found deep genetic breaks between taxa currently regarded as species, and in two groups we found paraphyly at the species and subspecies levels. Our population genetics analyses showed extensive admixture between some taxa despite their deep genetic divergence. We present a revised taxonomy for the group, discuss the biogeographic patterns that we uncover, and suggest potential aspects that warrant further study.}

\normaltalk{Glacial Cycles Promote the Diversification of the Southern Lapwing (Vanellus Chilensis) and a Secondary Contact Zone}{Bel\'{e}n Bukowski, Leonardo Campagna, Mar\'{i}a J Rodriguez Cajarville, Gustavo S Cabanne, Pablo L Tubaro, Dar\'{i}o A Lijtmaer}{The diversification patterns of the birds of southern South America, and particularly Patagonia (southern Argentina and Chile), have been less studied than in other regions of the Neotropics. We analyzed the evolutionary history of the Southern Lapwing (Vanellus chilensis), a widespread species characteristic of Neotropical open habitats, focusing on the differentiated mitochondrial lineage previously found in Patagonia and the role played by glacial cycles in its differentiation. We performed a phylogenetic/phylogeographic analysis including the four recognized subspecies, using mitochondrial and genomic DNA, and also studied their vocalizations. The results suggest that the main diversification event within V. chilensis occurred approximately 600,000 years ago, resulting in a marked phylogeographic structure with 1.5\% mitochondrial genetic distance between the Patagonian lineage and the rest of the species. The genomic analyses confirmed this differentiation, showing considerable gene flow between the two lineages and supporting the presence of a contact zone in northern Patagonia. Vocalization were significantly different between the two lineages, and those recorded in the contact zone were intermediate in their temporal and frequency characteristics. Taken together, our results confirm that the populations from southern South America are clearly differentiated from the rest of the species and suggest that this differentiation occurred due to the isolation of the Patagonian populations in refugia during the Pleistocene glaciations, followed by an expansion from these refugia and the establishment of a zone of secondary contact in northern Patagonia. Our study highlights the importance of glacial cycles in southern South America as drivers of avian diversification.}

\normaltalk{Drivers of Differences in Migratory Timing in Neotropical Migrants - a Common Yellowthroat Case Study}{Taylor Bobowski, Christen Bossu, Kristen Ruegg}{Migration, the seasonal movement of animals, allows organisms to exploit seasonally favorable conditions across a large geographic area. Despite this common purpose, distinct populations of the same species can exhibit dramatically different patterns of movement. Large, population-specific differences in timing of migration in neotropical migrants have been shown, but the underlying causes of these patterns are still unclear. Here, we test the hypothesis that populations migrate at different times due to selection on genes regulating migratory timing, which differs between distinct breeding populations. We use Common Yellowthroats as a model species to test this hypothesis due to their large breeding range, clear population differentiation, and a library of over 2000 samples collected over an eleven year period during the spring migration. We assess key sites along the migratory corridor in order to test for  population specific differences in migratory timing and survey genes linked to migratory behavior.}

\normaltalk{The Effects of Ionizing Radiation on Mutation Rate in Fukushima Barn Swallows (Hirundo Rustica)}{Caitlin M Doherty, Wataru Kitamura, Timothy A Mousseau, Andrea Bonisoli-Alquati}{Studies have shown that exposure to ionizing radiation can result in changes to morphology, life history and mutation rate in different animal species. The 2011 Fukushima-Daiichi nuclear accident marked the second-largest radionuclide release in history, resulting in population declines of bird species in contaminated regions. Mutation accumulation as a result of chronic radiation exposure of Chernobyl birds has been invoked to explain their declines. Previous studies in Chernobyl barn swallows (Hirundo rustica) also documented higher rates of germline mutations. Yet, no investigations exist of genome-wide mutation rate in birds exposed to ionizing radiation in Fukushima. Here we provide the first explicit test of mutation rate. We are sequencing the genomes of 12 birds from four two-generation pedigrees. We will align them to the highly contiguous barn swallow reference genome produced by our lab. Strict bioinformatic criteria will be used for the detection of de novo mutations and subsequent estimation of mutation rate, across the parental and first filial generations. Ionizing radiation can induce mutations through direct energy deposition on DNA molecules and oxidative DNA damage, causing small to large-scale deletions when damage is incorrectly repaired. We expect increased mutation accumulation in Fukushima barn swallows. Knowledge of the effects of nuclear disasters at the genomic level will help provide a mechanism for the observed phenotypic changes in birds exposed to radiation, and clarify the potential evolutionary consequences.}

\normaltalk{Comparative Analysis of Repeated Sequences in Avian Genomes}{Ramiro Echeverria, Peter Arensburger}{Transposable elements are selfish genetic elements capable of movement within a host genome via a transposition or retro-transposition mechanism. As they move and become inactivated by a variety of mechanisms (e.g. inactivation by the host) they may leave a footprint specific to their taxonomic type. Birds appear to be somewhat static in their chromosomal evolution, with low rates of gene gains and losses in multiple gene families (Zhang et. al. 2014). Phylogenetic studies have suggested that there has been reduced transposon activity in birds compared to other vertebrate genomes. Avian genomes have experienced linage-specific erosion of repetitive sequences, gene loss, and large segment deletions (Zhang et. al. 2014). All of these have been linked to a variety of transposable elements, emphasizing the importance of identifying and comparing such elements and to compare transposable element sequences between bird lineages. In order to understand the abundance and compare sequences of transposable elements within all birds, a broad sample of the avian phylogeny needs to be analyzed. The avian phylogeny can be broken down into two major clades. Neognathae which consists of the Neoaves clade and Galloanseres clade. The Neoaves consists of Passerea and Columbea clades.The Palaeognathae are basal in the avian phylogeny. To better understand how transposable elements have affected avian phylogeny we conducted a bioinformatic analysis of a variety of bird species spanning basal primitive bird species to the modern derived bird species in the phylogenetic tree.}

\normaltalk{An Investigation into Phenotypic Variance and Microevolution in Introduced Populations of Feral North American Rock Pigeons (Columba Livia).}{Sarah Rackowski}{In captivity as well as in the wild, the rock pigeon (Columba livia) exhibits variation in anatomical traits.  In their native range, wild rock pigeons have diversified into 12 subspecies and in captivity, hundreds of breeds have been generated by artificial selection. Rock pigeons have populations throughout much of the world, and in North America, feral pigeons have been shown to be phenotypically and genetically distinct from their European counterparts. However, little is known about variation in rock pigeon traits among different North American sub-populations. To characterize this regional trait variation, I took anatomical measurements, with specimens coming from the Eastern Seaboard and from  Illinois. These sites were then grouped as follows: (1) Chicago, (2) Maine and Massachusetts,  (3) New York City to Washington D.C, (4)North Carolina to the Florida keys. Specimens from Region two showed significantly increased length of leg feathering compared to other groups, one sub-population in region 2 also showed the ``roller" trait  Specimens from Region 3 has significantly shorter tails than other groups) and when Regions  3 and 4 were combined, they showed a significantly longer bill length than regions 1 and 2.Coloration also varied from group to group, with ``light" morphs making up around 60\% of Region 3 and ``checkered" morphs making up around 60\% of region 2.}

\normaltalk{Elevational Variation in Body Size and Hematological Parameters of Andean Coots from the High Andes of Peru}{Alexis D\'{i}az, Kevin G McCracken, Emil Bautista}{Variation in environmental conditions can induce a diverse pattern of adaptation and phenotypic plasticity among populations that breed along steep elevational gradients. Patterns of morphological variation have been studied by testing Bergmann's and Allen's ecogeographic rules, which refer to changes in body and appendage sizes as a result of temperature, which changes with altitude. Moreover, many organisms adapted to high-altitude exhibit low hematological responses as well as other adaptations (i.e. functional mutations in hemoglobin variants) that promote survival in such hypoxic environments. We compared ten morphometric measures and three hematological parameters of Andean coot's populations sampled from low (LA)- and high (HA)-altitude regions in Peru. We collected 98 adults and analyzed patterns between sex and altitude for those populations. Both males and females from the HA had significantly greater body mass, and longer wings, tarsus length, and tail than those from LA. Thus, measurement related to body size (body mass) variation supported Bergmann's rule, whereas appendage size (wings, tarsus length, and tail) variation did not support Allen's rule. Although analysis of the ratio of wing/bill/tarsus length to body mass did yield a significant difference with altitude, supporting the hypothesis of heat conservation. With respect to the hematology, our study revealed neither significantly high elevated hemoglobin concentration nor hematocrit in the HA population, suggesting that excess polycythemia is not a necessary response to hypoxia in this species. Our results might support the hypothesis of duration of high-altitude ancestry, in which species residing at high-altitudes the longest might be the best adapted.}

\normaltalk{Ecomorphological Distinctness at the Macroevolutionary Level in Birds}{Gala Cortes Ramirez, Adolfo G Navarro Sigüenza}{The morphology of an organism is closely related to its lifestyle. Convergence is one of the major patterns in evolution. It refers to the pattern in which similar ecological pressures, acting on distantly related species, lead to the evolution of similar morphologies due to adaptation to similar environments or lifestyles. We used ecomorphological diversity in the class Aves as a proxy for their lifestyle and examined the morphological space occupied by birds. We aimed to characterize the broad scale patterns of morphological disparity within the class, to test to what extent allometric patterns tend to converge, resulting in similar morphologies. Because morphological variation is not determined by adaptation alone, we took into account the effect of phylogeny, phylogenetic membership, and body size on the data. We found evidence that birds morphology had a tendency to exhibit convergent morphology, and species are clustered rather than evenly distributed throughout the volume of the morphological space. Few flying bird taxa occupy an exclusive region in morphospace that is not occupied by other orders, and the largest disparity is found in Passeriformes and Trochilidae. The distribution of morphology is weakly related to number of species per taxonomic group. Our results suggest convergent morphological evolution can be found at macroevolutionary levels and that it can arise not only from convergence due to adaptation but also from other constraints.}

\normaltalk{Opposing Mitonuclear Co-Introgression and Divergent Selection in Siberian Songbirds}{Ellen GM Nikelski, Alexander S Rubtsov, Darren E Irwin}{Though previously considered a neutral marker of population history, research suggests that mitochondrial DNA (mtDNA) is subject to selective forces such that favoured mtDNA haplotypes can sweep through populations and adaptively introgress across species boundaries. Yet, changes in mtDNA must be accompanied by similar changes at specific nuclear genes (mitonuclear genes) to maintain mitochondrial functioning. As such, instances of mtDNA introgression are hypothesized to drive co-introgression of compatible mitonuclear genes. The yellowhammer and pine bunting are Holarctic songbirds that meet and hybridize in central Siberia. Though highly diverged phenotypically and ecologically, allopatric populations differ little in mtDNA--a pattern suggestive of mtDNA introgression. In this study, I performed Genotyping-By-Sequencing on over three hundred pure and hybrid yellowhammer and pine bunting samples across their ranges. I then characterized nuclear differentiation between species to distinguish how mitonuclear co-introgression and divergent selection on plumage traits drive nuclear genetic patterns. Adaptive introgression at mitonuclear genes was identified using genome-wide scans of Tajima's D, and candidate plumage genes were detected through admixture mapping. Preliminary analysis showed low genome-wide Fst, but peaks of high Fst on numerous chromosomes that separate allopatric populations into distinct clusters on a PCA. Evolution of incompatibilities between mitochondrial and nuclear DNA has gained support as a driver of divergence between taxa. Yet, selection on mtDNA can also lead to homogenizing mitonuclear co-introgression that opposes divergent selection. Understanding the interplay between such competing forces is an important, but understudied, step necessary to disentangling the complex process of speciation.}

\normaltalk{The Juru\'{a} River is a Barrier to Five Bird Taxa}{Marquette Mutchler, Glaucia Del-Rio, Bruna Costa, Gisiane Lima, Anna Hiller, Bianca Matinata, Jessie Salter, Donna Schmitt, Marco Rego, Robb T Brumfield}{Large rivers such as the Amazon River and its tributaries are hypothesized to be biogeographic barriers that shape species distributions of many terrestrial vertebrates. Due to the remoteness and size of the region, limited work has been done to examine the exact impacts of these tributaries on local avifauna, especially in tributaries in between the Ucayali and the Madeira Rivers. The Juru\'{a}, the third-largest white-water tributary of the Amazon River, has historically been reported as insignificant in shaping the distributions of birds ranging across its banks. Such a view is based on how narrow it is when compared to other Amazonian rivers, and also on its dynamic nature. The lack of vouchered genetic material has compromised any efforts to challenge the perception that the Juru\'{a} river is not a barrier for birds. Here, we investigate the Juru\'{a} River's efficacy as a dispersal barrier by studying patterns of plumage variation in ~90 taxa of terra firme birds distributed across its banks. We conducted a three-week-long scientific expedition along the Middle Juru\'{a} River in which we collected and analyzed 913 bird specimens and associated genetic material. Plumage pattern analysis and maximum likelihood phylogenies based on UCE markers for 80 birds revealed phenotypic and genetic differentiation across the Middle Jurua in five species of birds: Willisornis poecilinotus, Myrmoborus myotherinus, Epinecrophyla amazonica, Epinecrophyla haematonota, and Lepidothrix coronata. These results suggest that the Juru\'{a} is a potentially important barrier to low-dispersal understory birds, despite the dynamic nature of the river.}

\normaltalk{Are Island  Birds Giants? Testing Foster's Rule Across all Passerines}{Aaron L Veale}{Species on isolated island systems often experience different environmental conditions compared to their counterparts on the mainland. These conditions can provide unique challenges and opportunities. This isolation can even lead to insular dwarfism in large animal species, and insular gigantism in small species -- a process known as Foster's rule. The goal of this study was to perform a large-scale analysis to determine if members of the order Passeriformes (perching birds) follow Foster's rule, with island species exhibiting comparatively larger body sizes. Being one of the most diverse groups amongst vertebrates, passerines often also have high rates of dispersal and can be found on many unique island habitats. I analyzed the masses of over 5,000 extant species endemic to either islands or continental regions around the world. I then conducted phylogenetically-controlled analyses to account for shared ancestry and demonstrate clear support for Foster's rule in passerines.}

\normaltalk{Who's Thy Father? Assessing Paternity in a Highly Polygynandrous Passerine using RADseq Data}{Angel G Rivera-Col\'{o}n, Alec B Luro, Julian M Catchen, Mark E Hauber}{Polygynandry and extra-pair paternity (EPP) are common across the animal kingdom, yet it remains unclear why most of these species engage in EPP. Among birds, EPP appears to be present at moderate-to-high levels amongst most biparental socially monogamous species, where studied, particularly in passerines. A proposed explanation is that this behavior allows parents to increase offspring fitness by maximizing their offspring's genetic diversity. It is unknown, however, whether and how parental birds can discern genetic similarity between each other. In a population of the American Robins (Turdus migratorius) breeding in central Illinois, a previous study found exceptionally high rates of EPP (~72\% of all broods, 48\% of all nestlings), albeit with a limited genetic sampling of only 3 microsatellite loci. Using restriction-site associated DNA sequencing (RADseq), we genotyped over 400 robin individuals to confirm the high prevalence of EPP in this central Illinois population. Additionally, we tested the hypothesis that females seek extra-pair mates to increase offspring fitness by increasing the genetic diversity of their brood. We predicted that (i) within a brood, heterozygosity would be greater in the nestlings sired by EPP males relative to those sired by their social father, and (ii) the EPP male would be less related to the female than her socially paired male mate.}

\normaltalk{Are False Nests in Estrildid Finches a Defense Against Brood Parasitism?}{Danielle Goodvin, J J Price}{Some African finch species in the Estrilda and Amandava genera (family Estrildidae) build cup-shaped ``false" nests on top of their main roofed nests. The function of these false nests is unclear; however, they have been speculated to function as decoys in preventing brood-parasitism by whydahs and indigobirds (family Viduidae), which specialize on parasitizing the nests of estrildid finch species. We compiled a list of nest types, known brood parasites, and nest locations for all African estrildid species and mapped these characters onto a phylogenetic tree for the Estrildidae. Interestingly, we found a relationship between the occurrence of false nests and parasitism by a particular brood parasitic species, the Pin-tailed Whydah (Vidua macroura). We used Pagel's (1994) discrete likelihood correlation method to test for coevolution between the presence of false nests and parasitism by V. macroura. Our results found a significant correlation between the evolution of false nests and parasitism by V. macroura (P =0.006). We will test for the coevolution between the presence of false nests and (1) parasitism by any species of Vidua and (2) the location of the nest on or above the ground. Future research may include documenting how these false nests prevent or deter parasitism by V. macroura and why this specific host-parasite relationship has resulted in the evolution of this unusual nest structure.}

\normaltalk{Genome-Wide Patterns of Divergence in Passerina and Cyanocompsa Buntings}{Libby Megna, Matthew D Carling}{Recently, genomic datasets have been harnessed to elucidate the underlying genetic architecture of speciation and illuminate important evolutionary processes. Here, we take a novel approach to correlate patterns of genome-wide divergence with degree of reproductive isolation among three closely related species pairs within Cardinalidae (Aves: Passeriformes). These species pairs represent varying degrees of reproductive isolation (hybridizing vs. non-hybridizing) and different geographic relationships (allopatric vs. sympatric). Our study design provides empirical data to inform theory of genome-wide divergence by making comparisons of 1) different geographic outcomes of speciation, and 2) different levels of reproductive isolation upon secondary contact. We generated a reference genome for Passerina amoena using the Chromium platform; sequencing generated ~760 million reads and the resultant assembly is 0.92 Gb. We will align whole-genome resequencing data (~1.5 billion reads) from 56 individuals total of P. amoena, P. cyanea, P. ciris, P. versicolor, Cyanocompsa parellina, and C. brissonii to this reference. P. amoena and P. cyanea hybridize extensively in the Great Plains, while P. ciris and P. versicolor do not hybridize despite allopatry in Texas and Mexico. The Cyanocompsa species are completely allopatric. We will quantify patterns of genome-wide divergence for these species pairs and identify specific genomic divergence patterns associated with degree of reproductive isolation by comparing among species pairs. For example, comparing the location of divergence peaks between the hybridizing, sympatric species pair to those between the non-hybridizing, sympatric species pair will allow us to investigate how gene flow impacts patterns of genome-wide divergence.}

\normaltalk{Intraspecific and Interspecific Aggregations and Habitat Use by Threatened Sheldgeese (Chloephaga Spp.) in Austral Patagonia, Argentina}{Natalia A Cossa, Laura Fasola, Ignacio Roesler, Juan C Reboreda}{Studies about species aggregations and habitat use provide important information about the nature of the species and how those species meet their requirements, but also have key implications for its conservation and management. Migratory sheldgeese (Upland Goose Chloephaga picta, Ashy-headed Goose C. poliocephala and Ruddy-headed Goose C. rubidiceps) are threatened birds that breed in southern Patagonia. We studied the aggregation patterns of these species and the habitat they use within part of their reproductive area by conducting nine road surveys between spring 2013 and summer 2016, covering ~900 lineal km and ~900 km2. Most groups (82.3\%) were monospecific, but 16.4\% included two of the species and 1.3\% the three species. The most abundant species was the Upland Goose, which was present in 98\% of the groups, followed by the Ashy-headed Goose in 18\% of the groups. The Ruddy-headed Goose was present in only 3\% of the groups, in only 15 sites and a maximum of 49 individuals were counted.  Most individuals concentrated in grass lawns, meadows and Festuca grasslands. Groups were larger in meadows, wetlands and during summer and autumn (post-breeding). Population density increased from spring to autumn and decreased in winter, when 68-82\% individuals migrate. Surveyed area was probably a pre-migration concentration site, which would explain the larger groups in autumn. Our results provide resources to prioritize conservation actions by identifying important habitats and proper time of the year to apply these actions.}

\normaltalk{Multiple Point Fixed Transects for Rapidly Estimating the Minimum Number of Distinct Individual Parrots}{LoraKim Joyner}{Most populations of parrots in the Americas occur in patches fragmented by habitat degradation and wildlife extraction patterns. Density in these patches can vary considerably, making density population studies, given conservationist's limited time and financial resources may, often not the best option for rapidly locating flocks for conservation efforts. Alternatively, fixed transects yielding parrot foraging, nesting, and roosting behavior and locations, provides a rapid assessment of the minimum number of distinct individuals (MNDI). Transects are uniquely designed to fit the terrain, and reported locations and behavior of parrots. With the MNDI, communities and resources can be targeted to quickly address further wildlife trade and environmental degradation issues.  Repeating these rapid assessments yields more precise relative density of various parrot species and allows for analyzing population trends over time and throughout the year. We have used this technique throughout the Americas as a first step in identifying ``hot spots" where birds still occur, where it is possible to support communities and businesses, and where we can concentrate our efforts. Biologists need to be aware of parrot ecology and behavior to conduct accurate counts, but after initial training, become proficient in removing possible duplicate birds from the counts and incorporating community members in the effort. This counting technique then becomes a method for raising consciousness and awareness, focusing people on parrot biology and behavior and hence increasing their appreciation of the species, and providing workers, students, and community members with a concrete and satisfying method for contributing to their future.}

\normaltalk{Responses of Polylepis Birds to Patch and Landscape Attributes in the High Andes}{C Steven Sevillano-Rios, Amanda D Rodewald}{Habitat loss and fragmentation can devastate biodiversity, especially at regional and global scales. However, generalizing to individual species is challenging given the wide variety of intrinsic and extrinsic factors that shape species-specific responses -- particularly among species that are specialists, generalists, or adapted to naturally patchy landscapes. In this study, we examined how patch and landscape attributes affected bird communities within Polylepis forest ecosystems, which are patchily distributed within landscapes of Puna grasslands and shrublands in the High Andes of Peru (3,300 -- 4,700 m). We surveyed birds in 59 Polylepis patches and 47 sites in the Puna matrix, resulting in 13,210 observations of 88 bird species, including 15 species of conservation concern specialized on Polylepis. Data were analyzed using Multi-Species Occupancy-Models (MSOM) and cumulative species-area curves. Species richness was generally greatest at mid-to-low elevations, within small fragments, and in landscapes with comparatively little forest cover; this was especially true for birds associated with the Puna matrix. Consistent with the hypothesis that Polylepis specialists are adapted to naturally patchy landscapes, we found no evidence that Polylepis specialists were sensitive to patch size, though two of nine species were positively related to forest cover within 200m. Our work shows that small patches of Polylepis have high ecological value and that conservation of species of concern may depend more on retaining at least 10\% forest cover within landscapes than on the presence of large patches of Polylepis.La p\'{e}rdida y fragmentaci\'{o}n de h\'{a}bitat puede devastar la biodiversidad, especialmente a escala regional y global. Sin embargo, generalizarlo a especies individuales puede no ser apropiado dada la variabilidad de factores intr\'{i}nsecos y extr\'{i}nsecos que determina respuestas especificas -- particularmente entre especies que son especialistas, generalistas or adatadas a paisajes naturalmente fragmentados. En este estudio, examinamos c\'{o}mo la extensi\'{o}n del tama\~{n}o del parche y la cantidad de bosque en el paisaje afecta la comunidad de aves dentro de los ecosistemas de bosques de Polylepis, que se distribuyen de manera irregular dentro de paisajes de pastizales y matorrales de Puna en los Altos Andes del Perú (3,300 -- 4,700 m). Examinamos las aves en 59 parches de Polylepis y 47 sitios en la matriz de Puna, lo que result\'{o} en 13,210 observaciones de aves de 88 especies, incluyendo 15 especies de inter\'{e}s para la conservaci\'{o}n. Los datos se analizaron utilizando modelos de ocupaci\'{o}n de múltiples especies (MSOM) y curvas acumulativas de especies-\'{a}rea. En general, la riqueza de especies fue mayor a elevaciones medias-a-bajas, en fragmentos peque\~{n}os y en paisajes con una cubierta forestal comparativamente baja; esto fue especialmente cierto para las aves asociadas a la matriz de Puna. Consistente con la hip\'{o}tesis que las aves especilitas a Polylepis est\'{a}n adaptadas a un paisaje naturalmente fragmentado, no encontramos evidencia que los especialistas a Polylepis fueran sensibles al tama\~{n}o de parche, aunque dos de nueve especies fueron asociadas positivamente a la cobertura forestal dentro de 200m de radio. Nuestros hallazgos resaltan la contribuci\'{o}n ecol\'{o}gica que cumplen los parches peque\~{n}os de Polylepis en los paisajes altoandinos y que la conservaci\'{o}n de especies amenazadas depende m\'{a}s en retener al menos un 10\% de cobertura forestal ($>$ 400 ha) de estos valles glaciares que la sola presencia de grandes bosques de Polylepis.}

\normaltalk{The Effects of Pishing and Playback on Avian Fitness}{Mariel Ortega, Jacquelyn K Grace}{Playback and pishing are auditory stimuli used by researchers, tour guides, and amateur birders to facilitate viewing or detection of a bird. Playback involves playing territorial songs or contact calls, and pishing involves making a small, repetitive noise to stimulate a territorial response. Concerns have been raised about potential negative impacts of playback and pishing on avian long-term fitness, as a result of behavioral shifts related to playback and pishing. We investigated the effects of pishing and playback on behavior, body condition, corticosterone levels, and reproductive success on resident, territorial birds in College Station, TX. Pishing and playback treatments were administered for five minutes in sets of five days, with two day rest periods for four weeks. Birds were captured before and after treatment for measurement of body condition and feather corticosterone, and we located nests to count eggs and young hatched. The results of this study will be able to guide regulations about the use of playback/pishing, an ever-increasing concern as the popularity of birding increases.}

\normaltalk{Mapping the Distribution Range of the Red-Fronted Parrotlet and Evaluating Its Protection in Costa Rica}{David Segura}{Most of the endemic birds of Costa Rica have their distribution ranges limited to the Central Volcanic and Talamanca Mountain Ranges, which have some territories with permanent government protection under the category of National Parks. The Red-fronted Parrotlet (Touit costaricensis) is one of these species, but due to its populations limited to inaccessible places, periodic execution of altitudinal migrations to lands surrounding the mountain ranges and their silent behavior, it has been poorly recorded and studied by birdwatchers and scientists, resulting in an information gap on their actual range and threats. Due to this reasons, Potential Distribution Models have been created to have an approach on which are the landscape and bioclimatic features that it prefers, to determine if they are within protected areas. The Maxent Software was used to create the models, where climatic variables, geographic elevation models and several land uses were selected. The resulting distribution range is widely protected due to the presence of National Parks and Private Reserves, although there are sites with low or no forest connectivity, which, due to the altitudinal migration status of the species, may expose it to risks. This project is the first one to describe the range of distribution of the species based on environmental factors. Several future research strategies are proposed to favor the conservation of this endemic psittacid.}

\normaltalk{Lost in Translocation: Effects of Mitigation Translocation on Burrowing Owl Survival and Fidelity}{Dejeanne Doublet, Martha J Desmond, David H Johnson, Fitsum Abadi}{Given the rapid pace of urbanization, mitigation translocations are increasingly utilized to move animals away from development. Although widespread, the efficacy of these programs is not widely documented. Burrowing Owls (Athene cunicularia) are commonly subjected to translocations because of their association with development, declining populations, and fossorial nature. We investigated differences in survival and fidelity of resident and translocated owls and evaluated factors that influenced these rates in Arizona. From 2017--2019, we tracked the fates of 42 resident and 43 translocated adults using VHF radio-telemetry. Results indicated that translocated owls (T) had lower annual survival compared to residents (R) in 2017 (ST = 0.35 ± 0.13 [SE], SR = 0.69 ± 0.12) and 2018 (ST = 0.01 ± 0.01, SR = 0.69 ± 0.11). Annual was also generally lower for translocated owls compared to resident owls in 2017 (ST = 0.54 ± 0.18, SR = 0.62 ± 0.14) and 2018 (ST = 0.05 ± 0.06, SR = 0.79 ± 0.11). Models assessing survival and fidelity of only translocated owls showed that the number of males in release groups had the strongest influence on survival (ß = -0.74, 95\% CI = -1.07, -0.40) and fidelity (ß = -0.73, 95\% CI = -1.17, -0.29). Owls that were released in groups with more males had significantly lower survival and fidelity. Based on the results from this research, I recommend releasing owls in pairs or as singles and using an ecological timeline for releases that include reduced captivity periods for improved translocation success.}

\normaltalk{Factors Affecting Burrowing Owl Nest Site Selection in Artificial Burrows in Arizona}{Kimberly I Fonseca, Dejeanne Doublet, Martha J Desmond, David H Johnson, Fitsum A Gebreselassie}{Urbanization has resulted in habitat loss and degradation for bird species worldwide. The Burrowing Owl (Athene cunicularia) is a species that commonly comes in conflict with development. One method for mitigating this conflict is to relocate owls to artificial habitat. In this study, we analyzed how Burrowing Owl nest-site selection is influenced by local-scale factors in Artificial Burrow Systems (ABS). ABS are constructed throughout Arizona specifically for translocated Burrowing Owls. Factors we evaluated included angle and height of the tunnel entrance, whether a nest chamber had a single or double entrance(s), nearest nest distance, number of nests, and number of satellite burrows within a 75 m radius. We developed Bernoulli models using R-INLA to examine the influence of these site characteristics on site selection. We assessed models under a Bayesian framework, utilizing WAIC to rank models. Preliminary results indicate that nest-selection is skewed towards (1) single entrances (by 6-fold), (2) shorter tunnel length (ß = -0.54 ± 0.28), and (3) lower nest density (ß = -0.29 ± 0.23). Double entrance burrows are constructed by wildlife managers to provide an escape for nestlings. However, owls may avoid double entrances because of reduced humidity compared to single tunnel entrances. In this study owls significantly preferred single entrance burrows, shorter in length. Therefore, we recommend the construction of single entrance ABS, with tunnels short in length ( $<$ 7 m ). In addition, wildlife managers should consider the ABS nest distance and nest density between burrows, to increase nest-site selection.}

\normaltalk{Kakariki (Cyanoramphus Auriceps) Distribution in Upland Sites of Abel Tasman National Park, New Zealand}{Douglas A Robinson Jr}{Kakariki (Cyanoramphus auriceps) were once found throughout New Zealand, but now are constrained to native forests found on offshore islands and protected tracts of land, including national parks.  Reintroduction efforts of Kakariki in Abel Tasman National Park (ATNP) from 2014-2017 resulted in the release of 54 individuals; however, few systematic surveys have been done to document the release success.  From February-June 2019, 12 `walking' surveys and 63 point counts on three rat/stoat traplines were conducted in upland regions of ATNP where Kakariki had been previously detected.  A total of 149 Kakariki were counted during the surveys, which likely corresponded to ~55 individuals across the three routes.  The average detection rate per 100m `walking' survey segment of 0.0412 +/- 0.2553 differed little from the average reported from similar surveys conducted in Nov 2012-2016.  Kakariki detection at 21 point count locations averaged 0.215 +/- 0.473 (SD).  The persistence, but low detection rate, of Kakariki in upland sites suggests efforts to reduce the major obstacle to population success, predation by invasive mammalian predators (stoats and rats), are successful at creating an environment capable of sustaining a small population of Kakariki.  More detailed information is needed to determine the long-term persistence of Kakariki in ATNP, particularly whether the population is buoyed by reproductive success in the ATNP and/or sustained via immigration from nearby native forests (e.g., Kahurangi National Park).}

\normaltalk{Breeding site fidelity of two Austral migratory flycatchers is related to sex and habitat}{Karlla Barbosa, Vanessa Alegre, Thiago Costa, Milton Ribeiro, Alex Jahn}{Breeding site fidelity is a widespread behavior in migratory birds, but our understanding about the behavioral and ecological drivers underlying return rates of austral migratory birds is still incipient. We assessed whether breeding site fidelity of Fork-tailed Flycatcher (Tyrannus savana, FtFl) and Southern Streaked Flycatcher (Myiodynastes maculatus solitarius, StFl) in Brazil is related to sex and habitat type (urban versus rural). Between 2013 and 2018, we captured, banded, and made systematic observations of 133 individuals of FoFl and 49 individuals of StFl in different rural and urban areas. Our results revealed higher return rates to breeding sites in these two species in comparison to other austral migrants, being $>$23\% in FoFl and $>$53\% in StFl. Return rates were higher in males (37\% FoFl and 61\% StFl) than females (23\% FoFl and 53\% StFl), and in rural (33\% FoFl and 64\% StFl) vs. urban areas (27\% FoFl and 47\% StFl). These results illustrate the influence of demography and habitat on the behavioral ecology of austral migrants, which still remains very poorly understood. We suggest future individual-level ecological and behavioral studies on austral migrants in different habitats (including urban settings). Such research promises a novel perspective as to how individual austral migratory birds perceive their surroundings. They will also help inform future conservation initiatives focused on poorly understood migratory species, as well as their habitats.}

\normaltalk{Behavioral Flexibility of Tidal Marsh Birds in Response to Nest Flooding}{Sam E Apgar, Chris S Elphick}{As anthropogenic stressors threaten species with extinction, it is essential to understand the existing flexibility species have to respond to change. We studied the behavioral responses of four sympatric tidal marsh bird species (Saltmarsh Sparrow Ammospiza caudacuta, Seaside Sparrow A. maritima, Clapper Rail Rallus crepitans, Willet Tringa semipalmata) to nest flooding, a principal cause of nest failure. As sea levels rise, these species will likely experience nest flooding more often. Limited evidence suggests that older sparrow chicks can climb up vegetation to avoid drowning and that Clapper Rails can manipulate nest structure during floods and retrieve flooded out eggs. To assess the frequency and conditions under which species engage in behaviors used to mitigate nest failure during flooding, we used infrared radiation cameras to film nests during nightly high tides occurring around the full and new moons (n = 63 Saltmarsh Sparrow, 16 Seaside Sparrow, 27 Clapper Rail, 21 Willet videos, 2017-2019). Data demonstrate that rails are often at nests during flooding events, manipulating eggs or nests whereas adults of the other species are largely absent. Chicks of the sparrow species can climb as early as 4.5 days old, but until chicks are 8.5 days old, only half of each brood climbs and survives flooding. This research suggests that most species maintain few critical adult responses to nest flooding. However, variation exists in the age chicks can climb, suggesting that it may be possible for populations to evolve more uniformly early climbing behaviors. Earlier climbing would enhance nest survival.}

\normaltalk{Evaluation of Body Size Proxies in Birds using 3D Light Scanning Technology}{Michael D Brawner, Rebecca C Terry}{The realization that rapid phenotypic change occurs over ecological time periods combined with the threat of continued global environmental change has spurred a critical reevaluation of the factors that structure body size through time in recent years.  Central to this renewed interest in temporal size trends is the biogeographic rule known as Bergmann's rule, the commonly observed inverse relationship between body size and temperature.  However, recent efforts to uncover evidence for Bergmann's rule over ecological time periods have produced equivocal results.  Part of this ambiguity lies in a lack of consensus on the methods used to determine body size.  Many researchers use just a single morphological measurement as a proxy for body size such as mass, or wing length, but such measures can vary considerably over the annual cycle depending on life history events, calling into question their reliability as a proxy for structural size.  We take a novel approach using 3D structured light scanning technology to address the question of whether commonly used external measurements can accurately represent the structural size of birds.  Using 3D scans of sternae from historical specimens of American kestrels (Falco sparverius) and measurements from paired study skins, we quantify the correlation between structural size, as measured from a skeletal element, and size measured from external traits.  Our results provide the first accounting of size comparison between commonly used body size proxies and a structurally derived measure of size.}

\normaltalk{Linking Data Analysis Predictive Modeling to Conservation Decisions for the Eastern Black Rail.}{Conor McGowan, Whitney Beisler, Nicole Angeli, Catlin Snyder, Nicole Rankin, Jarrett Woodrow, Jennifer Wilson, Erin Rivenbark, Amy Schwarzer, Christy Hand}{The US Fish and Wildlife Service has initiated a re-envisioned approach for providing decision makers with the best available science, called the Species Status Assessment (SSA), for endangered species decision making. The SSA report is a descriptive document that provides decision makers with an assessment the species status by assessing the current conditionss and predicting future status of the species. These analyses support all manner of decisions under the US Endangered Species Act, such as listing, reclassification, recovery planning, etc. Here we describe our approach to analyzing available data to assess current condition of eastern black rail across its range in a dynamic occupancy analysis. We used the results of the analysis to develop a site occupancy projection model where the model parameters (initial occupancy, site persistence, colonization) were linked to environmental covariates, such as land management and land cover change (sea-level rise, development, etc.). We used the projection model to predict future conditions under multiple sea-level rise and habitat management scenarios. Occupancy probability and site colonization were low in all analysis units and site persistence was also low, suggesting low resiliency and redundancy currently. Extinction probability was high for all analysis units in all simulated scenarios except one with significant effort to preserve existing habitat, suggesting low future resiliency and redundancy. With results of these data analyses and predictive modeling, the US Fish and Wildlife Service concluded that protections of the Endangered Species Act were warranted for this subspecies.}

\normaltalk{Quantifying the Impact of Food Subsidy Reduction on a Highly Subsidized Generalist Predator}{Kristin M Brunk, Elena H West, M Z Peery, Anna Pidgeon}{Food subsidies can impact species survival, reproduction, density, and behavior. Generalist predators are particularly good at taking advantage of anthropogenic food subsidies and this can lead to spillover predation. We studied a subsidized population of a generalist predator, the Steller's Jay, in a park with high levels of human activity, where spillover predation may affect species of conservation concern. Our goals were (1) to determine if a visitor education campaign implemented in 2013 has been successful at reducing or eliminating food subsidies to Steller's Jays, and if so, (2) to understand how the reduction in subsidies has affected the ecology and fitness of jays. We evaluated subsidies using stable isotope analysis of feather samples and d13C as a proxy for human food, determined jay density and fecundity using point counts and juvenile to adult ratios, assessed body condition using feather growth bars, and calculated home range sizes and the degree of home range overlap between birds using radio telemetry. We then compared these metrics to similar data collected prior to the visitor education campaign. We found that jay density has decreased, average home range size has not changed, and overlap between jay territories has decreased. Further, we found that anthropogenic food enrichment and body condition of birds have not changed. Our preliminary results suggest that visitor education has potentially lessened available subsidies, which has led to fewer jays in campground areas in the post-management era, but equivalent body condition and human food enrichment for those that remain.}

\normaltalk{Do Surveys Tell the Whole Story? Comparing Atlas Data to Roadside Point Counts}{Evan Adams, Amber Roth, Glen Mittelhauser, Adrienne Leppold}{Breeding bird atlases are important for understanding population status and trends, setting conservation priorities, and engaging with the public. Recent atlases have collected a diversity of data on bird populations, in particular pairing extensive point surveys conducted by experts with intensive efforts to describe bird breeding status by community scientists. While these data have similar properties, their survey methodologies are divergent and provide different information on the breeding populations in survey areas (i.e., atlas blocks). Using data from two breeding seasons of the Maine Bird Atlas, we explore the relationships between point count and atlas data using a paired study design.  While data collection is still ongoing, we find that atlas surveys and point counts provide different perspectives of the avian community in each atlas block. Only 6\% of species that are thought to breed in a block were found in the point count surveys of the same area; this rate increases marginally to 8\% when you only consider blocks that have enough atlassing effort to be complete. Divergence between the two surveys are a result of complex differences in effort, detection rates, identification rates, and mismatch of survey spatial scale. Given their complementary nature, however, the integration of these surveys can quantify misidentification rates and sampling bias to achieve improved estimates of breeding bird presence or density. We describe how hierarchical models can achieve these results and maximize the utility of atlas data for furthering breeding bird conservation.}

\normaltalk{Using Citizen Science Bird Data to Test the Feasibility of Modeling Species Distributions with Spatially Biased Data}{Willson Gaul, Dinara Sadykova, Niall Keogh, Hannah White, Lupe Le\'{o}n-S\'{a}nchez, Paul Caplat, Mark Emmerson, Tom\'{a}s Murray, Jon Yearsley}{Biological records provide information about species occurrences over large spatial scales, but the data are spatially patchy, with lots of data from some locations and no data from others.  Species distribution models (SDMs) can ``fill in" unsampled locations by predicting which species are present, but only if models can make good predictions despite being trained with spatially biased data.  In Ireland, citizen science data collection for birds is intensive and relatively spatially even.  We degraded biological records of birds in Ireland by sub-sampling to introduce more severe spatial sampling biases based on the biases found in data for other taxonomic groups.  We tested the effect of the spatial bias and sample size on predictions of species distribution models.  Spatial bias at an intensity comparable to that found in average Irish biological records did not hurt predictive performance of SDMs compared to when models were trained with minimally biased data.  Increasing the total amount of data was more important than increasing the spatial evenness of data when the goal was to predict species occurrence using species distribution modeling.}

\normaltalk{Seabird Monitoring in an Important Bird Area of Puerto Rico}{Gloria I Morales Quintana, Ricardo A Berr\'{i}os P\'{e}rez, Adrianne G Tossas Cavalliery}{The White-tailed Tropicbird (Phaeton lepturus) is distributed along tropical oceans around the world, but its population in Puerto Rico is small ($<$300 pairs). According to historical data, this population has been declining, and at present is limited to a few sites in the island, mainly remote islets or keys. In the main island of Puerto Rico, tropicbirds can only be observed in the island's Atlantic coast, from Aguadilla to Barceloneta, with a breeding colony in cliffs from Isabela to Camuy, which are part of an Important Bird Area, as designated by BirdLife International.  We conducted 51 biweekly surveys from February 2018 to February 2020 to determine the size of the tropicbird population in the cliffs of Guajataca, in Quebradillas.  The number of individuals reported in these surveys ranged from 0 to 94 (mean = 40).   There were no significant differences in the mean number of individuals between the breeding seasons of 2018 and 2019.  In both years, the mean number of individuals halved from February to March with the initiation of the nesting activities when (e. g., egg laying and incubation), and gradually declined toward July when juveniles and adults left the site. No tropicbirds were observed from August to November. Our data suggests that this population consists of at least 50 breeding pairs.  Only one other seabird species was regularly reported in our surveys throughout the year, the Brown Pelican (Pelecanus occidentalis), but in low numbers (mean = 6.4 individuals, range 0-25).}

\normaltalk{Habitat Association Effects on Abundance of a Common and Generalist Insular Raptor: The Red-Tailed Hawk (Buteo Jamaicensis) in Puerto Rico}{Julio C Gallardo, Francisco J Vilella}{The Red-tailed Hawk (Buteo jamaicensis) nests across all elevations and habitats, including closed canopy forest in Puerto Rico and exhibits one of the greater densities recorded across the species' geographic distribution. However, the drivers that influence local abundance and distribution of this top island predator are poorly understood. This limit understanding of the functional role this raptor may play as a limiting factor for species such as the endangered Puerto Rican Parrot (Amazona vittata). During the breeding season of 2013-2014 (November-July), we conducted fixed-radius surveys (218 point counts/6 repetitions) along primary and secondary roads in eastern (Luquillo Mountains) and western (Cordillera Central) Puerto Rico. We developed open-population N-mixture models to assess the relationships between landscape variables (elevation, canopy closure, and slope) and abundance of the Red-tailed Hawk (N), simultaneously accounting for availability ($<$U+03D5$>$) and detection probability (r). Estimated abundance was ~0.05 hawks/ha (~4 hawks/site) across both regions, approximately five times greater than previously reported estimates for the island. Overall availability of Red-tailed Hawks was ~1 hawk/site, suggesting at least one individual was present at a given survey site while three others (total abundance) used the same space. Higher abundances were associated with steep slopes at mid-elevations (~300m) in the Luquillo Mountains, while no relationship between landscape variables and abundance were found in the Cordillera Central. The estimated abundance and availability of Red-tailed Hawks in the Cordillera Central may represent an important limiting factor for ongoing parrot reintroduction efforts in this area.}

\normaltalk{Changes in the Avian Community of Johnson County, Iowa (1907-2019)}{Jonah D Alderson, Heather A Sander}{Birds are declining at alarming rates worldwide due to factors such as habitat loss, invasive species, and global climate change. Identifying how bird communities have changed over long time periods is vital to understanding these declines and to planning future conservation actions. We used Kruskal-Wallis and Dunn's multiple comparison tests to compare songbird abundance data from the early 20th Century and contemporary field-collected data for Johnson County, Iowa to test whether different songbird groups showed significant increases or declines. Our results indicate an increase in exotic species abundance which, although some loss of native species has occurred, has led to increased species richness.  Most groups did not exhibit a significant change in relative abundance; however, decreases occurred across many groups, except for exotics, non-migratory species, urban-tolerant species, granivores, and species known to make use of human resources such as bird feeders. Changes in land cover also likely explain changes in community composition, particularly large increases in urban land, transformation of native grassland to agriculture, and patchy afforestation. The results of this study contribute to existing literature on the massive changes in avian communities seen around the world, but shows how these changes may differ on a local scale.}

\normaltalk{Fire Island National Seashore Piping Plover (Charadrius Melodus) Monitoring Project}{Justin R Wilson}{At the Fire Island National Seashore the loss of this habitat means the loss of a great number of species that dependent on it. Shorebirds are one of several vital signs recommended by the NPS Inventory and Monitoring Network. As sea levels rise, there are concerns that beachfront habitat will continue to shrink, and loss of habitat will negatively impact piping plover (Charadrius melodus) populations as well as other shorebird species. The piping plover is considered endangered and considered under the Endangered Species Act.   We worked as part of the monitoring and a collaborative research project for shorebirds recommended by the NPS Inventory and Monitoring Network the piping plover as one of the indicator species for monitoring the health of our coastline. There are four main goals of this project: 1)Continue NPS long-term monitoring of shorebirds at established survey points at Fire Island Seashore (FIIS) 2)Initiate a new sampling design with additional survey points throughout FIIS to adequately monitor birds on a local park scale 3)Assist the NPS Biologist in engaging with local partners and improving citizen science opportunities.  We conducted distance-sample point counts and broadcast surveys for piping plovers at 11 established survey points and followed SHARP monitoring protocols. The results will be shared with interested local organizations, such as the Seatuck Environmental Association and during Latino Conservation Day event at Watch Hill in July  and their work part of the rotating educational stations for that particular event.}

\normaltalk{Evaluation of Unmanned Aerial Vehicles for Surveys of Lek-Mating Grouse}{Jacqueline K Augustine, David Burchfield}{Unmanned aerial vehicles, or drones, are being utilized by wildlife biologists to monitor populations of birds and mammals.  Lek-mating prairie-chickens (Tympanuchus spp.) seem amenable to drone-based surveys because they are relatively large, and display in groups on hilltops with sparse vegetation. The goal of this study was to determine what flight characteristics maximize prairie-chicken detection on drone video footage, and to document the birds' reactions to the drones.  We tested three sizes of rotary-winged drones, flown at three heights, with three different camera angles by flying them over known prairie-chicken display locations.  We determined that the most prairie-chickens were detected using video footage from smallest drone, flown at a height of 100m, with a 10º camera angle. However, ground-based surveys routinely detected more birds than could be detected on drone footage. Prairie-chickens returned quickly after flushing, similar to their reaction to avian predators.  In order to develop guidelines for the ethical use of drones in wildlife research, there is a need for quantifying taxa-specific disturbance caused by drones. This study starts to close this knowledge gap by documenting disturbance to upland, lek-mating grouse of conservation concern, and suggest that the focal species' reaction to natural predators may provide clues as to how it may react to drones.}

\normaltalk{Distribution, Reproduction and Migration of the Silvery Grebe Complex in Argentina.}{Luc\'{i}a B Mart\'{i}n, C I Roesler}{Grebes have called the attention of researchers since the XX's century. However, Neotropical grebes are still largely unknown. There are 11 grebe species in the Neotropics, two of them have recently gone extinct and two others are critically endangered. There is an urgent need to understand basic aspects of Neotropical grebes. Our goal is to understand distribution, reproductive area and seasonal movements of the Silvery Grebe complex in Argentina: Andean Grebe (Podiceps [occipitalis] juninensis) (NT) and `lowland' Silvery Grebe (Podiceps [occipitalis] occipitalis) (LC). We searched and analyzed information from bibliography, citizen science (CC) and field observations. We found 37 published articles and 4062 occurrence records (CC), for 19 Argentinean provinces for both groups. For the `lowland' Silvery Grebe 3828 records and 126 reproductive locations, and for the Andean Grebe 51 records and 3 reproductive locations were confirmed. We were not able to assign to any group 183 records at 600-4700 masl, which are likely Andean Grebe. `Lowland' Silvery Grebe has a wider distribution than previously thought, and well marked seasonal movements: higher concentrations in Austral Patagonia over the summer, and in the pampas and at Patagonian coasts during wintertime. For Andean Grebes we found a rather restricted distribution, given the lack of certainty of many of the highland records. Neither is there enough evidence to establish the seasonality for this group. Monitoring high Andean wetlands in northwestern Argentina, is crucial in order to understand distribution, reproductive sites and seasonality (movements) of this near threatened group.Los macaes o zambullidores han llamado la atenci\'{o}n de los investigadores durante el siglo XX. Sin embargo, los grupos neotropicales son mayormente desconocidos. De las once especies neotropicales, dos se han extinto recientemente y otras dos est\'{a}n en peligro cr\'{i}tico por lo que conocer aspectos b\'{a}sicos de estos zambullidores, se ha vuelto urgente. Nuestro objetivo es comprender la distribuci\'{o}n, \'{a}rea reproductiva y movimientos estacionales del Complejo Mac\'{a} Plateado en Argentina: el Mac\'{a} Andino (Podiceps [occipitalis] juninensis) (NT) y el Mac\'{a} Plateado (Podiceps [occipitalis] occipitalis) (LC). Para ello, recopilamos y analizamos la informaci\'{o}n de la bibliograf\'{i}a disponible, ciencia ciudadana (CC) y observaciones de campo. Encontramos 37 art\'{i}culos y 4062 registros de ocurrencia (CC) de 19 provincias argentinas para ambos grupos. 3828 registros para Mac\'{a} Plateado de ``tierras bajas" con 126 ubicaciones reproductivas y 51 registros totales con 3 reproductivos para el Mac\'{a} Andino. No obtuvimos informaci\'{o}n para una asignaci\'{o}n correcta de 183 registros de 600-4700 msnm, pero probablemente pertenezcan al Mac\'{a} Andino. El Plateado de ``tierras bajas" tiene una distribuci\'{o}n m\'{a}s amplia de lo que se pensaba y un fuerte desplazamiento estacional: mayores concentraciones en la Patagonia Austral durante el verano y en la costa patag\'{o}nica y la pampa durante el invierno. Para el Mac\'{a} Andino encontramos una distribuci\'{o}n restringida, debido a la falta de certeza de muchos registros de las tierras altas, sin evidencias de estacionalidad. Es menester monitorear los humedales altoandinos en Argentina para comprender distribuci\'{o}n, localidades importantes y existencia de estacionalidad de este grupo Casi Amenazado.}

\normaltalk{Climate-Driven Shifts in the Non-Breeding Distributions of North American Waterbirds}{Jose R Ramirez-Garofalo, Shannon R Curley, Richard R Veit, Lisa L Manne}{Climate-driven range shifts have been documented for species worldwide. These shifts have occurred for species' breeding and non-breeding ranges, and have been both poleward and multi-directional (shifts with longitudinal components). While there have been many studies of shifts in bird distributions, there have been no published multi-species studies of shifting distributions of waterbirds on the North American continent. Here, we use a long-term citizen science dataset, The National Audubon Society Christmas Bird Count, to model shifts in the non-breeding distribution of North American waterbirds along both coasts of North America. We find that shifts in the distribution of waterbirds occurring on both coasts (e.g. Harlequin Ducks on east and west coast) are occurring at rates that are not significantly different from one another. We will further investigate the effects of broad-scale oceanic climate as represented by the Multivariate El Nino-Southern Oscillation Index (MEI) and the Hurrell North Atlantic Oscillation Index (station-based; NAO) using generalized additive mixed models.}

\normaltalk{Resiliency of Coastal Texas Bird Communities Following Hurricane Harvey}{Michael W McCloy, Selma Glasscock, Jacquelyn K Grace}{The impact of Hurricane Harvey along the central coast of Texas in August 2017 provided a valuable opportunity to investigate how bird communities respond in the year immediately following an acute disturbance event. We paired a localized dataset with a wider regional dataset to investigate how bird communities responded at two different spatial scales. We used an existing long-term dataset from the Monitoring Avian Productivity and Survivorship (MAPS) program to investigate immediate changes in avian diversity on a local scale at the Welder Wildlife Refuge (WWR). To investigate regional shifts, we used data from eBird that covered a six-county area in central coastal Texas. Project eBird is a large-scale citizen science project that collects observational data from birdwatchers in the form of ``checklists". Species richness, abundance, and diversity were analyzed for the breeding season using MAPS data from the three years immediately pre-Harvey (2015, 2016, and 2017) and in the one year post-Harvey (2018). Breeding season (May-July) data from eBird was analyzed for selected passerine species during the same time frame. Preliminary results indicate a high level of post-disturbance resilience within this coastal Texas ecosystem. Analysis of MAPS data indicate a significant annual increase in overall Shannon's Diversity Index values (P = 0.02), but no significant difference in overall species richness across all years (P $>$ 0.05). Continued localized monitoring alongside the integration of larger regional citizen-science datasets can give researchers clearer insights regarding how the effects of acute disturbances may be exemplified across a variety of spatial scales.}

\normaltalk{Oscillating Patterns of Nest Box Occupancy and Reproductive Performance in Tree Swallows}{Michael P Lombardo, Patrick A Thorpe, Collin Fox, Alexis Godfrey}{Tree Swallows are experiencing local declines in eastern North America.  There are likely multiple factors responsible for these declines.  To determine if our study population in western Michigan was experiencing a decline, we analyzed patterns of nest box occupancy (i.e., the proportion of nest boxes where at least one egg was laid) and reproductive performance (i.e., the proportion of nest boxes where at least one nestling fledged) using data collected from 1992-2018.  Excluding 1992, the first year that nest boxes were available, nest box occupancy rates did not significantly change from 1993-2018 although reproductive performance was punctuated by five years of high brood mortality due to predation.  Years of high predation were often followed the next year by lower nest box occupancy.  Plotting the means of nest box occupancy and reproductive performance over two year, three year, four year, and five year time frames revealed oscillating patterns of both.  That is, alternating periods of increasing and decreasing nest box occupancy and reproductive performance.  More research may reveal whether declining Tree Swallow populations elsewhere may be in the troughs of an oscillating pattern of nest box occupancy.}

\normaltalk{Variation of Barred Antshrike Abundance in Tropical Dry Forest Relicts Located Around an Important Hydroelectric Plant in Colombia}{Lilibeth A Palacio, Camilo Loaiza, Olga L Montenegro}{Human activities, both agricultural and development work in tropical dry forests generate modifications in habitat quality that affect wild animal population. In this study, we assessed the variation in relative abundance of Barred Antshrike (Thamnophilus doliatus) along three zones of tropical dry forest differing in conservation status, located in the environmental compensation area around El Quimbo hydroelectric, in Colombia. Based on the vegetation cover map, three zones with different disturbance states were established and in each of them 62 counting points were made for both visual and auditive detection of individuals in two phases, one passive and other active.   Using generalized linear models, we found a relationship between the zone with the best conservation status (less disturbance) and the highest Barred Antshrike relative abundance and unlike other insectivorous birds, climatic season is not a factor that affects this species abundance. Furthermore, this same zone was there the largest number of Barred Antshrike individuals were detected in pairs and with greater activity in singing, responding to possible changes in behavior given by possible resources limitation in areas with greater intervention human.}

\normaltalk{Moonlight Impacts Parental Behavior in a Diurnal Bird Species}{Colleen R Miller, Conor C Taff, Maren N Vitousek}{Light cycles control the timing of daily activities that can affect the success of an organism. With a continually brighter night sky, changed by artificial light at night (ALAN), it is important to understand how organisms exhibit light-based biological rhythms. While rhythms entrained to solar cycles are well understood in diurnal animals, we know less about lunar impacts. Some research has found lunar impacts on vertebrate life history, including aspects of reproduction such as hormone circulation and mating success. However, little is known about the impacts of lunar light on parental care, a key element of reproductive performance. We studied the relationship between moonlight intensity and the timing of offspring provisioning in Tree Swallows (Tachycineta bicolor) using radio frequency identification networks over several breeding seasons. We predicted that activity would begin earlier on days following brighter nights in a pattern following the lunar cycle. We found strong relationships between moonlight intensity and onset of daily activity, after controlling for dawn time. As predicted, birds began provisioning young earlier on days following full moons. Parents also began provisioning earlier on days with earlier dawns, but on nights when the moon was brightest, dawn timing had no effect on the onset of activity. If sensitivity to lunar rhythms enables birds to maximize foraging efficiency and impacts parental care patterns, then the ability to detect lunar cycles -- which can be impaired by ALAN - could have fitness consequences.}

\normaltalk{Important Ornithological Areas in Valle Central, Cochabamba and Sorroundings}{Dennis Camacho}{Cochabamba is one of the departments with the greatest diversity of birds in Bolivia with a total of 932 species. Much of this diversity is attributed to the wide range of habitats found within the department, and many of them are related to large cities, as is the case in the central valley of Cochabamba. The present work consists of collecting information on these habitats and determining that they have an ornithological relevance, for this purpose, census methodologies and bibliographic searches were used, as well as citizen science portals (eBird), the criteria used are to identify the species residents, migrants, threatened, accidental and species that were new records for the department or the country. You will see that the wetlands of the central valley as well as the sites near the Cordillera del Tunari are of fundamental importance for the local and visiting Ornitofauna, since they involve resting, nesting and passage sites for many birds, specifically Albarrancho and Potreros, places where species such as Roasted-breasted Sandpiper (Calidris subruficollis), Bobolink (Dolichonyx oryzivorus), endemic Cochabamba Mountain-Finch (Poospiza garlepii) and giant Conebill (Conirostrum binghami) were recorded, just to give some examples, these problems also occurred they are suffering an environmental crisis, due to the alteration, fragmentation and contamination that are of anthropic origin, and that make both the frequency and the number of individuals that can be seen in them, to be more affected, so it is advisable to continue with monitoring these places to see fluctuations in bird populations.}

\normaltalk{High Abundance of Bats in the Diet of a Barn Owl (Tyto Alba)  Couple in a Dry Forest of Tolima, Colombia}{Laura M Baldrich, Andres Link, Ronald Castellanos}{Barn owls (Tyto alba) are opportunistic animals that change their diet depending on prey availability, which is the reason why temporal and spatial variation in the diet of these organisms has been observed. In this study, the trophic habits of an owl nest located in a dry forest in Tolima, Colombia, were determined and recorded for the first time, analyzing 516 pellets collected between 2016 and 2017. To identify the prey, samples were cleaned and skulls in good condition (composed by upper and lower jaws, molars, and canines) were selected. Each prey item was identified up to the most specific taxonomic category allowed by the type of prey and the fragment found using identification guides and physical comparison with museum specimens. A total of 335 skulls belonging to vertebrates were identified. We found that the owls in this nest fed mainly on rodents (67\%) and bats (29\%), followed by other birds (3\%) and shrews (1\%). The proportion of bats in the diet exceeds the previously reported values of bats consumption in Colombia and South America, which may mean that these predators are exploiting rare resources in the sampling area, supporting their opportunistic behavior. This study, to our knowledge, is the first record of a diet with a high abundance of bats consumption in Colombia and South America.}

\normaltalk{Scale of Effect of Landscape Structure on Diurnal Raptors in a Fragmented Tropical Dry Forest}{Marisela Mart\'{i}nez-Ruiz, V\'{i}ctor Arroyo-Rodr\'{i}guez, Iv\'{a}n Franch-Pardo, Katherine Renton}{Theoretical models propose that the scale of effect at which landscape structure best predicts species responses depends on ecological specialization, landscape metrics, and response variables, but empirical support is scarce, especially for apex predators. We aimed to determine scale of effect for diurnal raptors, and to test for differences among ecological traits of habitat and dietary specialization, landscape metrics and response variables. We conducted 1.5 km transect surveys of diurnal raptors at 26 sites in the tropical dry forest of western Mexico from 2016-2018. We also measured landscape metrics (forest cover, matrix hardness, forest patch density, edge density) in 16 concentric landscapes (400 to 3400 ha) around each survey site. We then assessed the landscape size at which each landscape metric best predicted response variables (abundance, species richness, temporal turnover). Finally, we tested for differences in scale of effect among ecological traits of raptors, landscape metrics, and response variables. Our results demonstrated that landscape composition metrics of forest cover and matrix hardness decreased with increasing landscape size, and were the main predictors influencing raptors. Scale of effect differed significantly among landscape metrics, being larger for forest cover than matrix hardness, but did not differ significantly among ecological traits of raptors or response variables. Our results show that scale of effect was mainly driven by landscape metrics, with raptors being more strongly associated with forest cover measured over larger spatial scales. This supports previous findings for other taxa and suggests that forest cover influences dispersal success across larger scales.Modelos te\'{o}ricos proponen que la escala del efecto a la cu\'{a}n la estructura del paisaje predice mejor las respuestas de las especies depende de la especializaci\'{o}n ecol\'{o}gica, las m\'{e}tricas del paisaje y las variables de respuesta, sin embargo el soporte emp\'{i}rico es escaso, particularmente para los depredadores tope. Nosotros determinamos la escala del efecto para las rapaces diurnas, y probamos diferencias entre rasgos ecol\'{o}gicos de especializaci\'{o}n del h\'{a}bitat y diera, las m\'{e}tricas del paisaje y las variables de respuesta. Llevamos a cabo muestreos de rapaces en transectos de 1.5 km en 26 sitios del bosque tropical seco del oeste de M\'{e}xico de 2016 a 2018. Adem\'{a}s calculamos las m\'{e}tricas del paisaje (cobertura forestal, dureza de la matriz, densidad de parches de bosque y densidad de borde) en 16 paisajes conc\'{e}ntricos (400 a 3400 ha) alrededor de cada sitio de estudio. Posteriormente, evaluamos en tama\~{n}o de paisaje al cual cada m\'{e}trica del paisaje predice mejor las variables de respuesta consideradas (abundancia, riqueza, recambio temporal). Finalmente, probamos diferencias en la escala del efecto entre rasgos ecol\'{o}gicos, m\'{e}tricas del paisaje y variables de respuesta). Encontramos que las variables de composici\'{o}n del paisaje, cobertura forestal y dureza de la matriz, decrecen coon mayor tama\~{n}o del paisaje, y fueron los principales predictores influyendo en las rapaces. La escala del efecto difiere significativamente entre m\'{e}tricas del paisaje, con mayor escala del efecto para la cobertura forestal que para la dureza de la matriz. Sin embargo, no hubo diferencias significativas entre rasgos ecol\'{o}gicos de especializaci\'{o}n de las rapaces o entre variables de respuesta. Concluimos que la escala del efecto esta influenciada principalmente por las variables del paisaje, y las rapaces se asocian fuertemente a la cobertura forestal medida en escalas espaciales mayores. Esto da soporte a estudios previos con otros taxa ya que sugieren que la cobertura forestan influye en el \'{e}xito de dispersi\'{o}n en escalas espaciales mayores.}

\normaltalk{Important Sites for Endemic Bird Conservation in the Chaco}{Romina Cardozo, Ricardo Machado}{Habitat loss and fragmentation are the most important threats to biodiversity worldwide, in consequence preventive conservation actions must protect restricted species. The Gran Chaco is a biogeographic region in South America, is an open vegetation biome experiencing high habitat conversion due to cattle ranches expansion threatening bird's communities and endemic birds. Our main goal was to identify important areas for the Chaco endemic birds considering their distribution and the fragmentation and connectivity of the Paraguayan Chaco natural vegetation. To identify fragments clusters was used graph theory, habitat connectivity was based on a least-cost approach and was mapped the potential distribution of endemic bird richness. Results indicates two important regions for priority actions: potential sites for conservation and landscape management. First region represents the low fragmentation area with the largest and connected clusters, represented by Protected Areas and adjacent natural vegetation in Northern Chaco, was the region with least habitat conversion promoting patches connectivity, which mean resources abundance to maintain communities. Second region represents the high fragmentation area with small and isolated patches located in Central Chaco, corresponded to historical and high habitat conversion, suggesting that anthropic activities needs to promote connectivity patches to reduce impacts and improve bird's dispersal between fragments. This analyses could support endemic bird's conservation in a regional context, considering habitat loss in recent years, updating their endangered status and filling gaps about their remaining distribution. These areas can be used in zoning policies that promote and create strategies for landscape connectivity.La p\'{e}rdida y fragmentaci\'{o}n del h\'{a}bitat es la amenaza m\'{a}s importante a la biodiversidad, en consecuencia, acciones preventivas de conservaci\'{o}n deben proteger las especies restringidas. El Gran Chaco es una regi\'{o}n biogeogr\'{a}fica en Sudam\'{e}rica, un bioma con vegetaci\'{o}n abierta que est\'{a} experimentando una alta conversi\'{o}n de h\'{a}bitat debido a la expansi\'{o}n de tierras para ganado, amenazando las comunidades de aves y aves end\'{e}micas. Nuestro objetivo fue identificar \'{a}reas importantes para las aves end\'{e}micas del Chaco considerando su distribuci\'{o}n, la fragmentaci\'{o}n y conectividad de la vegetaci\'{o}n natural del Chaco Paraguayo. Para identificar los fragmentos importantes fue utilizada la teor\'{i}a de los grafos, la conectividad del h\'{a}bitat fue calculada considerando la dispersi\'{o}n por menor costo y fue mapeada la distribuci\'{o}n potencial del \'{a}rea con mayor riqueza de aves end\'{e}micas. Los resultados indican dos regiones importantes para acciones prioritarias: sitios potenciales de conservaci\'{o}n y gesti\'{o}n del paisaje. La primera regi\'{o}n representa el \'{a}rea con baja fragmentaci\'{o}n con fragmentos mayores y conectados, representados por Áreas Protegidas y vegetaci\'{o}n natural en el Chaco Norte, fue la regi\'{o}n con menos conversi\'{o}n del h\'{a}bitat promoviendo la conectividad, lo que se traduce en recursos abundantes para mantener las comunidades. La segunda regi\'{o}n representa el \'{a}rea de alta fragmentaci\'{o}n, caracterizada por fragmentos peque\~{n}os y aislados en el Chaco Central, que corresponde a una regi\'{o}n de alta conversi\'{o}n hist\'{o}rica del h\'{a}bitat, sugiriendo que las actividades antr\'{o}picas necesitan promover la conectividad del h\'{a}bitat para reducir los impactos y favorecer la dispersi\'{o}n de las aves entre los fragmentos. Este an\'{a}lisis servir\'{a} como base para la conservaci\'{o}n de las aves end\'{e}micas en un contexto regional, considerando la p\'{e}rdida del h\'{a}bitat en los últimos a\~{n}os, actualizando su estado de amenaza y visualizando datos sobre su distribuci\'{o}n actual. Estas \'{a}reas pueden ser utilizadas en pol\'{i}ticas de ordenamiento que promuevan la creaci\'{o}n de estrategias para la conectividad del paisaje.}

\normaltalk{The Avian Gut Microbiome and Its Effects on the Fitness of a Passerine Bird}{Shane E Somers, Gabrielle L Davidson, John L Quinn}{The gut microbiome refers specifically to the community of microbes found throughout the gastro-intestinal tract of an individual. This community is determined by both genetic and environmental factors. Altricial birds are an important study organism because they remain in the nest for several weeks after hatching. This may buffer them from environmental effects. Furthermore, altricial birds are the fastest growing terrestrial vertebrates which likely amplifies host level effects and related fitness consequences. Teasing apart the relative influence of these effects is vital to our understanding of how this system develops.    Nestling weight is a strong proxy for survival post-fledging which has been empirically demonstrated in great tits. As microbial diversity has been positively correlated with nestling body condition, I hypothesised that this relationship (positive relationship between diversity and mass) would hold in our birds and would translate into increased survival (to fledging). Finally, I hypothesised that adult survival (or detection the following year) would be positively related to diversity because of this same relationship between diversity and condition. Against expectations, diversity was negatively related to body mass in nestlings and there was no influence of diversity on survival in nestlings or adults. However a number of potentially probiotic taxa were identified which are associated with survival in both nestlings and adults.}

\normaltalk{Exploring MHC Evolution in Two Seasonally Sympatric Populations of Dark-Eyed Juncos}{Esther M Fernandez, Andrew J Pacheco, Danielle J Whittaker, Joel W Slade}{The evolution of migratory behavior depends, in part, on surviving exposure to novel pathogens throughout multiple environments (e.g. breeding and wintering grounds, and stopover sites). In contrast, the loss of migratory behavior should depend on adapting to local pathogens. A candidate gene family that undergoes pathogen-mediated selection is the highly polymorphic major histocompatibility complex (MHC). MHC genes encode cell-surface proteins that present pathogen-derived antigens to T-cells to initiate an adaptive immune response. Theory suggests that pathogen-mediated selection at MHC can favor locally adapted alleles (good gene effects) and/or a combination of alleles (heterozygote advantage). Here, we explore how avian haemosporidian parasites may explain MHC class I allelic diversity in two parapatric subspecies of Dark-Eyed Juncos, the migratory Slate-Colored Junco (Junco hyemalis hyemalis), and the non-migratory Carolina Junco (Junco hyemalis carolinensis). We sampled the two subspecies (migratory = 35; resident = 34) in spring 2018 at the Mountain Lake Biological Station (Pembroke, VA). We hypothesize that Slate-Colored Juncos are undergoing parasite-mediated balancing selection at MHC because they may be exposed to a greater diversity of avian haemosporidians during migration, in comparison to the resident Carolina Junco. To test this hypothesis, we will characterize individual MHC class I exon 3 (polymorphic antigen-binding site) with high-throughput sequencing, and characterize avian haemosporidian by nested PCR and Sanger sequencing. The results will be discussed. Ultimately, these results will give us insights into how differences in migratory behavior can shape the evolution of adaptive immune genes in songbirds.}

\normaltalk{Effect of Nest Predation on Flammulated Owl (Psiloscops Flammeolus) Nest Site Selection}{Jordan Ellison, Brian Linkhart}{Nest predation is a major factor affecting the evolution of life histories and habitat selection across avian taxa. Few studies have examined the relationship among multiple potential evolutionary responses to predation. Little is known about the strategies Flammulated Owls (Psiloscops flammeolus), a cavity-nesting species with a low reproductive rate, utilize to minimize nest predation. I examined how Flammulated Owls have adapted in response to predation by the North American Red Squirrel (Tamiasciurus hudsonicus). Habitat variables at owl nests across four study areas in central Colorado studied from 2010-2019 were quantified at nest trees and adjacent forest for comparison of available but unused sites. In 2018, squirrel density was estimated within one study area and found to be greater in owl territories than randomly selected unused areas of the same size, showing that owls do not avoid areas of high predator density. Nest cavity and nest tree heights were on average higher at owl nests than available but unused nest cavities and trees. Successful nest cavities and trees were also higher than those of depredated nests. Additionally, Flammulated Owls demonstrated preference for tree cavities excavated by Northern Flickers (Colaptes auratus). Selection for higher nesting cavities by Flammulated Owls may be an adaptive response to perceived predation risk, as owls in lower cavities experience decreased nesting success. Higher cavities may allow Flammulated Owls more time to respond to a predator approaching from the ground.}

\normaltalk{Introgression and Character Displacement in an Amazonian Piculet Contact Zone}{Hevana Lima}{Contact zones are natural laboratories for examining interspecific interactions such as competition and reproductive isolation. By intensely sampling a 600-km stretch of continuous Amazonian floodplains centered on their contact zone, we analyzed phenotypic geographic variation in Varzea (Picumnus varzeae) and White-barred (Picumnus cirratus macconnelli) piculets. At the contact zone, we detected a roughly 10-km-wide region containing varied individuals with mixed plumage traits, consistent with hybridization. Gradual size reduction in P. varzeae over hundreds of kilometers, beginning at contact with the larger P. c. macconnelli, suggests introgression of size traits beyond the hybrid zone. Plumage variation away from the hybrid zone, however, showed a pattern opposite to that predicted by hybridization. Varzea Piculet plumage was most different from that of the White-barred Piculet immediately adjacent to the hybrid zone, where birds were plain brown in contrast with the latter's heavily barred plumage. With increasing distance, Varzea Piculets displayed increasing plumage variability and barring, tending toward White-barred Piculet plumage; some individuals of P. varzeae sampled farthest from macconnelli were indistinguishable in plumage from the latter. We interpret this pattern as plumage character displacement near the hybrid zone to reinforce reproductive isolation in species that appear to be ecologically and vocally nearly identical and unable to co-occur. That the hybrid zone is very narrow and plumage differences are most distinct adjacent to it suggests strong selection against hybrids. Thus, hybrid introgression leads to plumage variability in the Varzea Piculet that is only expressed far enough from the White-barred Piculet to avoid further hybridization.Zonas de contato são laborat\'{o}rios naturais para examinar interações interespec\'{i}ficas como competição e isolamento reprodutivo. Atrav\'{e}s de uma intensa amostragem em um trecho de 600-km de florestas de v\'{a}rzea amazônicas com foque em sua zona de contato, n\'{o}s analisamos a variação geogr\'{a}fica do fen\'{o}tipo em Picumnus Varzeae e Picumnus cirratus macconnelli. Na zona de contato, n\'{o}s detectamos uma região com cerca de 10-km de extensão contendo indiv\'{i}duos variados como caracter\'{i}sticas de plumagem mistas, consistente com a hip\'{o}tese de hibridização. Uma redução gradual de tamanho em Picumnus varzeae ao longo de centenas de quilômetros, começando no contato com Picumnus cirratus macconnelli, sugere introgressão de caracter\'{i}sticas de tamanho al\'{e}m da zona h\'{i}brida. A variação de plumagem observada al\'{e}m da zona h\'{i}brida, entretanto, mostrou um padrão oposto ao previsto pela hibridização. A plumagem em Picumnus varzeae foi mais diferente da de Picumnus cirratus macconnelli imediatamente pr\'{o}ximo a zona h\'{i}brida, onde as aves apresentaram-se de cor marrom em contraste como a plumagem barrada de Picumnus cirratus macconnelli. N\'{o}s interpretamos esse padrão como deslocamento de car\'{a}ter de plumagem pr\'{o}ximo a zona h\'{i}brida como um reforço ao isolamento reprodutivo nessas esp\'{e}cies que parecem ser ecologicamente e vocalmente idênticas. Que a zona h\'{i}brida \'{e} muito curta e diferenças na plumagem são mais distintas nas regiões pr\'{o}ximas a ela sugerem forte seleção contra h\'{i}bridos. Assim, introgressão h\'{i}brida leva a variabilidade na plumagem em P. varzeae, essa variabilidade por\'{e}m s\'{o} \'{e} expressa quanto maior a distância com Picumnus cirratus macconnelli buscando evitar futura hibridização.}

\normaltalk{Evolutionary History of the Stripe-Backed Antbird: Implications of a Historical Connection Between Two Neotropical Dry Forests}{Maria W Pil, Lays Viturino de Freitas, Gustavo Cabanne, Luciano N Naka}{Allopatric populations of closely related taxa are often interpreted as evidence of past connections or vicariant events. The patchy distribution of Neotropical Dry forest has been subject to much debate concerning their origins and biogeographic connections. Here, we investigate the evolutionary and biogeographical relationship of the two largest patches of dry forests in the Neotropics: The Caatinga and the Chaco. To reach that end, we investigated the evolutionary history of the Striped-back Antbird (Myrmorchilus strigilatus), a species with two subspecies almost perfectly restricted to these two dry forests. We used DNA sequencing of ultra-conserved elements to sample 26 individuals collected throughout the Caatinga and Chaco and recovered 3,484 single nucleotide polymorphisms (SNPs). Bayesian structuring and phylogenetic analyses shows a clear separation between the populations restricted to the Caatinga and those to the Chaco, which may have occurred during the Pleistocene at about 0.63 (0.37-0.92) Mya. Within the Caatinga, we also found genetically structured groups. Principal component and discriminant analyses show that individuals are genomically grouped into three clusters: individuals located in São F\'{e}lix, a site adjacent to the Caatinga; individuals collected near Chapada Diamantina and a third group composed of individuals from different regions of the Caatinga. Given the allegedly low dispersal ability of this species, our study supports a recent connection between the two largest patches of Neotropical dry forests and point out a more complex evolutionary history in the geologically complex and heterogeneous Caatinga Dry forest.}

\normaltalk{Bird Assemblages of Mangrove Forest at a Coastal Strip of Veracruz, Mexico}{Adri\'{a}n Cipr\'{e}s, Beatriz dS Bolivar, Edward A Ellis}{Coastal areas are very productive in providing multiple goods and services. However, human activities such as urbanization and agricultural work create gradients of alteration in coastal ecosystems. In Mexico, mangrove forests are immersed in human-modified landscapes, threatening the birds that use them as habitat. The aim is to compare the composition of bird species in mangrove fragments in an urban alteration gradient in the central coastal strip of Veracruz, Mexico. We will calculate the urban and agricultural area in three gradients of alteration, using landscapes of 2km and 4km radius. Landsat 8 images will be used with ArcMap 10.5 for characterized the landscapes. To characterize bird assemblages, fixed-width transects surveying are made within the mangrove fragments. Diversity indices and statistical tests will be used to comparate the assemblages between fragments in alteration gradients. The preliminary results including the estimated richness of 146 species, 113 species have been registered within the transects. ``Arroyo Moreno" has been the fragment of mangroves with the greatest species richness (86), although it is in a landscape dominated by urban areas. The most dominant species was Northern Waterthrush (Parkesia noveboracensis), Mangrove Warbler (Setophaga petechia eritachorides), Golden-fronted Woodpecker (Melanerpes aurifrons) and White Ibis (Eudocimus albus). 15 species have been registered under some threat category by the Mexican law NOM-059-SEMARNAT-2019. Understanding the relationship between landscape elements and mangrove bird assemblages would allow us to propose conservation strategies for this habitat.Las zonas costeras son muy productivas al proporcionar múltiples bienes y servicios. Sin embargo, actividades humanas como la urbanizaci\'{o}n y la agricultura crean gradientes de alteraci\'{o}n en los ecosistemas costeros. En M\'{e}xico, los manglares est\'{a}n inmersos en paisajes modificados por humanos, amenazando a las aves que los usan como h\'{a}bitat. El objetivo es comparar la composici\'{o}n de especies de aves en fragmentos de manglares en un gradiente de alteraci\'{o}n urbana en la franja costera central de Veracruz, M\'{e}xico. Calcularemos el \'{a}rea urbana y agr\'{i}cola en tres gradientes de alteraci\'{o}n, utilizando paisajes de 2 km y 4 km de radio. Las im\'{a}genes Landsat 8 se utilizar\'{a}n con ArcMap 10.5 para caracterizar los paisajes. Para caracterizar los ensamblajes de aves, se realizan monitoreos con el m\'{e}todo de transectos de ancho fijo dentro de los fragmentos de manglar. Los \'{i}ndices de diversidad y las prueba estad\'{i}sticas se utilizar\'{a}n para comparar los ensamblajes entre fragmentos en gradientes de alteraci\'{o}n. Los resultados preliminares incluyen una riqueza estimada de 146 especies, 113 especies se han registrado dentro de los transectos. ``Arroyo Moreno" ha sido el fragmento de manglar con mayor riqueza de especies (86), aunque se encuentra en un paisaje dominado por zonas urbanas. La especie m\'{a}s dominante fue el Chipe charquero (Parkesia noveboracensis), el Chipe manglero (Setophaga petechia eritachorides), el Carpintero Cheje (Melanerpes aurifrons) y el Ibis blanco (Eudocimus albus). De acuerdo a ley mexicana NOM-059-SEMARNAT-2019, 15 especies se encuentran bajo alguna categor\'{i}a de amenaza. Comprender la relaci\'{o}n entre los elementos del paisaje y los ensamblajes de aves de manglar nos permitir\'{i}a proponer estrategias de conservaci\'{o}n para este h\'{a}bitat.}

\normaltalk{Post-Breeding Bird Use of Wildlife Openings in a Heavily Forested Landscape}{Hannah L Clipp, Christopher T Rota, Petra B Wood}{In forested landscapes, wildlife openings created by land managers are often meant to provide early-successional habitat and food resources primarily for game bird species such as Wild Turkey and American Woodcock and secondarily for other early-successional/shrubland birds. These wildlife openings may also provide resources during the post-breeding period (mid-June to mid-August) for bird species that breed in core forest. Therefore, we aimed to quantify the post-breeding bird community of wildlife openings within the Monongahela National Forest in West Virginia, specifically focusing on two guilds (early-successional vs. forest-interior) and age classes (hatch-year vs. after-hatch-year). In the summer of 2019, we used mist nets to capture 126 post-breeding individuals belonging to 15 early-successional and 13 forest-interior species within 12 wildlife openings where game birds had been detected earlier in the season. The most common early-successional species were Indigo Buntings and Song Sparrows, while the most common forest-interior species were Magnolia Warblers and Black-throated Green Warblers. The majority of individuals belonging to the forest-interior guild (80\%) were hatch-year birds. Preliminary results suggest that differences in guild richness and age ratios among wildlife openings may be related to local vegetative cover more than size or management regime. Ultimately, our study is the first to quantify the post-breeding bird community of wildlife openings within the Monongahela National Forest and document the sympatric use of wildlife openings by game birds and post-breeding songbirds, particularly young forest-interior birds.}

\normaltalk{Russian Olive Habitat along an Arid River Supports Fewer Bird Species, Functional Groups and a Different Species Composition Relative to Mixed Vegetation Habitats}{Sean M Mahoney, Anna Nellis B Smith, Peter J Motyka, Erick J Lundgren, Raemy R Winton, Bo Stevens, Matthew Johnson}{The establishment and naturalization of non-native Russian olive (Elaeagnus angustifolia) in southwestern US riparian habitats is hypothesized to have negative implications for native flora and fauna. Despite the potential for Russian olive establishment in new riparian habitats, much of its ecology remains unclear. Arid river systems are important stopover sites and breeding grounds for birds, including some endangered species, and understanding how birds use Russian olive habitats has important implications for effective non-native species management. We compared native bird use of sites that varied in the amount of Russian olive and mixed native/non-native vegetation along the San Juan River, UT, USA. From presence/absence surveys conducted in 2016 during the breeding season, we found 1) fewer bird species and functional groups used Russian olive habitats and 2) the composition of species within Russian olive habitats was different from the composition of species in mixed native/non-native habitats. Our results suggest Russian olive may support different bird compositions during the breeding season and as Russian olive continues to naturalize, bird communities may change. Finally, we highlight the paucity of research surrounding Russian olive ecology and stress the need for rigorous studies to improve our understanding of Russian olive ecology.   Se asume que la introducci\'{o}n y naturalizaci\'{o}n de la especie no nativa conocida como \'{a}rbol del para\'{i}so (Elaeagnus angustifolia) en los h\'{a}bitats ribere\~{n}os en el suroeste de EE.UU. tiene efectos negativos para la flora y fauna nativa. A pesar del potencial para la expansi\'{o}n del \'{a}rbol del para\'{i}so en nuevos h\'{a}bitats ribere\~{n}os, mucha de su ecolog\'{i}a aún se desconoce. Los ecosistemas \'{a}ridos de r\'{i}o son importantes sitios de escala y sitios de reproducci\'{o}n para aves, incluyendo algunas especies en peligro de extinci\'{o}n, por lo que entender como las aves usan estos h\'{a}bitats compuestos por \'{a}rboles del para\'{i}so tiene implicaciones importantes para el manejo efectivo de especies no nativas. Comparamos el uso que las aves le dieron a sitios con diferente cantidad de \'{a}rboles del para\'{i}so y h\'{a}bitats mixtos con vegetaci\'{o}n nativa/no nativa a lo largo del R\'{i}o San Juan, UT, EE. UU. De los muestreos en donde se registr\'{o} presencia/ausencia durante la temporada de reproducci\'{o}n encontramos que 1) menos especies y grupos funcionales usaron el h\'{a}bitat compuesto por \'{a}rboles del para\'{i}so y 2) la composici\'{o}n de especies dentro del h\'{a}bitat compuesto por \'{a}rboles del para\'{i}so es diferente a la composici\'{o}n de especies de h\'{a}bitats de vegetaci\'{o}n mixta nativa/no nativa. Nuestros resultados sugieren que los h\'{a}bitats compuestos por \'{a}rboles del para\'{i}so pueden mantener diferentes composiciones de especies de aves durante la temporada de reproducci\'{o}n y a medida que el \'{a}rbol del para\'{i}so continúe su naturalizaci\'{o}n, las comunidades de aves podr\'{i}an cambiar. Finalmente, resaltamos la escasez de investigaciones acerca de la ecolog\'{i}a del \'{a}rbol del para\'{i}so y destacamos la necesidad de realizar estudios rigurosos que ayuden a mejorar nuestro entendimiento de la ecolog\'{i}a del \'{a}rbol del para\'{i}so.  }

\normaltalk{Habitat Suitability for an Urbanized, Naturalized, and Endangered Species: Red-Crowned Parrots in the USA}{Simon Kiacz, Hsiao-Hsuan " Wang, Donald J Brightsmith}{Environmental alterations by humans have hampered biodiversity in general, but there are a number of species which have flourished by utilizing human modified environments. In the order Psittaciformes, over 40 species have established naturalized populations and virtually all of these populations are reliant to some degree on urbanized areas. One of these species, the IUCN endangered Red-crowned Parrot (Amazon viridigenalis), has established multiple naturalized populations throughout the United States. Although these populations may serve as beneficial reservoirs of individuals and genetic variation for the species, relatively little attention has been paid to understanding the habitat variables allowing them to persist. We use Maxent, a presence-only species distribution modeling program, to predict suitable habitat for Red-crowned Parrots throughout the contiguous United States to better understand the variables that are most strongly associated with species presence. Presence points used included nesting, foraging, and roosting locations gathered in south Texas over four years of field work and eBird reports from throughout the United States. The environmental layers used in Maxent included bioclimatic data, human density, and land cover variables. Our model can assist in the development of an integrated conservation strategy through (i) focusing future survey and research efforts on areas with a high likelihood of presence (ii) aiding in selection of areas for conservation and restoration and (iii) framing future research questions including those necessary for predicting responses to climate change.}

\normaltalk{Extra-Pair Copulation and Male Parental Investment in Cassin's Sparrow (Peucaea Cassinii)}{Garrett Visser, Claire Ramos, Scott V Edwards, Daren Card}{The Cassin's Sparrow (Peucaea cassinii) is a species of songbird that breeds in the southern grasslands of North America that is greatly understudied. Cassin's Sparrows are decreasing at a rate of 3.4\% annually in Colorado. While Cassin's Sparrows are monomorphic and seemingly monogamous, extra-pair copulations have been observed in our population and males have unusually large cloacal protuberances, suggesting possible sperm competition.  If extra-pair paternity is high in this species, then we predict that male parental care would be reduced. To test this, we quantified extra-pair paternity rates and male parental care. To observe parental care, we placed video cameras near nests and recorded the identity and frequency of individuals feeding at the nest. We tested parentage of the Cassin's Sparrow chicks using double digest restriction-site associated DNA (ddRAD) sequencing. This will allow us to determine which males have contributed genetically to a clutch. We will use this data to determine the frequency of extra-pair paternity and correlate it to male feeding rates to assess if extra-pair paternity influences male parental investment. We found that while females are responsible for incubation, nestling feeding rates are roughly equal between males and female. Extra-pair paternity in a sexually monomorphic bird species is considered unusual, and if we can show this behavior in Cassin's Sparrows, it will increase our knowledge on their breeding biology and may help future conservation of this species.}

\normaltalk{Comparative Breeding Success, Predators, and Nesting Substrate of Worthen'S Sparrow and Other Birds Using an Agricultural Landscape in Northern Mexico}{Graciela E Villanueva-V\'{a}zquez, Irene Ruvalcaba-Ortega, Alexander Pe\~{n}a-Peniche, Ricardo Canales-del-Castillo}{The Worthen's sparrow (Spizella wortheni) is an endemic and endangered species, with a small population size and a low breeding success. The understanding of the intrinsic and extrinsic factors associated with its nesting success rate is scarce, but indispensable for implementing actions aimed at its conservation. Therefore, we set the objective of comparing the breeding success of the Worthen's sparrow to other species that nest in similar substrates (shrubs and forbs) within the same agricultural landscape and their association with microhabitat characteristics (extrinsic). We monitored 191 nests of 4 species during three breeding seasons (2017- 2019) in La Concha, Galeana, Nuevo Le\'{o}n. Daily survival rate was variable among years for all species, the lowest during the first season, particularly for the Worthen's sparrow (0.8783). We found a strong (fc = 0.72) and significant (p $<$0.001) association between each species and its nesting substrate. Nest height significantly influenced the survival (p $<$0.05) for S. wortheni. The main predator of nests for all species were snakes (79\%). The results indicate that birds nesting in this area have relatively low reproductive success rates and a high inter-annual variation. However, these differences were wider for the Worthen's sparrow, indicating that intrinsic factors may be relevant.}

\normaltalk{Friend or Foe? the Influence of Neighbours on Mountain Bluebird Reproductive Success.}{Stephen Joly, Matthew Reudink, Thomas Dickinson, Nancy Flood, Jonathan Van Hamme}{Mountain bluebirds (Sialia currucoides) and tree swallows (Tachycineta bicolor) are secondary cavity nesters that will frequently use artificial nest boxes. Although they directly compete for nest boxes at the start of the breeding season, once ownership is established and eggs laid, they can coexist and nest in very close proximity. Both species are socially monogamous with both adults defending their nest from perceived threats, but tree swallows display a much higher intensity of nest defense behaviour. Do mountain bluebirds benefit by way of increased reproductive success from the defensive behaviour of a tree swallow neighbour? GPS coordinates for 300 nest boxes were recorded on 11 established bluebird trails routes in South-central British Columbia, Canada. For each breeding season (April-August) from 2011 to 2019, first egg date, hatch date, number of nestlings, number of fledglings and fledge date were recorded for all occupied boxes. Using the GPS coordinates, a distance matrix was created to determine the distance to nearest tree swallow and mountain bluebird neighbour and related to the proportion of eggs hatched and nestlings fledged from each mountain bluebird nest.}

\normaltalk{Cerulean Warbler Habitat Selection, Breeding Biology, Survival, and Space Use in the Ozark Region}{Jacob L Wessels, Than J Boves}{Cerulean Warblers (Setophaga cerulea; `ceruleans') are a declining species of Nearctic-Neotropical migratory songbird that breeds in deciduous forests of eastern North America. Because habitat loss and degradation are thought to be major causes of this decline, active habitat management has become an important part of the conservation strategy for ceruleans. However, ceruleans exhibit regional variation in habitat selection, reproductive success, and population trend, which complicate efforts to manage for them. Information on the ecology of local populations is therefore necessary to effectively conserve this species. Although population trend and migratory route data suggest that the relatively large population of ceruleans in the Ozark region of Arkansas and Missouri, USA may constitute a `natural population', ceruleans have been understudied there. To address this important knowledge gap, we studied their habitat selection, breeding biology, survival, and space use in the Arkansas Ozarks. Ceruleans demonstrated selection of various habitat features at all scales. Most saliently, they preferred areas with relatively large-diameter trees at all scales, which is largely consistent with studies from other areas. Inconsistent with previous studies, ceruleans in our area appeared to avoid white oaks at the territory scale, but this may have been confounded by selection of riparian habitat. We located 25 active nests and estimated entire-period nest survival at 32\%. We also estimated apparent within-season weekly survival at 0.95, return rate at 29\%, and mean territory size at 1.14 ha. Our results will help guide regional/local habitat management and provide parameter estimates for use in integrative full-annual-cycle population models.}

\normaltalk{Examining the Effects of Settlement and Habitat Preferences on Eastern Bluebirds Breeding Biology}{Melissa L Bailey, Joanna Hubbard}{The environmental conditions present early in an organism's life can greatly impact development of behavioral, morphological, and physiological traits, which can then affect overall quality and survival potential. Thus, understanding the factors that determine settlement and habitat preference aids in the greater understanding of offspring quality within a species. Eastern bluebirds (Sialia sialis) are a multi-brooded, secondary cavity nesting species, making them an excellent study system to observe the effects of habitat choice. Their inability to excavate their own nesting cavities paired with their aggressive territoriality causes there to be a natural limit on the number of high quality sites available to a population. In addition, previous research has found that nestlings exhibit variation in UV-blue plumage coloration and overall quality in response to developmental conditions. One condition which has received less attention is the effect of cavity age on settlement patterns and nestling development. To study this, three sites in Northeastern, Missouri were studied, two of which contained all newly made nest boxes while one contained a mix of old and new boxes. Preliminary results show variation in settlement patterns with greater levels of settlement in the sites that have a mixture of new and old boxes. Further analyses will examine variation in habitat preference between sites with predominantly old or new boxes as well as variation in reproductive output, reproductive initiation date, nestling body condition, and nestling ornamentation production between new and old boxes.}

\normaltalk{Chickadees Increase Provisioning Effort to Compensate for Poor Prey Quality During the Nestling Period}{Sarah Sen\'{e}cal, Julie-Camille Riva, Ryan S O'Connor, Christian Nozais, François V\'{e}zina}{The early acquisition of nutrition can significantly impact nestling fitness by increasing chances of survival and recruitment after fledging. The effort invested by parents towards the provisioning of nestlings is therefore crucially important and represents a key link between habitat resources and reproductive success. Recent observations suggest that provisioning rate (i.e., provisioning effort) has little effect on nestling growth rate. However, these studies did not include measurements of prey quality, despite the fact that parents could adjust their provisioning effort to account for diets of differing nutritional value. Here, we hypothesized that provisioning rates would negatively correlate with prey quality. We monitored breeding effort in black-capped (Poecile atricapillus) and boreal chickadees (Poecile hudsonicus) in the Forêt d'enseignement et de recherche Macpès, near Rimouski, Qu\'{e}bec, Canada. Provisioning rates were quantified using RFID systems fitted to nest boxes. Prey quality was determined through bomb calorimetry of nestling stomach contents obtained non-invasively. Mean daily growth rate was consistent across 8 years in both species. However, provisioning rate and prey quality differed by as much as 57\% among years. Consequently, parents brought 47\% more food to the nest in years of low prey quality. Thus, provisioning rate was higher in years of poor prey quality, and this likely explains the lack of a relationship between provisioning rate and growth rate often reported in this and other studies. These data support our hypothesis that parents increase provisioning effort to compensate for poor prey quality, thereby maintaining nestling growth rate.}

\normaltalk{Food Abundance as an Environmental Cue for the Start of Reproduction of Elaenia Albiceps}{Cristian A Gorosito, Diego T Tuero, V\'{i}ctor R Cueto}{The time of the year when birds reproduce has a great impact on their reproductive success. That is, if they reproduce when food availability for their nestlings is low, their survival could be reduced. Therefore, birds should synchronize the maturation of their gonads with the most favorable environmental conditions for reproduction. Photoperiod length is the main cue that birds use to achieve this synchronization. However, using other cues, such as food abundance, is necessary to refine the most suitable moment for reproduction. The aim of this work was to study the synchronization between the food types consumed by the long-distance migratory bird Elaenia albiceps and its reproduction. The effects of ripe fruit, arthropods and caterpillar abundances on the breeding intensity of this species were evaluated. Ripe fruit and caterpillar abundances were the food resources that had the greatest effects on the number of nests with eggs. These items could constitute important food sources for the preparation of reproduction in adults, providing protein and calcium for egg formation. Ripe fruit abundance also had a great effect on the number of nests with nestlings. It was observed a synchronization between the moment of maximum abundance of this food resource and the maximum abundance of nests with nestlings. Ripe fruits have a high antioxidant and calcium content, thereby they would be beneficious for nestling growth. Hence, ripe fruit abundance would be the main cue to refine the start of reproduction of E. albiceps in Patagonian forests.}

\normaltalk{Scarlet Macaw Nesting Ecology and Behavior in a Pristine Forest}{Gabriela Vigo Trauco, Donald Brightsmith}{Documenting parental care by wild birds is important for understanding reproductive success and promoting conservation. However, little is known about how wild parrots care for their offspring in the nest and how to use this knowledge to inform conservation actions. We documented wild Scarlet Macaw parental behavior in the lowland forests of southeastern Peru during 19 breeding seasons. In the $>$ 1900 hours of video analyzed, I found that females provide most feedings to chicks (\%\%) but feedings by the male increase significantly as the chicks age. I found that (1) chicks are fed on average 3.6 times per hour throughout the nesting period, (2) there are major diurnal and nocturnal chick feeding peaks, and (3) macaw parents store food in their crops for over 7 hours and use this to feed their chicks at night. We analyzed chick starvation due to brood reduction and found that chick starvation is the leading cause of chick mortality, 27\% of all second hatched chicks starve, and nearly all third and fourth hatched chicks starve. We found no evidence that death by starvation was caused by (1) sibling rivalry, (2) food availability, or (3) hatch weight. I did find that (1) direct control of food distribution within the brood favors first hatch chicks but specifically disfavors second hatch chicks that starve and (2) the larger the age difference between brood members the more likely the second chick would starve. This research has great potential to inform macaw population management and conservation.}

\normaltalk{Breeding Biology and Interspecific Interactions of Cypseloidinae Swifts in Central Brazil}{Renata N Biancalana}{Swifts are long-lived, highly aerial insectivore birds. They are still a particularly overlooked group of birds in the Neotropics, with limited data on their biology and distribution. Although some species are considered common, there are few studies regarding their conservation status or acknowledging the threats that they may face in South America, especially in Brazil. Four Cypseloidine species are breeding residents in the Cerrado, which is considered the most threatened biome in Brazil. Here I present new data on the breeding biology and interspecific interactions of swifts in Central Brazil. Field work was conducted between September 2015 and March 2017 in three states to monitor colonies of Sooty, Great Dusky and White-collared Swifts. Birds were banded, measured, and had blood samples and ectoparasites taken. Recaptured individuals in different years of the survey showed strong nest site fidelity, which is characteristic of this group. Great Dusky and White-collared Swifts were frequently seeing foraging, nesting and preening together, whereas Sooty Swifts were rarely seen on their nests during daylight. Great Dusky Swifts nests were made of moss and rootlets, and many were exposed to dripping water making them damp. Sooty and Great Dusky Swifts had one egg clutches, however they differed in breeding timing, with the latter laying the eggs earlier in the rainy season and not having a second breeding attempt as the former. Nestlings from all species had their bodies covered by dark gray semiplumes before feathers erupted. Nestling mortality was observed, and the main cause was falling from nests.}

\normaltalk{Direct and Indirect Factors on Bird Nestling Growth: a Case Study of the Red-Crested Cardinal using Structural Equation Models}{Facundo X Palacio, Luciano N Segura}{Nestling growth is a key life-history trait in birds, as it determines future individual survival and reproduction. However, how multiple spatial and temporal factors operate both directly and indirectly on this trait remains still little understood. We assessed how different factors (number of siblings, hatching order, time of breeding, isolated vs continuous forest patches, and botfly parasitism) affect nestling growth in the Red-crested Cardinal (Paroaria coronata) in east-central Argentina during three breeding seasons (128 nests and 278 nestlings). We estimated growth (tarsus length and body mass) parameters per nestling from nonlinear mixed models and explored different a priori hypotheses between growth and predictors based on biological knowledge using structural equation models (SEMs). Hatching order and botfly presence had direct negative effects on the upper asymptote and growth rate, whereas the number of siblings and time of breeding had a positive effect on both growth parameters. Time of breeding also had a direct effect on botfly presence, indicating an indirect negative effect of this predictor on nestling growth. Finally, continuous patches had a positive effect on botfly presence, indicating a negative indirect effect of continuous patches on nestling growth. Our results show that different environmental and temporal factors drive directly and indirectly nestling growth. As indirect ecological effects are widespread in nature, the dissection of direct and indirect relationships among variables would be a promising avenue to more comprehensively understand nestling growth in birds.}

\normaltalk{The Riddle of White Toes on Young American Crows}{Jennifer H Dewhurst, Kevin J McGowan, Anne B Clark}{During 30 years of banding nestling American Crows (Corvus brachyrhynchos) in Ithaca, NY, we have regularly recorded unpigmented toes and nails as well as poorly pigmented patches in flight feathers. Unpigmented feathers are associated with poor growth in some species, but the origin of unpigmented patches on fleshy areas appears unknown.  We investigated the co-occurence of unpigmented feathers and feet, and tested their association with condition (weight controlled for nestling age and year) and with temperatures during the laying period when early embryos might be exposed. Finally we leveraged our genealogical records based on banded birds, including years with genetic documentation of parentage, to ask if abnormal pigmentation might be heritable.  Of the 2969 nestlings in 925 broods banded between 1989-2018, 5.8\% (171 in 128 broods) had at least unpigmented toenails up to large patches of white on one or both feet. Unpigmented feet never co-occured with unpigmented feathers and, in contrast with significantly poorer condition in nestlings with abnormal feathers, those with unpigmented toes were in average or better condition. Temperature during laying did not predict the trait in broods. Unpigmented toes were, however, concentrated in a few family lineages and apparently inherited through both female and male lines where banded parents allow us to track it. No captured adults showed abnormal toes and photo-evidence suggests they disappear after fledging. We will present a limited pedigree analysis of unpigmented toes and discuss this trait in relation to hypotheses such as inbreeding, proposed for abnormal feathers.}

\normaltalk{Evaluation of FTA® Cards for the Molecular Detection of Mycoplasma Gallisepticum in Wild Birds from North Georgia}{Magan M Free, Neil Patel, Sawda Islam, Vivica A Pressley, Linda B Purvis, Dawn E Drumtra}{This study was carried out to determine the effectiveness of Flinders Technology Associates filter paper (FTA® card) sampling verses tissue sampling, to detect Mycoplasma gallisepticum (MG) infection in wild birds. MG is a bacterial pathogen affecting wild and domestic birds worldwide. This disease targets the respiratory organs, can cause conjunctivitis, as well as infectious sinusitis in a variety of birds. In wild birds, disease transmission has been documented to occur at communal foraging areas, such as bird feeders. Although wild birds do not usually show MG symptoms (with the notable exception of the house finch (Haemorhous mexicanus)), they remain potential reservoirs for the disease. Current detection techniques focus on swabbing mucus tissue of living or recently deceased birds. The swabbed sample is analyzed by serum plate agglutination (SPA) tests, hemagglutination inhibition tests (HI), or enzyme-linked immunosorbent assay (ELISA), which can be time consuming and not always convenient. The use of FTA® cards for sampling has been shown to be effective in previous studies for other avian species. However, no published work exists on its use for MG detection within songbirds. Through a partnership with Atlanta Audubon Project Safe Flight, we have access to a variety of native and migrating birds in North Georgia. Bird tissues from possibly infected organs were imprinted on FTA® cards and compared to traditional tissue sampling. All samples analyzed using molecular techniques. We hypothesize that FTA® cards are an efficient and dependable alternative collection method for molecular detection of MG in songbirds.}

\normaltalk{Interactions Between Avian Pox and Malaria in Hawaiian Forest Birds}{Michael D Samuel, Bethany L Woodworth, Carter T Atkinson, Patrick J Hart, Dennis A LaPointe}{Despite the purported role of avian pox (Avipoxvirus spp.) in the decline of endemic Hawaiian birds, few studies have been conducted on the dynamics of this disease, its impact on freeliving avian populations, or its interactions with avian malaria (Plasmodium relictum). We conducted four longitudinal studies of 3--7 yr and used generalized linear models to evaluate crosssectional prevalence of active and past pox infection. Our goal was to understand how species, season, elevation, malaria infection, and other factors influenced pox infection in three native ($<$U+02BB$>$Apapane, Hawai$<$U+02BB$>$i $<$U+02BB$>$Amakihi, $<$U+02BB$>$I$<$U+02BB$>$iwi), and one introduced (Japanese White-eye) birds across low-, mid-, and high-elevation forests on Hawai$<$U+02BB$>$i Island. We used multistate capture-recapture (longitudinal) models to estimate pox infection rates, recovery rates, and potential pox-related mortality. Pox infection rates were highest in low-elevation forests, followed by mid-elevations, and lowest in high-elevations. We found seasonal changes in pox prevalence in low- and mid-elevation forests, but not in high elevations where pox infection was low. Seasonal and elevation patterns of pox infection are like those for avian malaria, strongly implicating mosquito vectors as the primary source of transmitting both diseases. Most native Hawaiian birds recovered from pox infection within 6 months. Contrary to our expectations, we found no direct evidence that pox is a substantial mortality factor in the three native bird species studied. Birds with chronic malaria infection were more likely to have both active pox infection and healed pox lesions suggesting a synergistic interaction that may influence the evolution of pox virulence.}

\normaltalk{Identifying the Genetic Mechanisms Underlying Resistance or Tolerance to Avian Malaria in Hawaiian Honeycreepers}{Amanda K Navine, Kristina L Paxton, Patrick J Hart, Carter T Atkinson, Robert C Fleischer}{Inflated extinction rates among Hawaiian honeycreepers (subfamily Drepanidinae) due to avian malaria (Plasmodium relictum) and its mosquito vector (Culex quinquefasciatus) showcase how introduced diseases can decimate naive ecosystems. Most honeycreeper species only persist in high-elevation refugia from disease, where temperature limits development of both mosquitos and parasites. As climate change raises global temperatures we may see a rapid loss of these avian refugia. `Amakihi (genus Chlorodrepanis), are the only honeycreepers that remain at high densities in low-elevation areas where malaria prevalence is high. Experimental infections revealed different low-elevation `amakihi populations have evolved separate immune strategies (tolerance vs. resistance), while high-elevation populations remain highly susceptible. The genetic mechanisms underlying these immune strategies are largely a mystery. Therefore, we propose to experimentally infect a previously unexamined low-elevation population of Hawai'i `Amakihi to determine how genomic and transcriptomic profiles differ between survivors and fatalities of malaria infection, how these profiles differ from those of a high-elevation population we previously tested, and what genes are associated with immunological responses. RNAseq results from previous high-elevation `amakihi experiments showed that transcriptomic profiles of birds that succumbed to malaria infection differed from survivor and control birds. Differential gene expression analysis identified 421 genes that constitute these transcriptome-wide differences, the gene families of which we are currently investigating to assess the biological processes underlying recovery from infection. Elucidating the genes associated with immunological adaptations to avian malaria may be important for developing gene-based conservation strategies for our remaining 17 species of Hawaiian honeycreepers.}

\normaltalk{Infection with Mycoplasma Gallisepticum Augments Sociality in Juvenile House Finches}{Marissa M Langager, Dana M Hawley}{Acute pathogen infections are energetically demanding events that can induce substantial behavioral changes within hosts and thus influence pathogen spread. To date, few studies of vertebrates have examined how infection with a directly-transmitted pathogen influences host sociality. Juvenile house finches (Haemorhous mexicanus) are highly gregarious after reaching nutritional independence, forming large feeding flocks. In part due to this sociality, juveniles are particularly susceptible to infection by the bacterial pathogen Mycoplasma gallisepticum (MG). Any changes in sociality during active infection are likely to have consequences for the transmission of disease throughout these majority-juvenile flocks. However, it remains unknown how infection early in the juvenile stage ($<$3 months post-hatch) influences behavior and sociality in house finches. To test this, we inoculated 33 wild-caught juvenile house finches with MG or media alone (sham control) within 3 months post-hatch. At the peak of infection, all birds were run through a partner choice assay, where they were given the choice to feed and perch adjacent to a small flock or alone. We found that birds infected with MG were significantly more likely to spend time with the flock than healthy controls, indicating a higher degree of sociality. However, there was no relationship between the severity of disease and sociality, with more severely diseased birds displaying the same amount of social preference as those with less severe clinical signs. Our results have important implications for the spread of MG among hatch-year birds, which make up the largest proportion of infected and susceptible hosts in natural populations.}

\normaltalk{Wild Birds as Reservoirs of Mycoplasma Gallisepticum; the Influence of Diet on the Prevalence of MG Infections}{Vivica A Pressley, Megan M Free, Sawda Islam, Linda B Purvis, Dawn E Drumtra}{In wild birds, Mycoplasma gallisepticum (MG) was originally detected in birds visiting feeders in backyards in Washington D.C.. MG is a bacterial pathogen that is a cause of the conjunctivitis epidemic in birds (Fischer et al., 1997; Luttrell et al., 2001). Millions of dollars are lost in the poultry industry every year due to MG infections because these infections can quickly spread through a flock and lower the number of eggs produced by hens. The poultry industry's financial losses due to MG infections emphasize the need for a better understanding of how the disease is spread across species of birds (specifically from wild birds to domesticated birds). The University of North Georgia has access to a wide variety of wild birds from North Georgia due to a partnership with Atlanta Audubon Project Safe Flight. The trachea, duodenum, and lungs were harvested from the deceased birds and tested for MG using PCR and gel electrophoresis. Once this data is collected, it will be examined to determine the relationship between diet and the prevalence of MG infections. This research is still underway.}

\normaltalk{Parasite-Host Interactions and Biogeography of Avian Malaria in Five Host Species of Northern Central American Cloud Forest Birds}{Gregory T Gladkov, Rosa A Jim\'{e}nez, Rauri C Bowie}{At the heart of Nuclear Central America, the interface of two major geologic domains, the Chiortis and Maya Blocks, gave rise to particularly rugged topography. Consequently, climatological gradients fostered the accumulation of rich biodiversity in the region. While a number of studies have examined the prevalence, ecology, and biogeography of avian haemosporidian lineages around the world, our current knowledge of these parasites in northern Central America is limited. With avian tissue samples collected at fifteen localities throughout Guatemala and Chiapas, Mexico, we strove to document parasite prevalence and diversity in the region's cloud forests and to elucidate parasite-host interactions and biogeography. Using a standard barcoding approach, we screened DNA extractions from 330 individuals of five common cloud forest bird species (Henicorhina leucophrys, Chlorospingus flavopectus, Arremon brunneinucha, Basileuterus belli, Myioborus miniatus) for haemosporidian parasites in three genera (Plasmodium, Haemoproteus, Leucocytozoon). We detected 35 haemosporidian lineages, 21 of which were novel (i.e. not present in Malavi). Results indicated stark differences in parasite lineage diversity across host species with clear parasite-host interactions. In Basileuterus and Myioborus (Parulidae), a single Haemoproteus lineage constituted a majority of infections. Chlorospingus exhibited higher parasite lineage diversity than the other hosts, bolstered by its diverse Haemoproteus lineages. While all Haemoproteus lineages detected were novel, the converse was nearly true for Plasmodium lineages, supporting evidence of greater host specificity in Haemoproteus. In accordance with previous studies, we observed clear elevational trends in parasite prevalence, with Leucocytozoon more common at higher elevations and Plasmodium more prevalent at lower elevations.}

\normaltalk{An Experimental Test on the Susceptibility of Yellow-Rumped Warblers to a Widespread Avian Malaria Parasite}{Leticia Soares, Dr Elizabeth MacDougall-Shackleton, Dr Christopher G Guglielmo}{We experimentally exposed yellow-rumped warblers (Setophaga coronata) to a commonly occurring avian malaria lineage (Plasmodium cathemerium OZ14) to determine the timing and cost of infection development. We captured birds during fall migration in 2018, and held them overwinter in vector-free rooms. In May 2019, we used microscopy and PCR to characterize the parasitemia and genotype natural infections from blood samples. We multiplied Plasmodium infections in eight amplifier birds and used their blood to inoculate 38 recipients. We prepared a thin blood smear from each recipient bird 7, 13, 19, 25, and 34 days post-infection (dpi), and used Quantitative Magnetic Resonance to quantify fat mass. Average parasitemia peaked on 7 and 13 dpi (0.06±0.1\%), and steadily declined until 34 dpi (0.007±0.02\%). Parasitemia developed in 84\% of exposed birds, but only 24\% of individuals still presented active P. cathemerium infections on 34 dpi. Prior to Plasmodium exposure, 26\% of birds presented active parasitemia of Parahaemoproteus PASILI01, whereas 58\% of birds had active Parahaemoproteus parasitemia at 7 and 13 dpi. Parahaemoproteus parasitemia was positively related to P. cathemerium parasitemia on days 7, 13 and 19, but not on days 25 and 34 post-inoculation. By the end of the experiment, exposed birds had lost an average of 14\% of fat mass. Proportion of fat mass was positively related to Parahaemoproteus parasitemia, but was unrelated to Plasmodium parasitemia. Our results provide evidence that Plasmodium exposure increases the intensity of pre-existing Parahaemoproteus infections. We suggest that controlling the intensity of Parahaemoproteus co-infections is energetically costly for these birds.}

\normaltalk{Assessing the Impact of Haemosporidian Parasites in Raptors Health.}{Maria A Pacheco, Erica Miller, Ananias A Escalante}{As top predators and scavengers, raptors are of great ecological importance; however, many species are vulnerable. Thus, documenting their parasites and their effects on the raptors' fitness is a matter of importance. These birds are infected by a poorly defined pool of haemoparasite species/lineages. Importantly, stressors (e.g., lead poisoning, exposure to chemicals, or other infections) may trigger clinical disease in birds that otherwise can manage their haemosporidian infections. Leveraging on 96 samples collected by licensed rehabilitators, molecular diagnostics of haemosporidian parasites was performed. These samples belonged to 7 species of Accipitriformes, 2 of Falconiformes, and 4 Strigiformes orders from the Northeast of the USA. Percentage of infected individuals per species varied between 23 to 100\%. The targeted parasite mitochondrial genome allowed for the identification of twelve species/lineages with Haemoproteus spp. infections being the most common (60.5\%), followed by Leucocytozoon spp. (34.2\%), and Plasmodium spp. (5.3\%). Six of the twelve parasite lineages could be identified to species level. Among those, Haemoproteus syrnii, pathogenic to owls in Europe, was found infecting all owl species but Snowy Owls. Other parasite species include Haemoproteus catharti (infecting Turkey Vultures), Haemoproteus elani (infecting Red-tailed Hawks), Leucocytozoon buteonis (infecting Bald Eagles), Leucocytozoon toddi (infecting Red-tailed Hawks), and Plasmodium elongatum (infecting Cooper's Hawks). Six potentially new species of Leucocytozoon, Haemoproteus, and Plasmodium were discovered in this investigation. Overall, host switches appear to be common, and the diversity of haemosporidian parasites observed in the Northeast contrast with previous reports from the west of the USA.}

\normaltalk{Haemosporidian Bird Parasites in Dry Deciduous Forests Associated to Urban Areas on Guayaquil, Ecuador: a First Approach.}{Peter A Pibaque, Paolo Piedrahita}{Avian haemosporidian parasites (Order: Haemosporida) are a diverse group of microorganisms which develop part of its life cycle in the blood of avian hosts. Ecuador has been proposed as a reservoir for haemosporidian parasites since it harbors an ample amount of bird species. Research on this topic has been performed in all Ecuadorian regions but some ecosystems such as the deciduous dry forest along the Coastal region have not been evaluated to determine the presence of these parasites. In order to contribute with additional information to the currently available, we have analyzed avian blood samples in two different forests in Guayaquil, Ecuador: Bosque Protector La Prosperina and Area de Recreaci\'{o}n Parque Lago. Sampling was performed from the start of 2018s wet season (December) and finished at the start of 2019s dry season (June). Blood samples were taken from 37 individuals of 20 bird species captured by mist-netting. All samples were analyzed through optical microscope observations. Possible blood parasites from the genus Haemoproteus were found on a blood sample obtained from Arremon abeillei (Passerelleidae), a bird species which has no previous reports of blood parasite presence. These findings confirm previous research efforts that have hypothesized that Ecuador could be a reservoir of avian blood parasites. We suggest that future efforts in the Coastal region should be focused on the use of molecular biology techniques to aid in the complete identification of the possible blood parasites present in Ecuadorian birds.}

\normaltalk{Predictors of Brown-Headed Cowbird (Molothrus Ater) Brood Parasitism Incidence and Intensity in Prairie Reconstructions}{Kyla L Yuza-Pate, Matthew D Stephenson, Lisa A Schulte Moore, Robert Klaver}{The Brown-headed cowbird (Molothrus ater) is a songbird native to the North American Great Plains. An obligate nest parasite, females must lay their eggs in the nests of other bird species to reproduce. Prior research has documented a positive relationship between nest parasitism incidence, proximity of suitable perches for cowbirds, and nest visibility; however, the extent of cowbird parasitism has yet to be investigated in reconstructed prairies interspersed in an agricultural landscape. Due to the highly fragmented nature of perennial conservation practices on farms, spatial characteristics of the landscape may also predict cowbird parasitism incidence and intensity. From 2015-2019, we systematically searched for grassland bird nests on 9 farms and 2 prairies in central Iowa with varying vegetative diversity and reconstructed prairie. Nests were monitored for incidence and intensity of cowbird parasitism. Host nest visibility characteristics such as nest height and proximal vegetation density were measured in the field. Spatial characteristics relating to perch and habitat edge proximity were measured in ArcGIS. Mixed-effect models were evaluated using Akaike Information Criterion (AIC) for species known to accept cowbird eggs and that had more than 30 nests monitored during the study (Red-winged blackbird, n = 847; Dickcissel, n = 332; Vesper sparrow, n = 72; Common yellowthroat, n = 49). Host nest distance to habitat edge and perch distances had the most support over the species we evaluated. Our results may provide guidance for design and management of conservation practices, especially for species that are Species of Greatest Conservation Need in Iowa.}

\normaltalk{Master Illusionists or Amateur Artists: Do the Patterned Parasitic Eggs of the Generalist Brown-Headed Cowbird Resemble Host Eggs?}{Samantha L Rutledge, David Carr, Mark Hauber, Daniel Hanley}{Brown-headed cowbirds (Molothrus ater) are generalist brood parasites, successfully parasitizing well over 100 host species. Yet most cowbird hosts do not respond to the cost of parasitism by rejecting the foreign egg or abandoning the parasitized clutch. We investigated whether cowbird eggs, which are usually considered nonmimetic, utilize lighting conditions surrounding the host nesting site to camouflage their eggs inside the host nest. After comparing cowbird eggs to host and non-host species in nesting and alternative lighting conditions, cowbird eggs showed a greater resemblance to host species' eggs than non-hosts. Although the absolute degrees of eggshell similarity were relatively low, our findings suggest that, contrary to previous thought, cowbird eggs may have evolved to be somewhat mimetic. Also, cowbirds have likely adapted to exploit hosts that place their nests in light environments that make discriminating cowbird eggs from their own even more difficult. This slight -- rather than perfect -- mimicry of multiple host species may be advantageous for the cowbird as a generalist parasite, as it enables generalists to adapt to numerous host species and lighting environments.}

\normaltalk{Does Host Egg Removal by Brown-Headed Cowbirds Decrease Parasitic Egg Ejection Rate in Eastern Bluebirds?}{Zane Libke, Max Rollfinke, Elayna Daniels, Mark Stanback}{Parasitic Brown-headed Cowbirds (Molothrus ater) often remove a host egg when they lay an egg in a host nest. However, the fitness benefit of this behavior is disputed. We hypothesized that hosts recognize and eject parasitic eggs because their addition abruptly increases egg number, and that host egg removal evolved as a way for brood parasites to add a parasitic egg without increasing egg number, decreasing the likelihood that hosts will detect and eject a parasitic egg. Eastern Bluebirds (Sialis sialis) are neither uniform acceptors nor ejectors, so they are ideal candidates for testing this hypothesis. We manipulated bluebird clutches near Davidson, North Carolina to determine if host egg removal affects parasitic egg ejection rate. In our experimental nests (n=218), we added a model bluebird egg during egg laying and replaced it with a model cowbird egg a week later. In our control nests (n=174), we only added a model cowbird egg after clutch completion. We found no difference in ejection rate between the two treatments. These findings suggest that host egg removal by Brown-headed Cowbirds does not alter ejection rates in hosts, indicating that host egg removal must have a different fitness benefit.}

\normaltalk{Gut Parasites and Host-Parasite Evolution in Leach's Storm-Petrels}{Donald C Dearborn, Madison Liistro, Ross A Ackerman}{Host-parasite interaction in vertebrates is centered on genetic variation in the Major Histocompatibility Complex (MHC). To understand patterns of MHC variation within and between host populations, we need to determine whether parasites are locally acquired (enabling local adaptation of MHC) or are globally acquired in a way that shares parasites across host populations. To address this question in Leach's Storm-petrels, a species with well-characterized Class II MHC, we have begun screening parasites by sequencing DNA in blood, vomit, and fecal samples.  As a first step, we used high throughput sequencing with universal 18S eukaryotic primers plus a vertebrate blocking oligo.  In vomit samples pooled across individuals, we found low-abundance sequences of annelids, nematodes, platyhelminths, and apicomplexans.  A similar approach with blood samples yielded no detectable eukaryotic parasites. In contrast, sequences from feces show an abundance of coccidians (Apicomplexa: Coccidia), plus a much lower abundance of tapeworm sequences).  Subsequent work with additional primers has revealed that the coccidians are of two types -- one very similar to Sarcocystis, and another somewhat similar to Eimeria, Isospora, and Lankesterella.  Fitness effects in storm-petrels are not known, but data from Sarcocystis in sea otters and Eimeria in domestic poultry show the potential for marked pathogenicity under some circumstances.  Ongoing work is measuring prevalence and will explore likely sources and transmission routes.}

\normaltalk{Phonology and Syntax of Red-Crowned Parrot Duets in South Texas}{Abigail Pozulp, Karl S Berg}{Bird behavioral interaction systems show a preference for temporal precision. This preference is exemplified in the way many species are known to avoid masking of vocal signals. Contemporary studies of antiphonal duetting in songbirds suggest that overlapping notes are a sign of poor temporal coordination which can signify a weak pairbond, a possible cue for conspecifics seeking to usurp territory or mates. In Brownsville, TX, the antiphonal duets produced by a wild population of red-crowned parrots have a variable degree of temporal coordination with some duet elements often overlapping. Playback experiments will investigate whether element overlap is a cue used by conspecifics (both intrapair and extrapair) to determine duet coordination. Spectrographic analysis of audio and video recorded duets will measure the variation in timing of overlapping signals within and across a sample of mate pairs. Preliminary results suggest element overlap is more prevalent in red-crowned parrot duets. This may be an indication that partially overlapping notes in parrot duets is not undesirable like it is in antiphonally duetting songbirds.}

\normaltalk{Smarts and Symbiosis: Evidence for the Gut-Brain Axis in a Model Songbird}{Morgan C Slevin, David Bradshaw, Jennifer Houtz, Rindy C Anderson}{Recent years have seen a surge of research on the link between an individual's cognitive ability and its gut microbiome. With recent advances in understanding avian cognition, songbirds are an ideal system for investigating this gut-brain relationship. In a captive Zebra Finch (Taeniopygia guttata) population of 42 adults, we quantified individual variation in performance on cognitive tasks (novel foraging, color association, and color reversal) that measure motor learning and memory. We sampled the gut microbiome via cloacal swab immediately prior to testing, sequenced the bacterial taxa present, and assessed diversity and relative abundance in each sample using Qiime2 and R. Performance on the novel foraging task was significantly related to alpha diversity. Furthermore, beta diversity of individuals with poor performance on this task was significantly different from better performing individuals. This apparent microbiome dysbiosis in poor performers manifested as a relative lack of the putatively beneficial Lactobacillus genus and increased abundance of several presumed pathogenic taxa. However, this gut-brain relationship did not hold true for females' novel foraging performance or for both sexes' performance on the other two cognitive tasks that involved associative memory, suggesting there is still much more to be learned about the avian gut-brain relationship. This study provides some of the first evidence of an avian gut-brain axis, setting the stage for future research including understanding the gut microbiome in wild populations and manipulating the microbiome during critical developmental stages to understand potential downstream effects.}

\normaltalk{Weather, singing behaviour and song performance in a Puerto Rican endemic, Adelaide's warbler (Setophaga adelaidae)}{Juleyska Vazquez-Cardona, Hester Jiskoot, Peter C. Mower, Samantha W. Krause, David M Logue}{Birdsong responds to a host of external factors, including the weather. For example, some species start their dawn chorus earlier when the air temperature is higher, and later with the presence of precipitation. While it is clear that weather influences song, the extent of this influence remains unclear. In the present study, we describe the effects of multiple weather parameters on a variety of song delivery and song structure variables in a tropical, year-round territorial warbler. In addition to describing the relationships between weather and song, we will test predictions of the hypotheses that intense early morning singing causes birds' body temperature to increase, leading to higher song performance, and that this relationship is mediated by weather parameters that affect their thermoregulation. Our dataset comprises four years of field observations on Adelaide's warblers (n= 37,976 songs, 70 males) and concurrent weather data. This study will improve our understanding of the effect of weather conditions on song structure and singing behaviour and may provide insights into the impacts of climate change on acoustic communication in birds.}

\normaltalk{Acoustic Adaptation and Cultural Evolution in the Long-Billed Hermit (Phaethornis Longirostris)}{Hans Gonzembach, Marcelo Araya-Salas, Melanie Talavera, Paulina Gonzalez-Gomez}{Sensory drive is a common phenomenon that has shaped the evolution of many animal communication signals. However, there is little understanding of the role that acoustic adaptation plays in cultural evolution and learned acoustic signals. The Acoustic adaptation hypothesis proposes that songs with lower frequencies, narrower frequency ranges, and longer inter-element intervals should occur more frequently in densely vegetated areas compared with herbaceous habitats. Long-billed hermits are an ideal model organism for testing the AAH because of their lekking behavior and because they reside in a heavily vegetated habitat. To determine if habitat plays a role in vocal degradation in the Long-billed hermit, we conducted playback experiments at three different distances from the source of the speaker and at two different sites that had two different populations of lekking birds. We also evaluated the link between degradation, vegetation density and acoustic structure of these song types. We found songs that had a short lifespan degraded more but had a higher signal-to-noise ratio when compared with songs with a longer lifespan. We also found song types with a short lifespan had a higher sound pressure level when compared with songs types with a long lifespan. Our results partially refuted our hypothesis with regards to sound pressure level and signal-to-noise ratio. Overall, our results gave us a better understanding that transmission properties are affected by habitat structure and that these properties are linked with cultural evolution that affect the lifespan of song types.}

\normaltalk{Choice Responses to Variation in a Female-Specific Call that may be Reinforcing a Pattern of Microgeographic Divergence in an Island-Endemic.}{Christopher A Tarango, Cameron K Ghalambor, T S Sillett}{What mechanisms underlie ecological speciation? The homogenizing effects of gene flow have long been hypothesized to overcome local adaptation and prevent speciation in sympatry. However, recent evidence suggests that for so-called `magic traits', adaptive divergences, driven by ecological selection, are reinforced by associated divergences in mating signals, resulting in a clustering of genotypes in continuous landscapes. Yet there remains a lack of direct evidence of sexual or social selection acting on these divergent signals. Here, we use choice experiments of a proposed magic trait, the female-only rattle call of the Island Scrub-Jay (Aphelocoma insularis), an island-endemic species that exhibits microgeographic adaptations in bill morphology across two habitat types. We conducted simultaneous playbacks of two rattle calls, one from each habitat, to free living pairs and used remote acoustic sensors, in addition to human observers, to quantify the responses of territorial pairs.}

\normaltalk{Exploring Perception of Song Adjusted to Anthropogenic Noise}{Amber MB Ng, Sharon A Gill}{Many avian species produce altered song in anthropogenic noise, presumably to provide release from masking by high amplitude, low frequency noise. However, the costs and benefits of such alterations are infrequently directly studied. One potential cost is loss of information, wherein the vocal feature that is adjusted is a key component of the message being broadcast, resulting in a message that no longer carries accurate information. In birdsong, loss of information would occur if songs are perceived as unattractive or more threatening once altered. House wrens (Troglodytes aedon) are one species for which song transmission is significantly reduced in noise. House wrens increase the peak frequency of their songs in noise, but also in response to territory intruders. Because song frequency is implicated in both territory defense and noise adjustment, we hypothesized a possible loss of information in house wren male-to-male agonistic interaction. We simulated territory intruders using playbacks to determine whether male house wrens perceive noise adjusted song differently than unadjusted song. We quantified behavioral responses, including perch changes, distance from speaker, song, and flights past the speaker, to playback of synthesized songs of normal frequency, and otherwise identical songs increased by 500 Hz. Preliminary analysis indicates that playback type is not a significant predictor of male response. This suggests a loss of information cost does not exist in male-to-male interactions for a house wren intruder producing moderately frequency shifted songs.}

\normaltalk{Amplitude Variation as Honest Signaling in the Long-Billed Hermits}{Melanie L Talavera, Paulina L Gonz\'{a}lez-Gomez, Hans Gonzembach, Marcelo Araya-Salas}{Honest signals, which can provide information on sender's fitness and condition, are  enabled by constrained physiological traits and energy availability that only few can bear. Thus,  they can act as cues for mate choice. Amplitude is an acoustic parameter that has been  considered as a potential sexually selected signal. However, measuring amplitude in natural  settings can be challenging as it ca be affected by several factors that include recording distance  and orientation, environmental conditions, and equipment settings. Long-billed hermits  (Phaethornis longirositris) are a lekking species with behavioral characteristics that allow more  precise measurements of amplitude. They are faithful not only to territories, but also to  identifiable perches that are ideal for recording songs. This study focused on identifying if  amplitude variation between individuals exists and determining if it has a relationship with  morphological traits. Also, playback experiments were used to test whether amplitude is a plastic  trait that can be modulated in aggressive encounters and higher environmental noise. Our results  indicate that amplitude does not convey information of the body size, but it had a positive  correlation with lifting power, suggesting that producing higher amplitudes requires usage of the  available energy. Thus, amplitude may be an honest signal of the current condition of the sender.  On the other hand, environmental noise and aggressive encounters modulate higher amplitude  acoustic signals suggesting that amplitude is a plastic trait that changes with social and  environmental factors.}

\normaltalk{Correlation Between Exploratory Behavior and Oxidative Profile in Urban and Desert House Finches}{Sarah E Polekoff, Pierre Deviche}{Urban animals face unique environmental conditions. Past avian studies reported behavioral and physiological effects of urban living, but results of these studies vary across species and cities. For example, urban birds are often more exploratory than rural birds, but some studies found the opposite or no habitat-related differences in exploratory behavior. Differences between urban and rural birds may reflect differences in physiology but despite recent growth in the field of urban ecology, the relationships between physiology and behavior across urban and non-urban contexts remain poorly understood and under-studied. To address this question, we are investigating oxidative profile, exploratory behavior, and their possible correlation in urban- and desert-dwelling House Finches. We predict that oxidative profile measures will correlate with exploratory behavior within individuals and across environments. Compared to their desert counterparts, we expect urban finches to have higher plasma reactive oxygen metabolites (indicative of oxidative damage) and higher plasma antioxidant capacity, and to show more exploratory behavior. To test these predictions, we are sampling free-living finches at an urban site (Arizona State University, Tempe campus) and at two nearby Sonoran Desert sites (Estrella Mountain and McDowell Mountain Regional Parks). Finches are captured using seed-baited traps, bled, measured, and marked with leg bands for identification purposes before release. To test exploratory behavior, we use a mobile, collapsible wooden chamber with pegs inside. By identifying relationships between oxidative profiles and exploratory behavior, our studies may provide novel insights into the physiological mechanisms that underlie behavior and the adjustments of birds to urban environments.}

\normaltalk{Artificial Light at Nigh Drives Early Dawn Chorus Onset Times of the Saffron Finch (Sicalis Flaveola) in an Andean City}{Oscar H Mar\'{i}n G\'{o}mez}{Urban birds around the world have to cope with dominant city stressors as anthropogenic noise and artificial light at night (ALAN) by adjusting the temporal and spectral traits of their acoustic signals. It is widely known that higher anthropogenic noise and ALAN can disrupt the morning singing routines, but its influence in tropical urban birds remains unexplored. Here, I assessed the association between ALAN and noise with the dawn chorus onset of the Saffron Finch (Sicalis flaveola) in an Andean city of Colombia. I studied 32 urban sites which comprise different conditions of urban development. I annotated the time when the first individual of the Saffron Finch was heard at each site and then I obtained anthropogenic noise and ALAN measurements using a smartphone. Findings of this study show that Saffron Finches living in highly developed sites sang earlier at dawn than those occupying less urbanized sites. Unexpectedly this timing difference was related to ALAN instead of noise, suggesting that ALAN could drive earlier dawn chorus in a tropical urban bird. Saffron Finches could take advantage of earlier singing for territorial signaling, as expected by the social dynamic hypothesis. However, findings of this study should be interpreted carefully because the dawn chorus is a complex phenomenon influenced by many multiple factors. Future studies need to assess the influence of ALAN on the dawn chorus timing of Neotropical urban birds by taking into account the influence of confounding factors related to urbanization as well as meteorological, ecological, and social drivers.La luz artificial nocturna conduce a tiempos de inicio m\'{a}s temprano del coro del alba en el canario común (Sicalis flaveola) en una ciudad Andina. Las aves urbanas alrededor del mundo ajustan los rasgos temporales y espectrales de sus se\~{n}ales acústicas para afrontar los factores estresantes que predominan en la ciudad como el ruido antropog\'{e}nico y la luz artificial nocturna. Se conoce ampliamente que los mayores niveles de ruido antropog\'{e}nico y luz artificial pueden interrumpir las rutinas matutinas de canto, aunque la influencia de estos factores en las aves urbanas tropicales sigue siendo poco explorada. En este estudio evalu\'{e} la asociaci\'{o}n entre la contaminaci\'{o}n lum\'{i}nica y acústica con el inicio del coro del alba del canario coronado (Sicalis flaveola) en una ciudad andina de Colombia. Estudi\'{e} 32 sitios urbanos distribuidos en el norte de la ciudad que abarcaron diferentes condiciones de desarrollo urbano según la cobertura construida. En cada sitio anot\'{e} el momento en que se escuch\'{o} el canto del primer individuo del canario coronado y luego obtuve mediciones del ruido antropog\'{e}nico y la luz artificial nocturna por medio de un tel\'{e}fono inteligente. Los resultados de este estudio muestran que los canarios coronados que habitan sitios altamente urbanizados cantaron m\'{a}s temprano al alba que los que ocuparon sitios menos urbanizados. Inesperadamente, esta diferencia de tiempo se relacion\'{o} con la iluminaci\'{o}n artificial en vez del ruido antropog\'{e}nico, lo que sugiere que la luz artificial podr\'{i}a conducir a inicios de coro del alba m\'{a}s tempranos en un ave urbana tropical. Los canario coronados podr\'{i}an aprovechar el canto m\'{a}s temprano para se\~{n}alar la propiedad territorial entre vecinos, según lo esperado por la hip\'{o}tesis de la din\'{a}mica social. Sin embargo, los hallazgos de este estudio deben interpretarse cuidadosamente porque el coro del amanecer es un fen\'{o}meno complejo influenciado por múltiples factores . Los estudios futuros deben evaluar la influencia de ALAN en el momento del coro del amanecer de las aves urbanas neotropicales teniendo en cuenta la influencia de los factores de confusi\'{o}n relacionados con la urbanizaci\'{o}n, as\'{i} como los factores meteorol\'{o}gicos, ecol\'{o}gicos y sociales.}

\normaltalk{Nest cavity reuse by the Acorn Woodpecker: fitness benefits or constraints?  Reutilizaci\'{o}n de cavidades del nido por Melanerpes formicivorus: ¿beneficios de fitness o limitaciones? }{NA}{NA}

\normaltalk{Using Automated Radio-Telemetry to Assess Spring Migratory Ecology and Survival of Ipswich Sparrows (Passerculus Sandwichensis Princeps)}{Sydney Bliss, Phil Taylor, Marty Leonard}{Ecological barriers influence the behaviour of migratory birds. Population-level variation in behaviour at barriers can occur due to differences in experience and maturity across age and sex classes. Understanding how behaviour varies is important because it scales up from the individual level to affect survival and population size. We assessed spring migratory behaviour and survival of the Ipswich Sparrow (Passerculus sandwichensis princeps) which must pass two barriers, the Gulf of Maine (~400 km) and part of the Atlantic Ocean (~350 km), to reach its breeding grounds on Sable Island, Canada. To do this, we tracked 153 sparrows using automated radio-telemetry and assessed differences in behaviour by age (adult or juvenile) and sex. We also estimated daily survival probability at barriers as well as other legs of the migratory pathway using multi-state capture-recapture models. Sparrows both crossed and circumvented the Gulf of Maine, and, the bearing of juveniles was less direct than adults at the Atlantic Ocean. Sparrows also made abandoned flights at the Atlantic Ocean but not elsewhere along the migratory pathway. We did not detect sex-related differences in behaviour. Daily survival probability at the Gulf of Maine (0.97) was among the highest across migratory legs, and, was lowest at the Atlantic Ocean (0.91). Collectively, results suggest that the smaller Atlantic Ocean barrier may be more difficult for sparrows to pass than the larger Gulf of Maine, and, age but not sex affect the migratory behaviour of this songbird at barriers in the spring.}

\normaltalk{Sensory Process Sensitivity and Cognitive Test Performance in American Crows}{Rebecca Pearce, Anne B Clark}{Research on the evolution of animal cognition is increasingly focused on individual variation as a  complement to comparative studies. While linking fitness and individual cognitive abilities has proved challenging, many studies report a high degree of individual variation on cognitive task performance. Individual test outcomes, often used as a proxy for cognitive ability, could be influenced by other factors. One such factor is ``sensory process sensitivity", or the extent to which an individual's behavior is affected by changes in background environmental stimuli. Individuals more cued to their surroundings may be more likely to notice opportunities and better prepared to face threats. They may also incur potential costs of increased cognitive load and metabolic demand. Wild American crow (Corvus brachyrhynchos) families in a marked population were tested on territory for individual differences in sensitivity as compared to their performance on a simple cognitive task, predicting that sensitivity variations would influence how individuals handled the task. Crows were first familiarized with a short log with two holes containing food, and their approach latency, time near log and number of cautious behaviors were recorded as baselines. This setup was then presented four times on successive days, each time with one of four cords of differing thickness and color wrapped around the log. Finally the log was presented with holes/food covered by lightweight lids as a cognitive task. Baseline and sensitivity trial reactions of individual crows were compared to their approach to and success in retrieving food from under the lids.}

\normaltalk{The Effectiveness of Predator Guards on Reproductive Success of Cavity-Nesting Songbirds}{Cody M Kelley, Todd J Underwood, William P Brown}{Artificial nestboxes are an effective way of providing nesting sites for cavity nesting birds. Because nest predation is a major influence on reproductive success, predator guards have been added to nestboxes to reduce effects of predation. Previous research found predator guards to be effective at improving reproductive success of birds in artificial nestboxes. However, this observational research lacked proper controls. Thus, a well-designed experiment is needed to confirm these trends. In this study, we determined the effectiveness of two common predator guards (Noel and stovepipe) compared to controls without guards on Peterson-style boxes made of wood or metal using a randomized treatment design at two sites in Pennsylvania. In 2019, we monitored 72 nestboxes every 1-3 days to determine nest success with different treatments. Additionally, we placed trail cameras around nestboxes to document predator activity in response to the different treatments. Although predation varied significantly between sites, there was no significant difference in predation at wood compared to metal boxes. We found no significant difference in nest success between treatments at either site. Trail cameras captured red squirrels (Tamiasciurus hudsonicus) gaining access to nestboxes with and without guards. Thus, predator guards may not be effective in preventing access to all predators. Our results indicated that predator guards were not effective at improving nest success in artificial nestboxes. However, there was a slight trend toward higher success with stovepipe guards at the site with highest predation. Additional years of monitoring and a larger sample size may yield clearer trends.}

\normaltalk{Foraging Behavior of the Endemic Hummingbird Doricha Eliza in a Coastal Area of Yucat\'{a}n,M\'{e}xico.}{Rebeca Sanmiguel, Pamela Medina van Berkum, Jos\'{e} A Ramos-Zapata}{The Mexican Sheartail (Doricha eliza), is an endangered species of hummingbird endemic to Mexico. One of its populations is distributed in the north coast of the Yucat\'{a}n peninsula, where their habitat has been degraded by humans, reducing the floral resources. The aim of this work was to analyze the relationship between the floral resources and a D. eliza population distributed in a coastal dune vegetation in Chuburn\'{a} Puerto, Yucat\'{a}n, M\'{e}xico. We visited the area during September-December of 2019, and estimated: 1) the plant species visited and abundance of hummingbirds (point counts); 2) foraging behavior (focal observation) and 3) nectar quality and production of the visited species. The plant species visited by D. eliza includes Malvaviscus arboreus, Ernodea littoralis, Passiflora foetida and Suriana mar\'{i}tima. The foraging behavior of D. eliza was adjusted to the seasonality of floral resources were quantity and quality of nectar was more important than the number of flowers. A positive relationship between nectar production and number of visit were observed. We observed a density of D. eliza of 1.21/ind/ha; we did not observe inter or intra-species competition even when another hummingbird species (Amazilia rutila) was present in the study area. The low density of floral resources in this area may suggest a coupling in the foraging strategy. Even though the study area is very small, we observed a high D. eliza density, so we can conclude that this area must be conserved.}

\normaltalk{Validating the Use of a Handheld Meter for Measuring a Plasma Metabolite, Triglyceride, in the Field}{Christopher M Roelandt, Jill Witt, Amber Roth}{Plasma metabolite concentration analyses have been shown to be effective in predicting change in body mass and are useful as a measurement of body condition in birds. One such plasma metabolite, triglyceride (TRIG), has been shown particularly capable of predicting when birds are either in a state of fattening or fasting. These predictions can be a useful tool for wildlife and land managers interested in assessing the quality of habitat as measured by its ability to provide feeding resources to individual birds using that habitat. Plasma sample collection in a field setting can be challenging and methods aimed at reducing these challenges may help to expand their use in the field. I explored the use of a small handheld meter, CardioChek PA analyzer, for measuring TRIG concentrations as a means to reduce challenges that are associated with the use of plasma metabolites. Following the CardioChek PA manufacturer's guidelines, I tested results of TRIG against the results of the same sample completed in a laboratory analysis to determine the validity of using a handheld meter in the field. I found that the handheld meter was precise in its results but not able to accurately measure TRIG in the field, possibly due to the interactions with the extreme environmental conditions in the field. I recommend further research into the limitations of this handheld meter and developing methods to reduce the effects of light, temperature, and humidity in the field.}

\normaltalk{Natural Variation of Insulin-Like Growth Factor-1 and Telomere Dynamics Across Multiple Years in House Sparrow Nestlings.}{Aubrey E Sirman, Ád\'{a}m Z Lendvai, Mark E Clark, Jennifer Vangorder-Braid, Anuj Ghimire, Britt J Heidinger}{Post-natal growth has profound impacts on life-history traits later in life. Individuals may increase growth to maximize survival during early development, however, faster growth may incur long-term costs. Key to understanding these costs is examining the physiological mechanisms linked to variation in growth. Specifically, endocrine pathways enable organisms to integrate and respond to environmental cues.  and hormones such as insulin-like growth factor-1 (IGF-1) may play a critical role in this trade-off. IGF-1 is a potent stimulator of growth and has been linked with decreases in longevity and cellular aging across species. IGF-1 might influence cellular aging through its effects on telomeres. Telomeres are protective, non-coding caps found on the end of eukaryotic chromosomes that are thought to play an important role in cellular and organismal aging. To test this, we examined the relationships between free IGF-1 levels, growth rate, and telomere dynamics across three years in a free-living population of house sparrows (Passer domesticus) during the post-natal growth period (day 2 -- day 10 post-hatch). Free IGF-1 was sampled at days 6 and day 10 post-hatching and telomeres were measured at days 2 and day 10 post-hatching. We predicted growth rate would be positively associated with IGF-1 and negatively associated with telomeres.  Results will be discussed within the context of life-history theory and will provide insight into the relationship among growth rate, free IGF-1, and telomeres in a wild bird population.}

\normaltalk{Proximity to Mining Concessions and their Downstream Rivers in South-Central Mexico Affects Plumage Coloration in Streak-Backed Orioles}{Lynna M Kiere, Luis Antonio S\'{a}nchez-Gonz\'{a}lez, Marcela Osorio-Beristain}{Metal mining is economically important but detrimental to ecosystem health due partly to heavy metal (HM) pollution, which may be dispersed as runoff. Many mines are in high-priority avian conservation areas, and chronic HM exposure can have easily overlooked sub-lethal effects. Plumage coloration is important in avian social and ecological interactions and sensitive to environmental stressors, and several studies have demonstrated effects of local HM pollution on coloration. We investigated regional-scale effects of proximity to mines in the states of Morelos, Guerrero, and Oaxaca on plumage coloration in the Streak-backed Oriole. We measured patch color and area using reflectance spectrometry and photography in museum specimens and field-collected samples (n=116--168), then used geographic information systems to estimate each specimen's distance from the nearest mining concession and river with =1 upstream mine. We used mixed models to analyze the effects of mine distance, river distance, and threshold and interaction effects on coloration, accounting for potential confounding variables. There were several effects of mine proximity on carotenoid-based plumage; crown carotenoid chroma (CC) and rump structural reflectance (SR) improved with increasing mine distance, and mine and river distance had additive effects on belly SR. Breast SR and CC were affected by an interaction of mine and river distance, suggesting that downstream dispersal of mining waste may affect birds even when they are not near mines themselves. Carotenoid hue and melanin-based plumage were unaffected. Our results show regional-scale effects of proximity to mines and downstream rivers on coloration; further investigation is needed to understand their mechanisms and consequences.La miner\'{i}a de metales es econ\'{o}micamente importante, pero es da\~{n}ina para la salud ecosistemica debido en parte a la contaminaci\'{o}n por metales pesados (MP), la cu\'{a}l puede ser dispersada a trav\'{e}s de la escorrent\'{i}a. Muchas minas est\'{a}n ubicadas en \'{a}reas prioritarias para la conservaci\'{o}n de aves, y la exposici\'{o}n cr\'{o}nica a MP puede tener efectos sub-letales f\'{a}cilmente ignorados. La coloraci\'{o}n del plumaje es importante en las interacciones sociales y ecol\'{o}gicas de las aves y sensible a estresores ambientales, y varios estudios han demostrado efectos de contaminaci\'{o}n local de MP sobre la coloraci\'{o}n. Nosotros investigamos efectos a una escala regional de la cercan\'{i}a a minas en los estados de Morelos, Guerrero, y Oaxaca sobre la coloraci\'{o}n del plumaje en la calandria dorsirayada, un ave residente común en toda la regi\'{o}n. Medimos el color y el \'{a}rea de parches en ejemplares de museo y muestras colectadas en campo (n=116--168) usando espectrometr\'{i}a de reflectancia y fotograf\'{i}a. Luego usamos sistemas de informaci\'{o}n geogr\'{a}fica para estimar la distancia de cada ejemplar a la concesi\'{o}n minera y el afluente con al menos una mina r\'{i}o-arriba m\'{a}s cercanos. Utilizamos modelos mixtos para analizar los efectos de la distancia a mina, distancia a afluente y efectos de umbral y de interacci\'{o}n sobre la coloraci\'{o}n, tomando en cuenta potenciales variables de confusi\'{o}n. Encontramos efectos de la cercan\'{i}a a minas en varios aspectos del color basado en carotenoides; el croma de carotenoides (CC) de la corona y la reflectancia structural (RE) de la rabadilla aumentaron conforme aumentaba la distancia a la mina m\'{a}s cercana, y hubo efectos aditivos de los efectos de distancia a la mina y distancia al afluente m\'{a}s cercanos en la RE del vientre. El CC y la RE del pecho fueron afectados por la interacci\'{o}n entre distancia a la mina y al afluente m\'{a}s cercano, sugiriendo que la dispersi\'{o}n de desechos a trav\'{e}s de afluentes puede afectar a las aves a pesar de estar lejos de las propias minas. El tono y el plumaje basado en melaninas no fueron afectadas. Nuestros resultados muestran que hay efectos a escala regional de la cercan\'{i}a a minas y sus afluentes r\'{i}o-abajo sobre la coloraci\'{o}n; se requiere m\'{a}s investigaci\'{o}n para entender sus mecanismos y consecuencias.}

\normaltalk{Combining Morphological and Molecular Data to Reconstruct the Evolutionary History of Hawaiian Honeycreepers}{Luke C Campillo, Robert C Thomson, Rosana Zenil-Ferguson}{Adaptive radiations have long captured the attention of evolutionary biologists due to their rapid lineage diversification and acquisition of novel phenotypes across a short evolutionary time span. However, little is known about the long-term persistence of newly evolved phenotypes within these systems. Here, we examine phenotypic persistence in a well-studied natural system, the Hawaiian honeycreeper adaptive radiation. The morphological diversity harbored within extant Hawaiian honeycreeper lineages nearly covers the entirety of all songbird body types and bill shapes, but only about a third of all honeycreeper diversity is extant. Given the rampant extinction across the clade, we wanted to estimate what ecological or evolutionary factors contributed to species extinction. Moreover, we were interested in what phenotypes have persisted and which have gone extinct. However, this can only be accomplished if, and only if, those extinct species can be placed in the Hawaiian honeycreeper tree of life. In this talk, we will show the results from: 1) combining morphological and molecular data to place the extinct honeycreepers onto the phylogenetic tree in a fully Bayesian framework using RevBayes, and 2) using life history traits (e.g., feeding morphology, size of geographic range) to determine vulnerability to extinction and phenotypic persistence across the radiation. This is, to our knowledge, the first total-evidence tree for the Hawaiian honeycreepers, and represents a critical step forward in understanding their evolutionary history.}

\normaltalk{Decreased Nest Survival Associated with Low Temperatures in a High-Elevation Population of Mountain Plover (Charadrius Montanus)}{Allison K Pierce, Stephen J Dinsmore, Michael B Wunder}{Mountain Plovers (Charadrius montanus) are migratory shorebirds of conservation concern that breed on grasslands and xeric tablelands scattered along the western edge of the Great Plains. Previous research from populations breeding on lower-elevation sites in Colorado and Montana suggests nest survival increases in cooler, drier environmental conditions. However, no studies have estimated Mountain Plover nest survival on high-elevation habitats, which have distinctively different climate and nesting phenology; therefore, it is unclear if these general trends hold true for the species as a whole or are regionally specific. In this study, we monitored nests of a population of Mountain Plovers breeding at high elevation in South Park, Colorado, to study influences on daily nest survival as compared to those found in literature from lower-elevation breeding sites. In contrast to studies on lower-elevation sites where daily maximum temperatures limited nest survival, we found daily minimum temperature best predicted variation in daily nest survival in South Park; survival odds declined with decreasing temperature. These results suggest that negative effects of extreme temperatures (hot or cold) may be generalities that exist between locales but are potentially mediated through different mechanisms at varying temporal scales. Although we cannot definitively conclude from our study why minimum temperature is associated with lower nest survival, we suggest alterations in incubation behavior may be a plausible mechanism that warrants future study. Overall, our study demonstrates the importance of considering differences between weather and phenology among breeding locations when estimating nest survival effects.}

\normaltalk{Census of Acuatic Birds in the Venezuelan Insular Zone: Period 2006-2020}{Margarita Martinez}{The island arch of Venezuela represents 0.5\% of the total land area and consists of 311 islands between continental and oceanic. In the period between 2006 and 2020, the counts were made in the months of February and July in an island state and two federal dependencies. Isla de Aves (VE001) is an island located in the tropical cyclone zone, which also has restricted access, which is not the case for the island state of Nueva Esparta that is a continental island and has been continuously monitored since the beginning of the censuses. Four IBAS were censored, including two national parks, a natural monument, two RAMSAR sites and a wildlife refuge. Pelagic species were not included in these censuses. The results obtained for 22 events were: 1,446 records of 63,752 individuals located in 33 localities, of 16 families of acuatic birds represented in 69 species: 40 residents and 25 migratory, in addition to four unidentified. Among the resident families that presented greater richness and abundance were: Laridae, Recurvirostridae and Ardeidae, in the case of migratory: Scolopacidae; those families composed of individuals with either of the two residency status are: Anatidae and Charadriidae. Important reproductive colonies of Pelecanus occidentalis, Anous stolidus, Sterna fuscata were found and subjected to anthropocentric and climatic activities that have affected the decrease of their populations. The species Calidris canutus, C. melanotus and Charadrius vociferus considered unusual or vagrants have been observed with rings. Nueva Esparta state is a proven stopover site for Neartic migrants.}

\normaltalk{Contrasting Fall Migratory Movements in Bank and Barn Swallows}{Tara L Imlay, Hilary AR Mann, Sarah Saldanha, Philip D Taylor}{Past research suggests that the drivers of population declines for several aerial insectivores likely occur during the non-breeding period. While several recent studies have provided more information on the winter locations of different species, there are still considerable gaps in our understanding of migratory movements, particularly the routes used, departure timing, and pace of movements. We deployed nanotags on adult Bank (Riparia riparia) and Barn Swallows (Hirundo rustica) during the breeding season in New Brunswick, Canada, and tracked their fall migratory movements with the Motus Wildlife Tracking System. Our results show that most Bank Swallows and all Barn Swallows travel along the Atlantic coast of Canada and the United States. Bank Swallows departed earlier than Barn Swallows, and, for both species, the initial movements were in a direction consistent with their migratory route (i.e., minimal post-breeding exploratory movements). The pace of movements for Bank Swallows was slower than for Barn Swallows, suggesting that this species may undertake longer or more frequent stopovers during migration. Also, migratory movements occurred during the day. Collectively, our findings suggest that departure timing follows the cessation of breeding activity, which is earlier for Bank Swallows compared to Barn Swallows. Although all individuals move in a direction consistent with their migratory route, the slower pace of movements observed in Bank Swallows suggest that this species has more time to complete fall migration than Barn Swallows. Finally, the fall migratory movements of these species is consistent with fly-and-forage migration strategy that requires an abundant food supply throughout migration.}

\normaltalk{The Behavioral Ecology of Southwestern United States Turkey Domestication}{Catherine E Mendel}{Recent research has demonstrated an independent domestication event of Turkey (Meleagris gallopavo) occurred in the Southwestern United States between 200 BC--AD 500. This event was separate from the domestication of turkey within the Mesoamerican world approximately 2000 years ago. While aDNA analysis reveals this as a separate and distinct event, we still know little about how turkey husbandry was practiced in the daily lives of prehistoric agricultural groups from the Southwest, USA.  Past stable isotope analyses by the first and second author has demonstrated a mix of foddering and management practices. Turkeys that consumed almost an entirely maize diet appear to have been restricted in movement or confined in some way, while those that had access to insects and other non-maize foods likely had some degree of mobility around the pueblos. The results strongly resemble ethnohistorical accounts of turkey husbandry. Our current research evaluates the domestication process through the lens of evolutionary ecology principles, specifically the behavioral ecology of mutualistic relationships, and ethnohistorical accounts of the uses (both dietary and other) of turkeys in this region.  The addition of this theoretical approach permits for the incorporation of both human and turkey behavior; the benefits to both species; and builds on the paradigm which views domestication events as mutualistic relationships with informed perspectives from both species involved.}

\normaltalk{Movements and Ecology of Common Nighthawk and Sickle-Winged Nightjar from South American Grasslands}{Kristina L Cockle, Olga Villalba, Nestor Fari\~{n}a, Luis G Pagano, Alejandro Bodrati, Andrea R Norris, Kathryn EC Davis}{Migratory aerial insectivores are declining across North America, but hypothesized drivers -- habitat loss, prey declines, toxins -- may well act in South America. In northeastern Argentina, we studied the ecology of Common Nighthawk (Chordeiles minor; CONI), a declining long-distance migrant, and near-threatened Sickle-winged Nightjar (Eleothreptus anomalus; SWNJ), a South American grassland endemic, using spotlighting, banding, GPS-loggers and radio-tags (2012--2020). CONI arrived November-January, finished moulting primaries, and gained body mass (12\%) before departing in March. Two females spent May-August in Florida (Citrus WMA and Naples), then 1--3 months in Cerrado (Brasnorte and Ilha do Bananal) surrounded by soybean agroindustry, near wintering sites of CONI that bred in boreal forest, suggesting the Cerrado-Amazon ecotone may be critical even for migrants that ``winter" farther south. SWNJ (141 banded, 8 radio-tagged) exclusively used native tall grassland, avoiding pine plantations. Males were smaller than females, detected four times more often, displayed at specific points along roads, never had brood patches, and foraged over smaller, overlapping home ranges (male: 110 ± 57 ha; female: 367 ± 162 [mean ± SE, n = 6]) encompassing display sites, suggesting a lek-like mating system that may require hundreds of hectares of contiguous native grassland to maintain populations. Since 1996 the study region saw a doubling in plantation cover and 900\% increase in insecticide use, highlighting the urgency for studies of aerial insectivore diets, prey availability, and toxicity. A top priority for migrant and resident aerial insectivores is to protect native Cerrado and southern South American grasslands from agro-industry.}

\normaltalk{Magnetic Disturbances and Bird Migration on Radar}{Eric R Gulson-Castillo, Benjamin M Van Doren, Kyle G Horton, Benjamin M Winger}{Birds use a variety of cues for navigation during migration, including Earth's magnetic field. However, the magnetic field is not constant: it experiences sun-induced decadal cycles that are punctuated by a period of frequent magnetic storms. During these storms, the intensity of Earth's magnetic field varies erratically, potentially reducing its usefulness as a navigation cue. We hypothesize that magnetic storms reduce birds' ability to navigate accurately. We thus predict that these storms affect the density and directionality of bird migration. We test this hypothesis using a 20 year weather radar dataset from 36 radar stations in the central US and global indices of magnetic disturbance. We use mixed-effect models (both linear and additive) to control for spatial and temporal correlations while examining the influence of magnetic disturbance indices on bird migration relative to meteorological predictors.}

\normaltalk{Artificial Light at Night Reduces Functional Connectivity of Migratory Routes for Eastern Whip-Poor-Wills}{Alicia Korpach, Colin Garroway, Alex Mills, Val von Zuben, Christina Davy, Kevin Fraser}{Birds use aerial habitat to forage, communicate, and travel, and anything that fragments that habitat may limit their ability to use airspace efficiently. Artificial light at night continues to grow globally, thus nocturnally migrating birds are increasingly required to navigate airscapes fragmented by light. Attraction to artificial light by groups of nocturnal migrants is well documented, but there is also evidence of light avoidance, particularly during stopovers. We do not have a good understanding of how individual birds, and nocturnal species specifically, perceive and navigate artificial light along their full migratory path. We examined the question of artificial light selection versus avoidance during migration using a highly nocturnal species, the Eastern Whip-poor-will (Antrostomus vociferous). We created a dark-sky connectivity surface representing the most efficient dark-connected routes that birds could take on their southbound migrations through eastern North America and compared it to the actual routes of GPS-tracked individuals. We found that the birds were more likely to migrate through connected patches of dark skies, and migratory stopovers occurred almost exclusively in dark, rural, areas. These travel and stopover decisions were made at various spatial scales. Our tracked Whip-poor-wills also traveled more slowly in dark-connected skies, which is suggestive of a fly-and-forage migration strategy; this possibility should be tested further in future studies. The presence and the configuration of light in both urban and rural areas influenced movements of a nocturnal species, but our findings are relevant to all nocturnal migrants moving through highly light-fragmented regions in North American migratory passageways.}

\normaltalk{Migratory Behavior in Junco Hyemalis (Dark-Eyed Junco) in Berkshire County, Massachusetts}{Hannah L Wait, Daniel P Shustack}{Junco hyemalis (Dark-eyed Junco) is a widespread songbird in North America. The J. h. hyemalis subspecies can be observed year-round in Berkshire County, Massachusetts, but questions remain about its migratory behavior in this part of its range. For example, are some individuals year-round residents? Do others display altitudinal migration, like the more southern carolinensis subspecies? We are using deuterium ratios from secondary flight feathers of juncos to assess migratory behavior of populations in western Massachusetts and identify breeding locations of overwintering individuals as well as individuals captured during fall and spring migration. To date we have sampled 42 juncos for deuterium values. Male juncos (N=7) captured in Berkshire County during the summer displayed deuterium ratios of -67 to -52 dD, whereas overwintering male juncos (N=3) captured in December and January had dD ratios between -115 and -105. Deuterium ratios in precipitation suggest male juncos overwintering in Berkshire County travelled from breeding grounds in northern Canada. Juncos captured during migration had dD values between the values of the summer breeders and overwintering males, suggesting that juncos from across the breeding range north of Berkshire County migrate through our study area. Overwintering juncos were rare in our study system during the winter of 2018-2019, leading to low captures. Therefore, we have continued sampling during the winter of 2019-2020 in order to more fully describe isotopic values from overwintering juncos.}

\normaltalk{Should I Stay or Should I Go? the Decision of a Wilson'S Plover in a Vanishing Landscape}{Katie B Barnes, Kiah Williams, Erik I Johnson}{Louisiana's rich and diverse coastal habitats are disappearing into the Gulf of Mexico because of complex and substantial erosion issues. Beach-nesting birds are among those impacted by coastal land loss and sea level rise, in addition to other threats, which could influence territory selection processes. We hypothesized that between-year site fidelity in Wilson's Plovers (Charadrius wilsonia) may be influenced by their previous year's breeding success, access to high-quality habitat, or both. Louisiana supports about 2,500 pairs of Wilson's Plovers, which is approximately 31\% of the U.S. population. We banded adult Wilson's Plovers with field-readable codes to investigate survivorship, annual site fidelity, breeding productivity, and habitat use across multiple years in coastal Louisiana. Annual survivorship, accounting for imperfect detection, was 74.5\% (SE 6.5\%) and 78.9\% (SE 7.2\%) in males and females, respectively. For the surviving birds that returned each year, we used generalized linear mixed models to assess the significance of sex, breeding productivity, and distance to nearest mudflat in explaining variation in between-year site fidelity across 2017-2019. Our results indicated that males had a significantly higher probability (90.1\%) of site fidelity in subsequent years compared to females (68.4\%). Additionally, both sexes were more likely to be site faithful when territories were closer to mudflats. Breeding productivity, however, did not predict variation in between-year site fidelity. Understanding these metrics has implications for future coastal restoration planning as we begin to understand species conservation needs and how these birds respond to large-scale restoration projects that directly impact these habitats.}

\normaltalk{Two Sympatric Mitochondrial Lineages of Oriental Turtle Dove}{Mizuki K Mochizuki, Isao N }{In general, mitochondrial DNA shows low sequence variation within species and high sequence divergence among species. Therefore, the DNA barcoding using mtDNA is an effective tool for species identification, and sister species usually have over 2\% divergence in birds. A previous study on DNA barcoding of Japanese birds showed that all of conspecific divergence are allopatric except for one species, oriental turtle dove, which has two divergent mtDNA lineages in the same area. Such sympatric lineages are relatively rare in birds, and only known in Italian sparrow, common redstart, and raven.   Thus, we collected DNA samples from some subspecies of oriental turtle dove in East Asia and examined how the two mtDNA lineages are distributed. We also tested several microsatellites and introns of nuclear DNA to investigate the possibility of hybridization with other species or ghost introgression.  As a result, the two mtDNA lineages of oriental turtle dove were widely and sympatrically distributed not only in Japan but also in East Asia, and the two lineages likely cross each other. It suggests that two populations had been geographically isolated until the last ice age and it have crossed later and then separated into the current subspecies.}

\normaltalk{What Climate Conditions Does an Intratropical Migratory Bird Track? the Yellow-Green Vireo (Vireo Flavoviridis)}{Alejandra Ochoa-Gonz\'{a}lez, Octavio Rojas-Soto, Adolfo G Navarro-Sigüenza}{Migratory birds are expected to move geographically by tracking specific climatic conditions through time, which can be similar or different between seasons depending on species attributes. However, we still understand little about the distributional ecology of many species of migratory birds, especially those that migrate within the tropics. Understanding how birds relate to environmental conditions in intratropical migration could contribute to clarify the evolution of this trait in birds and might also be useful for conservation in the face of rapid climate change. Herein, we analyze the ecological setting of an intratropical and short distance migratory songbird, the Yellow-green virep (Vireo. flavoviridis) to test if it follows similar climatic conditions throughout the annual cycle. We used GBIF presence records and WorldClim monthly climatic layers to build Ecological Niche Models (ENM) for nesting and wintering seasons, and projected those into spring and fall migration to evaluate the climatic conditions tracked in transitional months. Also, we used a niche overlap analysis in order to test similarity of the nesting and wintering niches. Results suggest that V. flavoviridis track similar conditions through the annual cycle, and tends to follow the nesting climatic niche during the transitional months. Our results support the idea that intratropical short distance migration could be a primitive stage of long-distance migration towards temperate zones, and provide important ecological information for future conservation plans.}

\normaltalk{Evolution of Molt-Migration is Directly Linked to Aridity of the Breeding Grounds in North American Passerines}{Claudie Pageau, Christopher M Tonra, Mateen Shaikh, Matthew R Reudink}{To avoid energy allocation conflicts, birds generally separate the most energetically expensive life history events during the annual cycle: breeding, migration, and molt. Typically, in North America, most passerines complete their molt on the breeding grounds after breeding, but prior to fall migration. However, some birds have evolved a molt-migration strategy in which they delay molt until stopping over in the Mexican monsoon region en route to their wintering grounds. The push/pull theory by Rohwer et al. (2005) has often been proposed as an explanation for the evolution of this particular moult strategy, but has not yet been empirically tested. Poor conditions on the breeding grounds at the end of the summer would push birds to depart prior to molt, while productive stopover locations would pull them towards that location. We tested for a relationship between molt-migration and aridity of the breeding grounds as measured by normalized difference vegetation index (NDVI). Our results indicate that species whose breeding grounds are dry and unproductive during the post-breeding period (indicated by low NDVI values) were much more likely to evolve a stopover moult-migration strategy. Thus, our results strongly support the ``push" aspect of the push/pull theory.}

\normaltalk{Demographic Dynamics of a Central American Pine-Oak Assemblage in Response to Pleistocene Climate Fluctuations}{Jack Hruska, Joseph Manthey}{Despite being widely acknowledged as a driver of differentiation and extinction, the effects of long-term environmental change on evolutionary trajectories are poorly understood. To date, most evaluations have employed a species-specific approach when addressing this question, precluding the establishment of a theoretical framework for how communities of organisms will respond to shared climate perturbations. Not only will such a framework provide a better understanding of how present-day phylogeographical patterns have come to be, but also improve predictions of what organisms may be at greatest risk under future scenarios of prolonged climatic change. Here, we use whole-genome variants from an assemblage of four co-distributed birds, the House Wren (Troglodytes aedon), Gray-crowned Yellowthroat (Geothlypis poliocephala), Acorn Woodpecker (Melanerpes formicivorus) and Azure-crowned Hummingbird (Amazilia cyanocephala), across highland and lowland pine-oak forests of Nicaragua and El Salvador to examine how Pleistocene climate fluctuations have altered the temporal dynamics of effective population sizes (Ne) over time. Pine-oak forests of Mesoamerica reached their zenith during the Pleistocene Last Glacial Maximum and have undergone cycles of fragmentation during warmer interglacial periods. By sampling taxa that are both specialists (restricted to pine-oak forests) and generalists (not restricted to pine-oak forests), we demonstrate that specialists experience population dynamics that are most concordant with temporal patterns of pine-oak expansion and contraction.}

\normaltalk{Comparative Phylogeography Reveals a Pervasive Sister Relationship Between Bird Populations of the Northern and South-Central Atlantic Forest}{Fernanda Bocalini, Sergio D Bol\'{i}var-Leguizam\'{o}n, Lu\'{i}s F Silveira, Gustavo A Bravo}{The Pernambuco Center of Endemism (PCE) is the northernmost strip of Atlantic Forest (AF). The biogeographic affinities between avifaunas in the PCE, the southern-central Atlantic Forest (SCAF), and Amazonia (AM) have not been studied comprehensively, and extant patterns of genetic diversity in the PCE remain unclear. The interplay between species' ecological attributes and historical processes, such as Pleistocene climate fluctuations (Forest Refugia hypothesis) or the appearance of rivers (Riverine Barrier hypothesis), may have affected population genetic structure in the PCE. Moreover, the role of past connections between the PCE and AM and the elevational distribution of species in assembling the PCE avifauna remain untested. Here, we investigated the biogeographic history of seven taxa endemic to the PCE within a comparative phylogeographic framework based on 3,618 independent single nucleotide polymorphisms (SNPs) extracted from adjacent regions of ultraconserved elements (UCEs) and one mitochondrial gene. We found that PCE populations were more closely related to SCAF populations than they were to those in AM, regardless of their elevational range, with divergence times placed during the Mid-Pleistocene. These splits were consistent with a pattern of allopatric divergence with gene flow until the upper Pleistocene with no signal of rapid changes in population size. Our results support the existence of a Pleistocene refugium driving the current genetic diversity in the PCE, thereby rejecting the role the Rio São Francisco as a primary barrier for population divergence. Connections with Amazonia also played a significant role in assembling the PCE avifauna through subsequent migration events.}

\normaltalk{Insights into Mixed-Species Bird Flocks Dynamics Through Social Network Analysis from a Costa Rican Highland}{Pablo Mu\~{n}oz, Luis Sandoval}{Mixed-species flocks have been studied focusing mostly on composition, foraging efficiency, and protection against predators. However, the processes that determine their establishment, cohesion, and interactions inside the flocks are not understood yet. We propose to use the general model for community assembly on islands through social network analysis to identify the co-occurrence and interaction patterns that promote flocks. This model considers each flock as an island that can be colonized by the overall bird community of the area. We recorded flock composition (species and individuals) and interactions (attack, follow and forage) twice a week during 2019 in a Costa Rican highland forest. We conducted social network analysis on flocks using co-occurrence and interaction weighted matrices. Our analysis showed that flock formation is not a random process, because, from all species combinations in the area expected by chance, not all of them were recorded in the field. We found that species roles inside flocks may changed (leader to follower), related to the occurrence of other species. These results supported the general model for community assembly on islands because not all the species of the community formed flocks and its composition was not by chance. The presence of some species may be conditioned to the occurrence and interactions with another species, like the flocking woodcreepers, which are more common when the Ruddy Treerunner occurs inside the flocks. This settles evidence to our assumption that certain species combinations may repel, facilitate, or modify the behavior of others inside the flock.}

\normaltalk{Influence of Elevation on Canada Warbler Population Dynamics in the Central Appalachian Mountains, USA}{Stephanie H Augustine, Christopher T Rota}{Canada Warblers (Cardellina canadensis) are a Nearctic-Neotropical migratory songbird that has exhibited apparent declines in abundance over recent decades. This species occupies a wide range of environmental conditions throughout their range but lack substantial data regarding elements driving variation in demography and the strength of population migratory connectivity. The aim of this research is to (1) determine the relationship between demography and environmental conditions along an elevation gradient and (2) ascertain sex-specific migratory route and wintering locations of a population of Canada Warblers breeding in the central Appalachians. Our research takes place at six study sites with an elevation range of 526 -- 1282m spanning an approximate 130km north-south gradient in the Monongahela National Forest, West Virginia, USA. We will estimate apparent survival with a three-year mark recapture study; in 2019 we marked 104 birds, which we will re-sight in 2020 and 2021. Beginning in 2020, we will assess reproductive success by monitoring nests using game cameras. We will model these demographic rates as a function of elevation and additional environmental variables, which will elucidate the region-specific habitat-demography relationship. To determine sex-specific migration strategies, we will deploy 24 males and 9 females with light-level geolocator tags in spring 2020 and retrieve tags from returned individuals in spring 2021. This study is the first we know of to track individual Canada Warblers throughout the year and will establish a baseline for full annual cycle modeling. This presentation will include results from the first two field seasons of the project.}

\normaltalk{Multiple Sources of Evidence for Density Dependence in the Endangered Hawaiian Stilt}{Charles B van Rees, Gautam Surya, J M Reed}{Hawaiian stilts (Himantopus mexicanus knudseni) are an endangered subspecies of the Black-necked stilt endemic to the Hawaiian Islands. Despite long-term study, the main drivers of Hawaiian stilt population dynamics are poorly understood. We tested for density dependence using two sources of evidence: a 30-year time series of annual estimated range-wide abundance, and two 15+ year time series of reproductive success. Using separate methods with independent data, sources allowed us to make up for the potentially positive bias of one approach with the more conservative nature of the second. We compared nonlinear density-dependent and density independent population model fits to our time-series data, using both frequentist and Bayesian state-space approaches. Across both approaches, density-dependent models best fit observed population dynamics, with lower AICc and cross validation  statistics compared to density-independent models. Among density dependent models, a conditional model in which density-independent dynamics occur below a population size threshold (~850--1,000 birds), and then density dependent dynamics occur above that threshold, performed best across Bayesian and frequentist model comparisons, with the Ricker model ranked next or equivalently. Our analysis of reproduction data revealed a strong negative effect of local adult density on nest success (proportion of nests hatching at least one chick) at Kealia National Wildlife Refuge on Maui, where few alternative breeding habitats are available, but no such effect at another site where many nearby alternative wetlands are available. These congruent results across independent datasets and analytical approaches support the hypothesis that Hawaiian stilts exhibit density dependence across their range.}

\normaltalk{Las Salinas, Ci\'{e}naga De Zapata: An Important Area for Reddish Egret in Cuba}{Alieny Gonzalez, Daniela Ventura, Lourdes Mugica, Martin Acosta}{Reddish Egret is a species of interest for conservation efforts. Cuba, along with the remaining Caribbean area and Florida, belongs to the Eastern Management Unit, which is the most vulnerable and understudied. Our project highlights the importance of Las Salinas as a foraging and breeding habitat for the species. The objective was to characterize the population status of Reddish Egret during the 2009-2019 period in a foraging area and breeding colony within Las Salinas' lagoon system. The data was grouped into four seasons of importance for the species. The highest population numbers were recorded during the pre-breeding season (53.8 ± 9.9 ind.) and the lowest during the post-breeding season (11.9 ± 2.1 ind.). The population dynamics of the pre-breeding season was markedly different from the rest (Monte Carlo probability, Breeding= 0.01; Post-breed. = 0; Non-breed. = 0). In all seasons, the proportion of adults was higher than 50\%. However, differences were found among the non-breeding season (which had the highest proportion) and the rest of the seasons (Monte Carlo probability, Post-breed.=0.02; Breed.=0.02; Pre-breed.=0.01). In the breeding colony, 19 nests were detected in 2015, and 22 nests in 2017. The colony remained active from May to October, and the highest number of active nests were found in August. Our results indicate the significance of the pre-breeding season in the foraging area, and the breeding season in the colony site. Further monitoring programs of the species could focus on the months included in each of these two important seasons.Egretta rufescens es una especie de inter\'{e}s para la conservaci\'{o}n. Cuba, junto al resto del Caribe y Florida, pertenece a la subunidad de manejo Este la cual a su vez es la m\'{a}s vulnerable y la menos estudiada. El trabajo resalta la importancia de Las Salinas como \'{a}rea de forrajeo y de reproducci\'{o}n pues se caracteriza el estado poblacional de E. rufescens en el periodo 2009-2019 en un \'{a}rea de forrajeo y en una colonia reproductiva. Para el an\'{a}lisis los datos fueron agrupados en cuatro etapas de importancia para la especie. Los valores m\'{a}s elevados se registraron durante la etapa pre-reproductiva (53.8 ± 9,9 ind.) y los m\'{a}s bajos en la post-reproductiva (11.9 ± 2.1 ind.), siendo diferente la primera de las restantes (Probabilidad de Montecarlo, Rep = 0,01; Post-rep = 0; No-rep = 0). La proporci\'{o}n de adultos fue mayor del 50 \% de todas las etapas estudiadas. Sin embargo, se detectaron diferencias entre la etapa no-reproductiva, donde se registraron los mayores valores, y las restantes etapas (Probabilidad de Montecarlo, Post-rep. =0,02; Rep. = 0,02; Pre-rep. = 0,01). En la colonia se contabilizaron 19 nidos en el 2015 y 22 en el 2017. La colonia se mantuvo activa desde mayo a octubre, encontr\'{a}ndose la mayor cantidad de nidos activos en agosto. Los resultados del trabajo evidencian la importancia de la etapa pre-reproductiva en el \'{a}rea de forrajeo de Las Salinas, y de la etapa reproductiva para la colonia. Posteriores programas de monitoreo de la especie en el lugar podr\'{i}an enfocarse en los meses incluidos en cada una de estas etapas.}

\normaltalk{Genetic Diversity in Toll-Like TLR1LB and TLR4 Receptors in Migratory Populations of Ammodramus Savannarum in Northern Mexico}{Gabriel H Aldaco, Ricardo Canales, Jos\'{e} G Ham, Irene Ruvalcaba}{The assessment of genetic diversity within and among grasshopper sparrow populations can provide valuable insight into the evolutionary potential as well as the conservation challenges of a species that has seen a steady decline in population. Reduced genetic diversity can lead to reduced evolutionary potential and adaptability. Several studies have used toll like receptors (TLRs) to assess genetic diversity, these are innate immune genes related to pathogen resistance. In this study we evaluated genetic diversity in TLR1LB and TLR4 in two wintering grasshopper sparrow population in the north of Mexico (Saltillo and Durango). By assaying haplotype diversity, it was possible to infer an apparent lack of genetic structure in TLR1LB even though it had a higher number of haplotypes (11) none were shared among populations, however it was found that up to a certain degree there is evidence of population structure in TLR4, since it was found that there was a shared haplotype between populations. This could imply that TLR4, has the potential to be a highly variable marker that could be used to describe population dynamics of the grasshopper sparrow. As it has been pointed out in previous studies the high genetic diversity and apparent lack of genetic structure could be attributed to breeding pairs forming while different migrant populations mix in wintering grounds.  La evaluaci\'{o}n de la diversidad gen\'{e}tica dentro y entre las poblaciones de gorriones de saltamontes puede proporcionar informaci\'{o}n valiosa sobre el potencial evolutivo, as\'{i} como los desaf\'{i}os de conservaci\'{o}n de una especie que ha experimentado una disminuci\'{o}n constante de la poblaci\'{o}n. La diversidad gen\'{e}tica reducida puede conducir a una reducci\'{o}n del potencial evolutivo y la adaptabilidad. Varios estudios han utilizado receptores tipo Toll (TLR) para evaluar la diversidad gen\'{e}tica, estos son genes inmunes innatos relacionados con la resistencia a los pat\'{o}genos. En este estudio evaluamos la diversidad gen\'{e}tica en TLR1LB y TLR4 en dos poblaciones de gorriones de saltamontes de invernada en el norte de M\'{e}xico (Saltillo y Durango). Al analizar la diversidad de haplotipos, fue posible inferir una aparente falta de estructura gen\'{e}tica en TLR1LB a pesar de que ten\'{i}a un mayor número de haplotipos (11), ninguno se comparti\'{o} entre las poblaciones, sin embargo, se encontr\'{o} que hasta cierto punto hay evidencia de estructura de la poblaci\'{o}n en TLR4, ya que se descubri\'{o} que hab\'{i}a un haplotipo compartido entre las poblaciones. Esto podr\'{i}a implicar que TLR4 tiene el potencial de ser un marcador muy variable que podr\'{i}a usarse para describir la din\'{a}mica de la poblaci\'{o}n del gorri\'{o}n saltamontes. Como se ha se\~{n}alado en estudios anteriores, la alta diversidad gen\'{e}tica y la aparente falta de estructura gen\'{e}tica podr\'{i}an atribuirse a la formaci\'{o}n de parejas reproductoras, mientras que las diferentes poblaciones migrantes se mezclan en las zonas de invernada.}

\normaltalk{Transatlantic movement of domestic pigeons Columba livia domestica and potential threats for native island birds}{Gabriel S Santos, Maur\'{i}cio B Vecchi, Maria Alice S Alves}{Historically, Domestic pigeon (Columba livia domestica Gmelin,1789) (Columbiformes, Columbidae) breeders use the species to promotes race competitions which over the time could select traits to increase the flight performance and spatial orientation capabilities. Although the species has remarkable navigational abilities, it is possible that these birds sometimes become disoriented pushing the individuals to fly off-course and over distances larger than usual. Here we report a Domestic Pigeon ringed in the Canary Islands (Spain) observed in the Abrolhos Archipelago, eastern Brazil, and compiled similar events (N = 5) in the Southwest Atlantic Ocean coast. Our results indicate that these events are largely unnoticed in the scientific literature as most of them were found in non-scientific newspaper and academic conferences, thus, untraceable from scientific databases as Scopus and Google Scholar. As domestic pigeons are possible intermediate hosts of diseases, we encourage the creation of a systematized database of this kind of movements of these birds, including the profile of these vagrants and the circumstances in which these events occur, which can be a basis to be used for scientists and decision makers.}

\normaltalk{Regional Collaboration to Identify Research Needs and Develop Conservation Guidance for Offshore Wind Development}{Kate Williams, Kate M Press, Julia Gulka, Gregory Lampman, Pamela Loring}{State policies have become a vital driver of the nascent offshore wind industry in the United States. New York State has put substantial emphasis on regionally focused, science-based collaboration to develop guidance and fill data gaps around the potential risks and impacts of offshore wind energy development to wildlife along the east coast of the U.S. A range of offshore wind developers, managers, environmental advocates, and scientists have been involved in these efforts, which include: (1) Development of a scientific research framework to better understand the impacts of offshore wind energy development on birds and bats, (2) Identification of research and coordination priorities to help improve our understanding of cumulative impacts to wildlife as the offshore wind industry develops in the U.S., (3) Development of recommendations to mitigate impacts to birds during construction and operations of offshore wind facilities in the eastern U.S., and (4) Beginning to fill data gaps and research/coordination needs, including a recently initiated effort to develop guidance for deploying nanotag receivers and transmitters as part of post-construction monitoring efforts at offshore wind facilities. These efforts to understand, avoid, and minimize avian impacts from offshore wind development activities are informing state decision-making, reducing risks for developers, and improving conservation and management outcomes. The effectiveness of these collaborations is due in part to the regional, rather than state-specific, scale of interest; the incorporation of a wide range of stakeholder input with appropriate technical expertise; and a strong emphasis on the use of science to inform decision-making.}

\normaltalk{Updating Information About Audubon's Shearwater  (Puffinus Lherminieri Loyemilleri) at  Los Roques Archipelago (Venezuela)}{Juan C Fernandez}{Venezuela has about 314 islands, cays and islets in the Caribbean. Los Roques archipelago provides stopover and breeding sites for birds. However, it is necessary to identify critical areas and update information to ensure continuity of this important bird site. A total of 24 species (5 landbirds and 19 waterbirds) regularly nest at Los Roques. One of this breeding species is Audubon's Shearwater (Puffinus lherminieri loyemilleri), which breeds in unknown numbers, and without recent counted data available. Main objective of this study is to generate updated information about Audubon's Shearwater's breeding colonies, and to describe and evaluate threats at the breeding colonies. We have conducted nocturnal censuses to detect breeding Audubon's Shearwaters. Each island was divided into 10 x 10 m plots and using a simple occupancy method we visited random points to determine which plots was occupied by nesting shearwaters, spending 10 minutes in each plot to count the birds that responded to the song of the Audubon's Shearwater's male and female. Sarqu\'{i} Cay, located at the west-central section of the archipelago, is an uninhabited cay, without any human-made structure and no visible anthropogenic threats. During 2018, 13 singing birds were detected, and a total of 14 singing birds during 2019 censuses. We only found some trash on the sandy beaches. In the rocky areas, where the Audubon's Shearwaters were found, we only find some large pieces of plastic or wood. On this island there are no introduced predators and, apparently, there are no rats or mice.}

\normaltalk{Abundance and Management Strategies for Invasive Rose-Ringed Parakeets (Psittacula Krameri) in Kaua`i, Hawaii, USA.}{C Jane Anderson, William P Bukoski, Bryan M Kluever, Page E Klug, Shane R Siers, Aaron B Shiels, Eric A Tillman, Leonard A Brennan}{The rose-ringed parakeet (Psittacula krameri) is among the most invasive bird species worldwide. Rose-ringed parakeets were introduced to Kaua`i, Hawaii, in the 1960s. The population initially remained relatively small but began growing in the early 2000s and was estimated at 6,800 individuals in 2018. This population is causing agricultural damage by foraging on fruit and grain crops. The parakeets also congregate nightly in large roosts in urban areas, causing disturbance through loud vocalizations and threatening human property and health through excessive droppings. To address this problem, we are evaluating historic take records and estimating current rose-ringed parakeet abundance in Kaua`i to estimate annual population growth rate. Further, we will evaluate the efficacy of two population control strategies. We will evaluate whether parakeet-selective feeders can be used to deliver baits to parakeets while preventing access by nontarget species, providing inference to potential contraceptive or toxicant methodologies. Monitoring a roost culling effort from March -- June 2020, we will evaluate number of birds taken per unit of effort, age and sex ratio of birds taken, how varying levels of effort could be projected to influence future population size, and whether roost culling facilitates roost abandonment. We estimated the 2020 population to include =10,500 parakeets, despite the removal of $>$7,000 individuals in the preceding 15 years. These studies will be incorporated into an integrated pest management plan to reduce the invasive rose-ringed parakeet population and related impacts on Kaua`i.}

\normaltalk{A Comparison of Sagebrush-Obligate Songbird Abundance Across a Gradient of Habitat Quality in Eastern, Oregon}{Alan Harrington, Jonathan Dinkins}{A prominent conservation shortcut across many ecological systems is the umbrella species concept. This single-species conservation effort is rooted in the notion that indirect protection is provided for multiple species that occupy the same area as the focal species. Greater sage-grouse (Centrocercus urophasianus, hereafter ``sage-grouse") is a sagebrush-obligate and high-profile umbrella species for sagebrush ecosystems. Successful application of conservation investments for sage-grouse that benefit both sagebrush ecosystems and other sagebrush associated taxa adhere to the tenants of the umbrella species concept. However, empirical assessments are uncommon and lack consensus. Previous efforts to evaluate sage-grouse as an umbrella species have primarily focused on commonalities in land-cover associations, spatial habitat overlaps, and indices of co-occurrence of sagebrush associated species. Thus, we examined differences in abundance of sagebrush-obligate songbirds across a gradient of habitat quality among five Sage-Grouse Priority Areas of Conservation (PACs) representing 6,458 km2 of the sagebrush ecosystem in eastern Oregon, USA. We also directly compared abundance of sagebrush-obligate songbirds in available habitat among PACs to locations actually used by sage-grouse. In 2018 and 2019, we surveyed 147 random point locations twice per month and 160 sage-grouse locations from May through July. We will present preliminary results assessing sagebrush-obligate songbird abundance across a gradient of habitat quality, and in relation to sage-grouse use locations compared to available sagebrush habitat. Our results will inform management prescriptions about the potential effectiveness of conservation activities for sage-grouse and their extension to sagebrush-obligate songbirds.}

\normaltalk{Birdwatching and Noise Level Effects on Bird Species Richness and Abundance: Supporting Tourism Management in Natural Areas}{Ingrid Molina}{Birdwatching may be a low impact activity for an ecosystem, but large numbers of tourists in the same area or incorrect birding practices (i.e., playback use) may affect it negatively. Another factor that may affect birding destinations is the noise contamination produced by people and the car traffic nearby.  My objective is to analyze the effect of birders presence and anthropogenic noise levels on the richness and abundance of bird species inside a protected area. I conducted bird point counts every week from December 2018 to May 2019 in Carara National Park, Costa Rica. In each point, I annotated the richness and abundance of species, measured the noise level with a sound level meter, counted the number of tourists present, and described the surroundings (vegetation, visual range, and water availability). I found that the increase in noise and tourists presence decreases the richness and abundance of species. In most cases the locations near the highway showed less richness than far locations. Points with introduced and native vegetation mix allowed more richness and abundance of species. Three priority points for investment in infrastructure for birdwatching were identified. Generally, limiting the size of tourist groups and diminishing the noise (by assigning areas for birding and controlling bad practices) will help avitourism to remain sustainable.}

\normaltalk{Photo-Ptilochronology: A less Invasive Technique to Measure the Speed of Molt in a Passerine Bird}{Carlos E Valeris, Katherine Ch\'{a}vez, Guyleine Rios, Mariana Klesse, Rodrigo A V\'{a}squez}{Ptilochronology is a method that uses feather band measures to establish the feather growth rate, which is useful to determine the speed of molt. We adopted the ptilochronology technique under a different approach for its use in free-living passerines, with no feather extraction nor sacrifice individuals. We called this photo-ptilochronology and applied in 33 individuals of Rufous-collared sparrows (Zonotrichia capensis), from central Chile. We took standardized digital pictures of the ventral view of rectrices feathers over a blackboard and scaled with a metallic ruler. Each picture was evaluated for four independent observers, which selected the pictures that showed clear visibility of at least three feather bands, on at least one outer rectrices feather (r4 to r6). Pictures were processed and analyzed independently, with the program ImageJ®. There were no significant differences between observers' measures despite the different numbers of feather bands measured for the same individuals. We used an ad hoc value of 0,099 mm for standard deviation (SD) as a limit value to select an average like correct. Using this value, 42\% of average measures show higher SD but removing the most extreme value between the four observers' measures 94\% of the average score had SD below this limit. We encourage the use of photo-ptilochronology as a technique to estimate the speed of molt in free-living birds because it is feasible to apply, it is less invasive than ptilochronology, and can provide insight regarding this life-history trait in bird populations. (AFB170008-CONICYT).}

\normaltalk{Are Differences in Wing Shape in Migratory and Sedentary Yellow Warbler Populations Shaped by Feathers of Skeletal Features?}{Vera Ting, Teresa Pegan, Benjamin M Winger}{Many studies show that migratory birds typically have long and pointed wings, important for flight ability and efficiency, while sedentary birds typically have short and round wings. Wing shape is usually measured using hand-wing index (HWI), which is calculated based on the length of the first secondary relative to the length of the manus bones plus the longest primary (i.e. wing length). While past research has investigated differences in HWI among different bird populations, none have focused on whether these differences are due to changes in feather length, bone length, or both. In particular, the importance of skeletal evolution in shaping the morphology of migratory birds remains poorly explored. In this study, we compared feather and manus bone lengths in Yellow Warblers (Setophaga petechia) from Michigan, South America, and the Caribbean by measuring combined skin/partial skeleton specimens at the University of Michigan Museum of Zoology. Yellow Warblers from Michigan migrate, while South American and Caribbean populations do not. Because wing shape is important for flight ability, we predict that migratory Yellow Warblers will have more pointed wings than non-migratory populations. We also predict that manus bone length will be longer in migratory Yellow Warblers. Preliminary results indicate that in tropical subspecies, manus bone length was shorter and first secondary length was longer, leading to rounder wings in tropical subspecies. The results of this study are important in informing researchers about how different aspects of avian morphology respond to selection from the environment and behavior to shape overall phenotype.}

\normaltalk{Molt Strategy, Breeding Phenology and Seasonal Occurrence of the South American Painted Snipe in Southern Brazil}{Fernando A Faria, M\'{a}rcio Repenning, Cindy Barreto, Leandro Bugoni}{The South American Painted-snipe (Nycticryphes semicollaris, SAPS) is a secretive endemic South American shorebird that occurs in subtropical lowlands. Feather molt and breeding strategy associated with seasonal movements remains elusive. Monitoring a breeding population, we aim determine SAPS molt strategy and age cycle, and identify breeding phenology and seasonal occurrence. Surveys were carried out in coastal estuarine grasslands in Lagoa do Peixe National Park, Southern Brazil (31°55'S; 052°10'W) complemented by occurrence using photographic records (n=119) from online platforms (eBird and Wikiaves). We determined SAPS molt strategy and age using W-R-P cycle-based aging system. Breeding phenology was assessed based on active nests, brood patches, fledgling and newly juvenile plumage of birds. Based on 30 birds captured, we determinate the Complex Basic Strategy with Preformative molt limited (remiges retained) and Staffelmauser incomplete in the second Prebasic molt, in which secondary and outer primaries are retained. We successful aged birds until the Third Cycle Basic. We documented, at individual level, complete overlap between breeding (active brood patch) and molt. Breeding period started in August, with peak along November and strong drop throughout December. The last eggs were incubated in December and last chicks observed in mid-January. Birds were most frequent from October-February (90.4\% of records). Lack of fall and winter records can indicate migratory behavior or local movements. SAPS could track environmental conditions such as flooding and food availability. Overlaps of molt and breeding could be a local adaptation to short seasonal breeding period in a temporarily flooded habitat with pulse of food.}

\normaltalk{Behavioral Response of Carolina Wrens (Thryothorus Ludovicianus) to Songs Altered to Escape the Masking Effects of Anthropogenic Noise}{Leah C Crenshaw, Caralee A Shepard, Spencer M Skaggs, Tristian J Wiles, Maureen R McClung}{Noise from compressor stations associated with shale gas extraction can impact the behavior of birds and other wildlife. Previous work by the authors found that Carolina Wrens (Thryothorus ludovicianus) sang at higher minimum frequencies near compressor stations compared to quieter well-pads, however, it is unknown whether these altered songs elicit different behavioral responses from receivers. In summer 2018 and 2019, we exposed 16 male wrens to altered and typical songs under quiet and noisy conditions and analyzed their behavioral response to playback. A playback speaker and noise speaker were placed in each bird's territory for four trials with the following conditions: altered song/noise, altered song/quiet, unaltered song/noise, and unaltered song/quiet. Each 30-minute trial had three phases: pre-, during, and post-playback during which three observers recorded the number of songs and flights, closest approach to playback speaker, time spent within 5m of the playback speaker, and latency to approach the playback speaker. Two-way repeated measures ANOVAs for all variables showed no significant interaction between trial treatment and phase, and that only phase significantly impacted these response variables (p$<$0.04 in all pairwise comparisons). In all treatments, birds responded to playback with fewer songs, more flights, approach to the playback speaker, and increased time spent within 5m of the speaker. There was no difference between treatments in latency to approach (p=0.28). Thus, both altered and unaltered signals successfully indicated presence of a territorial intruder in quiet and noisy environments. However, Carolina Wrens are synanthropic, and this result might not hold for other species.}

\normaltalk{A Look into Coastal Rufous-Collared Sparrows' Vocalizations: Do They Modify their Songs to Reduce Surf-Noise Masking?}{Elisabet Canale, Dar\'{i}o A Lijtmaer, Pablo L Tubaro}{Acoustic signal transmission is deeply intertwined with the surrounding environment and is constrained by noise. Birds have been shown to adjust their song amplitudes and frequencies, and also to use a repetitive and predictable structure, to avoid the masking effect of both natural and anthropogenic noise. However, the effect of surf-noise on the song structure of passerine birds has never been studied. Here, we present the first analysis of this kind based on Rufous-collared Sparrows (Zonotrichia capensis) living along the coastal environment in central Argentina. This species' song includes an introduction composed typically by 2 to 5 whistled notes, and a final trill, composed by a variable number of rapidly descending notes repeated at a regular rate. In January 2020 we recorded the songs of 30 individuals living close to the coast (less than 200 m from the surf zone) or far inland (between 500 and 1000 m from the coast) where the surf-noise is negligible. We also measured noise levels after recording each individual using dBX software. The spectrographic analysis using the Raven software showed that the trills of coastal individuals have higher minimum frequencies and narrower bandwidths compared to those of the inland. We interpret these results as evidence that Rufous-collared Sparrows living near the surf zone modify their song to avoid the masking effect produced by the low frequency noise generated by sea waves.}

\normaltalk{Evidence of Incipient Song Divergence Between Introduced and Native Populations of the Warbling White-Eye (Zosterops Japonicus) in Hawai'i}{Jesse D Robinson, Caroline Dingle}{Variation in culturally acquired bird songs may play an important role in mediating evolutionary processes. Natural or human-mediated founder events can lead to reduced diversity and complexity of communication signals, potentially impacting the trajectory of evolution. The warbling white-eye (Zosterops japonicus) was introduced from Japan to all major Hawaiian Islands between 1929--1960. This species is ideal for investigating cultural evolution, as territories have expanded within islands, but not between. In the current study, we used playback stimuli to test whether white-eye songs have diverged from the ancestral song type (from Japan), and between the Hawaiian Islands. During playback trials, we presented the primary song of conspecifics as stimuli to individuals on two islands, O'ahu and Hawai'i. Three stimuli were tested -- familiar song (song from the same island) and unfamiliar song (song from the other island, song from Japanese populations). We measured latency of response, approach distance, vocal response, and display behaviors as aggressive responses. We predicted that, if songs have diverged between the populations, individuals would respond stronger to songs from the same island than to songs from the neighboring island or from the native range. Analysis of responses of birds on Hawai'i showed that individuals did respond most strongly to songs from their own island than to unfamiliar songs, suggesting song divergence has occurred since introduction. Future work will examine responses from birds on O'ahu and explore impacts of song variation on the success of white-eyes in Hawai'i.}

\normaltalk{Investigating Female Song and Territoriality of the Endangered Bahama Oriole Across the Annual Cycle}{Michelle Moyer, Sheridan Danquah, Scott Johnson, Shelley Cant, Colin Studds, Kevin Omland}{Given the historical focus on male song, behavioral studies of female song are less prevalent but are critical for determining the function of vocal communication and for fully understanding species' broader ecology. Previous studies have demonstrated that many tropical birds use female song to defend year-round territories. However, we have very little knowledge of the full annual cycle habitat needs of this species, and the pilot data we have are very ambiguous as to whether they maintain year-territories or not. I plan to investigate the context in which female song occurs in the endangered Bahama Oriole, and quantify the birds' habitat usage across the annual cycle. Understanding how and when the birds use habitats throughout the year will allow us to target specific habitat types as conservation priorities and to accurately monitor the population size. The results of this study will allow us to gain valuable insight into the function of female vocal communication in a tropical songbird and to provide recommendations for conservation strategies to the Bahamas National Trust.}

\normaltalk{Do Hummingbirds Change their Flight During Molt?}{Andr\'{e}s F D\'{i}az, Alejandro Rico-Guevara, Carlos D Cadena}{Hummingbirds are well known for their hovering flight, one of the most energetically expensive locomotion modes among animals. Hummingbirds must be capable to sustain their demanding across the year, even during molt. Molt is one of the most expensive events on the annual cycle, in which birds shed and replace their feathers periodically, yet the effects of molt on hummingbird hovering flight are not well known. We assessed whether molting and non-molting hummingbirds differ in their hovering flight by using high speed video analysis of three species (Anthracothorax nigricollis, A.tzacatl and Saucerottia cyanifrons ) from the Andes of Colombia.  We analyzed five molting individuals and nine non-molting individuals in which we measured kinematics for complete wing beats in top and side views. We found that molting hummingbirds tend to open and close their wings in narrower angles, meaning that they carry their wings further back and forth compared to non-molting individuals. In contrast, the frequency, wingtips angle, body angle, tail angle and beak angle remained unchanged between molting and non-molting hummingbirds. Our findings shows that hummingbirds are capable of sustain hovering flight kinematics even under impressive wing area reductions during molt season.}

\normaltalk{Does Myiozetetes Flycarcher Invest more in Egg Volume with Elevation?}{Lauren Rumpel, Tatiane L Silva, Neander M Heming, Miguel  Marini}{Elevational gradients are important for understanding the environmental and ecological drivers of geographic variation in avian life-history strategies. Changes in temperature, pressure, food availability and predation can influence birds reproductive traits. Based on this, we tested whether different altitudes and latitudes affect the egg volume in the Neotropical region. We tested the hypothesis that Social Flycatcher, Rusty-margined Flycatcher and Grey-capped Flycatcher (Myiozetetes similis, M. cayanensis and M. granadensis) invest more in egg volume at higher altitudes and in lower latitudes. The species of the genus Myiozetetes are resident and have closed nest. We used Image J Program to obtain egg volume of 297 eggs from 104 clutches of Myiozetetes. We controled clutch size by using most 3 to 4 eggs' clutches. Elevation was used as a continuous variable from 0 to 3500 meters a.s.l. in the Neotropical region. A mixed linear model showed that egg volume increases with altitude (p = 0.007), and decreased with latitude (p = 0.05). The increase in egg volume at higher altitudes may be related to higher investment in offspring quality, since environmental conditions are more severe than at lower altitudes in the same latitude. Our results supports the hypothesis that altitude influences the life history of birds in the Neotropical region, corroborating Bergmann's rule.}

\normaltalk{Do Birds Use Shared Song Types to Communicate with Neighbours? a Test with a Puerto Rican Endemic, Adelaide's Warbler}{Peter C Mower, Samantha W Krause, David M Logue}{Most songbirds sing repertoires of song types, but the functional significance of repertoires is poorly understood. One influential hypothesis is that birds use shared song types to mediate territorial aggression with their neighbours.  We test novel predictions of this hypothesis with a large dataset of audio recordings and GPS locations from color-banded male Adelaide's warblers. Specifically, we test whether a focal bird's distance to his neighbor's territory and the temporal proximity to a fight with that neighbor can be used to predict whether the focal bird will sing a song type he shares with his neighbor. This study combines bioacoustic and geographic techniques to offer a fresh perspective on the function of song repertoires.}

\normaltalk{Migratory Phenology of the Gray Catbird: A Circular Approach}{Lauren V Puleo, Andrew Boyle, Robert R Fitak, Anna M Forsman}{Avian migration is a cyclical biological event that occurs in synchronization with the seasons. These annual cycles are triggered by many factors including temperature, photoperiod, and resource availability. Understanding how environmental change affects the annual cycles of birds is vital for determining the stability of populations and migration ranges over time. Changes in migratory timing may subject passerines to suboptimal conditions on their breeding and wintering grounds. The Gray Catbird (Dumetella carolinensis) is a North American migratory passerine that consists of two populations with separate migratory routes. Every Fall the Northeastern population migrates to Florida and the Caribbean, whereas the Midwestern population migrates to Central America. Understanding how Gray Catbird migratory behavior is changing under varying environmental conditions is essential for interpreting the future success of this and other species of eastern forest birds that are predicted to be severely affected. For this project, we analyzed bird banding data collected at the Wekiva Basin Bird Banding Station in Altamonte Springs, Florida. Over a 10-year period (2008-2018), 353 Gray Catbirds were banded and morphometric data were collected. Here we present results from analyses of phenological changes in migrating Gray Catbirds using circular statistics and linear regressions. These results are interpreted alongside weight, fat score and climate data. Phenological changes may suggest the need for further research and conservation actions to protect migratory birds from environmental change effects.}

\normaltalk{Migratory Connectivity, Migratory Distance, and Wing Morphology in Eastern Whip-Poor-Wills Across the Midwest}{Aaron Skinner, Michael Ward, James Wright, Ian Souza-Cole, Thomas J Benson, Stephen Matthews, Frank Thompson, Christopher Tonra}{The Eastern Whip-poor-will (Antrostomus vociferus; hereafter Whip-poor-will) has declined by 70\% in recent decades, yet an understanding of when and where populations are limited throughout the annual cycle remains poorly understood. Migration is the period with the highest mortality rate in many migratory bird species, and measuring variation in morphology and migratory ecology helps us understand a population's or a species' potential for adaptation, making it critical for conservation planning. Despite recent interest in the migratory period of the whip-poor-will, small sample sizes and spatial scales have precluded a robust understanding of this critical phase of the annual cycle. In the summer of 2019, we deployed 94 archival GPS tags on Whip-poor-wills in four states across the Midwest. Data from individuals recaptured in summer of 2020 will be added to data from 11 previously recaptured Whip-poor-wills from Ohio. We will identify migratory routes, migratory patterns, calculate migratory connectivity, and determine if morphological measures of wing shape or size can help explain individual migratory distances traveled, migratory patterns at the population level, and migratory connectivity. Migratory connectivity is exceedingly important when predicting which populations of migratory birds are at high risk due to climate change and other pressures like habitat loss. Ultimately, this research will help pinpoint factors behind declines in this enigmatic species, highlight particular populations at risk, and help us understand how migratory birds may adapt in the coming centuries in the face of rapid anthropogenic environmental change.}

\normaltalk{From Tradition to Tragedy: Overcoming the Illegal Trade in Cayman Parrots}{Jane E Haakonsson, Karen Rosenthal}{The Cayman Islands are home to two endemic subspecies of Amazon Parrot: the Grand Cayman Parrot (Amazona leucocephala caymanensis) and the Cayman Brac Parrot (A. l. hesterna). Originally listed as a game bird, the Cayman Parrot was given full protection under local legislation in 1989. However, the ``grandfathering in" of existing captive parrots has made these regulations largely unenforceable. Despite legal protection, illegal trapping for the local pet-trade and persistent illegal hunting as a crop pest continues on Grand Cayman, and to a lesser extent, on Cayman Brac. Appropriate regulation and enforcement of the pet trade is an essential step in protecting Cayman Parrots, especially against a background of continued habitat loss. The parrot amnesty project set out to register all captive birds to allow members of the public a six-month period during which they could legalize their birds. A complimentary health check accompanied the fitting of a uniquely numbered stainless steel leg-band and the implementation of a microchip, enabling enforcement staff to readily identify wild caught parrots and thus enforce the law. A total of 326 birds were registered during the amnesty project across the two islands. Overall health condition, weight, heart and respiratory rate as well as feather samples were obtained for each bird. Genetic material will be explored, particularly in relation to bioinformatics, historic bottlenecks, kinship, effective population size and the genetic diversity represented in the captive population. Location and owner demographics were collected, including DOB, to explore the trajectories of this cultural practice.}

\normaltalk{Developing a Collaborative Motus Network Across the Great Plains and Chihuahuan Desert.}{Matthew M Webb, Maureen Correll, Stu Mackenzie}{Populations of grassland birds are steeply declining in North America. Understanding how these species move and use habitat across their full annual cycle is therefore crucial to their long-term conservation. Innovative technology to gather data on grassland birds during migration is currently lacking in the grassland provinces and states of Canada, the US, and Mexico. Bird Conservancy of the Rockies is working with the Partners in Flight Western Working Group and recruiting partners to develop a collaborative automated radio telemetry research network throughout the Great Plains and Chihuahuan Desert as part of the Motus Wildlife Tracking System. This network can also be used and developed to understand movement of other declining taxa (i.e. invertebrates, bats). We will initiate this collaborative effort by holding webinars to recruit partners and collectively identify research questions and resulting conservation goals. A network in this region will help researchers understand movement and other behavioral attributes across the full annual cycle of declining grassland birds including the Baird's sparrow, Sprague's pipit, and chestnut-collared longspur and apply that knowledge towards landscape-scale conservation of these species.}

\normaltalk{Wintering Ecology of Chestnut-Collared Longspurs in Oklahoma}{John A Muller, Nuwanthika Perera, Jeremy D Ross}{Winter ecology of migratory grassland birds in Oklahoma is poorly understood, despite the region comprising a substantial portion of many species' winter range. Chestnut-collared Longspur (Calcarius ornatus), in particular, exemplifies such cases and are of particular conservation concern after decades of global decline. Using a combination of field surveys, habitat assessments and radio-telemetry we are developing winter occurrence and habitat suitability models of Chestnut-collared Longspur relative to local and landscape features. We established walking line transects at multiple open-grassland sites throughout Oklahoma (including Tallgrass Prairie Preserve, Wichita Mountains NWR and Rita Blanca National Grasslands) to determine occupancy, abundance, and habitat associations of longspurs and their co-occurring species. We then conducted both local and landscape level habitat models, and are using these habitat models to help focus conservation actions for the benefit of the species.}

\normaltalk{Overwinter Survival and Abundance of Seaside and Saltmarsh Sparrows in Southeastern North Carolina}{Evangelyn Buckland, Marae Lindquist, John Carpenter, Raymond Danner}{Little is known about the overwintering populations of Seaside Sparrows (Ammospiza maritima) and Saltmarsh Sparrows (A. caudacuta) along the southern Atlantic Coast, yet populations of both species are undergoing declines across their ranges. Winter seasonal survival probabilities and abundance estimates remain two important unknowns for these species in North Carolina. We are estimating overwinter survival and abundance for both species by using a powerful approach of combining mark-recapture and radio-telemetry. We use radio telemetry data to test the assumption of a closed population and to measure capture rates by determining the proportion of radio tagged birds captured during mist netting sessions. We have been conducting research on Masonboro Island since January 2019, and Fort Fisher State Park since January 2020. To estimate survival, we analyze mark-recapture data with CJS models and radio-telemetry data using known-fate models. To estimate abundance, we are using closed population capture-recapture models. Initial survival analyses indicate that these species can face a disproportionate amount of annual mortality on their wintering grounds. Additionally, initial abundance estimates indicate a greater proportion of Seaside Sparrows to Saltmarsh Sparrows at our study sites. This investigation into the winter population biology of Seaside and Saltmarsh Sparrows contributes important information to be used by land managers and conservationists.}

\normaltalk{The Influence of Pine Management on Grassland-Nesting Birds Overwintering in Central Louisiana, USA}{Nancy M Raginski, Ashley M Long}{Longleaf (Pinus palustris), loblolly (P. taeda), and slash (P. elliottii) pine forests cover ~28 million ha in the southeastern U.S. and are most often used for commercial timber and recreational hunting. Management practices used to reach desired vegetative conditions in these forests (e.g., thinning, prescribed fire) may simultaneously influence distributions of overwintering songbirds, including several species of conservation concern that breed in grasslands of the northern U.S. and throughout Canada (e.g., Henslow's sparrow; Centronyx henslowii). From December to February of 2018--2020, we surveyed 25 study sites to examine the influence of pine management on grassland-nesting birds overwintering in central Louisiana. We recorded 1,012 detections of 17 species associated with grasslands   and detected 3.6--9 times more detections of grassland associated species in longleaf stands compared to slash and loblolly stands.    Within longleaf sites, we recorded twice as many detections of grassland-nesting birds when stands had an herbaceous understory, which were burned within three years of our surveys, versus shrubby understory. Our results also suggest that grassland-nesting species of conservation concern (e.g. Henslow's sparrow and Bachman's sparrow [Peucaea aestivalis]) used sites with greater herbaceous and bare ground cover and less shrub and canopy cover when compared to site-scale vegetation points. Our research will help inform more comprehensive management of grassland-nesting birds overwintering in pine forests of the southeastern U.S.}

\normaltalk{Winter Survival of Baird's and Grasshopper Sparrows in the Marfa Grasslands, Texas}{Mieke Titulaer, Denis J Perez Ordo\~{n}ez, Fabiola Baeza Tarin, Maureen Correll, Erin H Strasser, Russell Martin, Louis A Harveson}{Over the past 40 years, grassland birds have been one of the bird groups with the steepest declines in North America. Baird´s Sparrow (Centronyx bairdii) and Grasshopper Sparrow (Ammodramus savannarum) are two migratory grassland specialists that have lost 70-80\% of their total populations since 1966. During three winters from 2016-2019, we monitored these species in the Marfa grasslands, Texas, to determine their winter survival and limiting factors. During the three winters we placed radio-transmitters on $>$200 individuals of these species to track every individual 1x/day from mid-December to mid-March. We estimated ground cover in a 5 m radius circular plot for = 20 bird points per individual. We placed temperature loggers in 40 bird points and 40 random points to measure microclimate conditions. Winter survival was lower for Grasshopper Sparrow than Baird's Sparrow (three year mean = 65.92\% and 85.54\%, respectively), ranged from 47\% to 100\%, and the main limiting factor was minimum temperature. The minimum temperature was lower in short grass, suggesting that short grass does not offer sufficient protection against cold weather. These are the first demographic estimates of these two sparrows during the winter season in Texas. The data of this site will contribute to better understand the winter ecology of these and other grassland bird species. This information will help managers and researcher to make decisions to sustain thriving grassland bird populations in the region.}

\normaltalk{Effects of Chronic Noise Exposure on Nestling Stress Physiology are Context-Dependent in Tree Swallows (Tachycineta Bicolor)}{Anik P Obomsawin, Marty Leonard}{Anthropogenic noise is increasing in intensity and scope, resulting in changes to acoustic landscapes and largely negative effects on a range of species. In birds, noise can mask acoustic signals used in a variety of communication systems, including parent-offspring communication. As a result, nestling birds raised in noise may have challenges soliciting food from parents and avoiding predators around the nest. Given that passerine nestlings are confined to a nest and therefore cannot escape these challenges, noise may act as a chronic stressor. If noise exposure alters the stress physiology of nestlings, negative effects may persist into adulthood and ultimately reduce reproductive success. In this study, Tree Swallow (Tachycineta bicolor) nestlings were raised in continuous, white noise or in the absence of this added noise. We tested whether noise exposure affected baseline, stress-induced, and integrated feather corticosterone levels, as well as measures of immune function. Given that stress physiology and immune function may vary with the developmental environment, we also examined how effects of noise varied with brood size and nestling size. We found that overall, exposure to noise did not alter nestling stress physiology or immune function. However, light nestlings raised in noise exhibited lower baseline, higher stress-induced, and lower integrated corticosterone than their heavier counterparts. Overall, our findings suggest that noise can alter the stress physiology of developing birds; however, these effects depend on developmental conditions and the presence of other environmental stressors, such as competition. Furthermore, our findings highlight why populations are not uniformly affected by noise.}

\normaltalk{Non-Native Mammal Species Threaten Transboundary Tropical Seabird Hotspot}{Juliana Coffey}{Introductions of species beyond the limit of their native geographical distribution -- particularly terrestrial mammals - has driven extirpations and extinctions of species worldwide, their impacts of which are particularly significant on insular species.  Seabirds are especially susceptible to the adverse effects of introduced mammals to islands, due to the reliance on islands of many species for nesting.  Many seabirds display high philopatry, returning to the same island repeatedly to nest and raise young.   Several types of non-native mammals inhabit remote islands throughout the Grenadines -- an archipelago which hosts globally and regionally significant colonies of breeding seabirds, and which represents one of the few remaining strongholds for seabirds in the entire Lesser Antilles.  This paper presents a contemporary inventory of non-native mammal species on islands in the Grenadines in the context of breeding seabirds and protected areas, and explores the wider implications of complex sociocultural barriers to conservation.}

\normaltalk{Attractiveness Rather than Abundance in the Wild Leads to Selective Parrot Poaching}{Pedro Romero Vidal, Martina Carrete, Guillermo Blanco Hervas, Fernando Hiraldo Cano, Jose L Tella Escobedo}{Parrot poaching remains highly active in the Neotropics. Despite the large decrease and even halt of exports for international trade in recent decades, a high number of parrots are still suffering from domestic illegal trade. Understanding the characteristics, intensity and drivers of domestic poaching is essential to implement effective conservation measures for the species, which many of them are also threatened by other factors. Some studies on the area of wildlife criminology proposed that illegal parrot poaching is an opportunistic crime, with people taking the most abundant species and not showing a preference for the most valuable ones. However, these studies lacked an appropriate statistical approach and reliable information on the abundances in the wild for the species traded. Since 2011, we are conducting large-scale surveys in all the Neotropical region (covering 12 countries) in order to obtain reliable data on species abundance in the wild, also related to habitat changes, and as domestic pets. Our preliminary results show that some species are disproportionally poached according to their abundances in the wild, showing a positive selection by people. In addition, positively selected species seem to have some specific traits, related to size, coloration and the ability to imitate human speech, which makes them also more valuable in the local markets. This positive poaching selection, rather than habitat loss, constitutes the main threat to many Neotropical species and is driving to the verge of extinction at regional and country-level scales several species of amazon parrots and large macaws.}

\normaltalk{Differential Use of the Trophic Resource by Neartic Shorebirds in a Coastal Wetland in the Argentine Patagonia}{Joanna M Castillo, Luis O Bala, Luciana R Musmeci, Mar\'{i}a de los Angeles Hern\'{a}ndez, Rodolfo El\'{i}as}{Knowledge dietary requirements in stopover sites is a key to conservation of shorebirds and the habitats that they use. Intertidal flats of south of South America are important stopover sites for many species of shorebirds. The diet of Hudsonian Godwit (HUGO) and White Rumped Sandpiper (WRS) was evaluated during different stopover dates throughout 2018 and 2019 at Caleta Malaspina, Argentina through the faecal analysis. Frequency of occurrence of prey was calculated and compared. Also, the trophic niche breath and the size-class of the main preys was estimated. In addition, was studied the trophic supply through benthic samples, and was estimated the abundance and diversity of invertebrate community of each date. The shorebirds consumed items representing at least 12 taxa, but only a few taxa comprised the main resources consumed. The intraspecific dietary composition of shorebirds was similar, but interspecific composition was different in all dates. The main prey consumed by HUGO were polychaetes of Nerididae, Orbiniidae and Onuphidae families, while the WRS captured mostly polychaetes of Spionidae family, small crustaceans and clams. Also, the HUGO captured larger polychaetes than the WRS. The density and diversity of invertebrate community was similar among dates. These results show innovative data about the trophic resource use by shorebirds in the study area, and indicate a differential use of food resources in the taxonomic composition and in the size-class of the main items captured by shorebirds during their migratory stopover.}

\normaltalk{Isotopic Study of the Foraging Ecology of Six Cuban Pelecaniforms}{Antonio Rodr\'{i}guez, Martin Acosta, Lourdes Mugica, Carola Sanpera, Lluis Jover}{In the last decade, stable isotopes have confirmed their importance in foraging ecology studies in a wide variety of ecological scales (individual, population, community). We collected chick feathers of six pelecaniform species (Egretta thula, E. caerulea, E. tricolor, Bubulcus ibis, Eudocimus albus and Plegadis falcinellus) in three Cuban localities (South of Pinar del R\'{i}o, Zapata and Birama swamps) for isotopic signature determination of dN15 and dC13. Significative differences were found among dN15signatures, with higher average values for individuals of Egretta. In the case of dC13 signatures, differences were higher for Plegadis falcinellus and Bubulcus ibis indicating predominance of terrestrial preys in their diet. Main differences were found between Birama and South of Pinar del Rio for Bubulcus ibis and species of Egretta. In Birama Swamp the use of different trophic levels and particularly the ingestion of prey from terrestrial environments were more evident. Those similar values obtained for dN15 and dC13 isotopes among species may reflect differential foraging strategies like habitat use --microhabitat segregation, which can lead to niche partitioning to avoid competition. These results provided key elements for an ongoing understanding of the foraging ecology of waterbirds and the creation of an isotopic baseline in the Caribbean; that can improve management programs for their conservation.}

\normaltalk{Habitat Relationships of Forest and Early Successional Bird Communities in Iowa}{Benjamin M West, Stephen J Dinsmore}{Understanding bird-habitat relationships are in areas of high avian diversity may be critical to reversing recent avian population declines. Our study focused on breeding bird communities three Bird Conservation Areas (BCA) in south-central Iowa; these areas are primarily forested but also contain early successional and agricultural habitats. We visited a grid of 493 points twice each breeding season between 2016 and 2019, using point counts with generalized distance sampling (GDS) to estimate species densities at a point scale. For each point we calculated (1) observed richness and (2) an inverse Shannon diversity using GDS density estimates. We then related avian diversity to habitat using a generalized linear model with fifteen habitat variables, ranging in scale from landscape to site-level. We detected a total of 81 diurnal forest, edge, and early successional avian species in our study area, 25 of which were listed as Species of Greatest Conservation (SGCN) need in Iowa. Observed overall species richness at a point ranged from 12 to 38 (mean = 27, SD = 4), and SGCN richness ranged from 2 to 12 (mean = 6, SD = 2).  There were strong relationships between total species richness and leaf litter cover (-), shrub stem density (+), and the proportion of forest in 10 km (+). Our findings affirm these BCAs are important hotspots for Iowa's avian diversity, and that observed habitat relationships have the potential to inform future management of Iowa forests for greater avian diversity.}

\normaltalk{Preliminary Status Report on the Ferruginous Hawk (Buteo Regalis) in Kansas, USA}{Erica L Clark, Medhavi Ambardar, William J Stark}{In Kansas, the Ferruginous Hawk is listed as a Species of Greatest Conservation Need, Tier II. It is the largest Buteo species in North America, occurring in grasslands and nests along bluffs, buttes, rock outcrops and occasionally in isolated trees. A previous study of Ferruginous Hawk nesting in Kansas was conducted from 1979 to 1987 revealed that the most productive nesting territories were isolated from anthropogenic activities. Nesting sites also were inaccessible to predators on rocky ledges and the surrounding landscape comprised over 50\% rangeland. We revisited 68 of the 120 historic nest territories in the summer of 2019. When a nest was located we flew a drone above the nest to determine if it was occupied. During flight, photographs were taken to document the nest contents, including eggs/chicks. Videos were obtained of surrounding habitat for analysis and site comparison to historical land use conditions. Of the 68 territories searched, 10 territories (or 15\%) were occupied and averaged two chicks each. These nest sites were all placed on rocky ledges or columns. As a top predator Ferruginous Hawks are vital to grassland health. It has been suggested that Ferruginous Hawk declines in Kansas have occurred due to heavy extermination of their main prey source, Black-tailed prairie dog (Cynomys ludovicianus), and habitat loss. We will investigate these potential effects to determine best conservation practices for the persistence of breeding populations of Ferruginous Hawk in Kansas, and aid landowners is adopting compatible rangeland management practices.}

\normaltalk{Linking Species Distributions to Environmental Variables:  Range Maps vs. Points of Occurrence}{John T Rotenberry, Priya Balasubramaniam}{Successfully predicting species' shifting distributions under environmental change is predicated on accurately connecting current geographical distribution of a species with the underlying environmental variables presumed to influence that distribution.  Raw species distributional information necessary for making these connections comes in two distinct forms:  points of occurrence (specific geographical coordinates where a species has been observed), and expert-prepared range maps (polygons of varying complexity encompassing a species expected occurrence).  Each form has potential shortcomings:  range maps may overestimate the true occurrence of a species whereas occurrence points may underestimate it.  Previous comparisons of the two have focused on how they differ when estimating species richness; less attention has been paid to whether the two differ in their representation of a species' environmental associations.  We evaluated potential differences using globally distributed avian orders Accipitriformes and Galliformes (~500 species). For each species we overlaid range maps (IUCN) and point-of-occurrence data (GBIF) on global maps of four climate variables (WorldClim2) and elevation (GTOPO30) at a resolution of 10'.  Overall, median values of all 5 environmental variables were highly correlated.  We also acquired species' elevational mid-points (mid-point between minimum and maximum elevational extent) from the literature; median elevations from point occurrences and ranges were consistently lower ($>$420 m) than mid-points.  We concluded that occurrence points likely produce better estimates of underlying environmental variables than range maps, although differences were often slight, and that elevational range mid-points were biased high, and that elevation distributions based on either points or range maps provided better estimates.}

\normaltalk{Influence of Fire Diversity and Territory Size on Florida Scrub-Jay Fitness}{Meredith L Heather, Kara L Lefevre, Reed Bowman}{Florida Scrub-Jays rely on frequently burned oak-dominated scrub, preferring habitat approximately five to fifteen years postfire. Natural scrub fires often burn patchy, creating a mosaic of vegetation structure. Florida Scrub-Jays have relatively large territories relative to other species in the genus Aphelocoma. Small territories may have a higher risk of burning completely, creating homogenous extremely low and open habitat for several years that could displace jays or lead to lower fitness. Larger territories may be more likely to contain unburned patches, retaining structural heterogeneity that offer post-fire refugia. Florida Scrub-Jays from large natal territories have higher fitness than those from smaller territories. Using historic data from Archbold Biological Station, we analyzed how territory size related to fire diversity and which influenced jay fitness more (adult breeder survival and fledglings produced). Annual data from 2008-2018 included territory area, number of fires and area burned per fire, proportion of territory burned in the last nine years, fire diversity, number of fledglings, and breeder survival. To describe habitat heterogeneity, we used the Shannon Index to calculate diversity of patches with different fire histories within territories. Larger territories had significantly more fires and greater diversity of patches with different fire histories (Linear regression, adj r2=0.10, p $<$ 0.001). The number of fledglings was best explained by proportion of habitat patches in early post-fire succession ($<$9 yrs, p=0.008), but not territory size.   Breeder survival was best associated with number of fires (marginally significant, p=0.06) but not territory size or fire diversity.}

\normaltalk{Effects of Sea Level Rise on Nesting Populations of Colonial Birds in Barnegat Bay: 1976 to 2018}{Joanna Burger, Michael Gochfeld}{We monitored population levels, habitat use, and concentrations of heavy metal levels in colonially-nesting birds in Barnegat Bay since 1976.  Colony sites were monitored several times/year.   The number of nesting pairs (and colony sites) of Common Terns, Forster's Terns, Black Skimmers, Laughing Gulls, and Herring Gulls decreased significantly, although reproductive success did not decrease.  Great Black-backed Gulls increased markedly partly because they are significantly larger and arrive on the nesting colonies earlier; they outcompete smaller species of gulls and terns.  Great Egrets increased as nesting birds in Barnegat Bay, while Snowy Egrets and others egrets and night-herons have declined.  The most notable change in Barnegat Bay is a loss of nesting islands, and degradation of some of the remaining habitat, due mainly to sea level rise.  More than half of the islands used by Common Terns have disappeared or degraded.  Sea level rise is greater in the northeastern Atlantic coast than in other parts of the world.  Many islands have disappeared or become sub-optimal for nesting.  Whether an island is useful for nesting depends upon the species.  The habitat for species, such as terns, that nest on low-lying islands disappeared first, and these islands eventually were inundated and destroyed.  Shrub-nesting species (egrets) have not lost nesting habitat.  Except for mercury, most metals have declined over this same time period, indicating they are not likely to contribute to the overall declines.}

\normaltalk{Wintering Home Range Size Association with Grass Cover: Baird's and Grasshopper Sparrows Savannarum in the Chihuahuan Desert of Mexico}{Alexander Pe\~{n}a-Peniche, Irene Ruvalcaba-Ortega, Octavio Rojas-Soto, Ricardo Canales-del-Castillo, Erin H Strasser, Jos\'{e} H Mart\'{i}nez-Guerrero, Arvind O Panjabi}{Baird's and grasshopper sparrows are migratory and grassland-specialist birds of conservation concern; both species' populations have been declining for almost 60 years. Fine-scale spatial use of their wintering habitat is unknown, despite being fundamental to their long-term viability. We characterized home ranges for both species during three wintering seasons (2014-2017) in three sites in the Chihuahuan Desert of Mexico. We deployed tags on 292 individuals and located them daily using radio-telemetry. We determined home ranges using the 95\% of probability occurrence of the Autocorrelated Kernel Density Estimation. We did not find a significant difference in home range area between C. bairdii (Mean=10.2 ha; n=140) and A. savannarum (Mean= 8.4 ha; n=152; p=0.1). We sampled 11'202 vegetation 5 m-radius plots on the study area. In each plot we visually estimated percentage of ground coverage in five classes (grass, forb, shrub, bare, and other); as well as the cover (\%) of the three main grass genera, and mean vegetation height. We used GLM to test extrinsic (habitat variables) and intrinsic (sex and body mass) factors as predictor of home range. Mean grass (p=0.002) and forb cover (p=0.003) inversely correlated to Baird's sparrow home ranges.  Mean grass cover (\%) was also inversely correlated to grasshopper sparrow's home ranges (p$<$0.001). Higher grass and forbs cover are related to a higher availability of required resources (e.g. seeds and cover from predators and cold). The results of this study on the area configuration and spatial use of habitat characteristics can be critical to design successful conservation programs for declining species.}

\normaltalk{Salmonella Smugglers: Investigating the Role of  Migratory Birds in the Spread of Salmonella}{Julia Nadeau-Gneckow, Anna M Forsman}{Wild birds are regular carriers of enteropathogenic bacteria, including subspecies of Salmonella enterica, that affect the health of domestic poultry and humans. Salmonella usually colonizes asymptomatically in the avian gut and is shed for a variable period of time thereafter. The asymptomatic nature of Salmonellosis allows hosts to migrate relatively unhindered, despite infection. Through extensive literature review, I have characterized rates of Salmonella infection in migratory birds and their role in dispersing bacteria across long distances. Although rates of salmonellosis in wild birds are low, individuals sampled in close proximity to farms tend to have higher rates of infection. Untethered by borders, migratory birds have potential to serve as long-range vectors for zoonotic diseases like salmonellosis. For example, in the United States, birds migrate along the North Atlantic flyway, which encompasses multiple cattle-producing regions. Birds visiting different farms as they migrate may infect farm animals or contract infections present at the stop-over farm. Because Salmonella spreads through fecal-oral transmission, infections may spread when wild birds gain access to grain silos, water troughs, or pastures and defecate in these areas. The growing centralization of agriculture, coupled with industrial farming's overuse of antibiotics, is contributing to the global spread of zoonoses. My work discusses the importance of monitoring the role of  migratory birds in the dissemination of enteropathogenic bacteria as well as the cyclical nature of infection between domestic flocks and wild birds.}

\normaltalk{Evaluation of NRCS Cover Crop Programs as Habitat for Wintering, Migrating, and Nesting Birds in Tennessee}{Brittany Panos, David A Buehler, Craig Harper}{The U. S. Natural Resources Conservation Service administers the winter cover crop program to provide financial incentives to agricultural producers to sow herbaceous plant seeds to protect agricultural fields from soil erosion during the non-growing season (late fall through spring).  In Tennessee, the hectares of cover crops have increased to $>$100,000 ha in 2019.  Although benefits related to soil retention and water quality improvements have been documented,  potential benefits related to avian wildlife use remain largely unknown. We are providing an in-depth examination of the use of cover crop fields by birds during the stationary non-breeding period, during migration, and during the breeding season..  We are comparing the use of cover crop fields with no-till row-crop fields without cover crops. We selected a set of 80 fields with cover crops and 20 control fields without cover crops for evaluation across two counties in middle and two counties in western Tennessee. Avian use is monitored along two 100-m line transects in each field in a distance sampling framework every three weeks during the January-June sampling period.  Supplemental drive netting with mist nets and banding is used to further quantify avian use.  We are nest-searching during the breeding season for focal grassland species to document the role these fields may play in supporting breeding activity.  Results and preliminary conclusions from the first field season will be summarized and presented.}

\normaltalk{Evaluation of Rice Cultivation as Habitat for Birdlife in the Department of Tolima (Colombia)}{Elizabeth T\'{e}llez D\'{i}az, Miguel Á Quimbayo Cardona, Luis A Castilla Lozano}{Rice cultivation, seen as an artificial wetland, provides a passageway for migratory birds and shorebirds in Colombia. Crop production practices and its ecological role as a provider of habitat for birds in two conventional irrigation systems and swimming pools, during a productive period, and their relationship in different stages of the production cycle were evaluated. This study was carried out in the lower part of the hydrographic basin of the Salda\~{n}a river, Santa In\'{e}s and Pueblo Nuevo districts (Salda\~{n}a, Tolima - Colombia). The samplings had two localities and each locality had between 80 and 100 ha; Linear transects with counting points with minimum distances of 200 m between points were established, the observations were made within a radius of 50 m per point, between March and August in 2018. During 200 hours of observation, 15.141 individuals were recorded, distributed in 58 species, belonging to 17 orders and 28 families. Out of those, 53 were found in conventional irrigation and 46 in swimming pools. The families with the highest number of species were Ardeidae and Tyrannidae; with 7 species each. The most abundant species was Phimosus infuscatus with 4,797 individuals followed by Vanellus chilensis with 2,569 specimens. The relative abundances of species regarding the conventional irrigation system were recorded: vegetative phase: 14.14\%; flowering: 5.25\%; panicle: 5.06\%; harvest: 17.66\% and soil adaptation: 57.88\%. respecting the pool system, the vegetative phase: 58.514\%, flowering: 7.43\%; panicle: 5.28 \%\%; harvest: 0.88\% and soil adaptation: 58.14\%.}

\normaltalk{Bird Community Composition and Structure in Agroecosysyems from Tabasco, M\'{e}xico}{Mariana M Cordova, Juan Manuel K Gonzales, Stefan Louis A Weiss, Johannes Cornelis V der Wal}{In southeastern Mexico, deforestation and modification of large forest areas have been one of the most obvious causes of landscape alteration. Studying these relatively new agroecosystems allows us to understand how bird friendly they are. The objectives of this study were to contribute to the knowledge of bird communities composition and temporal structure in the productive landscape in Tabasco, Mexico. Five agroecosystems were selected: forest, cacao plantation, orchard, cornfield and pasture. Birds were sampled using point-counts in each habitat, which were visited once each season - dry, rainy and winter. We recorded a total of 1,539 birds from 89 species. The most diverse habitat was the forest, followed by the cacao plantation, pasture, cornfield and orchard. Bird diversity variated through the seasons, being winter the most diverse period due the migratory bird arrival. Bird community assemblage differed among all agroecosystems, being the cacao plantation the habitat most similar with the forest. Thus, even altered habitats can play an important role in the persistence of birds, as some species seem to respond differently to the degree of habitat fragmentation and transformation. Highly disturbed habitats such as pasture and annual crops tend to have less diversity due to a shortage of shelter, food and perch options. Agroecosystems are, to a lesser or greater extent, habitats for the avifauna, depending on the vegetation structure in these sites and in their surroundings. The pressure on the bird community can be attenuated by favoring characteristics of the productive landscape that are compatible with bird conservation.}

\normaltalk{Are Invasive Fruits Nutritional Traps for Birds? a Synthesis of Patterns in the Northeastern U.S.}{Susan S Pagano, Jenifer Rosete, Victoria Kwasinski, Erica Delles}{Wild fruits are widely utilized as a food resource by songbirds in the eastern U.S., and many species exhibit seasonal shifts to fruit--eating at times that correspond to seasonal fruit availability. Fruits can be variable in terms of their nutritional content with many native shrubs in this region providing energy-dense and calorific berries that are preferred by migrating birds. However, the introduction of non-native fruiting plants presents a novel food for birds, and there may be ecological consequences if the fruits are of lesser quality than natives that they outcompete. We summarize data collected over the past decade on the nutritional content of native and invasive fruits that are available to birds in late-summer and fall in western New York and the surrounding region. We also analyzed native and invasive buckthorns, and exotic honeysuckle fruits for phytochemicals that may have physiological and biological effects in avian consumers. We found a trend toward lower energy density and fat content in invasive fruits compared to native fall-ripening fruits, and this pattern parallels migratory bird preferences for native fruits in the fall. Further, the presence of anthraquinones, notably emodin, in the fruits of invasive buckthorn may be pertinent for the diets of migratory birds refueling at important stopover sites where it occurs. Thus, examining of the biochemical and nutritional composition of fruits may provide insight into their overall quality as a food resource during migration and other stressful times for songbirds.}

\normaltalk{Foraging Behavior and Efficiency of Reddish Egret in Las Salinas, Ci\'{e}naga De Zapata, Cuba}{Daniela Ventura, Alieny Gonzalez, Alejandro Rodr\'{i}guez-Ochoa, Lourdes Mugica, Mart\'{i}n Acosta}{The ecology of Reddish Egret is poorly understood within the Eastern Management Unit; a situation that is especially true for Cuban populations. Our project aims to improve the knowledge about this species' ecology, a necessary step in developing conservation strategies. The temporary characterization of the Reddish Egret's foraging behavior and success were studied by focusing on variations with age, group size, color morph, and water depths. Foraging individuals (n = 193) were video-recorded in the foraging habitat of Las Salinas, Ci\'{e}naga de Zapata, Cuba, from 2017 to 2019. In all seasons, stand and wait, waking and running were the most employed behaviors, with a frequency higher than two events per minute. Differences between seasons were only detected for active behaviors like open-wing feeding and canopy feeding. All behaviors were more frequently used in shallow water, with the exception of foot-stirring and open-wing feeding, which were used more frequently in deep water. Other behaviors showed differences between age (running) and group size (running, open-wing feeding and canopy feeding). The more specialized foraging behaviors of hopping, canopy feeding and open-wing feeding had the highest success rate (50\%, 37\%, and 35\%, respectively), but were employed infrequently. Regarding the foraging efficiency, the highest values were found during the pre-breeding season (32.82 \%), which may be related to the high energy requirements this period demands. Age and season were the most important predictive variables that explained the foraging efficiency of Reddish Egret.La ecolog\'{i}a de Egretta rufescens est\'{a} poco estudiada dentro de la subunidad de manejo este, situaci\'{o}n aún m\'{a}s evidente en las poblaciones cubanas. Este trabajo contribuye a aumentar el conocimiento sobre la ecolog\'{i}a de forrajeo de esta especie de h\'{a}bitos especialistas, necesarios para implementar efectivas estrategias de conservaci\'{o}n. El trabajo incluye la caracterizaci\'{o}n temporal de la conducta y \'{e}xito de forrajeo de Egretta rufescens en el h\'{a}bitat de forrajeo de Las Salinas. Se analiz\'{o} la variaci\'{o}n de ambos aspectos entre morfos, clases etarias, estatus social y profundidades de agua. Para ello se filmaron 193 individuos durante el periodo 2017-2019. En las cuatro etapas las conductas de forrajeo m\'{a}s empleadas fueron observar, caminar y correr, con una frecuencia superior a 2 eventos por minuto. Solo se detectaron variaciones entre temporadas para las conductas r\'{a}pidas como correr con alas abiertas y abrir las alas. La mayor\'{i}a de las conductas fueron m\'{a}s usadas en profundidades bajas, excepto mover la pata y correr con las abiertas, que fueron m\'{a}s empleadas en profundidad alta. Otras conductas fueron diferentes entre edades (correr) y status social (correr, correr con alas abiertas y abrir las alas). Volar, abrir las alas y correr con las abiertas fueron conductas muy exitosas (50\%, 37\% y 35\% respectivamente), pero infrecuentemente usadas. En cuanto a la eficiencia, los mayores valores se registraron en la etapa pre-reproductiva (32.82 \%), lo cual podr\'{i}a relacionarse con la alta demanda desde el punto de vista energ\'{e}tico que representa este momento del ciclo de vida. La edad y la temporada fueron las variables m\'{a}s importantes para predecir la eficiencia de forrajeo de E. rufescens.}

\normaltalk{"INFLUENCE of LANDSCAPE COMPOSITION on the FEEDING HABITS of the ANDEAN CONDOR (Vultur Gryphus) in CHILE"}{Fernanda A Salvo, Constanza N Rodr\'{i}guez, Melanie N Duclos, Fabi\'{a}n N Jaksic, Veronica N Quirici}{The accelerated growth of the human population has led to significant changes in landscapes, forcing many species to change their territories in search of new sources of food. Scavenger birds, as the Andean condor (Vultur gryphus), can adjust their home range, patterns of movement and eating behavior according to food sources availability. Therefore, the objective of this study was to evaluate the relationship between the diet composition of the Andean condor and different landscape compositions in three areas with contrasting levels of anthropization in Chile. We selected three group of variables: (i) Andean condor diet composition through the analysis of pellets from nine roosts, (ii) characterization of coverage and land uses, and (iii) areas with the presence of prey, all variables within three spatial scales (30, 75 and 120km) around each roost. Our results identified 14 Native prey species, six domesticated Introduced, two feral, and eight anthropogenic waste items consumed by the Andean condor. Our results showed significant correlations between diet composition of the Andean condor and landscape characteristics, showing more significant differences in larger scale within the buffers (120km). Therefore, we conclude that the diet composition of the Andean condor is associated with landscapes composition in its maximum potential area of use, mainly those modified by man, which are related to the availability of different food sources. This study provides a better understanding of the use of space by the Andean condor, relevant key points for future management and conservation strategies.El crecimiento acelerado de la poblaci\'{o}n humana ha llevado a cambios significativos en los paisajes, obligando a muchas especies a cambiar sus territorios en busca de nuevas fuentes de alimentos. Las aves carro\~{n}eras, como el c\'{o}ndor andino (Vultur gryphus), pueden ajustar su rango de hogar, patrones de movimiento y comportamiento de alimentaci\'{o}n según la disponibilidad de fuentes de alimentos. Por lo tanto, el objetivo de este estudio fue evaluar la relaci\'{o}n entre la composici\'{o}n de la dieta del c\'{o}ndor andino y las diferentes composiciones del paisaje en tres \'{a}reas con niveles contrastantes de antropizaci\'{o}n en Chile. Seleccionamos tres grupos de variables: (i) composici\'{o}n de la dieta del c\'{o}ndor andino a trav\'{e}s del an\'{a}lisis de egagr\'{o}pilas de nueve dormideros, (ii) caracterizaci\'{o}n de la cobertura y usos del suelo, y (iii) \'{a}reas con presencia de presas, todas las variables dentro de tres escalas espaciales (30, 75 y 120 km) alrededor de cada dormidero. Nuestros resultados identificaron 14 especies de presas nativas, seis de Introducidos domesticados, dos Introducidos asilvestrados y ocho de desechos antropog\'{e}nicos consumidos por el c\'{o}ndor andino. Nuestros resultados mostraron correlaciones significativas entre la composici\'{o}n de la dieta del c\'{o}ndor andino y las caracter\'{i}sticas del paisaje, mostrando diferencias m\'{a}s significativas en mayor escala dentro de los buffers (120 km). Por lo tanto, concluimos que la composici\'{o}n de la dieta del c\'{o}ndor andino est\'{a} asociada con la composici\'{o}n del paisaje en su m\'{a}xima \'{a}rea potencial de uso, principalmente las modificadas por el hombre, que est\'{a}n relacionadas con la disponibilidad de diferentes fuentes de alimentos. Este estudio proporciona una mejor comprensi\'{o}n del uso del espacio por parte del c\'{o}ndor andino, puntos clave relevantes para futuras estrategias de manejo y conservaci\'{o}n.}

\normaltalk{The Birds and the Bees: Pollination and Nectar-Robbing on at-Risk Plants in a Novel Ecosystem}{Pryce W Millikin, Samuel B Case, Corey E Tarwater}{The extinction of native species and introduction of non-native species often leads to the disruption of mutualisms, threatening mutualistic partners and the ecological services they provide. Pollination is a mutualism between animals and flowering plants and is critical for community structure and the maintenance of biodiversity. Introduced nectarivores have the potential to either fill the role of extinct pollinators or to take nectar without pollination (nectar robbing). In the Hawaiian Islands, the majority of pollinator birds have gone extinct following human arrival, resulting in many native plants relying on introduced nectarivores for pollination. Nevertheless, it is unclear whether introduced birds and bees are having a positive or negative effect on native plants. In the northern Waianae Range of Oahu, we deployed camera traps (n = 34) and conducted in-person observations (n = 40) on three critically endangered plant species in the Lobeliad family adapted for bird pollination. In 67\% of interactions (n = 1054) detected, animals robbed nectar from plants. Surprisingly, most of the nectar robbing was done by native birds (86\%) and the remainder was by introduced birds. We recorded 98 pollination events, of which 71\% were performed by bees and 5\% were performed by birds. Our results indicate that species interactions in a novel ecosystem can be unexpected, with bees substituting roles of extinct birds in pollination and native birds imposing negative impacts on native plants. Thus, we emphasize the importance of examining community-level interactions in managing populations of at-risk species in novel ecosystems.}

\normaltalk{"Effects of drought on the avian community on the short grass prairie"}{Angelina Canney, Claire Ramos, Clark Jones}{Climate change affects organisms on every level, from the biosphere on down to the individual. In the Southwest, climate change is predicted to increase the frequency and severity of droughts.  Precipitation in a dry environment can be the difference between organisms having a successful breeding season and an unsuccessful one. In 2018, southwestern Colorado endured an extremely dry spring in the short grass prairie, while in contrast, 2019, had above average precipitation. This allowed for a natural experiment to investigate the impacts of precipitation on avian communities in the shortgrass prairie. We conducted point counts at different areas in the same vicinity on the shortgrass prairie, to gain a snapshot of the avian biodiversity. We predicted that the drought year will show less diverse avian communities than the wet year. This research will allow us to build predictions for how climate change may impact avian communities in the shortgrass prairie going forward.}

\normaltalk{Phenotypic and Genomic Divergence Despite Recent Speciation in Seedeaters (Sporophila)}{Tram N Nguyen, M\'{a}rcio Repenning, Carla Suertegaray Fontana, Leonardo Campagna}{Recent speciation events can allow researchers to associate phenotypes with their causal genotypes. Finding recent species that diverge in traits of interest (e.g., relevant reproductive isolation and speciation) is therefore crucial to our ability to search for the genetic basis of such phenotypes. Here we investigate the genetic underpinnings of the phenotypic differences between a pair of species from the genus Sporophila, seedeaters that are thought to have speciated very recently. Once thought to be the same species, Sporophila plumbea and Sporophila beltoni have only recently (2013) been distinguished as sister species based on morphology, plumage pattern, breeding habits and song variation. Most notably, these species differ in the songs males use to attract females and defend territories and in bill morphology -- second year males of S. beltoni develop yellow bills while males of S. plumbea have black bills. Preliminary genetic studies using ddRADseq markers could not unambiguously differentiate these taxa; however, whole-genome sequences from 36 individuals (18 of each species) reveal 40 SNPs of high divergence encapsulated in just three narrow divergence peaks (average length of 60 kb) along the genome. These differentiated regions harbor six genes, one of which is a well-known pigmentation gene (EDN3 which encodes the protein Endothelin-3), which controls skin coloration in chickens. This study contributes to the growing body of literature indicating that genes involved in traits related to prezygotic isolation are important in the early stages of speciation.}

\normaltalk{Phylogeography and Species Limits of the Troupial Icterus Icterus Species Complex}{Layse A Lucas, Maria W Pil, Leilton W Luna, Luciano N Naka}{Delimiting species boundaries and understanding the distribution of independent evolutionary units within a species complex can shed light into the evolutionary history and help unravel the history of the landscape. With that in mind, we studied the genetic structure of the Neotropical troupials, a species complex formed by three allopatric species that inhabit the dry Neotropical lowlands. We used DNA sequencing of Ultra-conserved elements (UCE) to sample 21 individuals within the complex across its range. Based on 1,500 loci, and ~1,000 independent single nucleotide polymorphisms (SNPs), we identified four morphologically distinct genomic clusters, including i) Icterus icterus, from the W Colombian/Venezuelan lowlands; ii) Icterus croconotus, from western South American dry lowlands and part of W and C Amazonia; iii) Icterus jamacaii, from NE Brazil; and iv) an isolated population of I. croconotus, from the Roraima-Rupununi savannas. Our results provide evidence for the existence of an undescribed lineage in Roraima, and indicate historical genetic introgression between I. jamacaii and I. croconotus in C Brazil. In addition, paleoclimate species distribution modelling of Icterus croconotus suggests that the disjunct fragments of dry forest were more connected in the recent past, during the middle Holocene. These models predict greater habitat suitability in the Amazon region in present-day models, suggesting that the occurrence of I. croconotus in the Amazonian lowlands may be a recent historical event. These results suggest a complex history in this group, but also highlight the potential effect of recent climatic oscillations on current distribution patterns.}

\normaltalk{The Evolution of Nest Building Behavior in Ploceidae Weaverbirds}{Bobby Habig}{A major goal of evolutionary biology is to explain the emergence and interspecific diversity of complex traits, including behaviors. One strategy for illuminating our understanding of behavioral evolution is to conduct comparative studies of animal architecture. Extended phenotypes, and particularly the nests of weaverbirds, are the result of behavior, and yet have all the practical benefits of morphological traits. These features vary between species, apparently shaped by selective pressures that yield specific predictions. In this study, I use a molecular phylogeny and ancestral state reconstruction to map the evolution of nest construction in the Family Ploceidae. To explore the evolution of nest morphology in relation to nest building, mating system, and social organization, I also created separate models of ancestral state reconstruction and used transition rates between states to test the correlated evolution of discrete characters. The results of this comparative study thus address questions such as how rapidly certain nest structural features evolve, which features are ancestral versus derived, which tend to exhibit phylogenetic signal, and which evolve in response to environmental features.}

\normaltalk{Taxonomy of the Rusty-Margined Flycatcher (Myiozetetes Cayanensis Linnaeus, 1776) (Tyrannidae) Reveals an Endemic Species from Brazilian Atlantic Forest}{Juan Camilo R\'{i}os-Orjuela, Lu\'{i}s F\'{a}bio Silveira}{The genus Myiozetetes Sclater, 1859 (Aves: Tyrannidae) comprises four recognized species, molecularly and morphologically supported as a monophyletic group and are widely distributed in the neotropics. Myiozetetes cayanensis (Linnaeus, 1776) is one of the most common species and mostly distributed within the genus, having populations isolated by strong geographical barriers, which could lead to speciation events. Thus, today M. cayanensis complex is composed by four subspecies and not much is known about its phylogenetic status since the last contributions date back to the 1910s. Traditionally the boundaries between these taxa have been based on size and subtle plumage differences, being necessary a detailed analysis of morphology, plumage, and behavior to clarify the subspecies boundaries that exist within the group. As a first approximation to determine the valid species of the complex, a museum specimen-based morphological variation analysis of 31 body measures was performed, obtaining information of their external variation. Our data show that there is a significant morphological difference between the four currently recognized subspecies. There is wide morphological variation between two subspecies/populations that inhabits further biogeographically, while morphological variation is clinal between populations that are not isolated geographically. Also, M. c. erythropterus is the most diverge subspecies within the group, being a candidate for categorization as an independent species, which is endemic to the dry enclaves in the SE Brazilian Atlantic forest. It is necessary to complement these results using ecological and vocal data, to support this variation more consistently.  El g\'{e}nero Myiozetetes Sclater, 1859 (Aves: Tyrannidae) comprende cuatro especies reconocidas, sustentadas molecular y morfol\'{o}gicamente como un grupo monofil\'{e}tico y est\'{a}n ampliamente distribuidas en el neotr\'{o}pico. Myiozetetes cayanensis (Linnaeus, 1776) es una de las especies m\'{a}s comunes y mayormente distribuidas dentro del g\'{e}nero, teniendo poblaciones aisladas por fuertes barreras geogr\'{a}ficas, lo cual podr\'{i}a derivar en eventos de especiaci\'{o}n. De esta forma, en la actualidad el complejo M. cayanensis est\'{a} compuesto por cuatro subespecies, de las cuales no se conocen muchos datos sobre su estatus filogen\'{e}tico ya que las ultimas contribuciones datan de la d\'{e}cada de 1910. Tradicionalmente los l\'{i}mites entre estos taxones se han basado en el tama\~{n}o y en las diferencias sutiles en plumaje, siendo necesario un an\'{a}lisis detallado de la morfolog\'{i}a, el plumaje y el comportamiento que permita esclarecer los l\'{i}mites entre las subespecies que existen dentro del grupo.  Como una primera aproximaci\'{o}n para determinar las especies v\'{a}lidas del complejo, se realiz\'{o} un an\'{a}lisis de variaci\'{o}n morfol\'{o}gica basado en espec\'{i}menes de museo, usando 31 medidas corporales, obteniendo informaci\'{o}n de su variaci\'{o}n externa. Nuestros datos muestran que hay una diferencia morfol\'{o}gica significativa entre las cuatro subespecies actualmente reconocidas. Existe una variaci\'{o}n morfol\'{o}gica amplia entre dos subespecies/poblaciones que est\'{a}n m\'{a}s alejadas biogeogr\'{a}ficamente, mientras que la variaci\'{o}n morfol\'{o}gica es clinal entre las poblaciones que no est\'{a}n aisladas geogr\'{a}ficamente. Adicionalmente, M. c. erythropterus es la subespecie que m\'{a}s diverge dentro del grupo, siendo candidata a la categorizaci\'{o}n como especie independiente, la cual es end\'{e}mica de los enclaves secos del sureste de la mata atl\'{a}ntica brasilera. Es necesario complementar estos resultados con datos ecol\'{o}gicos y vocales, que permitan sustentar esta variaci\'{o}n m\'{a}s consistentemente.}

\normaltalk{Searching for the Genomic Basis of Differentiation in Plumage Patterning Within the Rufous-Collared Sparrow (Zonotrichia Capensis).}{Pablo D Lavinia, Leonardo Campagna, Ana S Barreira, Stephen C Lougheed, Pablo L Tubaro, Dar\'{i}o A Lijtmaer}{High-throughput sequencing is allowing researchers to unveil the genomic basis behind plumage color differentiation. The Rufous-collared Sparrow (Zonotrichia capensis) shows phenotypic variation throughout its distribution. In particular, Z. c. australis from southern Chile and Argentina differentiates from other subspecies by lacking the characteristic black lateral crownstripes (i.e., head uniformly grey or with only subtle traces of black). We sequenced whole genomes of 18 individuals (9 representing Z. c. australis and 9 from northcentral Argentina possessing the black crownstripes) to investigate the genomic changes behind this phenotypic differentiation. Based on the analysis of ~ 11 million SNPs, we detected three divergence peaks that contrast with a background of low genomic differentiation (FST = 0.015 $<$U+F0B1$>$ 0.018) between Z. c. australis and the other individuals. Differentiation at these three peaks combined (1,350 SNPs) was remarkably higher (FST = 0.29 $<$U+F0B1$>$ 0.09). However, 90\% of outlier SNPs (211/235 SNPs with an FST $<$U+F0B3$>$ 0.8) and virtually all fixed differences (51/53 SNPs with an FST = 1) were located within a single ~200-kb region mapped to chromosome 5. Outlier SNPs were concentrated within and downstream the Suppression of tumorigenicity 5 gene (ST5), which is involved in cytoskeletal organization and tumorigenicity in humans but has no characterized function in birds. Our results point out to ST5 as a gene of large effect controlling phenotypic differences between Z. c. australis and other subspecies, proposing this region with no previous association with pigmentation as a candidate gene for plumage color patterning.}

\normaltalk{Evaluating Sexual Dimorphism in Long-Eared Owl (Asio Otus) Plumage}{Emily V Griffith, Chris Neri, Nova MacKentley, Alec R Lindsay}{Sexual dimorphism in plumage coloration is a topic of particular interest because of its unique implications about the life history of a species. Owls (order: Strigiformes) are largely considered to have sexually monomorphic plumage.  However, Long-Eared Owls (Asio otus) are reported (Holt et al. 2016) to have a slight sexual dimorphism in their plumage which can be used to correctly assign sex of individual owls captured in the field. This study tested the efficacy of using those plumage characters for sex-assignment of individuals captured during spring migration at Whitefish Point Bird Observatory (Paradise, MI) by confirming the sex of captured individual owls using DNA-based tests of sex in the lab. Further, this study investigates additional dimorphic features of Long-Eared Owl plumage, including the potential role of UV-reflective pigments.}

\normaltalk{Unexpected Plumage Congruence with DNA and Song Evidence for Species Groups in the Hooded Pitta Complex}{Pamela C Rasmussen, Per G Ericson, Yanhua Qu, Martin Irestedt, Mozes Blom, Paul Sullivan, Frank Lambert, Frank E Rheindt}{The Hooded Pitta Pitta sordida has long been treated as a single polytypic species occurring from the western Himalayas through New Guinea, until recently being split by some based on plumage and song into three species: western sordida (tentatively including north Sulawesi forsteni on plumage alone); eastern novaeguineae; and Biak rosenbergii. DNA analyses showed two main clades, western and eastern, but that forsteni is sister to other taxa in the eastern clade. PCAs show seven plumage clusters on PC1 and PC2. Our in-depth plumage analyses group forsteni with the eastern clade, despite its resembling the western clade in overall duller plumage, and show that forsteni has previously unrecognized plumage autapomorphies. On the basis of our analyses and historical evidence, we show that the type of Pitta persola Brodkorb, originally thought from Palawan, is instead a Sulawesi forsteni. In the western clade, abbotti of the Nicobars is sister to all other taxa, and is the most distinct on plumage and song.  However, the next divergence, between Philippine region populations vs. the Himalayan brown-headed cucullata plus black-headed Sundaland populations, shows less congruence with plumage patterns. Five PCA clusters based on multiple plumage characters are congruent with other evidence for species status, while the others are based only on crown color, the most obvious plumage difference but one that does not reflect genetic or vocal divergence. Based on integrative analyses, we advocate recognition of three species (forsteni, rosenbergii, and novaeguineae) in the eastern clade, and two (sordida and abbotti) in the western clade.}

\normaltalk{"Melodia-Us" Morphometrics: Appraisal of Geographic Variation in Song Sparrow Beak Morphology and Its Connection to Song Characteristics}{Zena Casteel}{Song is an acoustic signal used for mate attraction and/or territorial defense across animal taxa. Yet despite serving very similar functions between diverse lineages, song is incredibly complex and variable, and the causes of this signal divergence are poorly known. While primarily under sexual-selection, song can be inadvertently modified or constrained by natural selection. In birds, beak morphology is known to be under intense selective pressure for foraging and/or thermoregulation, but changes in beak morphology have the potential to alter song structure, a major component of mate choice in many passerines. Song Sparrows breed from the Eastern seaboard to the Pacific Coast of the United States, south onto the Mexican Plateau and north into Canada. Not habitat specialists, they experience a huge range of climatic and ecological conditions across this range, and unsurprisingly are one of the most polytypic passerines in the world. Using geometric morphometrics on museum specimens, I analyzed variation in bill size and shape between Song Sparrows from five different regions across the United States. Bill morphology changed in multiple ways between groups, including both depth and protuberance of the upper mandible, and curvature of the ventral edge of the lower mandible. Allometric effects were also found, with beaks becoming shallower and straighter as they increased in size, and birds with the largest beaks (the Alaskan/Aleutian group) were significantly different from all other groups. This study provides a more complete picture of geographical variation in Song Sparrow bills than was previously available using linear measurements alone.}

\normaltalk{Pantropical gene flow in three booby species}{Danny Jackson, Erik Funk, Vicki Friesen, Dave Anderson, Tammy Steeves, Jamie Morris-Pocock, Scott Taylor}{What functions as a barrier to gene flow in pantropical seabirds? Previous work in the red-footed, masked, and brown boobies demonstrate population structure across the Isthmus of Panama, the Benguela Current, and, to a lesser extent, the Pacific Basin for all three species. The red-footed booby has shown to exhibit structure across Indonesia. It has been unclear what phylogenetic relationships each of these intraspecies populations has with each other, and if gene flow has occurred across these barriers after the initial colonization of these spaces. Here, we present analyses of several whole genome sequences across the ranges of these species, and discuss patterns of shared genetic variation that imply histories of gene flow around these barriers.}

\normaltalk{Evaluating the Trade-Off Between Investing in Coloration and Vocalizations in Finches (Fringillidae), a Passerine Group of Complex Signals}{Agustin I Casale, Pablo L Tubaro, Dar\'{i}o A Lijtmaer}{The interaction between multiple secondary sexual characters simultaneously expressed by an organism is one of the less understood aspects of sexual selection. In fact, different interpretations of the role of multiple simultaneous signals predict that the correlation of their complexity across species should be positive, negative or absent, three possibilities that were supported by the study of different avian groups. In this context, we analyzed the interaction between color complexity and song elaboration in the passerine family Fringillidae. We used a two-scale approach and studied this association both at the family level, by comparing the songs between the species with the most and the least complex color in each genus, and within the genus Crithagra, which shows gradual variation in color across its species. At both scales, we analyzed the association between color complexity, established by analyzing the number of plumage color patches in each species and their conspicuity, and three aspects of vocal elaboration: energy investment (song length and rate of syllable production), song complexity (repertoire index) and vocal performance (vocal deviation). A total of 1,342 songs from 449 individuals of 81 species were analyzed. We found an absence of association between the complexity of coloration and the three evaluated aspects of song elaboration at both scales. Our analysis also indicates that finches can have a remarkable vocal performance compared to other groups of birds. Taken together, our results suggest that visual and acoustic signals in this family could have different roles in the context of sexual selection.}

\normaltalk{Effects of sexual dimorphism on species birds composition in agricolas zones of Chile´s Norte Chico}{Mariangeles Petit, Angeline Bertin, Nicolas Gouin, Cristian Celis}{Sexual selection is an evolutionary force responsible for the most dramatic traits found in nature. Evidence indicates that it can be an important determinant of species survival in altered habitats. While arid zones are highly impacted by global change, our knowledge of how sexual selection influences the ability of species to cope with habitat alteration in arid ecosystems is limited. Here, we investigated whether landscape characteristics influence the presence of bird species with sexual dimorphism in an agricultural watershed of North-Central Chile. To do that, we analyzed bird communities in 26 riparian sites along the Limar\'{i} watershed and investigated whether landscape characteristics and biotic interactions influenced the relative frequency of bird species with sexual dimorphism. Our results show that the proportion of bird species with sexual dimorphism highly varied from one site to another, and was positively associated with the proportion of natural forest present in the neighborhood. In addition, it was negatively influenced by the abundance of the four most common species. Altogether our results suggest that species with sexual dimorphism could be less able to cope with habitat alteration and be sensitive to competitive interactions. }

\normaltalk{Conservative Plumage Masks Extraordinary Phylogenetic Diversity in the Grallaria Rufula (Rufous Antpitta) Complex of the Humid Andes}{R T Chesser, Morton L Isler, Andres M Cuervo, C D Cadena, Spencer C Galen, Laura M Bergner, Robert C Fleischer, Gustavo A Bravo, Daniel F Lane, Peter A Hosner}{The Grallaria rufula complex is currently considered to consist of two species, G. rufula (Rufous Antpitta) and G. blakei (Chestnut Antpitta). However, it has been frequently suggested that it comprises a suite of vocally distinct yet morphologically cryptic species. We sequenced nuclear and mitochondrial DNA for 80 individuals from across the distribution of the complex to determine the extent of genetic variation between and within described taxa. Our results revealed substantial genetic divergence within the rufula complex, and that G. rufocinerea (Bicolored Antpitta) was nested within the complex. Neither G. rufula nor G. blakei as presently defined was monophyletic. Although six of the seven recognized subspecies of G. rufula were monophyletic, several subspecies contained substantial genetic differentiation. Genetic variation was largely partitioned across recognized geographic barriers, especially across deep river valleys in Peru and Colombia. The G. rufula complex seems unusually diverse even among birds of the humid Andes, a prime location for cryptic speciation.}

\normaltalk{How Effective is the Birdsbesafe® Cat Collar at Reducing Bird Kills by Domestic Cats?}{Margaret B Jensen, Susan K Willson, Abby N Powell}{The global decline of songbird populations is a well-recognized conservation issue. Domestic cats are estimated to kill approximately 2.4 billion birds each year in the United States alone -- more than most other anthropogenic threats combined. As many pet owners are reluctant to keep their cats inside, collar-mounted anti-predation devices for domestic cats may be an important conservation tool. We examined the efficacy of the BirdsBeSafe® collar cover (BCC), a sleeve of brightly patterned fabric worn over a typical breakaway collar. The bright colors of the BCC are intended to alert potential prey to the cat's presence. By combining data from two studies in New York (2014 and 2019) and one in Florida (2019), all of which used similar methods, we tested the hypothesis that the BCC effectively reduces avian mortality caused by cats across ages, sexes, and locations. We tested 92 cats over twelve (NY 2014) or eight (FL and NY 2019) week periods during the bird breeding season; cats alternated two-week periods with the collar on and off. Across studies, the cats in our study killed 2.7 times fewer birds with the BCC than without (p=0.000). There was no significant difference in the efficacy of the BCC for cats of different ages, sexes, or locations. Our results suggest that the BCC can be a valuable tool in songbird conservation.}

\normaltalk{The Hidden Threat of Pesticides and the Role of Wetlands in Mitigating Swallow Declines in Agricultural Landscapes}{Christy Morrissey, Andrew Elgin, Egina Malaj, Robert Clark}{Intensive agriculture is associated with high pesticide and agrochemical use, reduced landscape complexity, and drainage or degradation of wetlands and other natural areas.  Modern agricultural practices are increasingly a threat to aerial insectivores that forage and breed in these habitats. In the Canadian Prairies, the most agro-intensive region of Canada, we have found evidence of landscape simplification and a concurrent increase in pesticide use over the past 35 years, notably neonicotinoids and other insecticides.  Based on an intensive 2 year survey of wetlands across the region, the most frequently detected insecticides were the neonicotinoids, thiamethoxam and clothianidin, at levels known to be toxic to insects and seven other insecticides were also detected. Tree swallows and other aerial insectivores routinely use these wetland habitats; 100\% of tree swallow blood samples as well as 14\% of nestling bolus samples contained neonicotinoids. Increasing agricultural intensification and insecticide use may be a hidden threat that is contributing to population declines.  However, we highlight some of the exciting new research that indicates the important role of intact wetlands and buffers in supporting aerial insectivores in Prairie landscapes. Thus, retaining and restoring wetlands and natural areas as a form of ``ecological intensification" represents a promising solution to biodiversity conservation in intensively cropped regions of North America.}

\normaltalk{Carcass Removal Time in Wind Farms on Neotropical Grasslands}{gonzalo daniele, Igor Berkunsky, Ayelen Lutz, Mariano LAstra}{Impact of wind farms on birds and bats has been poorly studied in Neotropical grasslands. During the last 5 years, more than 100 wind farms has been approved In Argentina. The Buenos Aires province is the most important region of wind energy production where more than 49 wind farms are currently operating and/or under construction. The rate of carcasses removal is a necessary input to estimate the frequency of collision. In this study we conducted a field experiment to estimate the carcasses removal and to identify the main scavengers. We estimate the survival time of two different birds and two small mammals. Between 35\% and 48\% of carcasses were removed in the first 10 h, and less than 15\% survive more than 30 h. At least two birds (Crested Caracara and Chimango Caracara) and two mammals (Hairy armadillo and Pampas Fox) were documented removing carcasses. Our results highlight the importance of consider the rate of removal on the detection of bird and bat collisions in wind farms of temperate Neotropical grasslands}

\normaltalk{What can Ornithologists Tell Us About Global Insect Decline?}{Eliza M Grames, David L Wagner, Chris S Elphick}{A global decline of insect populations would have a devastating impact on birds. Despite urgent calls for more research on insect declines, however, there has been no comprehensive analysis of insect population and biodiversity trends worldwide. Compiling the necessary data is, in part, hampered by its being scattered across disciplines. EntoGEM is a community-driven evidence synthesis project that aims to identify all studies and datasets, published and unpublished, that document long-term trends in insect populations. Many studies of food availability for birds have the potential to be a rich source of insect abundance data that have yet to be analyzed in the context of insect decline. To assess the scale of this untapped potential, we identified studies from the EntoGEM project where the primary focus of the study was birds and compared them to a random subset of non-bird studies in EntoGEM. We analyzed bibliometric relationships within and between the bird and non-bird subsets based on shared citations. We found minimal overlap in citations between the subsets, indicating that information within each discipline is relatively isolated, despite ample evidence for the existence of long-term insect datasets in studies of bird diets and food availability. Here, we identify for which taxa and regions there are likely to be long-term insect data collected by ornithologists. We also highlight some of the bird studies that the EntoGEM project has shown to contain long-term insect population data that have not been included in current discourse surrounding the global insect decline phenomenon.    /////    La disminuci\'{o}n global de insectos tendr\'{i}a un impacto devastador en las aves del mundo. Sin embargo, a pesar del llamado urgente a investigar la disminuci\'{o}n de insectos, no ha habido un an\'{a}lisis exhaustivo de sus poblaciones y sus tendencias en cuanto a biodiversidad a nivel global. EntoGEM es un proyecto de s\'{i}ntesis de evidencia que tiene como objetivo identificar todos los estudios y conjuntos de datos publicados y no publicados sobre las tendencias de los insectos a largo plazo. Múltiples estudios sobre la disponibilidad de alimentos para las aves tienen el potencial de ser una gran fuente de datos sobre abundancia de insectos, que aún no se han analizado en el contexto de la disminuci\'{o}n de insectos. Para evaluar la escala de este potencial sin explotar, analizamos las relaciones bibliom\'{e}tricas dentro y entre los estudios que se centraron en aves y los que no, en la base de datos EntoGEM. Encontramos una superposici\'{o}n m\'{i}nima en las citas entre disciplinas, a pesar de la amplia evidencia de la existencia de conjuntos de datos de insectos a largo plazo en estudios de dietas de aves y disponibilidad de alimentos. Identificamos los taxones y regiones en los que probablemente existen conjuntos de datos a largo plazo. Adem\'{a}s, destacamos algunos de los estudios de aves que el proyecto EntoGEM ha demostrado que contienen datos de poblaci\'{o}n de insectos a largo plazo que no se han incluido en el discurso actual sobre la disminuci\'{o}n global de insectos.}

\normaltalk{A Conservation and Economic Analysis of Birdwatchers Versus Non-Birdwatchers in Pennsylvania}{Ty S Sharrow, Valorie Titus}{In recent years due to decreased hunting participation along with the growing need for conservation action, there is a growing desire for supplemental funding from alternative sources towards environmental protection.  Birdwatchers may serve as such a source.  Before strategies for management and funding can be conceived, the economic impact that birders have, and what conservation attitudes they hold must be understood.  Pennsylvania can serve as a case study to understand such a relationship.  This research shows that Pennsylvania birders have a direct positive influence upon bird populations and existing conservation practices.  It also shows the extent of how Pennsylvania birders engage in the economy concerning their outdoor recreational involvement.}

\normaltalk{Effect of Forest Management on the Density and Abundance of Eight Dry Forest Specialist Bird Species of Northeastern Brazil}{Jonathan R Ribeiro, Flor Maria G Las-Casas, Weber Ge Silva, Luciano N Naka}{The Caatinga, the largest nucleous of seasonally dry Neotropical forests, is undergoing high rates of deforestation and habitat degradation, mostly due to illegal logging. Brazilian legislation encourages forest management to reduce illegal over-exploration of wood resources. The impacts of forest management schemes on biodiversity, however, remain largely untested. This study aims to evaluate the effects of forest management on the populations of eight dry forest avian species. We conducted distance-based sampling along transects to estimate bird abundance and density in managed and unmanaged forest patches in a privately-owned area at the Chapada do Araripe, in northeastern Brazil. We found that most species were negatively affected by forest management. Five species were less common in managed areas, and three presented significant density declines. The globally Vulnerable Cear\'{a} Leaftosser (Sclerurus cearensis) disappeared altogether from logged areas. Conversely, the Caatinga Antshrike (Thamnophilus capistratus), showed higher abundance and density in managed areas, a potential ecological winner following management. These abundance and density changes are likely related to changes in the structure of the forest, as indicated by the positive relationship between these variables and a known NDVI index. A successful forest management protocol should allow species recovery after regeneration. Our results indicate that 12 years of forest recovery were not enough to ensure the return of affected species to managed areas. Current management models may cause some species to go locally extinct, indicating the importance of maintaining non-managed plots and forest corridors to allow the survival and potential recovery of species in management areas.A Caatinga, o maior núcleo das florestas sazonalmente secas Neotropicais, vem sofrendo com desmatamento e degradação de habitats, principalmente devido à exploração ilegal de madeira. A legislação brasileira incentiva o manejo florestal para reduzir a exploração ilegal de recursos madeireiros. Os impactos do manejo florestal na biodiversidade, entretanto, são pouco conhecidos. Este estudo tem como objetivo avaliar os efeitos do manejo florestal nas populações de oito esp\'{e}cies de aves de floresta seca. Realizamos distance-sampling ao longo de transectos para estimar a abundância e a densidade de aves em fragmentos de florestas manejadas e não-manejadas em uma propriedade privada na Chapada do Araripe, no nordeste do Brasil. Constatamos que a maioria das esp\'{e}cies foi afetada negativamente pelo manejo florestal. Cinco esp\'{e}cies foram menos comuns e três foram menos densas em \'{a}reas manejadas. O vulner\'{a}vel Sclerurus cearensis desapareceu das \'{a}reas manejadas. Entretanto, Thamnophilus capistratus mostrou maior abundância e densidade nas \'{a}reas manejadas, sendo um potencial vencedor ecol\'{o}gico p\'{o}s-manejo. Essas mudanças na abundância e densidade possivelmente estão relacionadas à estrutura da floresta, como indicado pela relação positiva entre essas vari\'{a}veis e o NDVI. Um protocolo de manejo florestal bem-sucedido deve permitir a recuperação de esp\'{e}cies ap\'{o}s a regeneração. Nossos resultados indicam que 12 anos de recuperação florestal não foram suficientes para garantir o retorno das esp\'{e}cies afetadas às \'{a}reas manejadas. Os atuais modelos de manejo podem produzir extinções locais, indicando a importância de manter parcelas não-manejadas e corredores florestais para permitir a sobrevivência e a recuperação potencial de esp\'{e}cies em \'{a}reas de manejo.}

\normaltalk{Range-Wide Population Structure and Genetic Variation in Seaside Sparrows (Ammospiza Maritima) using a Next-Generation Sequencing Approach}{Kathryn EC Davis, Amie E Settlecowski, Carolyn Enloe, Thomas Virzi, Margaret E Hunter, Stefan Woltmann, Sabrina S Taylor}{As an obligate salt marsh species, Seaside Sparrows are highly vulnerable to threats including climate change, coastal erosion, sea level rise, and both natural and anthropogenic disasters. Of the nine recognized subspecies, two are extinct and one is endangered. However, genetic analyses of mitochondrial DNA (mtDNA) and microsatellite loci showed that current taxonomy does not accurately reflect underlying genetic diversity. Specifically, they identified (1) a lack of connectivity between Gulf and Atlantic populations, (2) called into question the distinction of the two subspecies found on the northern Gulf coast of Florida, and (3) revealed a potentially distinct population near Charleston, South Carolina. Given the importance of species, subspecies, and population designations in conservation frameworks (e.g. the Endangered Species Act), it is vital that we improve our understanding of genetic structure throughout the species' entire range. To this end, we used restriction site associated DNA sequencing to obtain several thousand single-nucleotide polymorphisms in 292 individuals from 24 range-wide sampling locations. From the resulting data, we will estimate genetic differentiation by calculating pairwise FST values and identify genetic clusters using principal component analyses and a Bayesian clustering approach. We expect our analysis of thousands of loci distributed across the genome to support the previous findings from microsatellite and mtDNA data and to reveal additional fine-scale genetic structure and variation. This robust analysis of genetic structure can then be used to identify and prioritize the conservation of unique populations of Seaside Sparrows.}

\normaltalk{Distribution and Habitat Preferences of Cercomacroides Antbirds (Thamnophilidae) in Western Amazonian Brazil}{Thiago VV Costa, Gabriel A Leite, Thiago O Laranjeiras, Alexandre Aleixo, Christian B Andretti, Claudeir F Vargas, Mario Cohn-Haft}{Three species of Cercomacroides antbirds occur in western Amazonia, apparently segregating either geographically or ecologically. All three occupy dense understory in or at the edges of forest. The Black Antbird (C. serva) inhabits terra firme forests throughout the region, in sympatry with one or the other remaining species, which in turn inhabit floodplain environments. The Blackish Antbird (C. nigrescens) is in v\'{a}rzea forest along the main rivers of the upper Amazon; the Riparian Antbird (C. fuscicauda) occupies shrubby vegetation along the middle and upper reaches of western Amazonian tributaries and is not known to co-occur with C. nigrescens. Our new records of C. fuscicauda, however, extend its known distribution considerably toward that of C. nigrescens, raising the question of whether the two come into contact, and if so how they interact, or whether habitat differences are sufficient to explain their lack of geographic overlap. Geographic replacement in other species pairs between the upper and lower reaches of the Juru\'{a} and Purus rivers, in the absence of any clear geographical barriers, is a known pattern as yet defying ready explanation. Interestingly, east of the Rio Madeira where C. serva is absent, C. nigrescens occurs in terra firme as well as v\'{a}rzea, suggesting release from competition or genetically distinct populations. Our results demonstrate how field data continue to improve accuracy of species distribution and habitat descriptions and call attention to the complex interplay between current interspecific interactions and historical diversification events that remain to be disentangled.}

\normaltalk{Delimiting Areas of Endemism of the Birds of Tropical Asia.}{Zheng Oong, Sushma Reddy}{Commonality in species distribution patterns implies possible shared histories between the species that were driven by the same geological, ecological, or evolutionary processes. In biogeographic analyses, areas of endemism are essential, because they help to identify common factors driving the non-random congruence of species in a given region. Tropical Asia has been recognized as a unique zoogeographical region since the time of Wallace, yet there has been no quantitative and spatial assessment of areas of endemism for the entire region. Past molecular phylogenetic and phylogeographic studies have tested the effects of biogeographic barriers, such as the Isthmus of Kra or the Pleistocene savanna corridors of Sundaland, on species distributions and diversification. The results, however, have failed to uncover a consistent pattern with respect to these barriers, highlighting the need for broad-scale assessments of areas of endemism and the establishment of zoogeographic boundaries based on species distribution. Using a global database of species distribution, we examined the diversity patterns across a broad sampling of avian groups with substantial presence in tropical Asia and identified areas of endemism for the regional avifauna using a grid cell-based approach. We assessed the congruence of boundaries of the areas of endemism with past geological events and/or major habitat transitions. Our analysis attempts to establish these quantitatively characterized areas of endemism as hypotheses of biogeographic boundaries, which can later be tested for their roles in diversification patterns in the region.}

\normaltalk{Ornithological surveys of the Selva de Florencia National Natural Park (Saman\'{a}, Caldas, Colombia): new records and distribution extensions}{Daniela G\'{o}mez, Kelly Orozco, Andr\'{e}s F Cardona, Milton Pineda, Mary L Bedoya, David Ocampo}{The Selva de Florencia National Natural Park is one of the last fragments of the Andean rain forest in the Central Cordillera of Colombia. With the purpose of updating the state of knowledge of the park's avifauna, we collected information obtained through monitoring carried out by park officials since 2012 and records of two expeditions made between 2017 and 2018. We recorded 357 species, of which 11 are endemic and 20 are in some threat category. We report 18 new records for the department and extensions in the distribution range for 12 of these species. The contact between the assemblies of montane forest species with those from the Middle Magdalena valley, added to the good state of conservation of the park and the regeneration of ecosystems, largely due to the restricted access generated by the armed conflict, could explain the great richness of bird species in the area (˜20\% of the Colombian species in less than 0.01 \% of its territory). This new information contributes to the knowledge of the bird distribution of in the Central Andes and stresses the importance of the Selva de Florencia National Natural Park as a priority area for the study and conservation of biodiversity.El Parque Nacional Natural Selva de Florencia es considerado uno de los últimos fragmentos de bosque pluvial andino de la cordillera Central de Colombia. Con el prop\'{o}sito de actualizar el estado del conocimiento de la avifauna del Parque recopilamos informaci\'{o}n obtenida mediante monitoreos realizados desde 2012 y registros de dos expediciones realizadas entre 2017 y 2018. Registramos 357 especies, de las cuales 11 son end\'{e}micas y 20 tienen alguna categor\'{i}a de amenaza. Reportamos 18 nuevas especies para el departamento, con ampliaciones en el rango de su distribuci\'{o}n para 12 de ellas. El contacto de ensamblajes de especies de bosques montanos con los de especies de influencia del valle del Magdalena medio, sumado al buen estado de conservaci\'{o}n y regeneraci\'{o}n de los ecosistemas, en gran medida producto del restringido acceso generado por el conflicto armado, podr\'{i}an explicar la gran riqueza de especies de aves en la zona (˜20\% de las especies de Colombia en menos del 0.01\% del territorio). Esta nueva informaci\'{o}n contribuye al conocimiento de la distribuci\'{o}n de la avifauna en los Andes Centrales y reafirman al PNN Selva de Florencia como una zona prioritaria para el estudio y la conservaci\'{o}n de la biodiversidad.}

\normaltalk{Precipitation Seasonality Drives Bird Species Composition in Southwestern Atlantic Forest}{Laura F Torezan, Luiz dos Anjos}{The Atlantic Forest, which is under intense pressure due to habitat loss and climate change, is composed by 15 ecoregions. The Upper Paran\'{a} Atlantic Forest (UPAF) covers the southwestern portion of the Atlantic Forest and is the largest ecoregion, spanning 471,204 Km2. Although the avifauna of UPAF has been known through several papers, a broader assessment of the entire region was never performed. We assessed bird diversity of UPAF to analyze if precipitation seasonality is correlated to species composition in this ecoregion. Bird species occurrence data were from citizen-science datasets (EOD - eBird Observation Dataset and Xeno-canto) obtained in GBIF.org. Grid cells (0.5° x 0.5°) were taken as sample units. A species richness map was elaborated, and the dissimilarities in species composition among grid cells were assessed with NMDS. The envfit function was used to test if precipitation seasonality (measured both by number of dry months and by precipitation coefficient of variation) is correlated to species composition. A total of 722 bird species were recorded, and the southern region of UPAF accounted for the greatest species richness. A significant correlation between species composition dissimilarities and precipitation seasonality was found both for number of dry months (p = 0.001, R2 = 0.515) and coefficient of variation (p = 0.001, R2 = 0.607). UPAF is likely to undergo an intensification of precipitation seasonality due to climate change, therefore bird species conservation in this ecoregion should take climate trends into account.}

\normaltalk{Evidence of Niche Conservatism among Pairs of Allopatric Avian Taxa in 2 Neotropical Dry Forests, the Caatinga and the Chaco}{Lays V de Freitas, Luciano N Naka}{Adaptation, or the adjustment of organisms to their environment, represents a key feature for the survival and establishment of biological populations. Seasonally Dry Forests (SDF), located along the South American Dry Diagonal, represent a natural laboratory to understand the influence of ecological conditions on the evolutionary process of species. Numerous species of different biological groups (animals and plants) have allopatric populations in the Caatinga and the Chaco, the two main blocks of Neotropical dry forests. Do these allopatric pairs of populations/taxa represent cases of niche evolution, and as such may be close to completing the speciation process, or represent cases of niche conservatism, and are more likely to have undergone non-adaptive mutations while in isolation? Aiming to answer this question, we tested hypotheses of divergence versus niche conservatism in 12 pairs of avian populations/taxa with allopatric distributions in these two dry forests. We used ordination technics and niche models to evaluate the degree of niche differentiation between these pairs. Our results indicated that in most cases (9/12), there is strong evidence of niche conservatism among allopatric populations. These results suggest that the current pattern of distribution of these populations fits with the natural tolerance of the species, and do not represent cases of ecological differentiation. We hypothesize that the differences observed in the phenotype and genotype of the isolated populations are possibly the result of neutral changes and not the direct result of adaptations to the environmental characteristics of both ecosystems.}

\normaltalk{Distribution Extension of Gallinago Undulata (Boddaert, 1783) (Birds: Scolopacidae) and Biogeographic Implications}{Marina FdA Maximiano, João Marcos G Capurucho}{The Giant Snipe, Gallinago undulata (Boddaert, 1783), is a species with disjunct distribution, being found both in the savannas of northern South America and in the enclaves of open vegetation in the southeast of the continent. Two subspecies are recognized that are isolated by Amazonia, G. undulata undulata (Boddaert, 1783), which occupies the northern distribution of the species, and G. u. gigantea (Temminck, 1826), in the southeast distribution of the species. The subspecies differ in relation to their size and weight, the latter being noticeably larger than the former. Despite presenting a very characteristic vocalization, the species is difficult to register in the field, which may be due to its nocturnal habits, being popularly known as one of the ``ghosts of ornithology". We present the first record of the species for Amazonas State, Brazil, made during a field expedition to the Mapinguari National Park, located in southern Amazonia. It fills part of the distribution gap between the two subspecies. Comparing morphological and molecular data with available voucher specimens, we aimed to relate the collected individual collected to populations of the species. Our results indicate that despite the greater proximity to the population of G. u. gigantea, the specimen is morphologically similar to G. u. undulata. These results add new information in the known biogeographic patterns of the species that will be useful in future studies and makes us question what is known about the biology of the species, which according to the literature is one of the few species of the genus that is not migratory.}

\normaltalk{Cryptic Genetic Diversity and Cytonuclear Discordance Characterize Contact among Canada Jay (Perisoreus Canadensis) Morphotypes in Western North America}{Brendan Graham, Carla Cicero, Dan Strickland, John Woods, Howard Coneybeare, Kimberly M Dohms, Ildiko Szabo, Theresa Burg}{Three recognized Canada Jay (Perisoreus canadensis) morphotypes with distinct plumage traits come into contact in western North America. Recent work demonstrated high genetic structure across the species' range; however, genetic variation in these contact zones remains unknown. We combined plumage and genetic data to determine whether differences in those traits correspond to where the morphotypes come into contact, and to assess the extent of hybridization between morphotypes. We categorized 605 individuals into one of the four morphotypes (Pacific, Rocky Mountain, Boreal, and Hybrid) based on plumage, and genotyped individuals at the mtDNA control region and 12 microsatellite loci to examine phenotypic and genetic structure within and outside of the contact zones. Our data showed cryptic genetic diversity and high cytonuclear discordance among morphotypes in the contact zones , which is  likely the result of recent and historical admixture. The Boreal and Pacific morphotypes both showed strong associations with genetic breaks, whereas the Rocky Mountain morphotype exhibited higher genetic diversity and was associated with multiple genotypes. Our work provides greater insight into plumage and genetic variation for Canada Jays as well as the distribution of the three morphotypes. Further, our analyses show the importance of combining plumage and genetic traits to examine contact zones within species and between closely related taxa.}

\normaltalk{Low Interannual Site Fidelity for Wintering Yellow Rails}{Christopher J Butler, Bailey Kephart, Jennifer Wilson}{Yellow Rails (Coturnicops noveboracensis) winter primarily in coastal prairies and wet grasslands from Texas to North Carolina, although small numbers winter inland north to Oklahoma. Between-year winter site fidelity of this species has not previously been reported, so we analyzed the recapture histories of Yellow Rails banded in southeastern Oklahoma and coastal Texas from 2007 to 2019. We banded 108 Yellow Rails in Oklahoma and 446 Yellow Rails in Texas during this time. Although Yellow Rails were routinely recaptured within-season, only five individuals were recaptured between years. All recaptured birds were in Texas. The apparent low interannual site fidelity (1.1\% in Texas, 0\% in Oklahoma) is similar to that reported for other grassland species wintering in grasslands in this area (e.g., LeConte's Sparrow, Henslow's Sparrow, Sedge Wren), and we speculate that low site fidelity could potentially be advantageous for species that use early successional habitats.}

\normaltalk{Estimating Non-Breeding Grassland Birds Population in Managed Ranches in West Texas, USA}{Alejandro Ch\'{a}vez-Trevi\~{n}o, Mieke Titulaer, Maureen Correl, Carlos E Gonzalez, Louis A Harveson}{Grassland-obligate bird species (GOBS) are a group of specialized birds that live in pristine grasslands. In North America, these species are declining dramatically. Habitat changes such as shrub encroachment is one of the causes for GOBS declines.   We used a paired treatment-control design to conduct bird surveys in the winter of 2018-2019 in shrub invaded areas that will be treated to remove woody vegetation, and control areas containing open pristine grassland (control). Sites are located in Marfa and Marathon Texas. The objectives of this work are to: compare bird populations between the two habitat types, and to start a long-term bird monitoring program restored sites.   We performed 30 (800m) transects per strata from January 1st to February 28th 2019 and 2020. We estimated pooled density of grassland birds using distance sampling. We considered priority grassland species in the analysis. Results show 66 birds per km2 in the control site, and less than one bird in the treatment site in Marfa, and 176 birds per km2 and 1 bird per km2 for control and shrub encroached sites in Marathon, respectively. After the shrub removal treatment takes effect we expect to see an increase in grassland bird abundance on shrub encroached sites.}

\normaltalk{Grassland-Breeding Bird Abundance in the National Capital Region}{Alexis C Rickert, Elizabeth L Tymkiw, Zachary S Ladin, Hannah G Redmond, W G Shriver}{Grassland-breeding birds are declining throughout North America due to afforestation, fragmentation, and conversion of natural habitat to human dominated landscapes. The anthropogenic factors that are disrupting the habitat of these specialist species can be detrimental to the integrity of their breeding grounds, thus reducing species abundance causing declining population trends. National Parks are federally protected lands that can provide the necessary habitat to support grassland birds, even when surrounded by an urbanized environment. We used data collected from the Inventory and Monitoring Program to estimate the abundance of five grassland and early successional breeding bird species (Field Sparrow, Grasshopper Sparrow, Prairie Warbler, Eastern Meadowlark, and Bobolink) from four National Parks (Manassas National Battlefield Park, Monocacy National Battlefield Park, Antietam National Battlefield, and Harpers Ferry National Historical Park) in the National Capital Region. We surveyed 242 point counts randomly located across the four parks each breeding season from 2014 -- 2017. We used the `gmultmix' function in the R package `unmarked' to estimate detection adjusted abundance (individuals / point). Average abundance for Eastern Meadowlark (5.03 +/- 2.46 SD) and Prairie Warbler (3.59 +/- 1.11 SD) was greatest at Manassas across all years. We detected Grasshopper Sparrows at all parks and the average abundance was consistent across years. Bobolinks were the least abundant across all parks and only detected at Monocacy in 2016.  Through this ecological monitoring program, we can demonstrate the importance of these federally protected lands for the conservation of species in urbanizing environments.}

\normaltalk{Are Anthropogenic Disturbances Leading to Homogenization of the Boreal Landbird Community in the Oil Sands Area of Alberta, Canada?}{Judith D Toms, Tazarve Gharajehdaghipoor, Megan Edgar, Thea M Carpenter}{In recent decades, the boreal biome in northern Alberta has been increasingly impacted by anthropogenic disturbances arising from multiple sectors, including forestry, agriculture, energy development and rural-urban development. Although the direct area disturbed is relatively small, (e.g., only 14\% of the oil sands area (OSA)), native habitats have become highly fragmented by these disturbances (e.g., 63\% of the OSA is within 200m of an anthropogenic disturbance) and associated changes have been well documented in many boreal landbird populations. We tested the hypothesis that the boreal landbird community is undergoing a process of biotic homogenization due to the loss of specialist species and an increase in generalist species tolerant of anthropogenic disturbances, using multiple analytical approaches. The uniqueness of bird community composition declined as the total amount of anthropogenic disturbance increased. Using ordination analyses, we quantitatively demonstrated not only that the bird community composition shifts in a predictable way as the amount of anthropogenic disturbance increases, but also that disturbed communities are more similar to each other than are undisturbed communities. Thus, our analyses support the hypothesis of biotic homogenization, even though the total area disturbed in this landscape is quite low (14\%). Further analyses are underway to determine the relative importance of different types of anthropogenic disturbances on the homogenization of the boreal landbird community.}

\normaltalk{Occurrence Of Red-masked Parakeet (Psittacara Erythrogenys) In Urban Environments Of Guayaquil, Ecuador}{Josu\'{e} Vizh\~{n}ay, Paolo Piedrahita}{The Red-masked Parakeet (Psittacara erythrogenys) is an endemic bird from the Tumbesian region. This bird is distributed from the coastal dry regions in western Ecuador to Lombayeque in north-western Perú. This species is threatened basically by illegal trafficking, deforestation and urban development. Data about the presence of this species in urban areas is poorly known.  The cities from the coast where this species has been reported are Guayaquil, Santa Elena, Playas and Manab\'{i}. Reports suggest this species is using parks and other green areas along the cities. The aim of this study is to contribute with more information about the occurrence of P. erythrogenys in urban environments. We monitored this species along nine parks in Guayaquil between May and June of 2018 in the morning at 6:30 -- 9:00 and during the afternoon at 16:30 -- 17:30. Our findings suggest that occurrence of this species in two of the main parks monitored is recurrent.  During the monitoring activities we saw this bird stay less time in the morning than in the afternoon. The activities which we registered during the presence of these birds in the parks were: feeding, preening, resting and perching, but we didn´t measure how much time the birds did those activities. In conclusion we think that this information could suggest that some parks or green areas of Guayaquil city harbour a small population of red-masked parakeet and it could be an important report to the authorities in charge of planning conservation strategies in the city.}

\normaltalk{Evaluation of Avifauna in an Important Coastal Lagoon in Northern Puerto Rico}{Naomi G Mercado-Velez, Adriana Bonilla-Crespo, Adrianne G Tossas-Cavalliery}{From November 2017 to February 2020 we assessed the avian diversity in Yeguada, a small but diverse lagoon in northern Puerto Rico. Thirty-six biweekly counts were conducted to determine if species richness and abundance changed through time.  A total of 33 aquatic species was reported, ranging from 8 to 17 species, and from 42 to 616 individuals per count.  The five most abundant species were Glossy Ibis (Plegadis falcinellus, mean = 79.2 individuals), White-cheeked Pintail (Anas bahamensis, 77.2), Blue-winged Teal (Anas discors, 54.9), Black-necked Stilt (Himantopus mexicanus, 45.6) and Cattle Egret (Bubulcus ibis, 19.2).  When classified according to foraging guilds, dabbling ducks were the most abundant (mean = 147.5 individuals), followed by large wading birds (101.8) and small wading birds (55.7). Fluctuations in water levels were related to the total number of birds present, with variations among species.  For instance, a drought from July to December 2019 caused significant declines in vegetation gleaners, dabbling ducks and diving birds.  All the five most abundant species declined significantly during that period, with Blue-winged Teals, Pied-billed Grebes (Podilymbus podiceps) and Common Gallinules (Gallinula galeata) totally absent.  A single individual of American Flamingo (Phoenicopterus ruber), present in the lagoon for 10 years, left the area after the drastic drop in water level.}

\normaltalk{Decline of Species but not of Functional Diversity in a Large Protected Area of the Atlantic Rainforest in Brazil.}{Helon S Oliveira, Luiz dos Anjos}{Some studies have showed decline of bird species number in large protected areas, an unexpected situation. We also detected decline of bird species in a large protected area of the Atlantic forest in southern Brazil, the Iguassu National Park with 187,000 ha in size, based on a nine years monitoring census using point counts. We recorded 4,930 contacts of 140 bird species in 30 points, which were placed each 200m in a trail in the interior of forest. The loss of species recorded per year was verified by using a generalized additive model (Edf: 1.72; F: 11.02; p = 0.001). Species such as Glittering-bellied Emerald, Golden-winged Cacique, Lineated Woodpecker, and Violaceous Quail-Dove was not registered since 2015. Nevertheless, there was no loss in functional range filled by taxocenoses (Edf: 1; F: 0.011; p = 0.92). This result shows that the ecosystem functions of birds are maintained, even in the face of the loss of species. Species that are redundant to those that have been lost keep their ecological roles. Communities without redundant species become more sensitive to ecological impacts, as the loss of functionally unique species results in the loss of ecological roles. The monitoring should continue to evaluate if the observed decline is recovered with species recolonizations, what indicates a natural dynamic of bird community, or if it persist, what would be of conservation concern.}

\normaltalk{You can See Marshes, but you Won't See Birds. Which are the Main Features Bird Prefer to Inhabit a Marsh?}{Jesus Carlo Cuevas, Luis Ignacio I\~{n}iguez-Davalos, Ram\'{o}n Cuevas-Guzm\'{a}n, Martha Pilar Ibarra-Lopez}{Marsh birds have decrease their populations in the neotropic. Among the main causes are the destruction and degradation of wetlands. With the aim of provide information about marsh birds populations and their habitats, we analyze bird abundances and its relationship with some environmental and structural features of marshes from a portion of Lake Chapala. We study 14 marshes sites during 2018 breeding season, measuring biotic and abiotic variables. Among the abiotic, we register water pH, relative humidity, and water and atmospheric temperatures. The biotic variables were related to vegetation: species richness, coverage percentage and stratification. To detect birds, we use recorded calls from eight species with previously registered in the area. Five sampling visits are required to detect all species. Fulica americana y Gallinula galeata were the most abundant and Rallus tenuirostris, Butorides virescens and Porphyrio martinica were the least abundant. The use of calls positively increase the number of responses of birds. Lake Chapala marshes have a considerable plant species richness, high similarity in composition and are dominated by Typha domingensis. The variables that best explain variation in bird abundance are time since sunrise, coverage and plant species richness.  Las aves de ci\'{e}negas han decrecido sus poblaciones en el neotr\'{o}pico. Entre las principales causas se encuentra la destrucci\'{o}n y degradaci\'{o}n de los humedales. Con el objetivo de proveer informaci\'{o}n acerca de las poblaciones de aves de ci\'{e}nagas y sus h\'{a}bitats, se analiz\'{o} la abundancia de las aves y su relaci\'{o}n con algunas variables ambientales y caracter\'{i}sticas estructurales de las ci\'{e}nagas de una porci\'{o}n del Lago de Chapala. Se estudiaron 14 sitios en ci\'{e}nagas durante la temporada reproductiva del 2018 y se midieron variables bi\'{o}ticas y abi\'{o}ticas. Entre las abi\'{o}ticas, se registr\'{o} el pH del agua, humedad relativa, y la temperatura del agua y atmosf\'{e}rica. Las variables bi\'{o}ticas relacionadas con la vegetaci\'{o}n fueron: riqueza de especies, porcentaje de cobertura y estratificaci\'{o}n. La detecci\'{o}n de aves se realiz\'{o} utilizando llamados de las ocho especies de aves con registros previos en el \'{a}rea. Se requieren cinco visitas para detectar a todas las especies. Fulica americana y Gallinula galeata fueron las m\'{a}s abundantes y Rallus tenuirostris, Butorides virescens y Porphyrio martinica fueron las de menor abundancia. El uso de llamados increment\'{o} positivamente el número de respuestas de las aves. Las ci\'{e}nagas del Lago de Chapala tienen una considerable riqueza de especies de plantas, alta similitud en su composici\'{o}n y son dominadas por Typha domingensis. Las variables que mejor explicaron la variaci\'{o}n en la abundancia de aves fueron el tiempo transcurrido desde el amanecer, la cobertura y riqueza de especies de plantas.}

\normaltalk{Distribution of Golden Eagle Nests in Two Bio-Regions of the State of Chihuahua. Monitoring and Conservation in Northern Mexico.}{Francisco M Puente Guevara, Javier Cruz Nieto, Enrique Cisneros, Miguel A Cruz Nieto, Francelia Torres Gonzalez, Maria DLA Olvera D\'{i}az, Daniel M Rangel Resendez, Sergio Jimenez Lezama}{A group of community monitors called ``Eagle Warriors" was created to monitor the golden eagle. Since 2017, surveys have been carried out to locate new territories both in the APFF Ca\~{n}\'{o}n de Santa Elena and part of the Sierra Madre Occidental, with these surveys it was possible to locate 22 reproductive territories, of which 31.8\% are located in the Sierra. The 22 represent 12.8\% of all couples nationwide and 37.2\% statewide. They have been followed up every season, in 2019 a productivity of 1 flyer per nest was obtained. The mortality of 9 eagles has been recorded, 5 adults and 4 juveniles, for two possible causes (poison, shot), of these 9, 5 were males and 4 females. According to the results, 13.6\% of the territories present a very marked threat from infrastructure (roads and mines). Given the importance of maintaining the habitat, restoration or maintenance actions were carried out on water storage piles, with the placement of escape ramps, to avoid the drowning of eagles and other species.}

\normaltalk{Detectability and Abundance of Snowy Plovers in Eastern Venezuela}{Josmar E Marquez, Adri\'{a}n Naveda-Rodr\'{i}guez, Juan C Fern\'{a}ndez Ordo\~{n}ez}{The Snowy Plover (Charadrius nivosus tenuirostris) is a partial migrant shorebird distributed in North America with year-round residents in the Caribbean. Although it is globally listed as near threatened and has a decreasing population trend, little is known about its population size and habitat preferences in Northern South America.  We performed a survey to estimate population size of Snowy Plovers in a saltpan located in Coche Island in Venezuela. In February and December 2019, we conducted repeated visits during the non-breeding season sampling 12 point counts. We modeled the relationship between detection probability and air temperature, time of the day and distance to salt piles. We included covariates for the number of feral dogs, horses and salt miners when modeling abundance. Detection probability was negatively affected by the distance to salt piles (0.49, 95\% CI=0.49-0.5). Abundance was not associated with the presence or abundance of domestic animals nor salt mining activities. We estimated there were 96 (95\% CI: 46-201) and 116 (95\% CI=61-220) Snowy Plovers in February and December, respectively. Although our analysis is preliminary, we speculate that Snowy Plovers are not affected by low densities of free ranging domestic animals and low intensity artisanal salt mining activities. Management and conservation actions could address environmental awareness in local communities for an early control of free ranging dogs and horses, and to maintain low intensity of salt mining activities.}

\normaltalk{Patterns in Bobwhite Quail Occupancy over 15 Years Across the State of Arkansas, USA}{Grace Christie, Marcus Asher, Connor Gale, Andrhea Massey, Cody Massery, Christopher Middaugh, Ellery Ruther, John Veon, Brett DeGregorio}{Northern bobwhite (Colinus virginiasnus) populations have been rapidly declining across their natural range in the eastern, central, and southern United States for decades. Estimates of range-wide decline calculated by the Breeding Bird Survey indicate an annual loss of 3\%, with the most dramatic declines reported from the southeastern US. Regional variation in the severity of the population decline indicates locality-specific causes. Existing models of landscape factors on bobwhite populations may not be generalizable due to local variation in habitat types and land use. Many of these studies have focused on abundance rather than occupancy, meaning that the study of the mechanistic basis of bobwhite-landscape relationships is needed to better guide conservation through habitat creation and management.  To address the limitations of past modeling studies concerning habitat-scale effects on bobwhite populations, we built an occupancy model capable of handling a zero-inflated data set. Occupancy models are particularly useful when answering questions about species in decline, given their ability to account for incomplete detection. The Arkansas Game and Fish Commission (AGFC) has been monitoring bobwhite using presence-absence surveys for approximately the last 15 years. Using their available survey data, habitat variables identified as critical to bobwhite populations by the AGFC, and the model we have built, we examined how bobwhite habitat and occupancy throughout the state of Arkansas has changed over this period.}

\normaltalk{Do Socially-Mediated Allee Effects Limit the Distributions of Trailing-Edge Populations?}{Samuel A Merker, Richard B Chandler}{Understanding how climate change impacts trailing-edge populations near low-latitude range boundaries requires information about the roles that abiotic and biotic factors play in limiting their distributions. Socially-mediated Allee effects can limit species distributions by suppressing growth rates of peripheral populations when social information is scarce. Using four years of observational and experimental data in the Appalachian Mountains we tested the hypothesis that Allee effects limit the distribution of a trailing-edge population of a migratory songbird (Cardellina canadensis).  During the third breeding season, social cues were experimentally added at 10 sites near the species range margin to induce colonization and increase density at unoccupied sites. No experimental sites were colonized after adding social cues and no evidence of positive density dependence was found. Temperature, precipitation, and negative density dependence strongly influenced population growth rates, suggesting that, contrary to theoretical studies, Allee effects may play a minimal role in limiting species' distributions.}

\normaltalk{The Mountain Plover (Charadrius Montanus) in Mexico}{Julio Merayo Garcia, Irene Ruvalcaba Ortega, Michael Wunder, Jose Ignacio Gonzalez Rojas, Antonio Guzman Velasco}{The Mountain Plover (Charadrius montanus) is an upland shorebird associated to short-grass prairies, particularly to prairie dog (Cynomys spp.) colonies. It has declined during the last 50 years and is classified as ``Threatened" in the Mexican List of Endangered Species. Its southernmost breeding population was discovered in Mexico barely 20 year ago and has been poorly studied, with two nesting records and under-sampled density estimations. We therefore aimed to increase information on the distribution and abundance of the breeding and wintering population in the Grassland Priority Conservation Area ``El Tokio", in northeastern Mexico and a system of temporary lagoons in Zacatecas and San Luis Potosi, through occupancy ($<$U+03C8$>$) and distance sampling. We generated a grid of 30 ha plots, and selected randomly 77-121 in 36 sites surveyed in 2018 and 2019. We sampled between Apr-July and Jan-Feb. Plovers were detected in ten prairie dog towns during the surveys. In breeding season, Occupancy in prairie dog towns was $<$U+03A8$>$ =0.22 (IC95\%=0.14 -- 0.33) in 2018 and $<$U+03A8$>$ 0.20 (IC95\%=0.13 -- 0.29) in 2019, while in winter was 0.09 (95\% CI: 0.05-0.17). Densities were estimated in 1.27 birds/km2$<$U+202F$>$(IC95\%=0.70 -- 2.29) in 2018 and 1.71 birds/km2$<$U+202F$>$(IC95\%=1.00 -- 2.93) in 2019, while in winter occupancy was $<$U+03A8$>$ 0.09 (95\% CI: 0.05-0.17), and densities were of 1.98 birds/km2$<$U+202F$>$(95\% CI: 0. - 4.49). Mean plover abundance was around 300 ind. and was similar along the year. Our data suggest that the Mexican population of Mountain Plover is highly endangered, being relatively small  and having a very reduced range.}

\normaltalk{Study of Mycoplasma Gallisepticum (MG) in Wild Birds from North Georgia Using Molecular Detection}{Sawda Islam, Magan Free, Vivica Pressley, Linda Purvis, Dawn Drumtra}{Mycoplasma gallisepticum (MG) is a tiny bacterial organism, which lacks a cell wall, is resistant to antibiotics, and is problematic in the poultry industry. This bacterium can transmit directly from bird to bird via respiratory secretions or can spread indirectly through contaminated dust, droplets, or feathers. MG causes chronic respiratory disease and can result in conjunctivitis. This reduces egg production and quality in poultry. Even though MG was thought to only affect the poultry industry, this disease has more recently been detected in wild birds in North America. In our study, we are documenting the prevalence of MG in window-killed birds in the Atlanta metro area. We hope to obtain information on the spread of MG in native wild birds. We dissected salvaged resident and migrating birds donated by Atlanta Audubon Project Safe Flight. For each bird, we collected trachea, duodenum, and lung tissues as these organs have a high possibility of harboring MG. We conducted PCR and gel electrophoresis on that organ tissue to detect the presence of MG. This project is ongoing.}

\normaltalk{Changes in Community Temperature \& Precipitation Indices Reveal Changes in Composition of Avian Communities}{Shannon R Curley, Jose R Ramirez-Garofalo, Lisa L Manne, Richard R Veit}{Species distributions are becoming increasingly altered by climate change.  Different responses by species to climate can shuffle community assemblages. In this study we used North American Breeding Bird Survey (BBS) data from 1990 -- 2017 from eastern North America to evaluate changes in avian communities along a latitudinal gradient. We used two community indices, the Community Temperature Index (CTI), which measures contributions of ``warm" or ``cool" dwelling species in a community and a newly derived Community Precipitation Index (CPI), which measures contributions of ``high precipitation" or ``low precipitation" tolerant species to evaluate spatiotemporal trends of community characteristics. We hypothesized that CTI and CPI would each increase over time, due to increased abundances of southerly distributed ``warm" and ``high precipitation" species increasing within communities. We used Generalized Liner Mixed Models (GLMMs) to model spatiotemporal changes in CTI and CPI at the regional scale (all BBS sites) and local scale (BBS sites partitioned by 5° latitude bands). We employed a jackknife analysis to assess the positive percent contribution of each species to the temporal trends of CTI and CPI. Generalized linear models (GLMs) were used to ask if abundance trends, habitat associations, or a species affinity for ``warm /cool" and ``high precipitation/low precipitation" climates, contributed to the positive percent changes from the jackknifed models. CTI and CPI both linearly increased at the regional scale; however, these trends were more variable at different latitude bands. Individual species contributions to these temporal changes also varied with latitude.}

\normaltalk{Bird Community Dynamics Across a Successional Gradient in a Tropical Wet Forest}{Christy Carello, Shane Way, Scott W Yanco}{Forest succession can drive concomitant changes in associated faunal communities. Landscapes containing patches at various stages of succession can be an important driver of gamma diversity by increasing habitat heterogeneity and, therefore, diversifying the communities of animal inhabitants. Thus, maintaining landscapes with high successional diversity can be an important consideration in habitat management.  We sought to understand avian community dynamics across a successional gradient created by reforestation initiated at various times in the past. This study was conducted in a tropical premontane wet forest in Costa Rica. Specifically, we examined the effect of successional stage on overall abundance, species richness, diet niche, migratory status, and community composition. We hypothesized that early successional forests would experience greater abundances of birds driven by an increased proportional abundance of migrants in those treatments (driven by a resident advantage in more mature habitas). We also predicted that community compositions would vary across successional stages. Using data from point counts conducted in 2018 we found that abundance and species richness varied as a function of successional stage, with the highest values of both observed in the earliest successional stages. Furthermore, the proportion of migrant species was highest in early successional habitats. Finally non-metric dimensional scaling suggested distinct communities in each successional stage. We suggest that an ephemeral vegetation structure present for only a short period early in succession creates a unique habitat that results in a distinct avian community. This highlights the importance of habitat heterogeneity in maintaining avian diversity at landscape scales.}

\normaltalk{Describing Bird-Plant Frugivory Networks in a Secondary Ombrophilous Forest in Southern Brazil}{Francisco A V\'{a}squez-Ar\'{e}valo, Pamela E Malmoria, Carla S Fontana, Christian Beier, Ivan CC Provinciato, Pedro M A Ferreira}{The relationships between animals and plants can vary and be related to different degrees of specialization and trophic levels. We describe the frugivorous plant-bird interaction networks in fragments of secondary mixed ombrophilous forest (early and advanced stage) at the RPPN Pr\'{o}-Mata, Rio Grande do Sul, Brazil. We obtained seed samples by collecting feces from birds captured with mist nets in six fragments (three per stage). Bipartite networks of bird-plant interactions per stage were built, using quantitative matrices with the number of seeds occurring. 16 species of plants were identified (12 in the initial stage and 9 in the advanced) consumed by 23 species of birds (19 in the initial stage and 13 in the advanced). In the initial stage, Myrsine lorentziana stood out as the plant with the highest number of interactions. In the advanced stage, a better  distribution of interactions between plant species was observed. The early stage network showed less specialization compared to the advanced stage (H2: 0.45 and 0.62, respectively). Connectivity was lower in the initial stage (0.17) than in the advanced stage (0.22). Network nesting (weighted NODF) was higher in the initial stage (29.6) than in the advanced stage (16.9). Finally, the estimated niche overlap of birds (diet overlap) in the initial stage was much higher (0.42) than in the advanced stage (0.17). Our results indicate that the successional stage may be an important environmental filter for the networks that frugivorous birds form with the species of plants consumed and potentially dispersedAs relações entre animais e plantas podem variar e estar relacionadas à diferentes graus de especialização e n\'{i}veis tr\'{o}ficos. Descrevemos as redes de interação planta-ave frug\'{i}vora em fragmentos de Floresta Ombr\'{o}fila Mista secund\'{a}ria (est\'{a}gio inicial e avançado) na RPPN Pr\'{o}-Mata, Rio Grande do Sul, Brasil. As aves foram capturadas com redes de neblina em seis fragmentos (três por est\'{a}gio) para obter amostras de sementes pela coleta de fezes. Foram constru\'{i}das redes bipartidas de interação ave-planta por est\'{a}gio, usando matrizes quantitativas com o número de ocorrência de sementes. Foram identificadas 16 esp\'{e}cies de plantas (12 no est\'{a}gio inicial e 9 no avançado) consumidas por 23 esp\'{e}cies de aves (19 no est\'{a}gio inicial e 13 no avançado). No est\'{a}gio inicial destacou-se Myrsine lorentziana como a planta com maior número de interações. Na etapa avançada foi observada uma melhor distribuição das interações entre as esp\'{e}cies de plantas. A rede do est\'{a}gio inicial apresentou menor especialização em comparação com o est\'{a}gio avançado (H2: 0.45 e 0.62, respectivamente). A conectância foi menor no est\'{a}gio inicial (0.17) que no avançado (0.22). J\'{a} o aninhamento da rede (NODF ponderado) foi maior no est\'{a}gio inicial (29.6) comparado com o avançado (16.9). Finalmente, a estimativa de sobreposição de nicho das aves (sobreposição da dieta) no est\'{a}gio inicial foi muito maior (0.42) que no est\'{a}gio avançado (0.17). Nossos resultados indicam que o est\'{a}gio sucessional desempenha o papel de um filtro ambiental importante para as redes que as aves frug\'{i}voras formam com as esp\'{e}cies de plantas consumidas e potencialmente dispersadas.}

\normaltalk{Does the Spectacled Owl (Pulsatrix Perspicillata) Provide an Ecosystem Service as a Consumer of Exotic Invasive Species?:  Preliminary Observations on Owl Pellets at Two Peri-Urban Areas of Manizales, Colombia}{Sarah N Chaves-Casta\~{n}o, M C Hernandez, Valentina N Mar\'{i}n, Jimena N Ramirez, H\'{e}ctor E Ramirez-Chaves}{The Spectacled Owl (Pulsatrix perspicillata) is a Neotropical species that inhabits humid forests, savannas, and coffee plantations. It has been registered in the Amazon, the Andean region, the Caribbean and part of the Pacific in Colombia. However, there is very little information about the ecological role of this species in regard to ecosystem services that it can provide in the country. Under this scope, it was quantified the estimation of faunal and invasive alien species remains in spectacled owl pellets at two peri-urban locations in the city of Manizales (1980-2150 mamsl), which is placed in the Central mountain range of Colombia.   From the analysis of 134  and 46 pellet samples in these two areas, it was found that 37\% of the diet was composed of the common rat (Rattus rattus) in the area with the greatest anthropic intervention, which was identified by morphological and molecular analysis (sequences cytochrome b). The consumption of other exotic species such as pigeons (Columba livia) is under review. The native species detected included marsupials (Didelphis marsupialis, Marmosa sp. and Marmosops sp), rodents (two species of Cricetidae), bats (Artibeus lituratus), birds and insects to a lesser proportion (10\%). The high percentage of exotic invasive species in the samples indicates an ecosystem service provided by the Pulsatrix perspicillata in surrounding areas. This observation contributes as a resource to planning ways of mobilising knowledge in human communities about the valuable role of this species.El búho de anteojos (Pulsatrix perspicillata) es una especie Neotropical que habita desde bosques húmedos, de sabana, hasta plantaciones de caf\'{e}. En Colombia ha sido registrada en la regi\'{o}n Amaz\'{o}nica, Andina, el Caribe, y parte del Pac\'{i}fico. Sin embargo, la informaci\'{o}n ecol\'{o}gica y los servicios ecosist\'{e}micos que puede brindar esta especie en el pa\'{i}s es aún escasa. Para evaluar el posible papel de esta especie como controladora de especies ex\'{o}ticas invasoras, se cuantific\'{o} la proporci\'{o}n de estas con respecto a la fauna nativa en dos localidades periurbanas de la ciudad de Manizales (1980-2150 msnm), cordillera Central de Colombia. A partir del an\'{a}lisis de 134 y 46 muestras respectivamente, se encontr\'{o} que un 37\% de la dieta estaba compuesta por la rata común (Rattus rattus), en el \'{a}rea con mayor intervenci\'{o}n antr\'{o}pica, la cual fue identificada mediante an\'{a}lisis morfol\'{o}gicos y moleculares (secuencias de citocromo b). El consumo de otras especies ex\'{o}ticas como las palomas (Columba livia) se encuentra en proceso de revisi\'{o}n. Las especies nativas detectadas incluyen marsupiales (Didelphis marsupialis, Marmosa sp. y Marmosops sp), roedores (dos especies de Cricetidae), murci\'{e}lagos (Artibeus lituratus), y en menor proporci\'{o}n (10\%), aves e insectos. La presencia de un alto porcentaje de especies ex\'{o}ticas invasoras sugiere un servicio ecosist\'{e}mico que Pulsatrix perspicillata brinda a las comunidades humanas aleda\~{n}as a las zonas de muestreo que debe ser evaluado y socializado con m\'{a}s detalle.}

\normaltalk{Ecology of Breeding Birds, Ticks, and Tickborne Pathogens in New England Young Forest}{Andrew T Magel, Sean Williams, Andrew C Vitz}{Secondary ``young forest" habitat is a rare habitat in New England. Local private and government conservation agencies have been increasing young forest habitat and the associated avifauna, such as Eastern Towhee (Pipilo erythropthalmus), Blue-winged Warbler (Vermivora cyanoptera), and Prairie Warbler (Setophaga discolor). The creation of young forest creates edge habitat as well, which supports relatively high abundances of parasites, such as ticks (Ixodida). As this habitat type increases regionally, it is important to understand the changes in the ecology of birds and ticks. Furthermore, an increase in tick populations would have human health implications due to the number of human pathogens transmitted by ticks. We investigated the ecology of birds, ticks, and tickborne pathogens by capturing breeding birds in mist nets in Jun-Aug 2019. We extracted ticks from birds and tested them for the human pathogens Anaplasma, Babesia, Borrelia, and Ehrlichia. We found 296 of 681 individuals and 34 of 48 species were carrying ticks. Out of 100 engorged ticks sampled for pathogens, 29 tested positive for Borrelia burgdorferi, 10 tested positive for Babesia microti, and 14 tested positive for Anaplasma. No ticks tested positive for Ehrlichia. All ticks were identified as black-legged ticks (Ixodes scapularis). Veery (Catharus fuscescens) and Eastern Towhee had the highest tick loads, which were 4.81 and 2.43, respectively. We emphasize the potential role of birds as hosts to ticks and pathogens in young forest habitat. Future work should compare these tick loads and rates of occurrence of pathogens to other habitat types in New England.}

\normaltalk{Using Morphological Data to Understand Coexistence of Hispaniola'S Two Tody Species}{Holly Garrod, Andrea Townsend, Robert Curry}{Sympatric species provide an ideal model for understanding the evolution and continued coexistence of similar species. Often morphological data is used as a proxy for understanding if species may be getting more similar through hybridization or more dissimilar through reproductive barriers, often seen through character displacement. The Broad-billed and Narrow-billed todies of Hispaniola provide an excellent example of two similar species that separate altitudinally, where sympatric zones exist at mid-elevations. To date, no records of hybridization have been recorded between the two species and little is known about their interactions within these sympatric zones. We used two sets of banding data to compare morphometric measurements between both tody species: one data set that sampled throughout the island and a second data set that focused on allopatric and sympatric zones within the Cordillera Central. Our preliminary results show evidence of character displacement when looking at the island as a whole, but evidence against character displacement within the Cordillera Central. We suggest genetic analyses may further be used to clarify the relationship and historic extent of gene flow between the two species.}

\normaltalk{Temporal Changes of Bird Diversity using of the Portachuelo Pass, P.N. Henri Pittier State Aragua, Venezuela}{Miguel Lentino}{The Portachuelo Pass is the place in Venezuela with the longest history of bird records in the country. The first documented records date back to 1947 with censuses interrupted in 1954 and 1971. Since 1989 there has been continuous annual monitoring, working 128.456 hours/net, 64.506 birds captured and 336 species recorded. 50.5\% of bird catches were concentrated in the open area of the pass. Monitoring has been performed using 14-20 mist nets of 12 mts. Maintaining annual censuses in a region for a long period of time allows us to detect changes in populations and diversity. In the last 72 years the composition of the community of birds using The Portachuelo Pass has changed. Between 1947-1989 30 species stopped using the Pass, 6 of them migratory and 2 parrots. Comparing the periods 1991-2001 vs 2002-2018, 74 species stopped using it, 14 of them migratory and 3 parrots.  Comparing more recent periods 2002-2010 vs 2011-2018 62 species stopped using the Pass, 12 of them migratory, and 5 new species were recorded using the Pass.  The average species per decade between 1990-2018 recorded in the Portachuelo Pass was 148, 114 and 91 respectively, representing a decrease of 39\%. The increase in temperature and a decrease in the precipitation due to climate change in the area, have had very significant effects on the distribution and abundance of birds.}

\normaltalk{Ten Years of Bird Banding in the Siboney-Jutici Ecological Reserve, Cuba}{Yasit Segovia-Vega, Freddy B Rodr\'{i}guez-Santana, Carmen B Plasencia-Le\'{o}n, Nicasio CBCU Vina-Davila, Jose R Fuentes-Caballero, Leydis Sanchez-Zaldivar}{The ``Juan Crist\'{o}bal Gundlach bird banding station" began operating in July 2010 at the Siboney-Jutic\'{i} Ecological Reserve, Cuba. Since its opening 9942 individuals of 56 species have been captured. The station operates for three consecutive days every month, in the morning hours, with nets located in fixed sites. The 62\% of the individuals were captures and 38\% recaptures. The 52\% of captured species are migratory, 34\% permanent resident and 14\% are endemic. The species with the highest number of catches are Columbina passerina and Tiaris olivaceus, both permanent residents. The migratory species with the highest catches are Setophaga tigrina, S. caerulescens and S. ruticilla. The 20\% of the total migratory individuals have been captured more than once and 10\% in more than one season. The years 2012 and 2014 were the ones with the highest total number of catches and 2017 was the year with the lowest number of total catches. The months with the highest average catches are March, April, October, November and December. The month of June has the lowest number of average catches. The establishment of the banding office has also allowed training of personnel and promoting the opening of two more bird banding offices in Cuba.}

\normaltalk{Birds as Ecological Indicators at the University of Pennsylvania}{Chloe E Cerwinka}{Birds are ubiquitous, intricately connected to habitat, sensitive to environmental changes and their presence can indicate ecosystem function. Situated along a major bird migration route, the University of Pennsylvania (Penn) provides important stop-over habitat for migrating birds and foraging and nesting habitat for resident birds. Analysis of diversity, abundance and behavior reveals how the campus landscape provides essential habitat. Point-count surveys and area searches at representative habitat typologies on campus were conducted during migration and breeding seasons in 2018 and 2019. Systematic and incidental data collection resulted in 84 species from 34 families, with a total of 3,777 detections of birds recorded. As Penn's Landscape Planner, I play a pivotal role in furthering the university's sustainability goals and influencing landscape design and management on campus. This research establishes quantitative data that can define best practices and inform future landscape standards. Enhancing bird-friendly design guidelines and improving campus habitat and management strategies could have a broad impact on landscape connectivity and bird species richness. Future monitoring studies are recommended in order to quantify trends linking campus environmental health with bird species richness.}

\normaltalk{Spatial and Temporal Variation in Bird Community in Three Different Urban Parks, Guadalajara, Jalisco, Mexico}{Kirey A Barrag\'{a}n-Far\'{i}as, Veronica C Rosas-Espinoza, Ana L Santiago-Perez}{Urbanization transforms wildlands into man-made systems. Within the urban landscape, urban green areas represent the last remnants of greenery within large city boundaries. However, urban green areas, offer food and refugee to different bird species. Birds play an important ecological role as pollinators, predators, and seed dispersers, but our understanding of the processes that determine how bird communities respond to urbanization is still basic. Urban parks differ in surface area, topography, and vegetation structure, so, we analyzed how bird communities differed in space and time in three urban parks in the Guadalajara Metropolitan Area. The results show that site and landscape variables explained the bird community's diversity. The strongest relationship was the distance to the nearest Natural Protected Area. But the presence of most of the species was related to tree and shrub species and abundance. Guadalajara Metropolitan Area parks showed to be different in their characteristics, making it a place with a wide array of resources available to the bird community. This makes possible the establishment of different species with variables needs and to contribute to the maintenance of the city's biodiversity.}

\normaltalk{Noise as a Potential Mechanism Underlying the Effects of Urbanization on the Avian Gut Microbiome}{Mae Berlow, Elizabeth Derryberry}{The rapid effects of human land development and urbanization present relatively recent but stark changes in the environment, which can have consequences for wild animals. For example, recent work suggests that urbanization affects the composition of animal gut bacteria. One evolutionarily unprecedented consequence of urbanization is noise, and noise pollution is most likely a biologically relevant stressor in urban environments. Noise pollution can interfere with many aspects of an animal's life by changing social behaviours, interfering with foraging and predator identification, and changing stress hormone levels. Sustained increases in stress hormones can have wide ranging physiological repercussions and may change the substrate for intestinal bacteria through alterations in gut physiology such as increased intestinal mobility, permeability, and inflammation, leading to changes in the composition and relative abundances of gut microbiota. We experimentally examine this relationship between noise and the gut microbiome through measurements of stress hormones and feeding behaviour.}

\normaltalk{Noise Level and Water Distance Drive Resident and Migratory Bird Species Richness Within a Neotropical Megacity}{Karlla VdC Barbosa, Amanda Rodewald, Milton C Ribeiro, Alex Jahn}{A large body of evidence indicates that urbanization profoundly affects ecological communities, but the extent to which patterns are generalizable across regions, such as in the Neotropics, remains unclear. We examined responses of migratory and resident birds to human disturbance and habitat attributes in São Paulo, Brazil, a tropical megacity in South America. In 2017--2018, we surveyed birds across 31 landscapes distributed across the urban landscape and evaluated competing models that included five non-correlated variables explaining variation in species richness:  ambient noise level, distance to water, tree cover, human population size, and impervious surface. We recorded 142 bird species, 128 of which were resident and 14 migratory. Richness of both resident and migratory birds declined with increasing noise level and distance to water, which best explained variation in bird communities among the sampled landscapes. Although resident and migratory birds presented similar response patterns to local and landscape attributes, noise level was the best predictor of migratory species occurrence, whereas distance to water best explained the occurrence of resident species. Our results suggest that, although tree cover is important to biodiversity in urbanized landscapes, proper management of urban water bodies and reduction of noise levels are also essential to maintaining avian diversity within tropical urban areas and suggest novel avenues for future research in tropical urban ecology.  Um grande conjunto de evidências indica que a urbanização afeta profundamente as comunidades ecol\'{o}gicas, mas esses padrões são generalizados para extensas regiões, sendo que nos Neotr\'{o}picos o efeito permanece incerto. Examinamos respostas de aves migrat\'{o}rias e residentes a distúrbios humanos e atributos de habitat em São Paulo, Brasil, uma megacidade tropical na Am\'{e}rica do Sul. Em 2017--2018, pesquisamos aves em 31 \'{a}reas distribu\'{i}das pela paisagem urbana e avaliamos modelos concorrentes que inclu\'{i}am cinco vari\'{a}veis, $<$U+200B$>$$<$U+200B$>$não correlacionadas, que explicam a variação na riqueza de esp\'{e}cies:  n\'{i}vel de ru\'{i}do ambiente, distância de corpos d´\'{a}gua, cobertura arb\'{o}rea, tamanho da população humana e superf\'{i}cie imperme\'{a}vel. Registramos 142 esp\'{e}cies de aves, das quais 128 são residentes e 14 são migrat\'{o}rias. A riqueza de aves residentes e migrat\'{o}rias diminuiu com o aumento do n\'{i}vel de ru\'{i}do e da distância da \'{a}gua, o que melhor explicou a variação nas comunidades de aves entre as paisagens amostradas. Embora as aves residentes e migrat\'{o}rias apresentem padrões de resposta semelhantes aos atributos locais e paisag\'{i}sticos, o n\'{i}vel de ru\'{i}do foi o melhor preditor da ocorrência de esp\'{e}cies migrat\'{o}rias, enquanto a distância da \'{a}gua melhor explicou a ocorrência de esp\'{e}cies residentes. Nossos resultados sugerem que, embora a cobertura arb\'{o}rea seja importante para a biodiversidade em paisagens urbanizadas, o manejo adequado dos corpos d'\'{a}gua urbanos e a redução dos n\'{i}veis de ru\'{i}do tamb\'{e}m são essenciais para manter a diversidade das aves nas \'{a}reas urbanas tropicais e sugerem novos caminhos para futuras pesquisas em ecologia urbana urbana.   }

\normaltalk{Bird Diversity Hotspots Mask Socioeconomic Blind Spots in Sampling Urban Birds: A Comparison of eBird and Systematic Surveys}{Deja Perkins, Michael W Caslin, Madhusudan Katti}{Citizen science has become a popular way to gather large datasets that also engages lay audiences with the scientific process. eBird has become one of the most popular platforms to gather data globally capitalizing on a preexisting demographic: birders. There is great potential to use eBird gathered data to inform urban development, park management and policies, however, one caveat of eBird data are the potential reporting biases of volunteers, which may not be representative of the urban landscape or demographics. In this study, we examine data from a new systematic point count within the ``Triangle" region of North Carolina, as well as similar long-term datasets from Tucson and Fresno, to compare with data from eBird to investigate differences in spatial resolution, distribution and diversity of urban birds in relation to underlying socioeconomic gradients. The systematic approach uses standardized point counts at locations semi-randomly distributed across the entire urban habitat matrix in Raleigh, Durham, Cary, Chapel Hill, modeled after long-running projects in Tucson, AZ (since 2001) and Fresno, CA (since 2008). We demonstrate that systematic point counts provide a more accurate representation of bird diversity and distribution across the urban matrix than eBird. Our analysis reveals both hotspots of eBird reporting and blind spots in coverage of lower income and racially diverse neighborhoods, highlighting how citizen science platforms can inadvertently perpetuate environmental injustice in cities.}

\normaltalk{Using Game Cameras to Determine more Accurate Nest Fates for Least Terns in Southern Louisiana}{Sarah M Bolinger, Erik I Johnson, Paul L Leberg}{Beach restoration and renourishment projects in southern Louisiana create new sparsely vegetated beach, thought to be attractive nesting habitat for Least Terns (Sternula antillarum). Our analyses suggest that these restoration projects improve nest survivorship during storms due to the increase in elevation, but how they affect predation is largely unknown. Nest fates are determined in the field using clues in the nest area, but many fates remain unknown in the absence of such cues. These unknowns are excluded from analysis, potentially biasing nest success estimates if some fates are easier to determine in the field than others. In 2019, we began a trial of nest cameras on beaches in southwest Louisiana to help us more accurately determine nest fates and look for human disturbance to nests. We compared nest data between camera absent/camera present conditions, as well as comparing both to synthetic nest data to determine biases across multiple model types. Daily survival rate (DSR) of nests does appear to be negatively biased in the absence of cameras. Continued monitoring by Audubon Louisiana and graduate students will provide insight into how these trends in predation, washout, and nest survival evolve in the years following restoration.}

\normaltalk{Loggers in Artificial Eggs Reveal Wild Scarlet Macaw Incubation in Natural and Artificial Nests}{Constance J Woodman, Donald J Brightsmith}{The use of loggers inside artificial eggs has offered new knowledge about how birds incubate and attend to their eggs. Newer technology offers true 3D position recording to more fully describe behaviors. This detailed data can inform artificial incubation practices for intensive species management programs. In the Peruvian Amazon, at the Tambopata Macaw Project, loggers were added to clutches in 21 nests after the first egg was laid. There were three nest types, natural, artificial wood, and artificial PVC boxes. The majority of hens had known nesting histories. Descriptions of macaw incubation behavior are created using linear models to detect relationships between temperature, weather, 24-hour cycling, and types of egg movement. The value of quantifying incubation to conservation projects is shown through a case study. In the case study, maternal behavior toward a logger egg placed after total nest predation. The logger was used to check if the female's level of maternal care would be appropriate for a foster chick, allowing the team to increase managed-nest productivity.}

\normaltalk{Evaluation of Small Unmanned Aerial Systems for Monitoring Wading Bird Nest Abundance and Survival}{Rostam E Mirzadi, Dale E Gawlik}{Wetlands around the world have been impacted from drainage and increased water allocation for human needs. In the Everglades of Florida, these alterations to the hydrologic regime have reduced the ability of wading birds to forage and reproduce successfully, resulting in a decline in numbers of nests and a shift in the timing and location of nests. Because wading birds have shown such a stark response to hydrologic changes, and much is known regarding their habitat requirements and historical nesting patterns, their reproduction is monitored as an indicator of ecosystem change and restoration progress. Monitoring birds over such a large area (40000 ha) with traditional fixed-wing aircraft is costly and can have high error rates in detection. Our study aims to determine whether the efficiency of wading bird monitoring can be improved through the use of small unmanned aerial systems (sUAS). To test the efficiency of sUAS for monitoring nests of colonies of Snowy Egret (Egretta thula), Tricolored Heron (E. tricolor), and Great Egret (Ardea alba), colonies will be photographed monthly. The resulting images will be rendered into a single RGB image through post processing software to estimate nest abundance, survival, and brood size at fledging. Results will be compared to the standard aerial surveys from fixed-wing aircraft. If sUAS surveys can be used to accurately record nest abundance, survival, and fledging success, it will be possible to survey a greater number of colonies and nests with a higher degree of accuracy and improved safety of observers.}

\normaltalk{Can Livestock Farming Prevent the Extinction of Threatened Grassland Birds? a Study Case in South America.}{Melanie Browne, Alejandro G Di Giacomo, Adri\'{a}n S Di Giacomo}{The Strange-tailed Tyrant (Alectrurus risora) is a threatened grassland bird that experienced a drastic 90\% reduction in Argentina. Remaining populations occur in the provinces of Corrientes and Formosa in northeastern Argentina where livestock production is the main land-use. We evaluated its breeding biology to use as a study case of the response of endemic threatened grasslands birds to differing grazing histories. The study was conducted in a protected area without cattle in Formosa, another protected area in Corrientes, and several livestock ranches in Corrientes. The daily survival rate (DSR) was significantly greater in the protected area of Formosa than in Corrientes and in livestock ranches, with a probability of nest success of 0.30, 0.15 and 0.07 respectively. Nest density was also significantly higher in protected areas than in livestock ranches (nest density= 0.74 ± 0.09 nests/ha, and 0.09 ± 0.01 nests/ha; respectively). Grazing history could explain the different DSR observed between the provinces. Cattle was introduced in Corrientes in the 18th century, while it was introduced in Formosa in the 1930s. Early persecution by farmers drove top predators to extinction in Corrientes, while in Formosa they are extant. Thus, nests' mesopredator abundance in Corrientes is greater than in Formosa. The lower DSR and nest density of livestock ranches could be influenced by current livestock practices (i.e. annual-fire, high animal stocking and pesticide use) affecting vegetation structure and food availability. These results are alarming considering that most threatened grassland birds' populations are in private lands devoted to livestock grazing.}

\normaltalk{Mating Strategy Variation and Plasticity in Avian Wild Conspecifics of Domesticated Species as a Potential Indicator of Domesticability}{Catherine E Mendel}{Investigations of domestication tend to focus on mammals while bird domestication is often a minimalized or understudied process.  This may be due in part to the newness of some of the rediscoveries of these domestication events.  Or, it could be due to a heavy mammal research bias similar to those of other academic fields.  Either way, a deficit in the knowledge of the systems of avian domestication exists.  The specific benefits of exploring avian mating strategies as potential indicators of domesticatability lies in the relatively recent domestication of at least some domesticated bird species.  This recentness allows us to pose some interesting comparative questions with the wild conspecifics (that are notably still extant) of the domestic lineages.  And, specifically about reproduction because after all domestication is a process in which humans assume some degree of control over an animal's breeding.  However, often when bird domestication is addressed as a system in anthropological literature, it is done so in sweeping reviews of animal domestication which encompass both mammal and avian taxa with dichotomous characterizations given for traits desirable or favorable for domestication in both. Notable are the dichotomous sexual behavior traits listed as favorable for domestication.  Herein, I explore the applicability of the sexual behavior dichotomy of promiscuous-favorable versus monogamous-unfavorable, and the traits juxtaposed within it to the avian mating systems of the wild progenitor species of four domesticates: Turkey (Meleagris gallopovo), Greylag Goose (Anser anser), Mallard Duck (Anas playrhynchos), and Red Junglefowl (Gallus gallus).}

\normaltalk{Reproductive Phenology and a New Nesting Site of Agami Heron in Southern Mexico}{Jorge Correa Sandoval, Xanny L Garc\'{i}a Reynoza, Griselda Escalona Segura, Paula L Enr\'{i}quez, Pablo Baldovinos Rogel}{Seasonal fluctuations of biotic and abiotic variables influence the nesting period of wetland birds. This study describes the heron Agamia agami's reproductive phenology at R\'{i}o del Este, Laguna de Terminos, Campeche, and at El Aguaje, La Encrucijada Biosphere Reserve, Chiapas over five years (2015 -- 2019). The reproductive colony at El Aguaje represents a new nesting site for Mexico. In R\'{i}o del Este, the nesting season lasted from July to December, while at El Aguaje, it went from June to December, with nesting periods of varying length (i.e., 114, 144 and 155 days). The nests were built among dense, partially submerged vegetation, with low disturbance. The variables that correlated positively with the nesting period in R\'{i}o del Este were precipitation, water level and temperature (F3, 56 = 32.59, p = $<$0.001, R2 = 0.63), while in El Aguaje it was precipitation, high tide and temperature (F3, 56 = 54.18 p = $<$ 0.001, R2 = 0.74). Climatic, hydrological, and oceanographic conditions influenced the establishment and loss of nesting colonies at both sites. The new nesting site expands the range of distribution for the species in Mexico. The conservation and wise management of wetlands are essential for the protection of the nesting sites of this species.Las aves de humedales requieren de condiciones clim\'{a}ticas, hidrol\'{o}gicas y oceanogr\'{a}ficas id\'{o}neas para anidar. En este estudio se describi\'{o} la fenolog\'{i}a reproductiva de la garza Agamia agami en R\'{i}o del Este y El Aguaje al sur de M\'{e}xico durante un per\'{i}odo de cinco a\~{n}os (2015-2019). El Aguaje representa un nuevo sitio de anidaci\'{o}n en M\'{e}xico. La temporada de anidaci\'{o}n abarc\'{o} los meses de julio a diciembre en R\'{i}o del Este y de junio a diciembre en El Aguaje, con per\'{i}odos de anidaci\'{o}n desfasados interanualmente (114, 144 y 155 d\'{i}as). Los nidos se colocaron en vegetaci\'{o}n densa, parcialmente sumergida y con baja perturbaci\'{o}n. El per\'{i}odo de anidaci\'{o}n se correlacion\'{o} positivamente con la precipitaci\'{o}n, nivel del agua y temperatura en R\'{i}o del Este (F3, 56 = 32.59, p = $<$ 0.001, R2 = 0.63), y con la precipitaci\'{o}n, pleamar y temperatura en El Aguaje (F3, 56 = 54.18 p = $<$ 0.001, R2 = 0.74). Las condiciones clim\'{a}ticas y oceanogr\'{a}ficas influyeron en el establecimiento y la p\'{e}rdida de las colonias de anidaci\'{o}n en ambos sitios. El registro del nuevo sitio de anidaci\'{o}n amplia el rango de distribuci\'{o}n reproductiva de la especie en M\'{e}xico. La conservaci\'{o}n y manejo de los humedales es esencial para la protecci\'{o}n de los sitios de anidaci\'{o}n de esta especie.}

\normaltalk{Local Weather Effects Explain Annual Variation in Northern Goshawk Breeding Performance in the Northern Great Basin, USA}{Eliana R Heiser, Allyson B Bangerter, Jay D Carlisle, Robert A Miller}{We investigated and quantified the effects of weather on breeding Northern Goshawks (Accipiter gentilis) from 2011 through 2019 in south-central Idaho and northern Utah, USA. Using data from the 42 territories where we found evidence of breeding in at least one of the nine years, we quantified breeding rates for 315 territory-season combinations, analyzed nest success and productivity in 134 breeding events, and estimated fledging date for 118 breeding events. We used generalized linear mixed models to examine 33 predictor variables from four general categories: precipitation, temperature, wind, and snowpack. We employed a binomial distribution to assess breeding rate, a Poisson distribution to assess productivity per breeding attempt, and a Gaussian distribution to assess fledgling date. Of the variables we tested, April precipitation, previous year's spring precipitation, maximum consecutive days with heavy precipitation, mean April-May temperature, minimum April-May temperature, mean February-May temperature, and year had measured effects. Although select weather patterns induced by climate change may benefit Northern Goshawks, we anticipate that climate change will have a negative overall effect on Northern Goshawk breeding success. Our results indicate that weather affects Northern Goshawk breeding success in the northern Great Basin, and we suggest that the relationship between raptor breeding success and weather be further investigated to enable predictions of how a changing climate will impact their populations.}

\normaltalk{Breeding Season Development of the Wilson'S Warbler Cloacal Protuberance, Among the Relatively Largest of Any Passerine Species Worldwide.}{William M Gilbert}{The relative cloacal protuberance (CP) size at maximum development, the CP volume index (hpr2g-1), is considered to be one indicator of sperm competition.  The mean maximum for 29 North American passerine species was determined to be 10 mm³g-1.   However, for 10 worldwide species with uncommon sexual behaviors or social systems, including polygynandry, the mean maximum was 33.6 mm³g-1.  I determined that central California Wilson's Warbler (WIWA) males have a mean maximum of ¬¬32.6 mm³g-1.  Also, SE Alaska WIWA males had a mean maximum of 38.3 mm³g-1.  I plotted the development of resident male (RM) WIWA CP volume indexes over the breeding season.  The mean maximum soon after migratory return was ~ 10 mm³g-1, and remained so for ~ 25 days.  This moderate size also was maintained ~ 14 days past the mean start of resident female nest building, when females were first fertile.  However, the population of WIWAs that I studied annually hosted a large subpopulation of non-territorial (floater) males and females.  Over five days following the mean first arrival time of large numbers of floater males in my study area, the mean CP volume index of RMs increased from 15 mm³g-1 to 25 mm³g-1.  The mean CP volume index of RMs varied between 25 and 43 mm³g-1 during five-day sampling periods throughout the remainder of the breeding season, ~ 85 days.La protuberancia de la cloaca (CP) del masculinos del mayor de especies de paseriformes se funcionar por el dep\'{o}sito del esperma e tombien un \'{o}rgano de intromission.  La talla relativo de la CP a el desarrollo m\'{a}ximo, el \'{i}ndice de volumen de la CP, (hπr2g-1), se considera uno \'{i}ndicador de la competici\'{o}n esperma.  El medio \'{i}ndice del volumen de la CP m\'{a}ximo por 29 especies Norteamericana de paseriformes de se determin\'{o} que era m\'{a}s o menos 10 mm³g-1.  Sin embargo, por El medio \'{i}ndice del volumen de la CP m\'{a}ximo por nueve especies en todo del mundo con conductas sexuales o sistemas sociales poco común, incluso polygynandri, el medio m\'{a}ximo estaba 33.6 mm³g-1.  Yo determin\'{e} el medio m\'{a}ximo \'{i}ndice del volumen de la CP por la Chipe Coroninegro en California central estaba ¬¬32.6 mm³g-1, entre las mas grande de especies en todo el mundo.  Tombien, mediciones de CP de Alaska sureste el medio m\'{a}ximo \'{i}ndice de volumen de la CP estaba 38.3 mm³g-1, sugerencia que el rasgo de un volumen de la CP relativamente grande es en toda de la especie.  Yo trazado el desarrollo de \'{i}ndice de volumen de CP \'{i}ndice de volumen de la Chipe Coroninegro durante la \'{e}poca de la reprodiucci\'{o}n.  El medio m\'{a}ximo \'{i}ndice de volumen de la CP de los masculinos residente (RMs) poco despu\'{e}s el retorno migratorio estaba moderado a ~ 10 mm³g-1, e permaneci\'{o} as\'{i} por ~ 25 dias.  Esto tambi\'{e}n fue ~ 14 dias pasado el memes comienzan de construcci\'{o}n de nidos de las hembras residente, quando hembras primero nos cnvert\'{i}amos furtil.  Sin embargo, la poblaci\'{o}n de Chipe Coroninegros que estudie anualmente alojado una subpoblaci\'{o}n grande de masculinos e hembras no territorial (flotadores), e durante cinco d\'{i}as siguiendo el medio de dia de llegada de números grandes de masculinos no territorial en mi \'{a}rea de estudio, el media del \'{i}ndice de volumen de la CP de RMs aument\'{o} r\'{a}pidamente de 15 mm³g-1 a 25 mm³g-1.  El media del \'{i}ndice de volumen de la CP de RMs variar entre 25 and 43 mm³g-1 durante per\'{i}odos de muestreo de cinco d\'{i}as durante todo el recordatorio de la \'{e}poca de la reproducci\'{o}n, ~ 85 dias.}

\normaltalk{The Influence of Natal Environment and Physiology on Offspring Growth and Development in Eastern Bluebirds (Sialia Sialis).}{Madeline C Sudnick, Jamie Christie, Kelly A Williams}{Variation in offspring development influences health and survival of individuals into adulthood and generates differences in local recruitment. Understanding how the natal environment including nest temperature and food availability influences offspring growth and development is important for determining success of individuals in changing environments. However, developmental consequences from poor natal environment may be mitigated or altered by the offspring's physiological response, including corticosterone levels. Under stressful environmental conditions, individuals may vary in the strength of their corticosterone response and in how limited resources are shifted among vital functions when corticosterone is high. This study examines the interactions between environmental conditions, parental behavior and nestling physiology and how they influence nestling development. I monitored Eastern Bluebird nests throughout the 2019 breeding season in Athens, Ohio. I recorded nest temperature, arthropod availability, and parental food provisioning rate during the field season. To evaluate nestling development, I measured the tarsi of each nestling three times between nestling days 6 and 11 to calculate growth rate, and obtained a blood sample to determine hematocrit. I measured corticosterone response by using EIA assay kits to find corticosterone levels in nestling plasma. Observing natal environment and offspring physiology will allow us to determine whether birds have innate ways to mitigate or redirect consequences of a climate with increasing variation.}

\normaltalk{Experience Counts: The Role of Female Age in Adjusting Morning Incubation Behavior in Relation to Temperature}{Kelly A Williams, Madeline Sudnick, Rachel Anderson, Meredith Fitschen-Brown}{Reproductive experience can impact how individuals allocate time and energy to reproduction and generate differences in reproductive behavior that leads to experience dependent variation in reproductive success. In order to understand if individual variation in parental behavior is related to environmental temperature and breeding experience, we observed the timing and duration of the first morning off bout in a wild, open cup nesting passerine bird during the incubation and early nestling period. We compared incubation behaviors and nest temperature of inexperienced (second year = first breeding season) and experienced (after second year) female hooded warblers Setophaga citrina. Females left the nest earlier on colder mornings suggesting an energetic constraint due to the long overnight incubation on bout during colder temperatures. During incubation, females increased the duration of the first morning off bout with increasing temperature. Similarly, during the early nestling period, experienced females had shorter off bout duration on colder mornings and increased duration with warmer temperatures. In contrast, inexperienced females increased off bout duration with colder morning temperatures. Experienced females maintained higher nest temperatures and higher minimum nest temperatures compared to inexperienced females. The ability of experienced females to maintain higher minimum nest temperatures and to adjust brooding behaviors during colder mornings is a potential mechanism that has consequences for nestling condition in a wild population. Our results highlight the need to examine experience related parental care behaviors in responding to environmental variation and the consequences to reproduction.}

\normaltalk{Nest Visitation Rates and Fledging Patterns in Kirtland's Warblers}{Ashley Olah, Kim Grveles, Sarah Warner, Davin Lopez, Christine A Ribic, Anna M Pidgeon}{Passerine ground nests are well concealed and difficult to observe, and basic patterns of activity at nests are not well understood. Similarly, little is known about the motivation, process, and timing of fledging. Length of time in the nest could be influenced by parent-offspring conflict, i.e., parents favoring fledging relatively early, or by nestling choice, i.e., nestlings achieving a minimum developmental stage before fledging. Fledging early in the day may be associated with reduced predation risk or alternatively, with greater likelihood of finding a parent to provide food (energetics). The period over which the brood fledges may be greater if competition among nestlings is reduced when siblings fledge. We documented nest visitation rates and fledging behavior in the Wisconsin population of Kirtland's Warblers using video cameras placed near 13 nests. We found that adult nest visitation rates increased with nestling age and the duration of visits decreased with nestling age. Approximately half (57\%) of the 14 chicks we captured fledging on video left the nest using terrestrial locomotion, and 43\% used wing assisted locomotion for fledging. Most young fledged (93\%) when no adult was present, and in only 36\% of fledging observations the adults flew to the chick after it fledged. The average amount of time between the last adult visit and when a chick fledged was ~ 6 minutes. Our results suggest that Kirtland's warblers do not exhibit parent-offspring conflict, but instead that energetics and nestling choice influence how and when fledglings leave the nest.}

\normaltalk{Does Daily Thermal Variability Affect Incubation Behavior of Two Ground-Nesting Birds?}{William Kirkpatrick, Rachel Beyke, Craig Davis, Sam Fuhlendorf, Sarah DuRant}{Research indicates that thermal variability may be a better predictor of organism responses to temperature than shifts in average temperature. However, studies examining behavioral responses to the thermal environment, including studies predicting species responses to climate change, focus on mean environmental temperature. In birds, parental responses to the thermal environment can have important consequences for nest temperatures and offspring phenotype. Although, prior research in birds finds broad patterns in incubation patterns and temperature, for instance shorter off-bouts early in the season when ambient temperatures are low, few studies have explored the relationship between incubation behavior and fine-scale patterns of temperature or thermal variability. Our goal is to determine if diurnal thermal variability affects the timing and duration of critical nesting behaviors in two quail species. Fifty-nine scaled and bobwhite quail nests were monitored throughout the 2015 and 2016 breeding season. Temperature loggers captured egg incubation temperatures and operative temperature near the nest. Using these thermal data and NestIQ, a machine learning software for analyzing nest thermal data, we can determine patterns among environmental temperature (daily thermal variability, mean daily temperature, maximum temperature, and temperature at the time of off-bouts), parental behavior, and temperatures experienced by eggs. Our hypotheses are that fine-scale temperature variability is a better predictor of incubation constancy than temperature mean over the same period. Our data will have important implications for predicting avian responses to climate change, because both mean temperature and thermal variability are predicted to rise substantially in the coming years.}

\normaltalk{Effects of Brood Size on Parent Provisioning and Foraging Behavior: A Field Experiment in Barn Swallows}{Sage A Madden, Molly T McDermott, Rebecca J Safran}{According to life history theory, reproduction is costly, and increased parental effort can lead to reduced survival and fecundity. Animal parents are thus expected to maximize lifetime fitness by balancing trade-offs between investment in current offspring versus investment in survival and future reproduction. In birds, brood size manipulation (BSM) experiments have been used to examine how parents balance these life history trade-offs. Previous research has investigated whether parents are willing to alter their visitation rate and the amount, type, or size of prey they bring to nestlings in response to changes in brood size, but our knowledge about how BSM affects the spatial foraging patterns of parents is limited. We set out to understand how BSM affects foraging patterns and other metrics of parental effort. We altered the brood size of North American Barn Swallows (Hirundo rustica erythrogaster) breeding in Colorado. Using GPS tags and on-site observation, we tracked visitation rate, brood growth rate, and foraging distance from the nest before and after manipulation. We found that female visitation rate significantly increased by 105\% following brood enlargement. Brood growth rate significantly increased by 1.05 grams/hour following brood enlargement. Foraging distance was unaffected by changes in brood size. Our results suggest that while Barn Swallow parents are willing to scale some aspects of feeding effort to nestling number, foraging distance is similar across brood sizes. These data contribute to a more complete understanding of how avian parents balance life history trade-offs when provisioning nestlings.}

\normaltalk{Comparison Of Parental Activity And Nest Predation In Worthen's Sparrow (Spizella Wortheni) On Two Different Anthropogenic Disturbance Level Landscapes In Galeana, Nuevo Le\'{o}n.}{Eliphaleth Carmona G\'{o}mez, Irene Ruvalcaba, Ricardo Canales del Castillo, Antonio Celis-Murillo}{Variation in environmental factors induced by different levels of anthropogenic activities may impact key behavioral parental-care-related patterns on birds, such as those proposed as related to nest predation. We examined the relationship of parental activity to nest-predation in the Worthen's Sparrow (Spizella wortheni), an endangered sparrow endemic to the Chihuahuan Desert with high nest predation rates, as well as the behavioral variation among the parents on sites of low and high anthropogenic disturbance level (LDL/HDL), and among both incubation and nestling stages. We found clear differences on behavioral patterns among both stages, and although the parents of both sites showed no significant difference on their behavioral repertoire size (LDL: Mean=18, 95\%CI=16--20; HDL: Mean= 18, 95\%CI=15-21), we found differences on the frequency and time of behaviors, such as the total time spent on nest per hour (LDL: Mean=11min, 04s, 95\%CI=-2min, 01s -- 24min, 07s; HDL: Mean=25min, 05s, 95\%CI=19min, 47s -- 30min, 23s; p= 0.03). We found a higher daily survival rate for nests on HDL sites, but we were unable to detect any clear relationship among nest survival and parental-care behavioral patterns. The clear variation on behavioral patterns among sites can be explained by the risk perceived by parents, which could be related, as well as the difference on nest survival, to differences on the nest-predator complex on both sites. Careful multi-species predator study and management might be needed in order to increase Worthen's nesting success.}

\normaltalk{Evaluation of the Physiological Condition and Status of Infection by Hemosporides in the Painted Bunting (Paserina Ciris, in Two Stages of Its Annual Cycle}{Leonardo D Mijares Villarreal, Ricardo Canales del Castillo, Irene Ruvalcaba Ortega, Gary Voelker, Jos\'{e} I Octavio Rojas}{Throughout the annual cycle, migratory birds suffer enormous stress, which causes immunosuppression, and in turn susceptibility to infections, including parasitic ones. That is why this study aims to determine the prevalence and intensity of hemosporidial infection, in addition to the physical condition in the painted bunting (Passerina ciris). A comparison was made between the reproductive stage (Coahuila) and the migratory stage (Sonora) using generalized linear models (GLMs) to find out the effect of sex and the stage of the annual cycle on prevalence, parasitaemia, body mass index (BMI). and heterophil-lymphocyte ratio (H: L). The detection of hemosporidia was carried out by means of microscopy and at the molecular level. Both individual condition indicators (H: L and BMI) were significantly higher at the resting site. However, males presented lower BMI, which results in a lower amount of energy reserves. Increased stress Physiological (H: L) may be due to the effect of migration and the molting process that takes place in the resting place, exerting an energy cost on the body. Although the prevalence did not show any significant effect on the predictor variables, the parasitaemia was significantly higher in males than in females. These results could be due to intrinsic factors resulting from the breeding season in males, which has an immunosuppressive effect. It is necessary to identify the importance of the resting place in migratory birds, since this information will serve to better understand the function of the resting place.  EVALUACIÓN DE LA CONDICIÓN FISIOLÓGICA Y ESTATUS DE INFECCIÓN POR HEMOSPORIDIOS EN EL COLORÍN SIETECOLORES (Passerina ciris), EN DOS ETAPAS DE SU CICLO ANUAL  A lo largo del ciclo anual, las aves migratorias sufren un enorme estr\'{e}s, el cual provoca inmunodepresi\'{o}n, y a su vez susceptibilidad a las infecciones, entre ellas, las parasitarias. Es por eso que este estudio tiene como prop\'{o}sito determinar la prevalencia e intensidad de infecci\'{o}n por hemosporidios, adem\'{a}s de la condici\'{o}n f\'{i}sica en el Color\'{i}n sietecolores (Passerina ciris). Se realiz\'{o} una comparaci\'{o}n entre la etapa reproductiva (Coahuila) y la migratoria (Sonora) utilizando modelos lineares generalizados (GLMs) para conocer el efecto del sexo y de la etapa del ciclo anual en la prevalencia, parasitemia, \'{i}ndice de masa corporal (IMC) y proporci\'{o}n heterofilos-linfocitos (H:L).  La detecci\'{o}n de hemosporidios se llev\'{o} a cabo por medio de microscop\'{i}a y a nivel molecular. Ambos indicadores de condici\'{o}n individual (H:L e IMC) fueron significativamente mayores en el sitio de descanso. Sin embargo, los machos presentaron menor IMC, lo cual resulta en una menor cantidad de reservas energ\'{e}ticas. El aumento de estr\'{e}s fisiol\'{o}gico (H:L) puede deberse al efecto de la migraci\'{o}n y al proceso de muda que se lleva a cabo en el sitio de descanso, ejerciendo un costo energ\'{e}tico al organismo. Aunque la prevalencia no mostr\'{o} ningún efecto significativo en las variables predictoras, la parasitemia fue significativamente m\'{a}s elevada en machos que en hembras. Dichos resultados podr\'{i}an, deberse a  factores intr\'{i}nsecos resultado de la temporada de reproducci\'{o}n en los machos, el cual tiene un efecto inmunosupresor. Es necesario, identificar la importancia del sitio de descanso en las aves migratorias, ya que esta informaci\'{o}n servir\'{a} para comprender de mejor manera la funci\'{o}n del sitio de descanso.}

\normaltalk{Effect of the Shiny Cowbird (Molothrus Bonariensis) Brood Parasitism over the Rufous-Collared Sparrow (Zonotrichia Capensis) Reproductive Success in Bogot\'{a}, Colombia}{Andr\'{e}s F Sierra}{The brood parasitism may be the explanation for the decrease in the population of some bird species, because this strategy generally carries negative effects to the host. In Bogot\'{a} some authors noticed a reduction in the abundance of some species being the brood parasitism the possible cause. During a year I searched Rufous-Collared Sparrow (Zonotrichia capensis) nests, a common host of the Shiny Cowbird (Molothrus bonariensis), with the aim of characterize and quantify the effect of the parasitism over the host reproductive success. Almost the half of the nests were parasitized, with a 3.3 mean parasite eggs (parasitism intensity) in those nests. I found statistical significative differences in the production of fledglings and in the hatching success. A high intensity of parasitism, multiparasitism, may cause that the host adults abandon their nest. The desertion of nests and the removal of eggs by the parasite females explains the lower fledgling production of parasitized nests. Thus, the negative effect over this aspect of the reproduction reduce the availability of new individuals to be recruited for the population.}

\normaltalk{New Avian Hosts for Taphropiestes Spp. (Coleoptera: Cavognathidae) and the First Report of Dermatitis Causing the Death of Nestlings}{Exequiel Gonzalez, Adri\'{a}n Jauregui, Luciano N Segura}{Among bird parasites, there is a poorly known group of beetles, Taphropiestes spp. (Coleoptera: Cavognathidae) that inhabits birds' nests. There is scarce information about their life cycle and whether they have direct interaction with their hosts. As part of a breeding biology study conducted in a forest in northeastern Buenos Aires Province, Argentina, we systematically searched and monitored nests of five species, (Pyrocephalus rubinus, Polioptila dumicola, Pipraeidea bonariensis, Satrapa icterophrys, and Elaenia parvirostris), and not systematically of other passerine species that nest in the same forests. We found 13 out of 700 nests with signs of Taphropiestes: six of P. rubinus, one of P. dumicola, three of S. icterophrys, one of Mimus saturninus, one of Tyrannus savana and one of Paroaria coronata. In all the cases we observed dermatitis, which consisted of lacerations and crusts, on the belly and chest regions, areas in direct contact with the base of the nest. In one nest of P. rubinus we registered the death of one of the two nestlings present as a result of the multiple wounds produced by the larvae, and in one nest of P. dumicola we registered the death of the only nestling in the nest, which lead to nest failure. This study provides for the first time detailed data of dermatitis caused by Taphropiestes spp. larvae in several species and its lethal effect on birds' nestlings, broadening the information available on the impact this parasite has on its hosts.}

\normaltalk{Genetic Structure and Biogeography of the Blue-Gray Gnatcatcher (Polioptila Caerulea)}{Darwin M L\'{o}pez, John Klicka, Kevin Epperly}{The Blue-Gray Gnatcatcher (Polioptila caerulea) is a migratory songbird with a wide breeding range across North America. A number of gnatcatcher subspecies have been described based on morphological variation, suggesting that this species may be divided into two or more isolated and independently evolving populations. Whether or not subspecies boundaries also coincide with genetic structuring of populations has not yet been assessed and their evolutionary history (biogeography) has not been explored. Tissue samples from the Burke museum specimens from across the gnatchatcher's breeding distribution were used for genetic extraction and sequencing (mitochodndrial DNA) in order to identify isolated (genetically structured) populations. Three phylogenetically distinct groups (haplogroups) were recovered, one in the eastern U.S., one in interior regions of the western U. S. and Mexico, and one along the Pacific coast. Montane barriers likely played role in the diversification process. Western and eastern birds are separated by Rocky Mountain and Great Plains habitat barriers while the two western groups are divided by the Sierra Nevada range. The development of haplogroup-specific migration patterns has probably also played a role, yielding populations that are now isolated in both their breeding and wintering areas. Future work will require genomic data to assess levels of gene flow in regions of secondary contact and song and morphometric data will be used to better define distributional limits.}

\normaltalk{How Does the Distribution of the Saffron-Cowled Blackbird Change over Time?}{Amanda P Marcon, Carla S Fontana, Glayson A Bencke, Adrian Di Giacomo, Adri\'{a}n B Azpiroz, Fabiana G Barbosa, Rafael A Dias}{Understanding the dynamics of the distribution of a species at different timelines is important to develop conservation guidelines. The Saffron-cowled Blackbird is a vulnerable grassland bird with a poorly known distribution. We compiled presence records of this species from the literature, museums and on-line citizen-science databases (1970-2018) and developed Ecological Niche Models for three time periods: (i) past (Last Glacial Maximum [LGM] and Middle Holocene [MH]), (ii) present and (iii) future (Representative Concentration Pathways [rcp] 4.5, 6.0 and 8.5 for 2070). Environmental data consisted of temperature and precipitation for all periods, using the Community Climate System Model 4.0 for the past and future. We previously selected the best feature classes and regularization multiplier, and modelled with Maximum Entropy. Northeastern Rio Grande do Sul (RS), Brazil, was suitable in all periods. Western RS and southern Paraguay (78,450 km2) were suitable during the LGM. Areas with the highest suitability during the MH (93,457 km2) and the present (49,422 km2) ranged from Santa Catarina (SC), Brazil, to Buenos Aires, Argentina. Rcp4.5 (45,643 km2) and rcp6.0 (48,138 km2) included part of SC and Paran\'{a}, Brazil. Rcp8.5 (38,893 km2) indicated a reduction of suitability towards northeastern RS, and, as rcp6.0, included western RS. Northeastern RS is an important area for conservation actions. Climate changes during LGM-MH propitiated extension to southernmost areas. The current restriction on suitability might relate to anthropic alterations. Future models predicted habitat constrictions to the south, possibly due to a warmer and drier climate that will negatively influence its breeding habitat.Compreender as dinâmicas distribucionais de uma esp\'{e}cie em diferentes per\'{i}odos de tempo \'{e} importante para desenvolver diretrizes conservacionistas. O veste-amarela \'{e} uma ave campestre vulner\'{a}vel de distribuição pouco conhecida. Compilamos registros de presença dessa esp\'{e}cie na literatura, museus e databases de ciência cidadã on-line (1970-2018) e desenvolvemos Modelos de Nicho Ecol\'{o}gico para três per\'{i}odos de tempo: (i) passado (Último M\'{a}ximo Glacial [UMG] e Holoceno M\'{e}dio [HM]), (ii) presente e (iii) futuro (Caminhos de Concentração Representativa [ccr] 4.5, 6.0 e 8.5 para 2070). Dados ambientais consistiram de temperatura e precipitação para todos os per\'{i}odos, usando o Modelo de Sistema Clim\'{a}tico Comunit\'{a}rio 4.0 para passado e futuro. Previamente selecionamos as melhores classes de caracter\'{i}sticas e multiplicadores de regularização, e modelamos com Entropia M\'{a}xima. O nordeste do Rio Grande do Sul (RS), Brasil, foi adequado em todos os per\'{i}odos. O oeste do RS e sul do Paraguai (78.450 km2) foram adequados no UMG. Áreas de adequabilidade alta no HM (93.457 km2) e o presente (49.422 km2) foram desde Santa Catarina (SC), Brasil, at\'{e} Buenos Aires, Argentina. Ccr4.5 (45.643 km2) e ccr6.0 (48.138 km2) inclu\'{i}ram parte de SC e Paran\'{a}, Brasil. Ccr8.5 (38.893 km2) indicou redução de adequabilidade ao nordeste do RS, e, como ccp6.0, incluiu o oeste do RS. O nordeste do RS \'{e} importante para ações conservacionistas. Mudanças clim\'{a}ticas durante UMG-HM propiciaram extensão ao sul. A atual restrição de adequabilidade pode estar relacionada a alterações antr\'{o}picas. Modelos futuros previram constrições de habitat ao sul, possivelmente devido ao clima mais quente e seco que negativamente influenciar\'{a} o habitat de reprodução.}

\normaltalk{Do Birds of Two Dry Forest in South America Follow Ecogeographical Rules?}{Mayara A Tanaka, Hevana S Lima, Gustavo S Cabanne, Luciano N Naka}{Bergmann and Allen's rules state that endothermic animals within a broadly distributed taxonomic clade will show differences in body and appendage sizes that are related to latitude and climatic variation. Here, we tested the validity of these rules on birds with allopatric populations in the two largest nuclei of Neotropical Dry Forests along the South American Dry Diagonal: the Caatinga of northeastern Brazil and the Chaco from SE Bolivia, W Paraguay, and N Argentina. We selected seven species/allospecies with allopatric populations/taxa along the Dry Diagonal, including Myrmorchilus strigilatus, Stigmatura budytoides, Suiriri suiri, Xolmis irupero, Thectocercus acuticaudatus, Agelaioides fringilarius/badius, and Icterus jamacaii/croconotus. For these taxa, we obtained morphometrical measurements to evaluate the relationship between body and appendages size and latitude. Despite finding evidence of geographic variation in morphology among five of the seven pairs investigated, we do not find strong evidences to validate Bergmann's rule, and only the complex Suiriri suiriri and Icterus jamacaii/croconotus was in concordance with the expectations of Allen's rule, with reduced bill and wing in the Chaco's higher latitudes. These results suggest that these ecogeographic rules may not be applied in every condition, and that within the dry tropics, other environmental variables, such as drought regularity, precipitation, and maximum temperature may be more important than latitude to define patterns of geographic variation.  As regras de Bergmann e Allen afirmam que animais endot\'{e}rmicos dentro de um clado taxon\'{o}mico amplamente distribu\'{i}do mostrarão diferenças no tamanho do corpo e dos apêndices corporais, que estão relacionadas com a latitude e a variação clim\'{a}tica. Aqui, testamos a validade destas regras em aves com populações alop\'{a}tricas nos dois maiores núcleos de Florestas Secas Neotropicais ao longo da Diagonal Seca Sul-Americana: a Caatinga do nordeste do Brasil e o Chaco do SE Bol\'{i}via, W Paraguai, e N Argentina. Selecion\'{a}mos sete esp\'{e}cies/aloesp\'{e}cies com populações alop\'{a}tricas/taxa ao longo da Diagonal Seca, incluindo Myrmorchilus strigilatus, Stigmatura budytoides, Suiriri suiri, Xolmis irupero, Thectocercus acuticaudatus, Agelaioides fringilarius/badius, e Icterus jamacaii/croconotus. Para estes tax\'{o}ns, obtivemos medidas morfom\'{e}tricas para avaliar a relação entre o tamanho do corpo e os apêndices e a latitude. Apesar de encontrarmos evidências de variação geogr\'{a}fica na morfologia entre cinco dos sete pares investigados, não encontramos evidências fortes para validar a regra de Bergmann, e em apenas os complexos Suiriri suiriri e Icterus jamacaii/croconotus em concordância com as expectativas da regra de Allen, com bico e asas reduzidas nas latitudes mais elevadas do Chaco. Estes resultados sugerem que estas regras ecogeogr\'{a}ficas não podem ser aplicadas em todas as condições, e que dentro dos tr\'{o}picos secos, outras vari\'{a}veis ambientais, tais como regularidade da seca, precipitação, e temperatura m\'{a}xima podem ser mais importantes do que a latitude para definir padrões de variação geogr\'{a}fica.}

\normaltalk{What Do Birds Tell Us on Grassland Conservation? Recent Findings on Community and Population Studies in Southern Brazil and Uruguay}{Carla S Fontana, M\'{a}rcio Repenning, Eduardo Chiarani}{In Brazil about 25\% of grasslands have been lost due to changes in the use of soil along the last three decades. Overgrazing and non-prescribed fire affects directly the grasslands dynamic (e.g. species richness and bird reproduction) in the southeastern South America grasslands. Considering the importance of these open environments to conservation, we intend show some results of studies conducted with grassland birds in southern Brazil and Uruguay in the last decade. We found that there is a relationship among bird community diversity and grassland management with cattle and fire. Generally, bird diversity do not change in native and managed areas; however, some grassland dependent species like Anthus spp., Emberizoides ypiranganus, and Sporophila spp., may be favored or not by specific vegetation structures maintained by fire and cattle. This is relevant for conservation of specific populations, which is desirable especially for threatened species. Additionally, habitat modification affects differently migratory or resident and generalist or specialist species. We demonstrate that territories of grassland seedeaters and insectivores species differed in response to fire, decreasing the number of territories immediately or one year after the fire, and reflecting on the recovery time for the population after the disturbance. Using light-level geolocators, we tracked and understood the migratory annual-cycle of Seedeaters, identifying crucial stopovers and how connectivity areas between temperate/sub-tropical grasslands (breeding zone) and Cerrado's savannahs (wintering areas) can occur. We begin to have a good scientific base to assist farmers, government and non-government agencies in the natural grassland's management for conservation}

\normaltalk{Urbanization Has Opposite Effects on Territory Size of Two Passerine Birds}{Roselvy Ju\'{a}rez, Eduardo Madrigal-Chac\'{o}n, Luis Sandoval}{Urbanization has been identified as one of the main threats for biodiversity, because it negatively affects survival, breeding success, and territory size. Understanding how different birds adjust territory size in response to urbanization is essential to better understand why some species are lost and others persist in urban environments. We evaluated the effect of urbanization on territory size of an urban avoider species, White-eared Ground-Sparrow (Melozone leucotis), and an urban adapter species, House Wren (Troglodytes aedon), at five Costa Rican sites. We measured the size of 30 ground-sparrow and 28 wren territories during 296 h of observation. We followed each individual for at least one hour per day for two days in two consecutive years and geo-referenced their locations. Territory size was estimated using the minimum convex polygon method. We measured urban surface and natural surface within territories. Ground-sparrow territory size decreased from high to low-urbanized sites. Wren territory size increased from high to low-urbanized sites. We found a positive relationship between urban surface and territory size for the ground-sparrow, but not for the wren. Ground-sparrows probably need to defend larger territories in urban environments to find all the resources required to survive because urban environments may provide insufficient resources such as food or shelter. Meanwhile, wrens defend smaller territories in urban environments where even small territories may provide sufficient resources. These results suggest specific behavioral adaptations developed by birds inhabiting urban environments.}

\normaltalk{Which Migratory and Over-Wintering Birds Use Residential Areas Versus Urban Forest Fragments?}{Ryan E Buron, Mark Hostetler}{As urban environments grow, it is critical to understand how birds utilize urban landscapes. Both residential areas and urban forest remnants can provide habitat for over-wintering birds and serve as stopover habitat for migrants. However, it is not clear which avian species use forest fragments versus residential areas and how species dynamically shift habitat preference throughout the year. Our objectives were to determine: (1) which birds primarily use urban forest fragments versus adjacent residential areas during migration and winter seasons; (2) and which avian functional traits (foraging, dietary, migratory, and vertical distribution guilds) are associated with birds that primarily use urban forest fragments versus residential areas. Conducted in Gainesville, Florida, 39 random point counts were situated in urban forest fragments, forest edges, and residential areas. Preliminary results suggest that species shift habitat needs throughout the year, potentially driven by differences in vegetation structure between fragments and residential areas. Also, birds that have been classified as ``specialists" (i.e. forest interior specialists) utilize a variety of habitats, including residential areas, during winter and migration. Results from the study will help conservationists, urban planners, city designers, and landscape architects identify important avian habitats in the urban matrix.}

\normaltalk{Persistence of Secondary Cavities as Nest Sites of the Endangered Red-Crowned Parrot in an Urban Landscape}{Kelley K Sampeck, Simon R Kiacz, Donald J Brightsmith}{Populations of secondary-cavity nesting bird species are often limited by the availability of suitable cavities for breeding, especially in anthropogenically modified landscapes. The Lower Rio Grande Valley of South Texas supports a population of endangered Red-crowned Parrots, an obligate cavity nester, that inhabits strictly urban areas. After decades of increase, the growth rate of this south Texas population has stagnated over the past four years, and the reasons for this remain unknown. The trees that provide a majority of nest sites for Red-crowned Parrots in this area are dead palms. Dead palms are an important parrot nesting substrate worldwide, but they often have a very high rate of turnover compared to other types of trees that support cavities. A scarcity of suitable cavities may be limiting the growth of the population, but the basic knowledge necessary to be able to model the persistence and turnover of these cavities is lacking. In order to investigate nest site persistence and rates of nest site loss in south Texas, we revisited 90 historic palm nesting trees first discovered between 2016 and 2018. Nests were checked on average 2.6 years after discovery, and over this time 45.6\% of palm trees were lost. The relatively rapid rate of palm tree loss highlights the need for urban planners and private residents to continue to plant palms in order to maintain important nesting substrate for this population of threatened parrots.}

\normaltalk{Protocol-Adherence Behaviors of Citizen Scientists Working with the American Kestrel Partnership}{Leticia M Camacho, Sarah E Schulwitz, Jeremiah Sullivan, Chris J McClure}{Citizen scientists can aid researchers interested in tracking continental-scale patterns due to their potential to collect data across vast geographic expanses, given that they adhere to established protocols. The American Kestrel Partnership (AKP) is a citizen science-based project of The Peregrine Fund that aims to gain a greater understanding of demographic trends in the widespread but declining falcon, American Kestrel (Falco sparverius). AKP partners across North America are asked to follow protocols regarding installing and consistently monitoring kestrel nest boxes during the breeding season. Here, we analyze nest box characteristics (e.g., entrance orientation, mounting surface, if cleaned out annually, etc.), protocol-adherence behaviors by partners (e.g., following AKP's instructions for box installation and monitoring), and the relationship of box monitoring to the number of boxes a partner owned (i.e., $<$10 vs. $>$10 boxes). We found that most nest boxes were installed according to AKP's basic guidelines and most were cleaned annually. On average, partners that owned ten or more boxes were more likely to monitor their boxes each year since box installation. Regardless of installation year or the number of boxes a partner owned, however, we found that box monitoring decreased steadily with each year since installation. This study elucidates the tendencies of AKP's citizen scientists and informs program managers on topics that may require alternative communication strategies.}

\normaltalk{Agroecosystems in Santander, Colombia: source of ornithological diversity and focus of research at the Industrial University of Santander}{Ang\'{e}lica Rodr\'{i}guez-Rey, Diana Carolina V\'{a}squez-Gonz\'{a}lez, Fernando Rond\'{o}n-Gonz\'{a}lez}{Colombia has a high number of biomes and ecosystems, in which the greatest taxonomic diversity of birds worldwide resides. The Santander department is located in the northeast and has three of the four strategic ecosystems in the country. Its topographic heterogeneity and variety of ecological conditions, such as altitude (40 m asl to 4345 m asl) and water resources (rivers and swamps), are used for anthropogenic interventions that include agricultural activities, which alter habitats and can accelerate loss of species. Although there is no consensus on the total number of bird species present in Santander, recent estimations indicates that it is home to nearly 50\% of the diversity reported for Colombia. Based on the above, since 2011 the Laboratory of Genetics and Biotechnology belonging to the Industrial University of Santander has generated lines of research that allow addressing aspects related to loss of taxonomic diversity, evaluation of genetic and functional diversity, involvement of hemoparasites among others, having as an object of study the avifauna associated with agroecosystems. The present contribution analyzed 1418 specimens, of which 220 species were identified, in an investigation aimed at detecting hemoparasites causing avian malaria. Thraupidae, Tyrannidae and Parulidae were identified as representative families, in agroecosystems mainly conformed by Citrus sp., Coffea sp., Rubus sp. and Theobroma cacao, in addition to the fact that taxonomic diversity is concentrated between 677 m.a.s.l. and 1583 m.a.s.l. The results help to highlight Santander's potential for birdwatching in rural agro-ecological zones.}

\normaltalk{Are elevational distributions of tropical birds limited by parasites? Insights from two biodiversity hotspots}{Paulo Pulgarin, Alexandra Class-Freeman, Benjamin Freeman, Robert Ricklefs, Daniel Cadena}{A core goal of ecology and biogeography is to disentangle the factors responsible for species distribution and diversity patterns. In tropical mountains, species are often restricted to narrow elevational zones and closely related species replace each other sharply along elevational gradients. Parasites have been hypothesized to be decisive agents in structuring host assemblages in various contexts, including elevational zonation. We analyzed the distribution of haemosporidian parasites (Plasmodium, Haemoproteus and Leucocytozoon) and their avian hosts along elevation gradients in two tropical mountain ranges: the Sierra Nevada de Santa Marta in northern South America, and the YUS Conservation Area in Papua New Guinea, to provide a first assessment of the plausibility of the hypothesis that parasites are involved in limiting the elevational ranges of tropical birds. We screened a total of 3088 individuals, of which 239 (29.1 \%) individuals were infected with haemosporidian parasites in Santa Marta, and 1031 (45.4\%) in New Guinea. Parasite lineages occupied wider elevational ranges in Santa Marta than in New Guinea, where most lineages were narrowly restricted with respect to elevation. Although regions differed in parasite diversity and degree of host-parasite specialization (both were higher in New Guinea), we found evidence for reciprocal elevational zonation between haemosporidian parasites and closely related species of bird as well as variation in prevalence patterns between bird species replacing each other with elevation in both mountain systems. Although these patterns are consistent with predictions of the hypothesis that parasites limit host geographic ranges, additional work including studies of vector ecology and experiments are necessary to determine whether patterns in fact imply causal relationships.}

\normaltalk{A multi-step approach to understanding the effects of backyard bird feeding on the abundance and diversity of birds in San Antonio, Texas}{Amanda Lamberson, Jennifer Smith}{Provision of wild bird food to birds is a pervasive activity in developed nations. Previous studies of potential effects of feeding wild birds have mainly occurred in temperate areas, in non-urban settings, and during the breeding season. Using a multi-step approach, we evaluated effects of wild bird food on the abundance and diversity of birds during the non-breeding season in San Antonio, Texas, an urban setting in a subtropical region as follows: 1) sales data were used to investigate what type, when, and where wild bird food is provided, 2) we used data from Objective 1 to inform a field-based study, and 3) we used surveys of residents to acquire site specific information about bird feeding habits. Results suggest that mixed seed and Nyjer seed were the two most commonly purchased and available foods. We randomly allocated 36 sites with either mixed seed, Nyjer seed, or no food (control) and used point counts to assess effects of food type on the diversity and abundance of birds. Preliminary results suggest that bird diversity was higher at sites provided with bird food compared to control sites, while bird diversity was equivalent between mixed seed and Nyjer seed sites. Treatment type effects on abundance of birds varied by species; Lesser Goldfinches Spinus psaltria were more abundant at both Nyjer and mixed seed sites whereas there was no effect of food type on the abundance of Eastern Phoebes Sayornis phoebe. Our results highlight that bird food likely contributes to effects of urbanization on birds.}

