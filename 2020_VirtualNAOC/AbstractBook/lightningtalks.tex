\normaltalk{Does Size Really Matter? Factors Affecting Juvenile Recruitment in a Neotropical Migratory Songbird}{Adele M Balmer, Lesley P Bulluck, Derek M Johnson}{Juvenile recruitment is a key life history trait that significantly influences population growth. Despite its importance, juvenile recruitment is often poorly understood because it is difficult to measure for mobile organisms. The rapid population declines seen in many migrant songbirds have increased the need to identify factors influencing juvenile mortality. The goal of this study was to understand how temperature and precipitation influence nestling body condition (mass adjusted for age), and to investigate how nestling body condition, brood size, mother's age, and hatch date influence juvenile recruitment in female Prothonotary warblers (Protonotaria citrea). Prothonotary warblers are Neotropical migratory songbirds known to have high site fidelity, making them an ideal species for studying population dynamics. We found an average annual recruitment rate of 6.48\% ± 2.86\% (range 0 - 8.78\%) for females hatched in our study population (n = 2,366) in Virginia, between 2010 and 2018. Hatch date and body condition were the strongest predictors of recruitment, with higher recruitment for females hatched earlier in the season. Body condition was positively correlated with recruitment only for birds hatched later in the season. We also found that an eight-day precipitation window starting three weeks before the hatch date positively correlated with nestling body condition (p $<$ 0.001). Temperature, interestingly, was not associated with nestling body condition. We plan to incorporate winter precipitation as predictors of recruitment and then use these results to model future population fluctuations under variable climatic conditions.}

\normaltalk{High Genetic Connectivity in a Threatened Songbird Despite Substantial Habitat Loss and Population Decline--the Tricolored Blackbird}{Kelly R Barr, Annabel C Beichman, Pooneh Kalhori, Jasmine Rajbhandary, Rachael A Bay, Kristen Ruegg, Thomas B Smith}{Much like the Passenger Pigeon, the Tricolored Blackbird once darkened the skies over California. Early naturalists observed breeding colonies that built over 100,000 nests in the early 1930s. Over the next century, the species lost most of both its preferred wetland nesting habitat and grassland foraging habitat, and as a consequence experienced a precipitous population decline by over an estimated 90\%. Using a de novo genome for the species, a large sample-set that covers their entire range, and restriction site associated sequencing (RAD-Seq), we analyzed genetic variation patterns with high resolution to assist the conservation and recovery of the species. We found neither limitations to gene flow and nor evidence for declining genetic diversity. Tricolored Blackbirds are panmictic across their range, including between colonies separated by wide areas of unsuitable habitat. We did detect evidence for a major demographic contraction by about 50\% at around 20,000 generations in the past. It is likely this is associated with Pleistocene climate change. The high levels of gene flow observed here suggest Tricolored Blackbirds are resilient to habitat fragmentation. These results suggest the species can be managed as a single evolutionary unit, and conservation efforts should be focused on protecting the largest remaining colonies and restoring foraging habitat in the core of their range.}

\normaltalk{Sex-Specific Causes and Consequences of Variable Testosterone Circulation in the White-Shouldered Fairywren (Malurus Alboscapulatus)}{Jordan Boersma, Erik D Enbody, John A Jones, Doka Nason, Jordan Karubian, Hubert Schwabl}{There is considerable debate about whether testosterone regulates traits similarly in male and female vertebrates. Meta-analyses of the relationship between male and female circulating testosterone have produced conflicting results, highlighting the need for empirical studies in species that exhibit variation in both male and female testosterone and the traits this hormone is known to mediate. The White-shouldered fairywren (Malurus alboscapulatus) demonstrates considerable subspecies-specific variation in testosterone circulation, behavior, and female ornamentation. Testosterone circulation in females appears to be linked to discrete female phenotypes: females from the subspecies with greater ornamentation and aggression circulated higher baseline testosterone, and unornamented females implanted with testosterone produced a major component of ornamental plumage and became more aggressive once the putative signal was acquired. Interestingly, males exhibited opposite patterns in baseline testosterone, as males from the unornamented female subspecies circulated the highest mean testosterone. Our findings challenge the idea that testosterone circulation is correlated and has similar function between sexes. We also present preliminary results from a long-term dataset quantifying variation in testosterone, social networks, and male sexual displays. We address how the social environment influences a suite of functionally linked traits and explore the causes and consequences of variable testosterone circulation between sexes.}

\normaltalk{Efficacy of Elemental Contaminant Detection in Developing Feathers from Seven Species of Birds of Prey in Central Florida}{Jennifer T Bouchenot, Melanie J Beazley, Samantha S Little, Eric W Goolsby}{Birds are a popular subject for elemental contaminant detection because of their predictable behavior and sensitivity to anthropomorphic influences. While feathers are often utilized as a non-destructive sample type, their resulting contaminant values are varied, even within the same individual. This uncertainty has made internal and destructive sampling a more reliable metric for measuring contaminant values. This study examines the efficacy of utilizing a blood feather as a novel non-destructive sample type.    The blood feather is an actively developing feather on a bird. Due to a unique staunching system, a pulled blood feather rapidly shunts the dermal papilla to stop any bleeding, and regeneration of a new feather begins. The growing blood feather is still physiologically analogous to the organism. This may yield a smaller variation in measured contaminant data within an individual and have a stronger correlation with internal values.    This study explores the efficacy of using a novel sample type, the blood feather, in quantification of elemental contaminants in seven species of Central Florida birds of prey using data obtained from an Inductively Coupled Plasma with Mass Spectrometry. Federal and state protected specimens are obtained from the Audubon Center for Birds of Prey, a local rehabilitation center. Samples are collected from birds that have succumbed to their injuries sustained in the wild. These include Red-shouldered Hawk, Osprey, Barred Owl, Cooper's Hawk, Black Vulture, Red-tailed Hawk, and Eastern Screech-Owl. Contaminant measurements from livers, feathers, and developing feathers of these species are analytically compared.}

\normaltalk{Rediscovering the Striped Sparrow in Aguascalientes, Mexico.}{Daniel A Carrillo Martinez}{Striped Sparrow (Oriturus superciliosus) is endemic to the highlands of north-western and central Mexico. It occurs in grasslands, mountain scrubs and other mountain forests in the Sierra Madre Occidental and on the Central Plateau. In the state of Aguascalientes, the mountain grassland remnants are found in two isolated sites that represent a small area on the highest zones of the state. This species was recorded by O. Salvin and F. D. Godman in 1888, and after 70 years, it was seen and collected by N. K. Johnson in 1959. The Striped Sparrow was not registered again in the state for 57 years until 2016, in a small and isolated population, while a study about the avifauna of the area was being done. Seven individuals were identified in an area known as Playa Cebolletas in Monte Grande, a common zone in Sierra Fria. Few months later, more individuals of the same species were registered in a close area called Playa Mariquitas. Information about the behavior, including roles when feeding were recorded: some individuals acted as sentinels while others where foraging. I noticed some pauses and specific stops when singing, when other individuals waited to sing until the current singer was done. Finally, I reported some mixed groups of this species with Lincoln's Sparrows. These sedentary and isolated populations allow us to do specific observations about this specie's behavior, also helping us to get the attention needed, so conservation efforts can be done in Sierra Fria, Aguascalientes.}

\normaltalk{Home-Range Size of an Andean Bird: Assessing the Role of Physical Condition}{Maria I Casta\~{n}o, Carlos D Cadena, Jorge E Avenda\~{n}o}{Because space-use patterns are a key aspect of the ecology and distribution of species, identifying factors associated with variation in size of territories and home ranges has been central to studies on population ecology. Space use might vary in response to extrinsic factors like habitat quality and to intrinsic factors like physical condition and individual aggressiveness. However, the role of these factors has been poorly documented in the tropics, particularly in high-elevation bird species. We report the home-range size of a Neotropical Andean bird, the gray-browed brush finch (Arremon assimilis), and evaluate the role of physical condition in explaining variation in home-range size among individuals. We performed spot mapping to estimate the home ranges of 14 territorial males in Bogota´, Colombia, using minimum convex polygons (MCP) and 95\% kernel density estimators (KDE). The mean home-range size estimated for the 100\% MCP was 0.522 ± 0.305 ha (range = 0.15--1.18 ha), whereas the 95\% KDE estimation was 0.504 ± 0.471 ha (range = 0.13--1.88). We calculated the real mass index of each bird as a proxy of physical condition to assess whether individuals in better physical condition had larger home ranges. Because we found no relation between our estimations of physical condition and home-range size, we conclude that space use in this species might depend more on ecological factors such as habitat quality or neighbor density than on individual traits.}

\normaltalk{"Chancing on a Spectacle:" Co-Occurring Animal Migrations and Interspecific Interactions}{Emily B Cohen, Dara A Satterfield}{Migrations of diverse wildlife species often converge in space and time, with their journeys shaped by similar forces (i.e., geographic barriers and seasonal resources and conditions); we term this ``co-migration." Beyond their significance as natural wonders, species with overlapping migrations may interact ecologically, with potential effects on population and community dynamics. Direct and indirect ecological interactions between migrant species remain poorly understood, in part because migration is the least-studied phase of animals' annual cycles. To address this gap, we conducted a literature review to examine whether animal migration studies incorporate multiple species and to what extent they investigate interspecific interactions between co-migrants. Among animal migration research papers published in 23 relevant peer-reviewed journals during 2008-2017, thirty percent reported two or more species with coinciding migrations, suggesting that co-migrations are common, although few of these studies investigated interactions among migrating species. Synthesizing those that did explore this phenomenon, we found examples of five types of ecological interactions between migrating species, including predator-prey, host-parasite, and commensal relationships. Deepening ecological knowledge of interspecific interactions among migratory animal communities will enhance understanding of the drivers of migration and could improve predictions about wildlife responses to global change. Further research focused on multi-species migrations could also inform conservation efforts for migratory animal populations, many of which are declining or shifting, with unexplored consequences for other co-migratory species.}

\normaltalk{Ecological Constructs Underlying Landscape Conservation: The Fundamental Role of Coffee Shade Restoration in West-Central Puerto Rico.}{Jaime A Collazo, Krishna Pacifici, Brian J Reich}{A common conservation objective is to maximize persistence of wildlife species at landscape scales, but increasingly, it is realized that achieving the stated objective will require greater reliance on integrating human-modified landscapes. This is the centerpiece of a strategy to promote avian persistence in west-central Puerto Rico--integrating non-protected, shade and shade-restored coffee plantations to smooth the matrix among protected forest tracts. Ushering such a strategy requires economic cost considerations (When?), but also an understanding of the functional relationship between management actions and ecological benefits (Where and How?). We showed that species richness and a diversity index, expressions of representation and redundancy, did not differ statistically among sun and shade coffee plantations, and secondary forest tracts across the landscape. We also showed that shade restoration, a key management action, promotes local avian persistence (resiliency) while retaining most of the ecological services available in intact, protected habitat. We illustrate the implementation of a strategy that builds upon existing resources and relationships between conservation agencies and farms implementing best management practices. We illustrate other implementation scenarios optimized to maximize landscape avian persistence and minimize costs given equal or varying conservation value among multiple tracts of protected habitat.}

\normaltalk{Breeding Biology and Nest Survival of Hellmayr's Pipit (Anthus Hellmayri) in Natural Grasslands in Buenos Aires Province, Argentina}{Mart\'{i}n A Colombo, Luciano N Segura}{Grasslands in central-east Argentina are among the most altered ecosystems in the world, and the biology of many native bird species remains poorly studied while the landscape keeps changing. Information on breeding biology of birds is important to understand their population trends. We present the first study on the breeding biology of Hellmayr's Pipit Anthus hellmayri, a poorly known ground nesting passerine, and we analyze the effects of habitat characteristics on its nest daily survival rate (DSR). We searched for nests during three breeding seasons (2017 to 2019) in a natural grassland area in Buenos Aires province, Argentina. For each nest we measured habitat features to assess their effects on DSR. The breeding season lasted from September to February, clutch size was 3-5 eggs, incubation lasted 13-14 days and nestling period lasted 10-14 days. DSR was 0.91, which results in a 7\% probability of surviving the entire breeding cycle. Predation was the main cause of nest failure, followed by abandonment caused by weather. DSR was negatively associated with visual obstruction and was higher in nests with lateral entrances than in those with vertical entrances. Less visual obstruction could be advantageous by allowing a greater field of view from the nest, and a lateral entrance could offer better protection from rain and aerial predators. Estimated success was very low compared to other grassland bird species, and may indicate that this Pipit population is declining. We emphasize the importance of preserving these habitats and monitoring populations of grassland birds.}

\normaltalk{Bird-Habitat Relationships Across Life Stages in Grassland Songbirds of the Northern Great Plains}{Maureen D Correll, Jacy S Bernath-Plaisted, Nicole M Guido, Katharine J Ruskin, Brian J Olsen, Arvind O Panjabi}{Few studies have simultaneously examined breeding season adult and juvenile survival in North American passerines relative to nesting success. These demographic rates often drive population dynamics such that management strategies may need to be tailored to specific life-history stages. Grassland songbirds continue to be among North America's most rapidly declining avian groups, and knowledge of factors that influence demographic vital rates is needed to address these declines. We collected radio telemetry data on adult and juvenile birds from 2015-2018 as well as nesting success data on Baird's sparrow and grasshopper sparrow breeding in western North Dakota and northeastern Montana. We modeled a suite of climate, temporal, and vegetative covariates to explain demographic rates and habitat selection in grassland birds and found that cumulative adult survival, nesting success survival, and juvenile survival was (79\%, 41\%, 23\%, respectively) for Baird's sparrow, and (74\%, 17\%, 54\% respectively) for grasshopper sparrow. None of the vital rates were not well-explained by the ecological variables we modeled, but vegetation structure and composition did predict juvenile survival for both species. Further, our study highlights that grassland birds prefer a narrow range of vegetation height and density for habitat nest site selection. These data and findings can help inform management recommendations as well as feed into larger integrated population models to identify when and species are limited within the annual cycle.}

\normaltalk{Conspecific Recognition of the Chatter-Call by Parasitic Shiny Cowbird Molothrus Bonariensis Nestlings}{Ignacio Crudele, Juan C Reboreda, Vanina D Fiorini}{The password hypothesis states that conspecific recognition in brood parasites develops when a young parasite encounters a unique species-specific signal that triggers the learning of other aspects of the producer of the signal. For cowbirds (Molothrus sp.), the chatter-call has been proposed to act as a password for species recognition. The aim of this work was to evaluate if Shiny Cowbird (M. bonariensis) nestlings recognize conspecific chatter-calls. Field work was carried out during October-February 2018-2019 and 2019-2020 at Reserve ``El Destino", Buenos Aires, Argentina. We tested Shiny Cowbird nestlings, reared by Chalk-browed Mockingbirds (Mimus saturninus) and House Wrens (Troglodytes aedon) and host nestlings as control species. Six-day-old nestlings were removed from the nest and we presented randomly the following treatments: 1) chatter-calls, 2) host-vocalizations, 3) non-host-vocalizations. Each treatment lasted one minute and they were interspersed with 5-minute silences. Seventy seven \% of cowbird nestlings reared by mockingbirds (32/42) and 73\% of those reared by wrens (19/26) begged during the chatter-call, without differences between them (GLM, p=0.8). Shiny cowbird nestlings reared by either mockingbirds or wrens responded more to conspecific chatter-calls than to heterospecific calls of mockingbirds (0\%) or wrens (25\%, GLMM, p=0.001). Mockingbird nestlings (N=26) did not beg during treatments. Forty five \% of wren nestlings (10/22) begged during host-vocalizations and 27 \% during chatter-call (GLMM, p=0.002). None of the three species begged during non-host-vocalization. Our results indicate that there is an innate specific recognition of Shiny Cowbird nestlings of the chatter-calls and support the password hypothesis.}

\normaltalk{The Evolutionary Riddle of the Sharpbill (Oxyruncus Cristatus), a Unique Neotropical Bird}{Andres M Cuervo, Sergio Bolivar, Elizabeth Bramlett, Robb T Brumfield, Elizabeth P Derryberry, Luis Silveira, Gustavo A Bravo}{Frank Chapman published The Riddle of Oxyruncus in 1939, describing the many puzzles of this species biology. In particular, the Sharpbill exhits 1) a rather chaotic geographic distribution of plumage color variation, a phenomenon better known as the leapfrog pattern, 2) a perplexing distribution of disjunct populations across a wide range, 3) local rarity in most parts, and 4) high evolutionary distinctiveness. The phylogenetic placement of Oxyruncus within the infraorder Tyrannides has remained enigmatic. Chapman did not have any of these recent data, nor did he know that Oxyruncus occurs in isolated populations of the outlying slopes of the Andes, but he advanced important hypotheses on the biogeography of this unique bird. We revisited The Riddle of Oxyruncus using phylogenetic and historical demographic analyses based on a phylogenomic dataset obtained from voucher specimens collected across its range. We found that Oxyruncus is a long, isolated lineage that is an early divergent and sister to a clade containing Piprites, Neopipo, Calyptura and all extant `Tyrannidae'. We found support for highly divergent clades within Oxyruncus: one from Central America, one from eastern tepuis and the Andes, and one from eastern South America. This structure coincides with examined vocal variation, and partly with phenotypic plumage patterns. Genetic structure across multiple, isolated Andean populations calls for a detailed assessment of species limits beyond our proposal of three species in the genus. We discuss the mechanisms involved in the origin and maintenance of Oxyruncus diversity to start solving its enigmatic patterns.}

\normaltalk{First Looks at Landscape-Level Bird Migration in Colombia Revealed by Weather Surveillance Radar and Acoustic Monitoring}{Jacob R Drucker, Vincent Lostanlen, Nicholas J Bayly, Andrew Farnsworth, Adriaan M Dokter}{Understanding the full annual cycles of migratory birds is fundamental for their conservation, particularly as they become a model system for studying the effects of global change. Technology such as weather surveillance radar, tracking devices, and citizen science databases have provided extensive insight into the landscape-level processes that affect bird populations migrating through North America. Yet how migratory strategies shift with the dramatically different climate and topography of the Neotropics remains poorly understood. Here we report the first documentation of the magnitude and structure of nocturnal migration in the Neotropics, leveraging a network of weather surveillance radars and acoustic monitoring to begin to paint a picture of the phenology and volume of nocturnal migration through Colombia. We find a more heterogeneous distribution of migratory waves in the fall than in the spring that may correspond to wind systems that vary longitudinally across the Caribbean. In the fall we also find more southeasterly headings of birds flying over Western Colombia, and more southwesterly headings over Eastern Colombia, hinting at a migratory divide along the eastern edge of the Andes. In both the eastern and western halves of the country, we detected concentrations of birds at low, mid, and high altitudes, with consistent directionality and higher velocities at mid and high altitudes, suggesting the presence of profitable winds at higher altitudes. We also provide migration traffic estimates during spring and fall. Nocturnal flight call monitoring reveals differences in species composition on either side of the Andes.}

\normaltalk{Ecosystem Engineering is Critical for Maintaining Mountain Plover Habitat in Northeastern Wyoming}{Courtney J Duchardt, Jeffrey L Beck, David Augustine}{The plight of North American grassland birds has been in the spotlight for decades, but recent work by Rosenberg and colleagues (2019) indicates that population declines are even more dire than previously thought. Although habitat loss is a major driver of these declines, the alteration of historic disturbance regimes often compounds these threats to grassland birds. Like many other grassland species, the mountain plover (Charadrius montanus; Near threatened [IUCN]) evolved in the context of intense and spatially variable disturbance regimes including ungulate grazing, fire, and burrowing mammals, and relies on the short vegetation structure that these disturbances create. In the Thunder Basin National Grassland of northeastern Wyoming we found that mountain plovers were almost entirely dependent on disturbance by Black-tailed prairie dogs (Cynomys ludovicianus) for nesting habitat, and were rarely observed off of active prairie dog colonies. Following a Sylvatic plague (Yersinia pestis) event in 2017 $>$90\% of prairie dogs were lost from the landscape, reducing colony cover from 16,000 ha to $<$100 ha. We observed rapid recovery of vegetation on the landscape and a concomitant decline in mountain plovers in 2018. Prairie dogs are often viewed negatively by stakeholders in western landscapes because of potential competition with livestock for forage, but our research highlights the importance of maintaining prairie dogs on public rangelands for mountain plover conservation.}

\normaltalk{Palomarin Field Station Data Explorer: Communicating the Value of a Long-Term Field Station}{Kristen E Dybala, Diana L Humple, Hilary A Allen, Ren\'{e}e L Cormier, Mark D Dettling}{The Palomarin Field Station, founded in 1966, is home to one of the oldest bird banding stations in the U.S., as well as long-term territory mapping, nest monitoring, and weather and vegetation data sets. Collectively, data collected at Palomarin have made numerous contributions to avian natural history, ecology, field methods, and conservation. Yet the value of the field station as a whole is not easily communicated to potential funders and the broader community, and the availability of these data is not readily apparent to potential research partners. To improve our ability to share and communicate about these data, we have developed the Palomarin Data Explorer, a series of web pages with interactive data visualizations to help tell our stories. Each page highlights one or more of our long-term datasets, sharing high-level patterns and observations, why these data are important, and how they are collected. All pages were developed in R using rmarkdown, and code is freely available via GitHub. We encourage other field stations to consider this approach to highlighting their data, and we encourage potential research partners to reach out to us about ideas for projects and grant proposals.}

\normaltalk{Extreme Long-Duration Wind Tunnel Flights Demonstrate Lean Mass  -- not Fat Mass - may Limit Flight Duration in Migratory Blackpoll Warblers}{Cory R Elowe, Derrick J Groom, Julia Slezacek, Alexander R Gerson}{During fall migration, Blackpoll Warblers (Setophaga striata) complete extraordinary long-distance migrations, flying non-stop for over 100 hours from the Atlantic coast of North America to South America. Simulations estimate that a 19 g bird is capable of flying up to 119 hours until fat stores are depleted, leaving a fat-free mass of ~8 g at the destination. Given this extreme arrival condition, we tested the underlying assumptions that (1) long-distance flight is primarily limited by fat stores and (2) protein contributes 5\% of flight energy. We captured Blackpoll Warblers during fall migration and flew them in the wind tunnel at the Advanced Facility for Avian Research at Western University in Ontario, Canada and measured fat and lean body mass before and after flight using Quantitative Magnetic Resonance (QMR). Most individuals flew 6 hours, but three flew voluntarily up to 28 hours -- the longest wind tunnel flight to our knowledge. After these extreme-duration flights substantial fat reserves remained ($>$ 3 g), but concave flight muscle and lean mass of ~8.5 g indicate that flight was limited by lean body mass rather than fat loss. In flights over 8 hours, lean mass contributions dropped from ~10\% to ~3\% of total energy, showing that the majority of lean mass is lost early in flight and that dynamic rates of protein catabolism may limit flight duration more than current models predict. Alongside recent studies demonstrating the effect of environmental conditions on lean mass loss, these extreme-duration flights may be especially sensitive to departure conditions.}

\normaltalk{Variation in Behavioral Syndromes in Brood Parasites of the Genus Molothrus not Explained by Genetic Variation in DRD4 Fragment}{Romina C Scardamaglia, Nicol\'{a}s Memelsdorff, Macarena V\'{a}zquez, Mar\'{i}a I Craig, Ariel E Vagnozzi, Valentina Ferretti}{Shiny cowbirds (Molothrus bonariensis) and Screaming cowbirds (M. rufoaxillaris) are brood parasites that differ in their degree of specialization in the use of hosts, the former being a generalist and the latter a specialist. Given this marked difference in parasitic behavior, we expect them to differ in some aspects of their personality, and for this difference to have a genetic basis. Here we look at: 1. whether there are genetic polymorphisms associated with a region of the DRD4 gene, a gene associated to differences in behavior; 2. whether these polymorphisms are related to different personality traits; and, 3. whether these differences are in turn related to sex and species. We captured individuals of both species in Buenos Aires Province, Argentina. We took a blood sample from each individual for genetic analyses and tested them to measure different personality axes. We found that Shiny cowbird females showed less exploratory behavior and were less neophobic than Screaming cowbird females. In addition, females of both species exhibited greater exploratory behavior than males, and males were less fearful of novel situations in both species; but we found no polymorphisms in the targeted region of the DRD4 gene for the samples analyzed. However, when we compared the amplified sequence to that of other non-parasitic species, we detected a deletion of three base pairs and two substitutions which generate differences in the aminoacidic sequence. Further work should explore the functional implications of these differences, and if they are associated to adaptations to brood parasitic behavior.}

\normaltalk{Hierarchical Modeling of the Response of Great Basin Avifauna to the Area and Fragmentation of Riparian Habitat}{Frank A Fogarty, Erica Fleishman, Alison Ke, Rahel Sollmann, Martha W Zillig, Jian D Yen}{Much research on the community ecology of birds and other taxonomic groups has focused on the theory that species richness increases as habitat area increases and decreases as habitat becomes fragmented. However, many measures of habitat fragmentation and, more generally, landscape pattern are not independent of total habitat area. We developed a new hierarchical model for count data that allows for inference about the effects of covariates at three model levels: site-level detection of individuals, site-level abundance of individuals, and regional presence of species. To the best of our knowledge, this is the first model to integrate detection-weighted estimates of abundance from count data with inferences about covariates that affect regional species richness. We used spatially and temporally extensive data on breeding birds at individual sites within montane canyons in the central Great Basin to examine the relations of individual species' presence and species richness with total riparian area and an appropriate, independent measure of fragmentation at the canyon level. Riparian areas in the Great Basin are management priorities due to their small area and relatively high species richness, and projected decreases in water availability. Our results indicated that total area of riparian cover in a canyon is significantly associated with the presence of many species and species richness, whereas there are few significant associations between fragmentation and these same response variables. Our results suggest that total area of habitat, and not its configuration, is a strong driver of avian community composition in the Great Basin.}

\normaltalk{The Effects of Territory Size and Territory Characteristics on Mating and Nesting Success of Male Cassin's Sparrow (Peucaea Cassinii) in Southern Colorado}{Augustus W Forrest, Garrett M Visser, Claire V Ramos, Nate Bickford}{In recent years, many grassland bird species in North American have been in steep decline. Cassin's sparrows have declined by 3.2 percent per year over the past ten years in Colorado. Very little is known about this species and its behaviors, including home range size and habitat selection. It is not clear how territory size and quality affect mate choice and nesting or fledgling success. Here we investigated how the size and the vegetation characteristics of the male's territory affects reproductive success. Male Cassin's sparrows were caught, color banded, and fitted with CTT life tags on their specific territories. Automated nodes then collected location data every 15 seconds for the length of the breeding season. This allowed for very precise assessment of home range size and habitat use. Nests were found and monitored to determine nesting and fledging success. The anticipated results are that the Cassin's sparrow male's with larger higher quality territories will have higher reproductive success. This knowledge of how the Cassin's sparrows use habitat could provide understanding of which characteristics of the landscape are needed to assist the conservation and persistence of this species.}

\normaltalk{Shifts in Functional Groups of Birds Across Varying Agricultural Landscapes}{Karina Garcia, Elissa Olimpi, Daniel Karp, Erin Wilson-Rankin, David J Gonthier}{The services and disservices provided by birds to agroecosystems can shift with variation in the landscape composition surrounding farms. However, bottom-up (resource concentration) and top-down (predatory bird of prey abundance) effects at the landscape scale may drive the abundance of beneficial and harmful bird groups in different ways. For fruit-eating birds, abundance within farms may be dependent on the concentration of fruit production in the surrounding landscape. For insectivorous birds, abundance within farms may be dependent on the available nesting and prey resources in natural and semi-natural habitats in the surrounding landscape. For both fruit-eating and insectivorous birds, abundance may be explained by the density of birds of prey that predate them. Across a network of 20 strawberry farms in California's Central Coast region, we observed that strawberry-eating bird abundance increased with the proportion of strawberry production in the surrounding landscape, but decreased with both semi-natural habitat in the surrounding landscape and the number of anti-bird practices used on farms. The abundance of foliage-gleaning, ground-gleaning, bark-gleaning, and hawking insectivores increased with the semi-natural habitat in the surrounding landscape, while the abundance of aerial foraging insectivores (e.g. swallows) increased with the proportion of strawberry production in the surrounding landscape. The presence or absence of predatory birds had no relationship with strawberry-eating bird or insectivorous bird abundance. Our results reveal that the concentration of resources is important in explaining bird abundance in agroecosystems, but that different functional groups of birds respond to different drivers across landscapes.}

\normaltalk{Bird Trophic Effects on Corn and Soybean Crops Within a Mixed Grassland and Agricultural Landscape}{Megan B Garfinkel}{Birds can have positive trophic effects on agricultural yield by eating arthropod crop pests. Although corn and soybean agriculture is extremely widespread across the Midwest region of the United States, few studies have examined bird trophic effects in these systems. Most corn and soybean fields do not host high bird densities, but crop fields that are adjacent to ``good" bird habitat such as prairie or grassland may show increased bird populations. I conducted a series of experiments in corn and soybean fields adjacent to prairies or grasslands to determine the net indirect trophic effects of birds in these crops. I first used exclosures to exclude birds from corn and soybean plants, and compared crop yield between exclosures and controls. The next year I continued the exclosure study in additional soybean fields. I found during the first year that birds provided net indirect services in corn and disservices in soybeans. The second year I found that although bird exclosures had higher levels of pest defoliation than control plots, there was no resulting effect on crop yield. These results indicate a need for further study on net effects of birds in corn and soybean agriculture. If we can show a net economic benefit provided by birds over time in agriculture adjacent to grasslands, this may help to incentivize conservation of grasslands and prairie habitat.}

\normaltalk{Patterns and Processes Altering the Attendance of Army Ant-Following Birds to Army Ant Swarms.}{Laura Gomez Murillo, Jonathan P Kelly, Corey E Tarwater}{Biotic interactions play a vital role in community structure and the maintenance of biodiversity. While studies continue to document the widespread disruption of these interactions, the underlying mechanisms leading to alterations in biotic interactions are less clear. Furthermore, biotic interactions are often subject to multiple environmental stressors, even though often studies only examine one stressor. Here we examined how habitat fragmentation and rainfall altered biotic interactions and the underlying mechanisms driving changes in interactions using a unique system in the tropics, army ants and the birds that follow their swarms, hereafter ant-following birds. We examined how traits of the swarm associated with profitability (e.g., ant species, swarm size and speed) and the community of ant following birds altered the attendance of birds to ant swarms in Panama. We found variation in the presence of army ants and ant-following birds across the fragmentation-rainfall gradient, with no or few obligate ant-following birds found in smaller fragments. We also found marked variation in the ant-follower bird community attending ant swarms, with attendance peaking in the middle of the rainfall gradient and in larger fragments. This coincided with the sites with the highest food availability at swarms. The ant-follower bird community was more diverse and with a higher abundance of birds when swarms were larger, slower, and insect biomass was greater. Given rapidly changing environmental conditions, examining the underlying mechanisms driving shifts in biotic interactions is critical for predicting how communities and biotic interactions may look in the future.}

\normaltalk{Dispersal Ability Correlates with Range Sizes in Amazonian White-Sand Ecosystems Specialist Birds}{João Marcos Guimarães Capurucho, Mary V Ashley, Brian R Tsuru, Jacob C Cooper, John M Bates}{Understanding how species attain their geographic distributions and identifying correlated traits are important objectives in biogeography, evolutionary biology, and biodiversity conservation. Despite much research, results have been varied and general trends have been slow to emerge. One reason could be the study of organisms that are adapted to different environments, as traits correlated to range size can vary depending on the habitat. We evaluated correlations between a set of organismal traits and range sizes in bird species specialized in Amazonian white-sand ecosystems (WSE) and their sister taxa. The WSE are naturally fragmented habitats that occur on nutrient poor and seasonally flooded sandy soils. We estimated the phylogenetic signal in the data based on three different phylogenetic hypotheses and used two range data sources, the BirdLife maps and ecological niche models. We assessed if results are consistent among phylogenetic and range hypotheses. We found that dispersal ability, measured as the hand-wing index, was correlated with range sizes in both WSE and non-WSE birds. WSE birds also had smaller ranges on average than their sister taxa, probably due to the fragmented nature of WSE. The results were similar and robust to the different data sources. Studying species that occupy similar habitats, instead of clades or artificial groupings with species occupying sets of distinct habitats, is important to identify traits correlated to ranges size that could be more informative for conservation programs in a rapidly changing world.}

\normaltalk{Woody Field Edge Effects on Bird Communities in Michigan Blueberry Fields.}{Melissa Hannay, Olivia Utley, Sarah Groendyk, Catherine Lindell}{Recent work suggests that complex vegetation structure surrounding crop fields may improve crop pest reduction by birds. This effect may result from a higher diversity of birds supported by complex vegetation structure compared to simpler vegetation structure. We investigated the numbers and types of bird species using Michigan blueberry fields in 2017 and 2018 during 120 hours of observation. We conducted surveys in 50 x 50 m areas of 20 study blocks of different blueberry varieties, including Bluecrop, a widely grown variety that ripens in July. We calculated the proportion of forest in 1-km buffers centered on each study block using satellite imagery from the National Agriculture Imagery Program from 2017. We predicted that bird species richness would be higher in blueberry blocks with greater proportions of forest cover within the buffers. We detected 34 and 29 bird species in 2017 and 2018, respectively, in all blocks, and 25 and 17 species in 2017 and 2018 in Bluecrop blocks. Our preliminary results did not show a strong relationship between the amount of forest surrounding blocks and bird species richness within the blocks. The most frequently detected species over all blueberry varieties and in Bluecrop were American Robins, Song Sparrows, and Red-winged Blackbirds, respectively. American Robins and Song Sparrow likely play dual roles as providers of ecosystem services and disservices in blueberry fields, consuming fruit while also consuming pest arthropods; their disservices likely outweigh services during the fruit-ripening period. Red-winged Blackbirds likely provide primarily ecosystem services through their consumption of arthropods.}

\normaltalk{Landscape Cover and Connectivity Mediates Avian Survival and Natural Pest Control in an Intensive Agroecosystem}{Sacha K Heath}{Co-management of agroecosystems for biodiversity conservation and food production can meet habitat requirements of farmland birds and harness beneficial avian services for growers. One strategy is to retain non-crop vegetation in uncultivated crop margins (i.e., hedgerows). Hedgerows are predicted to attract beneficial birds that reduce pests in adjacent crops, and to be of sufficient quality to support avian survival and reproduction. Further, hedgerow effectiveness is expected to be mediated by features of the landscape. I tested these predictions in a network of hedgerows in the intensive agricultural landscape of California's Sacramento Valley, USA. Collaborators and I employed field and remote surveys and a sentinel prey exclosure experiment to estimate bird community measures and pest predation probability in relation to non-crop vegetation at local and landscape scales. I also employed a mark-recapture study for two Zonotrichia sparrow species in hedgerows with varying degrees of landscape connectivity. Hedgerows harbored more bird species and individuals, and more pest predators than bare margins, and predator abundance and pest predation increased with increasing proportions of seminatural landscape cover. While within-winter apparent survival for adult Zonotrichia was relatively high and consistent (= 82±8\%) across hedgerow isolation categories, first-year sparrow survival was lowest (53±11\%) in entirely isolated hedgerows. This suggests that farm scale habitat enhancement can increase bird diversity, and support winter survival of select species if connected to similar habitats. Increasing pest predator diversity can also benefit growers. Yet, growers and birds appear to benefit most when farms are embedded in landscapes of greater natural complexity.}

\normaltalk{Bird Conservation in Human-Dominated Landscapes of the High Andes}{Liz B Hurtado}{The tropical Andes have exceptional biodiversity that has been shaped by past climate history, topography, and environmental heterogeneity. These factors coupled with the rich history of occupation, including ancient civilizations, have long shaped the region's landscapes and vegetation. As a result,  Andean landscapes are now dominated by a matrix of shrubs, remnant forest patches, and planted and fallow fields. This landscape conversion has greatly affected the wildlife communities of the region; however, the Andes are still considered a region rich in biological diversity and high endemism. This biodiversity has been studied in some sections of the High Andes ($>$4000 m elevation) but at lower altitudes where landscapes have been more intensively managed near areas with high human population density, there is no empirical information. Using the countryside biogeography approach and during the non-breeding season, my study aimed to investigate the response of bird community assemblages across five habitats in an agricultural matrix near the city of Cusco. The results show that fragments of native forest and thorny shrublands have great importance for the conservation of endemic and specialist birds whereas crops have high species richness but mainly of generalist and open-country species. One type of degraded shrublands lost most of its value for birds. To fight biodiversity loss, protected areas alone are not enough. Our efforts should also be focused on the study and management of habitat outside protected areas, especially in sites with long histories of human occupation.}

\normaltalk{Influence of Uncultivated Habitat on Barn Owl Nest Box Occupancy, Hunting Habitat Selection, and Rodent Removal in Winegrape Vineyards}{Matthew D Johnson, Allison E Huysman, Dane St George}{Many vineyard managers have installed nest boxes to attract Barn Owls (Tyto alba) to manage rodent pests, but our ecological understanding of this system is incomplete.  Uncultivated habitats near farms can strongly affect how mobile predators with large home ranges use the landscape and thereby impact the delivery of pest control services.  We used occupancy surveys, telemetry, and videography to examine Barn Owls nesting in boxes on winegrape vineyards in the heterogeneous landscapes of Napa Valley, California.  We found that nest box occupancy increased with the area of uncultivated lands -- especially grassland, oak savannah, and riparian habitats -- within a typical Barn Owl hunting radius around the nest boxes.  Telemetry revealed that these habitats were also selected out of proportion to their availability by hunting owls.  The amount of uncultivated habitat was correlated with prey delivery rate and prey species composition, with more voles (Microtus californicus) delivered to boxes with abundant grasslands nearby and more gophers (Thomomys bottae) delivered to boxes with oak savannah.  These results highlight the influence of landscape composition on pest removal by an avian predator, and they provide further incentive for producers to conserve natural habitats throughout the landscape.}

\normaltalk{Is it a Trap? Raptors, Rodenticides, and Rain in California's Agroecosystems}{Sara M Kross, Breanna Martinico, Ryan Bourbour, Emily Phillips, Roger Baldwin}{Farmers across the world have recently been installing nest boxes and artificial perches for birds of prey in an effort to boost natural rodent pest control services. However, because many farmers also utilize anticoagulant rodenticides to control rodent pests, raptors can suffer from both lethal and sub-lethal secondary poisoning. We have little understanding of how often raptors are exposed to rodenticides in an agricultural setting, and what effect this exposure has on their behavior and reproductive success. We seek to quantify the frequency and effects of secondary rodenticide poisoning in breeding barn owls, as well as in overwintering barn owls and red-tailed hawks in an agricultural landscape in California. Using blood samples gives us data on rodenticide exposure in live animals, and quantifies exposure in the ~2 weeks preceding each sample. Preliminary results show that no barn owls (n=41), and 36\% of overwintering red-tailed hawks, were exposed to rodenticides. Importantly, rodenticides are not the only risk for raptors in this landscape. In 2019, as a result of extreme rain events in California that spring, we recorded widespread nest failure in barn owls and 4 other avian species. Long term management for healthy raptor populations and natural pest control services in agroecosystems may need to account for natural history traits of rodent-consuming raptors, as well as population perturbations due to climate change.}

\normaltalk{Landscape Context Mediates the Physiological Stress Response of Birds to Farmland Diversification.}{Christopher E Latimer, Olivia M Smith, Joseph Taylor, Amanda B Edworthy, Jeb P Owen, William E Snyder, Christina M Kennedy}{Farmland diversification practices are increasingly adopted to help reverse biodiversity declines in agroecosystems. However, evidence for the effectiveness of this approach often comes from documenting the species attracted to particular farming systems or landscapes, rather than their underlying physiological states that ultimately determine population growth or decline over the longer term. Because physiological stress responses of individual birds often develop as a precursor to population declines, understanding the drivers of stress can offer insights into the mechanisms affecting population persistence in agroecosystems. We quantified three hematological and morphological indices that are widely used to capture variation in short and long-term stress responses for nine bird species with diverse life-history traits. Across 38 farms throughout the western U.S., we found that birds were generally less stressed on more-locally-diverse farms and in landscapes with higher amounts of seminatural cover. However, interactions between farm diversity and landscape context suggested birds were less stressed on more diverse farms in simpler landscapes, but more stressed and in poorer condition on more diverse farms embedded within complex landscapes. These results underscore that farmers might not see uniform effects of agricultural diversification schemes across different landscape contexts. However, practices, including increasing crop diversity, limiting field sizes, integrating (semi)natural habitats into fields, along with preserving natural habitats around farms appear to provide some benefits for birds, particularly in industrialized agricultural landscapes. Lastly, the strong modulating effect of landscape context suggests scalable conservation solutions will likely depend upon collective ``buy-in" and coordinated action across multiple landowners in a region.}

\normaltalk{The Case for Expanding Automated Telemetry Efforts Across the Caribbean}{Kara L Lefevre, Adam D Smith}{Today's ornithologists benefit from working in an era of rapid technological advances that bolster possibilities for studying animal movements. Growing initiatives like the Motus Wildlife Tracking System and Cellular Tracking Technologies are enabling astounding new insights based on automated telemetry, from hemispheric to local spatial scales. There is a need for increased investment in these methods to study avian movements in southeastern North America because the region functions as a crossroads of migratory pathways within the Americas. We will share progress toward developing a more robust Motus network in the southeast over the past five years. Compared to early 2016 when there were no receiving stations in Florida, there are now roughly 25 stations in the state. A total of 351 birds has been detected cumulatively by those stations as of December 1, 2019. Beyond providing improved insights about avian ecology in the southeast, more broadly these data are enhancing the understanding of movement linkages between the north and south, for studies of species such as American Redstart, Gray-Cheeked Thrush, Kirtland's Warbler and Red Knot. The Motus footprint has begun to expand to many parts of South America and there are a handful of stations in the Caribbean. We suggest that concentrated effort to expand Motus coverage across the Caribbean is a priority for improving basic data collection. It will reinforce collective efforts to document critical habitats and movement pathways in a disproportionately important part of the Americas, information that is integral to sound management and conservation of migratory birds.     ------    FRENCH version:    Les arguments en faveur d'une expansion des efforts de t\'{e}l\'{e}m\'{e}trie automatis\'{e}e dans les Caraïbes    R\'{e}sum\'{e}: Les ornithologues d'aujourd'hui tirent avantage de travailler à une \'{e}poque d'avanc\'{e}es technologiques rapides qui renforcent les possibilit\'{e}s d'\'{e}tudier les mouvements des animaux. Des initiatives croissantes telles que le Système de surveillance faunique Motus et les ``Cellular Tracking Technologies" permettent de nouvelles informations \'{e}tonnantes bas\'{e}es sur la t\'{e}l\'{e}m\'{e}trie automatis\'{e}e, de l'\'{e}chelle h\'{e}misph\'{e}rique à l'\'{e}chelle spatiale locale. Il est n\'{e}cessaire d'investir dans ces m\'{e}thodes pour \'{e}tudier les mouvements aviaires dans le sud-est de l'Am\'{e}rique du Nord parce que la r\'{e}gion fonctionne comme un carrefour sur les voies de migration dans les Am\'{e}riques. Nous partagerons les progrès vers le d\'{e}veloppement d'un r\'{e}seau Motus plus robuste dans le sud-est au cours des cinq dernières ann\'{e}es. Par rapport au d\'{e}but de 2016 où il n'y avait pas de stations de r\'{e}ception en Floride, il y a maintenant environ 25 stations dans l'État. Au total, 351 oiseaux ont \'{e}t\'{e} d\'{e}tect\'{e}s de manière cumulative par ces stations au 1er d\'{e}cembre 2019. Au-delà de fournir de meilleures informations sur l'\'{e}cologie aviaire dans le sud-est, plus largement ces donn\'{e}es am\'{e}liorent la compr\'{e}hension des liens de mouvement entre le nord et le sud, pour des \'{e}tudes des espèces telles que le Paruline flamboyante, la Grive à joues grises, la Paruline de Kirtland et le B\'{e}casseau maubèche. L'empreinte de Motus a commenc\'{e} à s'\'{e}tendre à de nombreuses r\'{e}gions d'Am\'{e}rique du Sud et il y a une poign\'{e}e de stations dans les Caraïbes. Nous sugg\'{e}rons que des efforts concentr\'{e}s pour \'{e}tendre la couverture de Motus dans les Caraïbes soient une priorit\'{e} pour am\'{e}liorer la collecte de donn\'{e}es. Il renforcera les efforts collectifs pour documenter les habitats critiques et les voies de d\'{e}placement dans une partie disproportionn\'{e}e des Am\'{e}riques, informations qui font partie int\'{e}grante d'une saine gestion et conservation des oiseaux migrateurs.}

\normaltalk{Weather Events Underlie Stopover Behavior During a Long-Distance Migration}{Jennifer A Linscott, Juan G Navedo, Mitch D Weegman, Sarah Clements, Bart M Ballard, Nathan R Senner}{The long, nonstop flights of migratory birds are powered by endogenous fuel stores, but they also rely--perhaps significantly--on wind support. When possible, migrants make behavioral adjustments in order to locate and exploit winds that are well-suited to flight, relying on cues from local weather patterns in order to time their departures, select optimal flight altitudes, and minimize energy expenditure. At other times, they may need to counter or compensate for adverse winds, expending more energy in order to maintain airspeed and arrive at suitable stopover habitat. Here, we examine wind-associated behavioral adjustments of Hudsonian godwits (Limosa haemastica), extreme long-distance migrants that travel from non-breeding grounds in southern Chile to subarctic breeding grounds in April/May. By combining movement data from satellite-tracked godwits with hourly ECMWF weather reanalysis data, we investigate the total cumulative wind resistance experienced by migrating godwits, as well as wind resistance at vulnerable points midway along their routes. We find that stochastic wind patterns encountered after initiating a crossing of the Gulf of Mexico can profoundly shape stopover decisions, stopover behavior, and flight trajectory throughout the remainder of their journeys. This finding, reinforced by interannual, intraindividual differences, emphasizes the opportunistic nature of godwit stopovers and the degree to which similar long-distance migrants may be underserved by site-specific stopover conservation plans.}

\normaltalk{Distributional Patterns of Monospecific and Mixed Colonies of Caciques and Oropendolas}{Fernando Machado-Stredel}{Caciques and oropendolas form an avian subfamily of ~20 species (Cacicinae), with most taxa exhibiting colonial nesting and polygynous mating systems, an unusual pattern in the Neotropics. In this study, the eBird dataset was analyzed until 2018 to locate potential colony sites of 14 Cacicinae species, using breeding codes associated with nesting behavior, since these colonies are widespread, conspicuous, and easily detected by citizen scientists. From 290,074 eBird lists, 1444 had breeding codes; this dataset was complemented parsing keywords in English, Spanish, and Portuguese through comments on lists without codes, obtaining 1756 additional lists, for a total of N = 2722 unique records. The species with most unique records was the Montezuma Oropendola (Psarocolius montezuma; N = 572), along with three other, well-represented species (P. decumanus, P. wagleri, Cacicus cela; 349-371 records). However, several gaps of knowledge persist, with colonies of five restricted species represented by only 4-24 records, and widespread species lacking information in some regions (e.g., P. decumanus in Peru). A subset of lists of mixed colonies showed known interactions, but mixed-nesting associations of 2-4 species that have not been reported previously in the literature were discovered as well. These results suggest that mixed-colonies may prove quite common, although many evolutionary and ecological aspects remain unexplored. Since natural history information is still scarce for some Cacicinae species, this project represents a first step in the study of the geographic distribution and mating and social systems of this radiation.}

\normaltalk{Double Brooding as a Weak Demographic Driver of Population Growth in a Threatened Barn Swallow Population in Eastern Canada}{Hilary A Mann, Tara L Imlay, Marty L Leonard}{Understanding how different demographic parameters affect population regulation in species at risk is essential for setting effective recovery goals. When examining reproductive parameters, the number of broods can be a better indicator of annual breeding success than the number of young produced per attempt. Therefore, we were interested in the role of double brooding in the population regulation of the Barn Swallow (Hirundo rustica), a species at risk in Canada. We estimated vital rates using mark-recapture and nest monitoring data from breeding sites in New Brunswick and Nova Scotia from 2012-2019. Then we used life stage simulation and perturbation analyses to determine the effect of the prevalence and success of double brooding on the population growth rate, in comparison to the success of single brooding and the survival of adult and juveniles. Double brooded pairs contributed 50\% of the population's fledglings and produced almost double the number of fledglings as single brooded pairs. However, increasing the prevalence of double brooding had relatively little influence on population growth rate, compared to the other parameters, especially adult survival. Although increasing adult survival has the greatest impact on population recovery, increasing the prevalence of double brooding will nonetheless have a positive effect on population growth rate, albeit less. We are currently investigating the factors associated with an increased likelihood of double brooding. Together, the results of our work will help inform and prioritize recovery actions for Barn Swallows.}

\normaltalk{Morphological, Vocal and Genetic Divergence Between Geographically Isolated Populations of an Andean Bird (Diglossa Cyanea)}{Silvia C Martinez-Gomez, Andres M Cuervo}{Populations that become isolated by barriers to gene flow can express divergence levels that vary among traits. Even when genetic divergence among populations is clear, phenotypic homogeneity can mask these differences. Widespread montane forest birds in the Andes mountains occupy latitudinally long, linear ranges that are highly discontinuous by ecological and topographic complexities (e.g., dry enclaves and lowland gaps). One such species is Diglossa cyanea (Thraupidae; Masked Flowerpiercer), which occurs within 1800 m belt of montane forest along 4500 km from the Bolivian Yungas to Venezuelan coastal range. We evaluated the extent of morphological, vocal and genetic divergence within D. cyanea, considering its geographically fragmented distribution. Population divergence is sharply defined by the Mara\~{n}on river valley in northern Peru. We found a deep divergence in 1) genetic data (ca. ~ 7\% uncorrected p distance of mtDNA sequences, n=122), and 2) song structure, (n=90), and 3) different slight differences in wing length (n=345) between the two populations on opposite sides of this valley. Although phenotypic differences across the Mara\~{n}on valley are recognized at the subspecies level, described differences are limited to subtle details in plumage. Because both phenotypic and genetic divergences are highly contrasting, our results suggest that the southern subspecies D. c. melanopis should be elevated to species rank. This study supports the importance of integrating different biological traits and their complementarity to better understand the evolutionary history of the species.}

\normaltalk{Conserving Birds by Managing Ecosystem Services on Coffee Farms}{Alejandra Mart\'{i}nez-Salinas, Adina Chain-Guadarrama, Natalia Aristiz\'{a}bal, Taylor Ricketts}{Coffee is one of the most important tropical crops on earth, considering both its gross production value and the number of families that depend on it for their livelihoods. Coffee is cultivated within some of the most biodiverse habitats on earth, and its cultivation intensification has threatened biodiversity and critical ecosystem services to coffee production such as pest control by birds and pollination by bees. Experimental evidence has shown that birds provide important pest control services and bees provide important pollination services to coffee productivity. However, both services have rarely been evaluated in conjunction to assess interactions. Using experimental field exclosures we measured the pest control and the pollination services provided by birds and bees respectively across a network of coffee farms (n=30) located within the Volcanica Central Talamanca Biological Corridor in Costa Rica following a management and shade gradient. Although evidence is incomplete, managing coffee farms as diversified agroforestry systems can enhance habitat and services delivered by birds and bees increasing coffee productivity and coffee farmers livelihoods.}

\normaltalk{Urban Nesting Habitat of the Invasive Monk Parakeet (Myiopsitta Monachus) on the Metropolitan Zone of Guadalajara, Jalisco, Mexico}{Perla S Gonz\'{a}lez-Ruelas, Zayra AG Mu\~{n}oz-Gonz\'{a}lez, Ver\'{o}nica C Rosas-Espinoza, Ana L Santiago-P\'{e}rez}{The Monk Parakeet (Myiopsitta monachus) is an invasive species, native from Sudamerica. It is favored in urban sites by the presence of exotic tall trees as eucalypts, pines, cypresses, and palms, as well as by artificial structures, like poles and electrical transformers and towers where it builds communal nests. The nest site selection is important for the species because it is used as site to nest, rest and perch. Our objective was to characterize the urban nesting habitat of the Monk Parakeet in the Metropolitan Zone of Guadalajara, which is the second biggest city in our country. We registered 61 nesting trees species, from which 72\% were eucalypts (Eucalyptus spp.), followed by palms (Washingtonia spp.), jacaranda (Jacaranda spp.) and ash (Fraxinus spp.). The average height of the nesting tree was 18.13 m, the average diameter at breast height (DBH) was 49.65 cm and the average canopy was 14.56 m. The nest was located on an average height of 16.79 m. The average distance from the nest to the nearest nesting tree was 11.75 m, and the average distance to the nearest no-nesting tree was 8.13 m. The average height and DBH for the nearest no-nesting tree was 14.88 m and 34.09 cm, respectively. The average distance from the nest to the nearest urban infrastructure was 17.93 m. The information from our study would help to identify the ideal conditions where the species could potentially establish and colonize, as well as, to implement management strategies to control the Monk Parakeet population.}

\normaltalk{Genomic Structure of a Three-Species Woodpecker Hybrid Zone}{Libby Natola, Darren E Irwin}{Hybrid zones provide biologists ``natural laboratories" to study the progression of speciation, and therefore the production of biodiversity. Most hybrid zones occur between two species or subspecies, but studies of hybridization between only two groups limit our ability to track divergence on a speciation continuum. Red-breasted, Red-naped, and Yellow-bellied Sapsuckers (Sphyrapicus ruber, S. nuchalis, and S. varius) have long been recognized to hybridize in pairs, and thus are a compelling example in the study of speciation. Moreover, Red-naped and Red-breasted Sapsuckers diverged less than 500,000 years ago, yet Yellow-bellied Sapsuckers split from this lineage over 1 million years ago, meaning these species constitute different stages of a speciation continuum. Here, we show that Red-breasted, Red-naped, and Yellow-bellied sapsuckers form a rare, emergent, three-species hybrid zone caused by the convergence of two independent two-species hybrid zones in British Columbia, Canada. We use Genotyping by Sequencing data to describe population genetic structure of hybridization among the three sapsuckers and to better understand to better understand how age of differentiation between three groups relates to the frequency and types of hybridization and the patterns of introgression among the three groups.}

\normaltalk{Conserving Natural Habitat Around Farms Limits Bird Disservices}{Elissa M Olimpi, Karina Garcia, David Gonthier, Erin E Wilson-Rankin, William S Snyder, Daniel S Karp}{Some birds are viewed as pests and vectors of foodborne pathogens in farmlands, yet birds also benefit growers by consuming pests. We ask how local (i.e., hedgerows, multiple crops) and landscape (e.g., more non-crop habitat surrounding farms within 1km) level diversification influence the likelihood of different bird species to function as beneficial, pest, and vector species. We captured birds and collected fecal samples, and surveyed crop fields for fecal contamination, on 20 California strawberry farms that varied in local management practices and landscape context. We captured 1,498 birds, representing 74 species, collected 1,328 fecal samples, and used molecular methods to characterize bird diets and screen for foodborne pathogens. We found very low incidence rates of pathogenic Escherichia coli and Campylobacter ($<$4\%), and did not detect Salmonella in any bird fecal samples. E. coli incidence was weakly negatively correlated with seminatural habitat, and Campylobacter incidence was significantly lower on farms surrounded by more seminatural habitat. We also found that fecal contamination in crop fields did not increase with seminatural habitat. From preliminary analyses, we found that birds were most likely to function as pests (i.e., consuming strawberries) on monoculture farms in landscapes with little remaining seminatural habitat. Finally, our preliminary analyses indicated that the influence of seminatural habitat surrounding farms had differential effects on invertebrate pest and natural enemy consumption for different bird species. Taken together, our findings suggest that conserving natural habitat in farming landscapes may help to limit the role of birds as both pests and vectors on farms.}

\normaltalk{Natural and Anthropogenic Effects on Distributions of Pacific Swallow Populations}{Brenda Ramirez, Dr Elizabeth Scordato}{Pacific swallows (Hirundo tahitica) are insectivores that nest on human structures across southeast Asia and the Pacific. This commensal relationship with humans makes them the ideal system in which to study the effects of human development on species distributions. Although human activity is a huge source of disturbance, we currently have a poor understanding of how long-term human landscape alteration has affected distributions of suitable habitat for a variety of species. Here we used species distribution models (SDMs) to generate predictions of suitable Pacific swallow habitat based on both climate variables and variables related to human population density and land use. Sighting data was collected in 2018 and was combined with sightings from the citizen science database eBird from 2016-2018. Environmental layers in these models include climate variables from Bioclim and human population densities and human land use from Hyde. Using MaxEnt in the R package ``dismo," preliminary SDMs show that annual precipitation and human population density have the largest effect on Pacific swallow habitat distributions. SDMs were then hindcast over the last 6000 years to predict distributions of available habitat since the spread of agriculture in Southeast Asia. Human activity is often not included in SDMs of past habitat, but humans have had dramatic effects on the landscape for thousands of years. Through these models, human impact on Pacific swallow distributions over comparatively long periods of time can be evaluated.}

\normaltalk{Drinking and Fighting: a Hummingbird's Paradise?}{Alejandro Rico-Guevara, Diego Sustaita, Tai-Hsi Fan, Daniel J Field, Kristiina J Hurme, F G Stiles, Margaret A Rubega, Robert Dudley}{Evolution of hummingbird bills is primarily understood to occur through coevolutionary change with the flowers from which they feed. The rationale is that the bill-corolla match enhances the benefits received for both the plant and the pollinator. Evaluating these benefits from the hummingbird's perspective requires thorough understanding of the mechanisms of nectar uptake and the formulation of testable predictions regarding net energy gain resulting from such a match. These mechanisms involve not only nectar intake within the tongue and bill, but also subsequent intraoral transport of fluid. We present here a new physical model for nectar intake by hummingbirds, which includes detailed assessment of the intraoral transport mechanism as well as a surprising discovery of simultaneous upper jaw bending in these birds. We study this novel bill bending through geometric morphometrics applied to kinematic analysis of virtual markers in high-speed footage, and also provide estimates of bill flexural rigidity from microCT scans. We then examine the selective forces involved in bill evolution, and address bill modifications within a previously unexplored context, namely conspecific and heterospecific fighting. Finally, we introduce the substantial variation present in several underappreciated sexually dimorphic bill traits (e.g., serrations, hooks, and daggers), which we suggest function as potential weapons but entailing trade-offs with nectar intake efficiency, and discuss their multiple independent occurrences across hummingbird phylogeny. Our approach offers a fresh perspective on the evolution of bills in hummingbirds, and its implications for future studies of coevolutionary processes with the plants they pollinate.}

\normaltalk{Grassland Bird Response to Nest Searching with Drones}{Bernadette W Rigley, Amy EM Johnson, October Greenfield, Jared A Stabach}{Grassland birds have suffered serious declines, and even extirpation, throughout North America. These declines are attributed to the deterioration and loss of grassland habitats, land-use change and intensification, loss of native species, and afforestation. Understanding population trends is a critical component for effective conservation. The most effective way to determine reproductive success is through nest monitoring, which is essential for developing comprehensive management plans that encourage population recovery. Traditional methods to monitor breeding success are labor-intensive and have even been shown to contribute to nest abandonment and predation. However, the use of unmanned aerial vehicles (UAVs), commonly referred to as drones, equipped with infrared cameras, may be a less intrusive alternative. Previous investigations conducted by the Smithsonian Conservation Biology Institute (SCBI) and James Madison University X-labs have shown the successful thermal recognition of active nests with the use of UAVs.  While this technology has proven to be a valuable tool in ecological research, little to no investigation has focused on the disturbance it may cause to nesting birds. The objective of this study was to compare the behavioral stress responses of birds to 2 nest search methods; systematic walking and drones. Nest search interactions (n=258) and respective behavioral stress responses (mobbing, flushing, alarm call) were recorded for 6 grassland species. Preliminary results from one season of surveys indicate no statistical difference in behavioral responses to the two survey methods (P$>$0.05). Data collection will continue this summer to further monitor behavioral stress responses of nesting grassland birds to these survey methods.}

\normaltalk{Are Eastern Bluebirds more Likely to Eject Cowbird Eggs that are in the Center of their Clutch or on the Periphery?}{Max F Rollfinke, Max Saucier, Adam Gelman, Elayna Daniels, Zane Libke, Mark Stanback}{Some songbird species have evolved the ability to recognize eggs laid by brood parasites by color, shape, or size and then remove them from the nest. Recent work suggests that hosts may use changes in the arrangement of eggs within the nest to identify such eggs. We tested this using model Brown-headed Cowbird (Molothrus ater) eggs in the nests of Eastern Bluebirds (Sialia sialis). Bluebirds exhibit a variable response to the appearance of cowbird eggs in their nests: about 2/3 of females eject such eggs. We placed a single model cowbird egg either in the center or on the periphery of fresh bluebird clutches containing 4-6 eggs. We observed a significantly higher rate of cowbird egg ejection when the cowbird egg was placed on the periphery of the clutch.}

\normaltalk{Soundscape Variation in a Gallery Forest Affected by Fire in the Bojonawi Nature Reserve (Vichada, Colombia)}{Santiago Ruiz Guzman, Dolors Armenteras Pascual, Oscar Laverde}{Fires in tropical regions have recently become a major problem due to their great impact on the structure and function of ecosystems. Despite the prevalence of fires in the region, their effects on wildlife are still little known, but few studies mainly in temperate areas have been published recently. The present study seeks to compare the acoustic communities, first analyzing the soundscape and second focused on vocal activity of birds in a preserved gallery forest and in forests affected cyclically by fires in the Bojonawi Nature Reserve located in the Orinoco region. We deployed six autonomous recording units (ARUs), three in the conserved gallery forest and three in the forest with fire regime. ARUs were distributed equally (ca. 100m from each other) and were programmed to record 2 minutes every 6 minutes for 5 days.  To characterize the acoustic communities 5 acoustic indices were calculated: the acoustic diversity index (ADI), the acoustic complexity index (ACI) in 5 frequency bands, the acoustic evenness index (AEI), the bioacoustic index (BI) and the total entropy. We only found differences for the ACI but just in the frequencies above 4 kHz. In further steps we will compare the daily acoustic patterns of bird species reported in the literature as susceptible to fires which inhabit and rely on well preserved gallery forest.}

\normaltalk{Assessing Effects of Habitat Amount vs Configuration on Avian Diversity in Managed Pine Landscapes}{Craig A Sklarczyk, Kristine O Evans, Dana J Morin, Daniel Greene}{Approximately 89\% of forests in the Southeastern Coastal Plain ecoregion are privately owned. However, little is known about how spatial patterning of working pine stands in different age classes influence bird diversity and abundance. Given their extensive coverage and growing demands for forest products, it is critical to enhance our understanding of how forest management activities influence avian diversity at the landscape scale. A recent hypothesis has been suggested that states the amount of species-specific habitat is more critical than the configuration of it on the landscape. To determine optimal amount and proximity of habitat conditions, we examined breeding-season avian communities in regenerating and mid-late succession managed loblolly pine stands in east-central Mississippi from 2019-2020. We assessed species associations with local and landscape characteristics using a Bayesian hierarchical modeling framework which combined distance-sampling and time-removal methods to address availability and perceptibility of birds. Our focal early-successional species showed varying associations with amount and proximity of habitat. Prairie Warbler, Northern Bobwhite, and Eastern Kingbird exhibited a quadratic relationship with the amount of regenerating pine on the landscape, whereas Field Sparrow exhibited strong positive associations with amount of regenerating pine.  Most species had slight to strong positive association with proximity of neighboring patches, although Northern Bobwhite and Field Sparrow exhibited a quadratic relationship with proximity of neighboring patches. We suggest associations with habitat amount and configuration are species-specific. Improved understanding of how spatial patterning may influence birds in working pine systems may lead to management applications that better target declining priority species.}

\normaltalk{Bird Diversity Pattterns in Three Modification Scenarios of a Mountainous Landscape of Southern Mexico}{Omar Suarez, Matthias Rös, Erwin L\'{o}pez}{Bird diversity at three different modification scenarios in a mountainous landscape of Southern Mexico was studied. The scenarios were defined by the amount of original vegetation cover (according to a continuous landscape approach) as intact ($>$90\%), variegated (between 90\% and 60\%), and fragmented (between 60\% and 10\%). The studied landscape was a mosaic of oak and pine forest, agricultural lands, cattle pastures, and orchards. In each modification scenario, one square of 100 ha was placed and subdivided into 16 plots. Point counts were conducted at each plot to register bird species and abundances and visited twice at two contrasting seasons. A total of 960 minutes of bird observation was completed, accounting for 75 bird species. At the landscape scale, 66 species were observed in the non-breeding season and 52 species in the breeding season. No statistical differences were found between landscape modification scenarios for alpha diversities of order 0 and 1 in any season, but diversity of order 2 was lowest in the fragmented scenario in non-breeding season and lowest in the intact scenario in the breeding season. Beta diversity was highest between the fragmented and the intact scenario, and lowest between the variegated and the fragmented scenarios. The findings of this work suggest that alpha bird diversity is not greatly affected by a reduction in original vegetation cover, but beta diversity does, because a set of species that were found in the intact scenario are absent or present in fewer numbers in the modified scenarios.}

\normaltalk{Raptor Evolution in Response to Megafaunal Extinction}{Oona M Takano}{The Quaternary extinctions in North America ~17,000 years ago marked the decline of the Pleistocene megafauna, which included giant herbivores as well as large, apex predators. Compared to mammals, the effects of these extinctions have been poorly studied in birds. The simplification of the mammal and bird communities likely caused changes in raptor diets because of reduced competitive pressure from other carnivores. The La Brea Tar Pits in California, U.S.A. have yielded thousands of raptor fossils, spanning glacial periods in the Pleistocene to warmer, modern climates. For raptors, toe size and talon shape are important traits determining the type of prey captured. I investigated evolution of Golden Eagle (Aquila chrysaetos) and Bald Eagle (Haliaeetus leucocephalus) foot morphology, and how these changes relate to dietary niche. All identifiable toe elements of ~500 modern and 600 Pleistocene eagles were measured with calipers, and geometric morphometric analyses for talon shape were conducted using the geomorph R package. Changes in Golden Eagle toe morphology indicate an average increase in size, while Bald Eagle toes have remained the same size and shape since the Pleistocene. These patterns are likely explained by the different dietary niches of the two eagle species, and a release from interspecific competitive pressure with other terrestrial predators for Golden Eagles. Examining shifts in raptor niches will allow a better understanding of bird community changes and individual species responses when dominant predators are removed, in light of modern extinctions.}

\normaltalk{Phylogenetic Relationships of New World Jays using Thousands of UCE Loci}{Whitney L Tsai Nakashima, Elisa Bonaccorso, Emiko M Schwab, Benjamin Scott, Amanda J Zellmer, James M Maley, Mario Cohn-Haft, Adolfo Navarro-Sigüenza, A Townsend Peterson, John E McCormack}{The New World jays (NWJs) are a monophyletic assemblage of 38 corvid species endemic to the Americas. They have successfully radiated across temperate, tropical, and subtropical habitats and provide an interesting system to study biogeography and colonization of the Americas. Previous studies explored the evolutionary history of NWJs, but could not resolve relationships within a paraphyletic group containing members of Calocitta, Cyanocorax (the largest genus of NWJ), and Psilorhinus, primarily due to incomplete taxon sampling. We present the first genome-scale phylogeny of all NWJ species (92 taxa total including almost all subspecies and five outgroup taxa) using next-generation sequencing of ultraconserved elements (UCEs). Our results provide new resolution on the phylogenetic relationships timing of diversification of the NWJs and assess the evolutionary history and biogeography of this diverse group of birds.}

\normaltalk{Does Differential Maternal Investment Exist Within European Starling Clutches?}{Alexa Tymkiw, Colleen A Barber}{Most passerine species incubate their clutches after the penultimate egg is laid, resulting in asynchronous hatching. Nestlings hatching from last eggs are often at least 24 hours behind in development from their siblings, and therefore much smaller in size. The brood reduction strategy proposes that this last-laid egg acts as insurance when food resources are unpredictable. If food is abundant, then this nestling survives, but if food is scarce, the smallest nestling dies without risking the entire brood. Females can compensate for the delay in the last egg hatching by increasing its size such that the size discrepancy among nestlings is reduced. Alternatively, females may lay a smaller last-laid egg such that the size discrepancy is even more distinct. Our objective was to examine whether female European Starlings (Sturnus vulgaris) laid a larger or a smaller last egg compared to the other eggs in their clutches, by measuring egg mass. European Starlings are cavity-nesting, ground-feeding, asynchronously hatching passerines with biparental care. They typically have two broods in a season, with clutch size ranging from three to six eggs. We studied 191 clutches over five years and found that in early-season clutches, females had significantly smaller last-laid eggs. However, in late-season clutches, last-laid eggs were not significantly smaller than other eggs in the clutch. Although food abundance typically declines throughout the season, last-laid eggs were not reflective of this trend. Perhaps the significantly smaller clutch size of late-season clutches permits less differential maternal investment within clutches than seen in early-season clutches.}

\normaltalk{A Local Test of Species Distribution Models Derived from Citizen Science Data in Costa Rica}{M\'{e}lusine Velde, Jacob Cooper, Holly Garrod}{eBird has quickly become one of the largest sources of distributional data on Earth. The quantity of existing eBird data allows for hitherto unprecedented models of species distributions throughout the annual cycle, as well as insights into patterns of local abundance. The dataset, however, is incredibly spatially biased in the tropics, with the majority of sightings coming from well-trafficked roads and known localities for specific bird species. Furthermore, a large quantity of tropical observations come from tourists who are not necessarily exact with their spatial precision nor familiar with the local avifauna. Many who use eBird data for science assume that `good data' (i.e., true positive observations) that passes filtering and eBird reviewing overwhelm the amount of `bad data' that exists within the system (i.e., false positive observations). In order to test the predictive ability of models derived from existing eBird data, we performed audiovisual surveys on private property adjacent to Parque Nacional Los Quetzales in San Jos\'{e} and Cartago provinces, Costa Rica. We visited during mid-December, a time period when many foreigners visit Costa Rica for ecotourism, and recorded our observations in an area lacking data for this time period. We then created distribution and occupancy models for six test species that vary in their ecology and detectability to compare the theoretical distribution and abundance to our observations. Our tests are specifically designed to determine the accuracy of citizen science derived models, and will serve to improve future distributional work in areas frequented by citizen scientists.}

\normaltalk{Distribution, Density, and Habitat Associations of Non-Breeding, Tidal Marsh Birds in Mississippi}{Spencer L Weitzel, Jared M Feura, Raymond B Iglay, Kristine O Evans, Scott A Rush, Mark S Woodrey}{Secretive marsh birds are difficult to survey for, especially during the non-breeding season when vocalizations are infrequent, complicating population estimation efforts. While some breeding population assessments have been carried out for these birds, dynamics of marsh bird populations, especially during the non-breeding season along the northern coast of the Gulf of Mexico, remain relatively unknown. To address the void in non-breeding season population information for marsh birds along the coast of Mississippi, we conducted line transect surveys distributed across a spatially-balanced sampling framework of Mississippi tidal marshes in December -- February from 2018--2020. Using hierarchical models for unmarked populations, we estimated species-specific densities of marsh birds across a mosaic of tidal marsh communities, spanning oligohaline to polyhaline emergent and open marsh systems. By February 2020, we had conducted 650 surveys, detecting 99 different species. Marsh Wrens (Cistothorus palustris) were encountered most (n = 1202) and at every marsh complex surveyed with an estimated coast-wide density of 2.89 ± 0.71 birds/ha during the first winter season. Clapper Rail (Rallus crepitans; n = 456) were also widespread but less dense (0.79 ± 0.42 birds/ha) especially when compared to the more site-specific Seaside Sparrow (Ammospiza maritima; n = 300) with an estimated density of 6.35 ± 2.04 birds/ha. This study provides the first statewide, non-breeding density estimates for tidal marsh birds along the Mississippi Coast, the first data of these kind for the northern Gulf of Mexico region, and increases our understanding of the non-breeding season avian community composition within tidal marshes.}

\normaltalk{Role of Foot Stomping During Female Mate Choice \& Species Recogination in Prairie-Chickens}{Chelsea J Wright, Jacqueline K Augustine}{Sonation consists of locomotion-induced sounds that have evolved for communication and are produced using specialized morphological features. Prairie-chickens (Tympanuchus spp.) use sonation produced by foot stomping during elaborate courtship display that includes other behavioral, morphological, and acoustic components.  In this study, we determine whether the foot stomping in Lesser (T. pallidicinctus) and Greater (T. cupido) Prairie-Chickens plays a role in female mate choice and species recognition. This study was located on privately grazed pastures in western Kansas, where ranges of the two species overlap. The drumming noise was recorded with a directional microphone placed on the ground facing a focal male. Acoustic properties of the foot-stomping sonation (number of foot stomps, length of stomping bout, stomping rate) were measured using Raven Lite on ten bouts of foot stomping per individual male of both species. We ran ANOVAs to examine the effect of female presence and species on rate and length of foot stomping with male as a random effect. We found that Greater Prairie-Chickens had a slower foot stomping rate compared to Lesser Prairie-Chickens indicating foot stomping could also be a signal associated with species recognition. Two results support the idea that foot stomping could be a signal used during female choice: 1) length of foot stomping bouts were longer when females were present, and 2) foot stomping rate and length were consistent for each male but varied among males. Sonation is an overlooked aspect of communication that helps us understand the relationship among signaling, morphology, and phylogenetic diversity.}

\normaltalk{A Comparative Study on the Evolution of Sexual Dichromatism in Muscicapidae (Old World Flycatchers and Chats)}{Min Zhao, Rebecca Kimball}{Sexual selection on males is often invoked to explain the why males often feature more elaborate coloration than females; however, it has been shown that sexual selection may not always be the primary driver shaping sexual dichromatism. When investigating evolution of sexual dichromatism, we should account for the interplay of both sexual and natural selection on coloration of both sexes, including assessment of ecological and life history traits that may correlate with sexual dichromatism. Muscicapidae, the Old World flycatchers and chats, is one of the largest and most diverse avian families (51 genera, 325 species). It exhibits great diversity in morphology, behavior, vocalization, range, habitat and life history, and shows extensive variation in sexual dichromatism, thus providing an excellent system to rigorously test hypotheses on the evolution of dichromatism. I conducted phylogenetic comparative analyses to investigate the correlation between sexual dichromatism and traits like migration, breeding elevation, insular vs. continental distribution, habitat openness and nest concealability, using a published phylogeny (78\% species). Analyses estimated that sexual dichromatism is the ancestral state. There have been more losses of dichromatism than gains, and most transitions involve changes in both sexes. Gains of sexual dichromatism were rarely due to gains of male brightness, but instead to losses of female brightness or changes in both sexes. However, the degree of change in males versus females could not always be determined with categorical dichromatism. My next step will be quantifying plumage coloration with spectrometry and incorporating a comprehensive species-level phylogeny built with genomic data.}

