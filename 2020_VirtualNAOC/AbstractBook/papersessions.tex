\normaltalk{Bird Diversity Recovery During Secondary Forest Succession in the Tropics: a Meta-Analysis}{Orlando Acevedo-Charry, T Mitchell Aide}{Many studies in tropical regions have focused on how vegetation and ecosystem processes recover following the abandonment of agricultural activities, but less attention has been given to the recovery patterns of the fauna. Our objective was to evaluate the recovery of tropical bird diversity during secondary forest succession. To evaluate the bird recovery, we conducted a literature search and a meta-analysis. We compared changes in species richness and compositional similarity during secondary succession to reference forests (mature or undisturbed). We also evaluated the response of forest specialists and the change in functional groups during succession in the tropical moist forest biome. Based on 58 studies, the general pattern was a gradual increase toward values similar to the reference forests. Species richness reached levels of the reference forests during succession, but this were not the case for species compositional similarity neither forest specialist species richness. Overall, bird recovery increased with successional stage, but other factors, such as, the geographic condition (slower recovery of species richness in islands), rainfall (faster recovery of species richness in regions of high rainfall), and the landscape context (faster recovery of species compositional similarity with more surrounding habitat amount) influenced bird recovery. Our results demonstrate the important role of secondary forests for bird diversity, but the slow recovery of species compositional similarity, forest specialists, and some functional groups (e.g. insectivorous birds) strongly argues for protection of old growth/mature forest and the importance of secondary forest persistence for bird diversity.}

\normaltalk{House Price Index Predicts Impact of Habitat Loss for Birds in the East but not West United States.}{Marlen Acosta Alamo, Lisa L Manne}{Habitat loss is one of the major threats for biodiversity worldwide. This is a non-random phenomenon since the environmental factors that make an area suitable for supporting high local biodiversity overlap with those that make it attractive for urban development. We tested the hypothesis that House Price Index (HPI) will have a higher than random impact on avian communities for both eastern and western United States. We used HPI as an index for attractiveness for development, and thus as non-random habitat loss. Using data on bird species from the North American Breeding Bird Survey, we simulated habitat loss in increments of five percent of total habitat, ordered by declining HPI, and estimated species richness, abundance and rarity index in the remaining area. HPI predicted decrease in total richness for eastern species while for western bird richness its effect was not different from random. Richness and abundance per route were always higher at zones of intermediate attractiveness. Rarity was higher at highly attractive zones in the East and at medium attractive zones on the West. Our results show the relevance of very attractive areas in supporting high levels of biodiversity. At the same time, we identified areas of intermediate attractiveness that offer opportunities for conservation in both eastern and western US by supporting high values of mean richness and abundance, and high numbers of rare species.}

\normaltalk{Differentiated Regions of the Genome Shape Phenotypic Differences in the Northern Flicker}{Stepfanie M Aguillon, Jennifer Walsh, Irby J Lovette}{Understanding the genetic basis of phenotypic traits can provide insights into how selection is acting in nature. Hybrid zones provide particularly tractable systems for studying links between genotype and phenotype in natural settings as they provide repeated instances of admixed phenotypes across the spectrum of genomic ancestry.   Here, we leverage the conspicuous plumage differences and the geographically extensive hybrid zone between yellow-shafted (Colaptes auratus auratus) and red-shafted (C. a. cafer) flickers in the Great Plains of North America to identify the genomic regions associated with the differences in melanin and carotenoid pigmentation across six distinct phenotypic traits. Despite being the focus of extensive study, to date no fixed genetic differences have been identified between the flickers. Here, we perform whole genome re-sequencing of flickers from allopatry and the hybrid zone to obtain a dataset of ~8.5 million SNPs. We identify an extremely low baseline of differentiation across the genome punctuated by a few regions of elevated differentiation. Within these differentiated regions we find clusters of significant associations between SNPs and the phenotypic differences between the flickers. Moreover, numerous genes known to be involved in both melanin and carotenoid pigmentation are located within these identified genomic regions, including CYP2J19 which has been implicated in yellow to red feather color changes in other systems. These findings demonstrate the utility of natural hybrid zones to connect phenotypic differences with candidate genes even in systems without existing genomic resources.}

\normaltalk{Post-Breeding Movements of Two Arctic-Breeding Shorebirds}{Marley Aikens, Hannah MacKellar, Glen Brown, Paul A Smith, Erica Nol}{Post-breeding departure dates of shorebirds in arctic-environments are assumed to be time-minimizing because the arctic-breeding season is contracted, and birds must make haste, after breeding, to reach staging and non-breeding areas efficiently. Some factors that are proposed to influence departure times are sex, with males in monogamous shorebirds, generally taking on more parental care than females, so leaving later; breeding success, with failed breeders departing immediately after their breeding attempt is complete; and distance to non-breeding grounds, with birds who migrate from further north, departing later than those who migrate from further south. We test these hypotheses with two species of migrant shorebirds: Semipalmated Plover (Charadrius semipalmatus) and Whimbrel (Numenius phaeopus) at two locations (Polar Bear Provincial Park, 53oN and Churchill, MB, 58oN) along the Hudson Bay coast in northern Canada. We used the MOTUS wildlife tracking system to determine departure dates. Female Semipalmated Plovers left prior to males, and failed breeders left prior to successful breeders. Failed breeders spent up to two weeks on the non-breeding grounds after nest failure, possibly to take advantage of any opportunity to re-nest. Plovers nesting at our more southerly study area left at the same time as birds at the northern site. For Whimbrel, failed and successful breeders left at about the same time. Northern breeding Whimbrel had later departure times than southern breeders, presumably because they started their breeding activities later as well. Long-thought expectations about biological factors influencing departure dates were mostly borne out through the use of modern tracking devices.}

\normaltalk{Stopover Territoriality in Subalpine Warblers and Its Role in Life History Stage Transitions}{Armando A Aispuro}{Stopover territoriality- the acquisition and defense of a temporary home range during migration- is rare or poorly documented. Obtaining, defending and advertising a territory is costly and it occurs during the most energetically demanding time of the year. Therefore, its energetic benefits are unclear. Alternatively, stopover territoriality during spring migration may be an advanced expression of breeding characteristics in preparation for arrival at breeding areas. In male songbirds, the breeding syndrome (e.g. singing, aggression and territorial behavior) should be fully expressed upon arrival at breeding grounds in order to remain competitive. Therefore, the transition from migratory to breeding life history stages should occur during migration. I studied stopover territoriality of the Subalpine Warbler (Sylvia cantillans), an Afro-Palearctic migrant in the Moroccan Sahara during spring migration. I color-banded birds, mapped their positions daily, and measured home range vegetation characteristics. I then conducted playback experiments and measured testosterone levels. Males defended territories by singing, approaching and chasing intruders. Some birds had high levels of testosterone during stopover. These characteristics resemble those observed at their distant breeding areas. Importantly, territoriality did not involve refueling benefits despite prolonged stopovers in small defined home ranges. Stopover territoriality and prolonged stopover in the absence of refueling may be characterized as a decreased expression in migratory behavior and an increased expression in breeding behaviors. I suggest that stopover sites not only function as resting and refueling points, but also as hormonal staging grounds where life history stage interfaces occur.}

\normaltalk{Hurricane-Induced Fluctuations in a Critically Endangered Population of West Indian Woodpeckers over Two Decades}{Michael E Akresh, Robert A Askins, David I King, Floyd E Hayes, Patricia E Barry, William K Hayes}{Prior studies have reported short-term impacts of hurricanes on avifauna, but few have examined long-term impacts. Over two decades (1993--2018), we surveyed a subspecies of West Indian Woodpecker (Melanerpes superciliaris nyeanus) on San Salvador Island, The Bahamas. We conducted passive and playback surveys to examine distribution, habitat occupancy, and population size on the island, and determined the effects of hurricanes on bird density. Woodpeckers were found only in the northern part of San Salvador, despite extensive surveys throughout other accessible areas. Birds occupied areas with taller coppice adjacent to sabal palm (Sabal palmetto) groves. After hurricanes with $>$160 kph winds passed over San Salvador, woodpecker densities declined to 35-40\% of pre-hurricane densities, but generally recovered back to pre-hurricane densities within 2--3 years. We estimated a population size of approximately 240 individuals (confidence interval = 68-408); however, the population fluctuates to far lower numbers immediately following hurricanes. Under IUCN red list criteria, M. s. nyeanus is classified as critically endangered, and could be especially sensitive to future hurricanes if they occur at a high enough frequency or intensity to prevent the population from rebounding. Our study shows that small, threatened bird populations can be resilient to the effects of hurricanes, but increased intensity of hurricanes, in combination with other threats, may limit this resilience in the future. Given the small size, isolation, and vulnerability of this population, we recommend preservation of the core habitat, continued monitoring, and further research.}

\normaltalk{A Tri-National Partnerships: Migratory Connectivity and Western Forest Bird Conservation}{John D Alexander, Sarahy Contreras, Luis E Morales, Christine A Bishop, Caitlyn E Gillespie, Kristen Mancuso, Jos\'{e}e E Rousseau, Sarah M Rockwell, Matthew Betts, Susan Bonfield, Wendy E Easton, Greg S Butcher}{Amazingly, through their migration many western landbirds connect a diversity of temperate forest habitats where they breed in Canada and the United States with the imperiled pine-oak, tropical cloud, and tropical deciduous forests of western Mexico and northern Central America, where they overwinter. Many of these species are of conservation concern; their populations are suffering declines that signal severe human impacts on the natural systems that they depend on. However, there is hope -- these migratory birds are also connecting an international group of partners from government agencies, non-government organizations, and academic institutions who are working together to track these birds throughout their migratory cycles and implement integrated conservation strategies to reverse their population declines. These strategies target specific threats that impact the forest habitats on which these migratory birds depend throughout their annual cycles (e.g., unsustainable timber harvest, livestock grazing, fire management, and water management). We will summarize how, through international collaboration and capacity building, our science and conservation efforts are meeting objectives outlined in Partners in Flight migratory bird conservation plans. We will present example projects focused on Rufous Hummingbirds and Yellow-breasted Chats that span the breeding, migration, and wintering ranges of these two migratory species. We will share research and monitoring results and show how this science is informing habitat protection and restoration efforts in Canada, the United States, and Mexico.}

\normaltalk{Female Signal Jamming in Cowbird Courtship}{H Luke Anderson, Ammon D Perkes, Julian S Gottfried, Hayden B Davies, David J White, Marc F Schmidt}{Acoustic signaling is vital to avian courtship, yet the role of female vocalizations is understudied. Here, we combined observational and experimental methods to assess the courtship function of the female chatter call in brown-headed cowbirds (Molothrus ater). While the chatter call is likely multifunctional, it is frequently used in social interactions and overlapping duets with males during the breeding season. Based on a combination of focal- and scan-sampling data from large naturalistic aviaries, we found evidence that chatter plays a role in pair-bond formation, as females preferentially chattered in response to songs from pair-bond males in the two weeks leading up to the median date of first copulation. We also found support for the hypothesis that chatter is used to signal jam male songs. Frame-by-frame video analysis revealed that the majority of female chatter calls were tightly time-locked to song, occurring less than 500 ms after male vocal onset. To test the effect of signal jamming on male song potency, we designed a laboratory experiment in which male song playbacks were jammed by various stimuli. Natural chatter more effectively reduced female copulatory responses to song than high-pass-filtered chatter, indicating that the low frequencies in chatter (2–4 kHz) are especially important for interfering with song. Our results suggest that sexual conflict is operating in cowbird courtship, with signal jamming serving as a mechanism by which females guard, resist, or select their mates. We also discuss ways in which cowbird vocal interactions may be inherently cooperative.}

\normaltalk{Tale of Two Cities: Competition Impacts Population Trajectories of Urban Yellow-Crested Cockatoos}{Astrid A Andersson, Jessica Lee, Caroline Dingle}{Introduced urban populations of endangered species may be key to the conservation of species impacted by wildlife trade or habitat loss in their native range, making it important to understand what factors influence survival of such ex-situ populations. The critically endangered Yellow-crested Cockatoo has become established in Hong Kong and Singapore; while the Hong Kong population has grown steadily over time, the Singapore population remains small. Competition for nest-holes, a rare resource in heavily urbanized areas, may play a role. In $>$500 hours of field observations, we documented significantly higher numbers of antagonistic interactions in Singapore than in Hong Kong, with top aggressors including other introduced species (Goffin's Cockatoos, Javan Mynas, Red-breasted Parakeets) as well as resident Oriental Pied Hornbills. In Hong Kong, very few antagonistic interactions were observed. Our results suggest that differences in competition between the two cities may at least partially explain the different patterns of population growth.}

\normaltalk{Differences in the Composition of Species of Water and Land Birds Between the Landscape Units of the Venezuelan Llanos}{Jesús E Aranguren-Alvarez}{The Venezuelan Llanos are the largest sedimentary basin in the country covering approximately one third of national territory. It is characterized by a relatively homogeneous topography, seasonally flooded areas and a vegetation of savannas, deciduous, semi-deciduous and gallery forests. The last regionalization proposed for this region is based on landscape units, which uses geomorphological and vegetation characteristics to define them. The llanera birdlife is represented by more than 400 species that use the Llanos as a feeding, reproduction and passage site during the autumn migration. Given the characteristics of the region, it is thought that bird species are distributed uniformly throughout the Llanos. Currently there are no biogeographical studies in birds that cover the entire region. Our goal is to determine if there are significant differences in the composition of species of water and land birds between landscape units. An exhaustive search of presence-absence data of birds including the eBird platform, museum records, scientific articles and inventories of research conducted in this region was performed. Grids were placed on the map of the landscape units and the species composition of the grids within each landscape unit and between landscape unit was compared by applying ANOSIM and SIMPER tests. The results show that there is a differentiation in the composition of species of water and land birds between some landscape units, however, this separation is weak. Despite this, it is becoming clear that the Llanos are not as homogeneous as initially thought, especially the southwestern and eastern Llanos.}

\normaltalk{Modeling Urban Bird-Habitat Associations with i-Tree in Santo Domingo, Dominican Republic}{Wayne J Arendt, David I King, Susannah B Lerman, Elvia J Mel\'{e}ndez-Ackerman, Mervin E P\'{e}rez, Jerry Bauer, none }{Planning effective conservation strategies in urbanizing areas requires an understanding of the variation of biodiversity along urban gradients and its association with specific habitat characteristics. i-Tree, a US Forest Service urban forest assessment tool, provides a standardized means for calculating the benefits of the urban green areas, including potential wildlife habitat. Our objectives in this study were to: 1) characterize the bird community in Santo Domingo, Dominican Republic, 2) describe habitat at our bird survey points using i-Tree, and 3) test whether i-Tree is a useful tool for planning biodiversity conservation in Santo Domingo, and potentially other urban areas in the Caribbean. We conducted standardized bird surveys during the fall, winter, spring and summer seasons of 2016--2019 at 74 survey points in Santo Domingo at points coinciding with i-Tree plots on which i-Tree characterization criteria and similar habitat measurements were quantified. These variables included physical factors (e.g., amount of impervious surface) and biological features (e.g., \% tree canopy cover, size and condition). We documented 100 species comprising 31\% of those using the Dominican Republic during their annual cycle. Bird species diversity, richness and abundance of Neotropical-Nearctic migrants, endemics, and exotic species were associated with i-Tree variables. The consistent relationship between species richness, abundance and distribution of breeding, wintering and migratory birds with habitat characteristics yielded from i-Tree measurements demonstrates the potential for i-Tree to contribute to planning efforts for birds in urban settings in the Caribbean.}

\normaltalk{Reintroduction Success of Yellow Cardinals (Gubernatrix Cristata) Recovered from Illegal Pet Trade in Argentina}{Melina Atencio, Marisol Dominguez, Romina C Scardamaglia, Juan C Reboreda, Bettina Mahler}{One of the main threats for the yellow cardinal (Gubernatrix cristata) is its capture for illegal pet trade. A reintroduction program for recovered yellow cardinals in Argentina has been carried out for the past three years. The objective of this study was to evaluate the success of these reintroductions through post-release monitoring of individuals. Mortality rates, settlement, reproduction and recruitment were determined. A total of 58 yellow cardinals were released in a private field in La Pampa province, Argentina, for three consecutive seasons (2017, 2018 and 2019); 39 of them were equipped with radio transmitters in order to carry out the monitoring via radio-telemetry. For these individuals, the percentage of predation events registered for the three seasons was of 30.6\%, occurring most of them (72.8\%) within the first two weeks post-release. Mean distance traveled between release site and settlement area was of 1,471 m (297.9 - 5,725.1 m). Survival was independent of vegetation densities, distance traveled from release site and year of liberation. Reproductive success of the population of yellow cardinals (released and local) withing the study area was registered for the breeding season of 2019. Reproductive pairs (n=9) were identified and nests (n=15) were monitored until the reproductive attempt ceased. All nests were parasitized by shiny cowbirds (Molothrus bonariensis) and suffered brood reduction as a cause of egg damage inflicted by the female parasite. The population presented null recruitment, mainly as a cause of brood parasitism. Based on these results, new strategies and conditions are discussed for future liberations.}

\normaltalk{eBird Status and Trends Update: Distributions and Abundances for 600 North American Species}{Daniel Fink, Tom Auer, Alison Johnston, Matt Strimas-Mackey, Orin Robinson, Shawn Ligocki, Marshall Iliff, Chris Wood, Ian Davies, Steve Kelling}{As of February 2020, eBird Status and Trend products provide ecological information for 610 North American birds, describing their ranges, relative abundance, and population trends at high spatial resolution. To generate these products, we use statistical and machine learning analyses combining eBird data with environmental data from NASA. The analyses are used to estimate the occurrence and abundance of species across the Western Hemisphere at weekly intervals and 2.9km resolution, as well as to estimate trends in relative abundance. These estimates are summarized to produce a suite of visualizations and data products available at https://ebird.org/science/status-and-trends. In this presentation we will provide an update about the 2018 version of Status and Trend products and what has changed since the 2016 version.}

\normaltalk{Adaptations to Light Predict the Ecological Niche and Disassembly of Avian Communities in Tropical Countrysides}{Ian J Ausprey, Felicity L Newell, Scott K Robinson}{The role of light in partitioning ecological niche space remains a frontier in understanding how terrestrial vertebrate communities assemble and respond to global change. Leveraging recent advances in biologging technology and intensive field surveys of cloud forest bird communities across an agricultural land use gradient in the Peruvian Andes, we demonstrate that eye size predicts (a) the exact light microenvironments used by wild birds, (b) the avian ecological niche, and (c) interspecific sensitivity to agricultural land use change. Mean light intensity levels used by 15 species carrying light sensors (N = 71 individuals) were best explained by eye size and foraging behavior, with larger-eyed species using darker microenvironments. Across the entire bird community, species using myopic ("near-sighted") foraging maneuvers had smaller eyes compared to hyperopic ("far-sighted") foragers, and myopic insectivores had larger eyes at lower foraging strata. Eye size strongly predicted sensitivity to four brightly lit habitat types within our agricultural land-use gradient. Species increasing in abundance in brighter habitats had smaller eye sizes, suggesting that light acts as an environmental filter when communities disassemble in human disturbed landscapes. We argue that eye size represents a novel functional trait that contributes to global patterns in terrestrial vertebrate community assembly and sensitivity to habitat disturbance.}

\normaltalk{Testing Physiological and Social Costs Associated to a Novel Plumage Trait in a Neotropical Songbird}{Jorge E Avenda\~{n}o, Carlos D Cadena}{An important step towards understanding the role of social selection (i.e. selection arising from competition for resources including mates; West-Eberhard 1983) driving signal evolution is to assess how do receivers perceive and respond to individuals exhibiting novel signals. How males respond to novel ornaments exhibited by other males may carry some physiological and social costs to senders and receivers, which ultimately may also influence the odds of fixation of novel signals. We simulated the presence of pectoral band (i.e. artificially painted) in a population of a Neotropical passerellid (Arremon assimilis) that lacks this trait, and compared physiological (i.e. physical condition, stress levels) and social-mediated costs (i.e. home range size) between banded and unbanded territorial males. Although we did not detect changes in body condition, corticosterone levels and home range sizes between banded and unbanded males before and after to plumage manipulation; we found that banded males were more susceptible to divorce and lose their territories compared to unbanded males that showed more stable pair bonds and maintain their territories along the study. Our results suggest that pectoral band is unlikely target of social selection in the context of competition for territories among males. However, sexual selection via female choice represents a candidate mechanism to explain the fixation and loss of plumage traits in this system.}

\normaltalk{Participatory Bird Census: a Post Hurricane Maria Citizen Science Program in San Juan Bay Estuary Watershed Ecosystems}{Coral E Avil\'{e}s-Santiago, Eliezer Nieves-Rodr\'{i}guez}{The San Juan Bay Estuary Program (SJBEP) is part of the National Estuary Program in the USA and as organization, implements actions to improve the water quality of the San Juan Bay Estuary watershed, an area in the metropolitan area of Puerto Rico that reaches 97 square miles and 8 municipalities. Three years ago, the SJBEP created a Citizen Science Certification Program to integrate citizens in a more structured way in two different areas: Participatory Bird Census and Water Quality Monitoring. The Participatory Bird Census Module trains and engages citizens to go in the field, identify and count birds in various ecosystems across the watershed using the Audubon and eBird protocols.  After Hurricane Maria struck the Island in 2017, the SJBEP, wanting to assess the state of water quality and bird populations after the disaster to direct relief efforts and plan long-term recovery, relied on the help of citizen scientists to collect important data about the state of the watershed. Our bird census volunteers have helped gather vital information on the state of the birds in different natural protected areas and urban forest during the past three years and growing citizen participation has allowed us to expand our reach to new areas across the watershed each season. Data collected was entered to eBird and to our online map Atlas del Estuario, where citizen science efforts are reported. This presentation will show bird census and citizen participation data collected within the Christmas Bird Count Season from December 2017 to January 2020.}

\normaltalk{Managing Nest Boxes for Invasive Species: What People Think, Know, and Do}{Tina B Phillips, Victoria Y Martin, Robyn L Bailey, Holly A Faulkner-Grant, David N Bonter}{Invasive species pose a significant threat to native flora and fauna. In North America, nest site competition from invasive cavity-nesting birds can limit the opportunities for successful nesting by native birds. Managing invasive species is costly and complex, requiring input from biologists, decision makers, and the public. An informed public, particularly birdwatchers, can play an important role in mitigating the negative effects of invasive avian species. However, little is known regarding birdwatchers' knowledge, attitudes, and behaviors toward non-native species. We examined the association between participating in a citizen science project and knowledge of, attitudes toward, and management of two non-native avian species: European Starlings (Sturnus vulgaris) and House Sparrows (Passer domesticus). We administered a survey to 934 nest-box monitors in the United States and Canada, 33\% of whom were also reporting their observations to NestWatch, a long-running citizen science project. We found that NestWatch participants were more likely to have negative views of non-native species, score higher on bird identification tasks, and manage for invasive species than non-participants. Overall, 57\% of respondents took management action against non-native species, and witnessing a nest usurpation was the most important predictor for management activities, followed by whether respondents believed non-native birds to be a problem at the continental scale. Our study highlights the important role citizen science may play in shaping attitudes and behaviors and increasing knowledge, as well as the importance of people's observations and perceptions of threats to engage in actions that can potentially support the survival of native birds.}

\normaltalk{Predictable Patterns of Plumage Gene Evolution in Wood Warblers}{Marcella D Baiz, Irby J Lovette, David PL Toews}{Neotropical wood warblers (Parulidae) are a colorful radiation of songbirds known for their striking plumage variation, even between closely related species. Plumage coloration is hypothesized to play a role in premating reproductive isolation and thus divergence in plumage traits may facilitate speciation. A previous study found that genomic divergence between two closely-related Vermivora species is limited to six small regions, four of which contain genes linked to plumage traits in other avian species. However, whether these genes underlie divergence in plumage coloration in other, more divergent Parulid species has not been investigated. To address this, we use whole genome sequencing of every species of Setophaga warbler (n = 34 species) to test whether divergence in these pigmentation genes predicts plumage color variation. By sequencing several individuals per species, we find that divergence upstream of the agouti signally protein (ASIP) gene generally separates Setophaga species with extensive melanic plumage on the face and throat from non-melanic Setophaga species. Similarly, divergence upstream of BCO2--a candidate gene for carotenoid processing--generally separates yellow from non-yellow Setophaga species. Our results are consistent with similarities in the genomic architecture of plumage color evolution in Parulids. We hypothesize that if plumage coloration plays a large role in premating isolation in Parulidae, rapid speciation in this group may be facilitated by repeated evolution in regulatory regions controlling a small number of plumage genes.}

\normaltalk{Comparative Genomics of a Sexually Selected Family of Birds}{Christopher Balakrishnan, Maude Baldwin, Morgan Wirthlin, Yasuka Toda, Manakin Genomics RCN}{Sexual selection has long been considered an important driving force underlying evolutionary novelty. In contrast to ecological adaptations, sexually selected traits provide a benefit primarily in the context of securing mates. Although a robust body of theory supports our understanding of sexual selection, little empirical data has been brought to bear regarding how strong sexual selection impacts patterns of genome evolution. A striking example of a clade under strong sexual selection is the neotropical family Pipridae, the manakins. These birds feature acrobatic courtship displays, morphological and endocrine modifications to support these displays and striking plumage colors, each of which varies across the manakin phylogeny. In this study we sequenced and assembled the genomes of five manakin species and compared those genomes to closely related avian outgroups. Using both population and comparative genomic approaches, we find evidence of selection on both protein coding sequences and conserved noncoding elements associated with sensory perception and muscle performance. Functional testing of taste receptors reveal important changes that occur before the radiation, implicating frugivory as a key factor promoting sexual selection. Taken together our analyses characterize patterns of genomic change preceding and during a radiation under strong sexual selection.}

\normaltalk{Do Himalayan Birds Wear Down Jackets?}{Sahas Barve, Viay Ramesh, Carla Dove}{Hundreds of avian taxa have independently evolved to live in frigid high elevation habitats around the world, and yet we know very little about the adaptations they have to keep warm in these extreme environments. A bird's plumage is an insulative barrier against the elements and montane birds have likely evolved specialized feather and plumage characteristics that shield them from freezing conditions. We sampled the feather morphology of 2249 bird specimens of 261 species collected between 75m A.S.L and 4804m A.S.L throughout the Himalayas. Linear mixed models that control for phylogenetic relationships between the species revealed that proportion of down in contour feathers increases along the elevational gradient, but more steeply in temperate versus tropical taxa. Conversely, downy barb density increased with elevation in tropical, but not temperate taxa. These alternate routes to increasing thermo-insulative properties of feathers underscores the significance of biogeography and evolutionary history as mechanisms driving physiological adaptation to environmental challenges in a hyper-diverse high-elevation avifauna.}

\normaltalk{Climate Change Affects Bird Nesting Phenology:  Comparing Contemporary Field and Historical Museum Nesting Records}{John M Bates, Mason Fidino, Laurel Nowak-Boyd, Bill M Strausberger, Kenneth A Schmidt, Christopher J Whelan}{Increasing data demonstrate that organisms respond to global climate change by advancing their spring phenology. Archived specimen-based data from museum collections extends the time period that can be assessed. We compare the nesting phenology of 72 bird species in northeastern Illinois, USA, using 1550 archived museum nest/egg records (1872-1963) and 3038 contemporary field records from 1983-2015. Overall, lay dates advanced ~ 10 days over the 143-year period. Twenty-four species (33.3\%) significantly advanced, one (1.4\%) significantly delayed, and forty-seven species (65.3\%) neither advanced nor delayed lay date. The variation in response across species illustrates there is much to learn about how species respond to human driven climate change driven. As museums archive historical data, these institutions are also ideal candidates to store contemporary field data as it is collected. Together, such information will provide the ability to track, understand, and perhaps predict responses to human-driven environmental change.}

\normaltalk{Opening the Black Box of Neotropical Stopover Regions Through a Multi-Scale Research Approach}{Nicholas J Bayly, Camila G\'{o}mez, Kenneth V Rosenberg, Keith A Hobson}{The Neotropics are occupied by a wide range of Nearctic-Neotropical migratory birds for up to eight months each year and yet how this region is used to power the great migrations of birds has been largely unappreciated thus far. Here we explore where, when and in what habitats long-distance migratory birds stopover and the contribution these stopovers make to the overall migratory journey. We examine how the Caribbean coast of Colombia is used both as a major fueling station prior to trans-oceanic flights, as well as for recovery by Blackpoll Warblers following endurance flights from North America. We also discuss the little-known role that overwintering sites play in fueling the first leg of spring migration. To reach these conclusions, we draw on multiple data sources and analytical techniques, including broad-scale occupancy models, banding data sets, tracking data, foraging observations, stable isotope analysis and flight range models. Together, this toolkit not only reveals how the abundance of migratory birds varies across extensive stopover regions but allows us to infer habitat quality, diet, tree use and more importantly, the contribution a given stopover makes to onward flights and the migratory journey as a whole. We propose that this multi-scale approach provides a rigorous means for identifying and describing the relative importance of stopover sites, and offers a framework within which conservation priorities and actions can be defined.}

\normaltalk{An Examination of Genomic and Acoustic Differentiation Between Eastern, Lilian's, and Western Meadowlarks}{Johanna K Beam, Erik R Funk, Scott A Taylor}{Understanding species boundaries is a fundamental, but challenging, component of describing and understanding the generation and maintenance of biodiversity. Examining differences among very recently diverged populations can provide insight into the traits and evolutionary mechanisms that drive divergence. The genus Sturnella includes two recently diverged species, the Eastern (Sturnella magna) and Western (S. neglecta) Meadowlark, the former of which has a complex of subspecies distributed across the Americas. Of the Eastern Meadowlark subspecies that occur in the U.S., S. m. lilianae is the only one with an allopatric range in the southwestern U.S. and central Mexico. It also has markedly different vocalization patterns than all other Eastern Meadowlark subspecies. In order to assess population differentiation, we performed whole genome sequencing of 35 birds as well as analysis of various song characteristics, including maximum and minimum frequencies. Results were visualized using principal component analyses and analyzed by running linear discriminant function analyses. S. m. lilianae exhibits high levels of genetic and vocal differentiation from both the Eastern Meadowlark and the Western Meadowlark, and likely forms a distinct evolutionary lineage. Additionally, the subspecies S. m. auropectoralis shows no genetic or acoustic differentiation from S. m. lilianae, suggesting that the subspecies falls within the lilianae group and not the magna group.}

\normaltalk{Sociality and Antipredator Behavior}{Amanda K Beckman}{Varying evolutionary pressures have resulted in the diversity and complexity of songbird sociality observed today. Some songbirds only associate with conspecifics during the breeding season, while others may form sizable flocks of con- and heterospecifics year-round. Songbirds also display a wide variety of cooperative behaviors (e.g., altruism, mutualism, and reciprocity). Benefits of sociality include improved antipredator behaviors, foraging, and individual fitness. However, increasing anthropogenic influence on the natural world means the cues and signals that songbirds use to communicate may become unreliable. Understanding how cooperative and antipredator behaviors evolved, how con- and heterospecifics interact, and the communication modalities used to execute these behaviors are essential to understanding the current and future social world of songbirds. The social complexities of songbirds are crucial to the success of management plans; if translocated individuals did not encounter each other previously then they may spend a disproportionate amount of time engaging in territorial disputes opposed to foraging and breeding (dear enemy effect). Social behavior has also been shown to be informative for determining if an invasive species can establish in a new location. Even if there is little information known about social or antipredator behaviors, investing time or resources to learn about the behavioral needs of songbirds can mean the difference between management success or failure.}

\normaltalk{Testing the Function of "flash" Plumage Signals in North American Passerines using Comparative Methods}{Lauryn Benedict, Jay McEntee, Harrison Jones, Kelsey Moran, Kasha Strong, Scott Robinson}{Avian coloration patterns are wildly variable. Despite this, there are plumage traits that occur similarly across taxonomic groups for which widespread explanations are lacking. Here we investigate why so many species have plumage ``flash signals" - color patches that are not visible when an animal is still but become visible with movement. The earliest interpretation was that flash signals warned conspecifics of danger and coordinated prey escape behaviors among flocking species. Other hypotheses suggest that flash signals function in prey capture, species recognition, or sexual and social selection. To test these hypotheses, we scored 325 species of North American passerines for the presence of flash patches on the rump, wings, and tail edges. We gathered flocking data for each species from the Birds of North America Reports and combined that with ecological information from the publicly available Elton Traits dataset. We used a model selection process to determine whether the following traits predicted the presence of flash signals in phylogenetically controlled models: winter flocking behavior (group cohesion hypothesis), mixed-species flocking (group cohesion and species recognition hypotheses), diet and flush-pursuit foraging (foraging success hypothesis), or plumage dichromatism (social/sexual selection hypothesis). Models controlled for habitat, migration behavior, and body size. Preliminary results indicate that flash signals are most strongly predicted by diet and plumage dichromatism. These results fit with multiple single-taxon studies that have shown flash signals to function in foraging and mate assessment, but run counter to the generally held assumption that such signals evolve to improve flock cohesion.}

\normaltalk{Genomic Data from Museum Specimens Illuminate Population Responses in a Tidal Marsh Songbird to a Century of Anthropogenic Change}{Phred M Benham, Jennifer Walsh, Rauri CK Bowie}{Accurately documenting the impacts human activities have on avian populations is a pressing goal in ornithology. Museum collections house long term series of specimens that often span human-mediated ecological change. Tidal marshes along the California coast have experienced dramatic change over the past century due to urbanization, agriculture, industrialization, and climate change. To understand how tidal marsh Savannah Sparrow populations have been influenced by these anthropogenic stressors, we sequenced 183 Savannah Sparrows from California using a custom designed exome capture array targeting exons from 14,000 genes and 3,500 non-genic regions. Sequenced individuals include modern and historic specimens from six tidal marsh locations and two populations from interior California. We also include specimens collected in the San Francisco Bay area from 6 time points (each decade from the 1890s to 1930s and 2015-2017) to assess whether population demography was more impacted by an earlier period of salt pond development or post-WWII urbanization. We performed a series of analyses to assess: (1) the impact of tidal marsh habitat loss to changing patterns of genetic diversity, population structure, and gene flow among tidal marsh populations; and (2) temporal variation in allele frequencies at loci locally adapted to salinity and temperature variation. With increasing tidal marsh fragmentation we find increased population structure among California populations of Savannah Sparrow, declining genetic diversity with increased habitat loss, and detect temporal shifts in allele frequencies associated with bill morphology genes that match observed phenotypic change in these populations.}

\normaltalk{Bird Friendly Cocoa: Linking Industry Sustainability Initiatives with Bird Conservation}{Ruth E Bennett, T Scott Sillett, Peter P Marra}{Cocoa, the main ingredient in chocolate, is traditionally grown in tropical agroforestry systems with high conservation value for birds. However, cocoa is also a driver of tropical deforestation, a fact that has come under scrutiny as global production increases. Cocoa sourcing companies increasingly face pressure to eliminate deforestation from their supply chains while increasing production and ensuring a stable future supply. Here we review how current industry sustainability initiatives--including zero-deforestation sourcing, eco-label certification, and sustainable intensification--impact bird communities. Using a meta-analysis, we show that endemic birds, frugivores, and insectivores are sensitive to both forest loss and cocoa intensification, while other bird groups maintain stable diversity in cocoa agroforests with at least 30\% shade cover. Industry sustainability initiatives align with bird conservation when habitat is retained either on or off cultivated areas, but the current push towards intensified production may have detrimental impacts on bird communities and associated ecosystem services. Current zero-deforestation agreements are well linked to governance frameworks that delineate and monitor primary forest patches in the top three cocoa-producing countries. However, eco-labels and sustainable intensification initiatives often overlook or weakly incentivize habitat retention in cocoa landscapes. The bird conservation value of these initiatives can be improved by defining where and how habitat will be conserved in or around intensified cocoa plantations.}

\normaltalk{The Role of Citizen Science for Bird Conservation in M\'{e}xico}{Humberto Berlanga}{CONABIO´s communitarian bird monitoring network, is a citizen science program designed to coordinate and operate bird monitoring activities in productive landscapes and natural protected areas, in order to document and assess the impacts of human activities, such as management and protection practices, as well sustainable productive practices on bird communities. We present the main achievements after the first 10 years of the Network, that currently continues growing and expanding not only in M\'{e}xico, but in Central America, incorporating systematic methods for bird monitoring based on eBird platforms, linked to local (resident species) and continental (migratory birds) research projects.}

\normaltalk{Genomic Phylogeography of the White-Crowned Manakin Illuminates a Continental-Scale Radiation Out of the Andes}{Jacob S Berv, Leonardo Campagna, Teresa J Feo, Ivandy Castro-Astor, Camila C Ribas, Richard O Prum, Irby J Lovette}{The complex landscape history of the Neotropics has generated opportunities for population isolation and subsequent diversification that place this region among the most species-rich in the world. Detailed phylogeographic studies are required to uncover the biogeographic histories of Neotropical taxa, to identify evolutionary correlates of diversity, and to reveal patterns of genetic connectivity, disjunction, and potential differentiation among lineages from different areas of endemism. The White-crowned Manakin (Pseudopipra pipra) is a small suboscine passerine bird that is broadly distributed through the subtropical rainforests of Central America, the lower montane cloud forests of the Andes from Colombia to central Peru, the lowlands of Amazonia and the Guianas, and the Atlantic forest of southeast Brazil. Pseudopipra is currently recognized as a single, polytypic biological species. We studied the effect of the Neotropical landscape on genetic and phenotypic differentiation within this species using genomic data derived from double digest restriction site associated DNA sequencing (ddRAD), and mitochondrial DNA. Most of the breakpoints we identify among populations coincide with physical barriers to gene flow previously associated with avian areas of endemism. The phylogenetic relationships among these populations imply a novel pattern of Andean origination, with subsequent diversification into the Amazonian lowlands. Our analysis of genomic admixture and gene flow establishes a complex history of introgression between some western Amazonian populations. These reticulate processes confound our application of standard concatenated and coalescent phylogenetic methods and raise the question of whether a lineage in the western Napo area of endemism should be considered a hybrid species. Lastly, analysis of variation in vocal phenotypes in the context of our phylogeny supports the hypothesis that Pseudopipra is a species-complex composed of 15-17 distinct lineages which have arisen in the last ~2.5 Ma.}

\normaltalk{Molecular Phylogeny of New World Swifts Based on Next Generation DNA Sequencing of Ultraconserved Elements (UCEs)}{Renata N Biancalana, Samantha Palhano, Therese A Catanach, Matthew R Halley, R T Chesser, Jason D Weckstein, Alexandre Aleixo}{Swifts and swiftlets comprise a globally distributed group of small to medium-sized insectivorous birds sister to hummingbirds. Swifts are an overlooked group in the New World, and, due to their aerial habits, are difficult to sample and identify in the field, and also frequently misidentified in collections. Few molecular studies have used mitochondrial and nuclear markers to reconstruct the phylogenetic relationships between swift species, these have mostly focused on Old World species. In this study we inferred a phylogeny of New World Apodidae using DNA sequences generated through next generation sequencing of ultraconserved elements (UCEs) enriched libraries and full mitochondrial genomes. We used 96 samples from 32 taxa, with robust geographic sampling across the Americas. Our findings suggest that New World swifts are not monophyletic with respect to the Old World swifts, that within the Cypseloidinae the genus Streptoprocne is not monophyletic, and that within the Apodinae the genus Tachornis is paraphyletic. It also revealed the potential for cryptic species in the genus Chaetura, which agreed with previously published studies.}

\normaltalk{Drivers of Declines in Common Loon Productivity in Ontario, Canada}{Kristin M Bianchini, Douglas C Tozer, Robert Alvo, Satyendra P Bhavsar, Mark L Mallory}{Common Loon (Gavia immer) reproductive success is an important indicator of aquatic ecosystem health. Existing evidence points to long-term declines in loon productivity; however, the reason for these declines is not well understood. Our objective was to identify drivers of the temporal decrease in loon reproductive success in Ontario, Canada. We analyzed reproductive data from the Canadian Lakes Loon Survey, a citizen science monitoring program managed by Birds Canada that has run annually in Ontario since 1981. Using 37 years of data from over 1600 lakes, we assessed how 16 factors affect loon reproductive success and determined whether these factors influence temporal trends in loon productivity. Our analysis estimated a declining trend in loon reproductive success of -0.10 young per pair per year between 1981 and 2018. The number of young per pair was lower on lakes that were smaller, farther west, more acidic, and that had fewer fish and higher fish mercury concentrations. Additionally, declines over time were steeper on lakes that were farther west, more acidic, and had fewer fish. We also found that chick production was lower following high winter precipitation and cool April temperatures. Surprisingly, reproductive success was higher on lakes where Bald Eagles (Haliaeetus leucocephalus) and Double-crested Cormorants (Phalacrocorax auritus) were present, and human disturbance and shoreline development did not influence chick production. We suggest that multiple interacting pathways are influencing declines in loon productivity. This information will be important for focusing future conservation efforts to help understand and mitigate threats to loon populations.}

\normaltalk{The Genomic Landscape of Admixture and Drivers of Reproductive Isolation Between Bullock's and Baltimore Orioles}{Jennifer Walsh, Shawn M Billerman, Vanya G Rohwer, Bronwyn G Butcher, Irby J Lovette}{Hybrid zones are excellent model systems for evolutionary studies, since with generations of ongoing gene-flow and backcrossing, they create a diversity of recombinant genotypes and phenotypes. Moreover, growing evidence indicates that hybridization and introgression are important forces that can shape the evolutionary trajectory of a species. Here, we explore a hybrid zone of historic importance in North American ornithology between the Baltimore (Icterus galbula) and Bullock's Oriole (I. bullockii) from the central Great Plains region. These species and their hybrids have been the subject of research, debate, and interest since the 1930's. While past research has revealed extensive admixture between them, as well as a westward-shifting hybrid zone, the mechanisms that foster their reproductive isolation are unknown. Using whole genome re-sequencing of pure and hybrid individuals, we tested for regions of the genome exhibiting elevated divergence. Genome-wide differentiation is moderate, with an average Fst of 0.21 between the two species and 72,356 fixed differences (representing 1\% of the genome). However, approximately 61,000 of these fixed differences are located on the Z chromosome, offering evidence of a possible structural inversion. We identified clear peaks of divergence across the genomic landscape, offering several new lines of evidence for loci contributing to reproductive barriers between Baltimore and Bullock's orioles. Comparison of divergence estimates between allopatric and admixed orioles provides insight into the role of introgression in shaping variation on a genome-wide scale.}

\normaltalk{Bird Population Trends: BBS and eBird}{Allison D Binley, Daniel Fink, Adam Smith, Richard Schuster, Scott Wilson, Tom Auer, Alison Johnston, Amanda Rodewald, Viviana R Gutierrez, Orin Robinson, Joseph R Bennett}{Migratory species cover such a large area between breeding and nonbreeding habitats that they require monitoring on a scale beyond what is achievable using traditional methods. The Breeding Bird Survey (BBS) completes longitudinal surveys across North America to examine annual trends in avian populations. However, the BBS does only roadside monitoring and only in the breeding season, leaving substantial gaps both spatially and temporally. eBird is a less structured, opportunistic citizen science project that takes advantage of the greater spatial extent achieved by crowd sourced monitoring, but consequently has more sources of variation to account for. The aim of this study is to determine how the BBS and eBird monitoring programs compare in producing broad-scale population trend estimates. Both datasets are modeled using the same Bayesian hierarchical framework so that any inconsistencies in the resulting estimates are due to differences in the collected data from each program. Preliminary results for Wood Thrush (Hylocichla mustelina) show agreement between the two programs across the breeding range. The results of this research will help to guide efforts for two of the largest ecological monitoring databases in the world and pave the way for the complementary use of both datasets and potential integration.}

\normaltalk{An Overview of the Use of Drone Technology for the Study and Conservation of Birds}{David M Bird, Dominique Chabot}{In an era of rapid global change, ornithologists are relying on an ever-increasing suite of novel tools to answer questions and solve problems related to avian research, monitoring, and management. Over the past 10 years, the use of unmanned aerial vehicles, or drones, has exploded in popularity in ecological studies in general, and ornithology in particular. Drones have many advantages over traditional survey and research techniques, as they may minimize human mortality due to plane crashes, reduce cost and disturbance, increase accuracy, and allow the collection of high-resolution data over large and/or otherwise inaccessible areas.   Some of the major areas of application of drones that have emerged in avian research and conservation include photographic population surveys, including breeding colonies and non-breeding aggregations and the use of different types of camera sensors (e.g. visible, thermal-IR); 2) individual nest inspections, most often involving raptors; 3) radio-tracking surveys, involving the use of radio telemetry sensors; 4) acoustic surveys, involving the use of song-recording sensors; 5) avian habitat research and monitoring, involving high-resolution 2D and 3D mapping; and 6) bird dispersal, either for nuisance birds or to deter birds from hazards.  This review provides an update on the progress, including success and failures, for ornithologists using, or planning to use, this technology.}

\normaltalk{Nature's Palette: An Open-Access Digital Repository of Spectral Data}{Pierre-Paul Bitton, Rabeya Akhter, Alison Farrell, Aidin Talaeizad, Adrian Fiech}{The roles of light environment, colouration, and colour vision have been fundamental questions in animal behavior and ecology for the last several decades. To address these questions, the objective quantification of light and the energy reflected at different wavelengths (reflectance) with portable spectrophotometers are often necessary, and typically require thousands of measurements. These usually come from ambient natural or artificial light sources (irradiance), transmittance properties of the medium (air, water), and reflective properties of animal integuments and the substrates in their environment. However, very few of these data are made publicly available to other researchers, and even those that are through repositories are not curated in a way that makes them easily included in other studies. Consequently, the opportunities for large-scale studies are limited, and substantial resources are wasted on duplicating data for different research projects. Here we present the prototype of an open-access online repository of spectral data. It allows researchers to upload, search, and download all types of spectral data. We highlight its current state of development, and present our future vision of its functionality. We believe that this repository will provide a fantastic resource of spectral measurements and encourage large-scale projects through reuse of existing data.}

\normaltalk{Adapters, Avoiders and Exploiters: How Three Terms Shaped Urban Bird Research.}{Robert B Blair}{In 1994, I used the terms ``adapters", ``avoiders", and ``exploiters" to describe the patterns of bird distributions that I documented on a rural to urban land-use gradient in Palo Alto, California. Since that time, the use of those terms has exploded in the urban-bird literature and crossed over to describe the distributions of many different taxa including bats, butterflies, termites, generic urban ``wildlife", and even plants. The framework provided by thinking of distributions in this way has allowed urban ornithologists to explore much more than where birds ``are" but also to explore the processes that drive these different distributions, the effects these have on community composition, how landscape configuration can bring about these patterns, and the drivers that are causing biotic homogenization at the global level. In this talk, I will trace the terms' use, how they have furthered (and perhaps hindered) urban-bird studies, and what that means for future research in conservation.}

\normaltalk{An Agent-Based Model of Winter Foraging and Energetics of North American Waterfowl: Applications of a Decision-Support Tool}{Robert Blenk, Sean Fogenburg, Kevin Ringelman, Matt Miller, Jeff Schank, John Eadie}{For many wintering populations, the amount of energy available on a landscape is a limiting factor. This is held to be true for wintering waterfowl across much of North America. However, current dominant methods of evaluating energetic carrying capacity for waterfowl are not spatially explicit and are unable to take into account the considerable costs and limitations associated with acquiring food-- namely the costs of traveling to and from foraging patches, actively finding and acquiring food, and metabolizing that food. We present the Spatially-explicit Waterbird Agent-based Modeling Program (SWAMP), a spatially-explicit decision-support tool for conservation researchers and managers, and present an application of this program in California's Butte Basin, critical winter habitat for the conservation of North American waterfowl. By running parallel simulations representing different possible future conditions for waterfowl, we provide insight as to the relative importance of factors such as continuing drought, urban expansion, changes in rice agricultural practices, and further restoration of natural wetlands.  Notable among our results is the heretofore undervalued importance of continued flooding of rice fields as a crop management technique even relative to restoration of natural wetlands. SWAMP is also currently in the process of undergoing expansion into other landscapes, as well as adaptations for application to the conservation of other species such as the endangered Siberian Crane.}

\normaltalk{Long-Term Banding Provides Insights to Weather Influenced Survival and Recruitment of Bananaquits}{Clint W Boal, Brent D Bibles}{The Caribbean region is a biodiversity hotspot and a priority for ecological conservation efforts. Understanding how environmental conditions influence survival of Caribbean birds is an important information need given the predicted increase of drought, and increased frequency and intensity of storms. We used 10-year capture-mark-recapture data set to examine weather influences on survival and population growth of Bananaquits (Coereba flaveola). Bananaquits are widespread throughout much of the Caribbean and may serve as viable representative species to assess forest passerine community response to environmental changes in the region. We assessed survival, population growth, and age structure in context of monthly and seasonal precipitation and temperatures, drought conditions based on the Standardized Potential Evapotranspiration Index (SPEI), and occurrence and intensity of storm events. Our analysis was based on capture histories for 466 individuals. Our models suggested age, sex, occurrence of storms, and SPEI all influenced survival. Survival from hatch year to adult increased for males but decreased for females. The incidence of storms and drier than average conditions in the preceding December-February both decreased survival. Our recruitment and population growth analysis indicated storm events influenced survival, sex influenced recapture probability, and storms and winter SPEI influenced population growth. During non-storm years, population growth was 1.11, but dropped to 0.76 in storm years. Increasing frequency of storms, such as back-to-back years, would likely push the growth rate lower. A protracted pattern of increased storm frequency, especially if coupled with subsequent dry winters, may lead to localized extinctions.}

\normaltalk{Neurogenomic Underpinnings of Testosterone Modulated Cooperative Lek Behaviours in the Wire-Tailed Manakin (Pipra Filicauda).}{Peri E Bolton, Christopher N Balakrishnan, T B Ryder, Ignacio T Moore, Brent M Horton}{Testosterone (T) is an important modulator of trade-offs in sexual and social traits in vertebrates. In Holarctic oscines, high T is typically associated with increased aggression at the expense of prosocial behaviors such as parental care. In the wire-tailed manakin (Pipra filicauda) Territorial males produce cooperative lekking displays with subordinate Floater males who generally have lower circulating T. Cooperative behavior predicts paternity among Territorial males, and ascension of Floaters to Territorial rank. Recent observational and experimental work has demonstrated that T modulates individual variation in cooperative behavior, with status specific effects. High T among Floaters is associated with territorial ascension and high T among Territorial males reduces cooperative behaviors. One possible underlying mechanism for this specificity is that Territorial and Floater males differentially express genes in the androgenic pathways, enabling status specific sensitivity to T. In this study we sought to characterize social status specific gene expression and regulatory networks using RNAseq across steroid responsive tissues in the avian Hypothalamic-Pituitary-Gonadal (HPG) axis, and the Social Behavior Network (SBN) of the brain. We characterized tissue specific patterns of gene expression using weighted gene co-expression network analysis, and identified Galanin as a hub of a neuroendocrine regulatory network. Our study is the first to use this comprehensive whole system level approach to address complex social behavior in birds, or any model organism. Our results are discussed in the context of an integrative understanding of the evolution of androgen mediated phenotypes in birds.}

\normaltalk{World Migratory Bird Day Conservation Theme Joins Communities for Conservation}{Susan Bonfield, Miguel E Matta}{World Migratory Bird Day (WMBD) is a global education campaign that serves as a call to action for bird conservation. In 2020, the program focuses on the methods researchers use to track bird migrations to inform conservation throughout their life cycles. We explore geolocators, banding, feather analysis, citizen science, Icarus, and other technologies that detail the timing, length, and distance of bird migrations, as well as identify the places where birds nest, stop to rest and refuel, and spend the non-breeding months. Hosted at diverse venues, WMBD serves to connect researchers with the public, communities with conservation, and people to nature and birds by providing unique opportunities to observe birds, learn about their migrations, and take action to protect them follow simple activities. This presentation serves as an introduction and overview to the specific examples of how we track birds, the materials we create to communicate with the public, and how WMBD motivates on-the-ground conservation actions.}

\normaltalk{Perfluorooctane Sulfonate (PFOS) Concentrations in Birds Around the World}{Andrea Bonisoli-Alquati, Raul D Flamenco}{Perfluorooctane sulfonate (PFOS) is a highly persistent carcinogenic and endocrine-disrupting chemical used in a variety of consumer products. Several studies indicate its potential to bioaccumulate and biomagnify in food webs. A member of the infamous per- and polyfluoroalkyl substances (PFASs), PFOS was largely phased out of production in 2000s, yet its chemical stability might make such measures ineffective in reducing its environmental concentrations. No comprehensive analysis exists of PFOS concentrations in birds though time and space. This systematic review and meta-analysis explores temporal, geographic, and taxonomic variation in PFOS concentrations across birds in relationship to their ecological traits and the protocol for analysis. We collected more than 500 estimates of PFOS concentrations in blood, liver, and eggs from more than 100 bird species across all continents. PFOS concentrations in liver and eggs have increased since the 1960s, while blood concentrations have decreased since the phase out. As a result, PFOS concentrations were over three times higher in liver and eggs compared to blood in recent years. The overall concentrations and temporal trends also depended on ecological traits. Aquatic birds had higher concentrations than terrestrial ones. Piscivores and opportunists had the highest concentrations among all feeding guilds, while herbivores had the lowest. Larger clutch sizes and higher longevity also positively predicted PFOS concentrations across species. Our results support extensive ecotoxicological risk from PFOS in wild populations of birds, inadequately addressed by the phase out of PFOS production. Interspecific variation is largely consistent with PFOS bioaccumulation and its biomagnification through food webs.}

\normaltalk{Defining the Genomic Health of Declining Burrowing Owl Populations and Informing Captive Breeding Programs}{Christen M Bossu, Kelly Barr, Debra Chromczak, Thomas Smith, Lynne Trulio, Kristen Ruegg}{The use of genomic data has power to combine information from neutral and adaptive loci to delineate conservation units, provide a basic understanding of the genetic health of populations, and inform translocation and captive breeding programs. Western Burrowing owl populations are sharply declining across their continental US range due to habitat loss, secondary poisoning by rodenticides, mesopredator release, and impacts at wind energy sites which has resulted in extensive financial investment in captive breeding and translocation efforts. Here, we work with conservation practitioners to leverage improved genomic methodologies for assessing population structure in Burrowing Owl populations across the range and measure the extent and severity of inbreeding. Our results support the idea that resident Burrowing Owls are genetically distinct from migratory Burrowing Owls, but no population structure exists across migratory populations. Moreover, we find high levels of inbreeding in resident populations, suggesting that smaller, often declining resident populations have lower genetic health than their migratory counterparts. Working in collaboration with captive breeding programs in Northern California, we develop a decision tree approach for integrating insights gleaned from genomic-level analyses into on the ground conservation efforts.}

\normaltalk{Using Motus to Test for Breeding Season Carry-over Effects in a Long-Distance Migrant}{Brendan P Boyd, Sue M Hayes, Bridget J Stutchbury}{Habitat loss and fragmentation on the breeding grounds of many bird species has been found to have a negative effect on immediate breeding success; however, long-term impacts on breeding adults via negative carry-over effects have not been studied. The Motus Wildlife Tracking System is an automated radio telemetry array that, for the first time, can link breeding season habitat quality to fall migration and annual survival. Individuals occupying small forest fragments are expected to experience high rates of brood parasitism and nest predation, which could directly delay fall migration due to timing constraints from late re-nesting or indirectly delay migration if adults are in poorer condition. If these small fragments are comparably worse to live in, blood sampling is expected to show increased baseline levels of corticosterone, the stress hormone in birds. During the 2016-2019 breeding seasons, I fitted 117 adult Wood thrushes with coded radio-tags in a variety of forest fragments to track their movements using Motus. In addition, for two years, I collected blood samples from birds that were tagged. Wood thrushes are a declining long-distance migrant that is large enough to carry a radio-tag with a one year battery life, allowing detection of adults who return to the study area the following year.  While neither a clear relationship between fragment size and fall migration date or baseline corticosterone has been identified, Wood thrushes breeding in small forest fragments were less likely to survive migration and return the following year than those breeding in large fragments.}

\normaltalk{More than Meets the Eye: High-Speed Video Reveals Aerobatic Performance and the Production of Mechanical Sounds in Mating Displays}{Alice Boyle, Elsie Shogren, Luke Nguyen, Lainy Day}{Male manakins attract mates with diverse physical and acoustic displays. Understanding details of these displays has revealed novel mechanisms of sound production, the functional significance of other traits, and afforded insight into the process of sexual selection. However, some behaviors remain a mystery due to their high speeds including those of Corapipo altera. Most of this species' displays are performed at speeds resolvable by the human eye, occurring silently on or near an understory log. However, final, pre-copulatory elements involve an above-canopy flight followed by a very high-speed dive and hop-about-face involving three distinct sounds ("flap-chee-wah") produced in unknown ways. We recorded 94 high-speed video clips capturing sub-canopy portions of this display, performed by nine males at four logs, and synchronized video with audio recordings. The flap occurs as the male emerges from a $>$35 m/s head-first dive by fanning wings open with a shuddering motion. The sound likely results from aeroelastic flutter of modified outer primaries, a mechanism of sound production previously undocumented in Pipridae. Speed decreases to ~20 m/s via a ``parachute," then the male swoops over the log, twisting, head-down, opening wings to bank and land backwards; the ``chee" occurs during this twist. The final ``wah" is vocal, occurring during the apex of an aerial cartwheel. Individuals were remarkably consistent in flight paths and landing locations when displaying alone, yet respond flexibly to female location. Individual differences hint at the ontogeny of performance and variation associated with female choice.}

\normaltalk{Does Intercropping Switchgrass in Private, Working Pine Forests Affect Avian Diversity and Abundance?}{Rebecca D Bracken, Daniel U Greene, Darren A Miller, Scott A Rush}{Wildlife conservation on private, working forests has received increased recognition in recent years, especially for at-risk species, including many songbirds. Although forest management effects on avian diversity and abundance have received much attention, less is known about effects of alternative practices, such as intercropping switchgrass (Panicum virgatum) for biofuel, within managed loblolly pine (Pinus taeda) stands. We tested the hypothesis that intercropping switchgrass alters species composition relative to non-intercropped stands (control). We conducted point counts within both loblolly pine stands intercropped with switchgrass and controls for 9 years within a forest landscape consisting predominantly of loblolly pine stands in Kemper County, Mississippi. We surveyed 5 blocks of each treatment, 3 surveys each per treatment, with 5 replicates each year. We compared number of detections of each bird species within intercropped and control treatments for each year.  We detected 69 species, with 71\% overlap in species assemblages between intercropped and control plots. We detected 11 species, including Sedge Wren (Cistothorus stellaris) and Eastern Meadowlark (Sturnella magna), only in switchgrass plots.  We detected 9 species, including Ovenbird (Seiurus aurocapilla) and Worm-eating Warbler (Helmitheros vermivorum), only in control plots. We found little effect of switchgrass on abundance of individual species. Further evaluation of interplay between local and landscape features will be explored within working forests to best apply management practices to optimize performance of these systems relative to forest production and conservation goals.}

\normaltalk{Resolving Difficult Nodes in Manakin Phylogeny using Low-Coverage Whole Genome Sequencing}{Min Zhao, Noor D White, Rafael N Leite, Andre E Moncrieff, Sarah Kurtis, Robb T Brumfield, Rebecca T Kimball, Edward L Braun}{Phylogenetic relationships among many groups of birds have proven surprisingly difficult to resolve in a convincing manner, even with large datasets. Many groups of birds include difficult to resolve clades; the manakins (Pipridae), a family of Neotropical suboscines, represent one such group. We targeted two clades within manakins, the genus Lepidothrix and the clade comprising the genera Chiroxiphia and Antilophia, for whole genome sequencing. We achieved $>$95\% coverage of typical single copy regions by mapping reads to those regions as long as we had at least 10x coverage of the genome for each taxon. This makes it possible to extract and analyze a broad set of phylogenetic markers, including introns, exons, ultra-conserved elements (UCEs), and other regions. We then integrated the whole-genome data with a more limited set of UCEs for all manakin species to produce a comprehensive and strongly corroborated manakin phylogeny. Finally, we assessed discordance among gene trees to understand why specific nodes have remained challenging to resolve.}

\normaltalk{The Role of Environmental Heterogeneity in Driving the Adaptive Diversification of the Antbirds (Aves: Thamnophilidae)}{Gustavo A Bravo}{The study of continental radiations offers opportunities to understand the mechanistic underpinnings of species and phenotypic diversity. The Thamnophilidae, commonly known as antbirds, is one of the largest Neotropical suboscine families comprising ca. 240 insectivorous species that are mostly restricted to the lowlands and lower montane forests of Central and South America. Antbirds are diverse in their vocalizations, plumage, and body size and shape. They exhibit high species richness at many Neotropical sites, especially in Amazonian forests, where as many as 45 species may co-occur, but they have also reached the most prominent South American dry regions. Here, I integrate museum-based genomic, phenotypic, and distributional data to assess how phenotypic diversity accumulates and how it associates with environmental variation. Using comparative and genomic analyses, I show that not all phenotypic traits evolve at the same rates and that some of them involve in tandem in association with environmental filters, such as those imposed by temperature and precipitation regimes. For instance, darker species occur in humid and warm habitats, whereas those with lighter plumages occur in dry and seasonal environments, consistent with predictions by Gloger's rule. The comparison of whole deep-coverage genomes of 14 species representing both dry and humid habitat species suggests that these plumage-habitat associations are not explained by unique convergent genomic mechanisms, but rather by a combination of differences in coding and non-coding regions leading to similar phenotypes. These results highlight the potential of an integrative approach combining cutting-edge genomic and comparative methods to study specious continental radiations.}

\normaltalk{Are Populations of Tropical Birds Really Regulated by Density Dependence?}{Jeffrey D Brawn, Henry Pollock, Thomas J Benson}{A tacit assumption about the evolutionary ecology of many tropical species is that density dependence is a strong regulating force on their populations.  While plausible, this assumption has received little critical attention; suitable data sets are not common and field experiments are generally infeasible.  Using data from a capture-mark-recapture study that spans over 40 years, we took a model-based approach to questions about the importance of density dependence on temporal variation in population growth rates (lambda).  We employed an analytic technique recently presented by Tenan et al. (2019, Ecology) and modeled the capture histories of 20 understory species sampled in Panama.  This technique is an extension of the temporal symmetry or Pradel model.  Preliminary results indicate that density dependence can be a strong regulating force on lambda, but this effect is not consistent among or within species.  Thus, when possible, the assumption of density dependence should be assessed empirically.}

\normaltalk{Nests in the City: Complex Patterns of Nesting Behaviors in Two Populations of an Urban-Adapting Songbird}{Samuel A Bressler, Eleanor S Diamant, Pamela J Yeh}{Urbanization poses novel challenges to bird life. To succeed in the city, birds must express beneficial traits produced from selection, phenotypic plasticity, or environmental filtration. Urbanization has been shown to impact avian nesting phenology, behavior, and success. However, the extent to which these traits evolve similarly in isolated urban populations is less understood. We examined whether nest characteristics and success vary between two urban Southern Californian populations of Dark-eyed Juncos (Junco hyemalis). In Southern California, this ground-nesting songbird historically bred only in the mixed-pine forests of the mountains; but, in recent decades, this species has successfully colonized several cities throughout Southern California. Here, we examined 1) which nest characteristics predict success; 2) whether pairs modify nest characteristics in response to prior nest outcome in a population on the UCLA campus in Los Angeles; and 3) how nest height and success rates compared between this UCLA population and a population in UCSD, approximately 170 km away. We found that first egg date was significantly correlated with nest success, and height above ground was positively but not significantly correlated with success at UCLA. In addition, pairs changed nest substrate, but not height above ground based on prior outcome. Finally, the UCLA population more frequently nested above ground than the UCSD population, and exhibited higher success rates due to greater success of on-ground nests. Our findings suggest that adaptations to urban environments vary across populations, and that the selection pressures behind novel nesting behaviors are complex and multifaceted.}

\normaltalk{Are Naturalized Parrots Priority Invasive Species Warranting Control?}{Donald J Brightsmith, Simon Kiacz}{Naturalized species may be a leading threat to global biodiversity, but this contention has been hotly debated. The largest global driver of invasive species management is the Convention on Biological Diversity which called on member nations to identify priority invasive species and control or eradicate them by 2020. About 50 parrot species (15\% of all species) have naturalized populations. Two species are particularly successful with over 40 populations each: Rose-ringed and Monk Parakeets. Much has been written about their negative ecological and economic impacts, and in response to their success, they have been repeatedly labeled priority invasives and targeted for control. However, our literature review suggests that few impacts have been quantified by scientific study and nearly all naturalized parrot populations have no or low-level impacts. We could find only two cases where naturalized parrots were having population level impacts on other species: Rose-ringed Parakeets displace greater noctule bats in Sevilla, Spain and slightly suppress periurban nuthatch populations in Western Europe. By comparison, the IUCN Red List shows that 1,312 species of birds and mammals are threatened by non-native invasive species, and our work suggests only two (0.15\%) are being impacted by parrots. Economic impacts of naturalized parrots in the US and Europe are similarly small: naturalized parrots cause $<$ 0.005\% of the total damage caused by all invasive species combined. Given the very small relative ecological and economic impacts, we feel it is difficult to label naturalized parrots as Priority Invasive Species.}

\normaltalk{Mapping of Key Conservation Areas in the Era of Big Data: Exploring eBird to Identify Important Bird Areas and Key Biodiversity Areas}{Joanna X Wu, Taylor Broadhead, Chad Wilsey}{As the U.S. partner for Birdlife International, Audubon leads the charge to identify, monitor, and protect over 2,800 Important Bird Areas constituting over 417 million acres (1.6 million km2). One challenge the U.S. and other IBA/KBA partners face is with so many sites, they often lack the capacity to conduct time-intensive surveys to update species information. Meanwhile, eBird has become the world's largest biodiversity-related community science project, with database containing more than 737 million bird observations. In this project, we explore using the rich data presented in eBird to update species information in Important Bird Areas of the U.S. We select the species that fell within IBA boundaries and looked for those that met criteria of being an Important Bird Area, (e.g. IUCN Red List endangered, more than 1\% of global population observed). Approximately 65\% of Audubon's 1092 existing global IBAs can be updated with more recent species information via this method. In addition, eBird has also identified that 69\% of our 2640 IBAs at the state or continental level may qualify as global IBAs/KBAs. We caution that this exploratory analysis is preliminary, and the criteria for Key Biodiversity Areas have been updated recently. However, it is a powerful prove-of-concept that many Important Bird Areas may be able to be updated by a publically available mega-database rather than resource-intensive, ground-based survey efforts.}

\normaltalk{AZE in the Dominican Republic: Conservation Status, Threats and Initiatives for Improvement}{Jorge Brocca}{The Dominican Republic contains 38 KBAs and 4 AZE sites. They present great conservation challenges for the country and its inhabitants. SOH Conservaci\'{o}n has been making interventions in the most important KBAs in the country, demonstrating ways in which civil society, the private sector and governments can work together for conservation success. In this talk, we will discuss initiatives at several sites, such as Bahoruco National Park, Domingo Fuerte Monument, Valle Nuevo National Park, and Loma Charco Azul Biological Reserve.      La Republica Dominicana contiene  38 KBA y 4 AZE. Las mismas presentan grandes retos de conservaci\'{o}n para el pa\'{i}s y sus habitantes. SOH Conservaci\'{o}n viene realizando intervenciones en las mas importantes KBA del pa\'{i}s, demostrando que la sociedad civil en conjunto con el sector de cooperaci\'{o}n y el sector privado puede ayudar a los gobiernos a desempe\~{n}ar una mejor tarea si se unen y trabajan en conjunto. SOH Conservaci\'{o}n ha impactado mas de 300,000 hect\'{a}reas en parques como Parque Nacional Bahoruco, Monumento Domingo Fuerte, PN Valle Nuevo, Reserva Biol\'{o}gica Loma Charco Azul, entre otros.}

\normaltalk{Hurricane Katrina Impacts the Breeding Bird Community in a Bottomland Hardwood Forest of the Pearl River Basin, Louisiana}{David Brown, Thomas Sherry}{We studied breeding bird communities and vegetation before and after Hurricane Katrina severely damaged the extensive bottomland hardwood forest of the Pearl River basin, south Louisiana. Wind felled many trees and branches and the canopy opened considerably (57\%). Following the storm blackberry thickets expanded to cover almost all of what was previously a patchily open forest understory. The bird community changed distinctively following the hurricane, driven primarily by increased density of species that prefer dense understory habitat that was regenerating following the storm. Individual bird species that increased in density included one year-round resident, Carolina wren, and five breeding migrants, white-eyed vireo, Swainson's warbler, Kentucky warbler, hooded warbler, and yellow-breasted chat. These patterns were predictable responses to the opened canopy and increased density of understory vegetation. However, over three years following the storm, most species, especially canopy breeders, showed no distinct numerical response to the hurricane, which suggests that the initial bird community was resistant to hurricane disturbance. Only one species, Acadian flycatcher, declined significantly after the hurricane, presumably because of loss of its preferred open understory breeding and feeding habitat. Our results show the important role of hurricanes in structuring forest bird communities by altering understory habitat. We expect habitat changes will continue as invasive plant species further change forest community structure, and as large storms increase in frequency in relation to global climate change. Thus, we also expect continued changes to the bird community, which may include additional future declines.}

\normaltalk{Energetic Consequences of Different Migration Strategies: An Annual Cycle Perspective from Lesser Black-Backed Gulls}{Morgan Brown, Willem Bouten, Kees Camphuysen, Judy Shamoun-Baranes}{Migratory patterns of birds are incredibly diverse, not only between species, but also within species. To understand the persistence of such diversity, the costs and benefits of different migration strategies needs to be assessed. One of the primary costs of migration is thought to be the high energetic demand of flying between wintering and breeding areas. However, the increased energetic cost of flying to lower latitudes could potentially be offset by reduced thermoregulatory costs as wintering areas approach the equator. The objective of this study is to examine whether migrating further is indeed more energetically costly within the context of the complete annual cycle. We use lesser black-backed gulls (Larus fuscus) breeding around the North Sea as a model, where breeding colonies contain a mix of individuals who winter anywhere between the UK (latitude of 53°) and West Africa (latitude of 16°). Individuals were tracked year-round using GPS and acceleration, and detailed time-activity budgets were constructed for each annual cycle. Variation in resting costs were estimated by a bio-energetics model validated with respirometer measurements, and activity costs are based on results from an energetic study performed during the breeding season. We show the relative contribution of locomotion and thermoregulation to annual energy budgets among different migration strategies. Furthermore, we compare how energy expenditure is distributed throughout the year to identify whether certain strategies may experience energetic bottle-necks. Potential fitness implications of variation in annual energy expenditure caused by using different migration strategies will be discussed.}

\normaltalk{Evaluating the Effectiveness of Marbled Murrelet Habitat Conservation Planning: Disentangling Management from Environmental Factors}{Kristin M Brunk, Sal Chinnici, Anna Pidgeon, M Z Peery}{Rigorous evaluations are imperative for gauging effectiveness of conservation policies and guiding future conservation investments. Habitat Conservation Plans (HCPs) are a commonly utilized policy under the Endangered Species Act that guide habitat conservation on non-federal lands. However, evaluating HCPs is challenging, and the effectiveness of this conservation policy is ambiguous. We used a quasi-experimental framework and 17-years of inland monitoring data to evaluate a high-profile HCP developed to protect a large tract of old-growth and residual old-growth forest used for breeding by Marbled Murrelets. The probability of murrelet presence and behaviors indicative of breeding were significantly lower on private HCP lands than on adjacent protected public areas, based on multi-state dynamic occupancy models, but trends in occupancy did not differ between private and public land over time. Inland murrelet counts were higher on public than private land but declined over time equally on both ownership types, based on generalized linear mixed models. Collectively, our findings suggest that this HCP protects important murrelet nesting habitat on private land and that observed declines in inland murrelet counts may largely be due to broad-scale environmental factors rather than the HCP. We note, though, that habitat quality on the private HCP land is not equivalent to that observed on protected public land. Our results highlight the importance of utilizing controls to put conservation policy evaluation results into context. This is particularly important when the focal species may be affected by broad-scale environmental factors outside the scope of an individual HCP.}

\normaltalk{Predicting Pasture and Forest Landowner Intention to Create Early Successional Habitat}{Lesley P Bulluck, Hannah Coovert, Rene Valdez, Dustin Martin}{Management practices of private landowners are essential for effective conservation. Early successional habitats (ESH) are a priority in the eastern US, and efforts to create ESH on private lands have focused on forest landowners and timber harvests. Private pasture lands present an additional opportunity, yet our understanding of landowner values and attitudes about management strategies in pastures (i.e., reducing mowing or grazing practices) is lacking. To address this, we surveyed private landowners in five Virginia counties who own at least 25 acres above 2000 feet elevation. This is a high priority region for declining bird species. Our objective was to understand what influences private landowner intention to carry out six management strategies for the benefit of wildlife in the next five years: reduced mowing, reduced grazing, timber harvest, prescribed fire, use of machinery, and use of herbicides. We used boosted regression trees to determine what factors best predicted intention to carry out each strategy. We predicted landowner intention to manage (accuracy $>$ 75\%) with only a small subset of predictors for open pasture and timber management strategies. Landowner values were important yet not consistent across different management strategies; those likely to reduce mowing/grazing valued ecological aspects of their land (i.e., pollinator habitat and water quality) whereas those likely to harvest timber valued hunting and revenue. Past experience with wildlife management was the strongest predictor of intention to reduce mowing and grazing. Expanding outreach efforts to include pasture management options for creating ESH would engage a broader set of landowners.}

\normaltalk{Mapping Trait-Based Extinction Risk: Identifying Areas of Conservation Concern for Parrots in a Changing World}{Kevin R Burgio, Rachel L White, Flavia Monta\~{n}o-Centellas, Katie E Davis, Steven J Presley, Michael R Willig}{Parrots are one of the most at-risk groups of birds, with over 40\% of species listed as near-threatened or worse by the IUCN. Generally, parrots are at risk for many of the same reasons as are other birds, namely climate change and habitat loss; however, parrots also face other threats such as widespread trapping for the pet trade. Historically, one way to assess a species extinction risk has been to evaluate which traits seem to be shared among species within that taxon that are threatened or extinct. Until now, these analyses have not been spatially explicit, and therefore, not particularly useful for conservation planning. We used phylogenetic generalized least squares regressions within the R `caper' package to evaluate traits directly linked to the significant drivers associated with population declines in parrots (e.g., logging and cavity-nesting dependency, desirability for the pet trade, and traits related to climate). We then used the global distribution of all 398 extant parrot species to quantify community-level extinction risk, based on the assessment of the traits of each of the species found in each 50 × 50 km community throughout the entire global range of parrots. We highlight areas of particular concern around the world that do not correspond with more traditional ``hot spots" or areas high in species richness that would benefit from further evaluation and elevate these areas to promote more attention from conservationists.}

\normaltalk{USAvian: a Tool for Enhancing Connections among and Within the Bird Management, Conservation and Research Communities}{Jessica L Burnett, Kevin R Burgio, Auriel M Fournier}{The ubiquity and sometimes idiosyncratic nature of bird conservation and management in the US prevents even well-informed ornithologists and bird enthusiasts from fully understanding the bird conservation landscape at various scales. As a result, there is risk of non-communication among spatially- and taxonomically-adjacent conservation groups, potentially leading to overlap in project scopes.     The USAvian project aims to serve as a resource for multiple end-user types (e.g., bird enthusiasts, land managers, decision makers, educators) to understand and identify the components of the bird conservation landscape in the US. Primary information sources include conservation and management groups and their jurisdictions, conservation project extents, and species ranges and distributions. The visualization service and underlying integrated database comprises spatially and temporally referenced layers for management and conservation group jurisdictions, conservation efforts, and basic species natural histories.      This talk provides a high-level overview of the USAvian tool and its underlying data sources, highlighting potential applications, and  its role in identifying conservation gaps at various taxonomic and spatial scales.     This project relies heavily on the integration of many unstructured information sources, including verbally-derived institutional knowledge. However, integrating unstructured information into a usable format remains a challenge across and beyond ecology and ornithology. Incorporating the human dimensions information required to develop USAvian further exacerbates this issue. We also discuss the challenges associated with integrating the various data sources used in USAvian and highlight opportunities for the conservation and management communities to contribute to streamlined data integration efforts.    Data, source code and more info: www.github.com/trashbirdecology/usavian}

\normaltalk{Oxidative Damage Increases with both Degree of Simulated Bacterial Infection and Ectoparasite Load in Tree Swallow Nestlings}{Mike W Butler, Emma N Stierhoff, Julianna M Carpenetti, Alyssa M Addesso, Sarah A Knutie}{Mounting an immune response destroys pathogens, but this response comes at a physiological cost, including the production of oxidative damage. However, many investigations into the effects of immune challenges employ a single high dose of an immunostimulant, meaning the consequences of less concentrated doses, which may more effectively represent naturally occurring immune challenges, are poorly resolved. We used nestling tree swallows to test how degree of immunological challenge modifies oxidative physiology, circulating nutrient levels, and body mass, and how these metrics interact with ectoparasite load. We injected 14-day-old nestlings with either 0, 0.01, 0.1, or 1 mg lipopolysaccharide (LPS) per kg body mass, collected blood 24-h later for quantification of oxidative damage (i.e., d-ROMs) and circulating nutrient levels, recorded body mass, and collected nests for ectoparasite identification. We found that nestlings injected with 1.0 mg LPS/kg body mass, which is a common dosage in ecoimmunological studies, lost significantly more body mass than individuals in other treatment groups. Additionally, there was a dose-dependent relationship between concentration of LPS injection and d-ROMs, with each successively higher amount of LPS resulting in more oxidative damage. Lastly, nestlings experiencing greater ectoparasite loads had higher levels of oxidative damage, and also greater levels of circulating glycerol. Thus, while low-intensity immune challenges may not affect body mass maintenance, these challenges still result in detectable increases of oxidative damage, and are compounded by ectoparasite presence. Also, at least with respect to ectoparasites, nestlings may be mobilizing nutrients such as glycerol to fuel associated immunological responses.}

\normaltalk{Population Genomics Elucidates Connectivity of Roseate Terns in North American}{Paige A Byerly, R T Chesser, Robert C Fleischer, Paul L Leberg}{Anthropogenic disturbances such as population declines and habitat fragmentation are known alter dispersal patterns among subpopulations of wildlife, potentially leading to reproductive isolation that can contribute to reduced genetic diversity. While these effects are well-documented in terrestrial wildlife, barriers to dispersal of highly vagile taxa such as seabirds are less understood. Roseate Terns (Sterna dougallii) are a globally distributed seabird species, but populations tend to be patchy, and the species is declining across most of its range. In North America, the Roseate Tern subspecies S. d. dougallii is delineated into a federally endangered Northwestern Atlantic population and a threatened Caribbean population, both of which experienced major declines in the early 20th century. There is thought to be no gene flow between these two populations, an assumption based on geographic separation, morphological differences, and a lack of band returns; however, there may be potential for interbreeding, and the question of movement among these populations warrants further investigation. We investigated connectivity of Roseate Tern populations in North America to evaluate if historical population declines and habitat fragmentation have contributed to contemporary declines by inhibiting dispersal. We used SNP genomic markers generated via high-throughput sequencing to evaluate relatedness, population structuring, and population genomic parameters such as inbreeding among breeding populations of Roseate Terns in the Northeastern United States, Florida, and the Virgin Islands. Our results suggest that historical dispersal patterns may have been altered by anthropogenic activities and population declines, and that low genetic diversity may inhibit recovery of both populations.}

\normaltalk{Age-Related Differences in the Distribution of Migrating Landbirds on the Northern Coast of the Gulf of Mexico}{Liliana N Calderon, Theodore J Zenzal, Wylie C Barrow, Barry Wilson, Jeffrey J Buler}{Migrating landbirds may be constrained by factors extrinsic to habitat, such as demographic traits, when stopping over near large water bodies like the Gulf of Mexico (GOM). We investigated variability in the age of migrating landbirds in relation to the geographic location of stopover sites along the northern GOM (i.e., distance from the coast and longitude). We mist-netted migrants at six stopover sites in Alabama and Louisiana during spring and autumn 2018 and 2019 that range from 1 to 75 km from the coast. We found evidence of clockwise loop migration across seasons for younger birds using relatively narrower routes more likely driven by endogenous cues while older birds are distributed more broadly. We found higher proportions of young birds at coastal sites relative to inland sites during the spring, but not during the autumn season. Our study provides new insights into the stopover distributions of birds in close proximity to the GOM and we emphasize the conservation of such habitats where migrants stopover along the coast as they are threatened by habitat destruction and climate change.}

\normaltalk{Monk Parakeets as a Globally Naturalized Species}{Carlos E Calzada Preston}{Monk Parakeets (Myiopsitta monachus) are amongst the most widespread naturalized parrot species, with multiple factors influencing their successful establishment and spread. For this project, I have: 1) illustrated the species' trajectory through the invasion framework, 2) documented their current global distribution and 3) described the natural history of monk parakeets in both native and naturalized populations. Using the CITES, eBird and GAVIA databases, I tabulated the amount of monk parakeets imported into each country, the countries into which monk parakeets have successfully established, and their population growth trends. I identified that a significant factor behind monk parakeet establishment is the amount of birds imported into each country. From 1981 to 2016, 1,295,805 monk parakeets have been traded internationally, being imported by a total of 103 countries/territories and anywhere between 12 to 40 countries importing parakeets in any given year. The analyzed literature shows that monk parakeets have greater fledging success and breeding productivity within naturalized populations than in the native range. Monk parakeets are also unique among Psittacidae in that they construct stick nests rather than relying on pre-existing tree cavities, significantly increasing the potential nesting locations for the species. Finally, monk parakeet's tolerance of other species and use of variable food resources has been shown to aid their survival in novel locations.}

\normaltalk{Genomic Islands of Differentiation in a Rapid Avian Radiation are Driven by Recent Selective Sweeps}{Hussein A Hejase, Ayelet Salman, Leonardo Campagna, Melissa J Hubisz, Irby J Lovette, Ilan Gronau, Adam Siepel}{Numerous studies of emerging species have identified genomic ``islands" of elevated differentiation against a background of relative homogeneity.  The causes of these islands remain unclear, however, with some signs pointing toward ``speciation genes" that locally restrict gene flow and others suggesting selective sweeps that have occurred within nascent species during or after speciation. Here, we examine this question through the lens of recently obtained genome sequence data for five species of southern capuchino seedeaters, finch-like birds from South America that have undergone a species radiation during the Pleistocene.  By applying newly developed statistical methods for ancestral recombination graph inference and machine-learning methods for the prediction of selective sweeps, we show that the striking islands of differentiation in these birds appear to be generally associated with relatively recent, species-specific selective sweeps, most of which are predicted to be ``soft" sweeps acting on standing variants. Many of these sweeps coincide with genes associated with melanin-based variation in plumage, suggesting a prominent role for sexual selection. At the same time, we also identify signatures of possible selection against gene flow at a few loci. These observations shed new light on the complex manner in which natural selection shapes genome sequences during speciation.}

\normaltalk{Comparison of Heavy Metal Exposure in Resident and Migratory Raptors in Mexico}{Meagan L Campbell, Jaime R Von Osten, Enrique A Guti\'{e}rrez, Ernesto R Inzunza}{Increased habitat loss and contamination in the tropics cause ornithologists to suspect that birds encounter elevated contaminant exposure during migration, and while overwintering. Our research question is: Where are migratory raptors exposed to higher metal concentrations during their annual cycle? We hypothesized that sampling heavy metals in blood and feathers would help us identify the boreal or austral exposure. During the fall of 2016 and 2017, we sampled juvenile (hatch year) Sharp-shinned Hawk (Accipiter striatus), Cooper's Hawk (A. cooperii), Aplomado Falcon (Falco femoralis), Roadside Hawk (Rupornis magnirostris), and Short-tailed Hawk (Buteo brachyurus) in Veracruz, Mexico (a primary migratory corridor for raptors in the fall). We analyzed 194 blood and corresponding feather samples through voltammetry for zinc, cadmium, lead, mercury, copper, and aluminum. We compared mean values of metal concentrations and found that feather concentrations were 1-2 orders of magnitude higher than blood, and mean values of migratory and resident species had comparable metal levels. The only significant differences between resident and migratory species (wet weight, ppm) were: Hg in blood (resident $>$ migratory, P$<$0.01), Zn in feathers (migratory $>$ resident, P$<$0.01), and Pb in feathers (migratory $>$ resident, P$<$0.05). This study provides information about the temporality of heavy metal exposure pathways in raptors outside of their breeding areas in North America. Higher Zn and Pb levels in feathers of migratory species indicates elevated exposure in nesting areas, where the feather developed, whereas higher Hg levels in the blood of resident species alludes to higher exposure in Mexico.}

\normaltalk{Past, Current and Future Distributions of Puerto Rican Fauna: Implications from Climate Change}{Marconi Campos Cerqueira Junior, Noelia A Nieves-Col\'{o}n, Nashally A Folch-Mercado, Mitchell Aide}{Climate change is altering species distributions even in protected and remote areas, and although there has been a general shift toward higher altitudes and latitudes, the response of species can vary greatly. Here, we evaluate the past (1980-2000), current (1998-2018) and future (2040-2060) distribution of 28 bird species in Puerto Rico. The historical data set was based on GBIF data and the contemporary data set was based on acoustic monitoring. We created species distribution models using downscaled climate data for the three time periods. Distributions models and maps for each species were based on ~700 sampling sites. The results show an heterogenous response of the bird community.  There was little difference in species distribution between the historic and contemporary models, but the distributions of most species are reduced as Puerto Rico becomes dryer.  Five endemic and endangered bird species (Accipiter striatus, Antrostomus noctitherus, Amazona vittata, Buteo platypterus, and Setophaga angelae) had a very narrow distribution occurring only in few sites and regions suggesting that climate change will jeopardize the future of these species. Our results highlight species vulnerability to climate change and illustrate how acoustic monitoring provides an easy and powerful way to monitor animal populations.}

\normaltalk{Identification of Important Shorebird Sites in the Insular Caribbean}{Jessica Ca\~{n}izares, J Michael Reed}{North American migratory shorebirds have declined by an estimated 37- 60\% since the 1970s. For species such as these that depend on multiple sites throughout their life cycle, stopover and overwintering site quality and persistence influences successful migration and may drive larger population trends. Even though species spend a disproportionately small amount of their annual cycle in them, protecting important sites during migration is a conservation priority. The Caribbean provides key links in the Atlantic Flyway between the continents and support a number of migratory shorebirds. However, the current status of shorebird habitat in the Caribbean is poorly known. Human population growth and land development have resulted in losses of Caribbean wetlands and competition for limited space and resources is ever-present. Our goal was to identify new priority areas for shorebird conservation in the Caribbean, and to better understand the status, abundance, and distribution of shorebirds in the region. To do this, we analyzed eBird observations from the Caribbean over ten years (2010-2019). eBird serves as the repository not only for citizen science data but also for structured surveys such as the Caribbean Waterbird Census and International Shorebird Survey. We identified sites in Cuba and the Dominican Republic that qualify for special international designations (e.g., Western Hemisphere Shorebird Reserve Network Site, or Important Bird Area) based on both aggregations of shorebirds and species-specific population thresholds. In addition, we recommend further investigation at sites with high potential for conservation importance for shorebirds in Turks and Caicos Islands and Jamaica.}

\normaltalk{2010-2019: Ten Years of the Caribbean Waterbird Census}{Jessica Ca\~{n}izares, Jeff Gerbracht, Lisa Sorenson, J Michael Reed}{Wetlands in the Caribbean are critical habitat for resident bird species and migrants that rely on these unique ecosystems for stopover and overwintering sites. To fill knowledge gaps of wetland species in the Caribbean, BirdsCaribbean launched The Caribbean Waterbird Census  (CWC) in 2010, a region-wide waterbird and wetland monitoring program. The CWC has been instrumental in identifying wetland sites with high species diversity and abundance across the region. For example, CWC counts were used to determine the Cargill Salt Ponds in Bonaire  qualified as a Site of Regional Importance under the Western Hemisphere Shorebird Reserve Network (WHSRN); it was designated in 2018. The main criterion used for nomination was overall shorebird abundance ($>$20,000/yr; 15 species totaling 31,986 [95\% CI = 17,242-59,340]  in Jan-Feb 2015) in addition to $>$1\% of the flyway population for Short-billed Dowitcher (Limnodromus griseus griseus/hendersoni; 1,705 individuals), and species of conservation concern Red Knot (Calidris canutus rufa; 571) and Snowy Plover (Charadrius nivosus; 248)). Other sites of high species diversity and abundance include Humedal Sur de Pinar del R\'{i}o in Cuba (65 species) and the Monte Cristi wetlands in the Dominican Republic (63 species). CWC monitoring has also revealed significant threats at Monte Cristi, with hundreds of waterbird traps removed by local partners on multiple surveys. The CWC is an invaluable program for wetland and waterbird conservation in the region as it captures unique and important data about the status, distribution, and migration patterns of wetland species in the Caribbean, in addition to wetland status and threats.}

\normaltalk{An Assessment of the Determinants of Recruitment in Tree Swallow (Tachycineta Bicolor)}{Esther Carle-Pruneau, Dany Garant, Marc B\'{e}lisle}{Like most aerial insectivores, Tree Swallow (Tachycineta bicolor) populations are showing significant decline over the last decades in certain parts of their distribution range. Potential causes of this decline include the intensification of agricultural practices and increased use of pesticides. These practices should affect all aspects of population dynamics at several spatial scales. However, the basic processes underlying population dynamics, such as the determinants of recruitment, are still largely unknown in this species. Here we assessed the determinants of recruitment in Tree swallow using a study system consisting of 40 farms located along an agricultural intensification gradient that has been monitored since 2004 in southern Qu\'{e}bec, Canada.   More specifically, we analysed the patterns of natal dispersal distances and assessed if habitat features were related to local (native) recruitment in this species. Our results show that females dispersing at much greater distances than males. While dispersal distance of females was negatively related to conspecific density and type of agricultural crop, dispersal of males increased with the level of interspecific competition (density of House Sparrows, Passer domesticus). Also, males from young females will disperse further than males from older females. Despite these differences in dispersal distances, local recruits tended to settle and breed in habitats with characteristics similar to their natal habitats. Familiarity with natal habitat features thus seems to affect the settlement patterns of local individuals in our study system. Altogether our results suggest that habitat characteristics influence the dispersal behaviour of Tree Swallows and consequently their recruitment into the population.}

\normaltalk{Fragmentation and Precipitation Effect on the Diversity of Antwren Flocks in Panama Isthmus}{Michael D Casta\~{n}o Diaz, Juan L Parra, J P Kelley}{While many studies have examined the impact of single environmental drivers (such as precipitation or habitat fragmentation) on the structure of mixed species foraging aggregations (MSFAs), no studies have assessed their response to multiple, co-acting environmental pressures. We investigated spatial variation in richness, abundance, and density of antwren flocks, a type of MSFA common in the Neotropics. To understand precipitation and habitat fragmentation influence the structure of antwren MFSAs, we studied 72 antwren flocks across 19 forest plots along the Panama Canal ranging in size from 9.2 to $>$1000 hectares in size (and 1500-3100 mm of annual rainfall). We found a positive correlation between the fragment size and observed flock density (R2 = 0.73) and flock species abundance (R2 = 0.74).  We detected no flocks in forest fragments smaller than 15 hectares; in other plots, we found a total 1302 individuals from 107 associated species (the highest number recording for central Panama) with an average of 10.6 species and 18 individuals per flock. We found a complex, nonlinear interaction, between fragment size and precipitation for both estimated flock species richness and total number of individuals, and an additive impact of fragment size and precipitation for Shannon diversity. These models were highly explanatory (explained variance = 70-84\%). Our results emphasize the importance of considering suites of co-acting environmental factors when investigating the impacts of global environmental drivers on avian species interactions in the tropics.}

\normaltalk{Phylogenetic Relationships, Population Structure, and Species Limits Within the Sharp-Shinned Hawk Complex}{Therese A Catanach, Matthew R Halley, Samantha Palhano, Renata Biancalana, Jeff A Johnson, Russell Thorstrom, Julio Gallardo, Jason D Weckstein}{The Sharp-shinned Hawk (Accipiter striatus) complex is widely distributed across the New World, breeding from western Alaska to Argentina.  Although this complex is currently treated as a single species, based on size and plumage variation some authorities have suggested it is better treated as a species complex.  Using DNA sequences of the flanking regions of ultraconserved elements and full mitochondrial genomes from over 50 samples from across its geographic distribution we inferred a phylogeny for the Accipiter striatus complex.  In particular, we densely sampled across the Caribbean and the Neotropics, where 8 of the 10 recognized subspecies are breeding residents.  In addition to molecular data, we analyzed wing, tarsal, and tail measurements from approximately 200 Sharp-shinned hawk specimens from all taxa except the endangered Puerto Rican Sharp-shinned Hawk.  Using the resulting phylogeny in combination with information on plumage and morphological variation we then determined species limits within this clade.   Our findings confirm that Sharp-shinned Hawk is a species complex and contains several unique, geographically restricted lineages with low populations, and thus several are of conservation concern.}

\normaltalk{100 Years of the USGS Bird Banding Lab: Where are We Now and How are We Preparing for the Next Century of Bird Banding?}{Antonio Celis-Murillo}{The USGS Bird Banding Laboratory (BBL) is an integrated scientific program established in 1920. It has supported the collection, archiving, management and dissemination of information from banded and marked birds in North America. These data have been critical for studying biological patterns and processes, such as population demographics, trends, morphology and physiology and spatial behaviors of resident and migratory bird populations. The resulting science has been used to inform management and conservation practices. In this talk I intend to introduce the changes planned by the USGS Bird Banding Lab as we position ourselves to support and advance the science that depends on the capture, banding and/or tagging of birds over the next century. For instance, the BBL is redesigning its data base management system, improving the banding and encounter data submission process, enhancing the curation, archiving and data request process, creating an electronic permitting process, among other major changes. These updates are being made with the goal of developing best practices, while supporting an increased volume of banding data produced by our banding community, a wider array of banding projects, and adapt to many new emerging technologies used in conjunction with the capture and banding of birds.}

\normaltalk{Yellow-Breasted Chats from Different Subpopulations Respond with Greater Aggression to Local than Foreign Calls: A Playback Experiment}{Marie E Chappell, David N Bonter}{Birds use song to attract and recognize potential mates. In species like the Yellow-breasted Chat (Icteria virens), large ranges can lead to geographic song variation and potentially reproductive isolation. In an attempt to determine chats' recognition of geographic variation in song, and therefore the likelihood of speciation by reproductive isolation, we conducted playback experiments that tested the responses of territorial chats (N = 43 territories) in Kentucky and Arizona to local (same state) and foreign (different state) songs. When presented with playbacks of local and foreign chat vocalizations, as well as a non-chat control, focal birds approached the speaker more closely in response to the local song type (minimum mean distance ± standard error = 9.9 ± 2.0) than the foreign song type (16.2 ± 2.1; F = 10.97, P $<$ 0.001). Focal chats also made more visible flights in response to the local song type (1.44 ± 0.2) than the foreign song type (1.42 ± 0.2; F = 6.56, P = 0.002). However, they made a similar number of calls in response to both the local (3.9 ± 0.97) and foreign (4.2 ± 0.97) playback types (F = 2.79, P = 0.066). In all comparisons, chats responded more vigorously to chat songs than the control (House Wren song), indicating that chats do recognize the songs of their own species despite the geographic differences in song.}

\normaltalk{Feisty Females are Less Flashy and Fruitful: Female Aggression, Ornamentation, and Fitness in the Prothonotary Warbler}{Elsa B Chen, Lesley P Bulluck}{Aggressive conspecific interactions and their associated tradeoffs between survival and reproduction are a driving evolutionary force. In avian systems, aggressive males tend to provide less parental care and be more ornamented. The tradeoffs for females, as well as the signaling function of ornamentation in the context of aggression, are not well understood. By comparing female aggression with ornamentation and reproductive success, this study assessed the reproductive tradeoffs of female aggression in the Prothonotary Warbler (Protonotaria citrea). During incubation, we conducted 5-minute staged nest intruder trials using audio of female chips paired with a 3-D printed decoy. During the late nestling stage (6-10 days post-hatching), provisioning effort was quantified with videos. We found that aggressive females were the least ornamented but did not provision their nestlings differently than non-aggressive females. Additionally, moderately aggressive females had offspring with the highest mean body condition, while the most aggressive females had lower quality offspring and mates that provisioned less than the average. The costs of aggression may be offset by mate quality for moderately aggressive females; however, highly aggressive females, who tend to have mates that provision less often, do incur these costs. Our results support previous research that the tradeoffs of aggression in females may be less severe than those experienced by males, and that mate quality is an important interacting factor. Our findings also suggest that female ornamentation may play a role in both sexual and social selection by signaling individual quality to males and aggressiveness to females.}

\normaltalk{Performance Response of a Cold-Adapted Seabird to a Rapidly Warming Arctic}{Emily S Choy, Ryan O'Connor, Tony Gaston, Francois Vezina, Grant Gilchrist, Kyle H Elliott}{The impacts of climate change are occurring most rapidly in the polar regions. Thick-billed murres (Uria lomvia) are one of the most abundant seabirds in the Canadian Arctic with a circumpolar distribution, and are an indicator species for monitoring the effects of climate change. Recently, there have been declines in murre populations reported in many regions, hypothesized to be the result of changing oceanographic conditions. Climate change may have direct effects on murres and through increasing ambient temperatures, leading to hypothermia. During the breeding season, murres face the risk of mortality due to water loss from overheating and mosquito parasitism.  From June to August 2019, we used open flow respirometry to measure the thermal tolerance of murres during the breeding season on Coats Island, Hudson Bay, Nunavut. In our pilot study, a maximum upper limit of heat tolerance (when murres display signs of heat stress) of 38.8°C was recorded (n = 10). Body temperature, resting metabolic rate, and total evaporative water loss increased significantly with ambient temperature in murres.  To continue our research, we will examine data on murre mortality collected over the past 20 years to determine whether our predicted the period coincides with when mortality actually occurs. As a cold-adapted Arctic bird with one of the highest energetic costs of flight, the effects of increases in ambient temperature due to Arctic change may have severe consequences on murre survival.}

\normaltalk{Using Heart Rate as a Proxy of Metabolic Rate to Examine the Foraging Energetics of Arctic Seabirds in a Changing Climate}{Emily S Choy, Shannon Whelan, Jonathan Green, Grant Gilchrist, Kyle Elliott}{The impacts of climate change are occurring most rapidly in the polar regions. Thick-billed murres (Uria lomvia) are one of the most abundant Arctic seabirds with a circumpolar distribution, and are an indicator species for monitoring the effects of Arctic change. Recently, there have been declines in murre populations reported in many regions, hypothesized to be the result of changing oceanographic conditions, but mechanisms remain unclear. With the recent Atlantification of northern Hudson Bay, thick-billed murres have switched their diet from $>$50\% Arctic cod (Boreogadus saida)  to $>$50\% capelin (Mallotus villosus), a north Atlantic species. Thus, climate change is leading to indirecl effects, namely decreases in energy consumption as a result of prey shifts. To identify the indirect energetic effects of shifting prey bases, we examined heart rate as a proxy for metabolic rate in murres on Coats Island, Hudson Bay, Nunavut, Canada, and in black-legged kittiwakes (Rissa tridactyla) on Middleton Island, Alaska, USA. During our 2019 pilot study, the heart rate of murres (n = 7), and kittiwakes (n= 12) demonstrated high variability among different behaviours. We also used accelerometers, doubly-labelled water and thyroid hormone analyses to quantify foraging behaviours and energetic expenditures in 18 murres. As a cold-adapted Arctic bird with one of the highest energetic costs of flight of all vertebrates, the effects of prey changes and may have severe impacts on murre survival.}

\normaltalk{Flight Efficiency Explains Differences in Natal Dispersal Distances in Birds}{Santiago Claramunt}{The factors responsible for variation in dispersal distances among bird species have remained obscure. Previous comparative studies found incongruent results and little support for theoretical models. Here I re-examine factors that influence dispersal distances in birds by taking into account the cost of aerial locomotion assessed through wing morphology. First, I show that flight efficiency, as estimated by the aspect ratio or the lift-to-drag ratio, is a strong predictor of dispersal distances among resident species. Migratory species showed a complex pattern. Most migrants showed a trend of increasing dispersal distances with increasing flight efficiency similar to that shown by resident species but some migrants with relatively low aerodynamic efficiency showed greater dispersal distances than expected. Ecological, behavioral, and life history factors had a small or nil influence on dispersal distances with most of their influence, if any, likely mediated by adaptations for the use of space reflected in flight efficiency. I suggest that dispersal distances in birds are not determined by adaptative strategies for dispersal per se but predominantly influenced by the energetic cost of movement.}

\normaltalk{Snowy Owl Irruption Patterns and the Potential Impact of Climate Change}{Cade L Coldren}{A number of bird species breeding at high latitudes irrupt every few years, wintering in large numbers far south of their normal winter range.  Mechanisms causing these irruptions are generally believed to be related to disruptions in food supply, but much remains unknown of those relationships.  The Snowy Owl (Bubo scandiacus) is one such species, but the causes of their irruptions are still speculative.  In recent years, climate change is impacting the breeding grounds of Snowy Owls in multiple ways, but how that might influence their irruption patterns is unknown.  Using Christmas Bird Count (CBC) data, augmented with other sources, I looked at how their irruption patterns may have changed in the last 100 years, specifically whether irruption frequency, spatial extent, or magnitude are increasing.  I found that irruptions are occurring significantly more frequently in the eastern and central states of the US, but not in the west.  Spatial extent does not seem to be changing, both within a region and continent-wide.   Regarding magnitude, the peak number of owls seen on individual counts appears to be increasing in the central and western regions (P $<$ 0.05), but mean number of owls per hour of effort, mean number of owls per count, and the proportion of counts with owls do not appear to be changing in any significant manner.  Lastly, I speculate on the ways in which climate change-driven mechanisms, such as rain-on-snow events in autumn, may be driving the observed increase in frequency of Snowy Owl irruptions.}

\normaltalk{Survival of Radio-Tagged Captive-Reared Hispaniolan Parrots in a Protected Area Through and after Hurricane Georges in the Dominican Republic.}{Jaime A Collazo, Thomas H White, Francisco J Vilella, Simon Guerrero}{We report survival estimates of 25 captive-reared Hispaniolan Parrots (Amazona ventralis) released in Parque Nacional del Este, Dominican Republic. Parrots were released between 29 June and 16 September 1998, before hurricane Georges made landfall in 22 September. Unlike 1997, when we released 24 parrots, none of the 25 parrots released in 1998 prior to hurricane Georges suffered early post-release mortality (up to ten weeks). Two adjustments to pre-release protocols in 1998 may have contributed to differences between years (increased exercise and reduced blood sampling). One death was attributed to the direct impact of the hurricane. But survival rates dropped 23\% in the ensuing eight weeks. Severe and extensive tree defoliation led to increased mobility to compensate for scarce foraging resources. Prior to the hurricane, movements averaged 16.9 km as compared to 26.5 km post-hurricane. Our data point at the short-term vulnerability of frugivores in the aftermath of hurricanes, but also at the high degree of resiliency displayed among released birds to cope with such extreme weather events.}

\normaltalk{Post-Hurricane Assessment of Abundance and Diversity of Land Birds in Coastal Dry Forest at Jobos Bay Salinas, Puerto Rico}{Fred C Schaffner, Ivelisse Rodr\'{i}guez, Ian J P\'{e}rez, Mariangely Col\'{o}n}{On September 20, 2017, Hurricane Mar\'{i}a struck the island of Puerto Rico; just ten days after Hurricane Irma ravaged the Island's north coast. This storm had devastating effects on habitats throughout the island of Puerto Rico, including massive crown loss and defoliation of mangroves and coastal secondary dry forest at Jobos Bay National Estuarine Research Reserve (JBNERR). Through mist netting sampling of resident and migratory land birds in an interhabitat corridor connecting mangroves with coastal dry forest, and a second site a mesquite (Prosopis pallida)-dominated dry forest from 2017 to 2019 we present an assessment post hurricane recovery of land bird abundance and diversity at this site.  We found reduced total species abundance and diversity, a slow increase in small granivores such as Common Ground Doves (Columbina passerina) and Black-faced Grassquits (Tiaris bicolor); a nearly complete absence of hummingbirds, as well as a decrease in Northern Waterthrushes (Parkesia noveboracensis); but an apparent increase in birds of open habitats including Grey Kingbirds (Tyrannus dominicensis), Northern Mockingbirds (Mimus polyglottos), and Eurasian Collard Doves (Streptopelia decaocto).}

\normaltalk{Proximity to Colonial Avian Predator Impacts Nest Site Selection of Imperiled Tidal Marsh Sparrows}{Alexandra M Cook, Jonathan B Cohen, Alison R Kocek}{Nest site selection of highly specialized species may be impacted by interspecies interactions when high quality habitat is limited. In tidal saltmarshes, the highest quality nesting habitat often occurs in the highest elevation portions of marsh, as this area receives less frequent tidal flooding. We discovered a population of imperiled Saltmarsh Sparrows (Ammospiza caudacuta) and Seaside Sparrows (A. maritima), nesting within a colony of Herring Gulls (Larus argentatus) and Great Black-backed Gulls (L. marinus) on a New York island. We hypothesized that sparrows nested within this gull colony because it contained the highest quality nesting habitat, and that sparrows would alter their nest site selection to reflect the increased risk of predation by gulls. We modeled sparrow nest and nest site characteristics as a function of distance from the gull colony center. When near to the colony center, the probability of nest site selection by Saltmarsh Sparrows and Seaside Sparrows was positively correlated with vegetation height and proportion low elevation marsh grass. This relationship did not hold for nests placed far from the colony or at a nearby site. Nest height did not change with distance to the colony center, but did differ between sites. These results may indicate that optimal sparrow nest height and architecture mitigate flooding risk, but when in close proximity to predators, concealment of nests in tall vegetation may also reduce depredation risk. Understanding how saltmarsh nesting predators can impact tidal marsh sparrow nest site selection can aid in conservation planning for these imperiled species.}

\normaltalk{Unravelling the Drivers of Avian Diversification}{Christopher R Cooney}{Bird species exhibit a bewildering array of phenotypic diversity and for over a century the avian radiation has been at the forefront of many types of biodiversity research. Nowhere is this more the case than in the fields of macroecology and macroevolution, where the study of birds and their traits has played a central role in furthering our knowledge of the basic processes governing global-scale patterns of biodiversity. However, while much progress has been made, many fundamental questions still remain. In this talk, I will describe new insights from my own work into macro-scale patterns of avian diversity, including recent work identifying key drivers of ecological (bill shape) and sexual trait (plumage colouration and song) diversity across bird lineages. In addition, I will touch upon more recent, ongoing efforts to connect the evolution of such traits to patterns of speciation, which has the potential to shed new light on the eternal question of why some avian groups and branches of the tree of life are more diverse than others.}

\normaltalk{Phylogeography of Albertine Rift Birds, with Implications for Future Conservation}{Jacob C Cooper, Joshua I Engel, Charles Kahindo, Ben D Marks, John M Bates}{The Albertine Rift is a biodiversity hotspot, with montane forest habitat in the region presumably experiencing several periods of connectivity and isolation in relation to climate cycling and recent human activities. Each major mountain range has montane forests remaining at higher elevations, and some recent research has postulated that corridors of habitat and climate suitability between these forest patches will allow for the persistence of regionally endemic bird species as climate change continues. To ascertain the effects of climate change on distributions and genetic structure and diversity, we gathered DNA sequence data for 12 montane bird species sampled throughout the Albertine Rift. We hypothesized that there would be similar genetic structure across species, but report inconsistent phylogeographic patterns, with some Albertine Rift endemics showing large amounts of unstructured genetic diversity across geographic space, while others show clear regional diversification. We find that Mt. Kabobo has the most genetically distinct populations in the Albertine Rift which is consistent with it being an isolated peripheral region geographically. We also find differentiation between the Western Rift and the Eastern Rift with varying levels of structure for different species. Our data improve our understanding of connectivity across the Albertine Rift, and inform how climate change may affect genetic diversity within this biodiversity hotspot.}

\normaltalk{Not Just Fat: The Effect of Dietary Antioxidants on Fuel Deposition Rates of Passerines at a Stopover Site}{Clara Cooper-Mullin, Wales A Carter, Scott R McWilliams}{Burning fat as fuel is oxidatively challenging for migratory songbirds, and much work has focused on how a bird's antioxidant system can protect against damage to cells, organs, and tissues during flights and when recovering at a stopover site. However, since fats are also vulnerable to oxidative damage, antioxidants acquired from seasonally abundant fruits at a stopover site may serve to protect fats during fueling and, in turn, allow birds to refuel more efficiently. We conducted a field experiment at an offshore stopover site on Hermit Thrushes (Catharus guttatus) to test the hypothesis that availability of dietary antioxidants allows birds to refuel more quickly as compared with birds not provided dietary antioxidants. We used a 2 X 2 factorial experiment (ad libitum or maintenance food availability; dietary antioxidants or no antioxidants) and measured fuel deposition rate, change in non-enzymatic antioxidant capacity, and change in oxidative damage over 3-6 days of field captivity. Thrushes on a maintenance diet did not significantly change body composition during captivity, whereas individuals on an ad lib diet increased fuel stores during captivity (P $<$0.001). Hermit Thrushes fed an ad lib diet increased non-enzymatic antioxidant capacity, and individuals given antioxidants increased non-enzymatic antioxidant capacity more than those provided ad lib food alone (P = 0.001). Hermit Thrushes given an ad lib diet with antioxidants had increased fuel deposition rates compared to those not given antioxidants (P = 0.01), indicating that antioxidant availability in a bird's diet on stopover allows individuals to build fat stores more quickly.}

\normaltalk{Nocturnal Flight Behaviour of a Long-Distance Neotropical Migrant During the Post-Fledging Period}{Dominic A Cormier, Philip D Taylor, Jennifer M Brown}{Nocturnal flight behaviour in migratory songbirds is influenced by a variety of intrinsic and extrinsic factors including wind conditions. During the post-fledging period, the long-distance migrant Blackpoll Warbler engages in extensive regional movements in a series of nocturnal non-migratory flights. Using automated telemetry and environmental track annotation services, we assessed the relationship between weather conditions, time-of-year, and age of nocturnal flights of Blackpoll Warblers in the Gulf of Maine region during the post-fledging period. We hypothesized: Adults are less time-constrained than hatch-years, therefore will make more efficient use of winds aloft; all individuals become more energy-constrained as they approach migratory departure, thus wind support would increase as the season progressed; cloud cover impedes navigation, thus wind support aloft will increase with cloud cover; flight direction will influence wind support aloft, but the nature of this relationship is uncertain. Overall, adults favored more supportive wind conditions aloft than did hatch-years, and all individuals flew with more supportive wind conditions as the time of migratory departure approached. Direction of flight and cloud cover had little effect. However, during September, adults favored less supportive winds compared to hatch-years contrary to predictions, suggesting that time-constraint is not the only factor affecting wind support. Adults, with both better orientation abilities and previous knowledge of the local landscape relative to hatch-years, may be willing (or able) to fly in less favorable tailwinds because they are more goal-orientated in their movements and with a greater ability to compensate for wind drift to maintain a desired course.}

\normaltalk{Discriminating Non-Breeding Origin and Migration Chronology in a Long-Distance Migratory Shorebird Staging in the Canadian Prairies}{Jess Cosentino, Ann McKellar, Keith Hobson, Juan Navedo, David Newstead, Enver Ortiz, Christy Morrissey}{Approximately 40\% of Canada's shorebirds have been lost in the past half century however the causes for these declines are not well understood. Each spring an estimated 75,000 Sanderling (25\% of North American population) stopover at the inland saline Chaplin Lake, Saskatchewan on their way to Arctic breeding grounds. Transcontinental, long-distance migrants such as these are experiencing the steepest declines, however little is known about where these birds overwinter or their use of Chaplin Lake and the Midcontinental route. Using combined stable isotope and radio telemetry approaches, my objectives are to (i) identify wintering origin of the Chaplin Lake population and (ii) investigate corresponding northward migration schedules. Over 600 Sanderling primary covert feathers collected at Chaplin Lake between 2012-2019 have been assigned to 3 main general non-breeding biomes throughout the Americas through clustering of d 13C, d 15N and d2H isotopic criteria. We found 3 groups of birds are using Chaplin on spring migration (Cluster 1: 73\%, Cluster 2: 6.7\%, Cluster 3: 20.3\%) with variation among years. Comparison with isotope ratios of feathers obtained from known wintering areas: Chiloe Island (Chile), Paracas (Peru) and Padre Island (Texas, USA) suggests birds may winter across 3 distinct latitudes. We are also currently tracking 70+ nanotagged individuals tagged at these 3 locations across the Motus Wildlife Telemetry network, providing insight into individual migration behavioural strategies. The results of this study will aid in identifying potentially vulnerable populations and better characterizing the relationship between non-breeding origin, migration chronology, and use of the Midcontinental flyway.}

\normaltalk{Nonideal Nest Box Selection in a Tree Swallow Population Breeding in Farmlands}{Ève Courtois, Dany Garant, Fanie Pelletier, Marc B\'{e}lisle}{Birds are expected to select a nesting site using cues that should reflect, directly or not, the fitness outcome of the different site options. However, human-induced environmental changes can alter the relationship between habitat characteristics and their fitness consequences, leading to a maladaptive behaviour. The most severe case of such nonideal habitat selection is the ecological trap, which occurs when individuals prefer to settle in poor-quality habitats while better ones are available. We studied the adaptiveness of nest box selection in a Tree Swallow (Tachycineta bicolor) population breeding in a network of 400 nest boxes installed in a gradient of agricultural intensification in southern Qu\'{e}bec, Canada, over a 10-year period. We first examined the effects of multiple habitat characteristics on nest box preference to identify potential settlement cues. We then assessed the links between those cues and the reproductive performance of individuals that established early and late. For example, forest cover within 100m of nest boxes appeared a reliable cue for late breeders as individuals preferring to settle in poorly forested habitats experienced greater fledging success. However, late settlers also preferred to settle in habitats that showed a high density of conspecifics in the previous year while such habitats were associated with a lower fledging success. We thus not only identified determinants of habitat selection in a secondary-cavity nester, but also detected potential maladaptive cue use dependent on the timing of settlement. Our results stress that nest boxes used as conservation tools can lead to ecological traps if unwisely placed.}

\normaltalk{IBAs and KBAs in Canada:  Bridging the Gap for Birds}{Andrew R Couturier, Amanda Bichel, Dean Evans, Sandra Marquez}{Birds Canada leads the Important Bird and Biodiversity Areas (IBA) Program in Canada and maintains an online database and website to share details about each site. Nearly 600 IBAs were designated in the late 1990s as part of a comprehensive identification process; roughly half of these sites have stewardship groups working at them to conserve birds and habitats. Birds Canada is also part of a consortium of NGO partners -- namely, Wildlife Conservation Society of Canada and NatureServe Canada -- that is supporting the establishment of Key Biodiversity Areas (KBAs) in Canada. KBAs encompass all types of biodiversity, including birds. Indeed, the KBA approach is modelled heavily on the very successful and long-running IBA Program -- IBAs can also be KBAs if they meet the criteria. Birds Canada is leading the work to assess our existing suite of IBAs against KBA criteria, both at the global and sub-global levels. This presentation showcases progress to date on the cross-walk of IBAs to the KBA standard, shares lessons learned, and illustrates some of the challenges of the work, e.g., Canada is the second largest country in the world, with many remote and data-poor sites. Next steps are discussed with respect to integration of data on birds and other biodiversity, and also continuing the conservation work of IBAs under a new KBA umbrella.}

\normaltalk{From One to Many: Using Archived Banding Data to Scale Up Studies of Parulid Migration Phenology}{Kristen M Covino, Kyle G Horton, Sara R Morris}{Migratory birds may adapt to changing environments through phenological changes to migration timing. Studies focusing on changes in migration timing are critically important, however acquiring information that encompasses the necessary geographic range and occurring over a temporal span long enough to detect these changes poses challenges. Data housed at the North American Bird Banding Laboratory has the potential to empower research at the scales necessary to investigate phenological change in a variety of migratory birds. As a proof of concept, we analyzed nearly 150,000 records of Black-throated Blue Warblers (Setophaga caerulescens) from spring and fall migration spanning from 1966-2015. In this species, early spring migration occurred 1.1 days earlier per decade and peak spring migration occurred earlier by half a day per decade. Additionally, fall migration protracted by almost 2 days per decade, demonstrating the importance of investigating both migratory seasons in studies of phenological change. In order to gain a more complete understanding of the patterns of phenological change in North American migratory birds, we are expanding our study to the Parulidae family. We are analyzing approximately 5 million records of 24 wood warbler species during spring and fall migration from 1963 to 2017. Our study will investigate whether patterns across species and seasons are consistent and thus will greatly impact our understanding of migration timing shifts at relevant scales both geographically and temporally. Additionally, we will demonstrate the utility of the long-term dataset housed at the North American Bird Banding Laboratory.}

\normaltalk{Ecological Drivers and Evolutionary Consequences of Interspecific Territoriality in North American Songbirds}{Madeline C Cowen, Jonathan P Drury, Gregory F Grether}{Behavioral interference between species can impact the phenotypic evolution of the species involved, but the extent to which such behavior persists through evolutionary time is thought to depend in part on whether the behavior is adaptive. Interspecific territoriality is often considered a maladaptive form of interference that results from misdirected intraspecific aggression and, for most cases, is therefore transient through time. Other evidence, however, suggests that this behavior may be adaptive when two species compete for resources or mates. In this talk, I discuss the results of phylogenetic comparative work on North American perching birds that tests these hypotheses at different evolutionary timescales. Using Birds of North America, Breeding Bird Survey, eBird, and an extensive search of behavioral studies, we determined which species pairs engage or do not engage in interspecific territorial behavior and compiled relevant life history traits and occurrence data for these species. We found that 32\% of species are interspecifically territorial with at least one other species. Among sister species of perching birds, interspecific territoriality appears to be a maladaptive behavior in some species pairs, and adaptive in others. Territoriality between perching birds at a deeper timescale (not just sister species) appears to be adaptive and associated with resource competition and reproductive interference. The territorial signals associated with this behavior also vary by timescale: plumage for within-family species pairs and song at between-family species pairs. Together, these findings suggest that interspecific territoriality can emerge in different ecological contexts, but that many species maintain territorial aggression toward competitors.}

\normaltalk{Conspecific Brood Parasitism in Red-Breasted Mergansers: Patterns of Host Choice and Reproductive Success Throughout a Host's Lifetime}{Shawn R Craik, Rodger D Titman}{The addition of eggs to a nest by a conspecific parasite can reduce host fitness via lower hatching success. Despite this, it remains largely unknown whether CBP throughout the lifetime of a host is non-random and whether rates of CBP at an individual's nests affects fitness over its lifetime. We followed nesting by 15 female Red-breasted Mergansers (Mergus serrator) over their lifetime in a population where CBP is a well-developed component of the breeding system (~40\% of nests parasitized). We examined: 1) whether lifetime rates of CBP at a host's nests deviated from that expected by chance; 2) evidence of cues used by parasites in selecting a host nest; and 3) whether nesting success and hatching success over a host's lifetime were related to the proportion of seasons in which she incubated a parasitized clutch. Whereas some females incubated a parasitized clutch during the majority of years (up to 80\% of seasons), nests of other individuals either rarely or never received a foreign egg. Despite individual variability in rates of CBP, the observed distribution of nests parasitized during a host's lifetime did not deviate from that expected by chance. Rather, we found that lifetime rates of CBP were strongly linked to the date of nest initiation in each year regardless of host age and body condition. Initiation of a nest during the first few weeks of a season was just as likely to produce a parasitized nest as a non-parasitized nest, whereas a nest initiated later in the year almost never received foreign eggs. Individuals incubating parasitized clutches during multiple years did not suffer from lower nesting or hatching success over their lives relative to other females, although heavily parasitized clutches of ≥14 eggs were more likely to lose an egg during incubation. The generally low costs of CBP to hosts during the nesting stage helps explain why behaviours effective in defending against CBP (e.g., egg rejection) have not evolved in Red-breasted Mergansers.}

\normaltalk{Reproductive Success and Parental Care of Birds Colonizing a Tropic Restoration Forest}{George C Cummins, Eben H Paxton, Tad C Theimer}{Restoration is an important conservation tool for species threatened by habitat loss, but measuring the conservation benefit of restoration actions is often based on metrics such as presence/absence or species richness that may be misleading indicators of the actual quality of habitat. In Hawai$<$U+02BB$>$i, native koa (Acacia koa) is a native tree that is easy to propagate and grows quickly, and thus is a favored species for forest restoration projects across the islands. In this case study, we measured the quality of a koa restoration forest for birds by comparing the nesting success and nest attendance behavior of four native bird species and two introduced bird species breeding in former pasturelands replanted with koa relative to old-growth, mixed `ohi'a (Metrosideros polymorpha) -koa forests in Hakalau Forest National Wildlife Refuge. We found no difference in either nesting success or nest attendance behaviors for all species, with the notable exception of Hawai$<$U+02BB$>$i `Elepaio (Chasiempis sandwichensis). While this species had similar nest success across our study sites, it took shorter off-bouts and spent more time incubating in the koa restoration forest than in old-growth `ohi'a-koa forests. One explanation for this is arthropods were more accessible in the koa forest, allowing this aerial insectivore to spend less time foraging and more time incubating. In contrast with these results, densities of most forest birds are lower in the koa reforestation areas despite similar reproductive output, suggesting that the restoration habitat can provide quality habitat, but at lower densities.}

\normaltalk{A Holistic Approach to the Conservation of Critically Endangered Ridgway's Hawk (Buteo Ridgwayi) in Dominican Republic}{Marta G Curti, Thomas I Hayes, Russell Thorstrom, Christine Hayes}{The Ridgway's Hawk is one of three island-endemic buteos found worldwide and one of 11 diurnal raptors listed as critically endangered by the IUCN. Once found throughout Hispaniola, this species declined to less than 300 individuals isolated in Los Haitises National Park  in Dominican Republic. The Peregrine Fund began working with this species in 2000 and after two decades, their conservation efforts have reversed the decline of this species, and now the population is growing. Hands-on management of nests to prevent nestling mortality caused by Philornis spp. has led to significant increases in productivity. In 2011, we knew of only 37 pairs of hawks, which fledged 18 young. In 2019, we monitored 144 pairs which produced 126 fledglings. Translocating young hawks into areas where they have been extirpated has resulted in a new breeding population in Punta Cana. This burgeoning population now includes 19 pairs and a total of 49 wild-hatched young have fledged in this area where no hawks had been reported for more than 50 years.  We also engage in an extensive environmental education program that targets adults and children with the goal of finding culturally appropriate sustainable solutions to human/raptor conflicts. To date, more than 13,000 people have been reached directly through these efforts. We also focus on community development programs wherein we hire and train locals -- 20 of which are currently working directly in conservation efforts. These, coupled with successful partnerships with local entities, have proven to be successful strategies in the recovery of this species.}

\normaltalk{Cultural Disruption in Response to Severe Population Decline in an Iconic Costa Rican Parrot}{Christine R Dahlin, Dominich L Hellmich, Grace S Vidaurre, Molly Dupin, Kaitlyn Eckley, Timothy F Wright}{Populations of many species worldwide are declining or becoming fragmented in the face of human over-exploitation of the environment. While the impacts on population numbers are easier for biologists to assess, more subtle effects of human pressure may be difficult to acquire. Cultural traditions in animals must be passed down through social learning, and thus these traditions may be impacted when populations decline. We examined whether yellow-naped amazons, Amazona auropalliata, maintained vocal dialects after an 11 year time span. Our recent survey fell after a dramatic decrease in the parrot's population and a change in IUCN status to endangered. Our previous research on the yellow-naped amazon provided evidence that they maintained dialects over an 11-year time span in Costa Rica with only slight shifts in geographic boundaries and acoustic structure, during a time when the population was less fragmented.  This recent survey indicated greater shifts in acoustic structure in the most recent 11-year time span as compared to the previous period, as well as a higher incidence of bilingual sites and possible development of new call types. We hypothesize that population fragmentation may be contributing to these changes.}

\normaltalk{Effects of Disturbance and Stewardship on Least Tern Productivity in Coastal Mississippi}{Abigail Darrah}{Coastal Mississippi supports up to 2,500 breeding pairs of Least Terns (Sternula antillarum) along a mere 26 miles of man-made beach, making this region a critical area for this declining species. These beaches are heavily used by the public during the summer, and a stewardship program that includes symbolic fencing and active stewards has been put in place in attempt to protect Least Tern colonies from disturbance. I assessed the effectiveness of these efforts as a conservation measure by determining if productivity was correlated with extent of disturbance or stewardship. Observers monitored nest success and number of fledglings produced at Least Tern colonies in Harrison County, Mississippi, during the breeding seasons of 2017-2019. Concurrently, stewards recorded all sources and durations of disturbance events observed during their work shifts and recorded the proportion of the colony that reacted by flushing or engaging in defense. Least Tern nest survival probability was positively related to colony size and stewardship hours, and was negatively related to weekends and evidence of owl presence. Total productivity was also positively related to extent of stewardship during the incubation stage, whereas productivity was negatively related to extent of stewardship and frequency of avian predator disturbance during the chick stage. The results of this study demonstrate that active stewardship can have a positive effect on Least Tern productivity by increasing nest success, whereas current stewardship practices may not be sufficient to increase chick survival, even in a system with a paucity of ground predators such as coastal Mississippi.}

\normaltalk{Diet-Induced Changes to Host Gut Microbiota is Linked to Foraging Innovation in a Wild Bird}{Gabrielle L Davidson, Niamh Wiley, Amy C Cooke, Crystal N Johnson, Fiona Fouhy, Michael S Reichert, Ivan de la Hera, Jodie M Crane, Ipek G Kulahci, R Paul Ross, Catherine Stanton, John L Quinn}{The microbial community in the gut is influenced by environmental factors, especially diet, which can moderate host behaviour through the microbiome-gut-brain axis. However, the ecological relevance of microbiome-mediated behavioural plasticity in wild animals is largely unexplored. We presented wild-caught great tits (Parus major) with a problem-solving task and showed that performance was weakly associated with natural variation in the gut microbiome. We then manipulated the gut microbiome by feeding birds one of two diets that differed in their relative levels of fat, protein and fibre content: an insect only diet (low content), or a seed, nut and beef suet diet (high content). We presented these same individuals with the same problem solving task after the dietary manipulation to test whether any observed dietary-induced gut microbiome alterations predicted their subsequent problem solving performance. Microbial communities changed substantially and were less diverse among individuals given the insect but not the seed diet. Individuals were less likely to problem-solve after being given the insect diet, and problem-solving performance was positively associated with microbiome diversity. The same microbiota metrics that were altered as a consequence of diet were also those that correlated with variation in problem solving performance. Although the effect on behaviour may not be solely or directly a consequence of the gut microbiota manipulation, this nevertheless represents the first evidence for an association between behaviour, diet and the gut microbiome in a wild animal system, and points to an overlooked but potentially crucial mechanism underlying individual behavioural plasticity in evolutionary ecology.}

\normaltalk{Measuring Direct and Indirect Effects of Conventional Oil Infrastructure on Grassland Bird Abundance in Alberta}{Jasmin P Dawson, D R Norris, Ellen H Esch, Patricia Rosa, Nicola Koper}{Grassland-obligate bird species have declined rapidly since the 1970s, which is largely attributed to habitat degradation, such as natural resource extraction, increasing non-native vegetation communities, and livestock grazing. Majority of the remaining grasslands in Alberta exist on public lands that are leased for oil extraction and cattle ranching. Previous studies have demonstrated that the presence of conventional oil infrastructure decreases grassland bird abundance, yet the mechanisms behind this relationship remain uncertain. This study aims to examine whether grassland bird abundance is influenced by the presence of conventional oil infrastructure and whether cattle grazing mediates the effects of oil infrastructure on grassland birds via neophilia. We hypothesize that (1) conventional oil infrastructure influences grassland bird abundance through neophobia, and (2) the effects of conventional oil infrastructure are mediated through variation in cattle grazing via neophilia. In order to understand how the presence of conventional oil infrastructure and cattle grazing influence grassland bird abundance, we assessed grassland bird abundance, presence of conventional oil infrastructure, cattle fecal pat abundance, and vegetation characteristic data. Our structural equation models show that the presence of infrastructure resulted in ecological traps for wetland, cavity, short-grass vegetation, and arboreal nesting birds. This study reinforces previous findings that oil infrastructure has adverse effects on this at-risk bird population. To ensure the future of grassland birds, it is imperative to restore the landscape by implementing conservation policies to limit further infrastructure development and continue restoration efforts in grasslands.}

\normaltalk{Exploring the Genomics of Migration Timing in a Long-Distance Migratory Songbird, the Purple Martin}{Evelien de Greef, Kira E Delmore, Wesley Brashear, Alexander Suh, Kevin C Fraser}{Optimal timing is essential for migratory birds to keep pace with seasonal resources. To anticipate birds' future capabilities to adapt to changing climates, it is fundamental to better understand the mechanisms driving or constraining migratory timing. Numerous studies have used a candidate gene approach to study migration genetics, which is limited to small portions of the genome and have provided highly variable results. By integrating tracking data with genomics, we used an unbiased approach to examine known and de novo genes that may underlying migration timing in long-distance migratory songbird, the Purple Martin (Progne subis). Data from geolocator-tracked birds displayed a wide range of breeding arrival phenotypes (Jan 29 -- Jun 9), capturing high variability in migration timing across their North American breeding range. We assembled and annotated a draft reference genome for this species and obtained whole-genome sequencing data for 96 individuals. We assessed population structure and differentiation in four populations, demographic history in three subspecies, and conducted a genome-wide association study to identify regions underlying the migratory phenotypes. Our results shed new light on our understanding of the Purple Martin, the population structure of songbird populations in North America and genetic basis of migratory timing in this group of organisms.}

\normaltalk{Influences of Year-Round Weather Events on the Breeding Phenology and Success of Migratory Alpine Horned Larks}{Devin R de Zwaan, Anna Drake, Kathy Martin}{In alpine habitats, early-season conditions are unpredictable and short breeding seasons limit reproductive opportunities, such that arriving and breeding earlier or later than the optimum may be particularly costly. Given early-season energy limitations, the influence of environmental conditions across the annual cycle on breeding phenology may have pronounced fitness consequences. For alpine Horned Lark (Eremophila alpestris) in northern B.C., Canada (54.8°), we used 12 years of breeding data and 3 years of migration data from light-level geolocators to link breeding phenology to environmental conditions across the annual cycle and identify critical periods in time or space that influence reproduction. Average clutch initiation varied up to 11 days among years but did not advance from 2003 to 2019. Colder, wetter conditions at wintering habitat, and colder temperatures upon arrival at the breeding site delayed clutch initiation, independent of arrival time. Extreme cold at an extended staging area (mean = 41 days) just prior to arrival at the breeding site carried over to prolong offspring development rate, potentially by influencing parental investment. Nest survival decreased with both later clutch initiation and prolonged offspring development, such that females that nested earlier and fledged offspring quicker were up to 40\% more likely to reproduce successfully. We highlight the potential importance of staging areas for songbirds breeding in highly seasonal environments and discuss implications of individual breeding decisions for predicting population-level responses to climate change across mountain habitats.}

\normaltalk{Tracking Malaria Parasite Infection in Birds Migrating Through Coastal Texas over a 6-Year Period}{Spencer C DeBrock, Peter P Marra, Emily B Cohen, Sarah A Hamer}{Avian migration can significantly impact disease dynamics in a population, since it is energetically expensive to simultaneously migrate and mount an immune response. It further provides a mechanism for large-scale geographic parasite dispersal; for example, birds can be infected by diverse species of Haemosporidia- vector-borne blood parasites that may cause avian malaria, impacting host fitness. Our objective was to compare parasite diversity and load in birds migrating into North America in order to reveal patterns useful in source-tracking infections and understanding host-parasite interactions that impact health. At our field site in coastal Texas, we used mist nets to trap migrating birds in the Springs of 2014-2020. Birds were banded, measured, screened for ectoparasites, and released after a small blood sample was taken. Using PCR and sequencing of the Haemosporidian cytB gene on a subset of collected blood samples, we found an overall infection prevalence of 48\% (n=198/414), with no associations between infection status and muscle score, fat score, or foraging guild. We used the MalAvi database of avian Haemosporidian parasites to determine the lineages of blood parasites present in our sampled birds. Among a subset of the samples, at least 13 lineages are present, including seven which have not been described in North American birds before. Future work will involve ectoparasite testing for Rickettsia spp. and Borrelia burgdorferi, as well as tracking focal species with innovative nanotags to assess potential parasite dispersal. This work may help define host-parasite relationships in migratory birds, including the influence of co-infection on migratory behavior.}

\normaltalk{How Removing a "soulless" Invasive Shrub Affects the Community Structure of Songbirds}{Leanna DeJong, Elizabeth M Toman, Stephen Matthews}{Amur honeysuckle (Lonicera maackii) is a highly invasive shrub that threatens ecosystems throughout the Eastern USA. Regarding birds, the shrub represents an ecological trap and affects avian species composition. Due to its detrimental effects on ecosystems, managers have invested substantial effort towards removing the shrub. Although many have explored the impact that Amur honeysuckle has on birds, few have examined how birds might be impacted by its removal. We investigated how the community composition of avian species in rural riparian forests is impacted by the removal of Amur honeysuckle. To accomplish this, we identified plots within rural riparian forest in southwestern Ohio that are invaded by or removed of the shrub. At these plots, during the 2019 peak breeding season, we performed avian point counts and collected vegetation data to capture differences in the avian community and habitat respectively among plots. We found that the avian community differed between plots invaded by vs. removed of Amur honeysuckle, with species overall more abundant in plots removed of the shrub. While removed plots had higher abundances of woodpeckers, species that prefer open woodland habitat, and the Acadian flycatcher, the abundance of other species, e.g., Northern cardinals, did not differ between plot types. A better understanding of how Amur honeysuckle removal impacts birds will help optimize management strategies and mitigate impacts to avian species.}

\normaltalk{Mito-Nuclear Incompatibilities Between Rhegmatorhina Hoffmannsi and Berlepschi Across an Amazonian River}{Glaucia Del-Rio, Marco Rego, Bret Whitney, Fabio Schunck, Silveira Lu\'{i}s, Brant Faircloth, Robb Brumfield}{Investigating genomes of birds in hybrids zones provides a good opportunity to assess the reproductive isolating mechanisms responsible for speciation. Rhegmatorhina hoffmannsi and berlepschi are sister species that have divergent mitochondrial haplotypes across the Aripuanã River, but the centers of the plumage and nuclear transitions are located ~180 km to the north of the mitochondrial break. Because the mitochondrial haplotypes do not cross the river barrier, we hypothesize that cytonuclear incompatibilities could be responsible. To test this hypothesis, we sequenced a reference whole-genome (N50=10.8Mb) for hoffmannsi, and resequenced nuclear and mitochondrial genomes (~11X) for 67 individuals spanning the hybrid zone. We compared genomes using Fst outlier analysis, and conducted haplotype analysis for candidate genes. We found evidence for mito-nuclear incompatibilities between populations inhabiting different banks of the Aripuanã River. However, according to haplotype frequencies, males that are heterozygous for the nuclear incompatibility genes, but which have berlepschi mitochondria, are still viable and/or fertile. We suggest that these males are enabling the asymmetric movement of hoffmannsi nuclear genes over berlepschi populations on the right bank of the Aripuanã River. We also found that genes responsible for differences in plumage color between the two species are not physically linked to the genes responsible for the mito-nuclear incompatibilities. Such lack of linkage between speciation genes and plumage color genes suggest that the hoffmannsi plumage color alleles will keep moving northward, which could eventually result in extinction of the berlepschi phenotype.}

\normaltalk{Integrating Abundance and Tracking Data to Identify Important Places for a Boreal-Breeding Migratory Songbird}{William V DeLuca, Timothy D Meehan, Sarah P Saunders, Nicole Micheal, Jill L Deppe}{Conservation planning for animals that migrate at hemispheric scales is notoriously challenging. The advancement of eBird-derived abundance estimates throughout the annual cycle provides an exciting step toward identifying critical locations for conservation. Additionally, advances in tracking technology for small migratory birds have resulted in a wealth of data that can contribute unique information to guide conservation decisions. We used light-level geolocation data from ten breeding sites across the boreal forest for $>$50 blackpoll warblers (Setophaga striata) and eBird Status and Trends weekly abundance data to identify key locations for the species during pre and post breeding migration. We used a novel approach to explore eBird abundance data during the migratory periods by applying a temporally dynamic least cost paths analysis to identify spatial concentrations during migration. We then integrated individual-level tracking data and used the integrated spatial surface to identify locations of conservation importance and compare them to locations identified using only the weekly eBird abundance data. The integration of dynamic abundance data with individually tracked migratory birds offers an exciting new approach that can ultimately be applied to multiple species to develop a comprehensive, hemispheric conservation design.}

\normaltalk{Effects of Urbanization and Landscape on Wild Avian Gut Microbiomes}{Elizabeth Derryberry, Mae Berlow}{Urbanization is rapidly transforming habitats around the globe. In the process, many bird species have been extirpated and those that persist are under extreme selection pressures. Urban and rural habitats are often different in a number of factors important to native organisms, including food availability, environmental stressors, and prevalence of disease, which can lead to divergence in phenotypic traits, including behavioral, physiological and morphological features. Recent research highlights that urbanization is also changing the gut microbial communities found in a diverse group of hosts, including birds. Notably, the effects of urbanization on gut microbiomes have not been uniform -- some studies find higher microbial diversity and others lower diversity in urban hosts -- leaving uncertainty as to how urban habitats are shaping gut microbial communities. This is a clear need to understand these effects, because changes in the gut microbial community can affect an animal's development, nutrient absorption, and pathogen defense, among many other traits likely important to the host persisting in urban environments. To better understand these effects, we investigated the gut bacterial communities of white-crowned sparrow populations along an urbanization gradient in the San Francisco Bay area. We asked how gut bacterial communities vary with the local environment and host morphological characteristics. We found direct effects of environmental factors, including urban noise levels and territory land cover, as well as indirect effects through body size and condition, on alpha and beta diversity of gut microbial communities. Elucidating these effects provides a better understanding of how urbanization affects wild avian physiology.}

\normaltalk{Using Historical Demography to Forecast Climate Vulnerability in Migratory Birds}{Matt DeSaix, Christen Bossu, H L Gibbs, Peter Marra, Thomas Sherry, Thomas Smith, Michael Webster, Kristen Ruegg}{In the past several decades, North American breeding avifauna declined by approximately 3 billion individuals.  In order for these declining species to persist under ongoing environmental change, they must be able to shift their distribution and/or adapt to changing selection pressures.  If range shifts are not possible, adaptation may be a species' only option for persistence under future climate change.  A species' current adaptive genetic variation is a product of historical processes, as well as contemporary gene flow and migratory patterns.  The objective of this study is to determine how historical demography can shape a migratory species' ability to adapt to climate change.  The focal species of this study is the American Redstart (Setophaga ruticilla), a neotropical migratory songbird with large breeding and nonbreeding ranges across a variety of habitats.  Specifically, the steps we will take to achieve these goals include: 1) identifying neutral population genetic structure; 2) describe ecotypes and the adaptive genetic variation associated with these groups; 3) hindcast species distribution models to Last Glacial Maximum (LGM) to infer historical separation; and 4) test explicit post-LGM expansion models with population genetic inference.  Understanding how historical processes have shaped current population structure and adaptive variation will be crucial for expanding this framework to the full annual cycle of migratory birds.}

\normaltalk{Nestlings Beg with Higher Voices in Noise}{Maheshi E Dharmasiri, Colleen A Barber, Andrew G Horn}{Increased anthropogenic noise impacts avian species by inhibiting transmission of signals to receivers. Males of many species adapt by increasing the minimum frequency of their songs above the low frequencies of traffic noise. We examined minimum frequency and frequency range of nestling begging calls in European starlings (Sturnus vulgaris) exposed to a playback loop of loud, constant traffic noise (experimental) for 10 days and compared them to controls exposed only to ambient traffic noise. We predicted that experimental nestlings would increase the minimum frequency of their calls, resulting in a narrower range of call frequencies compared to controls. We analysed begging calls on day 14 of the nestling period, on the 10th day of playback. 5\% frequency was significantly higher in experimental than control broods and did not differ between controls and post-experimental trials (right after playback ended). We found no difference in 90\% bandwidth of begging calls between control and experimental broods. These results illustrate that nestling European starlings are already well-adapted urban exploiters who adjust their begging calls in response to anthropogenic noise.}

\normaltalk{The Effects of Nest Parasitism on Incubation Behavior and Embryonic Development}{Alexander J Di Giovanni, Michael P Ward}{During the incubation period, most avian embryos are reliant on their parents for development. In parasitic species like the Brown-headed Cowbird (Molothrus ater; hereafter ``cowbird"), this developmental burden is transferred to the parents of the host species. Cowbirds parasitize many host species, thus their embryos must be able to develop within a wide range of incubation strategies utilized by their host parents. To investigate the development of cowbird embryos, we examined two aspects of the incubation stage: 1) the variability of cowbird embryonic heart rate within different host species, and 2) how incubation behaviors by hosts differed with parasitism. Embryonic heart rate was measured using a non-invasive ballistocardiogram on eggs every three days. Incubation behaviors were examined using iButton temperature loggers to record changes in nest temperature. We found no significant effects of host species on cowbird embryonic heart rate; cowbird heart rates were similar despite being incubated by different host species. Host species incubation behaviors did not differ with parasitism status in three of the four species we investigated. Only Indigo Buntings (Passerina cyanea) differed, with greater numbers of bouts off the nest per day, as well as a lower percentage of time of the day spent incubating when cowbird eggs were present. Cowbird embryos appear able to develop, regardless of the incubation behavior of their host parents, however some species may alter their incubation when parasitized. In the 2020 field season we will conduct an experiment to further understand how hosts may change their incubation behavior when parasitized.}

\normaltalk{Spatial Distribution Assessment, Conservation and Management Recommendations for the West Indian Whistling Duck in Puerto Rico}{Sheylda N D\'{i}az-M\'{e}ndez, Jacob C Cooper, None None}{The development of predictive habitat distribution models has rapidly increased in ecology. Habitat use in amalgamation with these models can be used for management decisions in conservation and management of a species. Puerto Rico provides suitable habitat for large avian populations including the West Indian Whistling-Duck (Dendrocygna arborea). This work defines habitat use using spatial distribution models to determine the factors that influence distribution of this avian species along different zones on the island. The objectives were to 1) relate species presence to habitat characteristics, 2) provide an occupancy model to guide management decisions and recommendations in Puerto Rico Using presence/absence data from eBird, we analyzed the correlation in landcover variables and produced an occupancy model for the West Indian Whistling Duck. Results show cropland mosaics to be important habitat for this species, as well as woody savannahs and wetlands. Based on these results, decisions for conservation and management recommendations were produced including working closely with farmers in Puerto Rico and monitoring the species year-round. An important part of the recommendations includes an aggressive education campaign to optimize West Indian Whistling Duck and habitat conservation at a national level.}

\normaltalk{Migration Patterns of Wild vs Captive-Bred Asian Houbara Bustards, Chlamydotis Macqueenii}{Luisa M Diele-Viegas, Yves Hingrat, Justin Calabrese, William F Fagan}{Captive breeding has been increasingly used in conservation management to re-establish threatened species. However, for migratory species, the success of this strategy depends on whether the individuals can survive the post-release period, migrate successfully, overwinter, and survive to breeding age. Migratory performance relies on a combination of social cues, physiology, and genetics, and may be shaped by the period a captive-bred animal is maintained in captivity. Thus, the successful recovery of migratory species may be challenging due to the mortality risk faced by all first-time migrants and by a potential erosion of migratory behaviour over multiple generations of captive breeding. Understanding the similarities and differences in the migration ecology of wild and captive-bred individuals is thus crucial to evaluate the success of conservation efforts. The Asian Houbara Bustard, Chlamydotis macqueenii (Otididae) is a ground-dwelling migrant species inhabiting desert and steppe areas from the Middle East to Mongolia and China, and is considered vulnerable due to over-hunting, over-grazing, and poaching. The International Fund for Houbara Conservation (IFHC) has employed a variety of conservation actions to support this species, among which is a robust captive breeding and release program. Since 1995, over 1700 bustards (762 wild and 973 captive-bred) have been tracked through GPS-satellite telemetry, generating an unprecedented dataset on the movement of these migratory birds. Based on this dataset, we describe and compare the migration ecology of wild and captive-bred Asian Houbara Bustards, characterizing their flight paths, stopover behavior, and destinations.}

\normaltalk{No Evidence for the Lombard Effect: Black-Collared Starlings Sing at Higher Frequency but not Higher Amplitude in Urban Noise.}{Caroline Dingle, Ka Kiu Wong}{Low frequency traffic noise in cities potentially masks animal signals, interrupting important fitness-related functions including mate choice and territory defense. Birds commonly sing at higher frequencies to avoid such masking. However, birds are also predicted to sing at higher amplitudes in response to increasing background noise. It remains an open question whether observed frequency shifts are an adaptation to avoid signal masking or a passive by-product of singing at higher amplitude. Few studies have tested both amplitude and frequency of wild bird song in the field due to the difficulties of accurately measuring amplitude in the field. We studied the impact of urban noise on songs of Black-collared Starlings (Gracupica nigricollis), a bird which sings from open perches facilitating amplitude measurements. We measured the amplitude and frequency of starling songs across a noise gradient in a dense urban area. We found that starlings increased the minimum frequency but not the amplitude of their songs as the amplitude of background noise levels increased. Contrary to our predictions, we found an inverse relationship between song amplitude and song frequency, with birds singing at lower frequencies as they produced louder vocalizations. This study provides evidence that birds can alter both the volume and frequency of songs independently. Shifts in song frequency may therefore be a direct adaptive response to noise and are not necessarily a simple by-product of increased song amplitude.}

\normaltalk{Do Weekly Cycles of Human Disturbance Affect the Behavior of an Urban Adapter Bird?}{Pedro Diniz, Edvaldo F Silva-Jr, Regina H Macedo}{The weekend effects hypothesis predicts that weekly cycles of human disturbance impact animals´ behavior and physiology. This hypothesis is supported in the context of recreational activities in natural environments, but it is unknown whether it applies to urban animals. We tested this hypothesis by comparing several time-budget and vocal behaviors between weekdays and weekends (and holidays) in an urban adapter duetting bird (Rufous Hornero, Furnarius rufus) living in a university campus in central Brazil. Human disturbance (traffic and pedestrian flows, noise) largely increased during weekdays. We expected that the birds would spend more time in territorial vigilance and less time foraging, and would sing for longer and at a higher pitch on weekdays than on weekends to deal with periodic variation in human disturbance. As expected, birds spent more time in sentinel behavior and less time foraging on weekdays than on weekends, but these results were not statistically significant (except for a weak effect on male foraging effort). In addition, birds were equally likely to sing solos, start duets and answer partner-initiated duets on weekends and weekdays; and did not alter phrase duration in duets with human disturbance. Unexpectedly, birds emitted slightly higher-pitched duets on weekends than on weekdays. Our results provide little support for the weekend effects hypothesis, suggesting that urban adapters might be habituated or indifferent to periodic variation in human disturbance.}

\normaltalk{Transgenerational Effects of Experimentally Increased Maternal Corticosterone on Parental Behavior of Male House Wrens}{Rachael A DiSciullo, Charles F Thompson, Scott K Sakaluk}{In animals such as birds where embryonic development occurs outside of the body of the mother, prenatal effects on offspring can occur in response to the mother's physiological state both before the egg is laid and after. To what extent these maternal effects have a lasting impact beyond nestling development is little known. To address this, we simulated the transient increase in maternal corticosterone that occurs as a result of prenatal maternal stress by orally dosing female house wrens (Troglodytes aedon) with corticosterone-injected mealworms prior to and during egg-laying. In the following breeding season, we captured and identified the now-adult offspring of the corticosterone-treated mothers that recruited as breeders to the study site. To evaluate how prenatal exposure to maternally derived corticosterone affects adult male phenotype, we simulated conspecific territory intrusions and recorded provisioning effort of these recruited sons. We found that males whose mothers had consumed corticosterone-injected mealworms were both more likely and quicker to attack a simulated territory intruder compared with males of control mothers. There was no effect, however, of experimentally increased prenatal maternal corticosterone on adult male nestling provisioning. This study demonstrates that, indeed, a mother's exposure to corticosterone while eggs develop has transgenerational effects on some but not all aspects of the phenotype of her offspring, and that these effects may often be advantageous rather than detrimental.}

\normaltalk{Intra- And Interspecific Territoriality of the Collared Forest-Falcon in the Tropical Dry Forest of Western Mexico}{Abel Dom\'{i}nguez-Pompa, Marisela Mart\'{i}nez-Ruiz, Katherine Renton, Reyna A Castillo-G\'{a}mez}{English  Understanding the dynamics of intra- and interspecific interactions enables us to determine the factors influencing the establishment and spatial distribution of territories for threatened Neotropical raptors that are top predators in ecosystems. We evaluated intra and interspecific territoriality of the Collared Forest-falcon (Micrastur semitorquatus) in the Chamela-Cuixmala Biosphere Reserve, Mexico. We established 1 km length transects to survey Collared Forest-falcons, and locate territories. We then conducted playbacks of pre-recorded vocalizations of conspecifics and of sympatric diurnal raptors with similar ecological requirements: the Great Black-Hawk, and the Crane Hawk. We measured the intensity, latency and duration of territorial response of Collared Forest-falcons to playbacks of conspecifics and sympatric raptors. Collared Forest-falcons showed a stronger intensity of territorial response to playback of conspecifics compared to interspecific territorial response. However, the latency and duration of territorial response was similar for conspecifics and for interspecific response to playback of the Great Black-Hawk, while Collared Forest-falcons showed a low territorial response to the Crane Hawk. These results demonstrate that intraspecific competition is the main driver regulating the distribution of Collared Forest-falcon territories. However, Collared Forest-falcons also experience a high degree of competition with the Great Black Hawk that has similar ecological requirements (hunting strategy, diet, habitat use, and body size). By comparison, the specialized hunting strategies of the Crane Hawk that seeks out prey in cavities may reduce interspecific competition with this species. Therefore, a combination of intra- and interspecific competition may regulate territories of top predator raptors in tropical forests.    Espa\~{n}ol  Comprender las din\'{a}micas de interacci\'{o}n intra- e interespec\'{i}ficas permite conocer los factores que influyen en el establecimiento y distribuci\'{o}n de los territorios en rapaces neotropicales, las cuales son consideradas depredadores tope en ecosistemas y se encuentran altamente amenazadas. Evaluamos la territorialidad intra e interespec\'{i}fica del Halc\'{o}n Selv\'{a}tico de Collar en la Reserva de la Biosfera Chamela-Cuixmala, M\'{e}xico. Establecimos transectos de 1 km de longitud para muestrear y localizar territorios de los halcones. Realizamos provocaciones auditivas utilizando vocalizaciones pre-grabadas de conespec\'{i}ficos y rapaces diurnas simp\'{a}tricas con requerimientos ecol\'{o}gicos similares: Aguililla Negra Mayor y Gavil\'{a}n Zanc\'{o}n. Medimos la intensidad, duraci\'{o}n y latencia de respuesta territorial de los halcones a las provocaciones de conespec\'{i}ficos y rapaces simp\'{a}tricas. Los halcones mostraron mayor intensidad de respuesta hacia conespecificos comparado con la interespec\'{i}fica. Sin embargo, la latencia y duraci\'{o}n de la respuesta territorial fue similar para conespecificos y para el Aguililla Negra Mayor, mientras que los halcones mostraron baja respuesta al Gavil\'{a}n Zanc\'{o}n. Los resultados demuestran que la competencia intraespec\'{i}fica es el principal factor que regula la distribuci\'{o}n de los territorios de los halcones. Sin embargo, los halcones experimentaron una mayor territorialidad hacia el Aguililla Negra Mayor, que presenta requerimientos ecol\'{o}gicos similares (estrategia de caza, dieta, uso de h\'{a}bitat y tama\~{n}o corporal). En comparaci\'{o}n, la estrategia especializada de caza del Gavil\'{a}n Zanc\'{o}n, que busca presas en cavidades, puede reducir la competencia interespec\'{i}fica con esta especie. Por lo tanto, una combinaci\'{o}n de competencia intra- e interespecifca puede regular los territorios de rapaces en el bosque tropical seco.}

\normaltalk{The Influence of the North American Monsoon on Breeding Abundance of Western Molt Migrants}{Paul J Dougherty, Matthew D Carling}{Conservation efforts are most successful when they take into account the full annual cycle of a species. With an increasing number of studies finding evidence that birds often molt in habitats distinct from breeding and wintering territories, there is a need to determine how conditions in molting areas influence populations trends. Many Passerine species that breed in western North America are molt-migrants, which migrate to monsoonal areas of the southwest after breeding to carry out prebasic molt. We analyzed Breeding Bird Survey and Daymet climate data to investigate how characteristics of the fall monsoon (total precipitation, start date, duration, and intensity) influence breeding abundance of western molt-migrants the following spring. In order to isolate the influence of the monsoon on molt-migrant population size, we compared breeding abundance trends of each molt-migrant species to the trends of a closely related species that is not known to rely on monsoonal productivity. Our analyses show that there is a significant positive relationship between monsoon precipitation and annual breeding abundance of some molt migrants, especially those that are not known to suspend prebasic molt. The strength of this relationship varies by location, with populations that breed farther north usually more sensitive to variation in the monsoon. The relationship between monsoon precipitation and molt-migrant breeding abundance demonstrates the need to protect molting areas and to closely monitor these species as climate change alters the timing and intensity of the monsoon.}

\normaltalk{Phylogenetic Analyses Reveal Drivers of Adaptive Interspecific Territoriality}{Jonathan Drury}{Historically, many biologists have dismissed observations of territorial aggression between species as the result of misdirected aggression or niche divergence being blocked by some extrinsic factor. However, several recent studies have challenged this view, instead suggesting that interspecific territoriality may be adaptive for the same reasons that intraspecific territoriality is adaptive. Focusing on North American passerines, we conducted the first large-scale, phylogenetic analysis of the distribution and determinants of interspecific territoriality. We found that interspecific territoriality is common, with individuals from nearly a third of all species defending territories against one or more other species. Contrary to the prevailing view, we also found abundant support for the hypothesis that interspecific territoriality is an adaptive response to resource competition and reproductive interference--not just a rare occurrence restricted to recently diverged lineages--and that interspecific territoriality constrains the evolutionary divergence of territorial signals. This study highlights one avenue through which behavioral interactions (in this case, interspecific territoriality) can shape phenotypic diversity on a macroevolutionary timescale.}

\normaltalk{Using Bird Specimens to Reconstruct 200 Years of Atmospheric Soot and Its Impacts on Signaling Traits}{Shane DuBay, Carl Fuldner}{Natural history collections are powerful resources for tracking environmental pollutants through time because specimens provide durable records of the past environments from which they were collected. Our research leverages specimens to reconstruct historical emission of atmospheric soot, a byproduct of burning organic matter, like coal, and a major contributor to climate change. We sample $>$10,000 bird specimens from across the British and Irish Isles to reconstruct the spatiotemporal distribution of soot pollution over the last 190 years. These data not only allow us to assess the historical extent of soot around industrial centers, but they also provide the necessary foundation to assess the ecological and evolutionary consequence of soot pollution on the birds themselves. For example, what happens when sexually selected plumage patches are coated in soot, obscuring plumage signals that have evolved in cleaner environments? What are the consequences of soot deposition for visual predators when animal prey coloration is homogenized with the surrounding environment? This work builds toward a robust, spatially informed inventory of atmospheric soot, while providing the necessary foundation to assess the impacts of soot pollution on signaling traits.}

\normaltalk{Maladaptive Nest Site Selection and Reduced Nest Survival in Female Sage Grouse Following Wildfire.}{David J Delehanty, Ian F Dudley, Peter S Coates}{In the North American Great Basin, increased wildfire frequency and associated replacement of sagebrush (Artemisia spp.) with invasive annual grasses are predicted to result in further population declines of greater sage-grouse (Centrocercus urophasianus; hereafter sage-grouse). We found that female sage-grouse made maladaptive nesting decisions by continuing to occupy habitat degraded by fire resulting in significantly reduced nest success. We conducted a before-after-control-impact (BACI) study design to estimate sage-grouse nest survival in burned and unburned areas before and after the 126,000-ha Rush Fire in northeastern California and northwestern Nevada, USA. We also measured post-fire microhabitat influence on sage-grouse nest survival in burned and unburned areas. Sage-grouse nest survival probabilities in the burned area were reduced from 0.52 (95\% CRI = 0.31--0.74) before to 0.19 (95\% CRI = 0.06--0.42) after the wildfire. Female sage-grouse continued to nest in the burned area but did not select for big sagebrush (Artemisia tridentata) which was substantially reduced in abundance, and instead selected alternative substrates such as perennial grass for nest cover. Female sage-grouse exhibited nest-site fidelity at a reproductive cost, continuing to nest in areas that the fire had converted to low quality nesting habitat.}

\normaltalk{The Trials and Tribulations of Counting Parrots in the Wild; are Citizen Science Databases an Informative Alternative to Traditional Field Counts?}{Molly K Dupin, Christine R Dahlin, Timothy F Wright}{In order to create effective conservation plans for any species, up-to-date, accurate information about population status is essential. Traditionally, these data have been collected by counting wild individuals in the field; however, this process is time-consuming and expensive. Regardless of expense, recent, rapid taxa-wide decline has made detection of endangered species critically important. Among birds, parrots are particularly threatened, with forty-six percent categorized as near threatened or worse. The yellow-naped amazon, Amazona auropalliata, has experienced well-documented decline in the southern portion of its range, Nicaragua and Costa Rica, resulting in their 2017 reclassification to `endangered' by BirdLife International. Despite this reclassification, there persists a knowledge gap about the status of populations in its northern range of Guatemala, Honduras and southern Mexico. We aimed to fill this gap by comparing population assessments using traditional roost counts and the citizen science database, ebird. We conducted roost counts at 74 locations across Mexico, Guatemala, the Bay Islands of Honduras, Nicaragua, and Costa Rica during non-breeding seasons between 2016-2019. The mean number of birds detected using traditional counts was 33.0 ± 55.3, and 16.5 ± 35.3 for ebird. Data for this species was relatively sparse on ebird, and some sightings seemed like misidentifications based on our own surveys. There were several major roosts that we detected using traditional counting methods which were not represented on ebird. Our results highlight the need for more effective approaches to counting wild parrot populations, such as passive acoustic or video monitoring or the development of conservation-based censusing networks.}

\normaltalk{Possible Shared Parental Care in Lark Sparrows (Condestes Grammacus)}{Diego Duran, Alex Sinker, Claire V Ramos}{The Lark Sparrow (Condestes grammacus) is songbird that breeds in the southern grasslands of North America. These birds are considered to be monogamous, however anecdotal observations suggest multiple female Lark Sparrows may feed at the same nest. This suggests that there could be cooperative breeding in this species, which has not previously been described. Here we conducted quantitative observations of parental care in Lark Sparrows to determine if cooperative breeding is common in this population. This research was conducted at the U.S Army Chemical Depot in Pueblo County in Southeastern Colorado. We mist netted the parents to the known nests, and the birds were banded with an aluminum numbered band and colored bands for identification purposes. The nests were also video taped for two hours just after sunrise to determine what adults were feeding at the nest. The results showed no evidence of Lark Sparrows cooperatively breeding. However, observation indicate that Lark Sparrows may be less territorial than previously thought. In the future we plan to determine parentage of chicks in the nest and whether there is functional overlap between adjacent territories.}

\normaltalk{The Status of Dominica's Parrot Conservation and Research Program and the Imperial Parrot Post-Maria}{Stephen Durand, Paul R Reillo}{Since 1980, Dominica's Forestry, Wildlife and Parks Division has operated an internationally renowned parrot research and conservation program.  Throughout its tenure, the program achieved unparalleled protection of Dominica's endemic parrots and numerous conservation milestones including the establishment of a very productive partnership and collaboration with the Rare Species Conservatory Foundation and the creation of the Morne Diablotin National Park (January 2000).  Notably, the program protected the parrots from local exploitation and international trade, with all conservation and recovery strategies employed on-island.  Both species, and all captive parrots at the Parrot Conservation and Research Centre (PCRC) survived devastating Hurricane Maria in September 2017.  In March 2018, a Dominican government official conspired with known, convicted criminals from ACTP (Assoc. for the Conservation of Threatened Parrots), without Forestry's knowledge or participation, to secretively export 10 Amazona arausiaca and 2 A. imperialis held at the PCRC to ACTP's facility in Germany.  Since then, Forestry's parrot research and conservation efforts largely have been paralyzed.  This presentation will discuss the status of the program before and after ACTP's actions, responses from the international scientific community, remedial steps taken, ongoing GPS/GIS surveys to assess the Imperial population and its conservation status post-Maria, and strategies to restore Forestry's governing authority over Dominica's wildlife.}

\normaltalk{Does Variation in "sexually Selected" Traits Reflect Current Opportunity for Sexual Selection Across Species?}{Carla C Vanderbilt, members of the Manakin RCN female mate choice working group, Emily H DuVal}{Plumage dichromatism, elaborate displays, and lekking behavior are considered hallmarks of historically intense sexual selection.  However, it is unclear whether and how such characteristics relate to current inequalities in reproductive success and the potential for current selection to act. We investigated these relationships in 15 species of manakins (family Pipridae), a clade considered to be strongly influenced by sexual selection.  To first identify a reasonable fitness proxy available in the maximal sample of species, we compared correlations among fitness measures from five species for which research groups had collected multiple possible fitness proxies.  We found that rates of display for females reasonably approximated rates of success from copulation rates and genetically determined siring success, and that rates of displays without females present did not. We then used rate of displays for females proxy to calculate Is, the opportunity for sexual selection, in 15 species.  Is is the variance in success divided by the squared mean success, and estimates the upper limit of realized sexual selection in a population. Is values varied considerably among species, ranging from 0.58-3.00 (mean 1.77).  We conducted a phylogenetically controlled comparative analysis to test for correlations of Is with display complexity, vocal diversity, spatial organization, dichromatism, social organization, and spatial organization. Our analysis suggests that traits commonly assumed to reflect current sexual selection are not consistently related to the current intensity of sexual selection across species.  This highly collaborative project was made possible by significant contributions from more than 40 members of the NSF Manakin Genomics RCN.}

\normaltalk{A Quantitative Framework for Inferring Vision-Related Behavioral Capabilities in Extinct Birds}{Catherine M Early, Ryan C Ridgely, Lawrence M Witmer}{Brain endocasts provide the most direct evidence of neuroanatomy and behavior in extinct birds. The optic lobe and the Wulst are structures that are visible on the endocasts of most birds. They overlie the optic tectum and hyperpallium, respectively, two brain structures in the visual pathways of birds whose relative size seems to correlate with emphasis on visual information. Researchers have inferred potential functional capabilities and behaviors of extinct birds from relative endocast structure size, but it is unknown if the size of the endocast structure faithfully represents the size of the brain structure. To test this assumption, we measured the surface areas of the optic lobes and Wulsts and volumes of the optic tecta and hyperpallia in a diverse evolutionary sample of extant birds. Our regressions of these values yielded strong, significant, positive relationships between the volumes of the brain structures and the surface areas of the overlying endocast structures. We used a phylogenetic prediction method based on Bayesian inference to calculate the volumes of the brain structures of a few extinct birds based on the surface areas of their endocast structures. Phylogenetic ANCOVAs indicated that the relative optic tectum and hyperpallium volumes of most of the extinct birds studied did not significantly differ from the extant sample. However, the optic tectum of Dinornis robustus was significantly smaller than any of the extant birds in our sample. Our results provide an analytical framework within which hypotheses on the evolution of the brains and behaviors of birds can be tested.}

\normaltalk{Open, Accessible, and Customizable Analysis of North American Breeding Bird Survey Data using the R Package bbsBayes}{Brandon P Edwards, Adam C Smith, Marie-Anne R Hudson, Charles M Francis, Keith L Pardieck, David J Ziolkowski Jr}{Data from the North American Breeding Bird Survey (BBS) have been readily available to download from the United States Geological Survey website for several years, with the data set being updated on an annual basis. Although the goal of this freely-available data set is to allow scientists to perform their own analysis on the data, or to replicate status and trend estimates published yearly by the United States Geological Survey (USGS) and Canadian Wildlife Survey (CWS), the rich and complex nature of the BBS data set often presents a major barrier. The R package bbsBayes was developed to mitigate this barrier in analysis of BBS data. bbsBayes provides an accessible package for anyone in the conservation community to estimate population trajectories (time-series) and trends (rates of change) for any of the 500+ bird species monitored by the BBS, and to allow more advance users to easily access the data and model-templates necessary to customize an analysis for their research. The package provides the user with functionality as simple as downloading and subsetting BBS data, to functionality allowing a user to replicate trend estimates produced by the USGS and CWS, to functionality as complex as generating population trajectories and trends for custom combinations of strata. Here, we present details of this functionality, explore a variety of different analyses and visuals that can be produced using bbsBayes, and make suggestions of how bbsBayes could be integrated into the typical workflow of a full BBS data analysis.}

\normaltalk{US Imperiled Species are Most Vulnerable to Habitat Loss on Private Lands}{Adam J Eichenwald, Michael J Evans, Jacob W Malcom}{To stem the ongoing loss of biodiversity, conservation practitioners must distinguish between effective and ineffective approaches for protecting species habitats. Using Google Earth Engine and 31 years of Landsat images, we quantified changes in the habitats of 24 vertebrates listed under the US Endangered Species Act (ESA) and on the International Union for Conservation of Nature (IUCN) Red List across categories of land ownership (eg federal, state, private) in the continental US that are subject to different conservation-focused legal restrictions. These estimates exclude changes attributable to agricultural conversion and burned areas. The imperiled species we evaluated lost the least amount of habitat (3.6\%) on federal lands, whereas losses on private lands without conservation easements were more than twice as high (8.1\%). Differences in annual percent loss before and after ESA listing, and between ESA-listed and Red List species, indicate that the ESA limited habitat loss and was most effective on federal lands. These results underscore the importance of federal lands in protecting habitat for imperiled species and highlight the need to improve habitat protection on private lands for long-term conservation.}

\normaltalk{Against the Wind: Accelerometry Demonstrates Demonstrates Finescale Tactics for Changing Windscapes}{Kyle Elliott}{Windscapes affect energy costs for flying animals, but animals can adjust their behavior to accommodate wind-induced energy costs. Theory predicts that flying animals should decrease air speed to compensate for increased tailwind speed by increasing wingbeat frequency or amplitude. In addition, animals are expected to vary their foraging effort in time and space to maximize energy efficiency across variable windscapes. We examined the influence of wind on seabird (thick-billed murre Uria lomvia and black-legged kittiwake Rissa tridactyla) foraging behavior. Wingbeat amplitude varied with windspeed while wingbeat frequency varyied with body mass illustrating different tactics for dealing with altered flight costs. As predicted, birds adjusted their airspeed to compensate for crosswinds and to reduce the effect of a headwind.  During high winds, murres switched from feeding their offspring with schooling fish, which required substantial above-water searching, to amphipods, which required less above-water searching. Thus, adults buffered the adverse effect of high winds on chick growth rates by switching to other food sources during windy days or increasing food delivery rates when weather improved. We present fine-scale evidence for how birds adjust their flight in response to changing costs.}

\normaltalk{Evolutionary Genomics of a Widespread Continental Radiation: The Motacilla Wagtails}{Erik D Enbody, Per Alström, Navaneeth Menon, Rebecca Harris, Yang Liu, Fumin Lei, Leif Andersson}{The Motacilla wagtails are a classic example of an adaptive radiation that has resulted in an enormous diversity in male ornamented plumage in a short time period. Within the Motacilla wagtails, two widespread species, Motacilla flava and Motacilla alba, are characterized each by as many as fourteen phenotypic subspecies across the Northern Hemisphere. Despite this high phenotypic variation, little genetic differentiation is seen between subspecies when sampling thousands of genetic loci, which has made the evolutionary history of Motacilla wagtails difficult to resolve. However, these characteristics provide a potentially powerful model system to describe the molecular basis of sexual selection in a widespread adaptive radiation. Using whole-genome resequencing of all Motacilla wagtail species and nearly all described subspecies, we search for signals of selection in the genome that have shaped the global diversification in wagtail plumage. In addition, we present a denovo chromosome-scale genome assembly of M. alba to facilitate analysis. We find that few regions of the genome are differentiated and we discuss the processes that have led to genomic divergence, including signals of classic hard selective sweeps in both species. Our results highlight shared vs. unique loci that underlie the combinatorial patterning within both M. flava and M. alba and contrast different patterns of evolution at the genome level between both these two species and in the context of other Motacillidae members. We discuss the role of sexual selection in continental radiations by leveraging a powerful, common, and widespread group with a vexing taxonomic history.}

\normaltalk{Genetic Basis and Long-Term Dynamics of a Color Polymorphism in Darwin's Finches}{Erik Enbody, Peter R Grant, G R Grant, C G Sprehn, Arkhat Abzhanov, Mariya Dobreva, Owqen Osborne, Chao Wang, Leif Andersson}{Adaptive radiations provide a powerful lens for observing the process of lineage diversification through natural selection. A detailed example of the process by which locally adapted phenotypes respond to environmental fluctuations and contribute to population divergence comes from long-term field study of fluctuations in beak morphology among Darwin's finches (Thraupidae). Specialized morphologies in each population of Darwin's finches are thought to have played a critical role in speciation in this group. An intriguing observation is the presence of a stable beak color polymorphism: finch nestlings have beaks that are either pink or yellow. Unlike beak morphology, the nestling color morphs are present across species. The pink morph is commonly at higher frequency than the yellow, but frequencies of the two morphs vary among species. Color morph frequencies within families is consistent with a model of simple mendelian inheritance, and the yellow morph is recessively inherited. We find that a single SNP in the carotenoid processing gene BCO2 is associated with the yellow morph using whole-genome sequencing of thousands of individuals. We use this SNP to track frequencies of the morphs through time across the entire population monitored on Daphne Major over 40 years. We discuss the conditions that have maintained and perturbed the stability of the polymorphism, including introgressive hybridization, fluctuations in resource availability, and different dietary preference between species. Our findings contribute to a comprehensive understanding of the processes that shape adaptive radiations through detailed investigation into the role of a stable polymorphism in this iconic group.}

\normaltalk{Multiple (but few) Genes are Associated with Differences in a Color Trait in the Rapid Capuchino Radiation.}{Mar\'{i}a C Estalles, Leonardo Campagna, Sheela Turbek, Mar\'{i}a J Rodr\'{i}guez-Cajarville, Irby Lovette, Pablo L Tubaro, Dario A lijtmaer}{New sequencing technologies have revolutionized our understanding of the genetic basis of avian plumage coloration and its role in the speciation process. The southern capuchinos (genus Sporophila) are an ideal study system to explore this connection. This group consists of ten species that have very low genetic differentiation and vary almost exclusively in plumage coloration. The few divergent areas of their genomes are associated with pigmentation genes. However, because many plumage patches vary at the same time in this group, direct associations between genes and specific coloration patches have not yet been established. S. hypoxantha, S. palustris and S. ruficollis differ in a single coloration patch, the throat (cinnamon, white and black, respectively), while S. iberaensis differs in other coloration patches but shares a black throat with S. ruficollis. Here we compared whole-genome sequences of 58 individuals of these four capuchino species with the objective of elucidating the genetic underpinnings of throat coloration. We found a strikingly low level of genomic differentiation, restricted to three genomic regions, among S. hypoxantha, S. palustris and S. ruficollis. These regions included genes associated with coloration and are therefore candidates for determining throat color in the group. S. iberaensis differed from S. hypoxantha and S. palustris, but not S. ruficollis, in the same genomic regions associated with throat coloration. Taken together, our results are consistent with the idea that throat coloration is a polygenic trait that has contributed through sexual selection to the rapid capuchino radiation in the grasslands of South America.}

\normaltalk{Sensitivity Analysis for an Avian Reproductive Toxicology Model}{Matthew A Etterson, Nika Galic, Katharine Ruskin, Michael Thiel}{MCnest is an integrated exposure and effects model used by USEPA for avian ecological risk assessment for pesticides. Models like MCnest integrate information on exposure, toxicity, and life history to predict potential risk to modeled species. However, the underlying biological and ecological processes occur at different temporal rates and are often measured at different levels of resolution.  These considerations of scale and measurement resolution influence the importance of different parameters for conclusions about risk that can be illuminated using sensitivity analysis.  We present both local and global sensitivity analyses of the MCnest model together with small and large perturbations.  Among life history parameters, nest survival and adult survival rates are the most influential parameters determining seasonal productivity, followed by lengths of development periods (e.g., incubation and nestling).  Principal Components analysis on the 50 species modeled revealed that almost all ($>$95\%) of variation in breeding season parameters among species was explained by the first principal component, which corresponds closely to the altricial-precocial spectrum.  The latter result has important consequences for understanding interactions between chemical contaminants and nutrients transferred in ovo.}

\normaltalk{Urban Food Subsides Reduce Natural Food Limitations and Reproductive Costs for a Wetland Bird}{Betsy A Evans, Dale E Gawlik}{There is a strong conservation need to understand traits of native species that adapt to urban environments. Wetland birds have shown a strong phylogenetic signal towards urban tolerance; however, these species have largely been ignored in urban studies. In their historic ranges, wetland birds inhabit dynamic systems, traveling long distances to locate food. This evolutionary adeptness of exploiting resources may translate to success in urban environments; areas characterized by novel food opportunities. We used the Wood Stork (Mycteria americana), a wetland species of conservation concern, to determine if the ability to exploit resources in natural environments translated to exploitation of urban resources. We found that when wetlands offered optimal foraging conditions, storks nesting in both urban and natural wetlands had narrow diet breadths and high reproductive success. However, when natural wetlands offered poor foraging conditions, urban storks expanded their diet and consumed more anthropogenic prey than did storks nesting in natural wetlands. Urban storks also had greater productivity than did storks nesting in natural wetlands, and therefore paid a lower penalty when wetlands offered suboptimal foraging conditions. Our study provides a mechanistic understanding of how a wetland species persists, and even thrives, in an urban environment. We demonstrated that species inhabiting dynamic systems can exploit urban areas which result in increased reproductive performance during suboptimal natural system conditions. Urban environments may support biodiversity in a variety of ways, but species-specific mechanistic understanding will illuminate how to best mitigate the potential threats of urbanization.}

\normaltalk{Effects of Hurricanes on Resident and Winter-Resident Birds: 45 Years of Monitoring in the Gu\'{a}nica Dry Forest of Puerto Rico}{John Faaborg, Judith D Toms, Wayne J Arendt}{We have been conducting constant-effort mist netting in the Gu\'{a}nica dry forest of southwest Puerto Rico since 1972. Over this time period, several severe hurricanes have impacted the island; some of these hurricanes directly impacted our study area, while others mostly impacted other parts of the island. In order to assess the local and spillover effects of these hurricanes, we analyzed the capture and recapture data using program MARK, which allowed us to estimate survival rates and populations of both resident and site-faithful winter resident populations. Only two hurricanes (Georges 1999, Maria 2016) showed extensive damage to the forest canopy in Gu\'{a}nica. In both cases, longer-term analyses suggest there has been little to no effect on survival rates or populations over time. However, while captures showed a major decline after Georges, captures increased greatly after Maria. Extensive canopy loss and associated piles of debris after Georges may have forced bird movements higher into the canopy (and above the level of mist nets), or birds may have moved out of the area for a short period of time. Maria had less local canopy loss, and the increase in captures may reflect birds immigrating from other, more damaged forests on the island. Impacts of other hurricanes appeared to be negligible. Our results suggest that although bird populations may be impacted by both local and spillover effects of hurricanes on short time scales, they are resilient to such impacts over longer time scales.}

\normaltalk{Females that Look Like Males: A Potential Case of Intersexual Mimicry in a Hummingbird}{Jay J Falk, Dustin Rubenstein, Michael Webster}{A major source of variation in the animal kingdom stems from differences between and within sexes. Sexual dimorphism can be studied within species if variation in dimorphism exists (i.e. when one sex varies in similarity to the other). This type of variation is frequently found in males but is relatively rare in females. Female-limited polychromatism is a fascinating evolutionary conundrum because theories typically used to explain variation (e.g. sexual selection) do not readily explain the observed variation in females. Many hummingbird species contain female-limited polymorphism. In one, Florisuga mellivora, ~30\% of females have ornamented plumage nearly identical to that of the male, while the others have drab coloration. The function of male-like coloration in females has not been studied, but an association with male-like territorial behavior has been hypothesized. We observed interactions of wild hummingbirds with taxidermy mounts of male-like females, drab females, and true males at feeders. Attacks were more often directed toward drab females than both male-like females and true males, suggesting that male-like coloration may be associated with higher resource holding potential. However, measures of muscle capacity and body size showed no difference between the two types of females, whereas males were more aggressive and were morphologically different from both female types. These results are consistent with a hypothesis in which male-like females are mimicking males in order to avoid aggression and resource-related harassment. Therefore, we suggest that in this case male-like ornamentation in females may be a result of divergent behaviors between the sexes.}

\normaltalk{Engaging Youth in Science and Conservation Through BirdSleuth Caribbean}{Jennifer Fee, Lisa Sorenson}{Birds are an engaging and fun way to get youth interested in nature, science, and inquiry-based learning. Cornell Lab of Ornithology partnered with BirdsCaribbean to adapt their BirdSleuth International program for the Caribbean context. Through the BirdSleuth Caribbean program and curriculum, 51 Caribbean educators from 19 countries received training and teaching resources during two international workshops that help them get young people outside to build their science skills and develop a conservation ethic. The BirdSleuth Caribbean curriculum is available in English, Spanish and French. It is supported by a kit for educators that contains resources and materials needed to carry out the lessons and hands-on activities (bird ID cards, posters, games, field guides, binoculars, art supplies, presentations, etc.). Following the training, partners carried out local training workshops in their own countries. Results from the 5-year project show that the program was successful in promoting student environmental stewardship and establishing a network of conservation educators and teachers who are trained in the use of the curricula, including bird identification skills, inquiry-based learning, and citizen science activities. Ultimately, the program was used by more than 1,081 educators in 22 countries, reaching more than 37,755 children and 9,329 community members. Positive outcomes included the declaration of a protected bird sanctuary, planting of native trees beneficial to birds, clean-ups, development of birding trails, and use of eBird Caribbean to monitor local birds. All of the program's materials are available for free download from BirdsCaribbean's website, helping to ensure long-term sustainability of the program.}

\normaltalk{Behavioral responses to divergent vocal signals in allopatric Neotropical sparrows are explained by acoustic similarity}{ Ronald A Fern\'{a}ndez, Jorge E Morales-M\'{a}vil, Laura T Hern\'{a}ndez-Salazar, J. Roberto Sosa-L\'{o}pez}{Birdsongs have a main role in reproductive behaviors, encoding information for the recognition of individuals. Songs usually vary across geography with divergent patterns between populations. Together, the discrimination ability and songs' geographic variation, establish the bases for evolutionary processes related to reproductive isolation and population divergence. We assessed the vocal and physical responses of receivers to songs from allopatric populations in the Olive Sparrow, Arremonops rufivirgatus, a Neotropical birdsong with five isolated populations. We conducted playback experiments on two allopatric populations. Treatments were songs from local (Pacific or Peninsula), allopatric (Gulf, Chiapas, Central-America), and heterospecific populations (Black-striped Sparrow, A.conirostris). We asked whether Sparrows have differential responses among treatments and whether the response strength can be predicted by acoustic similarity between subject' songs and stimulus' songs. Our results indicate that Pacific and Peninsula populations responded differently to treatments; Pacific population showed a strong respond to both local and allopatric stimuli, and weak response to the control; Peninsula population showed a strong respond to local stimuli and weak response to allopatric and the control. We found that the intensity of the response can be predicted by the acoustic similarity between the focal and the stimulus songs, with individuals responded strongly to more similar songs. Further, we show evidence that suggest both populations used the same variables to recognize conspecific songs. Our findings support the hypothesis that song divergence could act as a pre-mating barrier and that birds' responses are dictated by the similarity in the fine structural characteristics of the signal.}

\normaltalk{Does Getting the Worm Help the Breeding Bird? Supplemental Feeding Affects Nestling Growth in some Taxa but not Others}{Facundo Fernandez-Duque, Sarah A Dzielski, Robyn L Bailey, David N Bonter}{Globally, millions of people spend billions of US dollars annually to provide food for wild birds with the intention of helping individuals and populations. The variety of foods offered continues to diversify, yet the consequences for wildlife of this widespread human activity are relatively unexplored. To investigate the potential costs and benefits of supplemental feeding, we experimentally examined the relationship between providing supplemental insect prey and nestling growth rates in two species of wild birds across two nesting seasons. Nesting adults in the experimental group were provided with 10 grams of insects per day from the day their clutch hatched until the chicks fledged. Results were mixed, with eastern bluebird (Sialia sialis) nestlings growing to a 5\% greater mass at fledging, on average, than offspring in nests not receiving supplemental food. In contrast, no differences in mass gain were detected between treatment and control groups among black-capped chickadee (Poecile atricapillus) nestlings. Models also supported variation in bluebird growth rates depending upon weather conditions with greater gains at higher mean temperatures and with less precipitation. Although people who feed wild birds are motivated by a desire to contribute to conservation and to help birds, the consequences of supplementing birds vary and feeding may not have the desired effects.}

\normaltalk{Phylogenomics Indicate Smaller Taxonomic Diversity in a Manakin Species than Currently Understood}{Carolina Ferreira, Erico Polo, Lucas Bandeira, Pedro Paulo, Fernando Te\'{o}filo, Lainy B Day, Alexandre Aleixo, Marina Anciães, Tomas Hrbek}{Many species of Neotropical birds have large distributions, with many of these divided into subspecies. However, following phylogenetic studies and taxonomic revisions, these subspecies are generally discovered to be species. The Lepidothrix coronata is endemic to the Neotropics and has one of the widest distributions among manakins, occurring from Costa Rica to Bolivia, with pronounced phenotypic variation along its distribution. In general, females have a discrete green color, while males are characterized by bright colors with various intergrades from green to black, and a blue crown. In addition to sexual dimorphism, the species is known for its lekking behavior. Based on geographic distribution and adult male color patterns, eight subspecies have been described. However, using partial representational genome sequencing to infer the phylogenetic relationships among seven of the eight L. coronata subspecies, we found no evidence for the existence of these taxa. Rather, we found evidence for the existence of four, geographically structured, evolutionarily independent lineages that diverged close to the Pliocene/Pleistocene boundary. The divergence of the lineages appears to be mainly related to vicariant events such as the establishment of the Amazonian fluvial system and the rise of the northern Andes, but we also found strong evidences that climatic fluctuations during the Pleistocene played a key role in the divergence of these lineages. The lack of correspondence between phylogeny and current subspecies taxonomy, indicates that adult male plumage color diversity in Lepidothrix coronata is probably the result of retention of ancestral polymorphism potentially promoted and maintained by sexual selection.}

\normaltalk{Post-Hurricane Mar\'{i}a Population Distribution and Abundance of the Elfin-Woods Warbler (Setophaga Angelae) in the Luquillo Experimental Forest.}{Laura L Fidalgo-DeSouza, Alberto C Cruz-Mendoza, Frank F Rivera-Mil\'{a}n, Jessica Ilse, Joseph M Wunderle, Jr}{The population of the endemic Elfin-woods Warbler (EWWA) has fluctuated widely before-after hurricanes in the Luquillo Mountains, Puerto Rico.  Knowledge of the EWWA response to hurricanes and its basic habitat requirements is crucial for conservation management of this threatened species. Hence, eight months after Hurricane Maria point counts were conducted during 4 visits from May to September 2018, followed by image-based habitat measurements at each site. Field data were used to estimate EWWA occupancy and abundance within the forest habitats in the Luquillo Experimental Forest.  Across the study area (11,427 hectares), EWWAs were detected in 48 of the 193 sites, but detections varied spatially. No evidence was found for elevational shifts by the EWWA after the hurricanes, as highest EWWA counts occurred between 600-900 m asl in the Palo Colorado forest, as previously documented by others before the hurricanes. Vegetation structure characteristics where EWWAs were predominantly detected had  higher forest structural complexity, intermediate percentages of understory vegetation, and mid to low canopy cover. Average occupancy probability was estimated at $<$U+03A8$>$ = 0.3150 (s.e. = 0.0442), and detection probability was estimated as P (1,2,3) = 0.3676 (s.e. = 0.0491) and P (4) = 0.2172 (s.e. = 0.0568). Using count data adjusted for changes in detection probability, we are comparing image-based forest structure measures and estimated abundance to characterize post-hurricane EWWA habitat, which we will report at the conference.}

\normaltalk{Oldest Modern Bird Fossil Clarifies the Earliest Stages of the Extant Bird Radiation}{Daniel J Field, Juan Benito, Albert Chen, John W Jagt, Daniel T Ksepka}{Our understanding of the earliest stages of crown bird evolution is hindered by an exceedingly sparse Mesozoic fossil record. The most ancient phylogenetic divergences among crown birds are known to have occurred in the Cretaceous, but stem lineage representatives of the deepest crown bird subclades--Palaeognathae (ostriches and kin), Galloanserae (landfowl and waterfowl), and Neoaves (all other extant birds)--are entirely unknown from the Mesozoic. As a result, key questions related to ancestral crown bird ecology, biogeography, and divergence times remain unanswered. We report a new Mesozoic fossil that occupies a position close to the last common ancestor of Galloanserae, filling a key phylogenetic gap early in crown bird evolutionary history. Asteriornis maastrichtensis, gen. et sp. nov., from the Maastrichtian of Belgium, is represented by a nearly complete, three-dimensionally preserved skull and associated postcranial elements. The fossil represents one of the only well-supported crown birds from the Mesozoic Era, and is the first Mesozoic crown bird with well represented cranial remains. A. maastrichtensis exhibits a heretofore undocumented combination of galliform (landfowl)-like and anseriform (waterfowl)-like features, and, along with a previously reported Ichthyornis-like taxon from the same locality, provides the first direct evidence of co-occurring crown birds and avialan stem birds. Its occurrence in the northern hemisphere challenges biogeographic hypotheses of a Gondwanan origin of crown birds, and its relatively small size and possible littoral ecology may corroborate recently proposed ecological filters influencing crown bird persistence through the end-Cretaceous mass extinction.}

\normaltalk{Assigning Migratory Behavioral States to Satellite-Derived Tracking Data of American Woodcock}{Alexander C Fish, Amber M Roth, Erik J Blomberg}{Miniaturized biologging devices are providing unprecedented insight into how organisms move at various spatial scales.  Migratory periods are of particular interest as many species are understudied during migration, primarily due to an inability to follow and relocate individuals throughout migration.  Recent advances in satellite transmitters allow for migrating birds to be remotely tracked, using periodic satellite uploads collected continually throughout migration.  Smaller transmitters sizes now allow for medium-bodied birds to be tracked, although battery size limits the number of collected gps locations.  Frequent collection, every day or every few days, allows for day-to-day description of movements between seasonally used areas and time spent at stopovers sites.  Reconstructing phenological data from raw movement data requires the ability to distinguish between different migratory states (e.g.  stopover vs settlement).  Using migratory movement data collected from over 200 American Woodcock (Scolopax minor) monitored between 2017-2020, we used two methods; dynamic Brownian bridge movement models and estimates of Brownian motion variance, and net-squared displacement to distinguish pre-migration, stopover, and post-migration behavioral changes.  We further evaluated how age, sex, latitude, longitude, and biological covariates influenced the phenology of migration during both spring and fall migratory periods.  We demonstrate the benefits of using automated approaches and compare between two methods to identify behavioral changes associated with migration and provide recommendations for applying this method to data collected over a similar time series.  We describe how pre-migration ranging and mid-winter dispersal movements have the potential to be misclassified as migratory movements and provide recommendations to avoid misclassification.}

\normaltalk{The Use of Birds as Bioindicators to Monitor the Impact of Recreational Activities in the Northeastern Ecological Corridor Reserve and Seven Seas Reserve in Puerto Rico}{Eliseo E Font-Nicole}{}

\normaltalk{Characterization of the Ovenbird Flight Song: Plasticity in Output and Composition}{Jennifer R Foote, Megan J Thompson, Kendriah Pearse, Nadine Paul}{The songs of most passerines have been well described, however songs from some species are more difficult to record and have yet to be characterized. The Ovenbird (Seiurus aurocapilla) flight song occurs rarely, is complex in its syllable composition, and its function is still not well understood. We used autonomous recorders to record Ovenbird vocalizations during the breeding season near Sault, Ste. Marie, ON. First, we examined weekly recordings over 24 hour diel periods to estimate temporal song output for both primary and flight songs (N=99259 songs, 21 individuals). Second, we examined 396 clearly recorded flight songs (N=23 individuals) and characterized syllable repertoires, examined syllable transition patterns, and estimated repeatability of song features. We found that song output decreased across the season for both song types, with flight song activity declining similarly across diel periods. We found temporal separation of song types over diel periods, with flight songs being most commonly sung at dusk and during the night. We find evidence that individuals were not consistent in flight song production, suggesting flight songs are plastically expressed. We found that syllable repertoires, song length, and syllable versatility varied among males and that males transitioned differently between syllables when singing. Males were significantly repeatable in flight song composition but repeatability was lower for terminal segments suggesting this song segment is more plastically adjusted. Our work contributes to understanding the structure and potential functions of warbler flight songs and demonstrates the utility of using autonomous recorders to study sporadic vocalizations.}

\normaltalk{Climate Change Impacts on Montane Avifaunas Across Latitudes: Patterns, Processes, and Next Steps}{Benjamin G Freeman}{Climate change is predicted to fundamentally rearrange montane avifaunas via setting in motion an ``escalator to extinction." In this talk I will examine the empirical evidence that (1) bird species are shifting upslope as temperatures warm, (2) high elevation species are in an extinction vortex, and (3) warming temperatures are shifting overall diversity patterns, drawing on my field studies of tropical avifauna's distributional responses to climate change as case examples. There is substantial variation among studies that have measured upslope shifts in montane avifaunas; I will test whether macroecological variables explain this variation. I will finish by (1) summarizing the likely mechanisms underpinning observed elevational range changes and (2) charting a way forward to gather the data necessary to answer outstanding questions.}

\normaltalk{Regularly-Drinking Desert Birds Have Greater Evaporative Cooling Capacity and Higher Heat Tolerance Limits than Non-Drinking Species}{Marc T Freeman, Zenon J Czenze, Ryno Kemp, Ben Smit, Barry van Jaarsveld, Blair O Wolf, Andrew E Mckechnie}{Surface water is a critical resource for many birds inhabiting arid regions, but the implications of regular drinking and dependence on surface water for the evolution of thermal physiology remain largely unexplored. We hypothesized that avian thermoregulation in the heat has evolved in tandem with the use of surface water and predicted that a) regularly-drinking species have a greater capacity to elevate rates of evaporative water loss (EWL) compared to non-drinking species, and b) heat tolerance limits are higher among drinking species. To test these predictions, we quantified thermoregulatory responses to high air temperature (Ta) in 13 species of passerines from the South African arid zone and combined these with published values for an additional four species. We categorized each species as water-dependent, regularly-drinking or water-independent, occasional-/non-drinking. In support of our predictions, both conventional and phylogenetically-independent analyses revealed that evaporative scope (the ratio of minimum thermoneutral evaporative water loss (EWL) to maximum EWL) during acute heat exposure was significantly higher among drinking species. Moreover, evaporative scope was significantly and positively related to heat tolerance limits (i.e., maximum air temperature tolerated before the onset of severe hyperthermia). These findings offer new insights into the coevolution of avian diet, movement ecology and thermal physiology in the context of trade-offs between dehydration avoidance and resistance to lethal hyperthermia in hot desert habitats.}

\normaltalk{Till Death Do Us Part: Mate Mortality Predicts Breeding Dispersal in Canada Jays}{Matthew Fuirst, Dan Strickland, Ryan Norris}{Dispersal in animals can be shaped by both environmental and social constraints. Among adults, breeding dispersal provides individuals the opportunity to acquire a new mate or territory. However, dispersing does not guarantee improvement in future reproductive success and the mechanisms behind breeding dispersal remain poorly understood in vertebrates. We examined the causes and patterns of breeding dispersal within a marked population of Canada jays (Perisoreus canadensis) that have been monitored since 1964 in Algonquin Provincial Park, Ontario. Across all years of the study, breeding dispersal occurred 64 times, which was 3\% of all potential breeding opportunities. We used generalized linear mixed models (GLMM) to examine the relative contribution of explanatory factors affecting breeding dispersal and dispersal distance and used linear models to quantify whether the frequency of dispersal events has changed over time. Dispersal tended to occur most frequently when an individual's partner disappeared, but this effect only occurred when the disappearing partner was older. Dispersal distances ranged from 500 m-35 km (mean ± SD: 2 ± 4), but no variables significantly influenced breeding dispersal distance. The number of dispersal events each year also significantly increased over time. Our results suggest that the primary motivation for dispersal in Canada Jays occurs when their mate dies, particularly when their mates are older. Thus, social factors rather than environmental factors seem to shape dispersal strategies in adults.}

\normaltalk{The Role of Plumage Coloration in the Diversification of High Elevation and High Latitude Finches}{Erik R Funk, Nicholas A Mason, Garth M Spellman, Kristen C Ruegg, Kevin Winker, Scott A Taylor, Erika Zavaleta, Jack J Withrow, Snæbjörn P\'{a}lsson, Tom\'{a}š Albrecht}{Variation in color and pattern make birds among the most colorful of all vertebrate classes. Differences in avian plumage color may inhibit species recognition and act as barriers to reproduction. In some systems, color differences may generate rapid barriers to reproduction; however, the application of whole genome data has revealed that, in many systems, speciation often occurs despite ongoing gene flow. Here, we present whole genome data on rosy-finches (Leucosticte spp.) and redpoll finches (Acanthis spp.), two radiations of high elevation and high latitude finches that exhibit plumage variation despite low levels of genetic variation. While rosy-finches exhibit population genetic structure aligned with morphological variation, redpolls exhibit an absence of population genetic structure across the entire Holarctic. The disparate patterns of population genetic structure between rosy-finches and redpolls suggests that isolating barriers between plumage types may be stronger in rosy-finches than in redpolls. While these differences may be due to the strength of assortative mating related to color variation, or other processes such as local adaptation or drift, we also find differences in the genetic architecture underlying each of the coloration traits. The genetic control of patch color variation in rosy-finches appears to be quite modular, with independent color patches controlled by discrete genomic regions, while phenotypic variation in redpolls appears to be controlled by a 55Mb inversion.}

\normaltalk{Twenty-Five Years of Persistence and Change in the Naturalized Parrots of Southern California}{Kimball L Garrett}{Naturalized populations of non-native bird species can show wildly variable trajectories which may include explosive population growth, stasis, and sometimes sharp declines to extirpation. Urban southern California has been a hotspot for naturalized parrots since at least the mid-1970s, with some dozen species known or suspected to breed and at least eight species maintaining populations $>$100 individuals (some $>$$>$1000). These include globally significant numbers of four species of high conservation concern in their native range (Psittacara erythrogenys, Amazona viridigenalis, A. finschi and A. oratrix). Based on population estimation efforts in the mid-1990s and an increasingly robust eBird database, I report here on current status and population trajectories of southern California parrots and discuss data deficiencies. Currently our best information relates to population numbers, geographical range, gross habitat associations, and common food items.  Important information gaps include nearly all other aspects of breeding biology, metapopulation structure, hybridization among ``artificially sympatric" species, specifics of resource use, and interspecific interactions both among parrot species and with native species. Community science has long been an important component in parrot population studies in urban regions, and eBird and iNaturalist have accelerated this participation in recent years. Building input from birders and the wider community will be fundamental to future studies of urban parrots as well as other naturalized bird species.}

\normaltalk{Modeling Species Interactions Using Spatial Capture Recapture}{Heather E Gaya, Richard B Chandler}{Patterns of species co-occurrence are often important for understanding species-level processes such as species diversity, population dynamics or niche overlap. Using co-occurrence data to Infer the relative roles of habitat characteristics and interspecific competition on species distributions can be difficult because competition arises as an individual-level process that can occur at fine spatial scales. To quantify the drivers of species co-occurrence, we present a two-species spatial capture-recapture model that includes a Markov point process in which an individual's location is dependent upon both landscape level covariates and the locations of individuals of other species. We applied the model to data on two ecologically similar songbird species -- Hooded Warbler and Black-throated Blue Warbler -- that segregate over a climate gradient in the Appalachian Mountains of North Carolina.  In spite of many ecological similarities between the two species, we found minimal evidence of competition between the Hooded Warbler and Black-throated Blue Warbler at our study sites. Rather, spatial variation in density of the two species was much better explained by climate variables. Unlike previous statistical models that attempt to infer competition from species-level co-occurrence data, the framework proposed here allows for inference at the individual-level and can be used to assess the spatial scale of biotic interactions.}

\normaltalk{Differential Behaviour and Heart Rate Physiology Responses to Familiar and Unfamiliar Predators in an Arctic Seabird}{Erica A Geldart, Oliver P Love, H Grant Gilchrist, Chris M Harris, Christina A Semeniuk}{Changing sea-ice phenology is altering Arctic food-web dynamics, with prey being exposed to predators they have not encountered in their evolutionary past. In particular, earlier sea-ice breakup has created a shortened seal-hunting season for polar bears (Ursus maritimus), resulting in their recent terrestrial foraging on alternative diet items such as seabird eggs. I study common eiders (Somateria mollissima) on Mitivik Island, Nunavut, Canada, a large seaduck colony that is exposed to established nest predators such as arctic fox (Vulpes lagopus), but has recently experienced an increase in polar bear nest predation as a consequence of the bear's climate-induced reduced spring diet. What is unknown is whether eiders differ in the degree to which they perceive, respond to, and expend energy in responding to bear presence in comparison to more evolutionarily familiar ones. Using artificial-egg heart rate monitors, we quantified eider hen responsiveness -- incubation behaviour (i.e, flight initiation distance and return-time to nest latencies) and underlying physiology (i.e., heart rate) -- across different predator types. Comparisons were made by quantifying responses during a standardized threat-simulation protocol where I simulated a predator approach that differed in familiarity: polar bear, arctic fox, and a control stimulus. Our preliminary results indicate eider hens are least responsive to bear predation threats than to both fox and control. This study allows us to determine whether eider hens incur any energetic consequences by responding to polar bear nest predation to anticipate its fitness costs and any downstream population effects.}

\normaltalk{Genomic Data Reveal Demographic History of over 300 Avian Species under Past Climate Change}{Ryan R Germain, Shaohong Feng, Carsten Rahbek, Gary R Graves, Guojie Zhang, David Nogu\'{e}s-Bravo}{Human-induced changes to global climate are affecting biodiversity at an unprecedented rate. Determine how contemporary populations will respond to climate change in the near future requires understanding the demographic history of species over paleo-ecological time scales of intense global climate fluctuations. Modern genomic tools allow for unprecedented insight into the demographic history of species via patterns of historical recombination events. Such molecular `log books' reveal changes in effective population size (Ne) over thousands-to-millions of years that can be cross-referenced with paleo-records to infer demographic change during periods of abrupt climate warming or cooling. We use full-genome sequence data from 343 bird species representing all major families to identify common demographic responses to climate fluctuations by zoogeographic realm, habitat type, and current conservation status. We further test a suite of species-specific traits to identify those associated with demographic sensitivity to past climate change, with the aim of determining the role of life-history variation in adapting to changing environments. The majority of species experienced their most dynamic demographic changes during the Last Glacial Period (~25--115kya), with tropical species experiencing the sharpest declines in Ne due to shrinking availability of suitable habitat. Species with higher fecundity, shorter life-spans, and smaller body sizes showed greater resilience to climate cooling during this period, and those of highest current conservation concern were among the most severely affected over this period of paleo-history. Our results reveal dramatic patterns of global demographic change due to past climatic events, and highlight how species history should inform conservation planning.}

\normaltalk{Asian Woollynecks in Nepal: Distribution, Ecology and Conservation}{Prashant Ghimire, Nabin Pandey, Rojina Ghimire, Bhuwan S Bist, Bibek Belbase, Aarati Basnet, Yajna P Timilsina}{Asian Woollyneck Ciconia episcopus, although a vulnerable stork, little is known about its status ecology and conservation. In 2016, we initiated research and conservation project to understand distribution, nesting and foraging ecology of Asian Woollyneck in Nepal. 7 permanent road transects in western lowlands (Rupandehi and Kapilvastu districts) were surveyed from 2016 to 2019 where 38, 41, 46, 48 Asian Woollyneck were observed respectively. In addition to lowlands, nesting site exploration was done in mid-hills (Dhading, Pyuthan, Arghakhanchi and Salyan Districts) where 14 nests have been recorded which were very close to settlements (66.31 ±18.49 m) and road (155 ± 56.08 m) posing diverse threats. Out of 77 districts of Nepal, at least one sighting had been made from 66 districts with wider distribution. Asian Woollyneck spent more time on maintenance (35.18\%) followed by foraging (32.12\%), with relatively little time on vigilance (10.32\%). Asian Woollyneck were more likely to forage farther distance to roads (P $<$ 0.05) and settlements (P $<$ 0.05). Vigilance behavior and disturbance area found associated (P = 0.096 at 90\%) with high disturbance being higher (12.97\%) aiding species to forage in human dominated landscapes. Asian Woollyneck coped with increased human disturbances by altering activity budgets in minor ways. Moreover, Conservation campaigns were conducted in 50 schools and 20 social groups with distribution of posters (n = 1000) and brochures (n = 2000). Positive change has been observed at local communities inferring that consistent efforts at local level can aid in species and ecosystem conservation.}

\normaltalk{The Colombia Resurvey Project: Evaluating 100+ Years of Changes in Bird Assemblages Since the American Museum of Natural History Expeditions}{Camila Gomez, Elkin Tenorio, C D Cadena}{Between 1910 and 1915, Frank M. Chapman led a series of expeditions resulting in thousands of bird specimens deposited in the American Museum of Natural History, as well as detailed field notes documenting a good proportion of Colombia's bird species and their status at the time. This invaluable treasure now poses an opportunity: by developing methods to make these historical data comparable with contemporary records, we have the possibility of evaluating changes in Colombia's avian assemblages over the past 100 years. One of the questions we are tackling, is how bird assemblages can serve to assess ecosystem health through time? Using exhaustive bird surveys from four time periods and functional traits of 234 species, we quantified shifts in the functional space of a montane forest site in Colombia, which has gone through extensive fragmentation and also some forest regeneration over 100 years. We found that the functional space at this site could be described using three major functional axes of body size, dispersal ability and habitat breadth. As a result of species extirpation and colonization, there have been shifts in this functional space over time. In spite of the recovery of some species after forest regeneration, the original functional state of the ecosystem has not been restored, and new additions to the avifauna have shifted functional space towards smaller size, higher dispersal abilities and wider habitat breadths. We argue that healthy ecosystems can be identified by their functional fingerprints and that shifts resulting from loss or gain of species reflect changes in ecosystem function.}

\normaltalk{Switches in Migratory Behavior and Speciation in Tyrant Flycatchers (Tyrannidae)}{Valentina G\'{o}mez}{Migratory bird species are often sister to year-round residents in the evolutionary tree of life. In theory, switches between migratory behavior and residency may result in the formation of new species because of correlated evolution among morphology, physiology and behavior. Comparative phylogenetic analyses in Tyrant Flycatchers suggest that migration has been lost more often than gained. These evolutionary transitions pass through an intermediate state in which lineages have both migratory and resident populations. Microevolutionary evidence supports this pattern showing that in Fork-tailed Flycatchers (Tyrannus savana), loss of migration has promoted shifts in breeding schedules, which coupled with geographic isolation of breeding populations has led to the evolution of premating isolation and correlated morphology. Migrant and resident birds differ in traits important for flight, such as wings and tails. In many Tyrant Flycatchers these traits are also used during displays. For instance, outer primary wing feathers differ in shape across Tyrannidae, and it has been suggested that they produce non-vocal acoustic signals. In Fork-tailed Flycatchers, outer primary wing feathers differ in the shape of a notch, which is significantly narrower in migratory birds. Behavioral studies using audio recordings and synchronized high-speed videos show that these feathers produce diverging non-vocal acoustic signals differing in pitch and amplitude. Moreover, flight simulations suggest that migratory feathers produce less turbulence when compared to those of year-round residents, suggesting that aerodynamic pressures may also affect shape evolution. Thus, loss of migration may indirectly influence conspecific communication by morphological evolution associated with flight efficiency, or vice versa.}

\normaltalk{Survival and Spatial Ecology of the Endangered Yellow-Shouldered Blackbird During the Post-Fledging Period.}{Jean P Gonzalez-Crespo, Alberto R Puente-Rolon}{The Yellow-shouldered Blackbird (Agelaius xanthomus) is an endangered species that is endemic to the archipelago of Puerto Rico. This species could be found throughout the main island of Puerto Rico, but due to habitat loss it is currently restricted to the municipalities of Ceiba, Salinas, Lajas, Cabo Rojo and Mona Island.  With a population of approximately 500 individuals, the population located on the Southwest of Puerto Rico ( Lajas and Cabo Rojo munipalities) is the largest and possibly healthiest of all. However, this population, which peaked at nearly 1000 individuals in the past, has seen a steady decline in population size since 2003; something which is believed to have been further aggravated by the impact of hurricane Maria. Apart from the main threats that affect the Yellow-shouldered Blackbird (i.e. habitat loss and alteration and brood parasitism by the Shiny Cowbird), it has been suggested that high mortality during the post-fledging period could be hindering recruitment. Therefore, insight on this life stage must be gained in order to establish more efficient conservation strategies. For our research, we monitored the activity of 29 fledglings, using radio telemetry, during the 2019 breeding season. Using the Kaplan-Meier Estimator, we were able to determine a survival rate of 0.75, which suggests that species isn't as vulnerable during this life stage. Additionally, we observed that mangrove forests serve as a critical habitat, during the post-fledging period.}

\normaltalk{Human Impacts on Avian Communication: Breaking Down the Source and Structure of Anthropogenic Noise Pollution to Inform Conservation Strategies}{Erin E Grabarczyk, Dominique A Potvin, Sharon A Gill}{Anthropogenic noise pollution overlaps with acoustic signals birds use to communicate.   Noise masking may limit acquisition of information, resulting in missed signals that could lead to maladaptive responses. In turn, noise may decrease reproductive success and survival and result in extirpation from otherwise suitable habitats. This begs the question, can noise mitigation strategies improve habitat quality to support diverse avian communities? Human activities may differentially affect patterns of communication, depending on the form and function, and plasticity and adaptability, of the signals being used. Furthermore, the behavioral responses of individuals may differ depending on the source and structure of noise disturbance at the location of signaling. Continuous noise sources, such as building generators, emit sounds from a stable location, leading to constant signal masking with predictable patterns of decay over space. In contrast, intermittent bursts of noise (i.e. passing cars, air brakes, sirens, trains, leaf blowers, etc.) are high-amplitude, cover a broad frequency range, and may occur unpredictably. Therefore, conservation strategies aimed at noise management may require a nuanced approach, such that some areas of a given site may require mitigation, whereas others may not. To highlight the complexities of noise pollution, we share research that quantifies heterogeneity of anthropogenic noise within and between natural areas and the responses of birds that inhabit them.}

\normaltalk{The Thermal Performance of Worn House Sparrow (Passer Domesticus) Feather Coats is Even Worse than you Think.}{Jordana M Graveley, Kevin R Burgio, Margaret Rubega}{Feathers are critical to the thermal state and energy balance of birds. The feather coat, as a whole, insulates birds by trapping a still layer of air next to the skin and acts as a physical barrier to heat loss. Despite previous attention to the energetic cost of maintaining core temperatures, relatively little work has focused on the thermal contribution of the feather coat, alone, as opposed to strategies by birds to conserve body heat through physiological and behavioral responses. Moreover, to our knowledge, no studies have directly measured the effect of feather wear and loss through the annual molt cycle on the thermal performance of the feather coat. To address this, we developed and used a new technique to quantify heat loss through a flat-skin by measuring the temperature at the surface of the feather coat with a thermal camera. We measured the thermal performance of the feather coat in male house sparrows collected pre-molt (worn plumage) and post-molt (unworn plumage). We expected worn feather coats to lose more heat than those with unworn feathers; nonetheless, the magnitude of heat loss through worn feather coats was still surprising. Worn House Sparrow feather coats lost as much as 23°C of the heat applied to their undersides at a constant temperature of 40°C. The thermal balance of living birds will be better understood when the contribution of the feather coat, itself, is directly measured across more species and conditions.}

\normaltalk{Odour-Based Discrimination of Sex and Major Histocompatibility Complex Genotype in Songbirds}{Leanne A Grieves, Elizabeth A MacDougall-Shackleton, Mark A Bernards, Greg Gloor}{Oily secretions from the preen gland contain volatile, odour-producing compounds. Thus, preen oil may function as a source of chemical cues that could mediate avian olfactory communication. Chemical cues in preen oil could be useful as reproductive cues if they convey information about sex and genetic quality and compatibility. The major histocompatibility complex (MHC) is a polymorphic gene family that plays a crucial role in vertebrate immune defense. Individuals with high MHC allelic diversity can respond to a broader suite of pathogens, and selection should favour the ability of individuals to assess the MHC genotype of potential mates. We used gas chromatography to characterize the preen oil chemical composition of adult song sparrows and tested for differences between the sexes and for correlation of preen oil composition with MHC genotype. Next, we tested for olfactory discrimination of sex and MHC-genotype by comparing the amount of time song sparrows spent with preen oil odour from males versus females, and from MHC-dissimilar versus MHC-similar and MHC-diverse versus less MHC-diverse potential mates. Preen oil differed between the sexes and showed positive covariation with MHC genotype. Song sparrows spent more time with preen oil from the opposite-sex, and from MHC-dissimilar and MHC-diverse potential mates. We conclude that song sparrows, and presumably other passerines, can use preen oil odour to discriminate sex as well as the MHC similarity and diversity of potential mates.}

\normaltalk{Habitat Use of Juvenile Baird's and Grasshopper Sparrows in the Northern Great Plains}{Nicole A Guido, Maureen D Correll, Brian J Olsen, Jacy Bernath-Plaisted, Arvind O Panjabi, Katharine J Ruskin}{Habitat loss and alteration are linked to population decline in grassland birds, but there is limited knowledge of how juvenile grassland birds use habitat during the post-fledging stage. Understanding how birds use habitat during this life stage is essential for developing effective management strategies to lessen and reverse decline. We collected juvenile habitat use data for two grassland specialists, Baird's sparrow and grasshopper sparrow, in the Northern Great Plains, USA. We analyzed post-fledgling habitat use with variables measured from the ground and from spectral data collected via Unmanned Aircraft System at juvenile used points, random points, and adult nest sites to identify habitat conditions specified to the post-fledge stage. We found that both species selected for high forb cover and that juvenile Baird's sparrows moved towards densely vegetated areas (e.g. wetland areas) after they leave the nest. Patterns of selection of dead grass cover, grass height, and exotic vegetation varied between species but were also influential in juvenile habitat selection. We found that juveniles of both species selected for habitat cover types that differed substantially from those present at nest sites. We demonstrate that habitat use varies between different life stages within the breeding period and between species of juvenile grassland specialists co-existing in the Northern Great Plains. Generally, we emphasize consideration of all life stages when developing a management plan for a certain area. Particularly, we present a novel recommendation that wetland areas be considered for the management of Baird's sparrows on breeding grounds in mixed-grass prairies.}

\normaltalk{Numerical Abilities in a Brood Parasite, the Brown-Headed Cowbird}{M\'{e}lanie Guigueno, Justin Arthur, Alex Coto, David F Sherry}{Species in which the reproductive roles of males and females differ can be valuable for understanding the evolution of cognitive abilities. In the brood-parasitic brown-headed cowbird, Molothrus ater, females are responsible for selection and parasitism of suitable host nests. This task would be facilitated by enhanced numerical abilities, which can be used to distinguish suitable host nests in the process of laying eggs from unsuitable nests that have begun incubation. We trained eight male and seven female brown-headed cowbirds to use touchscreens and discriminate between pairs of stimuli differing in number. On alternating days of egg and cone shaped stimuli, cowbirds were tested using combinations of 1-6 objects. In touchscreen numerosity discriminations, cowbirds were able to distinguish images on the basis of number alone. Although both sexes learned at the same rate and their performance was similar when there were relatively few eggs, only females were able to discriminate five eggs from six eggs. Numerical abilities in female cowbirds may be specialized for a brood-parasitic mode of reproduction.}

\normaltalk{Variation in Scale-Dependent Habitat Selection by Great Gray Owls (Strix Nebulosa)}{Katherine Gura, Bryan Bedrosian, Susan Patla, Anna Chalfoun}{Habitat selection can differ across spatial and temporal scales, in response to varying selective pressures.  Studies that explicitly incorporate multiple relevant spatiotemporal scales therefore strengthen understanding of important resources for a species.  We quantified the resource selection of Great Gray Owls across multiple spatial (home range and within-home-range) and temporal (breeding and non-breeding seasons; day and night) scales in western Wyoming, USA.  Between 2017 $<$U+0336$>$ 2019 we outfitted owls (n = 27) with GPS transmitters that collected hourly location data.  We created Resource Selection Functions using Generalized Linear Mixed Models and remotely-sensed data, and we identified top models using a backwards, step-wise approach based on AIC value.  We observed different patterns of resource selection between the varying scales at which we measured it, underscoring the importance of multiscale assessments.  Owls tended to place breeding home ranges within areas that were predominantly forested with northerly aspects, and they avoided developed and herbaceous habitats.  During the non-breeding season, we observed a switch in favor of developed areas, wooded and herbaceous wetlands, and southerly aspects suggesting that developed, riparian forest and south-facing slopes provide important winter habitat.  Within home ranges, owls avoided herbaceous wetlands during the day but selected strongly for them at night, indicating context-dependent habitat selection and that open wetlands comprise important foraging areas.  Determining resource requirements at multiple scales can facilitate more effective conservation of Great Gray Owls and their resources.  In general, habitat selection studies that incorporate scale dependence can lead to more accurate inferences about species-habitat relationships.}

\normaltalk{Linking Movement Ecology, Body Condition, and Physiology to Better Understand Stopover Ecology along the Gulf Coast}{Mariamar Gutierrez Ramirez, Joely G DeSimone, Michael S Griego, Cory R Elowe, Alexander R Gerson}{After migratory flights, birds require stopover periods to rest and replenish depleted energy reserves. Migratory birds use fat for energy during long flights, but they also use lean tissue, which can result in significant reductions in muscle and organ masses. Rebuilding lean mass is slow and energetically costly, and must occur before fat stores can be replenished. Therefore, the loss of lean mass during migration can impose physiological limitations that prolong stopover periods and delay overall migration time. We investigate how body condition after crossing the Gulf of Mexico affects stopover duration and refueling rate. We predict that birds with lower lean mass will require longer stopovers to regain reserves needed to resume migratory flights; alternatively, birds may compensate by increasing refueling rate. During spring migration 2016-2018 we banded songbirds on a barrier island in Apalachicola Bay, Florida. We used plasma metabolites and Quantitative Magnetic Resonance to accurately and non-invasively measure body condition of spring migrants on arrival. We used the Motus Wildlife Tracking System to track four long-distance migratory species (N = 82) from stopover with a regional array covering 5800 km2. Local stopover was influenced by body condition and differed annually and interspecifically. Northern Waterthrush with lower fat and lean masses stayed longer on the island. Refueling rate was greater in birds with greater fat mass, but was unrelated to lean mass. High refueling rate shortened local stopover duration for Yellow-billed Cuckoos. Our analysis will expand beyond local stopover to incorporate regional stopover within the Apalachicola Bay.}

\normaltalk{Crossing the Ocean: Gene Flow Between Sedentary Island Bird Populations}{Ethan F Gyllenhaal, Xena M Mapel, Jenna M McCullough, Robert G Moyle, Michael J Andersen}{Islands have provided a disproportionate amount to the development of allopatric speciation theory, because myriad populations are separated by open-water gaps. These isolated populations are often genetically distinct, but little is known about the dynamics of gene flow between them. To examine gene flow between island populations, we performed population genomic studies in three groups of sedentary Pacific island birds. First, we found that the widespread honeyeater Foulehaio carunculatus formed distinct populations across short distances, but gene flow still connects distant archipelagos. Second, we found that strikingly distinct species of the Symposicharus monarchs interbred in multiple instances when they came into contact both on islands and due to Pleistocene sea-level changes. Finally, we tested how levels of gene flow vary relative to island biogeographic predictions with the Fiji Bush Warbler Horornis ruficapilla. The four island populations were genetically distinct, but asymmetric gene flow occurred between them. Gene flow was higher from larger to smaller islands than vice-versa, and was negatively correlated with distance. Using a framework to compare estimates of gene flow to predictions from classic island biogeography, we assessed which gene flow edges deviated from expectations. We inferred fewer birds moving against the trade winds between the two large, distant islands. We also found gene flow from small to large islands was reduced beyond neutral expectations. This analytical framework is an important development to allow the study of island biogeography on a population genetic level and will facilitate study of factors that hinder gene flow between island populations.}

\normaltalk{Species Limits and Population Genetics of the Tepui Vireo}{Matthew D Hack, Edgar Benavides, Richard O Prum}{The Tepui Vireo, Vireo sclateri, is a poorly studied Neotropical songbird that is patchily distributed in montane forests above 500 meters on isolated tepuis from Mt. Duida in southern Venezuela to Roraima in eastern Venezuela and northwestern Guyana, southern Guyana, and Tafelberg and Wilhemena in Suriname. Prior phylogenetic studies have not sampled from multiple populations of V. sclateri. A detailed examination of the phylogenetic, population genetic, and vocal diversity of Vireo sclateri will contribute to our understanding of avian diversification among isolated tepuis. We sequenced one mitochondrial and two nuclear genes from 21 specimens of V. sclateri from populations in Venezuela, northern and southern Guyana, and two localities in Suriname. We also obtained double digest restriction-site associated DNA sequencing (ddRAD-seq) data on the same geographic sample of specimens. Lastly, we conducted a quantitative analysis of 205 vocalizations from 35 individuals from across the range, including newly discovered populations in northern Brazil. We find that individuals from Suriname and southern Guyana form a distinct clade from populations in Venezuela and northern Guyana. A discriminant function analysis of vocal variables supports clear behavioral differences between these clades. There is no precedent of considering these populations as separate species, but the cumulative molecular and vocalization results strongly suggest the existence of two cryptic species in V. sclateri. These biogeographic patterns in V. sclateri mirror the distributions of the sister species Lepidothrix serena and L. suavissima (Prum 1994), which may indicate generalized pattern of avian differentiation among tepui regions.}

\normaltalk{The Role of Eggshell Pigments in Signaling and Thermoregulation}{Daniel Hanley, Phillip A Wisocki, Patrick Kennelly, Indira Rojas Rivera, Phillip Cassey, Mark L Burkey, Mark E Hauber}{Birds' eggshells vary in color from blue-green through white-beige to brown, and these colors are produced by two known eggshell pigments. Museum collections are vital for establishing the egg color gamut, which has advanced our understanding of the information encoded on birds' eggs. Accordingly, compared to other avian coloration (e.g., bare skin patches, feathers), bird eggshell colors represent $<$1\% of the avian color space (defining the colors that birds can perceive). Eggshell coloration can covary with the quality of the habitat, laying order, and maternal quality, and may provide cues for egg recognition. However, eggshell colors can also provide physical and chemical benefits, including increasing structural strength, combating surface microbes, protecting the embryo from harmful ultraviolet rays and helping to regulate the egg's temperature (and that of the of the developing embryo). Such thermal benefits of eggshell colors have been a long-standing and contested question in ecology. To address this, we experimentally demonstrated that darker eggs heat more rapidly than lighter ones when exposed to solar radiation. Then, using museum-based reflectance data from the full phylogenetic diversity of birds that range across the entire globe, we found that birds living in colder habitats have darker eggs, especially in species with exposed nests. These data suggest that eggshell colors are important for thermoregulation in colder climates, but that a range of competing selective pressures impact eggshell colors in warmer regions. Museums with egg collections contain a vast wealth of data that is ideal for addressing fundamental questions in ecology and evolution.}

\normaltalk{Applying Finite Element Analysis to Understand the Functional Anatomy of the Skull of an Ancient Bird}{Michael Hanson, Philip Morris, Bhart-Anjan S Bhullar, Jen A Bright}{Cranial kinesis, the ability to move the bones of the upper jaw independently of the braincase, is an anatomical novelty common to modern birds, and a key innovation allowing birds to use their bills as dexterous substitutes for hands. Two forms of kinesis characterize the deepest evolutionary split among living birds: Palaeognathous birds exhibit rhynchokinetic rostral bending, where the frontal process of the premaxilla bends along much of its length, while neognathous birds typically exhibit prokinetic bending, with flexion occurring in a highly restricted region immediately in front of the orbit. The condition preceding the origin of these two different modes of flexing the upper jaw so far has remained elusive, but may have important implications for the vast difference in species diversity between these two clades. To resolve the condition preceding the origin of crown Aves, we performed finite element analysis (FEA), on a 3D digital reconstruction of the skull of the Mesozoic toothed bird Hesperornis regalis, with a model informed by data from extant taxa. Our results show that the upper bill of Hesperornis had a bending distribution comparable to modern prokinetic birds, despite its otherwise archaic facial and palatal skeleton, suggesting that the ancestral form of avian kinesis was mechanically comparable to that of prokinetic neognaths despite the absence of some typical features associated with prokinesis. Furthermore, our study provides a methodological framework for addressing and answering questions about form and function using FEA, especially in cases where the original biological material is no longer accessible.}

\normaltalk{Secondhand Homes: Varying Success of Secondary Cavity Nesting Birds in Relation to a Local Ecosystem Engineer}{Faith Hardin, Jacquelyn Grace, David Cairns, Tyler Campbell, Michael Morrison}{Ecosystem engineers mechanically change their environment which has a direct impact on local species. Woodpeckers serve this role as their cavities provide obligate nesting sites for secondary cavity-nesting birds. Additionally, as woodpeckers forage for insects they increase insect habitat heterogeneity which has been shown to be positively correlated with insect abundance. This increase in insect abundance may provide additional foraging opportunities for nesting birds leading to increases in abundance and species richness in the avian community. The goals of this project were to (1) determine if any insect orders were positively correlated within active woodpecker territories (2) model nest cavity characteristics that influenced both woodpecker and secondary cavity nesting bird's clutch sizes. Data was collected from 2018 to 2019 on East Foundation lands in south Texas. Abundances of insect orders Coleoptera, Orthoptera, and Diptera were higher in areas with an active woodpecker territory (n= 89) when compared to areas without (n = 89). The decay state and diameter of the nesting tree were significant predictors of woodpecker nesting success, and while predictors varied between species of secondary cavity nesters, all species had higher success when nesting within an abandoned woodpecker cavity compared to a naturally occurring one. Considering that secondary cavity nesters are in decline worldwide, understanding factors that influence not only them, but the primary excavators they rely on will allow us to create more informed management plans and develop biologically relevant studies.}

\normaltalk{Can You Hear Me Now? a Test of the Acoustic Adaptation Hypothesis in Seven Wren Species}{Braelei Hardt, Lauryn Benedict}{Many avian taxa produce long-range songs for the purpose of communication. These signals play an important role in mate attraction and territory defense, which are key behaviors for reproductive success. The Acoustic Adaptation Hypothesis (AAH) posits that long-range acoustic signals are adapted for minimized degradation in the environment in which they are produced, leading to optimized communication within preferred habitats. While previous experiments have found some evidence for the AAH in `open' vs. `closed' habitats, this remains contentious due, in part, to a lack of large-scale comparative experiments. We conducted playback experiments using songs of seven wren species with unique habitat preferences in seven different habitats, testing whether a species' song degrades less 1) in its preferred habitat, 2) during the breeding season, and 3) when compared to songs of the same species sourced away from target habitats. We measured three common forms of degradation--excess attenuation, blur ratio, and signal to noise ratio, and found partial support for the AAH in only three species. The effects of habitat and season were only present in the desert and canyon habitats despite significant differences in structure in most habitats, indicating that structure does not influence signal propagation as intensely as previously thought across species. Interestingly, only species who remain resident in breeding areas throughout the year seem to adapt their songs to seasonal differences. Our study provides a robust test of the AAH, and shows that its principles are extremely species and context dependent.}

\normaltalk{Standing Genetic Variation in Clock Genes Underlie Phenology Patterns of American Kestrels}{Julie A Heath, Christen Bossu, Sara E Simmonds, Kristen C Ruegg}{There have been dramatic shifts in migration and breeding phenology of migratory birds in response to climate change. Heterogeneity in phenology shifts, both within- and between-species, remains poorly understood.  We studied whether standing genetic variation in genes associated with the circadian ``clock" mechanism explained phenology patterns of American kestrels (Falco sparverius), a widespread falcon showing different responses to climate change across its range.  We used RAD-seq data from 287 kestrels sampled across North America to identify genetic variants in previously identified candidate genes associated with circadian rhythms. We identified 10 single nucleotide polymorphisms (SNPs) within candidate genes and then screened more than 1400 kestrel DNA samples, collected across the North American range and throughout the year, at these SNPs. We used an ordinal principal component analysis to examine covariation in genotypes.  The first component explained 14\% of the variance and was based on covariance between PEAK1 and TOP1 genes.  Principal component scores based on PEAK1 and TOP1 genotypes were associated with the timing of nesting and migration passage.  Interestingly, kestrels from eastern North American, where breeding phenology has not changed or has gotten later, had significantly less variation in PEAK1 and TOP1 genotypes compared to western kestrels, where breeding phenology has advanced.  These results support the hypothesis that the genetic composition of populations plays an important role in facilitating responses to climate change. Specifically, low genetic variation may constrain phenology shifts and reduce adaptive capacity, potentially leading to population vulnerability from phenological mismatch.}

\normaltalk{Evolutionary History and Community Assembly in Diurnal Birds of Prey}{Shanta R Hejmadi, Keith Barker}{Research into the evolutionary and ecological processes driving community composition typically focuses on local to regional scales, with little to no investigation of globally distributed groups. The avian orders Accipitriformes and Falconiformes, two independent clades of diurnal raptors, present an opportunity to study how community assembly across continents, latitudes, and habitat types might vary in ecologically similar groups. My work combines evolutionary and ecological lines of inquiry, using a dataset of linear measurements from $>$1100 specimens, range maps, and a previously published phylogeny, to explore phylogenetic, morphological, and biogeographic patterns in raptor communities across the world. Despite deep evolutionary divergences, these two groups show significant overlap in morphospace, with Falconiformes sharing nearly half of its total morphospace with Accipitriformes, and Accipitriformes sharing about one-fourth of its total morphospace with Falconiformes. This raises the question of how two functionally similar clades could both expand to worldwide distributions. Preliminary biogeographic studies suggest mid-late Miocene grassland expansion may have influenced both accipitriform and falconiform diversification. At a global level, phylogenetic and trait dispersion in communities show clear differences.  While most communities are not significantly clustered or overdispersed in either trait or phylogenetic composition, those that are exhibit opposite patterns; areas with high concentrations of phylogenetic overdispersion also exhibit trait clustering, and areas with high concentrations of trait overdispersion exhibit phylogenetic clustering. Further analyses will interrogate these patterns, elucidating the processes behind raptor community assembly around the world.}

\normaltalk{Avian Usage Trends of Reclaimed Boreal Forest Habitat in Canada's Oil Sands}{Virgil C Hawkes, Nathan T Hentze, Travis G Gerwing}{The Athabasca Oil Sands Region, located in northeastern Alberta, Canada, supports open-pit bitumen mining operations. Such development has direct impacts upon the environment; however, concurrent habitat reclamation is underway, providing an opportunity to assess the efficacy of reclamation practices relatively early in the reclamation process and to study wildlife usage patterns. We implemented a monitoring program to assess the re-establishment of wildlife in reclaimed upland areas relative to desired reclamation end points (mature forest). We analysed bird detection data from point counts conducted between 2012 and 2019 on six oil sands leases and a range of habitat treatment types. Bird communities in reclaimed (soil disturbed) plots were most similar to cleared (no soil disturbance) plots (78\% similarity), and least similar to logged (no soil disturbance) plots (50\% similar). Avian communities in our oldest reclaimed plots (35 years) varied from 20-65\% similar (average ~43\%) to mature forest plots. Plots disturbed by other activities, such as stand-replacing fire and logging, supported bird communities that were more similar to older reclaimed plots than they were to same-age reclaimed plots based on NMDS results, suggesting that the type of disturbance influences the temporal aspects of the recovery trajectory. Continued research is required to assess the functionality of reclaimed habitats and to determine what level of community similarity between reference and reclaimed plots is satisfactory. These data contribute to the understanding of reclamation success for birds using human-altered environments in Canada's boreal forest.}

\normaltalk{Shorebird Migration on the Gulf of Mexico, Linking Wetlands Across America}{John A Herbert, Caz M Taylor}{Coastal wetlands along the northern Gulf of Mexico (NGOM) provide essential habitat for migratory shorebirds, yet little is known how the landscape affects habitat use and migratory behavior. The semipalmated sandpiper (SESA, Calidris pusilla) is a near-threatened migratory shorebird that winters in South America and breeds in the Arctic, but also uses NGOM coastal habitat during migration. Using nanotags and Motus towers, we are analyzing what affects SESA spring migration while using stopover habitat in the NGOM. We are also connecting the NGOM to breeding, wintering and other migratory stopover habitat. From 2017-19, we deployed 104 nanotags to SESA in coastal Louisiana. We found stopover duration ranged from 2-22 days and average residence time was 7.8 days. We found that sites with more foraging habitat had longer stopover duration, suggesting that locations with limited resources (e.g. barrier islands) may have to depart sooner. We also found males had a shorter stopover duration than females, and individuals who breed in the eastern portion of the breeding range had longer stopover durations. Additionally, we have recorded SESA at over 100 different sites throughout the western hemisphere, and individuals using the NGOM during spring migration are using all of the North American flyways, which was previously unknown. Information gained in this research will provide land managers with new information to improve their conservation efforts for the coastal habitat that support shorebirds in the NGOM. Further, this research can benefit migratory bird conservation by linking wetlands through shorebird habitat use throughout North America.}

\normaltalk{Assessing the Effects of Hormone Implants on Female Tree Swallow Mating Behavior and the Cloacal Microbiome}{Jessica Hernandez, Lisa Belden, Ignacio Moore}{Animals host microbial communities in and on almost every surface of their body, and these microbial communities can be shaped by an animal's behavior. For example, sexual activity is one way through which bacteria can be transmitted between individuals. While considerable research has focused on the sexual activity or the microbiomes of animals, little research has focused on the intersection of the two. For this study, we administered estradiol and sham implants to female tree swallows at the beginning of the breeding season and then assessed their subsequent extra-pair mating behavior, cloacal microbiome diversity, and cloacal microbiome structure. We assigned nestling parentage (a proxy for sexual activity) to determine the minimum number of sexual partners per female. We collected cloacal swabs from females and used 16S rRNA gene amplicon sequencing to characterize each females' cloacal microbiome. Manipulation of hormonal activity via estradiol implants has been well established as an experimental method to increase sexual activity (i.e., solicitations, copulations) in female birds. Female cloacal microbiome diversity and structure did not differ between the treatment groups. However, independent of treatment group, there was a significant shift in both the diversity and structure of the cloacal microbiome across the breeding season. These results represent one of the few studies to document within-individual changes in the microbiome over the course of the breeding season and contribute to our understanding of the factors that shape the cloacal microbiomes of wild birds.}

\normaltalk{Birds Rarely Hybridize: a Citizen Science Approach to Estimating Rates of Hybridization in the Wild}{Geoffrey E Hill, Nicholas M Justyn, Corey T Callaghan}{The extent to which species hybridize is of central importance to taxonomy and species concepts in ornithology, but the only estimate of rates of speciation among wild birds was presented with no data.  In 1963, Ernst Mayr stated that, based on a lifetime of examining museum specimens,  1 in 60,000 birds was a hybrid.  The citizen science dataset eBird presented an opportunity to test Mayr's hypothesis regarding the overall rate of avian hybridization as well as to assess the frequency of hybridization by individuals belonging to different species, families, and orders of birds. Between 1 January 2010 and 31 December 2018, a total of 334,770,194 species records were reported to eBird within the United States. Of this total, 212,875 or 0.064\% were reported as hybrids. This estimate is higher than the rate of hybridization (0.002\%) reported by Mayr (1963) based on impressions from a career studying museum specimens. However, if the 10 most influential hybrid species are removed from the eBird dataset, the rate of hybridization decreases substantially, to about 0.009\%. We conclude that the rate of hybridization for individuals in most bird species is extremely low, even though the potential for birds to produce fertile offspring through hybrid crosses is high. These findings indicate that there is strong pre-zygotic selection working in most avian species.}

\normaltalk{Migration Patterns of Upland Sandpipers and Opportunities for Conservation}{Jason M Hill, Brett K Sandercock, Rosalind B Renfrew}{Integrated models of the ecology of migratory species require tracking of individual migratory organisms throughout the annual cycle. Here, we report the first information on the movement patterns of nine Upland Sandpipers (Bartramia longicauda) that were captured at breeding sites in Kansas and Massachusetts, and tracked with GPS and PTT tags to non-breeding sites in South America. Upland Sandpipers were extreme migrants that regularly made non-stop flights $>$5,000 km up to 7 days. Sandpipers traveled up to 20,000 km per year in their annual movements. Our project resulted in a series of new discoveries. Sandpipers regularly crossed major ecological barriers during migration, which included long oceanic flights, high elevation mountains, and tropical forests. We documented new staging sites at canefields in the mountain valleys of Colombia, grasslands in the Llanos of Venezuela, and at airports along the Atlantic Coast of the US. Unexpectedly, some sandpipers spent the non-breeding season on river islands in the Amazon basin, and pastures in the Cerrado ecoregion of Brazil; areas not previously known to host overwintering Upland Sandpipers. Collectively, the birds used sites across much of northern South America. Overall, the migratory patterns of Upland Sandpipers were more similar to migratory landbirds than to shorebirds. Upland Sandpipers should be buffered against habitat loss and degradation at local sites within their migratory range, but it may be difficult to protect specific sites or broad landscapes that would be needed to conserve a high percentage of the global population.}

\normaltalk{Using Molecular Methodologies to Investigate Adult Provisioning Plasticity and Nestling Developmental Consequences Triggered by Limitations in Preferred Prey}{Brandon D Hoenig, Brian K Trevelline, Steven C Latta, Brady A Porter}{For migratory songbirds, a successful breeding season is largely dependent upon the availability of preferred, arthropod prey.  Unfortunately, arthropods have undergone severe population declines in response to anthropogenic disturbances, and these losses may serve as the root cause for similarly drastic declines in insectivorous bird populations.  As alterations to arthropod prey availability can impact the breeding success of adult songbirds as well as the early development of their nestlings, it is imperative that researchers not only provide precise dietary characterizations for these species, but also illuminate the repercussions that birds face in response to preferred prey limitations. In this study, we utilized DNA metabarcoding coupled with stable isotope analysis to understand how variation in arthropod prey consumption influences the diet and development of nestling Louisiana waterthrush (Parkesia motacilla).  Louisiana waterthrush are riparian-obligate, migratory wood warblers that preferentially breed on high-quality streams and specialize on nutrient-dense, aquatic arthropod prey.  However, when aquatic taxa are limited by stream pollution or seasonal variation, adult Louisiana waterthrush are forced to provision their offspring with terrestrial taxa, which lack the important dietary lipids present in aquatic prey.  Our preliminary results suggest that decreased provisioning of aquatic prey is associated with a decline in nestling physiological condition and that nestlings in poorer condition display lowered concentrations of the glucose and lipid metabolites essential for their development. These results further demonstrate the importance of preferred, arthropod prey for migratory songbird species as well as the potential for molecular techniques to better inform conservation decisions for avian populations.}

\normaltalk{Environmental Correlates of Genetic Variation in the Invasive and Largely Panmictic European Starling in North America}{Natalie R Hofmeister, Scott J Werner, Irby J Lovette}{Populations of invasive species that colonize and spread in novel environments may differentiate both through demographic processes and local selection. European starlings (Sturnus vulgaris) were introduced to New York in 1890 and subsequently spread throughout North America, becoming one of the most widespread and numerous bird species on the continent. Genome-wide comparisons across starling individuals and populations can identify demographic and/or selective factors that facilitated this rapid and successful expansion. We investigated patterns of genomic diversity and differentiation using reduced-representation genome sequencing (ddRADseq) of 17 winter-season starling populations. Consistent with this species' high dispersal rates and rapid expansion history, we found low geographic differentiation and few FST outliers even at a continental scale. Despite starting from a founding population of approximately 180 individuals, North American starlings do not have a detectable genetic bottleneck, as models suggest an extremely large effective population size since introduction. In genotype-environment associations we find more than 200 single-nucleotide polymorphisms that correlate with winter temperature and/or precipitation against a background of negligible genome- and range-wide divergence. Such variants tend to fall in the coding regions of genes associated with metabolism, stress, and neurological function. This evidence for incipient local adaptation in North American starlings suggests that it can evolve rapidly even in wide-ranging and evolutionarily young populations. This survey of genomic signatures of expansion in North American starlings is the most comprehensive to date and complements ongoing studies of world-wide local adaptation in these highly dispersive and invasive birds.}

\normaltalk{Drivers of the Faster-Z Effect in Australian Grassfinches}{Daniel M Hooper, Trevor D Price, Simon C Griffith, Peter Andolfatto}{Sex chromosomes are well documented to exhibit higher rates of protein coding sequence evolution relative to the autosomes in a wide range of species. In birds, faster of rates of evolution for genes on the Z chromosome relative to the autosomes is known as the `Faster-Z Effect'. However, despite empirical evidence suggesting that the phenomenon is widespread across avian taxa the underlying mechanisms responsible for Faster-Z evolution remain unresolved. In order to evaluate evidence for the Faster-Z, and assess and magnitude and drivers responsible, we whole genome sequenced 19 species and subspecies that comprise the complete clade of Australian grassfinches (family Estrildidae). We analyze protein sequence and polymorphism data on this new species tree with independent measures of population size and proxies for the strength of sexual selection in order to evaluate the relative contributions of selection versus drift in producing the Faster-Z. Examining a complete clade of closely related species enables us to examine the evolutionary forces shaping Z chromosome evolution as they might relate to the speciation process directly. Our results suggest that in the Australian grassfinches, higher rates of evolution for Z-linked genes is best explained by the increased efficacy of selection acting on sex chromosomes. This result contrasts markedly with other recent studies highlighting a prominent role for genetic drift in producing the Faster-Z Effect.}

\normaltalk{Gene Expression in the Social Behavior Network of the Wire-Tailed Manakin (Pipra Filicauda) Brain.}{Brent M Horton, Thomas B Ryder, Ignacio T Moore, Christopher N Balakrishnan}{The vertebrate basal forebrain and midbrain contain a set of interconnected nuclei that control social behavior. These brain regions, which are conserved across the vertebrate taxa, are collectively referred to as the vertebrate social behavior network (SBN). While it is known that nuclei (nodes) of the SBN are rich in steroid and neuropeptide activity linked to behavior, simultaneous variation in the expression of neuroendocrine genes aross several SBN nuclei has not yet been described in detail. In this study, we used RNA-seq to profile gene expression across seven brain regions representing five nodes of the SBN in male wire-tailed manakins (Pipra filicauda) from the Ecuadorean Amazon. Using weighted gene co-expression network analysis, we reconstructed sets of coregulated genes, showing striking patterns of variation in neuroendocrine gene expression across the SBN. We describe regional variation in gene networks comprising a broad set of hormone receptors, neuropeptides, steroidogenic enzymes, catecholamines, and other neuroendocrine signaling molecules. Our findings show heterogeneous patterns of brain gene expression across nodes of the avian SBN, highlighting the importance of region-specific sampling in studies of the neurogenetic mechanisms underlying behavior, and providing a foundation for future analyses of how the regulation of gene networks may mediate vertebrate social behavior.}

\normaltalk{Phenology of Nocturnal Avian Migration Has Shifted at the Continental Scale}{KyleG G Horton, Frank A La Sorte, Daniel Sheldon, Tsung-Yu Lin, Kevin Winner, Garrett Bernstein, Subhransu Maji, Wesley M Hochachka, Andrew Farnsworth}{Climate change induced phenological shifts in primary productivity result in trophic mismatches for many organisms, with broad implications for ecosystem structure and function. For birds that have a synchronized timing of migration with resource availability, the likelihood that trophic mismatches may generate a phenological response in migration timing increases with climate change. Despite the importance of a holistic understanding of such systems at large spatial and temporal scales, particularly given a rapidly changing climate, analyses are few, primarily because of limitations in the access to appropriate data. Here we use 24 years of remotely sensed data collected by weather surveillance radar to quantify the response of a nocturnal avian migration system within the contiguous United States to changes in temperature. The average peak migration timing advanced in spring and autumn, and these changes were generally more rapid at higher latitudes. During spring and autumn, warmer seasons were predictive of earlier peak migration dates. Decadal changes in surface temperatures predicted spring changes in migratory timing, with greater warming related to earlier arrivals. This study represents one of the first system-wide examinations during two seasons and comprises measures from hundreds of species that describe migratory timing across a continent. Our findings provide evidence of spatially dynamic phenological shifts that result from climate change.}

\normaltalk{Building for Birds: A New Online Design Tool to Assess the Impact of Development}{Mark E Hostetler, Jan-Michael Archer}{Information regarding the impacts of different urban designs on avian species is largely out of reach for most decision-makers. We present a new online, evaluation tool called Building for Birds (http://wec.ifas.ufl.edu/buildingforbirds/web/tools.html). The goal of the Building for Birds tool is to provide city/county decision makers with a way to evaluate different development scenarios and how they affect habitat for species of forest birds during breeding/wintering and migration seasons. From a systematic review of empirical studies in U.S. and Canada, we have created lists of species for three avian habitat categories. Avian habitat categories are 1) forest fragments used as breeding/wintering habitat, 2) forest fragments used as stopover habitat during migration, and 3) tree canopy within residential areas used as breeding/wintering and/or stopover habitat. Inputs for the tool are the amount of conserved forest fragments and tree canopy in built areas; points are allocated per acre of forest and tree canopy conserved, and these points are different depending if a bird is using the site as breeding/wintering habitat and/or as stopover habitat during migration. Outputs are bird habitat scores and a list of potential birds that could be found in each of three aforementioned habitat categories. About 218 species use fragmented landscapes, including interior forest specialists that use relatively small forest fragments and tree canopy in built areas as stopover habitat. The tool allows decision makers to assess impacts of different development and city designs, and such understanding will enable decision makers to incorporate bird-friendly designs into the planning process.}

\normaltalk{Passive Acoustic Monitoring as a Successful Survey Method for a Critically Endangered Endemic Cracid, Trinidad Piping-Guan}{Mark F Hulme, Alisha F Koulen}{The critically endangered Trinidad Piping-guan, Pipile pipile, or `Pawi', is the only bird species endemic to the island of Trinidad. Historically widely distributed, recent observations have been restricted to the northern forests, due to hunting, deforestation and forest degradation. A cryptic arboreal species inhabiting challenging terrain the Pawi is difficult to survey and a 20-year-old population estimate of between 77 and 231 is imprecise and out-of-date. Pawi have distinctive calls and wing-drumming behaviour, meaning passive acoustic monitoring is, potentially, an efficient survey method compared with point counts or transects. We deployed remote audio recorders in areas of known Pawi occurrence, confirming that passive acoustic monitoring successfully detects this species, to our knowledge the first use of this technology to survey for an endangered Cracid. We used sample recordings to automate the detection of vocalizations using dedicated software and novel algorithms. Using paired detections and playback experiments we determined that Pawi wing drums were detectable up to 200m. Based on this, a 400m distance between recorders was used in the second year, testing the method over an extended area, and furthering our knowledge of the distribution of Pawi in varied habitat types in the Northern Range, where the birds seem tolerant of some human disturbance and where the population may be increasing. We plan to extend the surveys across the Northern Range, and into Central and Southern Trinidad, which would serve as a starting point to develop population estimates, focusing conservation efforts detailed in the Species Recovery Plan.}

\normaltalk{Improving Northern Coverage of the BBS Using New Technologies and Survey-Design}{Samuel Hache, Steven Van Wilgenburg, Adam C Smith, David Iles}{The limited road access in northern boreal regions results in an important lack of spatial coverage for the North American Breeding Bird Survey (BBS) and other avian monitoring programs.  Thus, the status of migratory birds nesting in these relatively pristine northern landscapes remains uncertain.  However, newly available technologies such as autonomous recording units (ARUs) along with increased collaboration among governmental, industrial, and indigenous partners provide the opportunity to address some of these spatial gaps in cost-effective ways.  Here, we present a sampling scheme for monitoring boreal birds to improve our ability to estimate population sizes, distribution, and trends. Specifically, we developed the BOSS design (Boreal Optimal Sampling Strategy), consisting of a statistically rigorous spatially-balanced and hierarchically structured sampling design with respect to ecozones, ecoregions, political jurisdictions, and the habitats within them. Bayesian analysis of the resulting data allows multiple sources of information to be integrated, including human and ARU-based point counts, single and repeat visits, and on- and off-road surveys. We demonstrate the benefits of the proposed approach by contrasting species-habitat models created using single visit counts versus models integrating data from both single-visit human counts and ``multi-visit" ARU data.    The BOSS data can also be integrated into trend analyses using BBS data to provide a more comprehensive understanding of the status of birds breeding throughout the boreal region of North America.}

\normaltalk{Species Responses to Elevation and the Dynamics of Bird Ranges from the Amazon to the Andes}{Jill E Jankowski}{The distributions that species exhibit along environmental gradients are geographical expressions of the range of ecological conditions that allow species persistence. A combination of abiotic and biotic forces can influence the size and shape of species distributions, as well as the position with respect to other species along gradients. Such patterns, when scaled up to whole communities with comparisons across species, can reveal the features critical to community structure and turnover across landscapes. Here we use an extensive point-count database from tropical forests of Manu National Park, Peru, to explore patterns of elevational range size and position, as well as the shapes of species' responses to elevation, across a broad spectrum of the region's diverse avian community. In our assessment, we ask (1) How does range breadth change from lowlands to highlands? (2) Do species show symmetric, bell-shaped distributions, or are distributions asymmetric with truncated range edges? (3) Do the positions and overlap of species ranges suggest species-habitat associations or other interactions that could influence range boundaries? We use our results to target community subunits whose distributions may be driven by key biotic variables or relationships among species, and consider how elevational range size and interspecific interactions can affect the sensitivity of montane communities, and its members, to changing climate regimes.}

\normaltalk{Nest Concealment but not Nest Size Affects Nest Depredation Probability in a South American Grass Wren Population}{Mar\'{i}a M Jefferies, Paulo E Llamb\'{i}as}{Nest depredation is the prime cause of nest failure in passerines. To contribute towards ascertaining how predators shape avian life histories, we evaluated the effects of vegetation cover and nest size over nest depredation probability in a Grass Wren (Cistothorus platensis) population at the Central Andes of Argentina. The Grass Wren is a small insectivorous territorial passerine that builds domed nests well concealed among vegetation. To analyze if visual exposure of nests increases nest depredation, we measured nest vegetation structure at two scales from the nest: micro (0.5m) and meso-site (1m, 3m, 5m and 10m). To assess if predation risk increases with nest size, we evaluated if nest volume was correlated with nest depredation probability. Over 9 breeding seasons, we measured vegetation structure at 304 nest sites and calculated the volume of 236 nests. Nest depredation was affected by vegetation structure at both micro and meso-site scale. Nest survival was higher for nests built on sites with taller grasses at 0.5m and at 5m from the nest. Additionally, nest depredation was lower for those nests located in Pampas Grasses Cortaderia selloana with greater vegetation concealment and placed at further distance from the plant edge. In contrast, nest depredation probability was not affected by nest volume. Our results suggest that nest concealment and location are more important in determining nest depredation than nest size.}

\normaltalk{Challenges of Managing and Integrating Large, Disparate Data Sets for Conservation Applications}{Miguel Jimenez, Chad Witko, Erika Knight, Lotem Taylor, Melanie Smith}{Technological advances continue to expand our ability to track migrating birds throughout their full annual cycle. Consequently, pathway data for migratory birds are rapidly growing and increasingly accessible. Our ability to synthesize these data and reveal broad-scale migratory patterns for conservation action requires the integration of disparate datasets across various spatial scales and taxonomic assemblages. The National Audubon Society's Migratory Bird Initiative is developing a migratory bird conservation platform that integrates tracking, banding, and abundance data for 520 North American species to inform conservation of migratory birds centered on such a database. As part of this ongoing initiative, we are reviewing and summarizing the existing tracking data for North America's migratory species and contacting data-holders about contributing their datasets for inclusion in a migratory bird conservation platform to help us advance public engagement and conservation. Here, we outline our standardized search protocol and approach for discovering and acquiring migratory datasets and discuss our successes and limitations in accessing these data. Further, we demonstrate the value of our database for summarizing the status of available tracking data in North America and identifying data gaps that could serve as important recommendations for future research. Finally, we highlight the inherent challenges that we encountered, along with solutions we've developed for successfully acquiring and managing datasets used in conservation efforts throughout the Americas.}

\normaltalk{Seasonal Movement of an Amazonian Terrestrial Insectivore Suggests Avoidance of Conditions Associated with Changing Climate}{Vitek Jirinec, Cameron L Rutt, Elisa C Elizondo, Patricia F Rodrigues, Philip C Stouffer}{Terrestrial insectivores that inhabit Neotropical rainforest are among the most sensitive of birds to human disturbance, but these species have also declined in seemingly pristine areas. What causes these birds to also disappear from undisturbed forest? Research from forest fragments indicates that one possibility is changes to forest microclimate, because edge effects cause exposed forest to become hotter and drier. Here we tested a prediction of this ``microclimate hypothesis" applied to seasonal variation in primary forest. For Black-faced Antthrush (Formicarius analis) home ranges, we hypothesized that 1) low-lying areas will remain wetter and cooler during the dry season and that 2) birds will move towards these low-lying areas during that time of year. From 2017-2019, we measured ground moisture and temperature along elevational transects and tracked individuals with GPS tags and temperature loggers in primary terra firme forest near Manaus, Brazil. Although ground moisture decreased at all sites during the dry season, more moisture persisted at low elevation sites, following our predictions. Ground temperature (daily minimum, maximum, and range) also increased during the dry season, with especially pronounced increases at higher elevations. During this time, the birds experienced warmer temperatures and used sites at lower elevations (moved 4-6m lower). The dry season may thus serve as a microcosm for climate change in Amazonia--the forest is drier, hotter, and more variable. These conditions appear to be unfavorable for terrestrial insectivores and may explain declines documented in Amazonia, although topographical microclimate refugia may moderate the effects of continuing climate change.}

\normaltalk{Insights into Detection and Habitat Use of Black- And Yellow-Billed Cuckoos, Two Declining and Poorly Understood Species}{Claire A Johnson, Thomas J Benson}{Black-billed and Yellow-billed Cuckoos have experienced extensive range-wide population declines over the last several decades. However, as cuckoos are patchily distributed and hard to detect, population size and trend estimates are not well supported and habitat requirements are poorly understood.  We set out to examine habitat use and detection probability for both species. We performed passive and call-broadcast surveys for cuckoos at 41 sites in northeastern Illinois and used radio telemetry to track Black-billed Cuckoos at a subset of sites. We examined the influence of habitat and landscape covariates on occupancy, and the effect of call-broadcast and temporal and environmental covariates on detection probability. Each species was detected at $>$ 50\% of sites, and while sites were more likely to have both species than expected by chance, each appeared to use different areas within these sites. Vegetation structure and composition did not consistently predict habitat use within and among sites for either species.  Detection probability was increased substantially (up to 2.6 and 11 times for Yellow-billed and Black-billed Cuckoos, respectively) by using call-broadcasts, but varied throughout the breeding season and between species. Black-billed Cuckoos ranged widely within sites, and home range size varied among individuals. Our results suggest call-broadcasts are essential for understanding distribution, population trends, and habitat use.  Site level variables may be more important than fine-scale habitat variables for determining use.  Increasing detection probability and understanding  area and habitat requirements of these two cuckoos will make it possible to target sites for management or protection.}

\normaltalk{Black and Yellow Rail Habitat Associations in a Rapidly Disappearing Landscape}{Erik I Johnson, Justin Lehman}{Little is known about the status of the Black Rail (Laterallus jamaicensis) and Yellow Rail (Coturnicops novaboracensis) in Louisiana, where over 2,000 km2 of coastal wetlands have been lost since the 1930s. Between May 2017 and April 2019, we conducted systematic point counts (at 33 sites) during the breeding and non-breeding seasons for Black Rails, and drag-line surveys (at 16 sites) for both species during the non-breeding season. We tallied 38 Black Rail detections at 21 of 152 point count locations (at 33\% of sites). Occupancy analyses from 1,239 point count surveys indicated a strong positive relationship with Spartina spartinae cover. Among 61 drag-line surveys, we tallied 36 Black Rail detections, including 28 captures of 25 individuals at 43.8\% of sites and 93 Yellow Rail detections, including 68 captures of 60 individuals at 81.3\% of sites. Yellow Rail relative abundance was uncorrelated with Black Rail relative abundance (Pearson R = -0.01, P = 0.970); Yellow Rails used a greater variety of coastal marsh habitats and sites. To understand space use, we deployed 0.9-g VHF radio transmitters on 13 Black Rails and estimated the 95\% minimum convex polygon home range size to be 0.71 ha (± 0.13 ha; range 0.22 -- 1.59 ha). Our surveys efforts demonstrated for the first time that the Black Rail is part of the core Louisiana avifauna and that Yellow Rails more widespread than previously known, which has important implications for coastal restoration and land management in a landscape threatened by sea level rise, subsidence, and erosion.}

\normaltalk{The Kirtland's Warbler Research and Training Project: Past and Present}{Scott B Johnson, David Ewert}{The Kirtland's Warbler (Setophaga kirtlandii) is a recently delisted (US) migratory songbird that breeds in Michigan, Wisconsin, and Ontario and winters almost exclusively in the Bahama Archipelago. Habitat loss and brood parasitism by the Brown-headed Cowbird caused a major decline in this species population. Conservation efforts on its breeding grounds have resulted in a significant population increase in the last 40 years.  Since 2002, the Kirtland's Warbler Research and Training Project determined, through intensive studies, that wintering Kirtland's Warblers in The Bahamas primarily use early succession coppice, particularly those areas with fruit of wild sage (Lantana involucrata), black torch (Erithalis fruticosa) and snowberry (Chiococca alba and C. parvifolia). The project also trained eight Bahamian students in field research and helped support educational opportunities in conservation biology for the students.  Four of the eight students are practicing conservationists in The Bahamas and a fifth is working toward his Ph.D. studying Caribbean pine forests on Grand Bahama.  Collectively this group has enhanced conservation of the Kirtland's Warbler and other species and native plant communities in The Bahamas.}

\normaltalk{Using Fecal Metabarcoding to Reconstruct the Winter Diet of a Declining Shorebird Species}{Elliot M Johnston, Amanda J Klemmer, Rebecca L Mau, Jeffery T Foster, Glen H Mittelhauser, Brian J Olsen}{Over-wintering populations of Purple Sandpiper (Calidris maritima) in Maine have declined by up to 50\% in the past 20 years. Exact reasons for the decline are unclear, but anthropogenic impacts such as coastal habitat degradation may be a contributing factor. Between 2001 and 2018, wild harvest of rockweed (Ascophyllum nodosum)--an intertidal brown macroalgae--more than quadrupled in Maine from 4.8--22.3 million landed pounds. Purple Sandpipers over-winter along macroalgal-covered rocky shorelines, creating a potential spatial overlap between this declining bird species and rockweed harvest. We investigated the winter diet of Purple Sandpipers to better assess whether this species could be vulnerable to changes in habitat and food base. We collected a total of 220 fecal samples in 2018 and 2019, and reconstructed diet taxa using high-throughput sequencing targeting the COI gene. In both years, the most common diet item was acorn barnacle (Semibalanus balanoides), with the taxon present in 95\% of fecal samples and representing the majority of sequence reads. The diversity of diet composition, however, increased with access to the lowest tides. While previous diet studies have found evidence of barnacle consumption, the large representation in our results is novel and surprising. Due to the ubiquity of acorn barnacles in rocky coastal habitats, the Purple Sandpiper food base is unlikely to be impacted by rockweed harvest at current harvest levels, but the importance of barnacles for the caloric needs of populations of over-wintering Purple Sandpipers remains unknown.}

\normaltalk{Is Severe Fire Good or Bad for Spotted Owls?}{Gavin Jones, Rj Gutierrez, M Z Peery}{We examined territory occupancy dynamics and resource selection by spotted owls in a post-fire landscape following one of the largest and most severe fires in California history (2014 King Fire). Spotted owls tended to abandon territories when most ($>$50\%) of their territory burned severely, but owls persisted in territories that experienced a considerable severe fire component (up to ~40\%). Space use by individual owls persisting in burned territories differed based on the extent and pattern of severe fire within their home ranges. Individuals whose home ranges contained 5-40\% high-severity fire tended to avoid severely-burned areas, but individuals selected severely-burned areas when it made up $<$5\% of their home range. In addition, when individuals used severely-burned areas they selected smaller patches and avoided larger patches ($>$115 ha) of severe fire, and rarely made deep ($>$100-m) forays into severely-burned patches. Spotted owls avoided areas that had experienced post-fire salvage logging, but some owls showed even stronger avoidance for unlogged snag forest. Our findings suggest spotted owls are showing adaptive responses to conditions characteristic of historical fire regimes, which were dominated by lower-severity fire and small high-severity patches. Shifts in disturbance regimes that produce novel landscape patterns characterized by large patches of high-severity fire may negatively affect this species. We emphasize that whether severe fire is ``good" or ``bad" for spotted owls is nuanced yet predictable, and depends on the landscape ecology of fire, severe fire patch size, and individual variation.}

\normaltalk{Pre- To Post-Fledging Carryover Effects Reflect Trade-Offs that Explain Life History Variation among Altricial Songbirds}{Todd M Jones, Thomas J Benson, Jeffrey D Brawn, Michael P Ward}{Phenotypic traits acquired in one life history stage can carryover and affect survival in subsequent stages. Such carryover effects are particularly relevant in young animals, as early life stages are thought to be critical with respect to animal life history evolution and population dynamics. In juvenile songbirds, carryover effects from the pre- to post-fledging period may be critical for survival, population viability, and life history evolution but remain poorly understood. To assess potential factors driving life history variation in songbirds, our research tested whether wing development carries over to influence patters of post-fledging survival within and among species. Our initial efforts focused on a single species, the Dickcissel (Spiza americana), where we first documented carryover effects of wing development during the early post-fledging period. Subsequently, we broadened our study to examine carryover effects among 20 co-existing species of an avian community in east-central Illinois, USA. We found evidence for pre- to post-fledging carryover effects at both the intra- and interspecific level, by which species and individuals within a given species that fledged at earlier ages had less developed wings, exhibited poorer flight ability, and experienced higher rates of mortality after fledging. Fledging age was ultimately the result of trade-offs with predation risk, with higher nest mortality rates favoring shorter nestling periods and less developed wings. Ultimately, our results show how morphological traits, locomotor performance, and age-specific mortality can trade-off and interact across juvenile life stages to shape animal life histories.}

\normaltalk{The Rebirth of Ashton Lagoon, Union Island, SVG: A Conservation Success Story}{Orisha Joseph, Kristy Shortte, Audwin Andrews}{Ashton Lagoon is the largest natural bay in the Grenadines, and is home to the largest contiguous mangrove habitat in the region, representing one of the last remaining mangrove forest stands in St Vincent and the Grenadines. Despite being designated as an Important Bird Area (IBA) and legally designated as a Conservation Area in 1987, a foreign developer was given permission to build a large marina, hotel, and golf course complex in 1994. The developer abandoned the project a year later, leaving the area severely damaged, including stagnant polluted water, and damage to the island's fisheries sector, recreational opportunities, and overall health of marine and bird habitat. Sustainable Grenadines (SusGren) and our key partners have been working towards restoration of the area since 2004. We faced many challenges, including difficulty obtaining government approval to carry out the restoration, prospective developers, and community frustration. Approval for the project was finally granted in 2017.  Thanks to generous funding and key partnerships from the onset of the project, SusGren were able to accomplish the following; the creation of seven breaches of the marina causeway, which has drastically improved water quality; construction of bridges and two bird observation towers to create opportunities for bird watching; planting of 3,000 red mangroves;  installation of interpretive signs and installation of three swale systems to allow for the exchange of water from the ocean into the mangroves among other activities. SusGren will present lessons learned and our plans to further improve biodiversity health and encourage sustainable livelihood opportunities.}

\normaltalk{The Oddest of Manakins: Behavioral Evidences for an Atypical Mating System}{Lia N Kajiki, Regina H Macedo}{The Antilophia genus has long been considered the sole example of monogamy among manakins, a family well known for its polygynous mating system in the form of leks. The species in this genus may represent singular circumstances that could characterize a link between polygamous mating systems (e.g., lek) and monogamy. In this study we investigated territoriality and parental care in the Helmeted Manakin (Antilophia galeata). We captured, banded and georeferenced adult males in a savanna area in central Brazil. We also searched for and monitored nests with camera traps during three consecutive breeding seasons. Taxidermized male decoys were displayed with associated playback at various distances from the center of possible territories and agonistic responses were recorded. Finally, we mapped the defended areas by connecting external points marking agonistic responses. There were 26 agonistic responses and seven territories confirmed with areas ranging from 390 to 13,027 m². We monitored 34 nests for a total of 20 hours of video recording, where only the female was seen conducting parental care. Our study confirms individual territoriality in the Helmeted Manakin, suggesting a unique behavioral pattern for manakins. The complete absence of male parental care is an indicator of a polygamous mating system, but further investigation is necessary to confirm the species´ mating strategy. Our results so far suggest that the social monogamy proposed for the species may not be justified, but the species is extremely territorial compared to other manakins.}

\normaltalk{Community Composition and Foraging Selectivity of Insectivorous Bird on Central Kenyan Shade Coffee Farms}{Deven Kammerichs-Berke, Matthew D Johnson}{Shade coffee has been recognized as a well-studied example of a land-sharing management strategy that creates habitat for birds while maintaining agricultural yield. While there is a general consensus that a shade strategy is more ``bird-friendly" than an industrial sun monoculture strategy, little work has been done to investigate the effects of specific shade tree species on bird diversity, and the capacity for different tree species to help deliver ecosystem services. Research in temperate regions has demonstrated that due to shared evolutionary histories, native plant species support higher abundance of native arthropods, which in turn can support a greater number of insectivorous. Using avian foraging observations, mist netting data, and arthropod sampling, we examined two shade tree taxa - indigenous Cordia sp. and introduced Grevillea robusta - on insectivorous bird communities on coffee farms in central Kenya. Our results indicate that richness, diversity, and overall abundance of both arthropods and foraging birds were higher on Cordia than Grevillea. Furthermore, multivariate analyses of bird community data indicated a significant difference in community composition between the two tree species. These results suggest that the  indigenous Cordia shade tree may be better at mitigating habitat loss and promoting ecosystem services in Kenya. Identifying differences in prey abundance and preferences in bird foraging behavior not only fills basic gaps in our understanding of the ecology of East African coffee farms, but also aids in developing region-specific information to optimize functional diversity, ecosystem services, and the conservation of birds in agricultural landscapes.}

\normaltalk{Hotspots, Blind Spots, and Social Inequality: What Drives Spatial Patterns in Urban Bird Diversity?}{Madhusudan Katti, Deja Perkins, Michael W Caslin}{Over the past two decades, it has become mainstream in urban ecology to view cities as social-ecological systems, but our understanding of how social dynamics drive ecological dynamics remains relatively superficial. Phenomenological studies have found interesting patterns in the relationships between income and bird diversity, consistent with a broader (but not universal) trend that has been dubbed the ``luxury effect" in urban biodiversity: wealthier neighborhoods indeed tend to harbor greater bird diversity in many cities. The spatial layout of bird habitat in cities is famously heterogeneous and simple income gradients do not capture the underlying social structure. We contend that the spatial structure of bird distribution in cities is driven by the spatial structure of social inequalities in human society, and present results from new analyses of bird census datasets from multiple US cities to examine how: popular citizen science efforts such as eBird recreate underlying patterns of social inequality through biased spatial sampling; past histories of racial segregation (redlining policies in housing) are reflected in bird distributions (just as tree cover has been shown to do in recent studies from Baltimore and other cities); and, hotspots of urban bird diversity mask blind spots in our sampling and analysis of bird distributions in ways that perpetuate socio-environmental inequalities. We call for a broader, more inclusive approach to studying urban biodiversity that engages diverse communities in research to help us develop a more complete understanding of urban ecology and empowers communities to reconcile biodiversity conservation with human development.}

\normaltalk{Integrating Monitoring, Community Participation and Landscape Conservation for Migratory Bird Conservation: Lessons from Belize.}{Elma E Kay, Ivanna A Waight-Cho, Abidas A Ash}{There are 593 bird species reliably recorded for Belize; approximately half are migratory. With almost 60 percent of native forests intact, the country appears to be a hotspot for migratory birds. Decades of bird monitoring remain insufficient to determine population trends even as the loss of key migratory bird habitat increases. We document efforts, particularly through collaborative initiatives, for an integrated approach to bird conservation combining monitoring and landscape conservation. In 2012, to assist preparation of Belize's first National Biodiversity Monitoring Program (NBMP), a diagnostic survey documented bird monitoring as the most popular terrestrial monitoring. In 2017, we partnered with several organizations to mobilize un-systematized bird monitoring records. Of 93,207 individual data records, we only mobilized 31\% due to poor data quality or disparity. Efforts that have focused on avian monitoring have fallen short of conservation targets for migratory bird species and their habitat. Data challenges have allowed for adaptation and recognition of the importance of re-doubling parallel efforts. Today we have revised the NBMP with improved point count methodologies. But even as this pilot commences, partnering with the Government of Belize and others, we are working to safeguard the Maya Forest Corridor. This effort will consolidate Belize's National Protected Areas System and secure the resilience of the country's critical bird habitat. Along with the rapidly increasing capacity for birding, bird tourism and community participation in monitoring, this approach presents an opportunity to scale-up migratory bird conservation in Belize while we obtain population trend data to assist specific management interventions.}

\normaltalk{Contributions of Banding Data to the Development of Avian Demography and Capture-Recapture Modeling}{William L Kendall}{Bird banding, and subsequent encounters of banded birds, has informed avian demography from its inception. During this process, the availability of such data has provided a basis for the development of capture-recapture models to answer various questions being asked by avian ecologists and managers about survival, migration connectivity and phenology, natal and breeding dispersal, abundance, and others. Beginning with the estimation of abundance of waterfowl by Lincoln in 1930, using band recoveries, to the estimation of apparent survival from recaptures of fulmars by Cormack in 1964, development of tools for estimating population parameters was often inspired by bird banding. The modeling of dead recovery data to estimate survival and harvest rates culminated in the band recovery framework of Brownie et al. (1976), which has been used annually since to help set migratory bird harvest regulations. Subsequent developments integrated band recovery data with recaptures or resightings, to further estimate fidelity to breeding areas, connectivity between breeding and wintering areas, stopover times, and band loss rates. Examples include applications to waterfowl, cranes, and albatross. Methods for data integration will continue to develop, especially as technology for band types and detecting banded birds continues to expand.}

\normaltalk{Population Level Processes on Distribution and Community Level Consequences: A Possible Mechanism for Coexistence}{Cody M Kent, Thomas W Sherry}{Although understanding species distribution and coexistence patterns is a central goal in ecology, we still lack understanding of how both population and community level interactions contribute. In order to understand these patterns, many studies have tested for negative local species covariance patterns thought to be associated with interspecific competition driving ecological segregation. However, at small spatial scales, population level processes, such as intraspecific territoriality or conspecific avoidance, leading to despotic distributions, may also play a role. Here we tested whether interspecific or intraspecific forces dominate in determining community patterns at a local scale in winter assemblages of parulid warblers. These birds are known to be generally food-limited in winter, and compete both intra- and interspecifically for limited food. We used data collected from two sets of point counts taken across the island of Jamaica, and compared them to randomized null distributions. We found that for the multiple common species, birds were more evenly distributed than predicted, consistent with intraspecific territorial behavior driving distributions. In addition, we found greater than expected point level species richness and strongly positive community covariance, the opposite of the typical prediction from interspecific competition. We conclude that intraspecific forces, including territoriality, play a stronger role in determining spacing patterns within these winter assemblages. Furthermore, these patterns appear to contribute to species coexistence, to the extent that despotic distributions lead to unused resources, allowing weaker competitor species to persist on unused or shared resources; and we suggest potential experimental tests for this hypothesis.}

\normaltalk{Genomic Mechanisms Underlying the Introgression of a Sexually Selected Trait in the Red-Backed Fairywren}{Sarah Khalil, Jennifer Walsh, Erik D Enbody, Daniel T Baldassarre, Michael S Webster, Jordan Karubian}{Showy male traits, such as carotenoid-based plumage in birds, have long captured the imagination of naturalists and evolutionary biologists. Carotenoids are responsible for most of the vivid red, orange, and yellow colors we see in birds, and extensive research has shown that these signals can impact individual fitness by mediating inter- and intrasexual social interactions. Understanding the mechanistic basis for and evolutionary history of the production and variation in these traits is key to fully understanding the adaptive function of carotenoid-based signals. Genomic studies of hybrid zones can help reveal these mechanisms because the combined natural phenotypic variation and reduced background divergence from admixture offers unique opportunities to identify genomic regions associated with phenotype differences. The Red-backed Fairywren (Malurus melanocephalus) provides a well-suited system to study the evolution of a sexually selected trait. Here, two subspecies differ drastically in the hue of the male red plumage patch, a sexually selected trait, and hybridize where their ranges meet. We leverage this natural hybridization to characterize genetic underpinnings of this variation by sequencing whole genomes of the 10 most ``crimson" and 10 most ``orange" males from a single population within the hybrid zone and comparing them to individuals from allopatric populations. We identified divergent genomic regions associated with plumage variation and tested for selection and introgression of specific genes between subspecies. Our findings provide novel insights into the genomic mechanisms and geographic spread of a preferred trait across sub-species boundaries by identifying loci responsible for the introgression of a sexually-selected trait.}

\normaltalk{Urban and Endangered: The Paradox of Endangered Species in Completely Anthropogenic Habitats}{Simon Kiacz, Donald J Brightsmith}{Urbanization is a leading threat to endangered species and the ecosystems they depend upon. While many endangered species exist within or near urban areas, these species generally depend upon remnants of native habitat. However, there are populations of threatened species which use nearly exclusively urban habitats. Approximately 17 species of parrot that are classified as `Near Threatened' or below by the IUCN have populations that depend, either wholly or in-part, upon urbanized areas. Since 2016, we have studied the endangered Red-crowned Parrot (Amazona viridigenalis) in south Texas where it depends entirely on a human-modified environment. Most nests are in introduced species of palms, roosts generally take place within a few meters of streets or homes, and food sources consist of trees planted in residential yards. Populations of threatened species such as this provide a cost-effective means to maintaining genetic diversity for the species and increase the biodiversity in urban areas. However, in order to maintain these populations, maintenance of critical habitat elements and the education of the general public about these species is crucial. Although some parrots have managed to establish populations in cities throughout the world, homeowners and governmental agencies should manage urbanization in a responsible way to preserve these populations into the future.}

\normaltalk{Use of Passive Integrated Transponders Increase Detection Rates of Individual Birds Attending Nests}{Alison R Kocek, Jonathan B Cohen, Alexandra M Cook, Chris S Elphick, Chris M Field}{Identification of individuals attending nests is often important in avian field work. For secretive species such as tidal marsh sparrows that cannot be easily identified by individual color bands in the field, capture of adults on nests is the primary method for identification. However, for a multi-year research project such as the Saltmarsh Habitat and Avian Research Program (SHARP), repeated recaptures of birds at the nest within and between seasons are required and have been found to lead to trap avoidance. Additionally, it is feared that this may also lead to occasional nest abandonment. To reduce both of these costs, we attached a Passive Integrated Transponder (PIT) tag to a band on Saltmarsh Sparrows and Seaside Sparrows in New York in 2014-2019. When a sparrow nest was found, we concealed a Radio Frequency Identification (RFID) reader antenna near the nest and if an attending adult had a PIT Tag band, the identity of that individual was successfully recorded every time.  Use of PIT Tags reduced physical capture of adults at the nest by 50-80\%, did not impact survival of the birds, increased detection rates by 10-15\%, and individuals were identified at the nest by use of RFID technology that were never physically captured that season. RFID technology has great potential to increase detection rates while reducing the cost of researcher induced nest abandonment for difficult to observe species which is especially important for species of conservation concern.}

\normaltalk{Variation in Space Use and Exposure to Potential Risk Factors During Migration are not Associated with Vermivora Warbler Population Trends}{Gunnar R Kramer, David E Andersen, David A Buehler, Petra B Wood, Sean M Peterson, Henry M Streby, Kyle R Aldinger, Lesley P Bulluck, Sergio Harding, Rachel Vallender, John P Loegering, Curtis Smalling}{Golden-winged Warblers (Vermivora chyrsoptera) and Blue-winged Warblers (V. cyanoptera) are Neotropical-Nearctic migrants experiencing varied regional population trends that have recently been linked to strong migratory connectivity and historical forest loss at population-specific nonbreeding areas. Preliminary evidence also suggests populations of Vermivora warblers may exhibit variation in space use during migration potentially leading to differential exposure to factors that influence mortality risk. Whether these factors experienced during migration are driving population trends of Vermivora warblers is unknown. We used geolocator data from 90 individual Vermivora warblers tracked during 2013-2017 to investigate whether variation in exposure to a suite of anthropogenic and natural risk factors was associated with recent breeding population trends. Overall, Vermivora warblers exhibited population-specific space use during migration and these differences were associated with variation in exposure to anthropogenic and natural risk factors. However, none of these risk factors helped explain additional variation in population trends after accounting for migratory connectivity (i.e., breeding and nonbreeding region). Our results suggest that factors experienced during migration are unlikely to be driving regional variation in recent Vermivora warbler population trends. We did find, however, that regional differences in projected changes in land use may differentially affect populations of Vermivora warblers during migration in the future. Maintaining suitable stopover habitat is critical for the successful conservation of migratory species, but our results suggest that factors experienced along migration routes are not currently limiting populations of Vermivora warblers.}

\normaltalk{Geographic Variation in the Duets of the Rufous-Naped Wrens Complex  (Campylorhynchus Rufinucha)}{Wiliam Ku-Peralta, Adolfo G Navarro-Sigüenza, Luis Sandoval, J R Sosa-L\'{o}pez}{Acoustic signals used in animal communication play a key role in mate attraction, species recognition and territory defense. Variation in acoustic signals may reflect population structure, lack of gene flow, and phylogenetic relationships. In birds, the study of geographic variation in acoustic signals has been useful for elucidating potential factors involved in phenotypic divergence and for establishing species limits. However, most of the studies on geographic variation have focused on calls and solo songs. We investigated the variation in the duets of the Rufous-naped Wrens (Campylorhynchus rufinucha). Our results suggest that duets vary across the distribution range of the species complex, matching the three taxonomic groups previously suggested (rufinucha, humilis and capistratus). We also found a marked song sexual dimorphism in two groups, rufinucha and humilis, that is lacking in capistratus. At the local level, we found that duets also vary among sites within groups, but less than between groups. Our results suggest that groups rufinucha and humilis sing the most similar duets, while capistratus performs the most divergent duets of the three groups. Further, when comparing duet variation across geography, we found that birds living closer to each other sing more similar duets. We suggest that the existence of three vocal lineages is probably the result of historical geographic isolation followed by other selective (i.e. sexual selection, social selection, habitat structure) and/or non-selective factors (i.e. drift, isolation by distance), and that variation found within groups may be the result of cultural drift or social selection.}

\normaltalk{The Behavioral, Physiological, and Fitness Effects of Urbanization on Wild Song Sparrows}{Samuel J Lane, Michael G Emmerson, Isaac J VanDeist, Valerie Brewer, Michelle L Beck, Scott Davies, Kendra B Sewall}{As urbanization spreads, understanding its impacts on wild bird conservation is increasingly urgent. One approach to discern the effects of urbanization on wild birds is to compare the traits and fitness of individuals from species found in both urban and rural habitats (urban adapters). Song sparrows are a common urban adapter, and territorial males in urban areas are more aggressive and bolder than their rural counterparts. However, the physiological regulation and consequences of theses behavioral differences are unknown. Boldness is associated with altered glucocorticoid stress physiology while aggression is classically associated with elevated sex steroid hormones, namely testosterone. Therefore, we hypothesize that more aggressive urban male song sparrows will have a blunted glucocorticoid stress response and elevated testosterone. Additionally, in birds that provide biparental care, aggression is inversely related to paternal care, so we expect urban males to have decreased nest visitation and reduced nestling development and survival. Four years of studies reveal no consistent differences in endocrine physiology between urban and rural adult males and no evidence urban males sacrifice parental care for territorial behavior. Rather, birds in urban habitats have higher fledgling success, despite significantly higher levels of brood parasitism by brown-headed cowbirds. Therefore, individuals who settle in disturbed environments may be of higher quality and/or benefit from the ecological conditions of urban habitats to achieve higher fitness. Future urban planning efforts will benefit from identifying the ecological variables that increase fitness of wild birds in urban habitats.}

\normaltalk{Evaluation of an Innovative Audio Technique for Monitoring Wading Bird Colonies}{Rachel C Larson, Dale E Gawlik}{Wading birds serve as global indicators of wetland quality and ecosystem changes because they are highly sensitive to hydrologic fluctuations. In Florida, wading bird populations are monitored to assess wetland restoration progress throughout the Greater Everglades ecosystem. Our study aims to improve wading bird monitoring by using autonomous recording units to detect vocal activity rates (VAR) of Snowy Egret (Egretta thula), Tricolored Heron (E. tricolor), and Great Egret (Ardea alba) nestlings in colonies on Lake Okeechobee, Florida. Because avian vocalizations reflect environmental conditions and the internal state of individuals, we are using VAR to estimate the frequency of nestling provisioning by adult birds and predict species-specific nest densities in colonies. Our results reveal a significant correlation (p $<$ 0.01, Rtextasciicircum 2 = 0.64) between the timing of adult wading bird calls and provisioning events, where adult calls accurately indicated 96\% of nestling provisioning events. We also found that average VAR differs among wading bird species and nestling age groups, enabling the use of automatic species recognition software to quantify species composition and nestling age structure within wading bird colonies. Developing a passive acoustic monitoring protocol for wading birds will allow us to detect daily changes in nest densities and provisioning rates based on call rates, increasing the temporal resolution of our nesting models. Large, shallow lake systems like Lake Okeechobee are common globally and many face similar monitoring challenges. Improving colony monitoring methods will advance the recovery of wading birds and the management of these highly modified wetland ecosystems.}

\normaltalk{Comparison of Psittacine Rehabilitation and Release Facilities and Implications for Conservation Planning}{Patricia J Latas, Brooke Durham, Nikki Buxton}{Here we compare 2 different psittacine populations: one from Belize Bird Rescue, a wildlife rehabilitation facility serving native psittacines from the wild, from captivity and from confiscations; SoCal Parrot, a wildlife rehabilitation facility limited to the naturalized psittacines of southern California, USA, where all patients are completely wild for many generations. In their native geographical ranges, psittacines face anthropogenic impact from human incursion, habitat degradation, loss of native food and nesting, polluted water sources, poaching, hunting, harassment and cruelty from the human population. Pet, companion and aviary parrots generally live a life of modern civilization: sedentary, bored, lonely, socially isolated; poor and inadequate nutrition resulting in obesity, ill-thrift, gout, arteriosclerosis, stroke; with consequences of indoor light, noise, air and water pollution leading to toxicities, physiological and mental stress, reproductive disruption, altered molt; disruption of photoperiod and ambient temperature; and diseases of human-associated captivity such as fungal, bacterial and virus amplification and community-acquired infections. Parrots in captivity also reflect the socio-economics of the human captors. Urban parrots suffer stress from general urbanization but are protected from human and other animal depredations and urban species have adapted completely to city life. They rarely suffer from infectious disease, trafficking, harassment or socio-economic factors but may have illness and injury reflecting urban hazards such as window and vehicle collisions.This comparative study outlines the problems and potentials from different facilities to provide psittacines that are appropriately prepared for successful release to the wild and serve as valuable adjunct to conservation planning.}

\normaltalk{Disease Surveillance and Risk Assessment of Urban Parrot Species}{Patricia J Latas}{Populations of wild urban psittacines are rarely surveyed systematically for disease and disease risk analysis. SoCal Parrot is a wild psittacine rehabilitation and release facility providing a unique and resource-rich opportunity for surveillance and monitoring of patients admitted for care. Here we present results from surveillance and simple risk assessment for common pathogens in 13 species of naturalized psittacines from Southern California, admitted for wildlife rehabilitation and release. Commercial PCR screening for pathogens spanned the years 2014-to present, and included Chlamydiophila spp., Avian Polyomavirus, Psittacine Feather and Beak Disease circovirus, Psittacine Herpesvirus, Avian Bornavirus, Tritrichomonas sp., and Toxoplasma gondii. Post-mortem analysis and histopathology was submitted from deceased specimens and included  of pathogen testing. Few pathogens were detected, indicating low likelihood of psittacine-specific infectious disease.  If psittacines in Southern California are considered at low-risk of infectious disease, they could be a valuable reserve for conservation populations, especially for endangered (Lilac-crowned Parrot, Red-crowned Parrot, Yellow-headed Parrot). We hope these data will inform decision-making for conservation of endangered (Lilac-crowned Parrot, Red-crowned Parrot, Yellow-headed Parrot) and threatened species represented in this study, and advise wildlife-human conflict management and public policy.}

\normaltalk{Pairing Status Moderates both Production and Responses to Anti-Parasitic Referential Alarm Calls in Male Yellow Warblers}{Shelby Lawson, Janice Enos, Niko C Mendes, Sharon A Gill, Mark E Hauber}{Defending a nest incurs temporal and energetic costs, and can even be dangerous for the defending parent. Accordingly, the extent and intensity of this costly behavior should reflect both the (perceived) risk and the value of the reproductive output. When facing costly brood parasitism by Brown-headed Cowbirds (Molothrus ater), Yellow Warblers (Setophaga petechia) use referential ``seet" calls to warn their mate of the nearby parasitic danger. Yellow warblers of both sexes produce this call only in response to cowbirds or seet-calling conspecifics, and almost exclusively during the laying and incubation stages of nesting, when the risk of brood parasitism is highest. We hypothesized that the reverse should also apply to male Yellow Warblers that live on a territory without a mate or nest, in that they should be less likely to respond to playbacks that simulate parasitic or other threats to reproductive effort. We presented playbacks to male warblers with a known partner and an active nest (paired) or to male warblers that had secured a territory but did not have a female mate (unpaired). We found that pairing status of males was positively related to their responsiveness to playbacks that simulated threats to the nest, including playbacks of parasitic cowbird chatters and the calls of nest-predatory Blue Jays (Cyanocitta cristata). Paired males were also more vocally responsive with seets during seet-playbacks than were unpaired males. Our data show that male yellow warblers adjust their responses based on the extent of current investment in and value of their reproductive effort.}

\normaltalk{Naturalized Parrots in Asia: The Pet Trade, Issues and Considerations in Urban Environments}{Jessica Lee, Anuj Jain, Ding Li Yong, Serene Chng, Caroline Dingle, Astrid Andersson}{Southeast Asia is renowned for its high levels of biodiversity and endemism (over 60 parrot species), which are matched with equally high levels of threats. Key conservation issues include the largely illegal and unsustainable trade of wildlife, as well as large-scale deforestation. Poaching for the parrot trade is recognised to be one of the sole causes of decline for many species. Changes in consumer demand and enforcement efforts has resulted in species turnover and shifting trade routes, making the trade in wild parrots dynamic and hard to track, and enforce regulations. Our presentation broadly describes the magnitude of the parrot trade in the region as well as downstream effects in urban environments, touching on issues around tackling the trade, introduced species and parrot repatriation, but also conceptually around non-native species retainment as reservoirs for conservation.}

\normaltalk{Hybridization in Birds is Linked with Social and Migratory Traits}{Gavin M Leighton, Lucy J Lu, Eliot Holop, Jessica Dobler, Russell A Ligon}{Hybridization is important for both theoretical and practical reasons. In practice, many evolutionary biologists use hybrid zones to unravel the process of differentiation among incipient lineages, and we use these hybrid zones to make inference about larger evolutionary processes. Consequently, hybrid zones inform our understanding of important evolutionary processes and the likelihood of hybridization may dictate the trajectory of differentiated lineages. However, we do not have comparative insight into the factors driving the likelihood of hybridization. To determine what factors influence the presence and extent of hybridization, we compiled a database of 1011 species from 202 genera within 16 families. We predicted that social and sexual traits may influence hybridization and compiled these traits for the species in the dataset. We found that multiple social and movement variables are associated with the extent of hybridization.  The broad-scale correlations between species-specific traits and hybridization across diverse avian lineages suggests commonalities in the processes involved in mating with heterospecifics. Our results have implications for how we think about, study, and understand hybridization processes and their influence on evolutionary trajectories.}

\normaltalk{Local Forest Type Associations Predict Responses to Pasture Restoration but not Fire for Tropical Resident Birds of the Maya Biosphere Reserve}{Anna M Lello-Smith, Amanda D Rodewald, Viviana Ruiz-Gutierrez}{Conversion of forest to pasture and fire synergistically threaten birds across the global tropics. Though forest restoration has potential to mitigate impacts of forest loss, human-caused fire can render restoration efforts ineffective. We investigated how burning mediates the value of restored pastures for resident tropical birds and identified species-specific responses to restoration and fire. From February to August 2019 we surveyed resident bird communities in Guatemala's Maya Biosphere Reserve (MBR) in mature forest (n = 144) and in regenerating pastures (0 - 12 years of regrowth) that were either burned (n = 67; from pre-fire to 5 months post-fire) or unburned (n = 106) in 2019. The propensity of forest-dependent birds (classified a priori based on broad habitat associations throughout their global range) to utilize unburned pastures varied among species and was strongly influenced by their association with local forest type. Intact forest within the MBR is naturally heterogeneous, encompassing a variety of forest types broadly classified into bosque bajo and bosque alto. Species associated with low-stature forest (bosque bajo) were more likely to use unburned pastures preferentially or equivalently to mature forest than species characteristic of tall forest (bosque alto). Interestingly, local forest associations failed to predict variable and idiosyncratic responses of species to fire. While several forest species responded positively or neutrally to fire, many forest species that used regenerating pastures declined after burning. This suggests that fire compromises the habitat value of regenerating pastures in the MBR for some, but not all, forest-dependent birds.}

\normaltalk{Phylogeography of the Eye-Ringed Flatbill (Rhynchocyclus Brevirostris; Passeriformes, Tyrannidae).}{Fernando Le\'{o}n-Garc\'{i}a}{The Eye-ringed Flatbill is a bird from the Tyrannidae family and its distribution comprises Southern Mexico and Central America. This species finds itself in humid biomes namely tropical rainforests and Cloud Forests. In Mexico's Pacific Slope it is present mainly in cloud forest whereas in the Atlantic Slope and throughout Central America it is found mainly in Tropical rain forests. This pattern of distribution would suggest differentiation within the species as it presents difference in altitude and dry vegetation as barriers. To assess this, we employed genomic data using RAD-Seq form tissue and Museum specimens. After the assembly and filtering processes, we ended up with nineteen samples and total of 17800 usable SNP's.  We used maximum likelihood and Bayesian approaches to make a phylogeographic study of R. brevirostris. The data suggests that the individuals form the Pacific Slope in Mexico are separated from the ones in Yucatan's peninsula. The individuals of the Pacific coast in Costa Rica are set apart from the ones in Panama. The individual of Cerro Pirre is the one that shows the most differentiation. Understanding the phylogeography of this species would give insight to processes of speciation in humid biomes and in species with distribution at different altitudes.}

\normaltalk{Comparative Ecology of City Birds: a Global Perspective}{Christopher A Lepczyk, Frank A La Sorte, Myla F Aronson}{While urban bird ecology has long focused on the natural history, ecology, and conservation of birds living in and around cities, much of our knowledge has come from individual locations. Over the past decade we have seen a large growth in datasets and working groups that have promoted a comparative ecology of cities perspective, with a large focus on birds. These comparisons and syntheses have provided important insights into urban bird diversity, ecology, and conservation. Here we provide an overview of the comparative ecology of city birds, from its earliest work to ongoing projects. We discuss common findings of urban birds across cities, where our missing gaps in knowledge are, and what the state of the science means for ornithology, conservation, and planning.}

\normaltalk{An Historical Perspective of Urban Bird Research: Going Back to Get to the Future}{Susannah B Lerman, Madhusudan Katti, Eyal S , Paige S Warren}{For many, ``urban birds" conjures images of pigeons, starlings, and house sparrows. These species hardly motivate conservation action. Yet as early as the 1970s, ornithologists began considering urban habitats not solely as detrimental to bird populations, but rather as opportunities for management and worthy of investigation. In this presentation, we share our historical perspective on urban bird research, and celebrate the future of the discipline. We use an `ecology in, of, and for the city' framework to highlight different approaches ornithologists have employed over the past 50 years. We start with `in the city' to demonstrate how comparisons between urban and wildland bird communities using traditional ecological methodology established a baseline, characterizing dominant patterns of bird community structure. We then examine how during the 2000's, research embraced a more holistic approach. Ornithologists recognized the city as an ecosystem, with humans serving as the dominant species. This `of the city' research integrated human social drivers to further understand additional underlying patterns. During this time, long-term and multi-disciplinary research sites were established and provided opportunities for mechanistic study. We detail how this interdisciplinary `of the city' approach led the way to research `for the city', which emphasizes co-production of knowledge, and integrates ecological, social and built aspects of the city. Here, research programs actively include city practitioners to advance urban sustainability that benefits birds and people. Exciting challenges await, including deeper analyses of feedbacks between people and birds, and, studies in tropical cities with more diverse avifaunas and complex social-ecological dynamics.}

\normaltalk{Backyard Birds: How Do Species Traits Influence People's Subjective Evaluations of Urban Bird Communities?}{Riley Andrade, Susannah B Lerman, Kelli L Larson, Janet Franklin}{The impacts of urbanization on bird biodiversity depend on human-environment interactions that drive local management decisions. However, ecological measurements such as habitat guild do not capture the complexity of interactions between people and birds in cities. Little research has considered how the public perceives and evaluates their local bird community. To address this gap, we used social survey and bird community data collected in the metropolitan region of Phoenix, Arizona, USA, to determine how bird assemblages influence people's subjective evaluations. We used a trait-based approach to classify birds by attributes that influence public perceptions, such as physical features (color and size), diet, and song. Our classification scheme identified four key groups: urban associated, drab generalists, showy species, and hummingbirds. The urban associated group was largely urban-dwelling species with loud calls and grey coloring, foraging low to the ground. Drab generalists were brown versus colorful, with a familiar song. Showy species consisted of noticeable aesthetic traits, such as colorful birds described as having a musical song. All the hummingbird species were clustered into their own trait group related to a nectar diet, color, foraging height, and a small body size. Both showy species and hummingbirds were related to positive attitudes, whereas urban species were perceived negatively. Our results highlight that people's views of their local bird community are primarily driven by aesthetics, especially color and song. Increasing people's interactions with iconic species with unique physical attributes could improve subjective evaluations and support for future conservation initiatives.}

\normaltalk{Nest-Site Partitioning and Potential for Interspecific Competition among Cavity-Nesting Birds in Old-Growth Longleaf Pine}{Heather E Levy, James A Cox, Robert J Cooper}{Competition theory states that members of a community sharing similar resources will narrow their niche to avoid competition. Cavity-nesting birds provide an interesting system to study nest-site partitioning because of strong interspecific dependencies. Old-growth longleaf pine (Pinus palustris) forests are unique in that a keystone excavator, the Red-cockaded Woodpecker (Leuconotopic borealis; hereafter RCWO) has persisted for several centuries, resulting in densities of cavity resources not typically seen in younger forests in the region. In this study, we measure the degree of nest-site partitioning among members of the cavity-nesting bird community and identify species with potential for interspecific competition. We predicted the community would demonstrate a high degree of partitioning and that larger species would outcompete smaller species for safer cavities. We measured nest cavities in old-growth longleaf pine forests in southwest Georgia in 2018 and 2019. Principal components and cluster analyses indicate the strongest partitioning based on selection of substrate type and decay class. We did not find evidence that body size drives partitioning in this community. Red-headed Woodpeckers (Melanerpes erythrocephalus), Red-bellied Woodpeckers (M. carolinus), and Northern Flickers (Colaptes auratus) often nested in similarly decayed snags, but commonly used RCWO cavities. Brown-headed Nuthatches (Sitta pusilla) and Eastern Bluebirds (Sialia sialis) frequently used late stage snags. Groups with substantial overlap may be subject to interspecific competition where nest-sites are limiting. We did not find evidence of nest-site limitation, suggesting that old-growth pinelands may be especially important to Northern Flickers, Red-headed Woodpeckers, and Brown-headed Nuthatches, species of concern in the region.}

\normaltalk{The Role of the Media in Promoting Awareness of Birds and Conservation Issues--BirdsCaribbean's Media Working Group.}{Emma C Lewis, Mark C Yokoyama}{Communicating important bird news across our scattered Caribbean islands is challenging. Over the years, BirdsCaribbean's Media Working Group (MWG) has found an efficient way for getting this done (virtually!). Key to our success is meeting monthly, combining our strengths and talents; brainstorming to identify critical issues, finding new audiences, and devising media strategies. Media campaigns are particularly important when responding to a crisis, such as an imminent threat to a Critically Endangered bird or Protected Area. Regional bird festivals and annual bird counts are key markers on the calendar, but there is more to these than simply ``getting the word out." Developing themes and topics that resonate, gathering stories from across the Caribbean, and tying it all together with powerful messaging, are all important aspects of the MWG's work. Our products include media releases, a monthly newsletter, videos, webinars, magazine articles, and more. Website articles stimulate interest in our diverse range of projects on the ground (including the use of striking visuals). Our experiences navigating the vibrant, but occasionally choppy waters of social media will be discussed. In all of these activities, timeliness, accuracy, and attention to detail are paramount. Whether we are appealing for post-hurricane assistance, telling conservation success stories, or highlighting issues such as wildlife trafficking, we aim for the highest standards in storytelling. Our members are fueled by a passion for birds and the humans who care about them - and hopes that their passion will be contagious.}

\normaltalk{Long-Term Demographic Changes of Black-Throated Blue Warblers (Setophaga Caerulescens) from Across the Breeding Range}{William B Lewis, Robert J Cooper, Richard B Chandler, T S Sillettt}{Climate change is leading to range shifts in many species towards more northerly latitudes and higher elevations but a better understanding of the impacts on demography is needed to predict the consequences of climate change. We hypothesized that climate change may be impacting productivity through effects on local food supplies, which we tested by comparing long-term datasets of Black-throated Blue Warblers (Setophaga caerulescens) breeding at the range-core in New Hampshire and near the low-latitude breeding limit in the southern Appalachians. Breeding birds at low elevations in the southern Appalachians are likely to be extremely vulnerable to climate change due to being near the low-latitude and low-elevation breeding limit and so should exhibit the strongest demographic changes. Temperatures generally increased and precipitation decreased in the southern Appalachians compared to in the range-core. Breeding populations held steady or slightly increased in the range-core but in the southern Appalachians were stable at the highest elevations and declined at the lowest elevations. Both food biomass and nestling mass strongly declined at low elevations in the southern Appalachians concurrent with population declines. These results are broadly consistent with the hypothesis that climate change is influencing these populations and suggest that the mechanism may be through reductions in food abundance and nestling mass, which could lead to reduced first-year survival and/or recruitment. Our long-term monitoring effort has provided a potential pathway through which climate change may be causing population declines and highlights the need for continued monitoring of bird populations in these areas.}

\normaltalk{Evolutionary History of a Manakin Hybrid Zone that Exhibits Asymmetrical Introgression of Traits}{Haw Chuan Lim, Kevin Bennett, Rebecca Dikow, Paul Frandsen, Matthew Fuxjager, Sarah K , Barney Schlinger, Michael Braun}{The process of biological diversification and the generation of new species lineages are the results of a complex interplay among geography, ecology, historical contingencies and genetics. Different model systems excel in illuminating different aspects of this complex evolutionary process. In the tropics, the bearded manakins (Manacus spp.) are emerging as a model system for studying speciation modulated by sexual selection and introgression of positively selected traits. They belong to the family Pipridae (55 species), which includes many lek-breeding species. Lekking males often have conspicuous plumage, and perform elaborate courtship displays at communal arenas. In NW Panama, the white-collared manakin (M. candei) and golden-collared manakin (M. vitellinus) come into contact and hybridize extensively. Previous studies have documented unidirectional introgression of male plumage traits (yellow collar and drab green belly) from M. vitellinus into M. candei. Compared against transitions based on morphometric and genetic data, clines for these two plumage characters are displaced ~50 km west into M. candei territory, with the plumage transition occurring abruptly across a geographic barrier (Changuinola River). In this study, we generated a de novo genome of M. vitellinus using data from PacBio and Illumina sequencing technologies, and hybrid assembly. We then sequenced the genome of 26 M. candei and M. vitellinus individuals, including 20 M. candei from both sides of the Changuinola River. We identified genetic loci that introgress from M. vitellinus into M. candei, loci that likely underlie the introgressing morphological traits. Demographic analysis revealed the history of divergence and hybridization between these two species.}

\normaltalk{Chasing the Tides: Overwinter Intertidal Movements and Density Estimates of Saltmarsh and Seaside Sparrows}{Marae C Lindquist, Evangelyn L Buckland, John P Carpenter, Raymond M Danner}{There are large gaps in knowledge regarding the wintering populations of Saltmarsh Sparrows (Ammospiza caudacuta, SALS) and Seaside Sparrows (Ammospiza maritima, SESP). Marsh species are losing essential habitat due to sea level rise, development, and other anthropogenic forces. Both species are listed as Species of Greatest Conservation Need in the NC Wildlife Action Plan and the USFWS will determine if SALS should be federally listed as Threatened or Endangered in 2023. Understanding densities, abundances, and habitat utilization of SALS and SESP throughout their winter stationary period is important to understand the impacts of sea level rise and design effective conservation solutions. During the winters of 2019 and 2020 we are researching SALS and SESP at two sites in Southeastern North Carolina to estimate density, abundance, space use, and site fidelity, using a combination of mark recapture and radio telemetry. Both species use regularly flooded tidal marshes and their movements track the tidal cycle, suggesting that both supratidal roosting and intertidal foraging areas are important to conserve. The extent of their daily range differs significantly between species, leading to different patterns of habitat use, which suggests a need for different management priorities for each species. Both SALS and SESP have high site fidelity within seasons, suggesting little flexibility in habitat use. We are combining these results with mark-recapture data to estimate density of SALS and SESP in habitats at our study sites. This study will provide information about habitats that must be maintained to conserve wintering grounds for SALS and SESP.}

\normaltalk{Linking Patterns of Change in Migratory Songbirds at Two Long-Term Bird Banding Stations}{Andrea M Lindsay, Lucas W DeGroote, Mark C Shieldcastle, Henry M Streby}{Songbirds are facing increasing challenges throughout their annual cycles, chief among them being climate change. Many studies of the effects of changing climate on songbirds use data from one location, however analyzing multiple datasets may give a more nuanced picture. We used data from two long-term bird banding operations, Powdermill Nature Reserve, Rector, Pennsylvania (PNR; 1961-2017) and Black Swamp Bird Observatory, Oak Harbor, Ohio (BSBO; 1992-2017) to compare changes in migration timing, morphology, estimates of population size, and refueling performance during both migratory seasons. Despite their relative geographic proximity (350km), songbirds migrating through these areas are experiencing different patterns in many of the variables tested. Notably, opposite trends in morphology occur: whereas songbirds caught at PNR overwhelmingly show a decrease in body size and wing length, wing lengths are increasing in most songbirds at BSBO. Additionally, although both stations are experiencing an overall decrease in capture rate over time, the trends for many species are inconsistent between the two stations. Likewise, directionality of change in arrival timing is similar at both stations but the magnitude of change, and which species show change, varies. However, birds at both stations show consistent patterns in gaining fat mass throughout the day as it relates to temperature and relative fat content in the morning. Our results highlight the importance of spatial replication in long-term studies in making broad inferences about how climate change affects entire species or communities during all parts of the annual cycle.}

\normaltalk{Model Selection for the North American Breeding Bird Survey}{William A Link, John R Sauer, Daniel K Niven}{Hierarchical models allow for analysis of population change from North American Breeding Bird Survey data while controlling for nuisance factors.  Many reasonable models can be proposed; choosing among these possible models is one of the more challenging aspects of BBS analysis.  We completed an assessment of a set of 4 alternative models for the analysis of BBS data using Leave-one-out Cross Validation (LOOCV).  LOOCV involves omitting an observation, fitting a model, and assessing the prediction of model fit for the omitted observation.  LOOCV is a time-consuming procedure when applied to all observations in BBS species datasets.  We applied it to a random selection of observations balanced by year for BBS data from 548 bird species. Models in our set differed in year effect structure and form of overdispersion. We also used a hierarchical model applied among species to evaluate posterior probabilities that each model is best for individual species.  Models in which differences in year effects were conditionally independent tended to be favored (78.5\% of species) over models in which year effects were modeled by a slope parameter and a random year effect, and models with t-distributed overdispersion were favored (68.5\%) over models with normally distributed overdispersion.  The model with difference-based year effects and t-distributed overdispersion was selected for 51.5\% of species. However, differences in estimated trends among models tended to be small, and significant difference in annual indices only occurred in 18 species.  Model selection based on Watanabe/Akaike information criterion (WAIC) only weakly corresponded to model selection using LOOCV.}

\normaltalk{Red-Whiskered Bulbuls (Pycnonotus Jocosus) Synchronize their Nestlings' Fledging by Compensatory Feeding in Broods with Asynchronous Hatching.}{Chen-yang Liu, Rui-chang Quan, Huan Li}{Hatching asynchrony is widespread in altricial bird species leading to size-related sibling hierarchies which are expected to lead to asymmetric food competition, with senior siblings superior to juniors. This could limit the juniors' development rate and result in fledging asynchrony. Here we studied hatching asynchrony, parental food allocation, and fledging patterns in Red-whiskered Bulbuls (Pycnonotus jocosus) from 2013 to 2019. We also manipulated the degree of hatching asynchrony by cross-fostering to assess whether the food allocation strategy and flegding pattern varied with the age gap. We found that 69.31\% of successful nests hatched asynchronously while 90.30\% of successful nests fledged synchronously. Age gap was significantly higher in asynchronously fledging broods than synchronously fledging broods. 90.30\% of control broods and 61\% of cross-fostering broods those under 3 age gap flegdging synchrony, while all broods fledging asynchrony when age gap was 3. Food allocation strategies within asynchronously hatching broods differed significantly between synchronously and asynchronously fledging broods. In synchronously fledging broods, parents fed significantly more food to seniors at an early stage but not at a late stage. In asynchronously fledging broods, parents fed significantly more food to seniors through the whole nestling period. Our results demonstrate that Red-whiskered bulbuls compensated junior siblings in the later nestling period by a higher food provisioning rate, which increased their growth rates and contributed to synchronous fledging of the brood. This study is the first to provide insights into the relationship between fledging asynchrony and food allocation strategy based on age gaps.}

\normaltalk{The Important Role that Caribbean Bird Festivals Play in Encouraging an Interest and Appreciation of Birds and Promoting Local Conservation Values.}{Eduardo M Llegus-Santiago, Sheylda Diaz-M\'{e}ndez}{The Caribbean is a lively region with a mix of cultures, languages and colorful people, just like our birds! In 2002, BirdsCaribbean introduced the idea of bird education through the Caribbean Endemic Bird festival (CEBF), celebrated every spring. The objective is to increase public knowledge, awareness and appreciation of the region's unique bird life--171 species of birds are only found in the Caribbean. The CEBF has been successfully celebrated for 19 years. Diverse activities are organized by local organizations in each Caribbean island/country, including presentations in schools, birding walks, art and photography competitions, bird games, tree plantings, clean-ups, radio and TV programs, and creative events like ``Whistle like a bird".  In 2006, we began celebrating World Migratory Bird Day in fall, to highlight the long journeys that migratory birds take each year and how critical Caribbean habitats are to their survival during migration and for overwintering. There have been many positive impacts and outcomes from these festivals. Examples include: joining local birding clubs, cleaning up habitats, planting native trees for birds, and reducing personal carbon footprints and use of plastic (including a campaign that resulted in plastic being banned). Key to success is a network of enthusiastic and dedicated coordinators, providing educational materials and small grants. We also face challenges, but one thing is certain: festivals are a great way to get people interested in birds. They learn about a specific theme every year and are inspired to get involved in conservation in some way.}

\normaltalk{Further Enigmatic Declines and No Recovery: Two Decades of Change in Bird Populations at Tiputini Biodiversity Station, Ecuador}{John G Blake, Bette A Loiselle}{Lowland forests of eastern Ecuador are among the most diverse anywhere. Our 20-year study on bird communities has taken place at the Tiputini Biodiversity Station. We established two 100-ha study plots in 2001 and have sampled birds using mist nets and direct observations each year since (Jan-Mar). We have accumulated ~8,300 captures on Harpia plot and ~8,575 on Puma; direct observations total ~33,700 on Harpia and ~29,000 on Puma. During 2001 to 2009, capture rates were approximately 57/100 mnh while observations averaged ~22/100 m of transect. Capture rates declined during subsequent years (2010-2020) and now are ~27/100 mnh; observations have declined to ~11/100m. Changes have occurred across a wide range of guilds, particularly insectivores. Declines from 2009 to 2014 were associated with severe La Ni\~{n}a events, suggesting high rainfall may affect reproductive success, either directly or through reductions in resources, particularly insects. Although recent years have not been characterized by such strong events, captures and observations have not started to recover, suggesting that impacts may persist for some time. That both mist nets and observations provide similar perspectives on declines argues against effects of methods. Further, habitat change is not implicated as the study area is surrounded by extensive, undisturbed lowland forest. Anecdotal accounts from bird-tour leaders confirm that bird declines are not limited to the Tiputini area. Additional years are needed to determine if populations return to previous levels or whether the current rates are the ``new normal".}

\normaltalk{Adaptive Variation in Avian Eggshell Structure and Gas Conductance Across Elevational Gradients?}{David Ocampo, Carlos D Cadena, Gustavo Londono}{Many tropical species have limited elevational distributions, which are potentially bounded by constraints imposed by physical conditions on physiological processes. Although some studies have examined variation in the physiology of adult birds with respect to elevation, little attention has been paid to the structure and function of eggshells, which mediate gas exchange between the embryo and the environment. At high elevations, dry air is expected to increase water loss from the egg; selection to avoid dessication might be expected to favor reduced gas conductance by means of increased eggshell thickness or reduced pore area. We used gas diffusion experiments and scanning electron microscopy to examine water vapor conductance rates and eggshell structures in 197 bird species distributed along a 2800 m elevation gradient in the Andes. As predicted, water vapor conductance across the eggshell declined with elevation among all species and among species within families, but not among individuals within species. Variation among species in eggshell conductance was lower at high-elevation sites, potentially indicating greater constraints at such sites. Structural changes in eggshells with respect to elevation varied among taxonomic families of birds, suggesting different potentially adaptive responses to common selective pressures in terms of eggshell thickness and pore density and size. We suggest that considering functional and structural traits of eggshells, which influence embryo development, may help one to better understand the elevational distributions of species and to forecast their responses to global climatic change.}

\normaltalk{Hybridization Dynamics and Population Genomics of a Manacus Hybrid Zone}{Kira M Long, Michael J Braun, Jeffrey D Brawn}{Hybrid zones are dynamic systems where sexual selection and natural selection act upon admixtures of two distinct species' genomes. Hybrid zone dynamics, whether the hybrid zone moves across the landscape or remains stable, are affected by the behavior, genetics, and ecology of the hybridizing organisms. Whereas hybridization is fairly common in nature, questions remain about the processes underlying the movement and persistence of hybrid zones. I am investigating the Golden-collared Manakin (Manacus vitellinus) and the White-collared Manakin (Manacus candei), two species with strong sexual selection and lek breeding behavior. These manakin species hybridize in Panama where females of both species preferentially select males with yellow throats. With sexual selection favoring the yellow phenotype, theory predicts that the yellow phenotype should spread into the white parental range. Yet, despite strong female preference, the yellow phenotype has not spread to fixation, implying that natural selection via other selection pressures must be counteracting sexual selection to maintain stability. I am investigating why this hybrid zone is stable by establishing which specific selective pressures are acting against hybrid individuals. Through the use of Restriction site-Associated DNA sequencing (RADseq) and ecological monitoring, we investigate the diversity and divergence between the parental and hybrid populations and find possible trends in selection pressures against hybrids. At the genomic level, we see a putative hybrid center and regions of divergence between hybrid populations compared to the parents. Additionally, monitoring nests has revealed that hatching rates are lower in hybrids, possibly indicating hybrid inviability.}

\normaltalk{Mitonuclear Incompatibilities with the Sex Chromosomes Suggested in the Emergence of Reproductive Isolation Within an Avian Hybrid Zone}{Kelsie A Lopez, Irby J Lovette, Daniel M Hooper}{The Long-tailed Finch (Poephila acuticauda) comprises two subspecies that differ markedly in bill color: the western yellow-billed, P. a. acuticauda, and the eastern red-billed, P. a. hecki. A phenotypic hybrid zone containing orange-billed birds exists where the ranges of the two subspecies meet. We studied genetic differentiation across their species range to better understand the extent of admixture between subspecies. Geographic admixture clines for the mitochondrion and Z chromosome were constructed using samples collected across an 1,800 km transect. MtDNA was sequenced for 52 individuals in order to evaluate genetic differentiation between subspecies. MtDNA diversity was assessed using a restriction enzyme fragment length polymorphism analysis to reveal mitochondrial haplotype identities for 630 individuals. Phylogenetic analysis indicates a subspecies divergence time of 0.47 MYA (0.32-0.66, 95\% HPD). Mitochondrial haplotype network analysis revealed two reciprocally monophyletic subspecies clusters separated by 22 fixed differences. Genetic admixture between subspecies is limited to the region defined by the boundary of the Kimberly plateau, which most likely played a role in vicariance. Geographic clinal analysis revealed steep and concordant mitochondrial and Z chromosomal clines, which suggests that both may contribute to reproductive isolation between subspecies. Since the mitochondrion and W chromosome are maternally inherited, the location of these clines may be indicative of Z-W interactions. Genetic exchange between subspecies is most limited at sex-linked loci. We speculate that this is possibly due to the accelerated accumulation of reproductive incompatibilities and chromosomal inversions on the sex chromosomes compared to the autosomes.}

\normaltalk{Tracking Avian Disease Prevalence Across Environmental Gradients in Oklahoma and East Texas}{Krisangel Lopez, Tamaki Yuri, Scott C Weaver, Michael A Patten}{Emerging infectious diseases represent a significant threat to global health and security, and wild birds play an integral role in pathogen dispersal dynamics. Wild birds are infected by an assortment of encephalitic viruses, with consequences for both wildlife and human health. Oklahoma and Texas serve as breeding locations and transitory migration routes for a wide range of bird species. We surveyed arthropod-borne virus (arbovirus) prevalence in birds collected at three sites in Oklahoma and eastern Texas, with the goal of learning about the roles of migrant and resident birds as amplifying hosts. We collected blood serum from 369 wild birds in the summers of 2017 and 2018.  Hemagglutination Inhibition tests were used to screen for nine encephalitic viruses (West Nile Virus, St. Louis Encephalitis, Eastern Equine Encephalitis, Western Equine Encephalitis, Highlands J virus, Venezuelan Equine Encephalitis, Fort Morgan Virus, Rocio virus and Ilheus Virus). Our results demonstrate local circulation of at least six different viruses in both Oklahoma and Texas. These results will help support the case for year--round surveillance of avian disease in the southern Great Plains, and also show that there might be a phylogenetic component driving sero-positivity among the suboscine passerine super family.}

\normaltalk{Social Networks and the Illegal Trade of Wild Birds: a Call of Attention}{Fernando G L\'{o}pez, Mar\'{i}a E Rebollo, Miguel Á Santill\'{a}n, Igor Berkunsky, Juan M Grande}{Globally, social networks grew linearly in recent years. Facebook is the most popular with more than 1650 million users. Internet commerce is growing in different areas, social networks are an example of that. This trade can hide illegal activities such as wildlife traffic. We searched for ``wild birds buy/sell groups" on Facebook in Argentina. We found 128 groups of bird trade (79 public and 49 closed), with 96,785 users (756.1±1016.4; 3-5,569). In 18 of 23 provinces, we detect birds trading groups. A deeper analysis of the second largest group located in La Pampa province (our research institute is located there) we found buy/sell offers of 63 bird species of 22 families. The exotic bird trade represents 17\%, the rest are native and they are protected by several provincial/nationals laws ant their trade is illegal. Almost 60\% are species native from the province, whose capture could be local. The high offers and interest for threatened species (especially songbirds and parrot), the prices (U\$S 25±44; 0-328) and the ease use that Facebook offers could contribute to increasing this illegal trade, generating an impact in bird populations, particularly serious for endangered species. The Internet has radically changed the way we see and the way we interact with the world. This work highlights the widespread use of Facebook for illegal trade of wildlife in Argentina. A strong intervention by international, national and local wildlife authorities to control illegal trade in social networks should be necessary as well as other similar studies around the world.}

\normaltalk{Community Perspectives -- Achieving more Equitable and Inclusive Research and Programming in Ornithology and Conservation}{Marilu Lopez Fretts, Jose G Gonzalez, Karen A Purcell, Cecilia Alvarez, Juan Flores, Yao A. Foli, and John Annoni}{The fields of ornithology and conservation although filled with potential to ``democratize" science, and engage with historically underrepresented communities, have not advanced significantly in increasing equity, diversity, and inclusion. Frameworks and approaches, even when following ``best practices" and ``culturally responsive techniques," have been informed largely by dominant culture worldviews and do little to decrease the gap between underserved communities and science institutions. Ornithology and conservation, much like the rest of the sciences, are riddled with examples of ``trickle-down engagement," to reach historically excluded participants, often resulting in projects that are ``community placed" rather than ``community based." Collectively, these practices reinforce scientific norms and institutional structures that lead to a perception of a scientific ``elite" that benefits only a few.     A team of Community Researchers in collaboration with the Cornell Lab of Ornithology share a different perspective and approach for doing science ``with" communities instead of ``on" communities. Through humor, honesty, using the arts, and storytelling, the authors, leaders from underserved communities, will share our community-based participatory research results in a presentation focused on understanding the role of partnerships in the implementation of equitable research. Our process includes a community review board of non-negotiables, working agreements, and methodologies that helped us gain and maintain trust and transparency and identify stark differences in perspectives. We discuss solutions to problems seldom addressed by science institutions to genuinely address equity, diversity, and inclusion in conservation science, leading to better, kinder, and more equitable research and programming in underserved communities.}

\normaltalk{Phylogeography and Evolutionary History of the Elusive Antwrens in the Neotropical Genus Euchrepomis (Thamnophilidae)}{Andreina L\'{o}pez-Marcos, Sergio D Bol\'{i}var-Leguizam\'{o}n, Thiago VV Costa, Bret M Whitney, Alexandre Aleixo, Lu\'{i}s F Silveira, Robb T Brumfield, Jorge L P\'{e}rez-Em\'{a}n, Gustavo A Bravo}{Neotropical diversity has been shaped by the complex topography and dynamic climate history of the region. Phylogeographic studies are crucial to describe such diversity patterns and the processes underlying their evolutionary history. Euchrepomis (Thamnophilidae: Euchrepomidinae) is a widespread genus of antwrens that is considered to be the sister of all other thamnophilids. It currently includes four species: E. callinota, E. sharpei, E. humeralis and E. spodioptila. The former two species inhabit Andean and Central American montane forests, whereas the other two are restricted to lowland Amazonia. Aiming to unravel its evolutionary history, we conducted a molecular phylogeographic study including representatives of all species and 9 out of 10 currently recognized taxa. We inferred the evolutionary relationships within Euchrepomis using two mitochondrial genes (ND2 and Cytb), one nuclear intron (Fib5) gene, and 1,869 ultraconserved elements (UCEs). Our results supported the monophyly of the genus and revealed pervasive non-monophyly across species. Specifically, the geographically disjunct race guianensis of E. callinota was found to be sister to the rest of the genus and both E. humeralis and E. spodioptila were found to be paraphyletic with respect to each other. Montane and riverine barriers were found to be more informative of evolutionary relationships than current taxonomy. A taxonomic revision of the genus is warranted because plumage coloration, the main character on which taxa descriptions were based, does not reflect phylogenetic signals. The diversification and current distribution of Euchrepomis was shaped relatively recently, with divergence events starting during the Pliocene and peaking during the Pleistocene.}

\normaltalk{Bird-Window Collisions in Urban Areas: Evaluating Risk Factors and Mitigation Approaches}{Scott R Loss, Jared A Elmore, Sirena Lao, Timothy J O'Connell, Corey S Riding, Georgia J Riggs}{Urban areas contain important bird habitats but also pose many human-related threats to birds, including numerous direct mortality sources (e.g., cat predation, chemical poisoning, vehicle/structure collisions). Collisions with building windows are a top source of bird mortality in cities, and reducing bird-window collisions has been identified as an important action to help halt and reverse declines of North American bird populations. Until recently, few studies have assessed variation and correlates of bird-window collisions or field-tested the effectiveness of approaches designed to make windows more bird-friendly. We conducted five years of bird-window collision surveys in Oklahoma and Minnesota to assess spatiotemporal variation and correlates of collisions and to inform potential mitigation. We found that: (1) bird-window collisions vary predictably in time, both seasonally and throughout the day and night, and in association with bird migration phenology and weather; (2) collisions vary spatially in association with factors (e.g., amount of glass, nighttime lighting, and vegetation) operating at multiple scales, from individual building façades, to entire buildings, to landscapes around buildings; and (3) collision risk varies among bird species and in association with life history (e.g., long-distance migrants are most vulnerable to collisions). We have used these findings to recommend collision mitigation practices, and we have recently initiated studies assessing approaches to reduce collisions, including using radar-estimated migration activity to predict collisions and installing markers and films on windows to reduce transparency and reflection effects. Further research and widely-implemented collision mitigation are needed to reduce this major urban threat to bird populations.}

\normaltalk{Migratory Bird Conservation on Working Lands in Latin America/Caribbean: Projects Supported by USFS International Programs}{Greg Butcher, Jim Chu}{South of the United States border, most conservation activities take place on working lands; even many protected areas include productive activities within their borders. Thus, conservation experiments are in place, and we need to learn from past and present activities to separate beneficial practices from problematic ones. One of the keys for long-term conservation is to help producers improve their livelihoods, for example by improving grass cover in northern Mexico; improving incomes through sustainable forestry in Chiapas, Mexico; payment for ecosystem services in Costa Rican farmland; certification for environmental stewardship in coastal salt flats and shrimp farms; and certification for improved sales/prices in grass-fed beef and in shade-grown coffee, cacao, and cardamom plantations. We will introduce Forest Service-supported projects in each of these areas.}

\normaltalk{Raising Awareness and Building Capacity for Conservation of Birds in Puerto Rico: The Puerto Rican Ornithological Society}{Gabriel Lugo, Nathaniel Gonz\'{a}lez, Luis Ramos, Lisandra Izaskun}{Puerto Rico has a rich diversity of birds, including 17 endemics, and dozens of neotropical migrants. It has been seriously challenged over the years with both natural and man-made disasters. Despite these challenges, since 1995 the Puerto Rican Ornithological Society (SOPI) has done an outstanding job sharing knowledge, raising awareness, and promoting the conservation of the birds in Puerto Rico and surrounding islands. We began with a  small group of volunteers, mostly biologists. But over time, our membership has grown and become more diverse, bringing new ideas and people. SOPI identified a need to create regular training workshops that helped integrate our members into bird monitoring and research. Since 2000 ~50 workshops have been given on the identification of shorebirds, warblers, waterbirds and terns. In 2017, we established our annual Shorebirds Festival with average attendance of more than 600 persons. In February 2019 SOPI collaborated with BirdsCaribbean to host a week-long international shorebird training workshop in Cabo Rojo, a WHSRN site. Every year SOPI is an active participant in the Caribbean Endemic Bird Festival and World Migratory Bird Day, delivering activities in schools, shopping centers, universities and TV, that reach thousands of children and adults. Agreements with BirdLife International, the National Wildlife Federation, BirdsCaribbean and local organizations such as Para La Naturaleza, together with whom we manage the Puerto Rico eBird.org platform, allowed for studies such as Important Bird Areas (IBA), Limicola Network, Caribbean Waterbirds Census, and assessing the impacts of Hurricanes Irma and Maria in 2017 on our birds.}

\normaltalk{Tropical Forest Fragmentation and Isolation: Is Community Decay a Random Process?}{David A Luther, Justin Cooper, Jared Wolfe, Rob Bierregaard, Andrew Gonzalez, Thomas E Lovejoy}{Habitat destruction and degradation are the leading causes of species declines and extinctions in the world. Human altered landscapes often leave fragments of previously contiguous habitat, which may be of significant conservation value. We assessed the effects of habitat fragmentation on the taxonomic diversity, community composition, and nestedness of avian communities before and after fragment isolation at the Biological Dynamics of Forest Fragments Project research site in the Amazon rainforest. Species in 10ha and 100ha fragments had less taxonomic diversity at the family and genus level after fragmentation. In addition, after fragment isolation, but not prior to fragmentation, the species in the 10ha fragments were a nested subset of the species in the 100ha fragments. Finally, within the fragments two distinct communities formed, those on the edge of the fragments and those at the interior of the fragments, indicating that even though edge species increased after fragmentation they did not penetrate the interior. The controlled isolation of fragments from contiguous forest, resulted in rapid changes in the taxonomic diversity and species composition and fragment size served as a driver of species assemblages across the landscape. We suggest that future research disentangle the influence of survival and reproductive success, on the viability of bird populations in tropical forest fragments.}

\normaltalk{Genome-Wide Analysis Reveal Patterns of Species Collapse Despite Strong Plumage and Ecological Divergence}{Vanessa E Luzuriaga-Aveiga, Jason T Weir}{The pace at which reproductive isolation evolves could be elevated under strong divergent selection, causing the rapid accumulation of phenotypic differentiation (i.e. ecological speciation). However, high levels of phenotypic divergence might evolve quickly but lead to minimal levels of species discrimination, in which case incipient species coming back into secondary contact may hybridize extensively causing both populations to collapse into a gene pool. Most studies of reversing speciation in birds have been focused on either poorly differentiated species (e.g. cryptic ravens) or adaptive radiations (e.g. Darwin's finches). But the evidence of speciation in reversal in continental species with strong phenotypic and ecological differentiation remains unclear. Here, we analyze the hybrid zone dynamics in a pair of young tanager sister species, Ramphocelus carbo and R. melanogaster, with strong phenotypic and elevational divergence in the transition between the Andes and Amazon of the Huallaga valley of Peru. We analyze hybrid zone width for genomic and plumage colouration data and uncover introgression across the hybrid zone. Coalescent modelling uncovered high levels of introgression, suggesting that reproductive isolation is weak and that the geographically restricted R. melanogaster will likely become extinct due to high levels of unidirectional gene flow from lowland R. carbo.}

\normaltalk{What's Cookin' in the Urban Tropics? a View from Latin America}{Ian MacGregor-Fors}{Urbanization poses important ecological pressures, directly and indirectly affecting avian communities. Yet, an important body of knowledge has documented that urban areas retain an important proportion of the world's avifauna, with interesting behavioral, ecological, and even evolutionary processes occurring. Approximately two decades after the consolidation of urban ecology as a discipline, research remains heavily biased towards temperate countries, with an important dearth of knowledge from tropical and subtropical regions. Among such regions, Latin America is of important concern, as it houses more than 500 million urbanites, has four of the most populated global mega-cities, faces extremely unequal social and economical scenarios, and houses the highest bird species richness worldwide. Fortunately, an increasing number of studies from the region have started to populated the literature. In this talk, I will go through the urban bird knowledge generated in Latin America and set it in the context of the global literature, highlighting similarities and dissimilarities in the recorded findings.}

\normaltalk{Mobilizing Motus: Strategies for Maximizing Efficacy for Conservation Science.}{Stuart A Mackenzie, Catherine Jardine, Danielle Ethier, Denis Lepage}{The Motus Wildlife Tracking System is a collaborative research network that uses coordinated automated radio telemetry to facilitate landscape-scale research on the ecology and conservation of migratory animals. Over the past 5 years Motus has grown to facilitate more than 300 projects, maintaining a network of more than 900 receiving stations across 28 countries. The database comprises more than 1 billion detection records from 23,000+ individuals of more than 220 species of birds, bats, and insects. These data have contributed to more than 100 peer-reviewed publications across a wide range of disciplines. Motus has made a substantial contribution to our understanding of numerous ecological systems and is furthering our efforts to direct conservation action. The growth and success of Motus to date has largely been organically driven by the priorities and ambition of individual research groups rather than a unifying vision. Numerous partners have made substantial investments to support the development of the network, but those have been comparatively small when compared to the potential of a coordinated and strategic design. As the geographic extent of the system, number of collaborators, projects, and technology continues to advance, there is an increasing need for strategic and coordinated approaches to infrastructure deployment and maintenance, project design, and data analysis tools in order to maximize efficacy for the benefit of targeted applications and conservation science. To this end, we will present a summary of potential future directions for Motus, and how the network can develop over the next decade.}

\normaltalk{Assessing the Consequences of Endogenous Factors in the Recent Range Expansion of Great-Tailed Grackles using a Bayesian Network}{Maggie P MacPherson, C J Logan}{Mapping species distributions in correlation with environmental variables has historically paired knowledge of species-habitat relationships with that of global change without information about the evolutionary underpinnings shaping distributions. We hypothesize that the capacity of species to change their geographic ranges is additionally limited by endogenous variables (e.g., individual abilities, species interactions). To identify causes of range limitation, we reviewed avian species distribution models including empirical or theoretical explanations of range limits (n=1310 articles). As landscape data is more accessible than endogenous data on responses to global change (e.g., genetic/metabolic pathways determining responses to change), we found a focus on correlations between abiotic selection pressures (e.g., landscape/climate variables) and species presence. Endogenous mechanisms and biotic interactions (e.g., dispersal or metabolic limitations, and competition) were invoked without evidence post-hoc as justification for range edges defined by geographic barriers. When species presence-abiotic variable relationships were not found, behavioral flexibility or diet-switching were invoked as post-hoc explanations for observed distributions. We identified seven variables involved in limiting ranges: behavioral flexibility/cognition, disease pressure, dispersal ability, food/habitat availability, genetics, life history strategy, and species interactions. We built a causal (Bayesian) network model to investigate range limits in relationship to these variables. We validated the model with great-tailed grackle data on all variables to identify drivers of their recent rapid range expansion. Although difficult and time-consuming to collect, we discuss the importance of data on endogenous variables for creating more robust predictions for species distributions under global change.}

\normaltalk{Improving Our Understanding of Migratory Bird Distributions in the Americas by Informing a Spatially Explicit Bayesian Network Model with Geolocator Data from Two Tyrannus Species}{Alex E Jahn, Maggie P MacPherson}{Remotely sensed abiotic landscape variables are used in the vast majority of models identifying species ranges, and the Eltonian Noise Hypothesis (ENH) states that species ranges at large extents and coarse resolutions are determined principally by abiotic factors. Bayesian networks are a flexible tool that facilitates testing the ENH by including the strength of both abiotic and biotic factors together in a causal framework. We tested the ENH using location data from geolocators deployed on 2 Tyrannus species. We identified potential causal relationships between abiotic and biotic factors driving species-habitat dependencies from a literature review of the ecology of T. tyrannus and T. savana. We then combined location data from geolocators with species-habitat knowledge to build a spatially explicit Bayesian network using the R package `daggity'. We used National Centers for Environmental Predictions remotely sensed data (2.5 degree resolution) for abiotic factors of known importance to these species, and adjusted the strength of all factors as appropriate for each annual life history stage. Our variable results will be discussed in the context of using locations from geolocator data (rather than observational data) and with respect to each life history stage. Testing the ENH is important for gauging the ability of species to respond to climate change by including both abiotic and biotic factors into a single model. In practice, our model identifies biotic factors that contribute high uncertainty to current range maps, highlighting the importance of collecting smaller-scale monitoring data to improve our understanding of migratory bird distributions.}

\normaltalk{Understanding How and Why People Feed Birds: Implications for Future Research on Supplemental Feeding}{Rachael P Mady, Victoria Y Martin, Ashley A Dayer, David N Bonter}{More than 57 million people feed wild birds in the United States, but we know little about how and why they feed them. Understanding the availability of food provided for wild birds could inform how researchers study the potential effects of this supplemental resource for birds. Further, much of our understanding of bird feeding is based on what people who are highly engaged in the practice (e.g. citizen scientists) do. We do not know if their feeding practices or motivations for feeding birds reflects the wider population feeding birds. To better understand food availability and the differences between those who participate in citizen science and those who do not, we surveyed (1) a representative sample of U.S. adults who feed wild birds around their home (n=1,096) and (2) participants in a citizen science program (Project FeederWatch, n=1,280). The most common types of food provided by the general population were an ``economy" mix of seeds and grains (58\% of respondents) and sugar water (55\%), while the most common types of food provided by FeederWatch participants were suet (89\%) and sunflower seed (80\%). The two groups also differed in how consistently they provided different food types. Both groups were strongly motivated to feed birds for personal benefits (e.g. joy, relaxation), however the strength and other types of motivational factors differed. Supplemental bird food is not always a consistent resource, and what citizen scientists do only partially reflects how and why the wider population feeds wild birds.}

\normaltalk{Diet Influences Metabolic Rates in Neotropical Birds}{Maria L Mahecha, Natalia Gutierrez, Gustavo Londo\~{n}o, Daniel Cadena, Camila Gomez}{Metabolic rates are an important metric of physiology because they serve as proxies for the energy that organisms need to survive. Metabolic rates are affected by intrinsic characteristics of organisms and by environmental factors. For instance, diet could affect metabolic rates due to differences in energy contents and digestibility of food items. Here we test whether diet affects metabolic rates in Neotropical birds. Between June 2017 and November 2018, we took measures of three components of metabolic rates (basal, maximum and aerobic scope) for 118 individuals from 28 species in three different localities in Valle del Cauca (Alto Anchicay\'{a} (600 m), Cali (1000 m), and Felidia (2400 m)). For each species, we obtained values of percentage of fruit and invertebrates in their diet (Wilman et al 2014). We then performed phylogenetic corrected linear models to test relations between diet, metabolic rates,, and elevation. We found that birds from higher elevations had higher metabolic rates than those from lower elevations. Basal metabolic rate (in all elevations), maximum metabolic rate and aerobic scope (in lower elevations) was lower in species that eat more fruit. Whereas, maximum metabolic rate and aerobic scope in high elevations did not change with diet. These patterns may be related to compound digestibility, absorption of energy by birds and the different challenges of living at high elevations (e.g thermoregulation). This study helps to understand different ecological factors that affect the physiology of Neotropical birds, and the importance in the energetic constrains of different life-history traits in birds.}

\normaltalk{Assessing Geographic Variation in Song Structure and Plumage Coloration in the Willow Flycatcher (Empidonax Traillii) Species Complex}{Sean M Mahoney, Matthew W Reudink, Bret Pasch, Tad C Theimer}{Animals communicate their fitness as potential mates through various modalities including acoustic and visual signals. Divergence in these signals can be an important driver in speciation and can lead to reproductive isolation when characters between populations become too different for populations to recognize each other as the same species. Rigorously quantifying character divergence has recently gained more attention in ornithology, specifically with species of conservation concern such as the endangered Southwestern Willow Flycatcher (SWFL, Empidonax traillii extimus), a subspecies of the Willow Flycatcher (E. traillii). The subspecies designation was largely based on genetics, because the mtDNA C-haplotype is more frequent in SWFLs. However other aspects of the SWFL's biology such as song and plumage coloration are incompletely known and have therefore been used to challenge its subspecific and endangered status. We assessed character divergence among willow flycatchers by 1) quantifying subspecific song structure differences, 2) testing for subspecific song recognition in male flycatchers using simulated territory intrusion playback experiments, and 3) assessed plumage differences among flycatcher subspecies by measuring spectral reflectance on museum specimens using spectrophotometry. We found 1) song differed among subspecies and SWFLs sang the most unique song, 2) individual birds responded more aggressively to songs that were structurally more similar to their own during playback experiments, and 3) when modeled in tetrahedral colorspace, subspecies did not exhibit plumage differences. Our results suggest SWFL song is diverging and therefore deserve continued protection under the Endangered Species Act.}

\normaltalk{Conformity to Bergmann's Rule in Birds Depends on Nest Design and Migration}{Mark C Mainwaring, Sally E Street}{Species' geographic ranges and range limits are thought to be determined by climate, and across climatic gradients the morphology of populations varies non-randomly. Ecogeographic rules seek to characterise such variation, with Bergmann's rule positing that organisms inhabiting colder environments are typically larger-bodied than those inhabiting warmer environments. While Bergmann's rule has been supported across a range of taxonomic groups, how organisms' behaviour may moderate its effect remains unclear. Here we investigate whether conformity to Bergmann's rule among birds of the Western Palearctic varies in relation to nest design and migratory behaviour, using phylogenetic comparative analyses. We test predictions using data on nest structure and location, migration, body mass, latitudinal distribution, environmental temperature and phylogenetic relatedness for a sample of $>$500 species. We find that conformity to Bergmann's rule depends on migratory behaviour: non-migratory species breeding at colder, more northerly latitudes are larger-bodied, while body mass is unaffected by climate in short- and long-distance migrants. Among non-migratory species, conformity to Bergmann's rule depends, further, on nest design: species with more open nests, who are thus most exposed to adverse climatic conditions while breeding, conform most strongly to Bergmann's rule. Our findings suggest that enclosed nesting and migration enable smaller bodied species to breed in colder environments than their body size would otherwise allow. Therefore, we conclude that organisms' behaviour can affect exposure to environmental selection pressures.}

\normaltalk{Locomotion and Foraging Strategies in Shorebirds of South Texas}{Jessica Majors, Karl Berg, David Newstead}{A well-known tradeoff between wingbeat frequency and body mass explains much of the diversity of flight styles in birds. Avian bipedalism is a more ancient adaptation, but has received less attention. Agile fliers (e.g. swifts, swallows, hummingbirds) often display limited bipedal abilities, while those with limited or no flight capability often have well-developed bipedal running (e.g. chickens and ostriches), suggesting tradeoffs in locomotor modality. Shorebirds (Charadriiformes) have well-developed flight and bipedal running, but whether these behaviors are correlated across species or exhibit performance tradeoffs within species is unknown. I used high-speed video, radio-telemetry and Optimal Foraging Theory to better understand locomotor strategies in wild Sanderlings (Calidris alba) and Piping Plovers (Charadrius melodus) in the Rio Grande delta and adjacent Gulf of Mexico shoreline in extreme South Texas. Preliminary results indicate that wingbeat and stride frequency are negatively correlated within both species, supporting the Principle of Allocation. If more efficient dispersal between foraging sites results in improved foraging opportunities, there may be selection for optimal biomechanical strategies. Understanding the movement ecology in shorebirds can help identify fitness tradeoffs in coastal populations under increasing pressure from development and rising sea-levels. If data are similar between these two species, Sanderlings could act as a model for threatened shorebird species, like the Piping Plover.}

\normaltalk{Using Genomics and Citizen Science to Investigate Hybridization of Parrots in Southern California}{James M Maley, Margaret E Schedl, Brooke Durham, Whitney LE Tsai, Ryan S Terrill, Kimball Garrett, John E McCormack}{Southern California is home to a large population of naturalized parrots, all former pets or descendants of pets. There are at least six species of Amazona parrots that are known or suspected of breeding in Los Angeles County. The most abundant species is Red-crowned Parrot, numbering in the thousands, followed by Lilac-crowned Parrots, likely numbering in the high hundreds to low thousands. These two close relatives are allopatric in their native ranges, but breed in the same areas of Los Angeles. Phenotypic hybrids have been suspected to occur but we wanted to determine if the two species are hybridizing based on genomic data. We also wanted to determine the abundance and extent of phenotypic hybrids, so we used the citizen science database iNaturalist to map putative hybrids based on photographs. To determine the level of genetic admixture we sequenced ultraconserved elements for both species collected between 1934 and 1955 from their native ranges. We also sequenced UCEs for 22 specimens salvaged in Southern California. Based on 25,647 single nucleotide polymorphisms we found evidence of several hybrids between the two species. There appears to be introgression from Lilac-crowned Parrots into Red-crowned Parrots, the extent of which requires further study to determine. Putative hybrids appear to be uncommon, but occur throughout the overlapping distributions of these two species in their nonnative range and parts of their native ranges where escapees have established. It remains to be seen if species boundaries will remain, or if these two lineages will merge.}

\normaltalk{Greater Roadrunner Habitat Selection in a Coastal Prairie-Brushland Mosaic: Implications for Coastal Prairie Restoration}{Derek R Malone, Clint W Boal, Terry L Blankenship}{The Greater Roadrunner (Geococcyx californianus) is a charismatic cursorial predator that requires a combination of open areas for foraging and brush for perches and nesting. Surprisingly, little quantitative data are available for roadrunner habitat selection, especially in coastal prairies and brushlands of the Gulf Coast. We are assessing how coastal prairie restoration efforts on the Welder Wildlife Refuge, in San Patricio County, Texas, may influence roadrunners by quantifying home-range and habitat selection on the refuge. Our goal is to understand how prairie restoration efforts can contribute toward grassland bird community recovery, while also accounting for species, such as the roadrunner, that require a mixture of vegetation communities. We used VHF transmitters to determine breeding season home range size and estimate habitat use for 17 roadrunners in 2018 and 2019. We obtained triangulated estimates of roadrunner locations during May -- August of each year. We examined home-range size and habitat selection using programs LOAS, the HRT 2.0 extension, and remote sensing data. Roadrunners had a median home range 20.4 ha and a 50\% core range of 5.5 ha. Selection ratios indicate that roadrunners selected for dense grass and bare ground and avoided sparse grass at the 1st order. At the 2nd order, roadrunners selected for dense grass and avoided sparse grass and bare ground. For the 3rd order they selected for sparse grass and bare ground while avoiding dense grass and shrub land cover types. We will be collecting additional data for 11 more individuals in 2020 to include for presentation.}

\normaltalk{Migration Ecology of Western Gray Catbirds}{Kristen A Mancuso, Megan E Fylling, Karen E Hodges, Michael B Lancaster, Christine A Bishop, Kate R Stone}{The Gray Catbird (Dumetella carolinensis) is a neotropical migrant that breeds throughout the USA and Canada, and overwinters in southern USA, the Caribbean, and Mexico. Little is known about the migration ecology for birds breeding at the northwestern edge of their range. Western populations are of interest because their range spans a large physical barrier - the Rocky Mountain Range. We investigated migration routes and overwintering sites of two western populations separated by 500 km: one west of the Rockies in British Columbia, Canada, and one within the Rockies in Montana, USA. Additionally, we examined migratory connectivity for these two populations. To track the movement of individual catbirds, we used GPS tags and geolocators. Catbirds migrated across the Rockies and tended to use low elevation corridors. Both populations followed an indirect, ancestral route eastward then southward to overwintering locations in Mexico and Texas. Individual catbirds used several distinct overwintering sites, progressively moving farther south to each new site. Tamaulipas was the state used the most during the overwintering period, but catbirds also overwintered in Texas, Veracruz, San Luis Potosi, Puebla, and Hidalgo. There was considerable overlap of overwintering sites for catbirds from both breeding sites, indicating weak migratory connectivity. Our research provides insight into the putative ancestral origins of the migration route in western Gray Catbird. Additionally, we have revealed the links between breeding, migration, and overwintering life stages, contributing to our understanding of their full annual cycle.}

\normaltalk{Who Benefits the Most from a Social Party, Those Who Host it or the Guests?}{Giselle Mangini, Facundo A Gandoy, Juan I Areta, Pedro G Blendinger}{Mixed-flocks are a ubiquitous phenomenon across forest environments involving species with different roles and ecological requirements. In these flocks, central species attract and lead the movement of the flock, and follower species follow the central ones. But, those who leads want to be followed despite probable resource competition? Two main evolutionary advantages have been proposed as motivations behind flocking behavior: better surveillance against predators and increased foraging efficiency. However, it is unknown whether central and follower species benefit equally from flocking. In order to identify the central species and to evaluate the foraging benefits that species might be perceiving in relation to their role, we surveyed mixed-flocks for 388h in the Argentina Yungas-foothill forest across 3 years. We recorded the species that were leading the movement in each mixed-flock and the foraging efficiency of each observed bird individual inside and outside mixed-flocks. Because central species behavior implies more than leading the movement, we compiled all our mixed-flocks data and performed network analysis to rank the most central species based on the probability of each species of being central under the neighborhood inclusion criteria. We found that species with higher leading behavior were also the most central by network analysis. Global foraging efficiency increased by 24\% when integrating mixed-flocks in comparison to efficiency outside them, with the central species benefiting the most while some followers were hampered. This result indicates that central species are seizing the foraging benefits that mixed-flocks offer whilst follower species could be getting better surveillance against predators.}

\normaltalk{Differential Migration is not Constant Across Time or Space}{Lisa L Manne, Marlen A Alamo, Jonathan Corazza, Shannon R Curley, Justyna Karas, Jose Ramirez-Garofalo, Aleksandra Reyfman, Zihe J Wang}{Differential migration, in which females of a species migrate different distances than males, has several proximal explanations.  It may have developed as a way for males to overwinter nearer to anticipated breeding season territories; individuals in better condition can migrate shorter distances away from breeding grounds, or displace individuals that are in lowered condition or of lowered social status.  Differential migration has been demonstrated for 50+ species worldwide, with another 90+ species as candidate differential migrants.  The fitness consequences of differential migration can be positive (males overwinter near to breeding territories) or negative (a disaster on wintering grounds removes most of a class of individuals from the population).  Here, using data from the US Bird Banding Lab, we study several Zonotrichia species (Z. albicollis, Z. querula) and subspecies (of Z. leucophrys) and we address the hypothesis that differential migration remains similar over time and space.  We compare the time periods 1980-1999 ("early") and 2000-2019 ("late"), and different migratory flyways.  We find that differential migration (by sex) is generally found in the early period, and can differ among flyways, but that in the late period, differential migration for these species and sub-species had often broken down.  We investigate drivers for these different outcomes, including age and sex of the individuals, population trends of the species/subspecies, and climatic drivers.  These results call into question the constancy of selective processes across a longitudinal gradient, and require managers to further consider changing conservation needs across different parts of the avian life cycle.}

\normaltalk{Genomic Architecture Shapes Phylogeographic Population Structure Across the Genome of the Brown Creeper (Certhia Americana)}{Joseph D Manthey, John Klicka, Garth M Spellman}{Birds' genomes are highly heterogeneous in their structure and content, including variation in chromosome size, gene and repetitive element density, and local recombination rate. In eukaryotic model systems, it has been demonstrated that variation in gene content and recombination rate may influence population structure in different parts of the genome. Here, we aim to test whether genomic architecture interacts with population demographics and results in heterogeneous population structure across the genome in birds. We use a Brown Creeper (Certhia americana) genome in conjunction with phylogeographic genomic sampling across the Brown Creeper's geographic range to assess the impact of genomic architecture on population structure. We found that ~60\% of the variation in phylogeographic population structure in the Brown Creeper can be explained by variation in genomic architecture. Genomic regions with relatively low recombination rate and fewer genes exhibit the overall population structure of the species. In contrast, genomic regions with relatively high recombination rate and more genes exhibit variation in population structure, largely distinguishing only small, isolated populations. More specifically, populations with relatively small long-term effective population sizes appear to exhibit strong genetic drift and divergent population structure in regions of the genome with high recombination rate. Overall, our results show that population structure can strongly vary across the genome predictably with the characteristics of the genome, necessitating whole-genome approaches for phylogeographic and population genetic studies in birds to prevent biased estimates of populations' and species' evolutionary histories.}

\normaltalk{For the People and for the Birds in the Americas -- Celebrate Urban Birds}{Josmar E Marquez, Karen Purcell, Marilu Lopez Fretts}{Celebrate Urban Birds (CUBs), a bilingual citizen science project from the Cornell Lab of Ornithology, strives to co-create inclusive, equity-based citizen science projects that serve communities that have been historically excluded from birding and citizen science. The project seeks to improve quality of life and promote better science by including missing voices, perspectives, and experiences. CUBs works with people of diverse ages, backgrounds, and cultures to co-create birdwatching and environmental education programming through the arts, local culture, and a focus on well-being and participatory science. In the last decade, CUBs has worked with more than 14 thousand community-based organizations throughout the Americas (over 90\% working with underserved communities), engaging more than half a million people in the sciences and birdwatching. CUBs has developed its methodology through years of community-led research focused on equity, diversity, and inclusion, funded by the National Science Foundation.     We share our methodology and key research results that have led the project to emphasize equity, transparency, trust and communication and actively work to create benefit to both the participating communities and the sciences. We highlight one of CUBs community partners, AveZona, and share our collaborative, co-created approach in Venezuela. Including the co-created local materials, supporting workshops for teachers and community leaders, and integrating the arts in impactful ways to reach low-income, marginalized communities. This approach not only applies to Venezuela, but to all of Latin America where CUBs is working side-by-side with local communities to build trust and create better impact.}

\normaltalk{Revisiting the Structure and Organization of a Peruvian Amazonian Forest Bird Community after Three Decades}{Ari E Martinez, Jose Miguel Ponciano, Thomas Valqui, Blaine Carnes, John Fitzpatrick, Juan Pablo Gomez, Eliseo Parra, Jacob Socolar, John Terborgh, Scott Robinson}{Documenting patterns of spatial and temporal change in bird communities remains a challenge for tropical ecology. In 1990, Terborgh et al. quantified the structure and organization of an Amazonian bird community on a 97 ha. plot in southeastern Peru. We revisited the same plot in 2018 to evaluate community-wide change after 30-years by repeating the same combination of methodologies as the original study. These methodologies included spot mapping, mist-netting, marking and following multi-species foraging flocks (MSFs) and point counts. We took a multi-pronged approach to compare abundances and estimated densities of over 130 species representing traditional territories. Our results suggest a majority of species have changed relatively little in their distribution on the plot. However, surveys of canopy MSFs found only one of  five MSFs from the original census. Conversely, understory mixed-species flock territories, almost saturate the plot in nearly  the same locales as the original study, excepting those that occupied microhabitats subject to rapid change (i.e. bamboo and river edge). For instance, one leading flock species which prefers bamboo habitats appears to have disappeared from the majority of flocks. This suggests that species occupying habitats insulated from fluvial dynamics are less likely to experience drastic changes in abundance.  The re-census effort highlights the challenges of attempting comprehensive long-term monitoring of hyper-diverse biological communities without detailed systematic time series samples. We suggest directions moving forward for the sustainability of long-term monitoring efforts of tropical bird communities.}

\normaltalk{Factors Influencing the Breeding Success of Brown Pelicans (Pelecanus Occidentalis) in Coastal Louisiana}{Juita Martinez, Paul Leberg}{During the 1960s brown pelicans (Pelecanus occidentalis) were considered extinct in the state of Louisiana. As a result of human reintroduction and translocation, Louisiana's brown pelican population peaked in 2005 but have been in flux ever since. As a long-lived species with an extending provisioning period we utilized motion activated cameras in 5-10 plots per island in order to monitor nests across Terrebonne and Barataria Bay, Louisiana. Three out of the five islands have been restored since the 1990s. With almost daily accounts of each nest throughout the season we are able to quantify nest success in terms of the number of fledglings per nest, factors contributing to nest failure and daily survival rates. We modeled the effects of environmental and ecological factors on nest success across all five islands. Brown pelican chicks were found to have a higher probability of at least one chick fledging on restore (R) islands vs. unrestored (UR) islands (p $<$0.01). We observed the probability of nest success was greater on a mixture of restored and unrestored islands, Philo Brice island(UR) (p = 0.01), Queen Bess island(R) (p $<$ 0.01) and Raccoon island(R) (p $<$ 0.01) as compared to Brandy(R) and Felicity island(UR). Measuring the reproductive success of these colonies will provide valuable insights into factors affecting variation in nest success and habitat use across the coastline to better inform future restoration projects and practices.}

\normaltalk{The Bioregion of the Llanos: An Area of <u+200b><u+200b>importance for Acuatic Birds. Case: Wetlands in Agrosystems}{Alexis Araujo Quintero, Margarita Martinez}{In the Central-Western Llanos of Venezuela, twice a year between 2006 and 2020, different wetlands have been monitored: rice fields, dams, prestamos and rockeries through direct counts of acuatic birds families recognized for Venezuela. As a result of these censuses, 21 families were found with 97 species: 75 residents and 22 migratory. 1,727 records of 68,485 individuals were accumulated in 40 places in five states. The most diverse and abundant families were Ardeidae and Scolopacidae, the latter represented mostly by neartic migrants and followed in order of importance by Threskiornithidae, Anatidae and Charadriidae. The greater abundance and richness of species coincides with the formation of colonies during the reproductive period in the rainy season for residents, and the presence of migrants is mainly conditioned to the cycles of rice cultivation (Oryza sativa), the genus Calidris is the most represented. It is important to highlight the presence of predominant aquatic species in this type of inland wetlands that migrate locally such as Himantopus mexicanus and Eudocimus ruber; in addition to those with Near Threatened (NT) status such as Cercibis oxycerca and E. ruber. On the other hand, the species of the genus Dendrocygna and Porphyrio martinicus are considered plagues for rice fields and Mycteria americana as game-bird whereby are eliminated by farmers. In relation to the importance of agricultural development, censuses should be continued in these wetlands because they house large concentrations of ducks as well as rockeries, where mixed reproductive colonies of storks, ibises and herons develop.}

\normaltalk{Exploring Population Dispersal in Wintering Grounds: Long-Term Data Analysis Suggests Low Body Condition and Short Theoretical Flights of Northern Waterthrush Crossing Portachuelo Pass During Fall Migration at Northern Venezuela.}{Miguel E Matta, Miguel Lentino, Sandra Giner, Renato De Nobrega}{During long distance flights, migratory birds use energy stored as fat and the amount they transported shape the limit of displacements between sites part of the annual route. In the Neotropic, few long-term studies have informed weight, body condition (ratio between weight /wing chord) and fat scores in relation to the capability to continue flights, especially when birds cross ecological barriers such as the open sea, arid areas or mountain ranges. Facing different group of migratory birds declining in the Western Hemisphere, we must to determine the energy requirements associated with migration. We evaluated the potential flight range of Northern Waterthrush (Parkesia noveboracensis) according to their body condition and fat reserve in the Portachuelo Pass, Henri Pittier National Park (Venezuela), during autumn migration season. Portachuelo Pass Bird Observatory has recorded biometric information of 3,441 individuals (adults and juveniles) between 1990 and 2017. The averages of weight (AHY 14.6 g; HY 14.4 g), body condition (AHY 0.19; HY 0.20) and potential flight distance (AHY 228.4 km, HY 219.7 km) were similar between ages groups. Considering all NOWA, 56.5\% of the total were classified in the zero-fat score and the average fuel load variated between 6,2-8.6\%. In general, considering all initial capture conditions of P. noveboracensis in the Portachuelo Pass, it would be necessary to make new refueling stops if the final wintering destinations were located outside Venezuela. Future actions implementing new tracking technology can validate our results of NOWA dispersion at northern South America.}

\normaltalk{Systematics, Species Delimitation and Mitochondrial Introgression in the Narcissus Flycatcher Complex.}{Herman L Mays, Isao Nishiumi, Bailey D McKay, William Mauck}{The Narcissus Flycatcher complex has traditionally comprised three taxa; the migratory Ficedula narcissina narssina in temperate Japan and the Russian Far East, the sedentary Ficedula narcissina owstoni in the Ryukyu Islands, and the migratory Ficedula narcissina elisae in Northeast China. Prior studies using song, plumage, morphology, mitochondrial DNA and three nuclear DNA loci have justified elevating each of these subspecies to the rank of species. However, this work involved limited sampling, particularly for F. n. narcissina and F. n. owstoni. Here we report a larger molecular sampling of 66 individuals across F. n. narcissina and F. n. owstoni, over four islands for each taxon, and employing 11 loci (one mtDNA and 10 nuDNA). We also collected plumage color data from museum specimens using UV-sensitive photography. Prior work suggested that variation F. n. owstoni was clinal with populations on the northernmost Ryukyu Islands being more similar to F. n. narcissina, but prior studies had no molecular data from this area where the two parapatric taxa meet. In addition to confirming many of the patterns found in prior work suggesting the F. n. narcissina and F. n. owstoni represent distinct species-level lineages we found that the northern population of F. n. owstoni shares mitochondrial haplotypes with F. n. narcissina. This study adds to our understanding of the population history and systematics of the Narcissus Flycatcher complex and suggests that the northernmost populations of F. n. owstoni may be an example of mitochondrial capture between a sedentary and migratory species.}

\normaltalk{Toward Reconciliation of the Four World Bird Lists: Hotspots of Disagreement in Taxonomy of Raptors}{Christopher J McClure, Denis Lepage, Leah Dunn, David L Anderson, Sarah E Schulwitz, Leticia Camacho, Bryce Robinson, Les Christidis, Tom S Schulenberg, Marshal J Iliff, Jeff Johnson}{Taxonomy is foundational to conserving the world's biota, and as such taxonomic disagreements can hamper conservation efforts. There are currently four world bird lists referenced by different stakeholders including governments, academic journals, museums, and citizen scientists. Consolidation of these lists is a priority. In reconciling lists, care must be taken to ensure agreement in taxonomic concepts--the actual groups of individual organisms circumscribed by a given scientific epithet. Here, we compare species-level taxonomic concepts for raptor across the four lists, highlighting areas of disagreement. In total, there were 665 species-level raptor taxa occurring on at least one list. Of these, only 453 (68\%) were agreed upon across all lists. The Howard and Moore Checklist of the Birds of the World contains the least species (528), whereas the IOC World Bird List contains the most (580), and these two lists are in the most disagreement. Of the disagreements, 67\% involved owls, and Indonesia was the country containing the most disagreed-upon species (169). Finally, we calculated the amount of agreement across lists for each avian Order and found raptor Orders spread throughout the rankings of agreement. Our results emphasize the need to reconcile the four world bird lists, highlight broad disagreements across lists, and identify hotspots of disagreement for raptors, in particular. Our methodology should serve as a template for reconciling the four world bird lists.}

\normaltalk{Historical Museum Specimens and Modern Citizen Science Data Reveal 100 Years of Change to Mexico's Birdlife}{Ryan S Terrill, James M Maley, Morgan W Tingley, John E McCormack}{Earth's environment is changing at an unprecedented and accelerating pace. To understand what is being lost, natural history specimens housed in the world's museums can provide a baseline for past biotic communities, allowing for a comparison to what we observe today. From 1933 to 1955, Chester C. Lamb (1882-1965) crisscrossed Mexico, collecting 39,000 bird specimens from over 300 locations for Robert T. Moore (1882-1958), a Los Angeles based ornithologist. In the ensuing decades, Mexico's landscape was radically transformed by logging, conversion to agriculture, and other major industrial projects. Here, we reveal some preliminary results from the Mexican Bird Resurvey Project, a country-wide assessment of 100 years of change to Mexico's birdlife that compares Lamb's specimens to modern citizen science observations from eBird. We focus on species that appear to be in decline across Mexico and what these species might tell us about trends in habitat loss and conservation priorities. We also discuss the methodological complications of comparing specimen data from a single collector to citizen science data collected by many observers.}

\normaltalk{Simply Complex: Variation in Manakin Displays via Networks, Lossless Compression, Fractal Dimension and Entropy}{David B McDonald, Nicholas J Oakley}{The spectacular courtship displays of Golden-winged Manakins in Ecuador vary in context-dependent fashion. We quantified the sequences of behavioral elements via lossless compression, ethogram network analysis, entropy and Zipf's Law. Overall, displays leading to copulations were ``simpler," had fewer elements that were repeated more often, and formed sparser etho-networks. Lossless compression (via Huffman encoding and the LZ77 algorithm) was useful in distinguishing the three display contexts: male-only displays, female-present displays, and those leading to copulations. Copulation bouts were the most compressible. We suggest that lossless compression may be useful in other situations, such as in comparing vocal repertoires. In partial contrast to the idea of greater simplicity for copulation bouts, we found, using Zipf's Law, that copulation bouts have a higher fractal dimension (arguably meaning greater complexity) than do male-only bouts. Nevertheless, high fractal dimension might also qualify as organizational simplicity -- a pattern within a pattern within a pattern. Thus, whether the different displays are more complex or simpler, depends in part on the analytical approach. Fun footnote: high-speed video shows that the ultra-rapid log-approach display is the avian equivalent of a full-front layout, with half twist, in human gymnastics.}

\normaltalk{An Evaluation of Avian Use of Marsh Terraces in Gulf Coastal Wetlands}{Madelyn McFarland, Brian Davis, Michael Brasher, Mark Woodrey, Larry Reynolds, Fernanda Vizcarra}{Louisiana's coastal wetlands support millions of migratory birds annually. However, Louisiana has experienced 90\% of the total decline of coastal wetlands within the continental United States, accounting for most of the loss of Gulf Coastal wetlands. Marsh terracing is one method used to combat coastal wetland loss. The restoration technique uses in situ sediment to construct segmented ridges in open water areas of coastal wetlands. An objective of marsh terracing is to improve marsh conditions and habitat for a diversity of species. Despite terraces being an increasingly useful component of coastal restoration efforts, previous research on their value as waterbird habitat is limited in spatial and temporal scale. Using both ground and aerial surveys, our study evaluates avian use of marsh terraces across multiple paired sites (terraced and non-terraced) in coastal Louisiana. Avian monitoring efforts focus on two primary guilds of birds, breeding secretive marsh birds and wintering waterfowl. Results from the first field season indicate that: 1) terraced sites were used predominately by non-focal species such as red-winged blackbirds, 2) there was low use of terraced sites by focal species such as rails, 3) and there was generally low use of both terraced and non-terraced sites by wintering waterfowl, although species abundances varied in space and time. Field efforts are ongoing, and data collection will be completed by July 2020. Future analysis will examine relationship between avian use and habitat characteristics of study sties (e.g., submerged aquatic vegetation, diversity and structure of emergent vegetation).}

\normaltalk{A Full Annual Cycle Perspective for Analyzing Stopover Monitoring Data: Integrated Population Models for Arctic-Breeding Shorebirds}{Anna M Tucker, Conor P McGowan, James E Lyons, Audrey DeRose-Wilson, Nigel A Clark}{Consideration of the full annual cycle of migratory species can provide useful insight into ecological research and conservation efforts. Unfortunately, collecting data needed to statistically estimate demographic parameters is often logistically difficult. For example, for species that breed in remote areas monitoring is often conducted during migratory stopover or at nonbreeding sites and fecundity can be difficult to estimate directly. Integrated population models (IPMs) provide a method for inferring demographic rates of interest using data collected through existing monitoring efforts from other points of the annual cycle. Here we present an IPM that uses mark-resight and count data collected during migratory stopover to estimate population growth rate and recruitment for two species of Arctic-breeding shorebirds. This IPM includes an open robust design submodel to account for the flow-through dynamics of migratory stopover and adjust expected counts. We use this model to estimate adult survival probability, population growth rate, and recruitment rate for two Arctic-breeding shorebirds, the red knot (Calidris canutus rufa) and ruddy turnstone (Arenaria interpres), using monitoring data collected during spring stopover in Delaware Bay, USA from 2005-2014. We found red knot and ruddy turnstone populations were most likely stable from 2005-2014; adult survival probabilities were consistently high (red knot: 0.90 (0.86, 0.93); ruddy turnstone: 0.93 (0.89, 0.96)) and average recruitment rate was 0.34 (0.01, 2.1) for red knot and 0.15 (0.001, 1.2) for ruddy turnstone. Our approach demonstrates the utility of IPMs for understanding full annual cycle population dynamics, even when data are only available from one point of the annual cycle.}

\normaltalk{Agriculture is Adapting to Phenolgical Shifts Caused by Climate Change, but Grassland Songbirds are Not.}{Maeve M McGowan, Noah G Perlut, Allan M Strong}{Migratory birds time their migration based on ecological cues that signal resource availability for offspring. However, as the climate changes, the timing of seasonal events may shift, and those that are changing are expected to do so more dramatically, potentially inhibiting the ability of some species to use them as accurate cues for migration.  We studied the relationship between phenological shifts and reproduction by long- and short-distance migratory songbirds--Bobolinks (Dolichonyx orzivoryous) and Savannah Sparrows (Passerculus sandwichensis)--in the context of a rapidly changing climate. This population breeds in hayfields and pastures in Vermont, USA, where farmers are also responding to climate change. From 2002-2019 we monitored nest initation dates and nest survival. We collected historical and projected precipitation and temperature data for the breeding grounds, and their respective wintering and stopover sites and historical and future projections of the North Atlantic Oscillation (NAO) and the Él Ni\~{n}o Southern Oscillation (ENSO). For Savannah Sparrows, the interaction between precipitation in breeding ground and in female's wintering site (Wilmington, North Carolina) best explained variation in nest initiation date. For Bobolinks, the interaction between breeding grounds precipitation and average temperature in their primary fall stopover site (Barquisimieto, Venezuela) best explained variation in nest initation date. Nest initiation dates were stable across the study. However, first haying dates in the Champlain Valley of Vermont advanced by ~10 days. These results suggest that the conflict between the timing of hay harvests and grassland songbird reproduction will increase, further threatening population processes for these species.}

\normaltalk{A Meta- Analysis of Edge Effects on Nest Success in Forest and Shrubland Birds of Eastern North America}{Savannah L McInvale, Michael E Akresh, David I King}{In Eastern North America, populations of some species of shrubland birds and mature forest birds are in decline. Numerous studies report edge effects decrease avian reproductive success, yet variation among species, sites and nesting guilds obscures overall patterns. In order to reveal general patterns of edge effects on nesting, we performed a meta-analysis to examine the effects of edges on shrubland and mature forest bird nest survival. Analysis included data from 20 studies, and examined nest survival of 11 shrubland species and 9 mature forest species.  Our results suggest nest success declines as nests are placed closer to edges, when accounting for random effects of species and study. Additionally, when subsetting the data into mature forest versus shrubland species, our results suggest a significant relationship between mature forest bird nesting success and distance to edge, but no significant effect for shrubland birds. Further analysis will include weighting data by the number of nests in each distance bin and study. Nonetheless, our initial results indicate that mature forest species may be more detrimentally affected by edges than shrubland species. Mature forest specialists may be less adapted to edge environments, or alternatively, they may account for higher nest depredation near edges with higher juvenile survival along edges during the post-fledging period. Our findings indicate that nest survival between bird guilds varies in response to edges, and an overabundance of edges may be detrimental to some species, but not for others.}

\normaltalk{Accurate Drone-Based Nest Counts of Five Colonial Marshbird Species using a Dual Visible-Thermal Camera Approach}{Ann E McKellar, Nicholas G Shephard, Dominique Chabot}{Waterbirds are important indicators of wetland health, but certain species are challenging to survey and monitor due to their sensitivity to disturbance and the difficulty of accessing their breeding habitats. This is especially true for colonially-breeding marshbirds, for which a multi-species survey protocol that maximizes accuracy of counts and minimizes disturbance does not appear to exist. Drones may represent an ideal new tool that does not suffer the trade-offs and drawbacks of traditional methods for surveying marshbird colonies. However, complex vegetation structures at marshes and the cryptic nature of some marshbird nests make them difficult to detect in aerial imagery. We made use of synchronous high-resolution visible imagery (0.8-2 cm) and thermal-infrared imagery (6-16 cm) captured from a drone to count nests of five species of colonial marshbirds (Western Grebe, Franklin's Gull, Black Tern, Forster's Tern, and Black-crowned Night Heron) at eight breeding colonies in Saskatchewan, Canada. We compared counts from the imagery to those obtained from traditional ground-based surveys to determine the accuracy and applicability of this approach. The two types of imagery proved highly complementary for surveying nesting marshbirds, as heat signatures often helped detect and confirm nests not easily spotted in the visible imagery, while the detailed visible imagery allowed species confirmation and differentiation. For four of five species, drone-based counts were within 5\% of ground-based counts. The use of our drone-borne dual visible-thermal camera approach proved extremely promising for counting colonial marshbird nests and may have applications across a wide range of wildlife monitoring programs.}

\normaltalk{Pump it Up: How Diet and Body Condition Affect Migratory Movements of Shorebirds and Songbirds}{Scott R McWilliams, Adam Smith, Clara Cooper-Mullin, Clay Graham}{Just like rest and refueling spots along our long stretches of motorways and autobahns, well provisioned stopover sites are crucial for songbirds during their seasonal peregrinations between breeding and wintering areas. We review what we know (and need to know) about (a) how diet and body condition affect the tempo and pace of fall migration for a suite of songbirds, and (b) how body condition upon departure from breeding grounds affects the tempo and pace of movements on migration to distant wintering areas for a shorebird. We conducted a field experiment on an island stopover site to test the hypothesis that songbirds with greater fuel stores and antioxidant capacity have shorter stopovers and depart in a seasonally appropriate direction compared to lean birds with low antioxidant capacity. Fat stores and antioxidant capacity of Hermit thrush, Red-eyed Vireos, and Blackpoll Warblers (not Yellow-rumped Warblers) primarily affected stopover duration but not departure direction. We also nondestructively estimated the body composition of American Woodcock just prior to departure from southern New England breeding grounds and tracked them using pinpoint GPS transmitters to their wintering areas (along the Atlantic coast to Lousiana). Increased fat stores of woodcock upon departure were associated with more rapid and less circuitous migration routes during fall migration. In general, body condition and access to dietary antioxidants seem important for both songbirds and a shorebird during fall migration.}

\normaltalk{Global Patterns of Phenotypic Macroevolution in Parrots}{Jon T Merwin, Elliot Lee, Ryan Terrill, Leo Joseph, Brian T Smith}{Parrots (Order: Psittaciformes) exhibit vast diversity in shape, size and color, and are distributed from savannahs to tropical rainforests across five continents. While most parrots are described as sexually monomorphic, a large number of parrots exhibit dimorphic plumage and many more may exhibit hidden cryptic dimorphism in the UV spectrum. Psittaciformes broad distribution across varied environments and the heterogeneous distribution of dimorphic plumage ornaments across their phylogeny provides an opportunity to compare the evolution of ornamental and adaptive traits in a global spatial and environmental context. To characterize and model color and body size evolution, we quantified plumage color and morphometric variation across male and female parrots. We photographed and measured museum specimens (male and female) of all parrot genera, and quantified color using an automated pipeline. First, we compared the morphological space occupied by different clades of parrots in order to test whether increasing regional phylogenetic diversity was correlated with phenotypic diversity. Second, we quantified the occurrence and magnitude of sexual dimorphism across their phylogeny and assessed the extent to which this dimorphism was correlated with environmental variation. Overall, we found that while most genera independently occupied a large amount of morphological space, both closely and distantly related genera occupied similar color space and unique colors were rare. These results suggest that parrots have likely become so colorful in order to adapt to wide ranging environmental conditions across individual continents, but shared adaptive strategies (i.e., plumage mechanisms) may have led to morphological convergence at a global phylogenetic scale.}

\normaltalk{Whole-Genome Sequencing Reveals Patterns of Introgression in the Skuas (Stercorariidae)}{Else K Mikkelsen, Jason Weir}{Hybridization between species can play an important role in the evolution of birds, allowing for introgression between diverging lineages before reproductive isolation is complete. Signatures of past gene flow can remain long after hybridization has ceased, and advances in genomic technologies have now made it easier than ever to characterize patterns of gene flow across the whole genomes of birds. In this study, we use whole-genome sequencing of all seven currently-recognized species of skuas and jaegers (Stercorariidae), in order to elucidate the history of hybridization in this family of seabirds.     We analyzed the phylogenetic relationships of the skuas using whole genome sequences, and examined patterns of discordance among nuclear loci as well as between the nuclear and mitochondrial genomes. We assessed evidence for historical gene flow between pairs of non-sister species. We also examined the phylogeographic history of the group by examining the long-term history of effective population size in each species, combined with information from their dated phylogeny.     We uncover evidence for historical hybridization between the Great Skua (Stercorarius skua) of the Northern Hemisphere and a lineage of southern-hemisphere skuas, involving a substantial portion of the genome, despite the current separation of their breeding ranges in different hemispheres. We reject a hybrid origin for the Pomarine Jaeger (Stercorarius pomarinus) and Great Skua, which had previously been proposed based on patterns of mitonuclear discordance. Patterns of historical gene flow suggest a complicated history of inter-hemispheric colonization among the skuas.}

\normaltalk{Genomic Differentiation and Local Adaptation on a Microgeographic Scale in a Resident Songbird}{Chloe Mikles, Stepfanie Aguillon, Yvonne Chan, Peter Arcese, Phred Benham, Irby Lovette, Jennifer Walsh}{Elucidating forces capable of driving species diversification in the face of gene flow remains a key goal in evolutionary biology. Song Sparrows, Melospiza melodia, occur as 25 subspecies in diverse habitats across North America, are among the continent's most widespread vertebrate species, and are exemplary of many highly variable species for which the conservation of locally adapted populations may be critical to their range-wide persistence. We focus here on six morphologically distinct subspecies resident in the San Francisco Bay region, including three salt-marsh endemics and three residents in upland and riparian habitats adjacent to the Bay. We use reduced-representation and whole genome sequencing approaches to characterize the evolutionary distinctiveness and local adaptive capacity of the six song sparrow subspecies in the San Francisco Bay region and deploy this information to determine priorities for conservation. Clustering separated individuals from each of the six subspecies, indicating subtle differentiation at microgeographic scales. Evidence of limited gene flow and low nucleotide diversity across all six subspecies further supports a hypothesis of isolation among locally adapted clades. We suggest that natural selection for genotypes adapted to salt marsh environments and changes in demography over the past century have acted in concert to drive the patterns of diversification reported here. Our results offer evidence of microgeographic specialization in a highly polytypic bird species long discussed as a model of sympatric speciation and rapid adaptation, and they support the hypothesis that conserving locally adapted populations may be critical to the range-wide persistence of similarly highly variable species.}

\normaltalk{The Body Language of Birds: Avian Communication Through a Mechanistic Lens.}{Meredith C Miles, Matthew Fuxjager}{Birds are central in studies of animal communication, where researchers have spent decades trying to understand how their vocalizations and plumage color evolved. Until recently, however, we knew very little about another mechanism of communication shared by all birds: gesture, or communication via body movement. Gestures can evolve whenever a pre-existing motor pattern is co-opted for use in a social context (e.g., flapping the wings to display instead of fly), making them highly evolvable. In addition to acting as a signal in its own right, gestures can modify how plumage ornaments and vocal signals are perceived by a receiver. This presentation will briefly review what we know about the evolutionary processes that shape gesture, before focusing on a functional study in Andean cock-of-the-rock (Rupicola peruana), a large cotinga in which males dance at a lek to compete for mates. We filmed dancing males and used a linguistic approach to analysis, identifying statistical signatures of language-like properties in dance sequences. Only four major body movements are incorporated into the dance, but gestures cluster together into distinct units of dance--much like written words are formed by combining letters. We also find that dance unit frequency differs between peripheral and central dancers, and that individual males use specific dance sequences more often when a female is visiting the lek. Altogether, our data suggest that multiple language-like properties unique to primates (e.g., syntax, morphology, infinite generative capacity) are in fact present in the body `language' of birds.}

\normaltalk{Cooper's and Sharp-Shinned Hawks Fill Discrete Niches Created by the Relative Abundances of their Prey Species}{Eliot T Miller, Oisin Mac Aodha, Emma I Greig, David N Bonter, Wesley M Hochachka}{In this paper we address two broad questions that guide our interpretation of species' niches in the face of a growing trove of citizen science data. The first is one of basic biology, ``do discrete niches exist?" The second is one of observer expertise and its statistical ramifications, ``how good are observers at distinguishing similar species, and how does this vary with expertise?" These questions are inextricably linked--if we cannot be sure of the identity of our study organisms, then we cannot be sure of the strength of our conclusions. The decisions made by top predators over what prey to target have traditionally been detailed after painstaking observation in the field by trained biologists. Here, we take an alternative approach and use reports submitted by citizen scientists to quantify niche partitioning between two broadly sympatric congeneric avian predators, Cooper's and Sharp-shinned Hawks. We find that the resource base, as quantified by feeder bird body mass abundance, is multimodal and approximately lognormal. Cooper's and Sharp-shinned Hawks appear to fill these modes in the resource base, with Sharp-shinned Hawks focusing their hunting efforts on abundant small prey items, and Cooper's Hawks taking prey from the two most abundant peaks in the resource base (both the small body mass peak and a lesser peak at medium body mass). To some degree, therefore, these hawks partition niche space as defined by prey body mass, and habitat differences presumably further minimize niche overlap. Through online testing with known gold standard images of the study species and potential confusion species, we find that citizen scientists struggle to distinguish these similar looking predators from one another, but our simulations show that citizen scientists' observation skills are sufficient to allow us to reach our conclusions.}

\normaltalk{Biotic and Abiotic Causes of Variation in Nestling Energy Expenditure Across Species and Geographic Regions}{Adam E Mitchell, Blair O Wolf, Thomas E Martin}{Energetic studies of adult birds have provided useful information about biotic and abiotic constraints on physiology. However, estimates of energy expenditures in offspring are severely lacking, preventing a thorough understanding of the factors that influence avian physiology and ontogeny. Nestling songbirds may be expected to expend more energy during colder weather due to shivering thermogenesis. Alternatively, parents may increase brooding behavior as temperatures cool to minimize costs. Rainfall provides similar predictions due to heat loss from wetting. Also, larger brood sizes are expected to reduce costs of thermoregulation by increasing thermal inertia, but limited data on nestling energetics prevents strong tests of this hypothesis. Furthermore, biogeographic patterns of nestling energetics are almost entirely unexplored, but global avian metabolism data and life history theory predict lower energetics in tropical regions. We tested these hypotheses by estimating energy expenditure of free-living nestlings using the doubly-labeled water technique. We sampled over 600 nestlings from 57 species at two temperate sites (Arizona, South Africa) and one tropical site (Borneo). Accounting for the effects of body mass, ambient temperature negatively impacts nestling energetics (p = 0.004). Contrary to our expectations, rainfall was not correlated with energy expenditure (p = 0.536) suggesting that parents alter their behavior during rainy periods to reduce costs to offspring. Brood size was negatively correlated with energy expenditure (p $<$ 0.001). Lastly, tropical species had lower energy expenditures than temperate species on average. Together, our results illustrate biotic and abiotic causes of variation in energy expenditure across species and regions.}

\normaltalk{An Integrated Population Model for Spatial Capture-Recapture and Distance Sampling Data}{Narmadha M Mohankumar, Trevor J Hefley, Katy Silber, W A Boyle}{Classical capture-recapture (CR) models are widely used to estimate survival rates and the abundance of species in ecology. However, classical CR models are unable to accommodate the spatial distribution of individuals; thus, they do not account for the spatially explicit processes created by resource availability, dispersal, migration, and human-caused factors such as habitat fragmentation. The development of spatial capture-recapture (SCR) models incorporate an explicit model for the spatial distribution allowing the ecologists to test hypotheses, including spatial processes that are inherent in ecological studies. Recent developments include integrating SCR and other ecological survey data into a unified modeling framework to obtain inferences on demographic parameters with the potential to reduce uncertainty with abundance, survival, and resource selection. Distance sampling (DS) is a common source of ecological data where distances of the objects are surveyed from randomly placed lines or point transects. Our approach introduces a unified modeling framework integrating SCR and DS by incorporating the spatial distribution. Our study leads to leverage the strengths of data sources, SCS, and DS and improves inference on demographic parameters with enhanced statistical power. We evaluate the performance of the proposed approach using both simulated data and real data for Grasshopper Sparrows on Konza Prairie Biological Station, Kansas.}

\normaltalk{Testosterone Modulates Status-Specific Patterns of Cooperation in Wire-Tailed Manakins}{T B Ryder, Roslyn Dakin, Ben J Vernasco, Brian S Evans, Brent M Horton, Ignacio T Moore}{Mechanisms mediating cooperative behavior are not well understood. We studied the influence of testosterone on cooperative behavior in the wire-tailed manakin (Pipra filicauda), a tropical songbird whose cooperative partnerships are crucial for fitness and form the basis of long-term coalitions and social networks. We used an automated telemetry system to monitor more than 36,000 cooperative interactions among male manakins over three field seasons, and subsequently examined how circulating testosterone related to cooperation using $>$ 500 hormone samples. Observational data show that in non-territorial males, high testosterone is associated with increased cooperative behaviors. In contrast, in territory-holding males, both observational and experimental studies demonstrate that high testosterone is negatively associated with cooperative behaviors. Within each status class, circulating testosterone explains significant variation in social behaviors among individuals. Collectively, our findings demonstrate that the influence of testosterone on a male's cooperative display behavior is dependent on his social status.}

\normaltalk{The Evolution of Innate Vocal Traits Supports the Vocal Learning Continuum Hypothesis in Birds}{Ines G Moran, Stilianos Louca, St\'{e}phanie M Doucet, Daniel J Mennill, Margaret C Stanley, Kristal E Cain}{Vocal communication is one of the many remarkable adaptations on Earth and how it evolved remains unclear. Around 75 millions years ago, a novel vocal adaptation provided a small group of animals with the ability to reproduce vocalisations from their surroundings, also known as vocal learning. According to the vocal learning continuum hypothesis, vocal learning evolved progressively from limited, moderate, to complex vocal learning brain pathways. Whether such evolutionary transitions are reflected in the evolution of innate avian calls, recently found to be produced in the same brain area as vocally learned signals, is unknown. Here, we test the vocal learning continuum hypothesis by investigating evolutionary acoustic patterns of an ostensibly innate vocal signal. We conducted an extensive literature review of bird call spectrograms from 372 bird species for one functional call group, namely the distress call. We then illustrated the diversity of distress calls worldwide, and highlighted acoustic patterns and quantified evolutionary changes among avian groups. We find that the structure of innate avian calls is relatively conserved throughout birds. Moreover, we find that the rate of evolution of distress calls among non-vocal learners and vocal learners is faster in vocal learners, and that the rate of evolution of distress calls follows a clear continuum from limited to complex vocal learners. Collectively, our results support the vocal learning continuum hypothesis and show that the evolution of innate vocal traits provide novel and valuable insights into the evolution of vocal learning and the evolution of the brain vocal learning pathways.}

\normaltalk{Landscape of evolution: Exploring the role of metabolic network structure in patterns of diversification  }{Erin S Morrison, Alexander V Badyaev}{To what extent does the complexity of a metabolic network reflect its current function vs. its function in the organism's past environments? Does specialization of a metabolic network to a particular environment trade-off with its ability to perform with a variety of inputs? We address whether a diet-dependent metabolic network co-evolves with ecological specialization by examining ecological correlates of carotenoid metabolic network diversity across 250 bird species. Production of carotenoid coloration in birds requires initial consumption of external carotenoids which are metabolically converted into plumage carotenoids. Thus, the starting points of carotenoid metabolism vary among ecological groups of species. We first categorized species based on diet and diet specialization. We then identified dietary carotenoids that were the starting points for each dietary classification and examined structural differences in metabolic networks between ecological groups. We specifically tested the relative contribution of dietary inputs, biochemical network connectivity, taxonomic relatedness, and ecological convergence to interspecific variability in the products of color-producing metabolic networks. The results of this study shed light on the role of metabolic network structure in facilitating adaptation to novel environments and resources.}

\normaltalk{Engaging Local Stakeholders in Reversing the Decline of the Endangered White-Breasted Thrasher by Reducing the Impacts of Invasive Species and Restoring Key Nesting Sites}{Jennifer Mortensen, Jeff Dawson, Izabela Barata, Saphira Hunt, Pius Haynes, Jeannette Victor, Aloysius Charles, Stephen Lesmond}{The White-breasted Thrasher (Ramphocinclus brachyurus) is endemic to Saint Lucia and Martinique and has been called one of the highest priority avian species for research and conservation in the Caribbean. This call has been answered in part by considerable ecological research since 1995, (including field studies of the species' range, habitat requirements, population size, demography, behavior, and population genomics) and completion of the first conservation plan for the Saint Lucian subspecies (2014). There is now, however, a critical need to continue that important work in the context of management interventions. Saint Lucia has an impressive track record of protecting its upper elevation wet forests and associated species (and in the process initiating the first Rare ``Pride" campaign). But lower elevation dry forests, used by more people, are seen as unimportant habitat. In broad terms, our current project is working to address the thrasher's major threats -- loss and degradation of critical habitat combined with negative impacts of non-native predators -- through engagement with key local stakeholders and schools. It is the first attempt to implement conservation interventions for dry forest avifauna in Saint Lucia. This project is ongoing, with year one accomplishments including: (1) development of a habitat suitability model that will guide site selection for a community-involved restoration plan, (2) installation of a camera trap network to estimate baseline invasive mammal abundances, (3) implementation of invasive mammal control and thrasher nest monitoring programs, and (4) outreach events at schools-based Environmental Clubs that include participatory exercises using camera trap data.}

\normaltalk{Homerange and Demographics of a Wintering Population of Chestnut-Collared Longspurs}{John A Muller, Nuwanthika Perera, Jeremy D Ross}{During the Winters of 2018-19, and 2019-20 we captured and banded 120 Chestnut-collared Longspurs (Calcarius ornatus) at the Wichita Mountains National Wildlife Refuge in Southwestern Oklahoma. Along with banding we attached 44 and 46 radio-transmitters during each winter respectively. We tracked individuals throughout the winter to determine seasonal movements and space use. We also took demographic data and found that the Wichita Mountains population is predominantly male (78\%) and older birds. We conducted both Minimum Convex Polygon and 95\% Kernel Density home range estimates on longspurs that had a minimum of 15 and 30 detection locations (recommended minimums for MCP and Kernel density). We found that individual Chestnut-collared Longspurs appear to use differing winter survival strategies where some individuals frequently moved distances of several kilometers between grassland sites while others stayed in a single small ($<$50ha) area. Some areas of the refuge were consistently used between seasons where other areas were only used during one of the two winters. This data is going to be used to inform state and local wildlife officials for management of the species during it's non-breeding period.}

\normaltalk{Integrated Open Canopy System in Coffee Crops (Coffea Arabica) as a Conservation Alternative for Nearctic-Neotropical Migrants and Resident Birds in Yoro, Honduras.}{David Murillo, Dar\'{i}o Alvarado, Fabiola Rodriguez, David King}{Honduras possesses 349,510 hectares of coffee crops (Coffea arabica), most are shaded and sun coffee farms, with a smaller scale being Integrated Open Canopy coffee (IOC coffee). The IOC coffee system consists of coffee crops surrounded by forest in a 1:1 ratio, which allows preservation and restoration of native ecosystems. Our research goal was to evaluate the role of IOC coffee as an alternative for the conservation of Nearctic-Neotropical migratory and resident birds. We studied ecological characteristics such as richness and abundance of species. From November 2018 to April 2019, we conducted point counts at 75 points, in a coffee growing zone of the Yoro Department, Honduras. Statistically, we found that species richness of migratory birds did not differ significantly (p $>$ 0.05) between the cover types of forest, IOC coffee, and shade coffee. Nevertheless, the richness of forest dependent species was higher (p $<$ 0.05) in forest and IOC coffee that on shade and sun coffee. Some forest dependent species like Slate-throated Redstart (Myioborus miniatus) and Ochre-bellied Flycatcher (Mionectes oleagineus) were more abundant (ji $<$ 0.05) on forest and IOC coffee than shade and sun coffee. The results of this investigation support those conducted at IOC farms in Costa Rica -- where this agroforestry land-sparing approach was first introduced. We show that IOC coffee is an alternative for the conservation of Nearctic-Neotropical migratory and resident birds, especially for the forest dependent species.}

\normaltalk{Shared Seafood, Segregated Spaces: Foraging Movements and Diet of Sympatric-Breeding Arctic Terns and Common Terns on Country Island, NS}{Avery Nagy-MacArthur, Mark Mallory}{Colonial-nesting seabirds may exhibit niche partitioning of diet or habitat to reduce competition and maximize resource use. Breeding seabirds require separate habitats for nesting and foraging, hence understanding marine habitat use beyond the colony is essential for effective conservation. Country Island, Nova Scotia, hosts regionally significant numbers of breeding Arctic Tern (Sterna paradisaea) and Common Tern (Sterna hirundo), and although a monitoring program has existed since 1998, knowledge gaps exist regarding marine habitat use and diet of adult terns. This project integrates diet analysis with GPS tracking of foraging movements to generate new insight into relationships between habitat use and diet in Arctic and Common Terns and assess interspecific spatial or dietary partitioning.  A total of 26 Arctic Terns and 27 Common Terns were tracked during incubation in 2018 and 2019 to identify foraging ranges and determine habitat use. We used stable isotope analysis of tern blood plasma and collected prey items to investigate interspecific dietary partitioning during the tracking period. Both species forage farther from the colony than previously thought and exhibit some spatial partitioning, with Arctic Terns foraging farther offshore than Common Terns, particularly later in the breeding season. Broad overlap in blood plasma stable isotope signatures does not suggest significant dietary partitioning is occurring between the species. As the first GPS-tracking study of Arctic and Common Terns in a Canadian colony, this study provides novel baseline data on dietary and habitat requirements, key information required for marine spatial planning and conservation activities in Atlantic Canada.}

\normaltalk{Using Tree Preferences of Insectivorous Birds to Inform Plant Selection and Restoration in Human-Dominated Landscapes}{Desiree L Narango, Melory Brandao, Peter P Marra, Robert A Rice, Kerry Snyder, Douglas W Tallamy}{When selecting habitat, birds make choices at macro- and micro-scales--from a patch of habitat in a landscape to foraging in an individual tree. Because of differences in feeding behavior and partitioning, as well as differences in resource availability and quality among tree species, birds exhibit foraging preferences toward particular plant taxa. In human-dominated landscapes, such as urban areas and agriculture, plant community composition tends to be both novel and cultivated. Restoration efforts to combat negative effects of habitat fragmentation and degradation have resulted in widespread tree planting, however little attention is given to the importance of plant identity in restoring bird habitat. Here, we report findings from two studies of foraging insectivorous birds: residential neighborhoods of Washington D.C., USA, and shade-grown coffee in Central and South America. We show that insectivorous birds exhibit strong foraging preferences on select native tree species that support the highest abundance of insect prey. However, the plants selected, and the magnitude of preference, can vary by sex, species, and time in the annual cycle. We also found that as habitats become more dominated by resource-poor tree species, birds forage more selectively suggesting risk-sensitive behavioral flexibility. In the context of our findings, we review patterns of tree preferences and bird diets from prior literature in natural systems to identify which plants might be disproportionately important for foraging insectivores. We also discuss case studies and opportunities where foraging behavior can inform restoration guidelines for stakeholders to create high-quality bird habitat in shared, living- and working landscapes.}

\normaltalk{A Drying Climate Bolsters Diversity but Reduces Demographic Rates for Utah's Riparian Birds}{Monte HC Neate-Clegg, Joshua J Horns, Martin Buchert, Theresa L Pope, Russell Norvell, Cagan H Sekercioglu}{Riparian zones are critical refuges for biodiversity in arid regions like the western United States. Birds in particular rely on these habitats for breeding and migratory stopover within a resource-scarce landscape. Climate change is likely to affect the distribution of resources across such landscapes and how birds use riparian zones. In this study we use 15 years of bird banding data from eight riparian stations across Utah to examine the effects of variation in climate and vegetation on birds at the community and population level. At the community level we used generalized linear mixed models to analyze the effects of temperature, precipitation, El Ni\~{n}o Southern Oscillation (ENSO), and Normalized Difference Vegetation Index (NDVI) on capture rates, species richness, diversity, and community composition. At the population level we used mark-recapture models to analyze the effects of ENSO and NDVI on rates of survival, recruitment and population growth. We found that capture rates, species richness and diversity were positively associated with ENSO but negatively associated with precipitation and NDVI. We also found a shift in community composition over time towards species found at lower latitudes or peri-urban areas. At the population level we found that both ENSO and NDVI were negatively associated with population growth rates for 38-39\% of focal species. Almost entirely, these decreases were driven by reduced recruitment rates (i.e. the addition of new birds to the population) rather than by reduced survivorship. We hypothesize that a drying landscape funnels more birds into riparian zones but lowers their demographic rates.}

\normaltalk{A Disconnect Between Upslope Shifts and Climate Change in an Afrotropical Bird Community}{Monte HC Neate-Clegg, Timothy G O'Brien, Felix Mulindahabi, Cagan H Sekercioglu}{Climate change threatens to push species to higher elevations and eventual extinction. Birds, in particular, are shown to be shifting upslope in the Neotropics and Southeast Asia. Yet previous studies have lacked the temporal resolution to investigate distributional dynamics over time in relation to climatic fluctuations, especially in the understudied Afrotropics. Here, we use 15 years of point-count data from across an elevational gradient (1767 -- 2940 m) in Rwanda, to assess elevational shift rates and dynamics in a community of Afrotropical birds. In general, species shifted their elevations upslope by 1.87 m/yr, especially at their lower elevational limits which shifted by 4.41 m/yr. Importantly, these shifts occurred despite the fact that local temperature and precipitation showed little trend over the study period. Moreover, the distributions of few species were associated with temperature, suggesting that temperature plays little direct role in determining elevational distributions in birds. Instead, upslope shifts may be more related to steady shifts in habitat (especially for habitat specialists) which lag behind decades of increased temperatures in the region. Precipitation appeared to have more of an effect than temperature in determining inter-annual elevational changes, allowing species to expand their range in years of higher rainfall. The birds of the Albertine rift face an uncertain future which will be shaped by climatic trends and their interactions with the ecosystem. Our results highlight the need to understand the mechanisms driving upslope shifts and the need to preserve contiguous blocks of forest across elevational gradients to allow wildlife to shift unimpeded.}

\normaltalk{Full-Year Association Between Primary Productivity and Migratory Avian Biomass Reveals Winter Refuge in North America's Subtropical Zone}{Wee Hao Ng, Daniel Fink, Tom Auer, Wesley M Hochachaka, Frank A La Sorte, Adriaan M Dokter}{Seasonal changes in resources are an important driver of avian migration, but studies of the relationship between avian abundance and resources at macro-scales are scarce. Leveraging the coverage provided by eBird citizen science data, we present a large-scale analysis of avian biomass movement covering the full year and almost the full extent of North America. We developed a spatio-temporal model describing the biomass distribution of all nocturnally migrating landbird species combined. Using predictions from the model, we assessed the full-year temporal correlations between weekly avian biomass and vegetation index measurements across the continent. We find a remarkable negative association between greenness and avian abundance and biomass in the subtropical climatic zones in the southeast, where avian biomass is highest in winter and decreases during spring and summer. This relationship is driven primarily by the efflux from the subtropical zone by boreal and mid-latitude breeding migrants, whose combined biomass is considerably larger than the influx of Neotropical passage migrants between April and May. At mid-latitude to boreal regions, we find that greenness is positively associated with avian biomass, although biomass usually peaks ahead of greenness in spring, suggesting that birds time their arrival to take advantage of peak productivity during breeding season. Our results highlight the importance of the US's southeast as a winter refuge for songbirds and of protecting birds in these regions during the non-breeding season.}

\normaltalk{Applications of BBS Stop-Level Data}{Neal D Niemuth, David J Ziolkowski}{The North American Breeding Bird Survey (BBS) was designed for operation at the route level and analyses have traditionally aggregated results across stops, reflecting a composite of the habitats along a route. But BBS data users in the 21st century are eager to improve the resolution of the BBS by taking advantage of the structure that lies below the route, at the stop level. The rapid evolution and integration of more powerful computing and spatial technologies (such as GPS, GIS, and remote sensing) have made it possible for the BBS program to begin moving in that direction. Future growth will open myriad possibilities for greater incorporation of covariates derived from land cover, climatic, and topographic datasets. The increase in spatial resolution will also offer new opportunities for integrating BBS data with other avian datasets, for developing new models that, among other things, can better control for the primary components of detectability at the stop level, and for testing hypotheses regarding factors that influence population change. To realize these opportunities, some of the remaining challenges that prohibited the survey from originally working at this level will need to be addressed; for example, BBS stops are difficult to assign unambiguously to habitats, stop counts are generally very low, and low counts tend to have high proportions of within-site variability.}

\normaltalk{Data from Citizen Scientists and Weather Surveillance Radar Explain Patterns of Bird Strikes at Three Commercial Airports in the US}{Cecilia Nilsson, Frank La Sorte, Adriaan Dokter, Kyle Horton, Benjamin V Doren, Jeffrey Kolodzinski, Judy Shamoun-Baranes, Andrew Farnsworth}{Much effort has been expended to reduce collisions between aircraft and birds, but increased knowledge about bird movements and species occurrence in areas where collisions occur is needed to improve decision support and inform proactive measures. Migratory birds pose a unique threat to aviation that is particularly difficult for individual airports to monitor and predict: movements may vary extensively in space and time at the local scales of airport responses. Here, we use two publicly available datasets, information on bird movements extracted from the US NEXRAD weather radar network and species occurrence from the eBird citizen science database. This allows us to map bird movements and species composition with low effort, but high temporal and spatial resolution compared to other large scale survey methods. We compare these results to detailed bird strike records from three major airports located in the vicinity of New York City. Weather radar based estimates of migration intensity can accurately predict when the probability of bird strikes is the highest, with 80\% of the variation in bird strikes across the year explained by the amount of movements seen on weather radar. We also show that eBird based estimates of species occurrence can, when including information on species body mass, further inform when the highest number of damaging strikes occur. By better understanding when, and where, different species are migrating, airports can predict periods of collision risks with greater temporal and spatial resolution than currently available. These forecasts may save time, money and, most importantly, lives.}

\normaltalk{Cumulative Effects of Resource Extraction and Climate Change on Suitable Habitat for a Steeply Declining Aerial Insectivorous Songbird}{Andrea R Norris, Krista De Groot, Kristina Cockle, Leonardo Frid, Kimberly M Dohms, Wendy Easton, Kathy Martin}{Olive-sided Flycatchers (Contopus cooperi, OSFL) are a steeply declining aerial insectivore and long-distance songbird, breeding in northern open coniferous forests and wintering in South America. The causes for the population decline of this widely distributed bird are likely due a multitude of stressors. We examined the cumulative effects of multiple anthropogenic and natural stressors on the future habitat supply for OSFL in one of their core breeding areas in northeastern British Columbia. First, we modeled OSFL habitat suitability by comparing additive and interactive Bayesian generalized linear mixed-effects models (binomial family) that predicted OSFL occurrence in 1462 point count surveys (1997-2011) from spatially- and temporally-matched forest inventory data. Probability of OSFL occurrence declined with clearcut size and was highest 15-20 years after logging, and in stands attacked by Mountain Pine Beetle (Dendroctonus ponderosae). Second, we used spatially explicit state-and-transition simulation models to project the cumulative effects of industrial development, forest harvesting, fires, insect outbreaks, and regrowth under two landscape management scenarios: business-as-usual, and conservation (restricted timing and size of cutblocks). Finally, we used our best-supported habitat suitability models to project habitat distribution for OSFL at a 90 x 90 m resolution in current conditions and future scenarios. Landscape simulation allows the incorporation of cumulative effects of multiple stressors into projections of suitable breeding habitat to design landscape thresholds and adjust management targets for species recovery. However, OSFL and other aerial insectivores spend 3/4 of the year outside of breeding grounds. Future conservation work should incorporate stressors across the annual cycle.}

\normaltalk{Female Songs in Space: Assessing Evolutionary Patterns and Drivers of Elaborate Female Song and Dimorphism in Feature Space}{Karan J Odom, Marcelo Araya-Salas, Lauryn Benedict, Michelle L Hall, Naomi E Langmore, Michael S Webster, Katharina Riebel}{Songbirds (Passeri) are known for their complex songs, from intricate solos by both sexes to combined, coordinated duets. However, in some species only males sing and female song is reduced or absent. We now know that female song occurs in approximately 2/3 of songbird species and that it likely existed in the ancestor of all songbirds. However, we know a lot less about the evolutionary processes responsible for the diversity and dimorphism in female and male song that we see in modern songbirds, especially why females have lost song in so many species. To investigate this, we compared female and male song structure for over two hundred songbird species from sound collections and scored dimorphism in how often females and males sing for over a thousand species from species accounts. We compared dimorphism in song structure and occurrence to a variety of natural history variables representing social, sexual, and natural selection pressures, including size dimorphism, mating system, bi-parental care, migratory behavior, and daily nest predation rates. We found that female song rates are most influenced by mating system: females sing less than males in non-monogamous mating systems. Female song is also influenced by bi-parental care and migratory behavior. Females primarily sing in species with bi-parental care and females sing the most in year-round territorial species. These patterns indicate that similarity of sex roles and potentially the reproductive benefits that females receive from coordinating breeding behaviors or joint resource defense are major factors maintaining similar vocal behavior in females and males.}

\normaltalk{Sexual and Natural Selection on the Plumage Coloration of the Helmeted Manakin}{Caio Vitor S Lima de Oliveira, Lia N Kajiki, Felipe M Gawryszewsk, Regina Helena F Macedo}{Plumage coloration has numerous functions in birds, especially in the breeding context, including the signaling of sexual maturity and current body condition. Although ornamentation has been mostly studied in males, there is some evidence suggesting that sexual selection may influence plumage coloration in both sexes. Furthermore, there may exist a trade-off between natural and sexual selection in some contexts: brightly colored females may increase the risk of nest predation. In this study we described plumage coloration of the Helmeted Manakin and investigated the correlation between spectral properties of the coloration and body condition for female, male and immature birds. We collected feathers from the crest, back and breast and measured their reflectance with a spectrophotometer. We found that plumage brightness in all three body parts varies among individuals and is significantly different between females and immatures, despite the visual similarity of their green plumages. Both females and immatures exhibit a peak of ultraviolet reflectance for feathers from the crest, back and breast. We observed several nest predation by primates, which are UV-blind. This result suggests the UV reflectance may produce a relatively more conspicuous plumage to co-specifics than to primate predators.}

\normaltalk{Migratory Bird Use of Oil Palm Plantations in Tabasco, Mexico}{Samuel L Oliveira, David J Flaspohler, Jared D Wolfe, Jessie L Knowlton}{Landscapes dominated by a single product have reduced the habitat available for migratory birds that inhabit forests and has generated questions about the value of agroecosystems for wildlife.  Oil palm plantations are among the fastest growing agroecosystems in the neotropics, yet little is known about how overwintering migratory birds use oil palm habitats. To better understand the potential value of oil palm as overwintering habitat for migratory birds, we surveyed birds in oil palm and native forest remnants in Tabasco Mexico. Specifically, we coupled bird captures, measures of vegetative structure, multivariate analysis, and generalized linear models to assess how oil palm development drives changes in migrant bird diversity, community assemblage, and abundance. Our study provided three important findings: (1) native forest remnants hosted more migratory bird species when compared to oil palm plantations; (2) migratory bird community assemblage differed between native forest and oil palm plantations; (3) changes in migratory bird abundance -- and subsequent changes in community assemblage -- was driven by dissimilarities in vegetative structure between native forest and oil palm plantations. Our results suggest that integrating more native trees and increasing understory structural heterogeneity throughout oil palm plantations represents a viable management action to improve the habitat quality of working landscapes for migratory birds.}

\normaltalk{Eco-Climatic Factors and Spatial Heterogeneity on Occupancy Estimate of Waterbirds in Peninsular Malaysia}{Martins C Onwuka, Zakaria BH Mohamed}{In Malaysia, multiple land use by humans has opened the way to substantial loss of the wetland ecosystem, and shrinkage of the populations, habitat and food bases of avian species. However, the study of the waterbirds population has become eminent to understand the complexity of wetlands ecosystem structure. The study aimed to determine the eco-climatic factors that influenced the occurrence of waterbirds and to develop their habitat suitability models in Paya Indah Wetlands (PIW) and Putrajaya wetlands (PW), Malaysia. The distance sampling point count technique using stratified random design was employed to survey and choose 57 and 54 point stations around 14 and 24 lakes of PIW and PW respectively. An automatic linear modelling algorithm and geographic information systems were employed to compute the importance ratios of seventeen eco-climatic factors. The result showed that all the observed waterbirds individual and estimated indices were significant. The automatic linear modelling algorithm results also showed that the maximum and minimum weights of the factors are land use/land cover and water dissolved oxygen for water birds respectively, in PIW, while the atmospheric pressure and NDWI respectively, in PW. Large areas of Putrajaya Wetland were classified more suitable for waterbirds than Paya Indah Wetland due to the favourable atmospheric pressure, Land use/land cover, rainfall, wind speed, relative humidity and NDWI. Thus, the models' adoption as management tools coupled with a robust population monitoring database will enhance the management effectiveness of the species and wetlands.}

\normaltalk{The Historical Biogeography of Caribbean Birds Based on aDNA from Fossils}{Jessica A Oswald, Michelle J LeFebvre, Brian J Stucky, Robert P Guralnick, David W Steadman, Julie M Allen}{Much of our understanding of the factors that have shaped modern diversity is based only on extant species. Since the late Pleistocene, the planet has been in the midst of a mass extinction event largely driven by humans. Therefore fossil data are necessary to fully understand modern biodiversity and phylogenetic relationships. This is especially true in insular regions where so many species were lost upon human colonization during the Holocene. The West Indies are today home to 171 endemic species. These extant species are related to taxa largely found in the New World, which reflects biogeographic patterns and historical source areas for the islands. Based on fossils, 78 species endemic to the West Indies have gone extinct during the past few thousand years. We used ancient DNA target capture approaches and mitochondrial genomes from related taxa to place the extinct woodcock Scolopax brachycarpa and the extinct ``cave-rail" from the Virgin Islands and Puerto Rico, Nesotrochis debooyi, into a phylogenetic context. Phylogenetic analyses were performed in RAxML and divergence time estimation was calculated in BEAST2. Based on these results, the extinct species of Scolopax and Nesotrochis have sister relationships with Old World taxa. Scolopax brachycarpa is sister to the Austral Snipes (Coenocorypha), now restricted to the Pacific. The extinct flightless genus Nesotrochis is sister to the flufftails (Sarothruridae), which are all flighted and largely restricted to Africa. These results change our understanding of the biogeographic processes that have shaped these islands and emphasize the importance of including recently extinct species in diversity analyses.}

\normaltalk{Increasing Temperatures Drive the Risk of Reproductive Failure in a South African Alpine Ground-Nesting Bird, the Cape Rockjumper}{Krista N Oswald, Elizabeth F Diener, John P Diener, Susan J Cunningham, Ben Smit, Alan T Lee}{Reproductive failure in birds is due predominantly to nest predation, and can be correlated with nest concealment (e.g. lower failure with more vegetative cover). However, predation risk depends on predator type, as predators vary in their ecology and sensory modalities (e.g. visual vs. olfactory). In many habitats, snakes (generally olfactory predators) are a major nest predator for small birds, and in such cases predation is strongly associated with warmer temperatures. Here, we investigated nest survival in a ground-nesting alpine species, the Cape Rockjumper Chaetops frenatus, endemic to the Fynbos mountains in southwestern South Africa. We collected three years of nest data, testing if nest survival was correlated with (1) more open habitat [i.e. ``Habitat Stage"; early post-fire versus late post-fire habitat (= 3 and $>$ 3 years since fire respectively)], and (2) nest concealment. We found that 46 of 54 nests failed due to predation, with temperature our most significant predictor. We also found greater survival slightly correlated with early post-fire habitat, and increased snake predation (our main source of predation; n = 19 of 28 identified predation events) at higher temperatures. Our findings illustrate that reproductive failure may involve a complex interplay of predator-type and habitat-level ecological factors.}

\normaltalk{A Participatory Approach to Managing Offshore Islands in Antigua and Barbuda}{Natalya Lawrence, Jennifer Daltry, Nathan Wilson, Shanna Challenger, Sophia Steele, Andrea Otto}{Over 122 species of Neotropical Migratory birds use the offshore islands as critical habitats. Resident and migratory birds face many pressures from the widespread introduction of black rats and other Invasive Alien Species, and rising human disturbance. From 2017, the OICP heightened its participatory approach to restoring and managing offshore islands, with a greater emphasis on engaging private landowners as active conservation stewards of their islands.  Through in-the-field training, collaborative meetings and invasive species eradication planning and implementation, OICP has added 83 ha of restored habitat for birds and other wildlife, with two more islands scheduled to be restored in 2019 or early 2020. Senior government officials and local television personalities are now championing the work to restore offshore islands as they have come to appreciate the herculean task to safeguard these islands for the benefit of wildlife and people. Here we will highlight the various activities undertaken to enhance this national participatory conservation effort.}

\normaltalk{The New BBS Strategic Plan}{Keith L Pardieck, Veronica I Aponte}{Started in 1966, the North American Breeding Bird Survey (BBS) provides the foundation for conservation planning and management of hundreds of North American birds at regional and continental scales. Today, more than 800 scientific articles utilizing BBS data and trends have been published. Though the BBS is also widely used and integrated into North American conservation efforts, it is at a pivotal moment in its history, where opportunities exist that will allow the program to take full advantage of its invaluable dataset. The new BBS Strategic Plan provides a visionary roadmap that will guide the program into the future. Accomplishing the three goals in the strategic plan will mean that the BBS program will modernize its data collection and management system, fill geographic gaps, leverage new technologies to expand the applications of its data and ensure an adequate volunteer base, have the resources to meet the needs and expectations of stakeholders, and meet its mandate as a publicly supported program.}

\normaltalk{The Influence of White-Tailed Deer (Odocoileus Virginianus) Browsing on Vegetation and Avian Communities in Fenced and Unfenced Timber Harvests}{Halie A Parker, Clark Rushing, Emily Domoto, Joseph E Duchamp, Dan Heggenstaller, Jeffery T Larkin, Michael C Tyree, Jeffery L Larkin}{Forests of eastern North America have been negatively impacted by excessive white-tailed deer browsing for decades. Previous studies provided insight regarding how deer-driven changes in forest structure, tree species composition, and microhabitat negatively impact forest birds. However, such studies used relatively small plot sizes with low replication, limiting the transferability of findings to operational-scale timber harvests. We studied the influence of white-tailed deer browsing on vegetation and avian communities in paired fenced and unfenced regenerating timber harvests in Pennsylvania. Proportion of browsed woody stems was higher in unfenced harvests. Fenced harvests had greater horizontal and vertical vegetation density. Woody stem density did not differ between fenced and unfenced harvests, but the average height for several woody species was taller in fenced harvests. Based on a previous study we predicted that some avian species would be more positively associated with fenced versus unfenced harvests. Breeding bird communities did not differ between fenced and unfenced harvests, but density of some species did differ. As predicted, Chestnut-sided Warbler (Setophaga pensylvanica) density (indiv./ha) was greater in fenced versus unfenced harvests, while Common Yellowthroat (Geothlypis trichas) density was similar between the two treatment types. Territory settlement results revealed Chestnut-sided Warbler selecting for fenced harvests, whereas the Common Yellowthroat was indifferent. Initial findings from this study suggest deer-induced impacts on vegetation in regenerating timber harvests can have considerable influence on territory selection decisions and abundances of some avian species and that the settlement surveys provide insight into habitat selection patterns not detected by breeding season surveys.}

\normaltalk{Cultural Conformity and Persistence in the Context of Differing Site Fidelity}{Timothy H Parker, Bridget Sousa, Stephan T Leu, Stacy Edmondson, Cecily Foo, Amy Strauss, Hanna Kahl, Kristen Ballinger, Eric Ross, Mareile Große Ruse, Maria Sandsten, Bram HF Verheijen}{Animal culture often shows geographic structure, with individuals in close proximity sharing more cultural features than individuals further apart. However, spatial extent of cultural features, along with the degree of conformity to local cultures, vary within and among species. Further, rates of cultural change presumably also vary, though documentation of temporal variability lags behind documentation of spatial variability, and mechanisms driving this variation have not been sufficiently explored. We hypothesized that conformity to local culture in Oscine songbirds and the persistence of culture over time and space are promoted by habitats that facilitate stable populations in which individuals show relatively high site fidelity. In contrast, sites in which habitat features cause rapid population turnover provide more vacant territories and so more opportunities for colonization. Colonization should drive more rapid cultural change, either through adult colonists importing foreign cultural variants or young colonists making errors as they learn the local song. To test this set of hypotheses, we examined temporal and spatial variation in vocal culture in a songbird (Dickcissel, Spiza americana) in two distinct habitat types. As predicted, we found high site fidelity in relatively stable native grasslands and much lower site fidelity in nearby cropland sites which were disturbed by farming practices during the breeding season. We also found indications of higher levels of song sharing and slower changes in vocal culture in our grasslands relative to croplands, though we found no evidence of different spatial scales of song sharing between these habitats.}

\normaltalk{The Impact of Group-Selection Timber Harvesting on Avian Diversity in West-Central New Hampshire}{Samuel Parks, Len Reitsma}{Rapid loss of Neotropical migrant songbirds (44\% population decline in the past 50 years) necessitates an urgent call for conservation. The greatest cause for these declines is habitat loss, both on the breeding grounds and to overwintering habitat. Throughout the Northeastern United States, cultural shifts in land-use have led to even-aged, mid-successional forests becoming the dominant forested habitat types on the landscape. This has caused a loss of young forest habitats required by scrub-shrub birds for nesting and by many forest-interior birds for critical post-fledging cover and foraging. We used point counts to assess songbird community response to group-selection harvesting (0.2-1.0 ha openings) in a mixed-wood forest and in an Eastern hemlock (Tsuga canadensis) dominated forest. Counts in the mixed-wood forest continued a 7-year-post-harvest study and those in the hemlock forest were conducted 1-year before and after timber harvesting. Group-selection openings in the mixed-wood stand created a multi-aged forest mosaic, leading to a significant increase in avian diversity and no significant loss of forest-interior birds seven years post-harvest. Territory mapping of 6 forest-interior bird species revealed that most birds incorporated =1 group opening into their territories. Pre-harvest point counts were conducted in the hemlock stand in 2019 and the post-harvest counts were in May-June 2020. We expect a similar increase in total avian diversity and abundance and expect this increase will take longer but also last longer due to a slower regeneration time characteristic of hemlock stands compared to mixed-wood stands.}

\normaltalk{Evidence First, Deduction Second, and Practice, Practice, Practice: Extending Avian Science and Conservation Through Citizen Science}{Julia K Parrish}{Hands-on citizen science offers an opportunity to crowd-source environmental data collection across stretches of space and time that no research lab could hope to attain.  Projects that offer repeated activity provide participants the opportunity to learn the practice of science. The Coastal Observation and Seabird Survey Team (COASST) is one such program: a 20 year old environmental citizen science project focused on what floats in on the tide as a metric of marine ecosystem health.  The program is constructed such that deductions (what species is this?) can be independently verified with evidence (measurements, photographs), which also allows accuracy-tracking at the individual level. COASST natural science data have been used for a wide range of basic science (e.g., impacts of harmful algal blooms) and actionable science (e.g., fishery bycatch forecasting), as well as the baseline against which hypotheses of historic indigenous hunting have been assessed. COASST data are also included in a range of tribal, state, and federal annual reporting.  Simultaneous studies of motivation and identity indicate that the natural history stories and actionable science arising out of COASST data at scale provide continued incentives to participate, creating a positive feedback loop which furthers participant expertise and excellence of the science.  Together, these studies suggest that: (1) people are highly motivated to contribute to place-based science; (2) with rigorous training, non-scientists are as good as scientists at data collection; (3) and program participants are hungry for ecosystem knowledge, and are ready to work to steward and conserve it.}

\normaltalk{Energy Landscapes Determine Winter Strategies and Distribution of an Arctic Seabird}{Allison Patterson, Grant Gilchrist, Dave Fifield, April Hedd, Greg Robertson, Kyle Elliott}{Thick-billed murres (Uria lomvia) from Coats Island, Nunavut, Canada, winter in the Labrador Sea. We tracked 44 murres through two consecutive non-breeding seasons, 2017/18 and 2018/19, using geolocator-temperature-depth recorders. During each winter, most murres switched between using cold (0.2 $<$b1$>$ 1.2 $<$b0$>$C), stratified water on the Labrador Shelf and warmer (3.1 $<$b1$>$ 0.2 $<$b0$>$C), mixed water in the Labrador Basin. Overall, 53\% of tracked murres used both cold and warmer water during winter, while only 35\% spent the majority of winter in cold water and 12\% spent the majority of winter in warmer water. We examined what factors influence the probability of transitioning between colder and warmer water and compared behaviour and energetics of birds using these two winter habitats. Use of warm water was highest in February (65\% of birds in 2018 and 47\% of birds in 2019). There was a strong positive relationship between ice cover on the Labrador Shelf and the percentage of birds using warmer water, suggesting that murres may be excluded from the shelf by ice in mid-winter. While in warmer water murres spent less time flying, had lower daily energy expenditure, made shallower dives, and foraged more at night than murres in cold water. These warmer pelagic waters may provide a lower risk and lower energy alternative habitat to the continental shelf. Variation in the energetic landscape of different marine habitat types appears to play an important role in the winter strategies of this species.}

\normaltalk{What is the Role of Environmental Heterogeneity in the Evolution of the Blue-Crowned Manakin?}{Pedro Paulo, Lucas N Bandeira, Carolina Ferreira, Fernando H Te\'{o}filo, Andre Moncrieff, Érico Polo, Arielli F Machado, Clessian Dias, Mario Cohn-Haft, Tomas Hrbek, Igor L Kaefer, Marina Anciães}{The genetic and phenotypic variation among populations can be influenced by neutral processes such as drift or by selective processes such as natural and sexual selection. Thus, investigating the role of the environmental heterogeneity can provide insights into the relative influences of neutral and selective forces on biological diversification in nature. Thus, our main goal was to investigate if environmental factors influence the genotypic and phenotypic variation in the plumage of adult males of the blue-crowned manakin (Lepidothrix coronata) by testing the hypotheses of Isolation by Adaptation (IBA), Isolation by Distance (IBD) and Isolation by Environment (IBE). As part of Pipridae family, the blue-crowned manakin presents the most curious case of plumage variation in adult males, featuring black, green and intermediate plumage colors between these morphotypes, besides different shades of blue crown throughout its Neotropical distribution. In this study, we compiled the environmental, genetic, and plumage data of specimens deposited in collections covering the total distribution of the species., we use the Multiple Matrix Regression with Randomization (MMRR) approach indicates a combined influence of neutral and selective forces in the evolution of genotypic and phenotypic differentiation among blue-crowned manakin populations.}

\normaltalk{CAN COASTAL RESTORATION PROJECTS PRESERVE BALD EAGLE (Haliaeetus Leucocephalus) BREEDING HABITAT in COASTAL LOUISIANA?}{Katie L Percy, Nicole L Michel, Lindsay Nakashima, Erik I Johnson}{Louisiana is in the midst of a land loss crisis. About 2,000 mi2 have been lost since the 1930s, and without action another 4,120 mi2 may be lost over the next 50 years. Louisiana's Coastal Protection and Restoration Authority has developed a \$50 billion, 50-year Coastal Master Plan. Our objective was to evaluate the potential for the Bald Eagle (Haliaeetus leucocephalus) to serve as a useful indicator of freshwater wetland health, and to develop a model that could be used to evaluate bird response to future land change scenarios. For this, we used aerial nest survey data of 49 nests from the Barataria Basin in southeastern Louisiana. We modeled the relationship between Bald Eagle nest occurrence and landcover using boosted regression trees (BRTs). Model performance was strong (BRT models explained 54.4 ± 0.1\% of deviance in nest occurrence) indicating positive associations of eagle nesting sites with fresh forested wetlands, freshwater marsh, and flotant marsh. Remaining landcover types (intermediate marsh, salt marsh, brackish marsh, upland, and developed) each explained $<$10\% of the model variation. Future suitability of Barataria Basin for nesting Bald Eagles was higher in 20 and 50 years with restoration action compared to without restoration action. Even so, suitability decreased between 20 and 50 years with restoration action, driven by continued predicted loss of fresh forested landcover. Because our results indicate that the Bald Eagle is a good indicator of fresh forested wetlands, we are extending the model to evaluate habitat relationships and land loss scenarios across coastal Louisiana.}

\normaltalk{Historical Patterns in Mercury Exposure for North American Songbirds}{Marie Perkins, Oksana P Lane, David C Evers, Amy Sauer, Evan M Adams, Nelson J O'Driscoll, Samuel T Edmonds, Allyson K Jackson, Julie C Hagelin, Jeremiah Trimble, Elsie M Sunderland}{Methylmercury (MeHg) is a global environmental contaminant that poses significant risks to the health of humans, wildlife, and ecosystems. Assessing MeHg exposure in biota across the landscape and over time is vital for monitoring MeHg pollution and gauging the effectiveness of regulations intended to reduce new mercury (Hg) releases. We used MeHg concentrations measured in museum specimen feathers (collected between 1869 and 2014) and total Hg concentrations (as a proxy for MeHg) of feathers sampled from wild birds (collected between 2008 and 2017) to investigate temporal patterns in exposure over approximately 150 years for North American songbirds. For individual species, we found greater concentrations for samples collected post-2000 compared to those collected during historic times (pre-1900) for six of seven songbird species. Mean feather concentrations measured in samples collected post-2000 ranged between 1.9 and 17 times (mean 6.6) greater than historic specimens. The proportion of individual songbirds with feather concentrations that exceeded modeled toxicity benchmarks increased in samples collected after 1940. Only 2\% of individual songbirds collected prior to 1940 had feather concentrations greater than 2.4 µg/g (a toxicity benchmark related to a 10\% decrease in nest success) compared to 35\% of individuals collected post-1940. Many species included in this study have a vulnerable or near-threatened conservation status, suggesting recovery actions are needed to address mercury pollution.}

\normaltalk{How Wet is Too Wet? Avian Community Responses to Wetland Augmentation and Creation}{Sean M Peterson, Steven R Beissinger}{Terrestrial habitats have been greatly altered by human activity. Addressing the conservation needs of most species requires investigating the relationship between human activity and ecosystem health. We studied the impacts of wetland management strategies on species richness, a key measure of ecosystem health, and avian occupancy in the Sierra Nevada foothills of northern California. Wetlands in the Sierra Nevada foothills are composed of a natural sites, natural sites augmented by the addition of irrigation water, and de novo sites fed entirely by irrigation water. We used automated recording units to record the occurrence of birds at 99 wetlands six to nine times per summer during 2017 and 2018. We analyzed detections with a Bayesian multispecies occupancy model that accounted for site-level measures of habitat and imperfection. We detected 41 bird species that occupied at least 10\% of sites, including six species of conservation concern. Species richness was greatest at sites where natural wetlands were augmented by irrigation water compared to natural sites and de novo sites. The creation of new wetlands in the form of large impoundments resulted in an overall reduction of species richness. The wetlands of the Sierra Nevada foothills are threatened by changing precipitation patterns and climate conditions as well as human encroachment. Properly managing these wetlands, which mostly occur on private lands, will be important for ensuring that bird communities and the ecosystems that support them continue to thrive in a rapidly changing world.}

\normaltalk{The Effects of Anthropogenic Noise and Light on Settlement and Success in a Community of Birds}{Jennifer N Phillips, Clinton D Francis}{Over evolutionary timescales the sensory environment has provided reliable channels for acoustic and visual communication and biological rhythms. Recently, the rise in multimodal anthropogenic pollution--noise and nightlight--present conditions that could fundamentally disrupt the ways in which organisms interact and respond to environmental cues. Because these stimuli typically co-occur in today's human-dominated world, understanding their combined influence is an urgent conservation priority. Here, we test the effects of noise and nightlight on bird nesting settlement and follow the success of nests in a pi\~{n}on-juniper forest in northwest New Mexico. We follow nesting attempts and success on 24 sites with 4 different treatments: Noise, Light, Light+Noise, and Control. We expected Noise to decrease nesting attempts but increase success due to lower predation rates, whereas Light increases avian nesting attempts but increases depredation. Thus, we expected Light+Noise will have mixed effects on bird nest success. Overall, we see trends that birds are nesting more often in lit sites, but have higher rates of abandonment compared to other treatments. Both Light, Noise, and Light+Noise tend to decrease the depredation rate, potentially providing a predator shield to songbirds. Overall, sensory pollution may be having both adverse and beneficial effects on nesting rates and success, such that their additive effects should be considered in conservation efforts.}

\normaltalk{The Case for Social Science Research in the Natural Sciences.}{Tina Phillips}{Addressing the recent sobering results from a landmark study that described the loss of nearly 3 billion birds since 1970, the need to understand the role of human behavior in bird conservation has never been greater. Understanding how to change behavior, however, also relies on understanding of individual beliefs, motivations, attitudes, and intentions. Social science research, with its established theories within various disciplinary traditions, examines these and other issues to understand individual attributes, social processes and phenomenon, and provide evidence-based and behaviorally informed solutions.    At the same time, public involvement in science and conservation is growing, especially through citizen science, the intentional engagement of the public in scientific research. Citizen scientists represent a potential army of conservationists, but like any other audience, they have diverse knowledge, motivations, expectations, attitudes, skills, and behaviors. Social science research within the context of ornithological citizen science provides opportunities to understand human behavior at a variety of scales among a target audience most likely to engage in bird conservation. However, while there is growing recognition of the value the social sciences can bring to solving real-world conservation problems, many natural scientists are still unsure about what social science really is, and how to assess the quality of social scientific research. This introductory talk will set the stage for the subsequent talks by describing what social science research is and some of the most commonly used theories and methods, why it's important in conservation, and best practices for its use and implementation.}

\normaltalk{Individual Condition is the Key to Understanding Long Term Effects of Early-Life Exposure to Invasive Prey}{Caroline Poli, Ellen Robertson, Julien Martin, Robert Fletcher}{Conditions during early development may influence key biological and behavioral processes that affect individual organisms over long time periods and can feedback to influence population dynamics. Silver spoon effects occur when an individual reared in favorable conditions incurs benefits, with outcomes that may influence survival, morphology, and fitness. For the critically endangered Snail Kite, recently-observed increases in survival, reproduction, and population size have been linked to invasion of wetlands in the southeastern US by an exotic snail. Exotic snails increase food availability for Snail Kites, and can affect habitat quality and individual quality. We sought to investigate mechanisms for potential silver spoon effects in Snail Kites and to understand the extent to which the snail invasion may affect survival on an individual level. We collected body measurements and conditions at the natal wetland for 4055 fledgling Snail Kites produced between 2002-2019. We monitored wetlands and recorded subsequent encounters for each individual. We modeled survival rates of Snail Kites using the Cormack-Jolly-Seber model to account for imperfect detection. Birds that fledged from wetlands with invasive snails were heavier than birds that fledged from wetlands without invasive snails. Juvenile survival was notably higher for birds that were heavier at fledging, and the effect carried over into adult survival, indicating a silver spoon effect that did not diminish with age. Our results highlight how early-life exposure to invasive prey can have long-term effects on an endangered bird, and point to individual condition at fledging as a key indicator for habitat quality.}

\normaltalk{Long-term Monitoring Reveals Widespread Avian Declines in Intact Tropical Forest}{Henry S. Pollock, Corey E. Tarwater, Judith D. Toms, Thomas J. Benson, and Jeffrey D. Brawn. }{Understanding the factors governing species' distributions and abundances is a central aim of ecology. Long-term studies are the gold standard for tracking temporal changes in abundance and diversity, yet are expensive, time-consuming, and difficult to maintain. Here, we leverage a 40-year mist-net study in central Panama to provide unprecedented temporal coverage of an understory forest bird assemblage. Our primary goal was to determine how the assemblage had changed over time in a relatively undisturbed forest. Secondarily, we compared our findings to two other long-term studies of tropical bird assemblages from lowland forests in Ecuador and Brazil to test the generality of our conclusions. We found declines of a suite of second-growth species that were originally present at the study site but likely dropped out due to natural forest succession. For most species and foraging guilds, however, capture rates (an index of abundance) fluctuated substantially yet remained relatively constant on average across the sample. Similarly, at the community level, alpha diversity remained relatively constant over time. Our results differ markedly from the two other long-term studies of tropical birds, which both found a) pervasive declines in capture rates and b) specific declines of understory and terrestrial insectivore foraging guilds. Thus, while some species and guilds may be declining in intact forests, our assemblage in central Panama appears to be relatively stable. Reconciling these discrepancies will require drilling down to the mechanisms underlying these opposing patterns. Our findings highlight the importance of long-term datasets for elucidating temporal changes in abundance and diversity.}

\normaltalk{Why is Sympatric Speciation Rare in Birds?}{Cody K Porter, Craig W Benkman}{Closely related bird species often show the type of niche differentiation involved in disruptive selection and speciation with gene flow, yet sympatric speciation appears to be extremely rare in birds. In general, niche differentiation between closely related bird species is strongest in the non-breeding season, when species use alternative resources that impose strong feeding tradeoffs. Conversely, bird species with divergent morphologies are regularly found converging in their use of abundant resources that impose minimal feeding tradeoffs during the breeding season. Comparative data suggest that sympatric speciation is most likely when lineages are adapted to using different resources with strong tradeoffs while breeding, yet direct empirical tests of this hypothesis are lacking. I will test this hypothesis using a comparison of reproductive isolation between red crossbill (Loxia curvirostra complex) ecotypes during periods of variable resource tradeoffs. My data supports the hypothesis that breeding during strong tradeoffs promotes sympatric speciation. This work may help explain the rarity of sympatric speciation in birds and other taxa more generally.}

\normaltalk{Effigies as Deterrents in the American Crow}{Elena Prado-Ragan, Jim Mountjoy}{The American crow, Corvus brachyrhynchos is considered a nuisance to humans by many, particularly crow individuals that congregate near human settlements. Crows are problematic when roosting in or near local municipalities since they are frequent agricultural pests, raid trash bins, litter surfaces, and corrode infrastructure due to the ammonia in their droppings. Further, there is human concern for the spread of zoonotic diseases where birds are present in large densities. Since few methods are effective for deterring crows, they are often poisoned, trapped and killed. Recent studies conducted at the University of Washington  have determined that crows gather around their dead in a circular fashion, and proceed to avoid the area completely for a period of time (Swift \& Marzluff, 2015). In this study we evaluated the efficacy of `` crow effigies", or dead decoy conspecifics as crow deterrents. During the winter months we presented ``healthy" upright models and ``dead" upside down models to nearby crow roosts in order to observe a potential change in crow roosting densities. The presentation of an effigy was significantly effective in deterring crows. Throughout the testing period, crow populations at roosting sites decreased by an average of 78\%. Using a crow effigy as a deterrent may prevent future crow deaths, and decrease the presence of unwanted crows at a particular location.}

\normaltalk{The Relative Importance of Natural Selection and Sexual Selection in Speciation in the Tanagers}{Rosalyn M Price-Waldman, Amelia J Demery, Nicholas A Mason, Allison J Shultz, Pascal O Title, Kevin J Burns}{A central goal of evolutionary biology is to understand why diversification rates vary so widely among clades. Previous studies have suggested that variation in evolutionary rates of traits primarily under natural selection (for example, morphology, dispersal ability, or climactic niche) may drive speciation. Other studies have suggested that variation in evolutionary rates of traits primarily under sexual selection (for example, plumage, song, or courtship behavior) may drive speciation. However, few studies have compared the effects of multiple traits on speciation or the quantified the relative importance of natural and sexual selection in shaping patterns of diversification. We compiled four datasets of traits under natural and sexual selection (plumage, niche, song, morphology) that have previously been linked to speciation rates in the largest family of songbirds, the tanagers. We used several phylogenetic comparative methods (phylogenetic multiple regression, phylogenetic structural equation modeling) to explore the relative contributions of plumage, niche, song, and morphology to speciation in the tanagers while also modeling evolutionary correlations between traits (for example, beak size constrains song frequency, plumage is shaped by habitat) and their effects on speciation rates. We show that the most variation in speciation rates can be explained by rates of song evolution, and that more variation is explained by the combined effects of signaling traits (song and plumage) than by morphology and niche. However, as a whole the best-fit model included all four trait types. Our results suggest that speciation at macroevolutionary scales involves a complex interplay between traits under natural and sexual selection.}

\normaltalk{Celebrating 32 Years of the Journal of Caribbean Ornithology--how a Humble Journal Has Contributed to both Ornithology and the Development of Young Scientists Throughout the Caribbean.}{Justin Proctor}{In 1988, Dr. James (Jim) W. Wiley conceived and published the first issue of El Pitirre, a newsletter for the Society for the Study of Caribbean Ornithology (now BirdsCaribbean). Thirty-two years later, the now Journal of Caribbean Ornithology (JCO) has grown into the leading peer-reviewed ornithological research journal in the Caribbean. Throughout its tenure, JCO has held true to its policy of maintaining open access to its content on the principle that making research freely available to the public supports a greater global exchange of knowledge. The journal has played a key role in the development of young scientists, mentoring them as they prepared their first papers for publication. It also offers resources for Second Language (ESL) authors making it possible for them to publish their work. Special Issues of the journal (e.g., on forest endemics) have provided a forum for the Caribbean ornithological community to gather and share important work that informs local conservation efforts. JCO recently completed its archives, providing a digitized, free, and open-access collection of every publication from every volume and issue dating back to Wiley's Volume 1, Issue 1 in 1988. This talk will discuss the path that this unique journal has traveled, its fast-paced evolution over the last couple of years, and the work that is currently underway for securing its successful trajectory for the future. Additionally, we'll discuss new ways that the journal can engage its community of researchers and conservationists, and opportunities for getting involved.}

\normaltalk{Static Organisms in a Changing System?}{Darren S Proppe}{Change is a foundational process in natural systems. The habitats utilized by songbirds have never been static. However, the rate of environmental change is increasing rapidly due to the direct and indirect impacts of human development. Rapid change is not without consequences, with a number of potential or realized mismatches having been identified. Interspecific phenological mismatches that result in resource scarcity are some of the most prominent examples. But behavior is also shaped by particular cue/response systems -- and many of these systems are subject to alteration by human activities. Birds may respond to cues and signals in ways that are no longer beneficial, and in some cases ingrained responses may even be detrimental. On the other hand, there are many documented examples of populations and species that alter behavior in response to changing external conditions. Critical functions such as migration, habitat selection, foraging, and communication appear subject to some level of plasticity. It is expected that species-specific responses to environmental change will vary considerably, which elevates the need for widespread inclusion of behavioral principles in conservation and management. This presentation is designed to spur creative thinking by exploring several examples of plastic behavior and their potential fitness correlates. It also serves as an introduction to some of the concepts that will be used to address particular behaviors and situations in this symposium.}

\normaltalk{Landscape Genomics of Desert Birds Reveals that Multiple Mechanisms Cause Heterogeneity Across Chromosomes and Species}{Kaiya L Provost, Stephanie Shue, Meghan Forcellati, Brian T Smith}{The evolution of intraspecific variation across the landscape acts as the basis for diversification. Selective and stochastic factors influence the spatial structuring of intraspecific variation, but how uniformly these operate across codistributed species and different portions of the genome are unclear. For example, selection may provide a homogeneous signature across the genome/phenotype, whereas genetic drift may lead to heterogeneity. Here, we investigate spatial drivers of genomic/phenotypic divergence in the Sonoran and Chihuahuan desert avifauna, characterizing which factors influence variation across the landscape. We re-sequenced genomes (n=221) and quantified morphometrics (n=364 museum specimens) for ten species. We tested four mechanisms that may structure intraspecific variation (contemporary demography, geographic distance, environmental resistance, and paleo-climate) and used general dissimilarity matrix modeling to select the model that best explained genomic, chromosomal, and phenotypic spatial patterns. Most species show population structure, some with admixture, between the deserts. Across datasets, environment and demography each explains 32--36\% of observed genotypic/phenotypic variation, while distance and paleo-climate explain 13--14\%. However, chromosomes rarely share the same best predictor within species. Collectively, our results show variation within and between genomes, chromosomes, phenotypes, and species are predicted by alternative spatial factors. This observed discordance and heterogeneity may characterize the early stages of evolution and diversification in avian communities.}

\normaltalk{Global Distribution and Abundance of Naturalized Parrots: Influences of Trade, Culture, and Climate}{Stephen Pruett-Jones}{Parrots are among the most endangered birds in the world, and yet they are also among the most widely distributed outside of native ranges because of the human pet trade. I review the data on the distribution and status of naturalized parrots worldwide. Based on the GAVIA database, no less than 75 species of parrots are now breeding or are established in 88 countries or territories. The United States hosts the majority of species, with 40 breeding or established species, and Spain is second with 14 species. A recent detailed analysis of records for the contiguous USA shows that 25 species are breeding in 23 states. The rose-ringed parakeet is the single most widely distributed species, now established in 52 countries, with monk parakeets found in 27 countries. Nevertheless, in many countries, e.g. the USA and Spain, monk parakeets are numerically the dominant introduced species. Across countries, the number of naturalized species declines with increasing latitude and increases with economic of the individual country. Many countries that now host established parrots have no records of importing the individual species, suggesting both gaps in our knowledge and widespread dispersal of parrots from one established population to new habitats.}

\normaltalk{Unexpected Genetic Structure in the Likely Extinct Hawaiian Ou (Psittirostra Psittacea)}{Natalia Przelomska, Helen F James, Logan Kistler, Michael G Campana, Robert C Fleischer}{Hawaiian honeycreepers represent one of the most extensive avian adaptive radiations, with great variation in morphology, ecology and behavior among $>$50 species that evolved over 5-7 my.  Most lineages within the radiation show differentiation among islands at the level of species or subspecies, but a few appear undifferentiated among islands (e.g. Iiwi, Apapane, Ou).  These are species that feed primarily on nectar or fruit, have been observed to have lower site fidelity than insectivorous and granivorous species, and have evidence of long-distance movements in search of food.  The Ou (Psittirostra psittacea) was a common, largely frugivorous species.  It was found in low and mid-elevation forests on the islands of Kauai, Oahu, Molokai, Maui, Lanai and Hawaii.  It was a nomadic forager, following seasonal availability of its favored fruits.  While some early authors found little to no differentiation among populations, others suggested taxonomic differences based on usually minor plumage or measurement differences.  We used bait capture and next generation sequencing of DNA extracted from 49 Ou museum specimens to assess levels and patterns of genetic variation and differentiation in Ou across their range on the main Hawaiian Islands.  Using thousands of SNP markers, we found that (1) Ou on Kauai are differentiated from those on the other main islands, perhaps to the level of subspecies or higher; and (2) populations on the other main islands show evidence of structure, but a good deal of overlap, suggesting more extensive gene flow among these populations as predicted from their behavior and morphology.}

\normaltalk{Nest Management Increase Breeding Success in an Endangered Grassland Bird}{Florencia M Pucheta, In\'{e}s M Pereda, Adri\'{a}n S Di Giacomo}{The Saffron-cowled Blackbird (SCB, Xanthopsar flavus) is an endangered passerine of the grasslands of southern South America. In Argentina, habitat loss restricted the species into only two small populations that are isolated from each other and are not protected. In 2015, we launched a project aimed to increase its breeding success through intensive nest management. We study whether enclosures could be installed around the nests of SCB without causing parental desertion, and whether protected nests could successfully exclude nest predators and increase nesting success. During three breeding seasons, we installed shade-cloths enclosures in 107 incubating nests, and metal mesh enclosures in 98 nests during nestling stage. Adult acceptance of the enclosures was monitored by direct observation, while nest predator's activity was monitored with camera traps. We recommend the gradual mounting of enclosures during the morning, together its enrichment with surrounding vegetation that facilitates parental entrance and allows them to better accept protections. The enclosures, both during incubation and nestling stages, significantly lowered the nest's predation rate ($<$U+03C7$>$2=13.83, p$<$0.01 and $<$U+03C7$>$2=27.14, p$<$0.01; respectively). This implies that reproductive success increased in protected nests (70\%) in contrast to unprotected ones (36\%). Our results indicate that these simple devices can be used to increase the nesting success of SCB's colonies prone to high predation risk, and attempt increase the population size during the next years.}

\normaltalk{The Value of Equitable Co-Creation of Bird Appreciation and Conservation Programs Focused on Culture, Pride and Participatory Science with Communities of the Peruvian Amazon}{Karen Purcell, Marilú L\'{o}pez-Fretts, Cecilia Alvarez, Juan Flores}{The Cornell Ornithology Laboratory has been working together with approximately 250 teachers, community leaders, parents and more than 2000 students from 55 communities in the Loreto region of  the Peruvian Amazon since 2017. Together with CONAPAC, a Peruvian NGO, the Lab has co-created the ``Celebrate Birds in the Peruvian Amazon" project. The project focuses on strengthening awareness and pride towards birds and the local environment through the arts; rescuing and valuing local knowledge of birds; promoting active participation in science; inspiring environmental conservation; promoting sustainable avitourism; and improving quality of life. Our goal is to understand how to work in an equitable and impactful way with remote and rural communities in the Peruvian Amazon. For the last three years staff at the Cornell Lab have met remotely with 30-50 teachers and community leaders who travel to Iquitos to meet and co-create the project. Lab staff do an intensive in-person training workshop and communicate with teachers via WhatsApp. Educational materials include art, stories, myths, photos and student work and feature more than 50 local authors. The six-month, annual project integrates birds into the local culture through stories, songs, poems, art and theater and follows the lead of the local communities. The results have been exciting, with more than 36 birdwatching paths created, dozens of birdwatching clubs, community norms changed, and new knowledge shared. We will share evaluation results, barriers, and opportunities, as well as the results of interviews with teachers, parents, community leaders and students.}

\normaltalk{Weight and Abundance Variation of an Aerial Insectivorous (Gray-Rumped Swift/ Chaetura Cinereinvetris) over 30 Years in North-Central Venezuela.}{German Quijano, Miguel Lentino}{Birds that feeds on aerial insects are presenting widespread population declines. It is thought that the main reasons of this decline are climate change, deforestation and an overall pattern of decline in insect diversity and abundance. Members of the Apodidae family (Swifts) are very fast birds that daily move several kilometers to feed exclusively on insects or spiders caught on the wing. Several species of this family are presenting population declines on temperate zones but there are no studies about this in Latin America. Given the decline in swifts and insects populations, it's possible that neotropical swifts populations and its weight can be declining too, but has not been noticed because the lack of studies of this birds in Latin America. Using 30 years of data (since 1991 to 2020) from a long term bird banding program, we estimate the abundance and weight variations of a Gray-Rumped Swift (Chaetura cinereiventris). The abundance was estimated using the numbers of captures/sampling effort per year and the weight variations using the mean weight per year. We found a significate diminution of the abundance of this swift in Parque Nacional Henri Pittier, Although the weight has been stable over the past 30 years. This indicates that number of Gray-Rumped Swift are declining but there are other reason than food resources affecting its abundance.}

\normaltalk{Unexpected Dimensions from Banding in Avian Space and Time: Connectivity, Physiology, Phenology, Habitat Values Illuminating Full Annual Cycle Events.}{C. John Ralph, John D. Alexander, Jared D. Wolfe}{We will explore this special form of monitoring through a spatial perspective: looking at data collection and analyses as we move from single-station banding efforts, and expanding out to continentally-scaled collaborations.  A place-based capture station model, aka observatory, is the basic minimum viable unit.  Many fascinating insights have come from such long-term stations.   We will briefly touch on insights from Europe and the Americas, looking at central life-history questions of connectivity, habitat value, and body condition across a broad array of species.  As we expand to a cluster of stations we will touch on our results from our and others' networks of stations in the tropics and temperate zones. Observatories as a catalyst for regional collections of stations' data, we have recently seen amazing examples looking at landscape scale habitat values, through innovative data mining of full annual cycle events driven by patterns of phenology and climate.  Finally, continentally-scaled banding programs result in interesting insights from a wide variety of taxa, including waterfowl banding, the monumental ``Vital Rates of North American Landbirds", and even hummingbird age and sex- specific migration details.  For the future, we will touch on our aspirational paradigms to approach bird conservation in new, bold, directions, to inform a fuller understanding of the full annual cycle of priority species.  Using that information, we are confident we will come to be intimately knowledgeable about the relative impact that various limiting factors that can be readily adapted into deliverable conservation.}

\normaltalk{Seasonal Variation in Fecal Glucocorticoids and Their Relationship to Reproductive Success in Captive Populations of an Endangered Parrot}{Brian Ramos-Guivas, Jodie M Jawor, Timothy F Wright}{Many species of birds are threatened with extinction and captive breeding programs to protect these species from extinction and produce individuals for eventual reintroduction are becoming more common. Under captivity, animals may have different energetic requirements and have different levels of glucocorticoid hormones compared to individuals in the wild, potentially altering reproduction and other key behaviors and complicating captive breeding. The Puerto Rican Parrot (Amazona vittata) recovery program provides a good platform to understand how glucocorticoids levels may relate to reproductive success under captive conditions. We measured fecal glucocorticoids of males of breeding pairs from two captive populations of Puerto Rican Parrots over two breeding seasons.  Fecal samples were collected overnight from males while females incubated eggs and/or chicks in the breeding cavity and fecal glucocorticoids were measured with a corticosterone ELISA kit. Our results showed that glucocorticoids levels during the pre-breeding of 2018 correlated to total eggs and to fertile eggs laid by pairs. We found seasonal differences among males' glucocorticoids levels, with higher glucocorticoid levels during the pre-breeding season of 2018 compared to both breeding seasons. The mean value of fecal glucocorticoids in 2017 was higher than in 2018 but not statistically significant. El Yunque captive population showed higher glucocorticoid levels compared to the Rio Abajo captive population. There was no clear relation between mean glucocorticoids of males and our primary measure of reproductive success, the number of fledglings produce by each male. These results provide a baseline for comparison with reintroduced populations of this endangered species.}

\normaltalk{Genomic Population Structure of American Wood Storks -- How many Populations are There?}{Kristina M Ramstad, Natalia J Bayona-V\'{a}squez, Silvia Nassif Del Lama, Stacey L Lance, A L Bryan}{American wood storks (WOST, Mycteria americana) are large wading birds native to the southeast US, the Caribbean, Central America, and South America. Because WOST use thermals to soar, they travel great distances over land but not open water. Thus, post breeding dispersal varies geographically among WOST and gene flow is likely restricted between colonies on opposite sides of the Caribbean Sea and the Gulf of Mexico. In this study, we are using a powerful RADcap genomics approach to test for genetic differentiation among nesting colonies of WOST throughout their range. Specifically, we are asking (1) Are WOST panmictic?, (2) Do nesting colonies represent different populations?, and 3) What is the level of connectivity between WOST at the extremes of their range? To date we have sequenced 73 WOST from two US colonies and 148 WOST from three Brazilian colonies at ~6000 single nucleotide polymorphisms (SNPs). These data suggests low but significant genetic divergence between US and Brazilian WOST and among nesting colonies within the US and Brazil. Additional sampling of WOST colonies in Mexico, Central America and Cuba will allow us to differentiate between clinal or geographically discrete patterns of genetic variation in continental WOST and to test the effect of oceanic isolation. Results of this work will be critical for defining the proper scale of management for WOST, understanding dispersal and connectivity among nesting colonies, and assessing the susceptibility of individual colonies to local extinction over the next century of climate change.}

\normaltalk{The Patterns of Genomic Divergence in a Three-Way Hybrid Zone of Flameback Woodpeckers in the Island of Sri Lanka}{Rashika Ranasinghe, Sampath Seneviratne, Darren Irwin}{Heterospecific gene flow and resulting hybrid zones in the wild provide an opportunity to study underline mechanisms of evolution and speciation. Woodpeckers (Order: Piciformes) are one of the few avian groups that undergo multispecies hybridization. The island of Sri Lanka has two species of woodpeckers that are hybridizing in sympatry: the endemic red D. psarodes distributed in the southern wet zone of the island and yellow D. benghalense distributed in the northern dry zone, resulting in phenotypic and genetic intermediates. We analyzed patterns of genomic divergence across this Flameback hybrid zone. The examination of single nucleotide polymorphisms (SNPs) across the genome revealed a unique three-way hybridization between island-endemic D. psarodes, and two geographically isolated D. benghalense populations. ADMIXTURE analysis showed extensive introgression among three genetically distinct Dinopium populations in sympatry. Elevated genetic differentiation between allopatric D. benghalense reflects evidence of third historical colorization of D. b. punticole from mainland India to northern Sri Lanka. Consequently, the North-western shore of the island (Mannar peninsular) which is closest to mainland India consists of the founder population of D. b. punticole from most recent invasion, whereas the North shore (Jaffna peninsula) contains the endemic subspecies D. b. jaffnenses. SNPs-based PCA showed that genetic diversity across these hybrid zones is concordant with their biogeographical distribution, which provides strong evidence for divergence due to isolation by distance with ongoing gene flow. These hybridization events between multiple-pairs of Dinopium woodpeckers provide a rare opportunity to study reproductive isolation and genomic architecture of speciation in an island biogeographic standpoint.}

\normaltalk{Characterizing Haemosporidian Parasite Prevalence in a Pennsylvania Bird Community}{Suravi Ray, Spencer C Galen, Jason Weckstein, Emily N Ostrow, Janice Dispoto, Lisa Kiziuk}{Very little is known about malaria parasites that infect non-human vertebrates, and how the life history of their hosts can affect parasite infection rates. Over five years during the spring and fall migratory seasons and the summer breeding season, we sampled 1453 birds of 66 species at Rushton Woods Preserve in Newtown Square, PA and screened them for avian haemosporidian parasite infections. We used a standard nested PCR and Sanger sequencing approach to amplify the cytochrome b gene sequences to identify parasite lineages from three haemosporidian genera, Plasmodium, Parahaemoproteus, and Leucocytozoon, and used BLAST searches against the comprehensive haemosporidian database MalAvi. We found 113 unique lineages of haemosporidians infecting the birds at Rushton Woods, out of which 33 lineages are potentially new. We used these data to better understand the dynamics of host-parasite infection rates in relation to host life history characteristics such as age, sex, migratory status, season, and the phylogenetic relatedness the host species. Using the results of the parasite screening, we determined which factors best explain variation in infection prevalence in the Rushton Woods avian community. We also used these data to assess whether individual haemosporidian lineages are host specialists or host generalists. Since this avian community includes a mix of resident breeding and migratory bird species, we also asked whether resident and migrant bird communities have different parasite assemblages. Furthermore, by screening individual hosts that were recaptured across seasons and years we also assessed changes in infection status over time. In summary, long term intensive sampling of haemosporidian parasites at Rushton Woods Preserve is contributing to our understanding of how variation in avian life history characteristics affect parasite transmission, diversity, and prevalence.}

\normaltalk{Breeding Biology of Two Neotropical Austral Migrants at the Center of Argentina: Vermilion and Fork-Tailed Flycatchers}{Mar\'{i}a E Rebollo, Alex E Jahn, Jos\'{e} H Sarasola, Joaqu\'{i}n Cereghetti}{Reproduction is a central aspect of bird life history strategies because it allows persistence of populations time and space. We studied the breeding ecology of Vermilion Flycatcher (VEFL, Pyrocephalus rubinus) and Fork-tailed Flycatcher (FTFL, Tyrannus savana) in the Espinal biome of central Argentina from 2015-16 to 2018-19 by monitoring 112 and 123 breeding attempts, respectively. We compared breeding data between the Parque Luro Provincial Reserve, where livestock has been excluded for five decades, and two private cattle ranches Los Álamos and Giuliani. The breeding season for both species lasted from early November to February. Clutch size, brood size, and number of fledglings were =3 for VEFL and =4 for FTFL. VEFL reduced clutch size with clutch start date by up to13.65\% and had a later clutch start date at the reserve than at Giuliani Ranch, explaining 13.53\% of the variation between sites. FTFL had a later clutch start date in the 2015-16 season than in the 2017-18 season, explaining 11.88\% of their variation between years. The incubation and nestling period were 12-13 and 14-15 days for VEFL, and 13-15 and 13-15 days for FTFL. Breeding success were 12\% and 24\%, respectively. Only one VEFL nest was parasitized by Shiny Cowbird (Molothrus bonariensis), while 34.96\% of FTFL nest attempts were parasitized. The breeding biology of both species is variable and similar to other areas of their distribution and we discuss possible drivers of these patterns in central Argentina.}

\normaltalk{Nest-Site Selection and Breeding Success of Two Neotropical Austral Migrants: a Comparison of Sites with and without Livestock}{Mar\'{i}a E Rebollo, Alex E Jahn, Joaqu\'{i}n Cereghetti, Sebasti\'{a}n A Pereyra Fernandez, Jos\'{e} H Sarasola}{Understanding the mechanisms underpinning nest-site selection offers novel insights about selective forces shaping the breeding ecology of birds, and provides vital information for conservation planning. We evaluated and compared nest-site selection and breeding success of Vermilion Flycatchers (VEFL) and Fork-tailed Flycatchers (FTFL) in areas with and without livestock in the Espinal biome of La Pampa Province, Argentina. VEFL selected nest sites with greater tree cover, whereas FTFL nested in taller trees with narrower trunks. We found a weak positive effect of nest height, diameter of nest-support branches and the absence of livestock on VEFL nest survival, and a weak positive effect of nest orientation on FTFL nest survival. However, although some habitat characteristics varied between the two areas, nest site characteristics of both species in areas with and without livestock did not differ significantly. The nest survival was 12.7\% for VEFL and 28.7\% for FTFL, and did not differ significantly between these two flycatchers. The breeding ecology of these birds has likely been molded by a broad suite of selective pressures, such as severe weather, brood parasites and predators that characterize habitats in central Argentina.}

\normaltalk{Prolonged Indirect Cascading Effects of Hurricane Disturbance on Reproduction of a Threatened, Endemic Amazon Parrot}{Katherine Renton, Alejandro Salinas-Melgoza, Elisa Maya-Elizarrar\'{a}s, Rafael A Rueda-Hern\'{a}ndez, Ver\'{o}nica Ru\'{i}z-Hern\'{a}ndez}{Extreme climatic events are increasing in frequency and intensity, but little is known of their indirect effects on threatened, endemic species. We evaluated the consequences of hurricane disturbance on food and nest-site resources, and reproduction of a threatened, endemic parrot. We determined fruiting phenology and tree-cavity availability in tropical dry forest during 2009-2019, prior-to and following landfall by a minor and a major hurricane on the Mexican Pacific coast. We also evaluated population density and reproductive output of the threatened Lilac-crowned Parrot (Amazona finschi) prior-to and following hurricane disturbance. ARIMA time-series modelling of fruiting phenology indicated that while tropical dry forest was able to recover phenological cycles following impact by a minor hurricane, the subsequent landfall of a major hurricane led to a level shift in fruiting phenology. Tree-cavity resources were reduced by half following hurricane landfall, and remaining cavities had significantly larger entrance diameters. Lilac-crowned Parrots also used nest-sites with significantly larger entrance diameters following hurricane disturbance, and although parrots recovered reproductive output after landfall by a minor hurricane, reproductive output remained low (0.90--1.2 nestlings/pair) during four years after disturbance by a major hurricane. Furthermore, while we determined a high density of 14.8 parrots/km2, abundance of Lilac-crowned Parrots was significantly related to the presence of tall trees. Our long-term data suggest that tropical dry forest may be resilient to low-level hurricane impact, but landfall by a major hurricane was an extreme climatic event that had severe and prolonged cascading effects on reproduction of a threatened, endemic species.}

\normaltalk{Sexual Selection and Mating Systems}{Matthew Reudink, Ken Otter, Jenn Foote, Ann McKellar, Nancy Flood}{Sexual signaling and the strength of sexual selection are often predicated on the differential ability of individuals to acquire resources necessary for survival, reproduction, and mate attraction.  Variation may arise through differential acquisition/defense of desirable resources used to attract mates via intrasexual competition, or investment of those resources into condition-dependent signals used in intersexual displays.  The link between sexual selection and individual condition is often dependent upon relative habitat quality.  Anthropogenic disturbance that affects overall quality of the habitat thus has the potential to disrupt sexual competition and signaling and associated reproductive success.   In addition, alteration in habitat quality can influence the strength of sexual selection if it alters patterns of extra-pair paternity and polygamy. We consider how anthropogenic effects alter competitive ability, or the ability of individuals to reliably express condition-dependent signals in both acoustic and visual signals, and the potential consequences these anthropogenic effects can have on mate choice and the strength of sexual selection, with an emphasis on our focal species, the mountain chickadee (Poecile gambeli).}

\normaltalk{Energy Consumption During Migratory Flight Formation}{Ortal Rewald, Thomas Ruf, Elisa Perinot, Johannes Fritz, Leonida Fusani, Bernhard Voelkl, Alexei Vyssotski}{A considerable part of the bird population performs worldwide seasonal long-distance migrations. One conspicuous feature of several bird species is their flight in V-shaped formation, which is thought to allow birds to save energy by utilizing the aerodynamic up-wash produced by the preceding bird. During a human-guided autumn migration from Germany to Italy (2019), 29 hand-raised northern bald ibises (Geronticus eremita) were equipped with data loggers with GNSS and accelerometers that allow us to determine the position of the birds within a formation, their wing-beat frequency, and overall dynamic body acceleration (ODBA). Four birds were additionally equipped with a portable device to measure heart rate (fH) and estimate their energy expenditure along the flight. We compared ODBA and fH during bouts of following flight to selected periods when the birds were not trailing another bird. Since the relation between fH and metabolic rate is not linear and differs between activities, we created a calibration curve for fH and VO2 during flight and rest using a custom-made wind tunnel that fits the size of a flying ibis. Four ibises were raised and trained to fly with a data logger in the wind tunnel and were habituated to wear a respirometry mask.  Having a calibrated fH equation and monitoring the birds over an entire migration route allowed us for the first time to estimate the metabolic rate and quantify the energy consumption of free-flying migratory birds depending on the flight technique and position within a flight formation.}

\normaltalk{Malaria and Chickadees: Testing Parasite-Mediated Selection in a Well-Studied Avian Hybrid Zone}{Ari A Rice, Robert L Curry, Jason D Weckstein}{Studies of avian haemosporidians have proliferated, yet we still know little about specific factors that make birds more or less likely to harbor these pathogenic organisms. In hosts that naturally hybridize, haemosporidian infections may select for individuals with certain genotypes and favor the persistence of one host species over another through parasite-mediated selection. We tested whether this process occurs in Black-capped and Carolina chickadees, which hybridize across a narrow but rapidly-moving contact zone in eastern USA. We predicted that the seemingly more competitive Carolina chickadees, based on their ongoing northward advance, would exhibit fewer infections but greater parasite diversity than Black-capped Chickadees and hybrids. We collected 960 chickadee blood samples from four sites in SE Pennsylvania and used molecular methods to genotypically identify chickadee individuals, detect haemosporidian infections, and identify parasite lineages based on their mtDNA. We identified 33 lineages from the three haemosporidian genera. Prevalence varied little among chickadee taxa. Meanwhile, (i) infections with multiple parasites were five times more common in Carolina vs. Black-capped chickadees; (ii) Leucocytozoon prevalence varied with locality and season; (iii) Haemoproteus prevalence increased nearly tenfold between summer and winter months; and (iv) hybrid chickadees carried the highest haemosporidian lineage richness. Thus, haemosporidians showed complex patterns of infection in chickadees, and their prevalence alone may be unable to determine whether they affect hybrid zone dynamics. To further test for parasite-mediated selection, we suggest measuring the intensity of each infection and conducting year-round sampling from additional chickadee populations.}

\normaltalk{Application of Bird Banding Data to Inform Migratory Bird Management.}{Ken Richkus, Scott Boomer, Patrick K Devers, Joshua Dooley, Pamela Garrettson, Mark Seamans, Guthrie Zimmerman}{The U. S. Fish and Wildlife Service's Division of Migratory Bird Management (DMBM) provides analytical support to inform regulatory decisions under the Migratory Bird Treaty Act and the Bald and Golden Eagle Protection Act. Operational banding programs have provided seminal information, serving as a basis for the development of international migratory bird management plans. Biologists with the DMBM use band and recovery data to estimate population vital rates including annual survival, and harvest rates for geese, ducks, mourning doves, and golden eagles. Band and recovery data have also been used to delineate subpopulations, movement, and linkages between breeding and wintering regions for waterfowl, double crested cormorants and gull species. In combination with other data, DMBM biologists use band and recovery data to derive estimates of goose abundance (i.e., Lincoln estimates) and productivity. Resulting parameter estimates are used in a variety of decision frameworks to determine sustainable levels of sport harvest or other forms of take such as incidental (e.g., accidental mortality from energy development) or intentional (e.g., damage or conflict reduction). We describe the crucial role of banding data in monitoring and managing North American migratory bird populations.}

\normaltalk{Short-Term Effects of Hurricanes Maria and Irma on Forest Birds of Puerto Rico}{John D Lloyd, Christopher C Rimmer, Jose S Far\'{i}a}{We compared occupancy in local assemblages of birds in forested areas across Puerto Rico during a winter before (2015) and shortly after (2018) the passage of hurricanes Irma and Maria. Using dynamic community models, we found significant changes in detectability, with some species becoming more readily detected after the storms and others becoming more difficult to detect during surveys. Changes in occupancy were equally mixed. Five species -- mostly granivores and omnivores, but also Black-whiskered Vireo (Vireo altiloquus), a migratory insectivore -- occupied more sites in 2018 than in 2015. Thirteen species were less common after the hurricanes, including all of the obligate frugivores. Declines in site-occupancy rates were not only more common than increases, but tended to be of greater magnitude. Our results support the general conclusions that bird species respond largely independently to changes in forest structure caused by hurricanes, but that some dietary guilds, notably frugivores, are more sensitive and more likely to show changes in abundance or occupancy following strong storms.}

\normaltalk{Maximum Population Growth Rates, Stability, and Resiliency of Island Endemic Birds to Caribbean Hurricanes}{Frank F Rivera-Mil\'{a}n}{Hurricanes are catastrophic events that increase the risk of extinction of island endemics.  The frequency of major hurricanes is increasing in the Caribbean.  Long-term monitoring and modeling allowed estimation of maximum population growth rate (rmax).  Assuming the existence of a stable equilibrium point, stability was defined as the constancy of abundance over time and resiliency as the time needed to return to pre-hurricane level (1/rmax).  rmax was the maximum growth rate after a significant decline.  For brevity, here I focused on three island endemics with different ecologies and life histories: the threatened Elfin-woods Warbler (Setophaga angelae; fast-reproduction insectivore) and the endangered Plain Pigeon (Patagioenas inornata; slow-fast-reproduction frugivore) on Puerto Rico and the near-threatened Bahama Parrot (Amazona leucocephala; slow-reproduction frugivore) on Great Inagua.  For the warbler, parrot, and pigeon, rmax had medians and 95\% credible intervals (CRI) of 0.642 (0.507-1.112), 0.250 (0.108-0.483), and 0.196 (0.105-0.441), resulting in return times of 2 years (CRI = 1-2), 4 years (2-9), and 5 years (2-10).  After a 56\% decline from Hurricane Ike in 2008, the parrot showed stability and resiliency, with abundance returning to pre-hurricane level (i.e., mean $<$U+F0B1$>$ SE = 11,639 +- 1,968 in 2003-2008 and 11,237 +- 1,988 in 2012-2017).  However, after 81-85\% and 64-94\% declines from hurricanes Hugo in 1989, Georges in 1998, Irma and Maria in 2017, the warbler and pigeon showed instability with wide abundance fluctuations (i.e., 2,518 +- 451 in 1999-2006 and 564 +- 135 in 2018; 19,345 +- 674 in 1999-2017 and 599 +- 237 in 2019).}

\normaltalk{The Rise and Fall of a Female Mating Preference for Male Heterozygosity in a Lek Mating System}{Pearl R Rivers, Emily H DuVal}{Female mating preferences are often assumed to be consistent and strong over time but the resulting mate choices can be highly variable, influencing the strength of selection on preferred male traits. In a population of lance-tailed manakins (Chiroxiphia lanceolata), a highly polygynous tropical lekking passerine, a 12-year study found that female mate choice favors more heterozygous males. However this result was not the case in a 20-year re-analysis. Instead, correlations between heterozygosity and mating success were present in only a subset of years. We tested three hypotheses to explain this variation in patterns of female choice. First, we tested whether variation in the effect of male heterozygosity on reproductive success was driven by a few highly heterozygous males that were attractive for other reasons, such as experience. Second, we assessed whether annual variation in individual heterozygosity among alpha males in the population affected our ability to detect female choice for male heterozygosity in some years. Finally, we hypothesized that variation in female choice could be driven by variation in parasite abundance and disease prevalence associated with large scale climatic variation; for example, if individual heterozygosity relates to disease resistance potential through MHC diversity. To test this final hypothesis we assessed whether variation in the effect of male heterozygosity on reproductive success was related to measures of El Ni\~{n}o--Southern Oscillation. By testing hypotheses about the drivers of variation in mate choice over time, this study contributes to our understanding of how variable female preferences and choices can influence sexual selection.}

\normaltalk{Monitoring Bird Diversity in Migration Stopover Habitat: Assessing the Value of Extended Duration Audio Recording}{Ellie Roark, Willson Gaul}{Conserving bird populations requires knowledge of bird distribution and habitat use at all stages of their life cycle, including during migration. Automated digital audio recording units (ARUs) can be deployed in the field for long periods of time to efficiently increase both spatial coverage and total amount of monitoring effort, especially at places and times where access is difficult and costly for human observers. ARUs have been evaluated for comparability to human observers during the breeding season in many habitats. However, birds behave and vocalize differently during migration. We compared in-person point count observations to extended duration ARU observations during spring migration on the southern shore of Lake Superior, to assess and refine the applications of ARUs in migratory stopover habitat. We analyzed recordings by conducting desk-based ``listening point counts" which can be completed by any observer with the skills to identify species aurally. We found that when ARUs sampled 1-minute intervals randomly from a five hour survey window, they detected multiple species never found during in-person point counts, and detected the arrival of some migrant species up to two days earlier in the season than did in-person point counts. Temporal trends in species richness that reflect spring migration timing were detectable using both ARUs and in-person point counts. We demonstrate a relatively easy, immediate application for ARUs during migration, offering increased monitoring possibilities in remote locations.}

\normaltalk{Developing Bird Tourism Opportunities and Sustainable Livelihoods in the Caribbean--the Caribbean Birding Trail}{Holly Robertson, Lisa Sorenson, Venicio Wilson, Rick Morales, Alison D Ollivierre}{The Caribbean Birding Trail (CBT) is a network of the best sites on each island for birdwatching, enjoying nature, and experiencing culture. The CBT is being developed by BirdsCaribbean to raise awareness of the unique birds and biodiversity of the Caribbean and to inspire a willingness to preserve them. To reach this goal we are to work with partners on every island to build the local capacity for a sustainable tourism market focused on birds, nature, and heritage. In contrast to mass tourism, the CBT is low volume, low impact, and high yield. Tourism of this sort strives to capture the economic impacts locally, and to empower local communities and small businesses to harness the tourism economy for their own direct benefit, while not harming the environment. Our activities have focused on building the supply (i.e, well-trained guides and infrastructure) and the demand (marketing and promotion). To date, we have held Interpretive Bird Guide Training Workshops in 6 countries, including Grenada, Jamaica, Dominican Republic, Bonaire, Cuba, and St. Vincent and the Grenadines. We have also recently launched a redesign of the CBT website that features the best birding sites, guides and tour operators in the islands. We are partnering with tour companies to promote birding trips to the Caribbean using our local guides, and we are assisting partners with preparing sites to receive visitors. Much more work is to be done and we invite you to join us!}

\normaltalk{Population History and Natural Selection Shape the Genomic Landscape of Diversity in Two Codistributed North American Woodpeckers}{Lucas Rocha Moreira, John Klicka, Brian T Smith}{Elucidating the unique and shared factors that shape genomic diversity among species will clarify the role of constraint and stochasticity in genomic evolution and local adaptation. In this study, we investigate the mechanisms underlying differences in the genomic landscape of diversity in two codistributed woodpecker species, the Hairy (Picoides villosus) and Downy (P. pubescens) Woodpeckers. These two species inhabit a wide variety of habitats in North America, exhibiting remarkable parallel patterns of geographic variation in plumage and body size. We used whole-genome resequencing on 140 individuals of Downy and Hairy Woodpecker from across North America to (1) investigate the role of mutation, recombination, and natural selection on regional patterns of genetic diversity, and (2) test for shared signatures of natural selection that might indicate parallel mechanisms for local adaptation. Despite shared population history and exposure to common selective pressures, Hairy and Downy Woodpeckers exhibited different patterns of genomic diversity. Downy Woodpeckers show higher nucleotide diversity and lower genetic differentiation in comparison to Hairy Woodpeckers, which can be largely attributed to differences in effective population sizes, recombination rates, and levels of gene flow. Signatures of natural selection indicate that non-parallel genomic regions underlie local adaptation in these two species. A closer look at the loci under selection reveals a range of functions, including metabolism, immunity, and thermoregulation. These findings suggest that differences in intrinsic properties of the genome, as well as historical stochasticity, lead to non-parallel evolutionary trajectories in the genomes of the Hairy and Downy Woodpeckers.}

\normaltalk{Urban Ornithology:  Moving from Pattern to Process}{Amanda D Rodewald}{Studies of urban birds increased dramatically in number over the last two decades.  The surging interest in urban ornithology has promoted expansion in the geographies, topics, and taxa studied as well as the complexity of the questions addressed.  We are now in an exciting period of transition, in which the field is maturing from one almost entirely focused on describing patterns to one that identifies and seeks to understand processes.  In this talk, I will highlight key insights about the behavioral, demographic, and physiological mechanisms that are thought to be responsible for widespread patterns described in urban ornithology.}

\normaltalk{Vocal Behavior and Microgeographical Variation in Song Types in a Neotropical Songbird}{Juliana Rodr\'{i}guez-Fuentes, Carlos D Cadena, Jorge E Avenda\~{n}o}{Characterizing the vocal repertoires of bird species is a first step to understanding the diversity of their vocalizations and their use in social and sexual contexts, as well as to uncover temporal and geographical patterns in vocal variation. Despite the high species richness of the Neotropical avifauna, vocal repertoire sizes and the function of vocalizations of species remain poorly documented compared to species from the temperate zone. We present the first description of the vocal repertoire, diel variation and song-type sharing of the Gray-browed Brushfinch (Arremon assimilis), an Andean understory bird, based on recordings of vocalizations obtained for 22 territorial pairs. We found that A. assimilis produces four types of calls, sex-specific solo songs, and duets. Males have on average 10 ± 2.1 different solo songs, which are not divided in sections like in other sparrows, but rather are a collection of high-pitched, modulated elements combined in various ways. Similarity in repertoire-use declined with distance between neighbors to the point that no song-types were shared among individuals located in sectors separated only 1 km from each other. Vocal activity peaked in the early morning for calls and male songs, and calls showed an additional peak near dusk. Our results provide the first evidence of female solo songs and duets in A. assimilis, which are seemingly rare behaviors among Neotropical sparrows. The pattern of microgeographical variation in song sharing we observed has not previously documented in similar studies of other tropical species.}

\normaltalk{Nutritional Implications of Aquatic Prey in the Diet of a Riparian Songbird}{Samantha L Rogers, Brian C Verrelli, Santiago Lima, Paul A Bukaveckas, Lesley P Bulluck}{Riparian songbirds provision their young with a variety of aquatic and terrestrial prey that supply energy and essential nutrients for growing nestlings.  Aquatic insects, in particular, are rich in long-chain polyunsaturated fatty acids (LC-PUFAs) that confer benefits for growth and neuronal development to vertebrates sequestering these lipids from their diet.  Prothonotary Warblers (Protonotaria citrea) provision nestlings with aquatic and terrestrial prey, although the proportions of these prey types vary among individuals and populations. This ongoing research aims to: 1) describe seasonal and spatial variation in the provisioning of aquatic insects to Prothonotary Warbler nestlings and 2) identify consequences of diet to nestling condition.  Fecal and blood samples were collected from nestlings at two sites along the James River in Virginia where emergent insects comprise a significant portion of nestling diet. Nestling diet breadth was estimated through DNA metabarcoding of arthropod mitochondrial CO1 sequences from fecal samples.  LC-PUFA content was quantified from plasma samples using liquid chromatography tandem mass spectrometry (LC-MS/MS).  Plasma is reflective of short-term dietary lipid changes and is less destructive than other tissue sampling methods.  A subset of LC-PUFAs in nestling plasma differs between sites and throughout the breeding season. Surprisingly, we found evidence for a negative relationship between fatty acid concentration and nestling body condition.  Forthcoming metabarcoding results will be presented alongside fatty acid data. This work is unique in that it combines diet analysis approaches to better understand the importance of aquatic prey for nestling success in riparian habitats. }

\normaltalk{Kinship and Fine-Scale Genetic Structure in a Cooperatively Breeding Grayish Baywing (Agelaioides Badius)}{Juan Manuel Rojas Ripari, Leonardo Campagna, Bettina Mahler, Juan Carlos Reboreda, Irby Lovette, De M\'{a}rsico Mar\'{i}a Cecilia}{Cooperative breeding is a reproductive system in which one or more adults (helpers) assist others in rearing their offspring. In birds, cooperative breeding occurs in 9\% of species and it is associated to remarkable variation in mating systems and patterns of social organization. Determining the genetic relationships within and among social groups is a necessary step to understand the conditions under which helping behavior can evolve and be maintained. We used ddRAD sequencing to study the social organization and fine-scale genetic structure in a Neotropical facultative breeder, the Greyish Baywing (Agelaioides badius). During southern breeding seasons (Nov-Feb) 2015-2016, 2016-2017 and 2017-2018, we sampled blood from adults and nestlings at 33 nests in a population from eastern Argentina. Helpers were mostly males (90\%, n = 21), and based on bioinformatics analyses of 523 SNPs, they were previous offspring of the breeding pair (5 males, 24\%), first-order relatives of the breeding male (9 males, 43\%), or they were unrelated to both breeders (5 males and 2 females, 33\%). Helpers did not gain paternity/maternity in the nests where they assisted and we did not detect any clear genetic structure at fine spatial scale (i.e., kin neighborhoods) in the study population. The results suggest that cooperative breeding in Greyish Baywings occurs primarily within family groups and it could be driven by kin selection. However, the occurrence of unrelated helpers also suggests that non-reproductive direct benefits such as increased access to resources may influence helping behavior in this species.}

\normaltalk{Patterns of Transport and Introduction of Non-Native Parrots to the United States}{Christina M Romagosa}{The wildlife trade is an important economic activity that creates global movement of millions of individuals annually. This anthropogenic transport of wildlife is a major threat to biodiversity by depleting wild populations and introducing invasive species, disease, and parasites. The transport and introduction of avian species have been shown to be non-random processes among higher taxonomic groups. Data on species transported to and from the United States were used for a synthetic review of US trade in live parrots over 30 years and related to introductions. Trade-related dynamics have led to changes in parrot species used for trade, individual quantities of those species traded, and their geographic origin. Avian groups that are traded preferentially, such as parrots, are more likely to contain species introduced to the wild and subsequently establish free-living populations. These trends in parrot trade and introductions reflect those seen in other regions outside the United States, and help inform how trade contributes to global biological invasions.}

\normaltalk{Importance of the BBS to the Future of Avian Conservation in North America}{Kenneth V Rosenberg}{In response to Rachel Carson's Silent Spring, Chan Robbins initiated the North American Breeding Bird Survey (BBS) in 1966 to track the health of the continent's birdlife and provide a foundation for their conservation. The BBS remains our primary source of long-term population data today--the recent publication in Science documenting the loss of 2.9 billion breeding birds since 1970 relied heavily on BBS trends for 434 (82\%) of the 529 species analyzed. Partners in Flight's (PIF) widely used Avian Conservation Assessment Database uses BBS results to directly assess population trends, rank the importance of Bird Conservation Regions, estimate global population size, and project future trends as a metric of conservation urgency. As multiple interacting threats to birds continue, exacerbated by climate change, the BBS is more important than ever as a continent-wide tool for assessing future conservation efforts aimed at reversing these pervasive declines. Implementing recommendations in the 2020 BBS strategic plan, especially partnerships to integrate the analysis of BBS with regional monitoring data and eBird, will address current limitations due to incomplete geographic coverage, aging demographic of BBS participants, roadside, habitat, and seasonal biases. Sustainable tri-national support will be essential to fulfill Robbins's vision of a continent-wide BBS--a finger on the pulse of bird populations, with the ability to both guide the allocation of scarce conservation resources and sound an early warning to avert a second silent spring.}

\normaltalk{To eBird or not to eBird? Recreation Specialization Explains Citizen Science Contribution Patterns.}{Connor J Rosenblatt, Ashley A Dayer, Jennifer N Duberstein, Tina Phillips, Nicholas W Cole, David C Fulton, Howard W Harshaw, Andrew Raedeke, Jonathan Rutter, Christopher L Wood}{Contributions from citizen scientists within a program vary widely in quantity and quality. Understanding these differences is crucial for increasing engagement, and to successfully manage a citizen science program. The recreation specialization framework, which describes how recreationists vary in skill, behavior, and motives, provides a potential approach to characterize citizen scientists by describing their involvement in complimentary recreational activities. To test this, we paired data from a survey of birdwatchers across North America with information on their eBird contributions (n=28,926). We identified four types of participants based on levels of eBird checklist contributions (non-retained, occasional, moderate, active) and compared the types with continuous specialization scales. We constructed two latent variables to represent appreciation and achievement motivations. For all three dimensions of specialization  (i.e., affect, behavior, cognition), there was a consistent trend of increased specialization scores with contribution amount, with the effect size for the behavioral dimension ($<$U+03B7$>$2=0.270) greater than that for the affective ($<$U+03B7$>$2=0.150) and cognitive ($<$U+03B7$>$2=0.150) dimensions. For motivations, there was a moderate effect size for achievement motives ($<$U+03B7$>$2=0.141) but a small effect size for appreciation motives ($<$U+03B7$>$2=0.048). These findings suggest behavioral factors (i.e. taking trips to view birds) are most strongly related to the amount of contribution. Additionally, all participants enjoyed appreciative aspects of birding, whereas active participants had stronger achievement motivations than others. eBird may be able to increase contribution rates through encouragement of local birding in scenic natural areas, which may appeal more to those not motivated by competition and less interested in traveling to birdwatch.}

\normaltalk{Data Integration for Autonomous Recording Units and Traditional Point-Count Sampling}{Beth E Ross, Jesse M Wood, Amy K Tegeler}{New technologies to survey wildlife, such as autonomous recording units (ARUs), may optimize resources if combined with traditional survey methods such as point-count surveys. While ARUs can be used to monitor changes in species across time, developing a monitoring program integrating ARUs with point counts using a single statistical model has yet to be explored. The goal for our project was to develop an occupancy model using different data sources with variable observational quality. To achieve this goal, we developed a community occupancy model based on ARU and point-count surveys conducted on avian species in the Piedmont region of South Carolina. The latent presence/absence state was modeled to include covariates related to forest management practices in the region (e.g., thinning and burning of loblolly pine), while detection probability incorporated covariates related to wind, time of day, and survey type (ARU or point count). As expected, our results indicated differing relationships with forest management practices for different bird species. While many species responded positively to increased basal area, some showed negative responses (e.g., Yellow-breasted Chat and Prairie Warbler). Detection probabilities were similar for both surveys (point counts = 0.31, 95\% CI = 0.29-0.33 vs. ARU = 0.28, 95\% CI = 0.26-0.30). Variance associated with occupancy was lower for the integrated model than either data set alone. Overall, our approach presents a method to allocate resources for different survey methodologies and could be applied to other remote technologies such as camera traps or unmanned aerial vehicles.}

\normaltalk{Spatial Non-Stationarity in North American Breeding Bird Distributions}{Josee S Rousseau, Matt G Betts}{Distribution models provide insights into species' ecology and distributions and are used to guide conservation and land management priorities. However, most distribution models assume stationarity, which refers to a model or process for which the parameters are fixed through space. If a species has a stationary distribution, the relationship between abundance and habitat variables should be constant across a breeding distribution. We used Breeding Bird Surveys, climate and remote sensing data, and a model transferability methodology to test whether 138 species of North American birds show consistency in habitat relationships across their distributions. We also assessed whether species' traits were correlated with levels of stationarity in distributions.  Lastly, we tested whether prediction accuracy between modeled regions decreased with 1) geographical distances, 2) level of extrapolation, and 3) were affected by a `boundary-core' effect. Our results suggest that very few species exhibit spatial stationarity in their habitat relationships.  Species with smaller breeding distributions were more likely to have stationary distributions. Lastly, the prediction accuracy between modeled regions decreased with geographical distances and level of extrapolation. Improving the accuracy of distribution models, through accounting for non-stationarity, should increase our ability to predict the presence and abundance of birds across breeding distributions and thus increase the success of conservation efforts.}

\normaltalk{Ecomorphology of Neotropical Bird Wings}{Ernesto Ruelas Inzunza}{One of Seebohm's rules posit that island birds have wings that are more rounded and carry more weight than those of its continental counterparts, with little research on its causes and consequences. Here, I provide empirical data of several parameters of mainland bird wing morphology and explore its ecological and evolutionary correlates. I collected wing data from 61 species of sedentary, Neotropical birds with elliptical wings from different habitat types and elevations in Costa Rica. From these, I calculated wing-loading (W) and aspect ratio ($<$U+039B$>$) and found a mean W is 20.73 N m-2 ± SE 0.69 and a mean $<$U+039B$>$ is 4.23 ± SE 0.05. Upon testing the response of wing metrics to factors that could influence their W and $<$U+039B$>$, I found that foraging guild and taxonomic family showed effects that are statistically significant, whereas elevation, vegetation type, sensitivity to habitat disturbance, and range size do not exhibit noticeable differences. Because factors associated with wing shape represent ecological (foraging guild) and evolutionary (taxonomic family) drivers, I also built a phylogenetic tree using sequences from three mtDNA regions. When mapping W and $<$U+039B$>$ values to the resulting tree, I found these are broadly distributed in all clades, suggesting that neither of them is a deeply rooted trait. Wing morphology hence might be a labile feature of bird morphology, and consequently these birds' dispersal, movements, foraging ability, and other characteristics associated with their flight are not traits deeply rooted in their evolutionary history. Ecomorphology of island bird wings remains unexplored.}

\normaltalk{Structured Bird Monitoring in Latin America: Applications of PROALAS}{Viviana Ruiz-Gutierrez, Anna Lello-Smith}{Understanding patterns of bird abundance and distributions is one of the biggest challenges facing avian research and conservation efforts across the Americas. Large-scale monitoring efforts, such as the Breeding Bird Survey in the US, have been successful at generating this information for much of the US and Canada. However, similar efforts are lacking for much of the Latin America. Although avian research and monitoring efforts do exist, these present a common set of challenges for making scalable inferences or integrating with existing efforts: 1) data are often project specific and not readily available, 2) protocols lack a consistent set of variables or compatible methodologies, and 3) protocols vary widely in objectives and often based on suggestions for monitoring birds in temperate regions. Here, we present the design and application of PROALAS- an occupancy-based, bird-monitoring program, specifically designed for regions with lower detection probabilities and higher diversity. We provide details on the factors that informed the protocol and describe specific aspects and sampling guidelines. We will present results of different applications of the protocol in Mexico, Guatemala and Colombia.}

\normaltalk{Accounting for Survivorship Bias in Estimation of Migratory Connectivity from Archival Tags}{Clark S Rushing, Aimee M Van Tatenhove, Andrew J Sharp, Viviana Ruiz-Gutierrez, T S Sillett, Paul W Sykes Jr}{Advances in tracking technology have provided important insights into the movement, demography, and dynamics of mobile species but even the smallest satellite geologgers remain too heavy for most migratory species. Archival tracking tags, which save weight by storing rather than transmitting location data, have revolutionized the study of small, migratory organisms but because these tags do not transmit data, observed migration data only come from individuals that survive and return to their tagging location. This survivorship bias is problematic for inferring migration patterns if certain migration paths have lower survival than others, a situation that is common in wild populations. In this paper, we demonstrate that survivorship bias can significantly distort patterns of connectivity inferred from archival tags but that this bias can be reduced using a novel integrated tag-recovery/survival model. Using simulations, we show that the integrated model not only reduces survivorship bias but also provides accurate estimates of regional non-breeding survival. We demonstrate the utility of the approach using tracking data from Painted Buntings, a declining migratory songbird that is thought be threatened by illegal pet trade in parts of its non-breeding range. Consistent with predictions about where Painted Buntings are most at risk of capture, the integrated model indicated that male buntings wintering in Cuba are 10-15\% less likely to return to the breeding grounds that birds wintering elsewhere in their range. Improving inferences from archival tags through further development of this approach will help this technology reach its potential to transform the study of migratory organisms.}

\normaltalk{Seasonal Dynamics of Flock Interaction Networks in Lowland Amazonian Rainforest}{Cameron L Rutt, Philip C Stouffer}{Although lowland tropical rainforests were once widely believed to be the archetype of stability, seasonal variation exists. In these environments, seasonality is defined by rainfall, leading to a predictable pattern of biotic and abiotic changes. Yet most studies of tropical organisms ignore seasonality, even though only the full annual cycle reveals niche breadth. Further, if human-modified habitats display more seasonal stress, then ignoring seasonality also has important consequences for conservation. We examined the seasonal dynamics of Amazonian mixed-species flocks--an important species interaction network--across three habitats with increasing human disturbance. We quantified seasonal space use, species richness and attendance, and four ecological network metrics for flocks in primary forest, small forest fragments, and regenerating secondary forest in central Amazonia. Our results indicate that, even in lowland rainforest, mixed-species flocks exhibit seasonal differences. During the dry season, flocks increased the size of their core areas, included more species, and displayed network structures that were less complex and less cohesive. We speculate that because most flocking species nest during the dry season--a time of reduced arthropod abundance--flocks are simultaneously constrained by these two competing pressures. Moreover, seasonal adjustments were most pronounced in forest fragments and secondary forest, habitats that are less buffered from the changing seasons. Together, this study suggests that the conservation value of human-modified habitats depends upon seasonality and raises important questions about how rainforest organisms will cope with an increasingly unstable climate.}

\normaltalk{Bendire's Thrasher Nest Survival in Relation to Vegetation Characteristics in the Southwest United States}{Allison J Salas, Martha J Desmond, Fitsum Abadi}{Arid land birds are among the fastest declining avian assemblages in North America. Within this assemblage, the highly cryptic Bendire's Thrasher (Toxostoma bendieri) is undergoing strong population declines and considered an IUCN Red List species. To enhance our fundamental understanding of the breeding biology of this understudied bird, it is crucial to examine key vegetation characteristics associated with nests, territories and surrounding landscapes; common nest predators; food availability; and possible interspecific competition. We examined Bendire's Thrasher nest survival in relation to local and landscape variables. Nests (n = 75) were located and monitored until fledging or failure during 2018 and 2019 within the Chihuahuan Desert. Of the 75 nests monitored, 43 nests successfully fledged young, with predation being the main cause of nest failure. The most common predator documented was Chihuahuan Ravens (Corvus cryptoleucus) as well as coyotes (Canis latrans), Javelina (Pecari tajacu) and rodents. Preliminary nest survival estimates were calculated using the nest survival model in Program MARK, through the RMark package in R. The preliminary daily nest survival estimate was 0.97 (SE=0.005) and the preliminary cumulative nest survival rate was 0.45 (SE=0.06), assuming a 26-day breeding period. Preliminary model results suggest that arthropod abundance and foraging space positively influence nest survival at the territory scale, while measures of habitat heterogeneity at the landscape scale, including mean patch size, edge density, and patch dominance negatively influence nest survival. These initial results suggest that factors influencing nest survival are operating on multiple scales.}

\normaltalk{Effects of Volcanic Ash Deposition on Bird Assemblages after an Eruption Event in the Colima Volcano, Mexico.}{Luis Enrique S\'{a}nchez-Ramos, Adolfo G Navarro-Sigüenza, R Irma Trejo-V\'{a}zquez, Andr\'{e}s Lira-Noriega}{Volcanic eruptions are a natural phenomenon that produce different types of disturbance that impact the community in diverse intensities and extensions. The disturbance effects can range from the total extermination of the biota to the loss of some species with particular traits. In this study, we aim to identify the effects of volcanic ash deposition on vegetation cover and structure, measure the physical and environmental damage, and evaluate their impact on the functional diversity of bird assemblages after an eruption that happened in July 2015 in the Colima Volcano, western Mexico. Changes in vegetation cover and structure were identified using a Chi-square test performed on SAVI obtained from satellite images, while environmental damage values were drawn from vulcanological reports. Changes in the composition and structure of the bird assemblages, that were surveyed 10 days before and 10 days after the eruptive event, were analyzed by a PERMANOVA using the functional attributes of the species and an ANCOVA for the structure. The results suggest major changes in plant cover and structure to the southern slope of the volcano. The assembly of birds was filtered by the environment where the determining functional traits are the main diet, tolerance to maximum temperatures, and precipitation. The main negatively affected diets are frugivorous, nectarivorous, and granivorous, the tolerant diets are omnivorous and insectivorous, and the advantageous diet is scavenger. We conclude that some attributes are constant in the environmental gradient while others are lost due to the environmental filter.}

\normaltalk{The Role of Learning, Acoustic Similarity and Phylogenetic Relatedness in the Recognition of Distress Calls}{Luis Sandoval, Yingtong Wu, Anna Petrosky}{Many groups of vertebrates produce distress calls when attacked by predators as a last attempt to survive. Few studies have been carried out to understand if recognition of distress calls involves learning or acoustic similarity to familiar calls. This study assessed the importance of these two factors as well as phylogenetic relationship in distress call recognition by tropical rainforest birds. We measured the responses to familiar and unfamiliar distress calls using playbacks of a sympatric species, Oorange-billed Sparrow (Arremon aurantiirostris), and a closely-related allopatric species, White-eared Ground-Sparrow (Melozone leucotis), respectively. In addition, to test differential recognition based on relatedness to the call sender (a nine-primary Oscine), species that approached to the playback stimuli were categorized into four taxonomic groups, namely nine-primary Oscines, other Oscines, sub-Oscines, and non-Passeriformes. We found that individuals called back to sympatric and allopatric calls within similar amounts of time, supporting the role of acoustic similarity in distress call recognition. Our results also showed that individuals approached faster and spent more time inside the observation zone for sympatric distress calls than allopatric distress calls, and both responses were stronger than the response to white noise (control), which supports the roles of acoustic similarity and learning in distress call recognition. We also found that sympatric distress calls elicited approaching reactions from closely related species more than white noise did. We concluded that learning, acoustic similarity, and phylogenetic relatedness play important roles in the response to distress calls.}

\normaltalk{Previously Undescribed Sexual Size Dimorphism and Geographic Variation of Plumage in Orange-Fronted Parakeets}{Siddharth Sannapareddy, James M Maley}{Parrots and other wildlife are under unprecedented threat from human practices, especially the pet trade. In order to further our understanding of this threat, it is important to understand evolution and biodiversity in the species that are threatened. Orange-fronted Parakeets are found from northwestern Mexico to Costa Rica and show differences across their range in the amount of orange between the bill and the eyes (the ``front"). We measured 179 specimens in the Moore Laboratory of Zoology to test several hypotheses regarding sex-specific and geographic differences. We found that the width of the orange front is significantly different between sexes across subspecies, demonstrating marked sexual dimorphism not previously described in the literature. We also found significant differences between subspecies when accounting for sexual dimorphism, with the northern subspecies having the narrowest orange front. Contrary to the literature, the subspecies eburnirostrum had the widest orange front, not the nominate form. Initial analyses suggest potential species-level differences within Orange-fronted Parakeets as currently defined. Our results could have significant conservation implications for this enigmatic parakeet.}

\normaltalk{MAPS: Standardized Cooperative Bird Banding to Inform Landbird Demography and Conservation in North America}{Jamese F Saracco, Danielle R Kaschube, David F DeSante}{The Monitoring Avian Productivity and Survivorship (MAPS) program is a constant-effort bird-banding network established in 1989 by The Institute for Bird Populations to provide data on vital rates of North American landbird species. Since its inception, $>$ 1300 MAPS stations have been operated, and the program has involved hundreds of partners, including individual bird banders, private organizations, and public agencies. Chief among these is the USGS Bird Banding Laboratory (BBL), which has facilitated growth of the MAPS program through provision of bands, permits, and subpermits; by authorizing MAPS as an avenue whereby qualified banders can contribute to broad-scale conservation science; and by working cooperatively to improve both the MAPS and BBL databases. MAPS has contributed records of $>$ 2 million landbirds of $>$ 450 species to the North American Banding Program. These records have been leveraged to provided new insights into the demography, population dynamics, and migratory connectivity of landbirds at spatial extents ranging from individual stations to regional and continental scales. Here we highlight some key MAPS contributions, including identification of temporal and spatial patterns in vital rates; determination of demographic drivers of population change; modeling relationships between vital rates and environmental covariates; use of MAPS data in integrated population models; determining age and sex in captured birds; and understanding molt and effects of pathogens on bird populations. We expect that the role of bird banding and MAPS in informing landbird conservation will only increase in the face of climate and habitat changes in the coming decades.}

\normaltalk{Visualizing the North American Breeding Bird Survey}{John R Sauer, James E Hines, William A Link}{Timely and authoritative analyses are an essential part of the North American Breeding Bird Survey (BBS) program.  The internet has been a particularly effective conduit for users to obtain results and conduct analyses of BBS data; advances in technology and modeling approaches require ongoing research to keep results current.  We developed a map server that provides visualizations of BBS results at multiple scales.  Like USGS Water Resources web applications, it is based on a scalable map.  BBS trend results for 548 species can be displayed at geographic scales ranging from individual survey routes to survey wide. Trend estimates, graphics of annual indices, and spatial displays of regional increase and decline can be displayed for states, provinces, bird conservation regions, countries, and survey wide.  For any region, a custom analysis can be displayed for a user-selected set of start and end years.  The map server facilitates visualization of the relationship between route-specific change and regional population trajectories; users can scroll among routes in a region and display a graph of change for a species on routes the cursor touches. The graphs contain a semiparametric smooth of the counts on the route along with yearly data points that are color-coded to differentiate observers on the route.  Underpinning all these results are additional details of the analysis in the form of metadata, summary spreadsheets that provide results for all species and regions, and program R data structures for each species that provide users with the raw data, models, and detailed model results.}

\normaltalk{Teasing Apart a Century of Land-Use and Climate-Change Impacts on Winter Bird Distributions}{Timothy D Meehan, Sarah P Saunders, Nicole L Michel, Brooke L Bateman, Jill L Deppe, Joanna Grand, Geoffrey S LeBaron, Lotem Taylor, Henrik Westerkam, Joanna X Wu, Chad B Wilsey}{Understanding the relative impacts of climate change and land-use change on shifts in avian distributions has implications for future policies and conservation decision-making. Because global change is a dynamic process, quantifying the mechanisms that shape species' distributions, rather than current distribution patterns, is essential for enhanced ecological understanding and improved projections under global change scenarios. We analyzed changes in the winter distributions of 20 common bird species from 1930 -- 2019 within the eastern United States using Christmas Bird Count (CBC) data and geostatistical occurrence models. We estimated the spatiotemporal changes in probability of occurrence of these species in relation to changes in climate (average min. winter temperature, cumulative winter precipitation) and land cover (proportion of species-specific habitat within a CBC circle). We fit null, climate-only, habitat-only, and climate-habitat models to estimate the variation explained by global change drivers in isolation, as well as in combination. Across species, climate change explained more variation in occurrence probabilities than did land use change, although both processes combined generally explained considerable (i.e., $>$ 30\% on average) variation in occurrences, depending on species. Our results demonstrate that the two most pervasive components of global change -- climate and land use -- have substantially shaped the winter distributions of many bird species over a 90-year period. More broadly, our results can be used to integrate mechanism (i.e., species-specific sensitivities to climate vs. land-use drivers) into projections of species' range shifts under future global change scenarios.}

\normaltalk{Characterizing Differences in Migratory Phenology of Lazuli Buntings Across an Elevational Gradient using RFID Technology}{Kim Savides, Clark Rushing}{Understanding drivers of migratory phenology is important for predicting avian population dynamics, especially under climate change. However, accurate determination of arrival and departure dates is often limited by low detectability of individuals at the start and end of the breeding season due to cryptic behaviors, lower vocalization rates, and shifting territory boundaries. These issues are often overlooked and are particularly problematic in recording departure dates and female arrival dates. We quantified migratory phenology of a long-distance migrant, the Lazuli Bunting (Passerina amoena), using sub-daily digital recapture data of radio-frequency identification (RFID)-banded individuals across an elevational gradient in Cache County, Utah, USA. We marked 183 buntings with RFID leg bands and maintained RFID bird feeders from late-April through late-September in 2019 and 2020. We recorded 44,794 individual feeder visits made by 155 tagged individuals in 2019. This high-resolution spatial recapture data enabled us to test hypotheses on individual and demographic level migratory phenology using a Bayesian capture-recapture model accounting for daily uncertainty in individual presence at the breeding site. Our preliminary results suggest lower elevation birds depart earlier than high elevation birds, and within individual breeding sites, adults depart at similar times regardless of sex, but before hatch-year birds. We will apply the same model framework to determine arrival dates of previously marked individuals in 2020. This modelling framework may be further applied to other frequent-detection datasets, including in the MOTUS network and burrow or cavity-nesting species, to better understand individual and demographic level uncertainty around migratory and breeding phenology.}

\normaltalk{Geographic Variation of Mercury in Breeding Tidal Marsh Sparrows of the Northeastern United States}{Christopher J Sayers, Mackenzie R Roeder, Lindsay Forrette, Daniel Roche, Gaetan L Dupont, Samantha Apgar, Alison R Kocek, Alexandra M Cook, Greg Shriver, Chris S Elphick, Brian Olsen, David N Bonter}{Saltmarsh Sparrows (Ammospiza caudacuta) and Seaside Sparrows (A. maritima) are species of conservation concern primarily due to global sea-level rise and habitat degradation. Environmental mercury (Hg) contamination may present additional threats to their reproductive success and survival. To assess site-specific total mercury (THg) exposure and identify environmental correlates of Hg detection across a large portion of the breeding range where these species co-occur, we sampled blood from adult male Saltmarsh and Seaside Sparrows at 27 marsh sites from Maine to Virginia, USA. The mean THg concentration (±1 SD) throughout the entire sampling range was 0.53 ± 0.29 µg/g wet weight (ww) for Saltmarsh Sparrows and 0.44 ± 0.32 µg/g ww for Seaside Sparrows. Individual THg concentrations ranged from 0.135--1.420 µg/g ww for Saltmarsh Sparrows and 0.153--1.530 µg/g ww for Seaside Sparrows. Model averaging from a suite of generalized linear mixed models supported species- and site-based differences in blood THg. We did not detect any effect of land cover variables surrounding sampled marshes, annual precipitation, bird mass, or latitude on sparrow THg concentrations. Saltmarsh Sparrows tended to have higher blood THg concentrations than Seaside Sparrows at the same marsh sites. We speculate that species-specific THg accumulation is influenced by differences in diet or foraging behavior between these species.}

\normaltalk{Partitioning of Acoustic Niches by Song Frequency and Timing of Singing in a Breeding Bird Community}{Joanna M Sblendorio, Maarten J Vonhof, Sharon A Gill}{Birds depend on vocal communication to deter rivals and attract mates. However, the availability of communication space is often constrained by heterospecific interference. Closely related species are expected to partition signal space to avoid interspecific matings, but how an entire acoustic community avoids signal overlap remains unclear. Strategies may include partitioning of frequency and temporal signal space, but birds may also alter timing of singing in relation to heterospecific song. We hypothesize that acoustic partitioning will occur within bird communities, and that variable mechanisms for partitioning exist. Specifically, we predict a negative correlation between frequency overlap and overlap in timing of singing. We recorded songs of 55 bird species (10-20 males/species) from a breeding community to analyze frequency and temporal partitioning, and deployed a grid of 14 recorders to passively record the community to analyze timing of singing. To create axes of signal space, we conducted a principal components analysis of song traits where the first three PCs accounted for 95\% of interspecific variation and reflected heavy loading of frequency characteristics (PC1), temporal characteristics (PC2), and bandwidth (PC3). The total volume of community signal space was 59.97 PCA units, and volumes of acoustic niches varied from 0.01 to 1.23 PCA units across species. Most species' acoustic niches were clearly partitioned, with the degree of pairwise overlap ranging from 0 to 7.4\%. Our study is a critical step in understanding whole-community partitioning of signal space, and how acoustic competition and community dynamics can drive signal evolution.}

\normaltalk{The Migratory Connectivity Project: Mining a USGS Hidden Resource, the Bird Banding Lab Encounter Database}{Amy L Scarpignato, Emily B Cohen, Peter P Marra}{Following birds as they migrate between geographically disparate places throughout their annual cycles has been challenging so we lack migratory connectivity information for many species. As a part of the Atlas of Migratory Connectivity of the Birds of North America, we synthesized 100 years of bird banding data from the USGS Bird Banding Laboratory Banding and Encounter database. We described migratory connectivity by plotting breeding to winter encounters for 310 species. We quantified the strength of migratory connectivity, MC, for 45 species. MC is negative when populations are further apart between seasons, positive when populations remain together between seasons, and zero when populations have no patterns in distribution between seasons. We classified MC as negative ($<$ 0), none (0), weak (0.1 - 0.3), moderate (0.4 - 0.7), and strong (0.7 - 1). MC ranged from -0.06 to 1. Only one species, Brown-headed Cowbird, had a negative MC. For the remaining species, 52\% were classified as weak (n = 23), 30\% moderate (n = 13), and 18\% strong (n = 8). MC varied within orders (n = 9) and within terrestrial and aquatic (0.34 ± 0.25, n = 19, 0.41 ± 0.32, n = 26, respectively) species but there was no difference between the two groups (t = 0.83, df = 42.78, p = 0.41). The strength of migratory connectivity has consequences for populations and species, thus addressing population declines of migratory birds requires a knowledge of their migratory connectivity and a full annual cycle approach to conservation.}

\normaltalk{The Impact of Natural History on Plumage Color in Cardinalidae}{Benjamin Scott, Kevin Burns}{The conventional paradigm of sexual dichromatism dictates that sexual selection pushes males to be vibrant and colorful to attract mates, while natural selection pushes females to be cryptic to blend into their surroundings. Plumage coloration arises from a complex concert of selection pressures incorporating regional ecology, the light environment, genomic architecture, as well as geographic and bioclimatic factors. Here, we analyze the evolution of coloration in cardinals and grosbeaks (Cardinalidae) from the avian visual perspective and test the relative roles of life history traits in facilitating plumage evolution. We analyzed the correlated evolution between sexes in response to habitat preference, migration, forest dependency, and strata. Additionally, I analyzed the evolution of signaling and countershading patches in relation to how well they blend into their respective background substrate. My results show that males and females evolve under different axis of selection, and the relative roles of life history traits differ between the sexes. In open habitats and dense forests, natural selection drives species to match their light environment, but in semi-open habitats sexual selection drives plumage coloration. Migration, long heralded as a driver of sexual dichromatism, does not drive the evolution of increased male plumage complexity but does lead to a reduction in female plumage complexity. We also used new metrics of trait evolution to show that evolutionary changes in male plumage reciprocally influence female plumage, and this transition aligns with biogeography origins.}

\normaltalk{Integrating Migratory Bird Data for Conservation: What are the Challenges and Opportunities?}{Jill Deppe, Nathaniel E Seavy, William V Deluca}{The growing amount of data on migratory birds provides an unprecedented opportunity: integrating those data to identify the most important places, most pressing threats, and most effective conservation solutions. Efficiently integrating those data requires we address three fundamental challenges: 1) Conceptual: Do we know the right questions to ask in order to advance conservation solutions effectively and efficiently? 2) Logistical: Spatial information on migratory birds is collected by many researchers, and even when it is housed in centralized repositories, assembling it can be challenging. 3) Statistical: What methods do we use to analyze different data types to provide unique, yet complementary, information on the occurrence and behavior of birds throughout the year? Our review of the field suggests that there is active work on all three of these areas. Symposium speakers are at the leading edge of efforts to tackle these challenges. Their presentations will highlight how they are engaged in addressing these three challenges by integrating data to develop conservation solutions.}

\normaltalk{Genomic Evidence for the First Neotropical Ring Species}{Glenn F Seeholzer, Michael J Hickerson, Robb T Brumfield, Brian T Smith}{In a ring species, a species' distribution expands around a barrier and diverges such that the species' distributional termini are reproductively isolated when they come into secondary contact. This paradoxical scenario was considered by Ernst Mayr to be the `perfect demonstration' of speciation, however, examples are rare. In the Dusky-capped Flycatcher (Myiarchus tuberculifer), two phenotypically and genetically divergent subspecies have distinct yet overlapping elevational distributions along the humid east-slope of the southern Andes in Peru and Bolivia, yet, are connected by a continuous ring of phenotypically intermediate populations in the lowlands surrounding the impermeable barrier of the northern Andes in Ecuador and Colombia. To date, no study has tested the key genetic predictions for ring speciation in this system. We obtained low-coverage genomes to determine whether 1) two distinctive forms coexist without gene flow along the east-slope of the southern Andes and 2) if these populations are connected by gene flow through the lowlands surrounding the northern Andes. Our genomic data partially supported both predictions. Two genetically distinct populations come into contact at 1500 m in Peru and Bolivia yet replace each other above and below this latitudinally extensive, albeit narrow, zone of parapatry. Admixture analysis and demographic models show limited, unidirectional gene flow from the highland to the lowland population in Peru, and hybridization in Bolivia. Finally, we find a pattern of isolation-by-distance around the northern Andes confirming the second prediction. These data provide the best evidence to date for the existence of a ring species in the Neotropics.}

\normaltalk{Genetic Architecture and Asymmetric Introgression of Sexually Selected Reproductive Barriers}{Georgy Semenov, Ethan Linck, Erik Enbody, Per Alström, Leif Andersson, Scott Taylor}{Genome-wide variation in introgression rates in hybrid zones provide a unique opportunity to examine the architecture of population differentiation. Arguably, the least understood introgression pattern occurs when a trait associated with reproductive barriers--thus implicated in lineage divergence--is geographically displaced from genome-wide population boundaries. We sequenced whole genomes of 82 individuals to study an example of such discordant introgression in the alba and personata subspecies of the white wagtail (Motacilla alba). Previous study of their hybrid zone in Siberia found a 300-kilometer displacement between a plumage signal mediating assortative mating, and all other phenotypic traits and genetic markers. We show that asymmetric introgression of plumage coloration in wagtails is a consequence of the simple genetic architecture of the trait, which has one major-effect gene, and the partially dominant expression of alba alleles in some phenotypes. Our results further suggest that anthropogenic habitat modification promoted secondary contact and hybridization of the wagtail subspecies, and appears to have altered the evolutionary trajectory of these lineages.}

\normaltalk{Flamingos, Lithium Mining, and the Limits of Sustainability}{Jorge S Guti\'{e}rrez, Johnnie N Moore, J P Donnelly, Cristina Dorador, Juan G Navedo, Nathan R Senner}{Saline lakes occur in some of the world's driest regions yet are home to globally important assemblages of an array of endemic taxa, ranging from microbes to wading birds. Recent evidence suggests that these lakes have a reduced ecological resiliency because the species occupying them are dependent upon lake salinity remaining within a narrow window. Lake salinity, in turn, is largely determined by variation in water availability, meaning that environmental perturbations that influence water levels can lead to rapid changes in the local biotic community. In this context, the increasing demand for Lithium batteries poses a potential risk to the biota of saline lakes: most Lithium is mined in and around saline lakes and, especially, from saline lakes in the `Lithium Triangle' of the Andes of Chile, Bolivia, and Argentina. To investigate whether Lithium mining has affected saline lake biota, we used a dynamic model of water availability in combination with a long-term dataset of biannual counts of three species of flamingos -- Andean (Phoenicoparrus andinus), James' (P. jamesi), and Chilean (Phoenicopterus chilensis) -- from five key saline lakes in the central Chilean Andes to directly link water, mining, and changes in flamingo abundance. Given the essential function of flamingos as top-consumers and the projected future increases in the global demand for Lithium batteries, including planned increases in Lithium mining in the Lithium Triangle, our results provide a much needed assessment of the effects of Lithium mining on saline lake ecosystems and hint at the impacts of even the most `sustainable' technologies.}

\normaltalk{Alternative Reproductive Tactics in a Cooperatively Breeding Bird with Low Kin Structure}{Shailee S Shah, Dustin R Rubenstein}{In cooperatively breeding species dispersal is generally sex-biased, resulting in breeding pairs with one natal and one immigrant parent. This preserves the social group's kin structure while guarding against inbreeding.  However, in the plural cooperative breeder, the superb starling (Lamprotornis superbus), both sexes can either remain in their natal group or disperse to a non-natal group. While females can only breed if they disperse, male breeders can either be natal or immigrant individuals.  If males can attain breeding positions in their natal group, why do some males disperse? We investigated this question using individual-level data from 9 social groups over 35 breeding seasons in Kenya. We found no significant difference between the mean and coefficient of variation of lifetime fitness of immigrant and natal males suggesting that dispersing vs. remaining are two equivalent, alternative reproductive tactics. Whether a male dispersed was determined by a combination of pre- and post-natal environmental factors affecting maternal and individual condition. The overall equivalence in fitness of the two tactics explains why dispersal of males persists despite the fact that it leads to low kin structure, as a result of immigrant males breeding, in superb starling social groups.}

\normaltalk{Bird Sleep and Torpor: A Continuous Physiological and Genetic Spectrum?}{Anusha Shankar, Isabelle N Cisneros, Sarah Thompson, Catherine H Graham, Donald R Powers}{Daily torpor--a controlled reduction in body temperature and metabolic rate--is a key energy saving strategy for small endotherms. In hummingbirds, past studies show the exclusive use of deep torpor, where body temperature drops by 23-34oC down to a species-specific minimum. As hummingbirds normally have some of the highest mass-specific vertebrate metabolic rates, using deep torpor to maximize nighttime energy savings seems an efficient energy management strategy. However, deep torpor presents a predation risk, likely prevents restorative sleep and inhibits immune function, and incurs steep rewarming costs. A shallower form of torpor (i.e., a decrease of 3-10oC) might sometimes be a better strategy to balance energy savings. Some mammal species can use both deep and shallow torpor, but this joint occurrence has not been reported in any bird species. Using infrared imagery in Arizona under natural temperature and light cycles, we found that all three hummingbird species we studied used shallow torpor, and it often occurred with deep torpor on the same night. Blue-throated hummingbirds (8.4g) spent an average of 25\% of the night in shallow torpor, while black-chinned hummingbirds (2.9g) spent only 5\% in shallow torpor. Rivoli's hummingbirds (7.6g) spent the most time in shallow torpor (35\%) and appeared the most flexible at managing nighttime metabolism. Since hummingbirds can use both shallow and deep torpor, it is possible that torpor is on a physiological continuum with sleep. I will next investigate the genomic basis of the difference between sleep, shallow, and deep torpor using transcriptomics.}

\normaltalk{Using Tracking Data to Explore the Range-Wide Migratory Connectivity of Painted Buntings}{Andrew J Sharp, Viviana Ruiz-Gutierrez, Andrea Contina, Eli Bridge, Jeff Kelly, Clark Rushing, Scott Sillett}{The strength of connectivity between breeding and wintering regions (migratory connectivity) can have consequences for genetic structure and resilience to environmental changes.  We used tracking data to show that the strength of migratory connectivity varies with spatial and temporal scale in a species with a complex population structure, the Painted Bunting (Passerina ciris). Painted Buntings occur in two geographically separated populations (Atlantic and Interior) with three identifiable genotypes. The two geographic populations differ in both migratory strategy and phenology. Our results show that painted buntings have very strong range-wide migratory connectivity, with Atlantic and interior populations remaining segregated throughout the annual cycle. Within each population, however, birds from different parts of the breeding range mix on the wintering ground, leading to weak connectivity within populations. In contrast to connectivity between breeding and wintering populations, we found evidence of a strong migratory divide in the molting areas used by birds in the interior population. This suggests that although low breeding-to-winter connectivity should promote resilience to habitat destruction during those periods of the annual cycle, preserving key molting areas could be critical to the  viability of the interior population. In the Atlantic population, which is declining faster and is much smaller than the interior population, low connectivity suggests that conservation efforts will need to take a holistic approach, rather than focusing on specific breeding or wintering populations.}

\normaltalk{Macroevolutionary Drivers of Parental Investment in Nest Construction}{Catherine Sheard, Sally E Street, Camille A Troisi, Andrew D Clark, Susan D Healy, Kevin N Laland}{Avian nests can provide crucial protection for eggs and chicks from predators and the environment. The construction of such nests, however, can represent a substantial energetic cost to the parents, suggesting evolutionary trade-offs both between generations and between the sexes. While biparental construction is common across the avian phylogeny, many species instead have female-only investment, with a smaller minority presenting male-only or group-wide building. The macroevolutionary dynamics of these behaviours remain largely unknown, though intraspecific studies of parental investment in nest building indicate several potential drivers of this variation. For example, nest building could primarily evolve together with all other aspects of parental care, subject to similar constraints as other types of sex-specific investment in offspring. Alternatively, differential investment in nests could serve as post-copulatory sexual signals or could be driven by the costs of maintaining the nest itself. Using a large global database of behaviours related to nest construction, parental care, and social interactions, I will present a comparative analysis of the broad-scale social, ecological, and environmental correlates of sex-specific investment in nest construction. I will demonstrate that `parental care' is not a single process, but that sexual selection can act independently on different aspects of care and has been shaped over time by variation in physical costs.}

\normaltalk{What Did MacArthur (1958) Miss? New Insights into Diffuse Resource Competition using Traditional Gut Content Analysis}{Thomas W Sherry, Cody Kent}{Going back to MacArthur's (1958) classic study, ecologists have framed interspecific competition in the context of resource partitioning, but does this metaphor hold up? MacArthur lacked quantitative data on the diets of the parulid warblers he studied, instead inferring the results of competition using foraging behavior as a surrogate. Recent diet data from a variety of taxa, including parulid warblers in multiple seasons, indicate high overlaps and only subtle differences among demonstrably competing species, calling into question the generality of ``resource partitioning" as widely understood by many ecologists. Our studies suggest instead that diffuse resource competition is far more widespread than generally recognized. Data presented indicate how coexisting competitors have evolved strategies to exploit particular substrates efficiently rather than exploiting different resources from each other. High diet overlaps among competitors persist because species retain the ability to exploit easy-to-obtain resources opportunistically exploited by coexisting competitors. These new insights into the evolutionary outcomes of competitor coexistence resolve important controversies, including conditions to establish interspecific competition, the ghost of competition past, and better metaphors than resource partitioning to understand the outcomes of resource competition.}

\normaltalk{Detecting Genomic Signatures of Natural Selection in a Strongly Sexually Selected Species}{Elsie H Shogren, Alice Boyle}{Sexual selection is a powerful evolutionary force in many avian lineages, notably Neotropical Manakins. Natural selection can constrain sexual selection, but it is not always clear how selection due to abiotic conditions limits sexual selection. Particularly in the tropics, the importance of precipitation as a source of natural selection is becoming apparent. For example, in very rainy locations and during especially wet years, estimates of apparent survival are lower for White-ruffed Manakins. The strength of sexual selection is determined by the proportion of males able to monopolize mating opportunities over their lifetimes (reproductive skew). Therefore, higher mortality will reduce skew by increasing male turnover, affording a greater proportion of males the opportunity to be reproductively successful at some point in their lives. Consequently, we predicted that the effective population size and genetic diversity of Z (sex) chromosomes would be higher relative to autosomes (A) in populations subject to higher rainfall and presumably, lower chances of survival and long-term social dominance.  We calculated genetic diversity for seven populations of White-ruffed Manakins breeding at sites receiving between 3500 - $>$8000 mm rain per year. Z:A ratios ranged from 0.51 -- 0.56 among populations, and this variation was consistent with our predictions; populations inhabiting sites with less rainfall seasonality and wetter dry seasons bore genetic signatures of lower reproductive skew. This study integrates behavioral, demographic, and genetic evidence to understand how climate variation influences the process of sexual selection. We also highlight the importance of rain in shaping the evolutionary trajectories of tropical taxa.}

\normaltalk{Spatial Distribution of the Boreal Owl and Northern Saw-Whet Owl in the Boreal Region of Alberta, Canada}{Zoltan Domahidi, Julia Shonfield, Scott E Nielsen, John R Spence, Erin M Bayne}{Understanding what factors influence the occurrence and distribution across the landscape is necessary for species conservation and management. We examined the role of climate, land cover, and human disturbance in shaping spatial distribution of the Boreal Owl (Aegolius funereus) and Northern Saw-whet Owl (Aegolius acadicus) in northern Alberta. Using autonomous recording units, we conducted passive acoustic surveys to detect both owls in Alberta's boreal forest region. Our analysis used climate variables for each survey location, and land cover and disturbance variables at each location at two spatial scales: a smaller local scale and at a larger landscape scale. A boosted regression tree analysis identified average minimum winter temperature as the most important predictor of Boreal Owl distribution. Boreal Owls were more likely to be present in cool environments with cold winters, and areas with a low percentage of grassland cover at the landscape scale. Cropland cover at the local scale was the most influential factor in the final distribution model for the Northern Saw-whet Owl, and they were more likely to be present in areas where cropland was interspersed with deciduous-dominated forests; these areas generally had cool summer temperatures and received less precipitation as snow. Linear features at the landscape scale negatively influenced distribution of Boreal Owls, but linear features at a local scale positively influenced Northern Saw-whet Owl distribution. Our study provides predictive distribution maps and new information about habitat use that can be applied in management and conservation of these two poorly studied species of owls.}

\normaltalk{Influenza a Virus Transmission, Infection, and Immunity in Reservoir and Spillover Hosts}{Susan A Shriner, Mikaela K Samsel, Jeremy W Ellis}{Influenza A viruses (IAVs) are endemic in wild waterfowl, but commonly spillover into poultry, causing economic harm. Because wild waterfowl and commercial poultry do not frequently come into direct contact, IAV spillover into poultry is likely mediated by bridge hosts or contaminated water. In this study, we assessed IAV transmission from contaminated water and compared infection and immune responses for mallards and chickens as representative reservoir and spillover species. We experimentally exposed groups of chickens and mallards to four IAV strains by providing contaminated drinking water. We then collected swabs and blood across the infection period to characterize viral excretion and immune responses. All birds in three of the virus groups became infected after exposure to contaminated water, but none in the fourth group became infected. Generally, mallards shed viral RNA sooner and at higher concentrations compared to chickens. All infected birds mounted an immune response by day 14 with mallards showing detectable responses sooner than chickens (mean difference = 5.7 days). For both species, antibodies were evident earlier for an H6N2 virus compared to the other subtypes (mean 11.0 days for chickens, 5.5 days for mallards). Overall, chickens were susceptible to infection with wild bird viruses and contaminated water was a viable route of transmission. Infection characteristics such as onset of infection post exposure, peak shedding concentration, peak shedding day, and infectious period all varied between species and subtypes. Linking these data to mechanistic models is the next step in predicting epidemiologic dynamics essential to managing epizootics.}

\normaltalk{Assessing the Effects of Salt Marsh Restoration on Breeding Birds in the Northeast, USA.}{Greg Shriver, Elizabeth Tymkiw, Chris Elphick, Brian Olsen, Adrienne Kovach, Jonathan Cohen, Katharine Ruskin, Mackenzie Roeder}{The Saltmarsh Habitat \& Avian Research Program is monitoring breeding bird responses to tidal marsh restoration actions associated with Hurricane Sandy.   Our focal species include Clapper Rail (Rallus crepitans), Willet (Tringa semipalmata), Seaside Sparrow (Ammospiza maritimus), Saltmarsh Sparrow (Ammospiza caudacutus), and Nelson's Sparrow (Ammospiza nelsoni).  We have evaluated the breeding bird response to the immediate effects of six types of restoration actions at 28 projects that include 58 sites from Virginia to Maine.  Restoration types include thin-layer deposition (27 sites), living shorelines (17 sites), hydrologic improvements (35 sites), vegetation plantings (9 sites), invasive species control (15 sites), and pole removal (3 sites).  We sampled the breeding bird community using point counts at each restoration sites (n = 389 points) and associated control areas (n = 377 points).  We estimated detection-adjusted abundance of tidal marsh obligate birds and visually estimated the vegetation cover by community type and dominant species.  We also developed a ``rapid" demographic protocol as an index of sparrow productivity that includes standardized, mist-netting events, line transects, and point counts conducted at randomly selected sites (n = 85) during two independent visits per season; we evaluated the accuracy of the rapid protocol using intensively-studied demographic sites (n = 11).   Initial results indicate that bird abundance declined immediately after all restoration types except pole removal.  Rapid demography metrics of abundance and productivity were positively related to abundance estimates at sites with intensive study, indicating that the rapid demographic protocol has potential to provide a useful index of reproduction.}

\normaltalk{The Evolution of Plumage Coloration and Its Underlying Mechanisms in Male and Female Tanagers}{Allison J Shultz, Jacqueline E Dall, Dakota E McCoy, Kevin J Burns}{Plumage color is an integral trait of avian biology, with functions including social signaling, predator avoidance, thermoregulation, and flight performance. Because of its diverse uses, the evolution of plumage color in avian radiations is shaped by a complex mosaic of natural and sexual selective forces. These forces can vary based on characteristics that include life history, geography, and light environment. While these extrinsic factors are important in shaping plumage coloration, there are also intrinsic constraints that may limit the expression or use of certain colors. Here, we investigate how underlying mechanisms affect color evolution in the largest family of birds, the tanagers. First, we compare how broad coloration mechanisms evolve across the phylogeny in males and females and link the evolution of coloration mechanisms to measurements of color (e.g. color complexity). For example, we show that the degree of sexual dichromatism can vary widely based on the underlying coloration mechanisms in males and females. Second, we investigate how feather microstructure (barb and barbule morphology) varies both across species and sexes in a subfamily of tanagers, the Tachyphoninae. Microstructure can impact plumage signals by increasing chromatic signals or decreasing reflectance (e.g. superblack plumage), but has rarely been studied in an evolutionary context. Results from one genus, Ramphocelus, show that males have evolved elaborate microstructures that increase contrast for signaling patches, but female microstructure remains fairly static. Expanding these results will allow us to examine how this often-overlooked trait contributes to the evolution of plumage signals.}

\normaltalk{Natural History Collections are Critical and Underutilized Resources for Contemporary and Future Studies of Urban Evolution}{Allison J Shultz, Benjamin J Adams, Kayce C Bell, William B Ludt, Gregory B Pauly, Jann E Vendetti}{Urban environments are among the fastest changing habitats on the planet, which has evolutionary implications for the organisms inhabiting them. Herein, we contend that natural history collections are critical resources for urban evolution studies; the specimens housed in these collections provide great potential for diverse types of urban evolution research, and strategic deposition of specimens and other materials from contemporary studies will determine the resources and research questions available to future urban evolutionary biologists. Natural history collections are windows into the past and, thus, provide a crucial historical timescale for urban evolution research. Here we highlight some of the current and potential uses for museum specimens for studying urban evolution research, analyze how urban vs. rural and native vs. nonnative species are being deposited in museum collections, and make recommendations to improve museum collections for the study of urban evolution. Our analyses of recent urban evolution studies demonstrate that museum specimens are currently underutilized and that while nearly all studies are generating resources that could be deposited in natural history collections, only 12\% of studies actually do so. Following best practices for the collection, preservation, and deposition of urban-living species as specimens in natural history collections will maximize future research potential and allow urban evolution studies to be replicated and re-visited. Museum collections are critical for understanding the biological consequences of urbanization and sample deposition must improve to ensure resources are available for future urban evolutionary biologists.}

\normaltalk{Precipitation-Induced Changes in the Investment on Eggs by Kingbirds}{Marcelo AdA Silva, Neander M Heming, Miguel  Marini}{Natural fluctuations in temperature and precipitation can influence directly or indirectly the reproductive effort of birds. However, it is still no clear how wheater conditions are used as cues by females to define their strategy on how to invest in the quality of their offspring. To access the effect of weather conditions on reproductive effort (egg volume), we used 428 clutches of nine Neotropical and Nearctic kingbird species (Tyrannus spp.) deposited at museums. We used linear mixed models (LMM), to evaluate the relationships between egg volume (mm3) and annual anomalies of minimum and maximum temperature (AnTmin, AnTmax), the minimum and maximum precipitation (AnPmin, AnPmax), seasonality of precipitation (AnPsea) and isothermality (AnTiso). Precipitation was the sole climatic variable with an effect on egg volume. The AnPmin has a robust and significant effect on the egg volume (F1,377=7.48, p$<$0.05; ß=0.08±0.02), showing a decrease in the investment on eggs in years with drier winters. There was also a strong effect of AnPmax (F1,400=6.64, p$<$0.05; ß=0.06±0.03), pointing to an increase of investment in eggs in years with wetter summers/springs. The fluctuation of precipitation regimes can explain a significant component of ecosystem productivity and possibly act changing the phenology of food supplies, altering a major source of resources to egg production by Tyrannus species. In view of the rapid and evident climatic changes, it is necessary to understand the physiological limits involved in investing in offspring, their relationship with climatic parameters and their consequences on population dynamics.}

\normaltalk{Latitudinal Variation in Song Complexity in the Blue-Black Grassquit}{Edvaldo F Silva-Jr, Pedro Diniz, Regina H Macedo}{Ecological and sexual selection hypotheses predict that bird song should be more complex at higher latitudes due to greater ecological opportunities in `sound space' and a shorter avian breeding season. Although bird song is a well-studied topic, there are few studies on how song complexity varies latitudinally across a broad geographical scale. The Blue-black Grassquit is a small, sexually dimorphic passerine with a large geographical range, occurring from Mexico to Argentina. We investigated the relationship between latitude and the Blue-black Grassquit´s song complexity, defined here as a function of frequency metrics and song duration. We expected song complexity to increase with latitude, as predicted by the ecological and sexual selection theoretical framework. To cover the species´ distribution, we used recordings from the Xeno-Canto database. Statistical and acoustic analyses were done in R and Raven software, and a principal component analysis was used to summarize song metrics. We found no relationships between latitude and song complexity metrics, except for energy frequency bandwidth (90\%), which decreases significantly in higher latitudes in the northern hemisphere. Therefore, individuals in higher northern latitudes concentrate their song energy within a shorter frequency interval, resulting in a simplification of their acoustic signal. This result is contrary to our expectation of more complex songs in higher latitudes. Because other factors can drive song complexity in birds (e.g. migratory behavior), further studies of the Blue-black Grassquit song are necessary to assess the mechanisms responsible for the observed pattern and to test ecological and sexual selection hypotheses.}

\normaltalk{Evaluation  of Northern Bobwhite Population Growth and Persistence on Remnant Prairies and Mixed Agricultural Landscapes}{Emily A Sinnott, Frank R Thompson, Mitch D Weegman, Alisha R Mosloff, Thomas R Thompson}{Northern bobwhite are a shrub-obligate species in steep decline that occupy grasslands, scrublands, savannas, agricultural fields, and other open landscapes.  Bobwhite are a year-round resident and complete their full life cycle in a single landscape.   Population recovery on altered landscapes requires a better understanding of underlying, habitat-specific demographic processes and sensitivities across the full annual cycle.  Our objective was to use a seasonal two-stage, two-sex Bayesian integrated population model to estimate population growth and project future viability of northern bobwhite on three extensive native grassland sites and two traditionally-managed Quail Emphasis Areas  in the Osage Plains of southwest Missouri.  We used year-round survival and nesting data for males and females, adults and juveniles and annual spring whistle counts on five conservation areas from 2016 to 2018.  We estimated sex-specific productivity rates because both females and males contribute to fecundity as nest-incubating and brood-attending adults.  We also estimated adult and juvenile summer and winter survival rates. We contrasted population growth rates and seasonal demographic sensitivities. Survival was estimated from a known-fate logistic exposure model, while abundance and productivity were estimated from state space models.  Preliminary seasonal survival results suggest there are age-specific habitat sensitivities that vary between winter and summer.  We demonstrate how a holistic modeling technique that allows simultaneous estimation of seasonal survival, fecundity, and population size can help us understand the complexities and underlying drivers of population change and better inform conservation.}

\normaltalk{The Effect of Social Interactions and Host Genotype on Preen Gland Microbiome and Preen Oil Composition in Dark-Eyed Juncos}{Joel W Slade, Johnathan M Greenberg, Helena A Soini, Cody R Lambert, Milos V Novotny, Kevin R Theis, Danielle J Whittaker}{In Dark-Eyed Juncos (Junco hyemalis), volatile compounds given off by preen oil may be an important chemical signal involved in reproductive behavior. While studies have found a connection between host microbiome and semiochemicals, there is scant evidence on the relative influence of social interactions and individual genotype on the abundance and types of semiochemical-producing bacteria within a host. Here we characterized preen gland microbiome, preen oil volatile profiles, and MHC class II genotypes in two Dark-Eyed Junco subspecies (Slate-Colored Junco (Junco hyemalis hyemalis) and the Carolina Junco (Junco h. carolinensis)). Over a period of 24 days, birds were placed in same sex and same subspecies flocks, then in mixed-sex pairs (same or different subspecies), and finally back into their original flocks. Every four days we sampled the microbiome and preen oil of each bird. We predicted that the preen gland microbial community and preen oil composition of group members would converge towards the end of each grouping stage. We predicted that individual variation in microbiome and preen oil may be influenced by MHC class II genotype. We found that preen gland bacterial community richness and diversity decreased over time, and that individual preen oil odor was not stable throughout the experiment. Also, we found that MHC similarity did not predict similarity in preen oil composition or bacterial community structure. Time spent in proximity did correlate with similarity between individual volatile and bacteria profiles. Ultimately, this research may lead to new perspectives on environmental effects on host phenotypes via symbiont colonization.}

\normaltalk{Aggregate Citizen Science Field Notes Reveal Seasonal and Directional Dispersal Flights in Eurasian Collared-Doves}{David L Slager}{Range expansions require the dispersal of individual organisms, but dispersal behavior is notoriously difficult to study. Eurasian Collared-Doves have colonized both Europe and North America this century, with both initial invasions proceeding northwestward via ``jump" dispersal. The European population has reached carrying capacity, but in the Americas, where exponential population growth continues, little is known about dispersal behavior. I queried citizen science field notes to investigate Eurasian Collared-Dove dispersal behavior in North America along the Pacific Ocean, a natural barrier to northwestward landbird dispersal. I examined coastal and pelagic records of Eurasian Collared-Doves to determine the seasonality and directionality of dispersal flights and to assess changes in the frequency of dispersal-related behavior across years. Most reports of Eurasian Collared-Doves flying in flocks along the coast and most pelagic sightings were in March-May, mirroring the spring natal dispersal documented during the European invasion. Almost all spring coastal flocks were flying north, consistent with the overall northwestward direction of the initial North American invasion and northward deflection of northwest-bound birds upon encountering the Pacific Ocean. Spring flying flock observations showed no significant change in frequency from 2010-2018 after controlling for overall increases in spring collared-dove reports. This refined understanding of Eurasian Collared-Dove dispersal behavior informs full life cycle management of this rapidly increasing invader in the Americas, raises new questions about evolutionary mechanisms driving this spectacular invasion, and illustrates the potential power of citizen science field notes for documenting dispersal behavior, a longstanding challenge in ecology.}

\normaltalk{Genomic Underpinnings of Honest Signaling in Sexually Selected Plumage Traits}{Nicholas D Sly, Amberleigh E Henschen, Corey R Freeman-Gallant, Linda A Whittingham, Peter O Dunn}{Honest signals require genetic mechanisms that link ornament production and individual quality, but these connections remain poorly understood. Here, we take advantage of a unique opportunity to investigate the genetic basis for honest plumage signals in Common Yellowthroats. Yellowthroat females choose mates based on ornamental traits associated with indices of fitness, but the trait selected differs by population. Wisconsin females prefer males with larger black masks, while New York females prefer larger yellow bibs. Male mask size in WI and bib size in NY are positive correlated with traits including immune response, body condition, and survival, making these two plumage traits similar honest signals of male quality despite being produced by different pigmentation pathways. We combined whole-genome sequencing and transcriptomics to identify gene pathways associated with mask and bib size in each population. We identified genes expressed in mask and bib feathers that are divergent between high and low quality masks in WI and bibs in NY, but not vice versa, as candidate genes for generating honest signaling in those traits. Finally, we discuss the divergence in how these gene pathways may generate honest signals in carotenoid versus melanin generated traits.}

\normaltalk{Evolving the BBS Field Protocol and Estimates for Tomorrow's Conservation Needs}{Adam C Smith, Colleen Handel, Marcel A Gahbauer, Jim Nichols, Viviana Ruiz Gutierrez, Wayne E Thogmartin}{The North American Breeding Bird Survey (BBS) is designed to measure trends in annual breeding populations of landbirds, using a strictly defined field protocol to help control variation in effort and detectability. BBS data are also used to map species distributions, model habitat relationships, evaluate the efficacy of conservation actions, and estimate population size. With technological advances and the increasing availability of other sources of information on bird abundance and distribution, we envision expanding the BBS field-protocol and integrating other data sources into the annual analyses. Thereby providing improved trend estimation better informing the other applications of BBS data.  We will highlight the key opportunities for the BBS to fill gaps in its coverage, and the critical aspects of our plan that promote reliable estimates of population change. The field protocol will evolve to include BBS-compatible counts in off-road areas, expansions of the survey to the North and South where opportunities for roadside surveys are limited, and surveys using audio recording units.  The efforts of thousands of BBS volunteers and long-term investments by government agencies have built an unmatched source of information tracking population changes in the natural world. Our vision for the future of the BBS is to meet the changing information needs of conservation and management by capitalizing on modern techniques and partnerships to build on the integrity of the existing 50-year time series.}

\normaltalk{Livestock Intensification and Landscape Simplification Increase Food Safety Risks Imposed by Wild Birds on Produce Farms}{Olivia M Smith, Amanda Edworthy, Joseph M Taylor, Matthew S Jones, Aaron Tormanen, Christina M Kennedy, Zhen Fu, Christopher E Latimer, Kevin A Cornell, Tobin Northfield, William E Snyder, Jeb P Owen}{Recent declines in wild bird  populations, largely a result of agricultural intensification, heighten the need to balance conservation with food production. However, farmers   often consider birds a threat, in part because birds harbor foodborne pathogens. Yet, there remains uncertainty about the links between landscape context, farming practices, and crop contamination by birds. We addressed this by collecting 2024 fecal samples from captured birds alongside 1215 from production areas across 50 farms spanning the US West Coast. We then estimated the prevalence of 3 foodborne pathogens across landscape and livestock intensification gradients. Finally, we quantified the number of brassica plants with bird feces. We found a pathogen prevalence of 11.3\%, 0.03\%, and 0.09\% for Campylobacter spp., Shiga-Toxin Producing E. coli, and Salmonella spp., respectively across all samples tested.    Campylobacter spp. prevalence in feces from production areas was positively associated with increasing mammalian livestock densities in the landscape and lower amounts of natural habitat in the landscape. Nonnative   birds, found at high densities in feedlots, were 3.2 times more likely to have Campylobacter spp. than native species. Despite the strong impact on pathogen prevalence, landscape context did not significantly increase the encounter rates of   bird feces on plants (3.3\% of plants examined had feces), although on-farm mammalian livestock density slightly did.  Altogether,  our results suggest that preservation of natural habitats within agricultural landscapes could benefit both wildlife conservation and food safety, contrary to current standards for ``best practices" for food safety.}

\normaltalk{Genetics of Invasive Parrots: Insights into Origins, Invasion Patterns and Adaptation}{Grace Smith-Vidaurre, Michael A Russello, Timothy F Wright}{Biological invasions are growing in concert with human connectivity and habitat alteration. Ecological damage caused by invasive species can lead to great economic costs, such that invasive species are often targeted for intensive management strategies. Invaders also offer compelling opportunities to ask ecological and evolutionary questions, as invasive populations arising from repeated, independent introductions can be considered replicates of natural evolutionary experiments. Such replicates can be used to identify source populations in the native range, and to infer the relative contributions of genetic, epigenetic and behavioral processes to invasive population adaptation and spread. Parrots have become invaders of great management and research interest, as populations have been established across the world in association with the pet trade. We summarize the literature on biological invasions to show how genetics tools can be used to evaluate origins, establishment and spread of invasive parrots, then discuss case studies with monk parakeets (Myiopsitta monachus) and ring-necked parakeets (Psittacula krameri), two invasive species for which genetic approaches have been employed. Genetics studies with monk and ring-necked parakeets have evaluated putative source populations in these species' native ranges, and documented genetic bottlenecks post-introduction that have likely influenced the genetic variation available for adaptation. Despite this work, we still know relatively little about many aspects of how invasive parrot populations adapt and spread, including their evolutionary responses to new selection pressures, the effects of sex-ratio biases on population growth, and whether gene flow within the invasive range ameliorates predicted genetic bottlenecks.}

\normaltalk{Identifying Changes in Seaside Sparrow Diet Following Large-Scale Disturbances using Metabarcoding}{Allison M Snider, Andrea Bonisoli Alquati, Stefan Woltmann, Philip C Stouffer, Sabrina S Taylor}{Seaside Sparrows (Ammospiza maritima) in the northern Gulf of Mexico rely on coastal saltmarshes that experience a variety of large-scale disturbances. These disturbances, including oil spills and hurricanes, can have immediate consequences for both sparrows and their invertebrate prey, but these events may also have persistent, indirect effects that last beyond the initial perturbation. As upper-level predators in the saltmarsh, sparrow response to disturbances is a unique indicator of how changes at the base of the food web can extend to higher trophic levels. To identify changes in sparrow diet composition following potential disturbances in the food web, we used DNA metabarcoding of two markers (COI and 18s) to identify prey species consumed by sparrows from 2011-2017. This encompasses the aftermath of two major disturbances, the Deepwater Horizon (DWH) oil spill (2010) and Hurricane Isaac (2012), allowing us to identify changes in sparrow feeding ecology after both events. Birds were captured on oiled, unoiled, and distant reference sites, but diet did not differ significantly between oiled and unoiled sites. This suggests sparrow diet did not reflect documented changes to invertebrate communities on oiled sites following the DWH spill. Although site-level oiling immediately following DWH was not a significant predictor of diet, we found extensive temporal variation. Beta diversity measures showed differences in diet composition each year before 2015, with diets in 2015-2017 appearing more similar. This may indicate that resources stabilized, possibly reaching a new baseline after Hurricane Isaac.}

\normaltalk{Spring Relapse of Avian Haemosporidian Infections in a Neotropical Migrant}{Dr Leticia Soares, Dr Christopher G Guglielmo, Dr Elizabeth MacDougall-Shackleton}{Avian haemospordian parasites infect not only erythrocytes but also the cells of other tissues, hence absence of parasites in the blood does not necessarily signify lack of infection. Increased prevalence during the birds' breeding season has been linked to the re-appearance of parasites in the blood during spring (i.e. relapse). To quantify infection relapse frequency, and whether this phenomenon is endogenous or photoperiod-dependent, we held 50 yellow-rumped warblers (Setophaga coronata) overwinter in vector-free rooms. We captured birds during fall migration (October 2018), collected blood samples to assess parasite presence using PCR, and assigned them to two treatment groups: early photostimulation (EP, N = 21, photostimulated early March) and late photostimulation (LP, N = 29, photostimulated late April). We collected a second blood sample from EP birds in late April (i.e., on spring photoperiod). Parahaemoproteus prevalence in among EP birds increased significantly, from 43\% in October to 81\% after photostimulation, indicating relapse, but Plasmodium and Leucocytozoon prevalence remained stable. We collected blood samples from LP birds in late April (winter photoperiod) and again in May (spring photoperiod). Among LP birds, prevalence for all three genera were similar between fall, winter and spring. However, the cumulative prevalence increased from October to April, reflecting new individuals testing positive for the first time for Parahaemoproteus (8), Plasmodium (7) and Leucocytozoon (3). We provide evidence that haemosporidian infections relapse during calendar spring, regardless of photoperiod cues. These findings suggest that relapse can be triggered by photoperiod-independent cues, although different parasites appear to have different clocks.}

\normaltalk{Bird Diversity Loss at Very Large Spatial Scales in the Colombian Choco-Andes-Amazon}{Jacob B Socolar, Simon Mills, James Gilroy, Torbjørn Haugaasen, David Wilcove, David Edwards}{Land-use change often homogenizes species assemblages, which can lead to regional-scale biodiversity impacts that are dramatically underestimated by individual local-scale field studies. However, proper quantification of regional-scale impacts is severely hindered by the extreme difficulty of obtaining comprehensive regional-scale biodiversity surveys, particularly in the mega-diverse tropics, where not only do regional species pools contain numerous far-flung and range-restricted taxa, but speciose local assemblages require massive survey effort for adequate characterization. Because sufficiently comprehensive survey data are unlikely to become available in any particular area, there is a pressing need to understand how homogenization drives regional-scale biodiversity loss in general. We address this issue using a large bird-occurrence dataset at paired sites in forest and regionally dominant land uses (smallholder agriculture and cattle ranching) in Amazonian Peru and across the Colombian Andes and lowlands. A uniquely comprehensive field effort yielded datasets of $>$450 species Amazonian Peru and $>$950 species across Colombia. We show that land-use strongly homogenizes bird communities across edaphic and hydrological gradients in lowland Peru, across elevation and precipitation gradients in Colombia, and across biogeographic barriers in both systems (Amazonian rivers in Peru; Andean mountains and valleys in Colombia). These effects are underpinned by differential responses of specialist and range-restriced taxa, but not all restricted groups respond in consistent ways. We additionally discuss challenges and approaches in incorporating complex biogeography into multi-species occupancy models for birds.}

\normaltalk{A Data-Driven Migratory Connectivity Model for Predicting Bird Migration Across the Americas}{Marius Somveille, Kristen Ruegg}{Billions of birds migrate every year across the Americas, exhibiting a remarkable diversity of routes and seasonal destinations both between and within species. Recent advances in data and models provide new opportunities to understand the causes of bird migration and predict its response to global change. This talk will present an analysis of species-specific regional migratory connectivity patterns in the Americas. We developed a process-based modeling framework based on the mechanism of energy efficiency, which leverages the power of genetic markers data, i.e. genoscapes, and eBird citizen science data. This model is able to explain the patterns of regional migratory connectivity for several species and is used to make predictions under global change. Using climate reconstructions, we investigate how past climate change since the last ice age has shaped current migration patterns. Understanding how migration has evolved in response to climate in the past will help us better predict how it will change in the future. Predicting the seasonal movements of migratory species is indeed critical for designing effective protection measures for birds on the move. Where should we target conservation efforts? Where will migratory birds go in the future and which regions will they cross? Our approach will inform the prioritization of conservation efforts at continental scale, exploring how various threats are likely to impact migratory birds and identifying which species is particularly at risk. This talk will also present opportunities for extending the predictive framework to integrate other sources of migration data, such as tracking and radar data.}

\normaltalk{Operation Feeder Rescue: a Multi-Island Bird Feeding Effort by Islanders to Provision Birds in the Aftermath of Hurricanes Irma and Maria.}{Lisa Sorenson}{Hurricanes Irma and Maria walloped numerous Caribbean countries in September 2017 destroying homes, toppling trees and darkening cities. As reports came in of damaged habitats, including leaves, flowers and fruits stripped off of every tree, and wetlands and other habitats filled with debris, it became clear that birds and our in-country partners needed help. Birds that had survived the storm were starving, especially nectar-feeding birds such as hummingbirds and Bananaquits. Seed eating birds like pigeons, doves, grassquits, parrots, and bullfinches were also suffering. BirdsCaribbean quickly launched a crowdfunding campaign to raise funds for hurricane recovery for birds and our partners. We raised over \$100,00 from more than 450 generous people. With this funding we shipped 4,201 nectar feeders, 1,850 bottles of nectar and 5 tons of bird seed to 18 islands/ countries impacted by one or both hurricanes. We estimate that over 60 species of birds were helped by the temporary provisioning and many likely survived a serious food shortage. Feeding the birds also brought psychological benefits to people dealing with loss and devastation--numerous people expressed how helping the birds lifted their spirits and brought them closer to nature. We also assisted with restoration efforts of habitats and protected areas as well as post-hurricane surveys assessment of birds and habitats through 15 small grants. Our experiences underscore the importance of hurricane preparedness and also the value of provisioning and bird monitoring programs to assess impacts and recovery of birds following severe storms.}

\normaltalk{Does Predator Presence Affect Female Vocalization Rates During Nest Building?}{Nadya M Sotnychuk, Kristin K White, Peter A Biro, Christa Beckmann}{Birds are iconic for their acoustic communication, such as songs and calls. Vocalizing near their nest can have benefits, such as synchronizing parental care. Despite such benefits, research has shown these vocalizations may also reveal the nest to predators. Therefore, birds are expected to be sensitive to risk and reduce vocalization rates when predators are near. Interestingly, the Grey Fantail (Rhipidura albiscapa), frequently vocalizes on and near their nests during nest building. Using a model presentation experiment, we tested if fantails would reduce vocalization rates at the nest during building in relation to predator presence. We recorded vocalization rates on and within 2.5 meters of the nest during each build trip, next we presented models (predator/control) nearby, and the following day observed the pair to determine if they altered vocalization rates based on recent experience. There was no effect of model type on vocalization rates (p=0.16). However, there was a decrease in vocalization rates across trial days (p$<$0.01) and birds vocalized more frequently on the nest than when they were near the nest (p$<$0.01). The reduction in vocalization rates across days, but no model effect, suggests a generalized antipredator response serving to reduce the detection of the nest.}

\normaltalk{Does Investment in Reproduction Decrease with Increasing Migration Costs in Neotropical Tyrant Flycatchers?}{Nadinni OdM Sousa, Caio VSL Oliveira, Nuno MdO Lima, Neander M Heming, Miguel  Marini}{Migrant life history strategies involve several energetic costs and risks to the survival of organisms, but also benefits from exploiting the best resources of two worlds. The main advantage of this strategy is to optimize energy acquisition and, therefore, increase fitness. We investigated the balance between the benefits and costs of migration among tyrant flycatchers breeding in the Neotropical region, through analyzing the trade-off between investment in reproduction (egg volume) and migration cost (migration type and travel distance). Egg volume was estimated from 895 clutches of 95 taxa (full migrants: n = 22; partial migrants: n = 23; residents: n = 50) through digital photographs taken in scientific egg collections using the EggTools plugin in ImageJ. We used linear mixed models controlling for the effects of latitude of breeding location and species' body size. We found that migration type predicted egg volume (F2,429 = 3.79, p = 0.023), where full migrants had larger egg volume than partial migrants (ß = -0.05 ± 0.02, p = 0.010) and residents (ß = -0.05 ± 0.02, p = 0.021). Travel distance did not predict egg volume and was dropped from the model (F1,601 = 0.23, p = 0.63). Despite high energy expenditure faced by migrants during their journey, our results suggest that species also invest more in producing larger nestlings, possibly to increase their chances of surviving and migrating to wintering grounds. This study shows the potential of museum egg collections to test ecological hypotheses by exploring large-scale changes in breeding parameters.}

\normaltalk{Using Citizen Science to Track Hummingbird Migration for Conservation}{Susa Bonfield, Nancy Sheehan, Juan Loera, Daniela Souza}{Hummingbirds occur only in the Americas, where they are the second most diverse family of birds. They are specialized nectar-feeders and serve as pollinators for a wide array of native plants. Six North American hummingbird have been identified as species with conservation challenges, and one of these species, the Rufous Hummingbird (Selasphorus rufus) is on the National Audubon Society's Watchlist as a Species in Decline (Howell 2003). Research also shows that climate trends are shrinking the interval between plant flowering and the arrival of at least one species, Broad-tail Hummingbird (Selasphorous platycercus) on more northern parts of their range. Yet a void exists in the ability of scientists to gather data about hummingbirds. Unlike other birds, hummingbirds are not easily monitored using typical survey methods. They don't sing and are missed in aural (by ear) surveys. Capturing and banding them requires a special permit, which relatively few biologists have. Students, youth, and families may serve as valuable contributors by gathering information about nectar-producing plants and hummingbirds at natural areas and at nectar-feeders (hummingbird feeders). Already, citizen scientists have provided information about floral nectar preferences in post-burn chaparral habitat in California, and through a trinational hummingbird conservation collaboration, the Western Hummingbird Partnership, we seek to expand citizen science activities in Canada, the United States, and Mexico to gather more information about changes in migration timing, floral nectar preferences, and threats to hummingbirds. This presentation shares the results of citizen scientist activities and how the data may be used to inform conservation.}

\normaltalk{Influence of Habitat Structure, Landscape Composition, and Food Availability on the Distribution and Abundance of Eastern Whip-Poor-Wills}{Ian F Souza-Cole, Michael P Ward, Thomas J Benson, Jeffrey T Foster, Rebecca L Mau, Chris Tonra, Aaron Skinner}{The eastern whip-poor-will (Antrostomus vociferus) is an aerial insectivore that has experienced drastic declines across much of its range in recent decades. Reductions of insect populations and habitat degradation are hypothesized to be contributing to their decline. To help understand proximate causes of this decline, we studied habitat use and food availability of whip-poor-wills in Illinois. We performed nocturnal surveys to assess occupancy and abundance of whip-poor-wills across a broad range of forested sites in Illinois and examined habitat use in radio-tracked birds. Along survey routes, we sampled for nocturnal insects with UV-light traps to evaluate whether moth and beetle abundance or diversity predicts whip-poor-will occupancy, and we collected additional data related to habitat characteristics, light pollution, and pesticide load. Further, we examined fecal samples from 115 captured whip-poor-wills using DNA metabarcoding to determine the composition of their diet. Whip-poor-will occupancy was 37\%, and landscape-level forest cover was a major determinant of whether sites were occupied .  Based on our radio-telemetry work, whip-poor-wills tended to spend active periods in areas with more open canopies and less dense understories compared to roosting sites. From light trap data, moth abundance varied widely among sites and was related to abundance of whip-poor-wills . In diet analyses, whip-poor-wills were mainly eating moths (86\% of samples), cockroaches (9\%), and beetles (6\%). Our results suggest that food availability and habitat structure may be  determining the distribution and abundance of whip-poor-wills and are likely important factors for crafting management plans for this species.}

\normaltalk{Investigating Dietary Niche Partitioning in Morphologically Similar Hummingbirds using Metabarcoding}{Austin R Spence, Erin E Wilson Rankin, Morgan W Tingley}{Hummingbirds are the second most diverse family of birds in the world, and their unique relationship with plants has long interested evolutionary biologists. Many tropical hummingbirds show morphological adaptations that allow them to partition their diet to specialize on flowers with corresponding ornithophilous inflorescence, limiting competition for nectar resources. Bee hummingbirds, the most recent clade to emerge from Trochilidae, show much less morphological variation in bill shape, yet multiple species regularly exist in sympatry across North America. To understand if and how North American hummingbirds partition their diets, we collected fecal samples from Anna's, black-chinned, and calliope hummingbirds in California. We used DNA metabarcoding techniques to sample for floral and arthropod diet items to examine resource use across an elevational gradient from sea level to 2500 meters. We found hummingbirds consume a greater diversity of invertebrates than plants, with spiders accounting for the majority of invertebrate diversity. Anna's hummingbirds have a larger diet breadth than black-chinned hummingbirds and calliope hummingbirds as a whole, but all three hummingbirds have similar diets when accounting for elevation. Bee hummingbirds are relatively new compared to tropical clades, and they show more generalization than their tropical counterparts. Historically, understanding the breadth and range of bird diets has been invasive or time consuming. Using modern molecular methods allows us to investigate classic ecological questions in a new way, as well as delve further into natural history of organisms and how they respond to anthropogenic change.}

\normaltalk{Assessing Spatiotemporal Use of Nonbreeding Diving Bird Species in U.S. Atlantic Waters: Potential Exposure to Offshore Wind Development}{Caleb S Spiegel, Iain J Stenhouse, Alicia M Berlin, Andrew T Gilbert, Carrie E Gray, William A Montevecchi, Lucas Savoy, M W Goodale}{Although few offshore turbines have been deployed in United States (U.S.) waters to date, many large scale developments are being planned that have the potential to affect marine birds. Off the Atlantic coast, where the first offshore wind facility began operating in 2016, developers hold several additional leases and are actively planning construction. To determine spatial and temporal use and movement patterns, we tracked diving birds identified as vulnerable to offshore wind development -- Surf Scoter (Melanitta perspicillata), Red-throated Loon (Gavia stellata), and Northern Gannet (Morus bassanus). We captured 239 wintering adults at sea from New Jersey to North Carolina, between 2012 and 2015 and fitted individuals with satellite transmitters. Tracks were analyzed using dynamic Brownian bridge movement models to develop spatial utilization distributions for each species. All species exhibited largely near-shore, coastal, or inshore distributions. Habitat use was concentrated in or around large bays, especially bay mouths. Northern Gannets covered a much larger area than the other focal species, ranging farther offshore and over greater latitudinal extents. Core use areas for each species was associated with shallower, colder, and more productive waters, compared to other locations within winter home ranges, supporting the idea that distributions were linked to forage resources. Results increase our understanding of how diving birds use offshore areas of the Atlantic coast and beyond. Based on these results, we believe our approach, in combination with other research (offshore surveys, other tracking work), can inform permitting, risk assessment, and pre- and post-construction impact assessments of offshore energy infrastructure.}

\normaltalk{Conservation Strategies at the Human/Bird Interface in the Caribbean}{Tabitha Stadler, Adam C Brown, Natalia Collier, Anderson Jean, Juliana Coffey}{Addressing the root causes of threats to bird populations often requires modifying human activities, which typically requires long-term behavior modification strategies. We describe examples of successful strategies to reduce human/bird conflict from within the Antilles. The first is from Haiti, where community support for the Black-capped Petrel is fostered through economic development, pride campaigns, and sustainable agricultural education activities which benefit farming families while simultaneously protecting nearby forest nesting habitat for the petrel. The second is from the Grenadines, where poaching of seabirds is reduced through pride and education activities, training for enforcement agencies and policy makers, and recruitment of community members into the Grenadines Volunteer Patrol. The challenges encountered and lesson learned are discussed as well as planned activities.}

\normaltalk{Vocal Duetting in a Resident Subspecies of Grace's Warbler (Setophaga Graciae Decora) in Belize}{Cynthia A Staicer}{A resident population of Grace's Warbler (Setophaga graciae decora) was studied in the lowland pine savanna near the Caribbean coast of Belize. The study objective was to determine how this resident subspecies' vocal behaviour compares to that of the previously studied migratory subspecies S. g. graciae (Arizona) and the tropical resident Adelaide's Warbler (S. adelaidae; Puerto Rico), which is thought to be derived from a resident population of S. graciae. Observations were made during both the breeding and non-breeding seasons. Song structure was more similar to that of S. g. graciae while certain behaviours were more similar to those of S. adelaidae. Males were paired, sang and defended territories during both breeding and non-breeding seasons. S. g. decora pairs produced a vocal duet consisting of a male song and high-pitched female `pip' calls, very similar to that documented in S. adelaidae. These duets were more commonly heard in the non-breeding season, when the male and female spent more time together foraging in the pines. Assuming S. adelaidae indeed evolved from a resident population of S. graciae, these observations suggest that the duetting behaviour seen in Adelaide's Warblers was the ancestral condition present in a colonizing population of S. graciae, likely decora.}

\normaltalk{Investigation into Phenology of Two Aerial Insectivorous Species and their Potential Insect Prey in Yukon, Canada}{Tara E Stehelin, Fiona KA Schmiegelow}{Global climate change has occurred disproportionately in northern terrestrial regions. One of the oft-cited impacts of climate change is asynchronous changes to phenological events between interacting members of communities, such as breeding timing for birds and abundance peaks of insects to feed young. Data to support the phenological asynchrony hypothesis remain scant, especially in North America. We investigated phenological patterns for evidence of asynchrony between two species of aerial insectivores (Olive-sided Flycatcher, Contopus cooperi and Western Wood-Pewee, C. sordidulus) and essential insect prey in Yukon, Canada, between 2013 and 2017. Investigation into the abundance and diversity of insect prey, as well as timing and success of breeding events, revealed potentially declining insect abundance and nesting success with year, advancement of laying dates, but no change in first territory defense dates with year, and high reliance on second nesting attempts. A daily insect abundance index was related to bird breeding phenology, but evidence of phenological asynchrony using Generalized Additive Mixed Models was not compelling, in part because insect abundance did not reveal obvious or predictable annual peaks. Insect abundance and diversity was relatively high across much of the breeding season. However, an offset of days between individual breeding phenology and estimated annual insect abundance peak influenced number of fledglings in OSFL, but not in WEWP. Thus, timing was important, but our results did not support a hypothesis of asynchrony between insect abundance peaks and nestling feeding in these species, despite rapidly changing climate in this northern study region.}

\normaltalk{Sympatric Divergence in an Isolated Island Finch Radiation Driven by Foraging Adaptations with a Simple Genetic Architecture}{Martin Stervander, Maja Tarka, Max Lundberg, Markus Lindberg, Weilin Wang, Cathlene Eland, Manfred Grabherr, Jacques Dainat, Laura Duntsch, Martim Melo, Peter Ryan, Bengt Hansson}{The endemic Nesospiza finches on the isolated Tristan da Cunha archipelago in the South Atlantic Ocean constitute a small, ongoing radiation driven by pronounced adaptation to divergent food sources. On each of two small islands, Nightingale (4 km2) and Inaccessible (13 km2), occurs a pair of small-billed generalists feeding mainly on grass seeds and large-billed specialists foraging on the seeds of Phylica trees. On both islands the birds show strong bill size-associated assortative mating, except in a confined part of the highland plateau on Inaccessible, where the two forms hybridise. We confirmed that the archipelago was colonised by a small-billed ancestor from the mainland of South America already four million years ago. The selection pressure for large bills was introduced with the colonization of Phylica trees at least one million years ago. A genome scan between small- and large-billed birds revealed two highly differentiated regions on chromosome 1 and Z, respectively, each consisting of two (low- or) non-recombining haplotypes, spanning several candidate genes for bill morphology. The haplotypes arose 40--80 thousand years ago and are shared between islands, implying single mutation events on one of the islands with succeeding introgression into the other. Bill size variation in the restricted hybrid zone was explained by dominance and epistasis at the two chromosomal regions. This study reveals how few large-effect mutations can contribute to extreme trait variation associated with strong differential food preference, habitat selection and assortative mating in speciating taxa.}

\normaltalk{Finding the Right Flight Path to Advance International Migratory Bird Conservation: One Agency's Perspective}{Rebecca LM Stewart, Natalie Savoie, Cynthia Pekarik}{In 2018, under the Canada Nature Fund, the Government of Canada initiated a new program to help address international threats to migratory birds, dedicating 2.4 million dollars over 5 years to conservation for shared bird species at risk and other priority bird species.  The Canadian program was modeled after similar long-standing initiatives such as the United States Fish and Wildlife Service's Neotropical Migratory Bird Conservation Act grant program, and the Association of Fish and Wildlife Agency's Southern Wings.  In its first year, the new Canadian program supported conservation actions, undertaken by 33 partners, for 42 at risk and priority bird species in 12 countries.  Projects engaged 4,000 community members and restored or protected 1,760 ha.   But what do these metrics really mean? And how do we stitch together these efforts and others' to evaluate our collective impact? The recently developed Atlantic and Pacific Flyways plans, and more importantly their broad adoption by funding agencies and conservation implementers alike, suggest that Conservation Business Plans (CBPs) provide the framework needed to address this challenge and to advance collaborative, and strategic, bird conservation at broad scales and across jurisdictions.   Further, CBPs allowance of multiple conservation targets should allow the integration of multiple and more diverse conservation targets, and thus have the potential to meet the needs of multiple jurisdictions and diverse stakeholders.}

\normaltalk{Understory Community Change over Three Decades in Undisturbed Amazonian Rainforest: Frugivores Win, Ground Insectivores Lose.}{Philip C Stouffer, Vitek Jirinec, Cameron L Rutt, Angelica Hernandez-Palma, Erik L Johnson, Luke L Powell, Jared D Wolfe, Thomas E Lovejoy, Richard O Bierregaard}{How temporally stable are Amazonian rainforest bird communities in undisturbed forest?  Patterns in intact forest may reveal the signal of broad-scale climate and forest structural changes.  We analyzed mist net samples collected from 55 sites spread over ~40km sampled identically in the early 1980s and again beginning in 2008 at the Biological Dynamics of Forest Fragments Project near Manaus, Brazil.  Overall capture rate was stable.  Capture rates of some understory guilds mirrored this pattern.  For example, obligate mixed species insectivore flock participants collectively comprised about 20\% of captures in both samples.  Other segments of the community showed directional change.  The most striking declines were for ground and near-ground insectivores, which dropped from 12\% to 6\% of captures. The 13 species whose abundance rank declined most precipitously were all from this group, including representatives of five families. Obligate army ant followers, which use similar food resources, increased from 15.5\% to 18\% of the net sample. The most striking increases were for manakins, which nearly doubled to $>$10\% of the sample. A phylogenetically diverse set of understory omnivores also increased, now comprising $>$5\% of captures. These results suggest that ground- and near-ground microhabitats have deteriorated, but that understory resources for frugivores may have increased. Obligate specialized foraging associations, either following ants or flocking, are not disadvantageous. If these results represent broad patterns in undisturbed Amazonian rainforest, we need to recognize that communities have changed predictably in recent decades, and that conservation concerns for terrestrial insectivores extend beyond disturbed landscapes.}

\normaltalk{Winter Survival of Baird's and Grasshopper Sparrows in Chihuahuan Desert Grasslands}{Erin H Strasser, Irene RuvalcabaOrtega, Jos\'{e} Hugo Mart\'{i}nez Guerrero, Mieke Titulaer, Ricardo Canales del Castillo, Quresh S Latif, Arvind O Panjabi, Maureen D Correll, Mart\'{i}n Pereda Solis, Alexander Pe\~{n}a Peniche, Daniel Sierra Franco, Denis J P\'{e}rez Ordo\~{n}ez}{Survival during the nonbreeding season can have wide ranging implications for migratory bird populations and should be taken into account to effectively conserve threatened species. Birds that winter in the Chihuahuan Desert grasslands are one of North America's fastest declining guild of avifauna, yet little is known about survival during the nonbreeding season. To identify factors that limit the survival of grassland birds on the wintering grounds we radio tagged and tracked 684 Baird's and 890 Grasshopper Sparrows during the winter months of 2012 through 2019. We collected information on causes of sparrow mortality, habitat conditions, local weather, and bird characteristics (e.g. sex) at three sites in northern Mexico and one in West Texas. We found that winter survival differed between species and was highly variable across winters and sites. Of the 1574 sparrows that we tracked, we registered 486 deaths and found that the primary cause of mortality was predation by Loggerhead Shrikes and raptors. To improve our understanding of the relationship between shrubs and wintering grassland sparrows we characterized shrub cover and height with imagery collected at the four sites via Unmanned Aircraft Systems (UAS). Within rasters derived from UAS imagery we classified objects as shrubs with over 90\% accuracy. With these data we provide further insight into the drivers of winter survival, relationships between shrubs and grassland birds, and consider how our results can guide management strategies in the Chihuahuan Desert to benefit grassland birds.}

\normaltalk{Single but where Do They Mingle: Tracking Daily Extra-Territorial Movements in a Cooperative Breeder}{Young Ha Suh, Vivienne Sclater, Reed Bowman, John W Fitzpatrick}{Prospecting, the behavior of gathering information on surroundings, is a mechanism that allows individuals to identify and disperse to high quality territories yet is elusive and challenging to study in the wild. To investigate prospecting behavior and its significance, we collected constant tracking data of prospecting Florida Scrub-Jays where offspring prospect for breeding opportunities while delaying dispersal. Using recently developed tracking technology, we tagged nonbreeders prior to the breeding season and tracked them throughout the year. To understand fine-scale movement, we obtained 2-minute averaged location estimates and quantified spatio-temporal characteristics of extra-territorial movements defined by territory boundaries that are annually mapped as part of our long-term study. We then conducted spatial analyses to test for habitat use. Preliminary results from 9 tagged individuals in February 2020 yielded an average of 2,203 points per tag over 2 weeks and showed strong individual variation. Prospectors were seen outside their natal territory for 50.6\% of the time (SD=13.6, range=34.6-81.2). Extra-territorial forays had a mean distance of 87.2m (SD=79.4, range=0.01-840.8m) and exhibited a strong leptokurtic distribution from the natal territory boundary. Prospectors spent more time in optimal habitat (e.g. scrubby flatwoods) outside their territory boundaries, which supports the notion that prospectors are scoping out high-quality habitat. Our results are one of the first to document fine-scale movement patterns and show high variation in prospecting behavior that could potentially affect dispersal decisions. Ultimately this study will shed light on an elusive behavior considered to be crucial to habitat selection and dispersal.}

\normaltalk{Birds, Windows and Trees are a Deadly Combination: An Investigation of Bird Window Collisions and Mitigation Effectiveness on a University Campus.}{Kimberly Sullivan, Rachel Sagers}{Window collisions are estimated to kill up to 1 billion birds a year in the United States.  Following a 2015 incident where 60 Cedar Waxwings collided with a façade of mirrored tiles on the Utah State University campus, the USU Bird Strike Project was formed to identify problem glass areas on campus and develop mitigation strategies.  Participants documented 874 collisions on 3992 surveys of university buildings over a three-year period.   Collisions occurred year-round with Cedar Waxwings and American Robins making up the majority of collisions.  Afternoon surveys had a greater likelihood of finding a carcass suggesting that most collisions occurred during the day.  Window area was not a significant predictor of collision rate but the number of trees planted within 15 meters of the side of a building was a significant predictor of collision rate.  Extensive plantings of ornamental fruit trees and shrubs on campus appear to attract birds to campus and lead to increased rates of collisions.  Data from game cameras indicated that approximately ten percent of carcasses were removed from buildings with frequent strikes before they could be found by a surveyor.  Cats and Black-billed Magpies scavenged the carcasses.  Effective migration efforts included installing shades in walkways and buildings that birds could see through and fritting window glass.}

\normaltalk{Environmental Conditions Modulate Compensatory Effects of Site-Dependent Regulation in a Food-Caching Passerine}{Alex O Sutton, Dan Strickland, Nikole E Freeman, D R Norris}{Although density regulates the abundance of most wild animal populations, the mechanisms responsible for generating negative density-dependence are unclear for many species. Site-dependent regulation occurs when there is preferential filling of high-quality territories, which results in higher per capita vital rates at low densities because a larger proportion of occupied territories are of high quality. Using 41 years of territory occupancy and demographic data, we investigated whether site-dependent regulation was a mechanism acting to regulate a population of Canada jays in Algonquin Provincial Park, ON. As predicted by site-dependent regulation, the proportion of occupied territories that were of high quality was negatively correlated with population density and periods of vacancy were shorter for high-quality territories than for low-quality territories. We also found evidence that per capita fecundity was positively related to the proportion of occupied territories that were of high quality, but only when environmental conditions, which influence the entire population, were otherwise poor for breeding. Our results suggest that site-dependence likely plays a role in regulating this population but that environmental conditions can modulate the strength of density-dependence.}

\normaltalk{Natal Survival and Dispersal Influence Population Structure in a Threatened Shorebird}{Rose J Swift, Michael J Anteau, Kristen S Ellis, Megan M Ring, Mark H Sherfy, Dustin L Toy, David N Koons}{Natal survival and dispersal have important consequences for populations through the movement of genes and individuals. For species reliant on dynamic habitats, availability and location of appropriate habitat will shift from year-to-year, requiring natal dispersal to locate appropriate breeding habitat. Metapopulation theory predicts either balanced dispersal among regions or source-sink dynamics, which can dramatically change population structure. We quantified natal survival, dispersal probabilities, and dispersal distances in a migratory shorebird, the Piping Plover (Charadrius melodus), with data from four cohorts (n = 2,667) in the Northern Great Plains (NGP). Juvenile survival (fledge to SY) varied among regions and was highest for river (0.77 95\% CI [0.65, 0.89]) and lowest for reservoir habitats (0.56 [0.46, 0.66]). Natal fidelity was highest for alkaline wetlands (0.66) and lowest for reservoirs (Lake Sakakawea: 0.28, Lake Oahe: 0.35). As habitat availability decreased, natal dispersal increased. Individuals that hatched later in the breeding season, on reservoir habitats, in areas with high natal chick densities, or less available habitat dispersed farther. Overall, high natal dispersal rates within the NGP indicate high connectivity among distinct regions driven by fluctuating availability of habitat with different water regimes on the Missouri River and in Prairie Pothole wetlands. However, reservoir habitats appear to be `sinks' with low fidelity, low juvenile survival, and high dispersal distances. Our results suggest that plovers in the NGP take advantage of dynamic habitats where they are available in a broad geographic area but further research on adult breeding dispersal is needed to clarify population structure.}

\normaltalk{Individual Vocal Distinctiveness and Neighbor-Stranger Discrimination in a Suboscine Bird, the Red-Capped Manakin}{Josephine T Tagestad, J P Kelley, Corey E Tarwater}{Examination of vocal traits can lead to insights regarding the health, age, identity, and behavior of individuals and populations. Studies of bird song and its function focus almost entirely on the learned song of oscine birds, while the less vocally complex suboscine birds are often overlooked. Suboscines are assumed to lack individual distinctiveness in their vocalizations, leading to assumptions regarding their information sharing capabilities. In this study, we first examined whether vocalizations of the suboscine, Ceratopipra mentalis, the red-capped manakin, are individually distinctive. Second, we tested whether birds responded differently to neighbor versus stranger calls. We examined 120 call recordings from leks across the Isthmus of Panam\'{a}, extracting cadence and frequency data from a call given by males to test for individual variation. We also performed playback trials to investigate neighbor-stranger discrimination. We found that individual variation is present, with an FDFA assigning identity to calls with 78\% accuracy. Lastly, we found that males vocalized more quickly in response to strangers than to neighbors, but movement responses did not differ and the vocal response to strangers rapidly declined. This study suggests that using rapid, non-invasive methods, such as vocal recordings, to estimate abundance is feasible, important given that C. mentalis is a vital seed disperser and is expected to decline with climate change. Furthermore, we argue for the need to test the long-held assumptions about suboscine vocal capabilities and impacts on behavior.}

\normaltalk{Coastal Marsh Birds: Continued Decline or the Next Conservation Success Story}{Paul J Taillie}{Coastal wetlands represent one of the most valuable ecosystems on the planet due to the varied ecosystem services they provide, including habitat for a unique suite of bird species.  Though coastal wetlands are often characterized as highly vulnerable to rising sea level, they may also be resilient to these changes via processes like vertical accretion and landward migration.  However, avian responses to changing marsh conditions, particularly marsh migration, are poorly understood.  We designed two related studies to better understand the vegetation dynamics as marshes migrate in one of the most vulnerable regions to rising sea level, and how marsh birds are responding. We show that in a little over a decade, extensive tracts of forest have transitioned to marsh, yet newly-created marshes do not appear to provide high quality habitat for marsh-associated birds of high conservation concern.  However, management approaches aimed at minimizing shoreline erosion, reducing habitat loss from land conversion and development, and increasing marsh habitat quality with prescribed fire have potential to work with rising sea level, rather than against it.  Thus, even at the front lines of a changing climate, opportunities exist to conserve birds of high conservation concern.}

\normaltalk{Use of Artificial Nests and Selection Requirements by Amazon Parrots in Belize}{Fabio L Tarazona-Tubens, Charles R Britt, Fitsum Abadi, Martha J Desmond}{Availability of nesting sites is a key resource influencing population dynamics of secondary cavity nesters. Supplementation of nest boxes is a common practice to increase nesting opportunities for cavity nesters, such as psittacids (i.e. parrots), many of whom are species of conservation concern. However, understanding selection requirements of most psittacids remains an important challenge for effective conservation actions and placement of nest boxes in the landscape. This study utilized logistic regression models to investigate selection of nest boxes    (n = 60) by Amazon parrots in relation to vegetation cover around the nest site and nest tree characteristics. Furthermore, we evaluated species-specific nest selection requirements by the most common species utilizing the nest boxes. Overall, 60\% of nest boxes were used by four species of Amazon parrots. White-fronted (Amazona albifrons) and Yellow-headed Parrots (Amazona oratrix) accounting for the majority of use; 27\% and 28\%, respectively. Increasing nest tree height had a strong positive effect on selection (ß = 0.20, 95\% CI: 0.05, 0.34); areas with less vegetation cover and greater visibility from the nest were also preferred, but these variables did not have a strong effect on selection. Species-specific analysis revealed that Yellow-headed Parrots selected nest boxes in taller trees, closer to previously successful nests; while White-fronted Parrots did not show strong selection preference for any measured characteristics. Our results highlight the potential use of nest boxes as a conservation tool and emphasize the importance of maintaining tall trees on the landscape that can be utilized for placement of nest boxes.}

\normaltalk{Alterations in Species Interactions and Population Ecology Across a Rainfall-Fragmentation Gradient in Panama}{Corey E Tarwater, J Dylan Maddox, Jeffrey T Foster, Laura Gomez Murillo, Michael Castano, Rebecca L Mau, J Patrick Kelley}{Natural and anthropogenic changes in the environment can have dramatic effects on population demography, species interactions, and community structure. In particular, habitat fragmentation can affect species residing in fragments, with underlying differences in rainfall making some species and communities more vulnerable. We worked along a rainfall-fragmentation gradient in Panama to examine how these variables influence individuals, populations, and communities, and whether there were synergistic, additive, or antagonistic effects. We observed fewer individuals in understory antwren flocks in wetter sites and a complete breakdown of flocks in the smaller fragments. For army ant following birds, contiguous forest in the middle of the rainfall gradient had the greatest number of individuals and species attending ant swarms, while the extremes of the rainfall gradient and small fragments had fewer birds and ants. We also examined individual and population-level variation in one forest understory insectivore, Thamnophilus atrinucha (antshrike). We found variation in population density, body condition, blood parasites, bill morphology, and movement across the gradient. Owing to the differences observed in bill morphology, we also examined diet and the association between bill morphology and bite force. Overall, we have found that species interactions are dramatically changing across the gradient and one species had substantial differences in morphology, behavior, and density. Contiguous forest in the middle of the rainfall gradient was the most speciose and differed the greatest in bill morphology and diet compared to the extremes of the rainfall gradient, while the smallest fragments had a breakdown in species interactions.}

\normaltalk{Mist Net Height and Avian Monitoring: Ground-Level Mist Nets Introduce Biases to Constant-Effort Bird Banding Data}{David J Tattoni, Katie LaBarbera}{Mist netting is widely for monitoring avian populations over time. Many banding stations primarily use ground-level mist nets, above which birds can occupy unsampled space. Therefore, data analyses require assumptions that the degree to which birds occupy unsampled vertical space is unrelated to variables of interest. Little work has empirically demonstrated the validity of these assumptions. We analyzed capture data from paired ground-level and aerial mist nets from 1993-2019 at the Coyote Creek Field Station, CA. We built binomial models in a Bayesian framework to estimate height bias in capture probability and tested for differences based on age, sex, season, and year, and developed a quantitative framework to correct for these biases. Additionally, we simulated apparent capture rates under conditions where no population change occurred but where height biases varied over time. Biases were found for 15 species. We found years with higher probabilities of aerial capture were significantly correlated to declines in overall capture rates. Our quantitative framework corrected for these biases to produce more accurate population trend estimates. Our results demonstrate ground-level nets may bias data in ways not previously recognized. We introduce the concept of apparent capture rates to account for confounding effects of height-related variation in capture probability on estimating population trends. This framework makes it possible to correct capture rates and improve the accuracy of population trend estimates. We show that the wider use of paired ground-level and aerial mist nets at constant-effort mist netting stations can provide the data necessary to make these corrections.}

\normaltalk{Understanding Population Regulation and Limitation in a Genoscape-Network of a Declining Migratory Songbird.}{Caz M Taylor, Ryan Harrigan, Kristen Ruegg, James Saracco}{To understand declines in migratory animals, we need to determine what limits and regulates populations. We applied a network population model to a genoscape-network for a declining songbird, Wilson's Warbler Cardellina pusilla. The network has six breeding nodes, each representing a genetically distinct lineage, and five winter nodes. The population model assumes density-dependent regulation in winter and breeding seasons and distance-dependent migration survival; and predicts the network migratory connectivity. Carrying capacities were estimated using habitat suitability modeling and strengths of density-dependence by fitting the model, using a Bayesian approach, to genetically derived connectivity observations and citizen-science derived estimates of relative breeding and winter population sizes.   Our model shows that five of the six genetic lineages were primarily breeding-regulated, meaning loss or degradation of their breeding habitat caused dramatically larger declines than loss of winter habitat. However, one genetic lineage, Western Boreal, was primarily winter-regulated and loss of winter habitat in in Central America and Mexico caused large declines in this group while loss of breeding habitat had only a small effect. Because a large majority (~85\%) of Wilson's warblers belong to the Western Boreal lineage, winter habitat loss had the biggest effect on overall declines in the global population. To preserve the largest number of birds in this species, conservation actions should focus on protecting winter habitat in Central America but maintenance of genetic diversity requires additional preservation of breeding habitat across the breeding range in Coastal California, Sierra Nevadas, Basin Rockies, Eastern Boreal, and the Pacific Northwest.}

\normaltalk{Life History Lessons for the Evolution of Delayed Plumage Maturation}{Liam U Taylor}{In birds with Delayed Plumage Maturation (DPM), younger individuals develop drab, juvenile-like, ``predefinitive" plumages whereas older individuals molt into the colorful ``definitive" plumages that characterize increased reproductive effort and success. Thus far, DPM research has focused on testing for direct advantages of predefinitive plumages (i.e., ``why is it good to be green?"), rather than testing for selection that can actually result in delayed maturation. Here, I turn to life history theory for two clear mathematical expectations about the evolution of DPM. First, a direct advantage of a predefinitive plumage is neither necessary nor sufficient for the evolution of DPM. Second, a necessary condition for positive selection on DPM is that individuals get better at something as they age. Because older birds with DPM are on average better at something than younger birds, we cannot test the function of predefinitive plumages using unmanipulated comparisons between young/predefinitive and old/definitive groups. Given this limitation, I review the maturation trajectories that foster differences between young and old birds in the first place. Maturation can include the mitigation of classical ecological costs (e.g., getting better at foraging for large seabirds, maintaining resource territories for small passerines) as well as more novel forms of social development (e.g., calibrating pair-bonds for long-lived monogamous birds, learning complex social regulations for polygynous lekking birds). Grounding DPM in life history theory not only provides a tractable framework for ornithological research, but also positions avian development to radically enrich the simplified examples of life history evolution.}

\normaltalk{Studies of Avian Radiations in the 21st Century, an Introduction}{Scott A Taylor, Martin Stervander, Leonardo Campagna, David P Toews}{Avian radiations, such as the emblematic Darwin's Finches of the Gal\'{a}pagos Islands and the Honeycreepers of the Hawaiian Islands, have fascinated ornithologists for centuries and continue to play a pivotal role in our understanding of natural selection and speciation. The most dramatic kinds of avian radiations are adaptive, where there is a high rate of species formation within a clade and where multiple lineages diverge and ecologically specialize, such that they are able to overlap in their distribution. Studying the mechanisms that promote these kinds of radiations allows us to learn about the fundamental processes that drive avian diversification, as well as the traits that contribute to important ecological differences among closely related species. The recent revolution of DNA sequencing technologies, as well as methodological advances within the field of behavioral ecology, the quantification of color, and the analysis of vocalizations, provides unprecedented resolution of the processes driving adaptive radiations and dramatically improves our ability to infer historical relationships and movements. This symposium will highlight new insights into avian radiations from a variety of perspectives, including migratory shifts, rapid diversification of plumage and vocalization patterns, and genomic analyses identifying genes under natural selection for ecological adaptation or sexual selection. Together, the talks in this symposium will discuss advances in our understanding of both the proximate mechanisms (e.g. through functional genetics) and the ultimate consequences (e.g. behavioral ecology) involved in avian adaptive radiations.}

\normaltalk{Hurricane Maria Alters Hummingbird Pollinator Relationships and Natural Selection on an Introduced Island Plant on Dominica, W. I.}{Ethan J Temeles, Gabriella A Bishop}{Specialized plant-pollinator interactions may break down following hurricanes due to mortality and a shortage of preferred pollinators. How such breakdowns influence the strength and direction of natural selection on interacting taxa is unclear. We report on the impacts of Hurricane Maria on a specialized pollination system involving the Purple-throated Carib (Anthracothorax jugularis) and its native and non-native Heliconia food plants. Seven months after the hurricane, the only Heliconia available to caribs was the non-native H. wagneriana, which has substantially longer flowers than either of the native heliconias. Whereas prior to the hurricane, female caribs, which have longer bills than males, were the main visitor to H. wagneriana, after the hurricane, shorter-billed male caribs were the main visitor to H. wagneriana. The absence of trait-matching between a short-billed pollinator and a long-flowered plant resulted in directional selection for shorter flowers because males preferentially visited plants with shorter flowers. We estimate a roughly 75\% decrease in the population size of purple-throated caribs, and results suggest the heaviest mortality occurred among short-billed males and larger-bodied individuals of both sexes, which would have higher nectar requirements and the most difficulty obtaining nectar. Native heliconias resumed full flowering approximately 1.5 years after the hurricane, and we observed similar lack of trait-matching in some populations that resulted in reversals in the direction and intensity of selection on floral traits. Our results suggest that hurricanes may be a major force in shaping evolutionary trajectories of plants and their pollinators on tropical islands.}

\normaltalk{Top-Down vs. Bottom-Up Regulation and Island Birds}{John W Terborgh}{MacArthur and Wilson's (1967) Theory of Island Biogeography stimulated an enormous amount of research on islands and island phenomena. In the intellectual environment of the 1960s and 1970s, interspecific competition was the default assumption for data interpretation. Now, more than 50 years later, the theoretical landscape of ecology has broadened substantially with an expanding appreciation of top-down forcing and its potential for structuring ecosystems and communities. In many circumstances, predictions about community diversity and organization based on top-down reasoning are the exact opposite of those derived from competition theory. Nowhere is this more pronounced than on islands.     My goal in this talk will be to bring out some of the contrasting predictions generated through top-down vs. bottom-up approaches and to weigh the evidence on each side in the hope of arriving at more confident conclusions. I will employ data from published studies, including some of my own from Lago Guri, Venezuela. Bird communities of Lago Guri islands responded to both bottom-up and top-down forcing, but under distinct circumstances that allowed us to distinguish the consequences. Implementation of some straightforward tests could shed new light on the status of island bird populations vis a vis the mainland, and perhaps lead to a second generation theory of island biogeography.}

\normaltalk{Investigating the Effects of Climate Change on Phenology and Demographics of Molt-Migration in the North American Monsoon using Museum Resampling}{Ryan S Terrill, Whitney LE Tsai Nakashima, James M Maley, Maggie Shedl, Sievert Rohwer, John E McCormack}{Global climate change can affect birds through phenological mismatches when the timing of climatic events changes relative to exogenous triggers for life-history events such as breeding or migration. Migratory birds must plan their annual molts to coincide with food availability, and many migratory songbirds molt their feathers in the region of the North American Monsoon (NAM) to coincide with monsoonal rainfall and subsequent primary productivity. With climate change, the NAM is becoming stronger and later in the year, as drought and aridification increases across western North America. We investigate how this phenomenon may impact the timing of molt in these birds by resampling museum specimens over a 90 year period, and fitting novel models of molt timing that combine molting and non-molting birds to historical and modern specimens. We also investigated temporal stability of migratory divides in use of the NAM as a molting ground through sequencing of Ultraconserved Elements (UCEs) from historical and modern specimens on the molting and breeding grounds in order to connect populations. We find long-term stability in the phenology and migratory routes of the NAM as a molting ground, which means that birds that depend on the NAM to molt may in the present of future be negatively affected by a later onset to the monsoon as species are unable to adjust the timing or region of their molt along their migratory pathway.}

\normaltalk{Puerto Rican Sharp-Shinned Hawk: Living on the Edge}{Russell Thorstrom, Julio C Gallardo, Melissa Murillo, Hana Weaver, Thomas Hudson}{We surveyed and searched for the endangered Accipiter striatus venator (Puerto Rican Sharp-shinned Hawk) in several montane forests (=500 m) of the Cordillera Central range and other montane areas during the breeding season from January to July 2015-2019. We recorded a high of 75 hawks in 2017, all restricted to the highest montane forests (=700 m). From 2015-2017 we documented 38 nesting attempts, including four second attempts, and a total productivity of 1.1 young fledged per nesting attempt (n= 42 young fledged). In September 2017, Hurricane Mar\'{i}a struck Puerto Rico causing widespread devastation to the forested habitat throughout the island, and especially in the montane forests of the Cordillera Central. Surveys and searches in 2018, post-Hurricane Mar\'{i}a, recorded 19 hawks an alarming 75\% population decline from 2017. In the population stronghold at Toro Negro Forest, 92\% of the nest trees from 2017 were snapped or blow downs. From 2018-2019, we documented 9 natural (non-managed) nesting attempts by 8 pairs and a total of 4 young fledged (n=2 nests) with a productivity of 0.4 fledged per nesting attempt, a 64\% decrease from the previous years. Of the 7 nest failures: three were egg predation (Pearly-eyed Thrashers Margarops fuscatus), two were nestling predation (Caribbean Red-tailed Hawks Buteo jamaicensis jamaicensis and Pearly-eyed Thrasher) and two were nest fly parasitism (Philornis sp.). Hurricane Mar\'{i}a had a devastating effect on this small island population, and the reason for the current conservation measures taking place to save this unique hawk.}

\normaltalk{Assessing Priority Bird Response to Open Pine Management in Eastern Mississippi}{Holly M Todaro, Kristine O Evans, Steven E Lewis}{Fire suppression, combined with lack of forest thinning and short-rotation monodominant management, has drastically altered the landscape in the southeastern U.S., leading to the loss of open pine ecosystems and associated bird species. Restoration of open pine ecosystems through forest thinning and prescribed fire is a common practice; yet, there remains uncertainty regarding the range of specific forest structural characteristics that are advantageous to these declining species. We conducted yearly point-transect surveys in five separately managed pine habitat units at Sam D. Hamilton Noxubee National Wildlife Refuge in eastern Mississippi. Using a combined distance and time removal sampling approach we assessed avian communities in each management unit from 2016-2019. We used generalized linear models to assess species associations with vegetation characteristics and determined that priority species abundance is associated closely with pine basal area and percent pine overstory canopy cover. Additionally, Bachman's sparrow, prairie warbler, and northern bobwhite abundance was positively associated with percent cover of native grasses. Hardwood midstory cover was negatively associated with both priority and non-priority species. These results suggest that open pine management with forest thinning, followed by prescribed fire enhances abundance of priority bird species as well as non-priority bird species by providing structural conditions preferred by open pine species.}

\normaltalk{Diet Variation in a Diverse Community of Wood Warblers Revealed by Meta-Barcoding}{David PL Toews, Andrew Wood, Andreanna Welch, Robert Fleischer, Adrienne Dale, Eliot Miller}{The parulid warblers of North America are a well-known avian radiation with a rich legacy of study. This includes early works by Robert MacArthur in the 1950s, which remain a cornerstone of modern community ecology. MacArthur's studies focused on possible niche differences of co-occurring species within the family, with a particular emphasis on fine-scale foraging variation. Understanding the exact types of forest insects these warblers are eating, however, has been difficult to quantify. This is partly because directly observing the foraging of these warblers is challenging and, even when stomach contents are studied directly, partially digested insect remains can be difficult to identify to taxonomic groups. Since 2017, we have been using high-throughput sequencing analysis of fecal samples (i.e. ``fecal metabarcoding") to identify the insects consumed as prey for dozens of individuals for many different co-occurring warbler species in New York and Pennsylvania. We discuss how we have used this new information of diet variation to estimate species-level multivariate hypervolumes, which are estimates of each species' niche as a function of diet/resource use. Furthermore, we use these derived hypervolumes to calculate niche overlap, niche partitioning among species within sites, and similarities and differences within species between sites. We also describe how invasive arthropods may alter patterns of niche overlap.}

\normaltalk{Long-Term Population and Community Dynamics in the Dry Forest of Gu\'{a}nica, Puerto Rico}{Judith D Toms, John Faaborg, Wayne J Arendt}{We have been monitoring resident and winter-resident bird populations conducting in the Gu\'{a}nica dry forest of southwest Puerto Rico since 1972. Bird monitoring is conducted via constant-effort mist netting, and rainfall patterns have been assessed based on data from a local weather station, supplemented with data from regional stations. We analyzed data using a mark-recapture framework that allows us to separately estimate survival rates, capture probabilities and recruitment of new individuals into the population. Over the 48 years of this survey, the structure of this slow-growing forest has remained relatively stable, other than short-term damage resulting from hurricanes. However, rainfall patterns have been changing over time, and populations of many residents and winter-resident species have shown consistent declines. We demonstrate that changes in survival underlie these population trend in some resident and winter-resident species, and that rainfall influences survival and recruitment rates for some resident species. Although this bird community appears to be resilient to pulse stressors such as hurricanes, they do not appear to be resilient to press stressors such as changing rainfall patterns associated with global climate change.}

\normaltalk{Identification and Evaluation of Mortality by Predation in Nests of Thick-Billed Parrot (Rhynchopsitta Pachyrhyncha) in ANP´s of Chihuahua, Mexico}{Luz F Torres Gonzalez, James Sheppard, Edwin Juarez, Jose Ignacio Gonzalez Rojas, Javier Cruz Nieto, Miguel A Cruz Nieto, Daniel M Rangel Resendez, Maria De Los Angeles Olvera Diaz, Francisco M Puente Guevara}{The thick-billed parrot is in danger of extinction (IUCN 2017), during the last 19 years we have observed a general average of 115.57 ± 23.32 breeding pairs per season with a minimum of 86 and a maximum of 177 in five nesting sites. In 2012 the first predation in a nest was documented. During the last two reproductive seasons, 15 and 33 trap chambers were placed in nest trees, for the identification of potential predators and all the active nests were checked for evidence of predation. As a result, 5 species were identified as potential predators (Lynx rufus, Nasua narica, Spilogale gracilis, Procyon lotor and Cyanocitta stelleri). The first photographic record of predation by Lynx rufus in four nests was made and the pattern of predation by Lynx rufus was also described. This presents an average distance of predation offal at 2.37 ± 2.23 meters, with a range of 0.25 to 7 meters away from the base of the predatory nest. The predation radius was on average 40.06 ± 5.77 centimeters with ranges from 29 to 50 centimeters. The preyed specimens only had portions of wings, legs, beak and feathers concentrated in the same radius. Although they showed no sign of teeth, if in some cases some broken feathers were detected mostly on feathers with secondary covers at the base of the socket. Predation was documented in 13.20\% of the population of breeding pairs for 2019.}

\normaltalk{Assessing Bird Populations While Providing Research Experiences to Undergraduate Students in Guajataca State Forest, Puerto Rico}{Adrianne G Tossas}{I assessed the bird population dynamics in Guajataca State Forest, northwestern Puerto Rico from 2009- 2019. A total of 1,641 individuals from 23 species was recorded in 140 point counts. For the first eight years, only one species, the Black-whiskered Vireo (Vireo altiloquus) showed an increasing population and 13 species remained stable. Nine species fluctuated but with an overall trend of decreasing populations, including two endemic species, Adelaide's Warbler Setophaga adelaidae, and Puerto Rican Vireo Vireo latimeri. The passage of category 4 Hurricane Maria through the study area in 2017 dramatically altered the habitat and species diversity. Tree uprootings, broken branches, and severe defoliation caused the canopy cover to open by 22\% compared to the previous year. Two years after the passage of the hurricane, six species showed population declines (e.g., Bananaquit Coereba flaveola, Puerto Rican Tody Todus mexicanus), while nine species increased (e. g., Puerto Rican Bullfinch Melopyrrha portoricensis, Puerto Rican Vireo), possibly due to improved detectability after the opening of the forest structure. Omnivorous species increased after the hurricane compared to declines in the insectivorous, nectarivorous and frugivorous foraging guilds. Over 100 undergraduate students helped collect the data of this decade-long project, as part of a field course in which they engage in their first research work. Their participation has allowed the project to expand to include monitoring of other bird groups in the region. At the same time, participants benefited by obtaining the scientific experience needed to apply for graduate programs or start their professional careers.}

\normaltalk{West Nile Virus Emergence and Changes in Genome-Wide Diversity in American Crows}{Andrea K Townsend, Anne B Clark, Kevin McGowan, John A Eimes}{Population bottlenecks can lead to loss of genetic diversity, a reduction in the ability to adapt to environmental change, and an increased likelihood of inbreeding. Bottlenecks can also reduce diversity at loci critical to immune defense, such as the major histocompatibility complex (MHC). Here, we examine changes in individual and genome-wide genetic diversity in American Crows (Corvus brachyrhynchos) after the initial emergence of West Nile virus (WNV), a vector-borne pathogen that killed ~40\% of local crows after it emerged in Ithaca, NY, in 2002. We  used samples collected between 1990 and 2011 to estimate changes in genome-wide and neutral diversity at (1) a panel of 30 microsatellite loci (n = 826 birds; 418 and 408 sampled pre- and post-WNV, respectively) and (2) a panel of SNPs (6078 variant sites; n = 120 crows pre- and post-WNV emergence) generated by ddRAD sequencing. After WNV emergence, individual heterozygosity estimates at both SNP and microsatellite markers declined, indicating that the occurrence of inbreeding increased after WNV emergence. Contrary to expectation, however, diversity at the population level did not decline after the epizootic, and new alleles were gained at both neutral and MHC loci. Immigration provides one potential explanation for the appearance of new alleles into this population: loss of birds in this previously stable population may have opened breeding opportunities for immigrants, bringing new alleles into the population and buffering it from declines in genetic diversity.}

\normaltalk{From g to Joules: Measuring Energy Expenditure of Black-Legged Kittiwakes with GPS-Accelerometry}{Fred Tremblay, Shannon Whelan, Emily Shepard, Scott Hatch, Kyle Elliott}{Doubly-labelled water (DLW) has been widely used to assess energy expenditure in birds, but animal-borne devices offer a less invasive alternative. An individual's energy expenditure is driven by internal (e.g. breeding stage) and environmental factors. Reproduction is costly life event and is often associated with shifts in behaviour. We hypothesized that energy expenditure increases throughout the breeding season, because energy requirements increase as offspring develop and parents increase expensive foraging activity to provide for offspring. We calibrated traditional DLW methods with modern GPS-accelerometry to estimate energy expenditure of black-legged kittiwakes (Rissa tridactyla) breeding on Middleton Island, in the Gulf of Alaska. We studied the movement of black-legged kittiwakes prior to breeding, during incubation and chick-rearing. We equipped 72 birds with GPS-accelerometers and injected a subset of birds with DLW (n = 39). We used DLW to assess field metabolic rate (FMR) and compare energy expenditure across breeding stages. We used paired DLW and GPS-accelerometry to produce activity-specific metabolic rates for time spent on colony, in flight, and on water. We observed distinct differences in activity budgets and foraging locations across breeding stages, although foraging distance was constant. Daily energy expenditure increased as the breeding season progressed. The calibration produced will facilitate estimation of DEE using only GPS-accelerometers, reducing the impact on animals, researcher effort, and expense relative to traditional DLW methods. Thus GPS-accelerometry can accurately track energy expenditure of wild birds in response to behavioral and environmental change.}

\normaltalk{Post-Fledging Ecology and Behavior of an Endangered Species: The Golden-Cheeked Warbler}{Evalynn M Trumbo, Michael P Ward, Jeffrey D Brawn}{The post-fledging period is often a period of low survival and rapid change in songbird behavior (e.g., habitat use, mobility, foraging rate). Information about the post-fledging period is therefore critical for effective management and conservation of birds. We investigated the post-fledging ecology of the endangered Golden-cheeked Warbler (Setophaga chrysoparia). We radio-tracked fledglings (n = 23) for at least four weeks and monitored habitat use, estimated survival, and how fledgling and parental behavior changed as fledglings aged. Overall, fledgling habitat was composed of Ashe juniper (Juniperus asheii; 74.7\%), and oak (Quercus spp.; 20.7\%). Daily survival rate of fledglings during the first four weeks out of the nest was 0.985 (95\% CI = 0.971-0.993). Whereas the proportion of juniper around fledglings was positively associated with their probability of survival, we found considerable uncertainty in the effects of habitat. As fledglings aged and foraged independently, their use of oaks increased, and they were more successful at foraging in oaks compared with junipers. Adults remained with fledglings for a minimum of five weeks. Female provisioning rates for fledglings were greater than male rates, but both parents fed less as fledglings aged. The sex of the parent caring for the fledging, however did not affect fledgling survival. Previous research recommends maintaining mature juniper-oak forests for breeding adults and it appears this recommendation also holds true for fledglings.}

\normaltalk{Drivers of Distributions: Linking Dispersal Ability to the Biogeographic and Evolutionary Histories of Tropical Birds}{Brian R Tsuru, Jacob C Cooper, John M Bates}{The geographic ranges of bird species are extremely diverse, sometimes differing in size by orders of magnitude and/or showing pronounced differences in pattern (e.g. continuity vs. patchiness) across the occupied range. Prior work has repeatedly linked dispersal ability, or the capacity of birds to cross landscape barriers, to aspects of birds' geographic distributions, such as overall range size. However, most studies assess this relationship at broad taxonomic scales. Comparisons of dispersal and biogeography among closely related taxa are largely lacking in the ornithological literature. Here, we compare and contrast variation in dispersal ability (assessed via Kipp's Index, a proxy for wing shape) and geographic range size in two groups of tropical birds: robin-chats (genus Cossypha, Muscicapidae) of sub-Saharan Africa and toucans and their close relatives (Ramphastidae, Capitonidae, and Semnornithidae) in the Neotropics. Within Cossypha and the Capitonidae/Semnornithidae, we find no significant correlations between dispersal ability and range size, but do see a strong relationship between these traits in the Ramphastidae. We note multiple cases in both groups where comparative assessments of dispersal ability inform differences between species in terms of their evolutionary histories and modern patterns of distribution. Our results highlight the complexity of pressures shaping the evolution of both wing shape and geographic range in tropical birds, and evince the need for further study at lineage-specific scales to better understand the processes driving the evolutionary and biogeographic history of these birds.}

\normaltalk{Assortative Mating by Plumage Coloration Genes in an Incipient Avian Radiation}{Sheela P Turbek, Melanie Browne, Adri\'{a}n S Di Giacomo, Cecilia Kopuchian, Rebecca J Safran, Scott A Taylor, Leonardo Campagna}{Post-zygotic isolation often takes longer to evolve than the time to speciation of many taxa, and pre-mating isolation has therefore been proposed as a powerful reproductive barrier maintaining sympatric species early in divergence. However, the ease with which assortative mating can break down has called into question the importance of pre-mating isolation during incipient speciation. We combined whole-genome and double digest restriction-site associated DNA (ddRAD) data with fine-scale behavioral analyses from two years of detailed field study to quantify assortative mating between two sympatric species of capuchino seedeaters (Sporophila hypoxantha and Sporophila iberaensis). Capuchino seedeaters comprise a recent, rapid avian radiation characterized by striking differences in male plumage coloration and song despite remarkably little ecological or genomic divergence. We found that the two species exhibit an extremely low level of background differentiation (genome-wide Fst = 0.006) that is punctuated by three narrow regions of elevated genomic divergence. These peaks, ranging from 30-50 kb long, encompass 11 genes, two of which are known to be involved in the regulation of coloration. Though S. hypoxantha and S. iberaensis lack temporal or spatial barriers to reproduction, and the rate of extra-pair paternity was extremely high, we documented complete assortative mating with regard to both social and extra-pair mates by the genomic regions that underlie differences in male plumage patterning. Given that females of the two species are phenotypically indistinguishable, our findings provide strong evidence that pre-mating isolation related to male sexual signaling may be sufficient to maintain species boundaries extremely early in divergence.}

\normaltalk{Introduced and Naturalized Parrots in the USA: Arrival, Geographic Distribution, and Current Status}{Jennifer J Uehling}{Naturalized parrots are common across the USA, from regions that are cold in the winter to regions that are warm year-round. Localities with diverse lists of established parrot species are typically warm year-round (California, Florida, and Texas), but parrots are also commonly sighted in more temperate localities including Chicago and New York City. The Monk Parakeet (Myiopsitta monachus) is the most common and best-studied of all introduced parrot species in the USA; other species have been less-studied. Here, I explore the status and distribution of all non-native parrot species in the contiguous USA and discuss how they became established in the wild. From 2002 to 2016, 56 parrot species were sighted outside of captivity in the USA, of which 25 species are confirmed to be breeding. These parrots originally arrived via the pet trade. However, there is not a strong relationship between the number of individuals of each species imported and the number of localities where each species has been sighted outside of captivity, and I explore possible explanations for the lack of such a relationship. Finally, I explore geographic regions -- southern California, Texas, and Florida -- where parrots have been especially successful at establishment, and discuss the public's response to parrots in these localities. Non-native parrots often captivate the general public in the USA but prove worrisome to invasive species biologists, and as such their management as introduced or invasive species can be controversial.}

\normaltalk{Mainstreaming Bird Conservation into Sustainable Development}{Amy Upgren, Andres Anchondo}{Bird conservation and sustainable development advance complementary goals. Birds provide economic and human health benefits, and are useful indicators of habitat restoration success and biodiversity value, all of which are objectives of sustainable development. We see clear advantages to amplifying the integration of bird conservation into sustainable development through collaborative efforts with multilateral agencies, development banks, and impact investors. In this talk, we will discuss opportunities for expanding the mainstreaming of bird conservation into broader sustainable development and will highlight several key examples of successful integration.}

\normaltalk{Alliance for Zero Extinction (AZE) Sites: Progress in Bird Conservation in the Subset of KBAs on the Brink of Extinction}{Amy Upgren}{Alliance for Zero Extinction (AZE) sites are a subset of KBAs that hold the most threatened species in the world, those that are Critically Endangered (CR) or Endangered (EN) on the IUCN Red List and exist at only one site in the world. These 853 sites globally cover a very small percentage of the Earth's surface, buy are disproportionally important for the conservation of biodiversity and the prevention of extinctions. In this talk, advances and challenges in conserving the AZE sites identified for extremely threatened birds will be highlighted, and opportunities to strengthen their conservation will be discussed.}

\normaltalk{Impacts of Hurricane Maria on Tropical Forests: Is this the New Normal?}{Maria Uriarte, Jill Thompson, Jess K Zimmerman, Jazlynn Hall}{Projected increases in cyclonic storm intensity under a warming climate will have profound effects on the structure, composition and function of tropical forests. Forecasting storm impacts on these ecosystems requires consideration of risk factors associated with storm meteorology, landscape structure, and forest attributes. Identification of the tree species characteristics that influence vulnerability to storm damage is also key to understanding the impacts of an intensified storm regime on forests. Here we first evaluate risk factors associated with damage severity caused by Hurricane Mar\'{i}a across Puerto Rican forests. Storm-related rainfall was a stronger predictor of forest damage than maximum wind speeds. Soil water storage capacity was also an important risk factor, corroborating the influence of rainfall on forest damage. We also compare tree damage data collected in the same forest in Puerto Rico after Hurricanes Hugo (1989, category 3), Georges (1998, category 3), and Mar\'{i}a (2017, category 4). Mar\'{i}a killed twice as many trees as Hugo, and for all but two species, broke 2- to 12-fold more stems than the other two storms. Species with low density wood were particularly vulnerable to uprooting and breakage for Hugo but not Mar\'{i}a. Expected increases of 20\% in hurricane-associated rainfall in the North Atlantic highlight the need to consider how such shifts, together with high speed winds, will affect terrestrial ecosystems. A better understanding of risk factors associated with tree species susceptibility to severe storms is also key to predicting the future of forest ecosystems under climate warming.}

\normaltalk{Female Screaming Cowbirds Avoid Repeatedly Parasitizing Host Nests: Genetic Evidence from a Specialist Brood Parasite}{Cynthia A Ursino, Meghan Strong, Juan C Reboreda, Christina P Riehl}{Avian brood parasites lay their eggs in the nests of other species, leaving the hosts to provide parental care to the parasitic offspring. The bookkeeping hypothesis predicts that female parasites should avoid laying multiple eggs in the same host nest. However, selection against repeated parasitism should be weaker when host nests are limited, or when hosts are able to rear more than one parasitic nestling. Here we use microsatellite genotyping of parasitic eggs to test whether female screaming cowbirds (Molothrus rufoaxillaris) avoid repeatedly parasitizing nests of their primary host, the grayish baywing (Agelaoides badius), in Argentina. Parasitism rates were extremely high: 96.5\% of 57 host clutches with an average of 5.7 cowbird eggs each. Individual females rarely laid more than one egg in the same host clutch (2 of 57 clutches). Females were much more likely to lay subsequent eggs in different host nests than to return to the same host nest. We found no evidence for kin structure among female cowbirds parasitizing the same host nest, since these were no more closely related than chance would predict. These results suggest that female screaming cowbirds frequently lay eggs in host nests that have already been parasitized by unrelated females. However, they typically lay just one egg per host clutch, even though grayish baywings are capable of rearing several nestlings. Since screaming cowbird laying is poorly synchronized with this host, avoidance of repeat parasitism may be adaptive if it allows females to spread the risk of failure among multiple host nests.     Keywords: multiple parasitism, repeat parasitism, virulence, microsatellite}

\normaltalk{Immunosuppression and the Cost of Reproduction in Male Birds: a Meta-Analytic Approach}{Jos\'{e} O Valdebenito, Naerhulan Halimubieke, Ád\'{a}m Z Lendvai, Jordi Figuerola, Tam\'{a}s Sz\'{e}kely}{The specific roles of males and females during reproduction determine sex differences in life history traits, behavior, and physiology that often translate into sex-specific risks of mortality. Sexual differences in the immune system in vertebrates has been mostly found in mammals, where sex differences seem to appear mainly mediated by the overall immuno-modulating effect of the sex hormones. In birds, recent studies found no sexual differences in immunity, although different studies have reported differences between males and females in same species. The breeding period in animals induces major physiological and behavioural changes that often relate to changes in immunity, even at sex-specific level. However, it is unknown if reproduction could modulate immunity across species and generate sex-specific immune changes in birds. Here we used a meta-analytic approach to test the hypotheses that immunity (white blood cell counts, the phytohaemagglutinin, bacteria-killing ability and hemolysis-hemagglutination assay) differs between sexes and during the breeding and non-breeding period across adult wild birds. We found that, overall, none of the immune variables tested showed significant sex differences, but heterophils, lymphocytes and the H/L ratio levels showed marked differences during the breeding period. Analyses conducted separately on males and females showed that males had higher levels of H/L ratio than females during the breeding period compared to the non-breeding period. Overall, our findings suggest that the largest sexual differences in the immune system occur during the breeding period, and that males, apparently, have higher levels of immunosuppression than females during this period.}

\normaltalk{Comparison of Organochlorine and PAHs Residues in Terns Eggs from Two Natural Protected Areas in the Gulf of Mexico}{Adriana V Moncada, Jaime Rendon}{Tropical ecosystems have different physical, chemical and biological characteristics than those in temperate climates. It is crucial to know the rate at which pollutants accumulate or are transported within the trophic network in order to know its availability. The main source of Persistent Organic Pollutants (POPs) exposure is food, especially for high trophic animals. Polycyclic aromatic hydrocarbons (PAHs) are detectable in eggs, which indicates that they bioaccumulate. We measured concentrations of POPs and PAHs in Alacranes Reef, and Terminos Lagoon (2010-2011), in 20 eggs of least and sooty tern. Higher levels of $<$U+2211$>$HCHs were found in Terminos in 2010 and in Alacranes in 2011. $<$U+2211$>$Dienes and $<$U+2211$>$DDTs were higher in 2011 in both species. Endosulfan higher in Terminos than in Alacranes. In 2011, the levels of $<$U+2211$>$DDT higher than in 2010 in the Least. The most abundant POP in 2010 was $<$U+2211$>$dienes for the Least and $<$U+2211$>$DDT for Sooties. In 2011, $<$U+2211$>$DDT was more abundant in Least and in the Sooty was $<$U+2211$>$heptachlor. Apparently the origin of PAHs was pyrogenic in 2010 in both species and only in Sooties in 2011. Concentrations of pp-DDE similar to those of masked boobies in the Pacific Ocean. In 2011, least had levels of pp-DDE reported as harmful for reproduction. In 2011, sooty had 370 times more pp-DDE than in 2010. The PAHs in both species were in the ranks for terns in Europe. There was no evidence that hydrocarbons from the BP spill in 2010 affected colonies studied.}

\normaltalk{Versatile Migratory Strategies and Evolutionary Insights Revealed by Tracks of Wild Eurasian Blackcaps}{Benjamin M Van Doren, Kira Delmore, Miriam Liedvogel}{Migration is ubiquitous in the animal kingdom and may play a key role in promoting reproductive isolation and underpinning responses to environmental change. Migratory divides are contact zones between populations with different migratory phenotypes and ideal natural laboratories for studying the evolution of migration. The Eurasian blackcap (Sylvia atricapilla) is a songbird that exhibits a migratory divide in Central Europe between populations that migrate southwest (SW) and southeast (SE) in autumn and has recently established a wintering population in Britain. We tracked 106 annual migrations of 98 blackcaps captured across their range to characterize both the migratory divide and novel wintering strategy. Blackcaps to the west and east of the divide used predominantly SW and SE directions, respectively, but close to the contact zone many individuals took intermediate (S) routes. At 14.0ºE, we documented a sharp transition (22 km) in migratory direction from SW to SE, implying a strong selection gradient across the divide. Blackcaps wintering in Britain took northwesterly migration routes from continental European breeding grounds. They originated from a surprisingly extensive area, spanning 2000 km of the breeding range. British winterers bred in sympatry with SW-bound migrants but arrived 10 days earlier on the breeding grounds, suggesting some potential for assortative mating by timing. Overall, our data reveal complex variation in songbird migration and suggest that selection can maintain variation in migration direction across short distances while enabling the spread of a novel strategy across a wide range.}

\normaltalk{Temporally Dynamic Resource Use of Overwintering Waterbirds During an Agricultural Cycle in Southern Spain}{Charles B van rees, David Aragon\'{e}s, Willem Bouten, Chris Thaxter, Viola H Ross-Smith, Eric Steinen, Javier Bustamante, Andy J Green}{The anthropogenic modification of landscapes is a ubiquitous and prominent driver of global change and biodiversity loss. Research on the space use and behavior of organisms tied to human-modified landscapes yields insights on potential human-wildlife interactions of ecological and societal importance in the Anthropocene. Waterbirds, being highly mobile and important connectors of wetland habitats, are an excellent focal taxon for research on the ecology of human-dominated landscapes. We combined GPS tracking, earth observation, accelerometry and behavioral observations to study the space use of Lesser black-backed gulls (Larus fuscus) overwintering in a mixed rice field landscape in Andalusia, Southern Spain. These gulls are flight and diet generalists, and appear to be favored by anthropogenic environments, exhibiting large population increases in recent decades. Their fine-scale space use has strong implications for the dispersal of potentially invasive plants and animals as well as nutrients and toxic heavy metals in areas of agricultural cultivation for human consumption. Analysis of over 53,000 GPS fixes and 14 classified remotely-sensed images showed dynamic space use within the rice field landscape based on the changing resources available throughout the rice harvest cycle. Late-season commuting between local recycling centers and flooded rice fields may introduce substantial heavy metal loading to fields where rice is grown.}

\normaltalk{Spatial and Temporal Variation in Survival and Connectivity of North American Pelicans}{Aimee M Van Tatenhove, Clark S Rushing}{Sensitive to human disturbance and contaminants, American white pelicans (Pelecanus erythrorhynchos) and brown pelicans (Pelecanus occidentalis) are large migratory waterbirds that have been the focus of intensive conservation efforts since the 1970s. Despite conservation efforts and population monitoring spanning decades, range-wide trends in survival and migratory connectivity of both species are poorly understood. This critical information gap limits our understanding of how threats, including acutely toxic pesticides and rapid anthropogenic changes to breeding, foraging, and wintering habitats, have incongruously impacted populations across their large North American ranges. Using band and wing tag resight and recovery datasets from the USGS Bird Banding Laboratory (BBL), we fit multi-state capture-recapture models to estimate connectivity and spatial and temporal variation in pelican survival over the past sixty years. We estimated nuisance parameters (band recovery and resight probabilities) using data spanning strata, improving estimates of species-specific parameters of interest. Movement between eastern and western populations was low for brown pelicans, likely due to their strictly coastal distribution, while white pelicans experienced moderate connectivity between east and west. Movement between the United States and Mexico/Central America, a wintering area for both species, was lower than expected in some regions. Survival for white pelicans increased through time, mirroring pesticide bans and wetland protections across the continent, while brown pelican survival decreased. These results provide important context for understanding past changes in North American pelican survival and abundance and for managing populations of migratory species in the face of future threats across their annual cycle.}

\normaltalk{Global Drivers of Variation in Nest Size and Parental Effort in Passerines}{Karina Vanadzina, Catherine Sheard, Kevin Laland}{What determines the size of a nest that a bird species typically builds? Is it driven mainly by temperature? How does it relate to the total effort parents invest in their young? Our phylogenetic comparative study is the first to investigate the global drivers of variation in nest size and its link to parental care using a comprehensive dataset of cup nest measurements from nearly 1,000 species of passerines. Field studies in Canada and Britain have shown that northern populations build larger nests with thicker walls compared to birds that nest in the south. An increase in nest size could also be driven by post-copulatory sexual selection as size can act as a signal of parental quality and cue differential investment in other aspects of parental care. An early comparative study in Palaearctic passerines found that nests nearly doubled in size in species where both parents built compared to species with uniparental building. It is unknown, however, whether this pattern holds true globally or if the effect is independent of the relationship between nest size and temperature. Here we use Bayesian phylogenetic mixed models to analyse the relationship between nest size, the total time spent caring for eggs and nestlings and environmental variables such as temperature and precipitation. The data was obtained from ornithological handbooks and from measurements of specimens at the Natural History Museum at Tring (UK), which hosts more than 3,000 passerine nests from all biogeographical realms.}

\normaltalk{Nature or Nurture? Dissecting Environmental and Genetic Effects in Shaping Gut Microbiota During Avian Development}{Paula Vargas-Pellicer, Sarah Knowles, Cristina Banks-Leite, Jarrod Hadfield}{The establishment of gut microbiota in young vertebrates appears to be dependent on environmental and endogenous factors. Birds, however, are not exposed to microbes at birth in the same way mammals are. Consequently, the roles that environmental factors and vertical transmission play are likely to influence the microbiome in distinct ways. Research on avian populations has suggested that gut microbiota structure is influenced by both genetic and non-genetic factors, but the comparative influence of such factors shaping bacteria communities are yet to be determined. To fill this gap, we performed cross-fostering experiments during four consecutive years (2014-2017) in a wild population of blue tits (Cyanistes caeruleus) in Dalmeny Estate, Scotland. Eggs were swapped between nests on the day of laying, and faecal samples collected from 468 marked nestlings to investigate the relative influence of nest environment and genetic effects. We find that rearing environment had a much stronger influence on the gut microbiota than kinship, as a nestling's microbiota was far more similar to that of unrelated nestmates than to that of siblings reared elsewhere. We find no convincing evidence for genetic effects on the blue tit gut microbiota: gut microbiota composition showed similar differences among full siblings, half siblings and unrelated nestlings, irrespective of whether they were raised together or apart. Overall, this evidence supports the conclusion that nest environment affects microbiota composition more than genetics. Our study emphasizes the relevance of external ecological mechanisms that shape gut microbial communities in altricial birds.}

\normaltalk{Private Finance Mechanisms and Policy Innovations to Restore Degraded Landscapes for Migratory Birds in Latin America}{Rene Zamora Cristales, Walter Vergara}{Climate Change represents a major threat to economic and human development especially in emerging economies around the world. Poverty and inequality have the potential to increase as a result of extreme events such as hurricanes or severe droughts. The effects of climate change are not limited to one sector but could affect the performance and cost of different supply chains around the world (Environmental Agency, 2013). In Latin America and the Caribbean Region (LAC), a significant amount of greenhouse gas (GHG) emissions are produced from land-use changes from forest to agriculture Additionally different agricultural practices such as cattle ranching have caused land degradation Landscape Restoration in Latin America is becoming an increasable feasible strategy to mitigate and adapt to climate change effects. For Latin American Countries, restoring landscapes is key to increase carbon stocks, water, and food security, and reduce vulnerabilities to extreme weather events (e.g. landslides, floodings, etc.) In the Latin American context restoration includes interventions such as reforestation, agroforestry, and silvopasture systems. Many governments in the region have committed to restoring more than 50 million hectares to the Bonn Challenge, Initiative 20x20 and the New York Declaration on Forest. However, there is a lack of knowledge of policy-makers, managers and landowners on where, when how to implement restoration activities in the landscape to improve sustainability including restoring habitat for migratory birds. Discrete interventions in the landscape can help to restore habitat. In this presentation, we will describe private impact investments in Latin America that are helping to restore habitat and connectivity of degraded lands in Latin America.  We will also discuss how policy instruments such as incentives can be used to promote nature-based solutions to climate change. Finally, we will discuss how monitoring of impacts at the landscape level can help to estimate bird habitat rehabilitation for migratory species.}

\normaltalk{Telomeres and the Courtship Behavior of a Lek-Breeding Passerine, the Wire-Tailed Manakin.}{Ben J Vernasco, Roslyn Dakin, Ariana D Majer, Mark F Haussmann, T B Ryder, Ignacio T Moore}{Telomeres are thought to reflect an individual's biological age and are increasingly being used by biologists to determine the factors that influence patterns of aging. Few studies have, however, examined how telomeres are related to reproductive behaviors, despite the fact that individuals are predicted to modulate their behavior based on age and residual reproductive value. Determining how telomeres relate to reproductive behaviors can therefore provide biologists with a molecular biomarker that can be used to understand variation in reproductive success as well as provide a novel perspective on classic hypotheses describing how reproductive behaviors should change based on an individual's age. Here, we longitudinally measured the telomere lengths and reproductive behaviors of known-age, male wire-tailed manakins over the course of 3+ years. We then quantified the repeatability of telomere lengths as well as how a male's telomere length relates to his reproductive behavior. Wire-tailed manakins are a lekking passerine and males form cooperative display coalitions wherein multiple males perform coordinated courtship displays, but only one male within the coalition reproduces. Male wire-tailed manakins must ascend a social hierarchy as floater males and only gain reproductive opportunities upon becoming a territory-holder. More cooperative floater males are quicker to become a territory-holder and more cooperative territory-holders sire more offspring. Our results show that telomere lengths were highly repeatable and that males with shorter telomeres exhibited higher reproductive investments. We discuss our results in the context of both the pace-of-life hypothesis and the terminal investment hypothesis.}

\normaltalk{Introduction to Avian Microbiome Research}{Elin Videvall}{There has been a recent surge of interest in animal-associated microbiomes. Given the many important roles of the microbiota, this trend is expected. However, microbiome research in birds is still lagging behind that of other vertebrates, despite the enormous potential birds have to provide answers to important questions regarding the ecology and evolution of host microbiomes. In this overview talk, I will provide an introduction to avian microbiome research. What is a microbiome? And why is it important? I will provide examples on how we can sample different microbiomes of birds and highlight some of the fascinating research being conducted to explore these microbial communities. I will finish with a brief outlook of the future of bird microbiome research.}

\normaltalk{Wild Scarlet Macaw Foster Parents as a Tool for Aiding Population Recovery}{Gabriela Vigo Trauco, Donald B }{The use of foster parents in avian population management is a technique with great potential to aid in the recovery of highly endangered species. However, few studies have investigated how to accomplish this successfully. Our research shows that Scarlet Macaws in southeastern Peru hatch 2-4 chicks per nest but just 1.4 fledge. Here about 24\% of all hatched chicks die of starvation and starvation is the most common cause of chick death. Parents always raise the first chick that hatches, but 27 \% of second chicks, 81\% all of third and 100\% of all fourth chicks are left to starve to death by their parents. Our goal was to develop and test new techniques to increase survival of wild Scarlet Macaw chicks by reducing chick starvation. We hypothesized that we could pull chicks at risk of starvation, raise them in captivity to about 18 days of age then move them to nests with only one chick to increase their chances of survival. Our results show that all relocated macaw chicks were successfully accepted by their foster parents (N=28 chicks, 3 consecutive breeding seasons) and 89\% of the translocated chicks fledged successfully. Overall, we increased fledging success per available nest from 17\% (1999 -- 2016 average) to 25\% (2017 to 2019) and decreased chick death by starvation from 19\% to 4\%. These findings show that the use of foster parents in the wild is a promising management tool to aid wild parrot population recovery in areas with low reproductive success.}

\normaltalk{Raptor Counts During the Autumn Migration in Cuba and Hispaniola (2018 and 2019)}{Nicasio Vina-Davila, Freddy Rodr\'{i}guez-Santana, Yasit Segovia-Vega, Carmen Plasencia-Leon, Jose R Fuentes-Caballero, Leydis S\'{a}nchez-Zaldivar, Flavia Tamayo-Escobedo}{The Greater Antilles is very important for raptor migration. Previous monitoring at points counts in Cuba and satellite telemetry studies showed that the highest global concentrations of migrating Ospreys (Pandion haliaetus) occur in the area. However, the autumnal migration of the Osprey in the area has not been sufficiently studied, especially at the crossing from Cuba to Hispaniola and at the crossing of the Caribbean Sea. We conducted exploratory counts in 27 locations, four in Cuba and 23 in Hispaniola, in September and October of 2018 and 2019; with the objective of identifying the main sites of concentration and crossing points of Ospreys before crossing the Pass of the Winds and the Caribbean Sea. Between 2018 and 2019, 28,000 birds of prey of 7 species were counted in 2868 hours of observation. The Osprey was the most abundant raptor with 25730 individuals and the main concentration and crossing sites of this species were identified in the Passage of the Winds at Punta Caleta, Guantanamo, Cuba and in the Caribbean Sea at Cabo Peque\~{n}o in Jaragua, Dominican Republic with 47,4 and 34,4 eagles per hour of observation respectively. The counts at both sites offer a unique opportunity to better understand the migration of this species and make reliable estimates of the North American population as well as to better design conservation strategies for the insular Caribbean, which is the site of the world's largest concentration of Ospreys during migration.}

\normaltalk{Hurricane Effects on Birds in Eastern Cuba.}{Nicasio Vina-Davila, Freddy Rodriguez-Santana, Carmen Plasencia-Le\'{o}n, Yasit Segovia-Vega}{The results of long-term studies and monitoring in eastern Cuba have allowed the assessment of the impact of hurricanes George, Dennis, Ike, Sandy, Mattews and Irma on biodiversity in general and avifauna in particular. The impact of these hurricanes on endemic and migratory birds has also been studied.  In four of the hurricane studies, pre- and post-hurricane comparisons have been made with banding studies at Juan Crist\'{o}bal Gundlach permant banding station at the Siboney-Jutic\'{i} Ecological Reserve in eastern Cuba. These studies have enabled identification of variables which, independently or by their interactions, determine the type and magnitude of the hurricane impact on birds.  Hurricane traits of importance to biodiversity include the trajectory, speed of movement, intensity of winds, and the intensity and duration of rainfall.  Ecosystem characteristics which influenced hurricane impacts include the slope of mountain areas and their orientation, geology, type of soil, distance to the coastline, as well as the coastal zone elevation and its extent.  Hurricane impacts can be limited by the conservation status of habitats and distributions of species.  Consideration of these elements or factors are required to undertake management actions that mitigate the impact of hurricanes on birds.    Los resultados de estudios y monitoreos a largo plazo han permitido la valoraci\'{o}n del impacto de los huracanes George, Dennis, Ike, Sandy, Mattews e Irma sobre la biodiversidad en general y la ornitofauna en particular en el oriente de Cuba. Se analiza el impacto sobre aves end\'{e}micas y migratorias.  En cuatro de los casos se realizan comparaciones con estudios previos al paso del hurac\'{a}n, siendo una fuente importante de informaci\'{o}n la data de la Estaci\'{o}n permanente de anillamiento Juan Crist\'{o}bal Gundlach en la Reserva ecol\'{o}gica Siboney- Jutic\'{i}. El estudio de estos huracanes permiti\'{o} identificar variables que de forma independiente o como resultado de su interacci\'{o}n condicionan el tipo y magnitud del impacto sobre las aves. Se reconocen como variables del hurac\'{a}n de importancia en su impacto sobre biodiversidad a  la trayectoria, la velocidad de desplazamiento, la intensidad de los vientos y la intensidad y duraci\'{o}n de las precipitaciones. Las caracter\'{i}sticas de los ecosistemas que m\'{a}s influyeron en las caracter\'{i}sticas del impacto son la pendiente de zonas monta\~{n}as y su orientaci\'{o}n, la geolog\'{i}a, el tipo de suelos, la distancia a la l\'{i}nea de costa, as\'{i} como la altura de la  zona costera y su profundidad. Tambi\'{e}n son elementos condicionantes del impacto el estado de conservaci\'{o}n del \'{a}rea y la distribuci\'{o}n de las especies. Los elementos aportados son v\'{a}lidos para lograr acciones de manejo que mitiguen el impacto de los huracanes sobre las aves.}

\normaltalk{Coevolutionary History of Coots, Feather Lice, and Bacterial Endosymbionts}{Stephany Virrueta Herrera, Sarah A Sonsthagen, Robert E Wilson, Kevin P Johnson}{Coots (Fulica) are a group of water birds (Gruiformes:Rallidae) comprised of ten species. Like most birds, coots are host to a wide variety of parasites. We obtained samples of one group of ectoparasites, feather lice in the genus Rallicola, from all extant coot species. Achieving this complete species sampling of feather lice presents a unique opportunity to compare the complete evolutionary histories at three scales: birds, parasitic lice, and bacterial endosymbionts of the lice. A phylogeny for the coots was estimated from targeted sequencing of both mitochondrial and nuclear genes. We next estimated the phylogeny coot feather lice (Rallicola) using data from whole genome sequencing. A phylogenomic data set from 1,107 targeted gene assemblies was analyzed using concatenated and coalescent methods and compared that to coot phylogenetic tree. While we found some evidence of cospeciation, host-switching of these parasites has also occurred. Finally, we constructed a phylogeny of the bacterial symbionts in the coot lice using the same genome sequencing reads, which contain genomic reads from these bacteria. Feather lice typically have a single symbiont species which likely provides nutritional supplementation to the feather diet of the lice. We found evidence that different louse species harbored different genera of bacteria, suggesting endosymbiont replacement over the evolutionary history of these parasitic lice.}

\normaltalk{Resistance and Resilience of Forest Bird Populations Subjected to Multiple Disturbances in the Luquillo Mountains, Puerto Rico}{Robert B Waide}{Avian abundance and community composition are affected by hurricanes, but the strength of the effects depends on the initial environmental conditions and the characteristics of each hurricane. Effects are more severe for intense storms and mature forests, while weaker storms and successional forests show lesser responses. Thus, response of bird populations to a hurricane will be determined in part by legacies from previous storms. Moreover, other kinds of disturbance (e.g., climate change, droughts, land use/land cover) may influence the resistance and resilience of bird populations to hurricanes. In a dynamic landscape such as the Luquillo Mountains, interactions among disturbances complicate efforts to predict long-term trends in avian populations. Since the establishment of the Luquillo Long Term Ecological Research site in 1988, four major hurricanes (Hugo, Georges, and Irma/Maria) and two droughts have affected bird populations in a landscape with significant legacies from historical logging and agriculture. In 1989, Hurricane Hugo and the following three-month drought negatively affected abundance of frugivores, nectarivores, and granivores, but numbers rebounded with a year of the hurricane. Foraging behavior and diet of insectivores changed after Hugo, but abundance was resistant to the storm. The response to Georges, a less intense storm, was similar but more muted. The combination of Irma and Maria, occurring within two weeks in 2017, again affected granivores and nectarivores most severely. Species richness and abundance of birds are resilient to hurricanes in Puerto Rico, probably because of the long history of evolutionary adaptation to such disturbances.}

\normaltalk{Hot Take: Temperature Influences Flocking Propensity and Species Interactions in a Subtropical Mixed-Species Flocking System.}{Harrison H Jones, Mitchell J Walters, Scott K Robinson}{Increased participation in mixed-species flocks is one hypothesized behavioral tactic used by forest birds to adjust to cold temperatures in winter. This phenomenon has been documented in temperate North America, but not at subtropical latitudes where temperatures are less extreme. To understand how temperature structures the size, richness, flocking propensity, and species interactions of subtropical mixed-species flocks, we performed paired mixed-species flock and point count surveys in upland hardwood forests in North-central Florida, USA. We described the composition of 92 mixed-species flocks across a ~20°C range of daytime temperatures, using linear mixed models and network analyses to test for effects of temperature. We found that both flock species richness and size were significantly smaller at higher temperatures while controlling for effects of time-of-day, canopy height, and local site effects. By comparing metrics from seven social networks corresponding to temperature ranges, we found that strength of flocking interactions was lower and that species were significantly less likely to co-occur with similarly foraging species at temperatures above 20°C. In addition, 93\% of flocking species a) joined flocks in all seven networks and b) showed decreases in both within-flock abundance and flocking propensity with increasing temperature. Dissimilarity of social networks was consequently best explained by changes in species co-occurrence patterns rather than species turnover within flocks. In contrast to boreal flocking systems, changes to flocking behavior were prompted by unseasonably warm weather rather than cold spells, suggesting that warming winter temperatures could lead to breakdown of social behaviors in our system.}

\normaltalk{Development of a Bird Damage Management System for Agriculture using UAVs Incorporated with Bird Psychology}{Zihao Wang, Andrea S Griffin, Andrew Lucas, Kc Wong}{A novel bird damage management system using Unmanned Autonomous Vehicles (UAVs or drones) incorporated with bird psychology is being developed. Bird damage to agriculture is a challenging global problem with no cost-effective solution to date. The emergence of small and affordable commercial UAVs brings new opportunities to the challenge. UAVs bring disturbance to wildlife birds and this effect is well known by ornithologists; the actual effect of the UAVs on wildlife bird were recently studied and quantified. Furthermore, recent experiments conducted in vineyards indicated that the UAV system had the potential to be as effective as netting. This known effect is now being capitalised to develop a cost-effective bird damage management system for agriculture. A novel implementation of bird psychology is utilised to enhance the UAVs' ability to deter birds. To reduce the operating cost of the system, sensor technology and trajectory planning algorithm were developed and tested to enable autonomous operation. Preliminary results from the experiments showed that the UAV system was effective at deterring pest birds, and it was able to detect and track pest birds autonomously.}

\normaltalk{Attracting Conspecifics and Manipulating Interspecific Interactions to Achieve Conservation Goals.}{Michael P Ward, Valerie Buxton, Janice Enos, Sean MacDonald, Nick Anich, Jinelle Sperry}{Humans alter ecosystems in ways that can change how species interact and behave, presenting challenges to conservation efforts. Behavioral manipulation with social cues may be a valuable management tool to achieve conservation goals. We will highlight two separate research studies conducted in Hawaii and Wisconsin where we experimentally manipulated social cues for conservation purposes. On Oahu, Hawaii, native frugivorous birds are extinct and native plant recruitment is depressed due to poor seed dispersal. To attract introduced bird species to native plants and facilitate seed dispersal, we broadcast vocalizations of these introduced birds at target native plant sites. We found that introduced birds were attracted to social cues and subsequently consumed native seeds, thus facilitating restoration efforts to increase seed dispersal and native plant recruitment. In Wisconsin, Kirtland's Warblers (Setophaga kirtlandii) are state endangered and are found to consistently breed in only one county located centrally in the state, despite suitable jack pine forests occurring in northern counties. To expand the breeding distribution of warblers, we broadcast conspecific vocalizations at suitable but unoccupied sites in northern Wisconsin. We successfully established three new populations of Kirtland's Warblers at sites located $>$200 km from the edge of their geographic range. Although these two research studies differ in many ways, they both provide opportunities to use behavior in conservation. We will discuss the implications of these two studies and how similar techniques may be applied to pressing conservation issues (e.g. climate change) for many different bird species.}

\normaltalk{The Puerto Rican Sharp-Shinned Hawk: Management Activities for Species Recovery and Conservation}{Hana C Weaver, Russell Thorstrom, Melissa Murillo, Julio Gallardo, Jeffrey Grayum, Thomas Hudson}{In 2018, the recorded population of The Puerto Rican Sharp-shinned Hawk Accipiter striatus venator dwindled to a minimum of 19 individuals following the devastating effects of Hurricane Mar\'{i}a, a category 5 storm that struck Puerto Rico in September 2017. A conservation effort was initiated to increase hawk productivity by collecting clutches from wild pairs, propagation and release activities, and nest management. In 2018, eight eggs (n=3 clutches) were collected with six hatching. All captive-reared young were released into the wild by hacking and productivity was 2.0 young per managed nest. That year, four unmanaged nesting attempts yielded productivity of just 0.5 young. In 2019, we repeated activities conducted in the previous year and implemented additional nest management to control infestations of the parasitic nest fly, Philornis. In our second season, 12 eggs (n=4 clutches) were collected with nine hatching: productivity of managed nests was 2.2 young and unmanaged nests yielded a low productivity of 0.4 young. In combined years, nest failures were caused by predation of nestlings (n=2) and eggs (n=3), and parasitism by Philornis (n=2). Hacked young were fed, closely monitored and tracked via radio telemetry until reaching independence between 59-104 days of age. One captive-reared female from the 2018 season dispersed 26.1 km, to another isolated forest reserve, and was briefly paired with a territorial adult male before she ultimately returned to the hack site where she demonstrated parental care to the newly released 2019 young.  In 2020, we continue management techniques to ensure conservation of the species and strive to determine survivability and dispersal of both hacked and wild-fledged individuals.}

\normaltalk{Conservation Priority Birds and their Habitat Relationships in Iowa Forests}{Benjamin M West, Stephen J Dinsmore}{Recent avian population declines emphasize the need to obtain accurate population estimates and better understand the habitat characteristics associated with densities of at-risk species. Our study focused on Iowa Species of Greatest Conservation Need (SGCN) in three primarily-forested Bird Conservation Areas in south-central Iowa. We used point counts with generalized distance sampling to account for imperfect detection, visiting a grid of 493 points twice each breeding season between 2016 and 2019. We incorporated fifteen habitat covariates into our models, ranging in scale from landscape to site-level, to estimate relationships between bird densities and habitat. Three covariates were mid-story metrics derived from a novel method of digital image analysis. We detected a total of 25 breeding avian SGCN that used habitats ranging from early successional to mature forest, with 9,882 SGCN detections across all years. Acadian Flycatchers (mean d-hat = 0.20 birds/ha, SD = 0.27) were associated with minimally-disturbed forests, indicated their density relationships with distance to forest edge (+), landscape-scale forest cover (+), shrub density (-),  and grass cover (-). Field Sparrows (mean d-hat = 0.40 birds/ha, SD = 0.57), in contrast, were associated with more disturbed areas, indicated by their density relationships with grass cover (+), distance to forest edge (-) and landscape-scale forest cover (-). This study increases understanding of bird-habitat relationships in the Midwestern US and has potential to inform Iowa forest management. The SGCN bird community utilized a many habitats in our study area, indicating that diverse habitats are important for bird conservation in south-central Iowa.}

\normaltalk{Investigating Fiordland Penguin (Tawaki) Dietary Plasticity using Stable Isotopes.}{Jeffrey W White, Herman L Mays, Thomas Mattern, John B Hopkins III, Pablo Garcia-Borboroglu, Ursula Ellenberg, Philip J Seddon, David M Houston, Brooke Crowley}{Fiordland penguins (or tawaki; Eudyptes pachyrhynchus) depend on the abundant resources of the Tasman Sea and Southern Ocean to fuel reproduction and sustain them through the molting fast. Tawaki occupy a broad range of marine habitat types during the breeding season, suggesting high adaptability and potentially indicate dietary plasticity. To identify differences in foraging strategies among marine habitat types (pelagic, continental shelf, or fjord), we analyzed stable isotope ratios of carbon (13C/12C, expressed as d13C values) and nitrogen (15N/14N, expressed as d15N values) in penguin blood and feathers over 2 and 5 year periods respectively. This allowed a comparison of dietary preferences across the breeding range during incubation and among sites during the pre-molt dispersal. We found that both d13C and d15N values differed significantly between tissues, suggesting the diets were different during those two periods of the annual cycle. During incubation, d15N values were higher in the fjord than other colonies and d13C values differed significantly between years. We found that d13C values were significantly different between some colonies, sexes, and years during the pre-molt. d15N values differed only between years, reflecting potential differences in diet or in marine conditions. These data indicate that tawaki do exhibit a degree of dietary plasticity depending on season and marine conditions. Unlike many penguins, tawaki appear to have a stable population. Understanding the foraging requirements of this enigmatic species will help conservationists predict how tawaki may respond to a changing ocean.}

\normaltalk{Female Singers Contribute to Territory Defense Physically, but not Vocally, During Playback Experiments in an Australian Songbird}{Kristin K White, Christa Beckmann}{In  light of the Australasian descent of passerines and evidence that female song is ancestral in songbirds, research into the form and function of female song in Australia is important to our understanding of birdsong evolution. Grey fantails are monomorphic, serially monogamous flycatchers in which females sing but do not duet, and little is known regarding male and female use of vocalizations. Using dual-speaker playback, we broadcast male song, female song, and alternating songs of both sexes (a ``pair") to 24 pairs of nest-building fantails. We hypothesized birds would respond more strongly to songs of same-sex playback, and `pair' stimuli would elicit a stronger response than single-bird playback. Pairs responded strongly, giving significantly more songs and calls to treatments than controls; however, their vocal responses did not differ among treatments (male/female/pair). Physical responses (number of passes, approach distances) did not differ based on the sex of either the responder or simulated intruder. There was a significant interaction between sex and treatment in the number of calls produced; males called more during treatments, while female call rate did not vary between treatments and controls. Similarly, both sexes had low song rates during controls, but only males increased song rates during experimental treatments. Together, these responses indicate that songs of males and females may pose similar threats, and while both pair members contribute to physical territory defense, vocal defense is primarily male-driven. Alternatively, fantails may not distinguish between male and female song, rendering our ``pair" treatment no different than single bird treatments.}

\normaltalk{Differential hurricane effects on two wild Puerto Rican Parrot populations on Puerto Rico and implications for conservation management.}{Thomas H White}{Hurricanes are stochastic natural disturbances in the Caribbean to which native species have evolved various degrees of adaptation and resilience. However, extensive anthropogenic modifications throughout the Caribbean have altered ecological relationships between native species and hurricanes, rendering most species increasingly vulnerable to adverse effects of hurricanes. These adverse effects become particularly evident with populations reduced in size and geographic distribution. In September 2017, hurricane Maria made landfall in eastern Puerto Rico as a rare Category 5 storm. Two separate wild populations of Puerto Rican parrots (Amazona vittata) existed on Puerto Rico at the time of hurricane Maria: The El Yunque National Forest population in the northeastern part of the island (53-56 parrots), and the Rio Abajo Commonwealth Forest (130-145 parrots) in the northwestern region. Of these, the El Yunque wild population was almost completely lost – eventually losing 98\% of the population –  while the Rio Abajo population suffered an approximately 45\% reduction post-hurricane. Differential effects were primarily due to distinct geographic locations, hurricane timing and trajectory, and post-hurricane access to, and management of, the affected populations. Results of hurricane Maria have resulted in substantial modifications of previous management strategies for these populations, especially in the case of El Yunque where strategies are aimed at restoring the wild parrot population while bolstering resilience to future such atmospheric events.}

\normaltalk{Using Citizen-Science Data to Identify Declining or Recently Extinct Populations of Bahamian Birds}{David Wilcove, Ruby Bagwyn, Zuzana Burivalova, Kylen Bao}{Birds restricted to islands are especially vulnerable to extinction.  We analyzed $>$ 307,000 occurrence records of Bahamian birds from the citizen-science database eBird to identify island-specific populations of breeding species that have gone unrecorded since 2012 and which therefore may be imperiled or even extinct.  Focusing only on those islands with $>$500 observation hours logged in eBird, we identified 22 island populations, representing 15 species, that have gone unreported since at least 2012.  These potentially declining or extinct island populations should be of concern to conservationists and merit follow-up searches by birdwatchers.  Citizen-science data may offer a new way to identify imperiled populations in places that lack systematic, long-term bird surveys.}

\normaltalk{Impact of Plant Traits on Extent of Temporal Overlap Between Fruiting Plants and Introduced Avian Dispersers; Implications for Seed Dispersal Success in a Novel Ecosystem}{Rebecca Wilcox, Corey Tarwater}{Temporal overlap in key processes is essential for species interactions to occur, but mismatches in these processes (peak abundance, diet), leading to alterations or disruptions in species interactions, are increasing with growing anthropogenic change. This is true for seed dispersal, where introduced species are a primary concern, but little has been done to examine how temporal variability in disperser abundance or frugivory and plant fruiting phenology might cause mismatches to occur and seed dispersal to break down. However, identifying the traits that influence temporal matching pinpoints mechanisms that facilitate this process. Here we examine what plant traits (plant status, seed size, fruiting duration, fruit abundance) influence temporal overlap with a community of introduced birds and if overlap is linked to consumption by dispersers. Oahu, Hawaii is a highly novel ecosystem consisting of introduced dispersers and invasive and native plants. We leveraged phenology data from 29 plant species across seven different sites paired with site-level relative abundance data for the four primary seed dispersers to assess extent of overlap. We found that fruiting duration impacts the extent of overlap, while fruit abundance, seed size, and plant status did not. This indicates that consistent fruit availability, separate from abundance, is most important for matching to occur and this is true, regardless of plant status. We found that overlap is linked to local consumption of fruits by birds, suggesting that increased overlap could facilitate dispersal of certain plants. These results highlight mechanisms that maintain species interactions and shape communities in a changing world.}

\normaltalk{Keeping Friends Close: Brood Clustering and Spatiotemporal Predator Avoidance Determine Fledging Success in a Precocial Bird}{Luke R Wilde, Rose J Swift, Nathan R Senner}{An accurate assessment of risk depends largely on the information available to an individual. In mixed-species assemblages, the exchange of information among species can be beneficial for predator avoidance and survival. For instance, species incapable of nest defense commonly associate with protector species for safety and information snooping. These associations can be context-dependent, whereby the original benefits depend on the density of conspecifics  and the protector species can transition to predators. In southcentral Alaska, Hudsonian Godwits (Limosa haemastica) nest among aggressive, colonial Mew Gulls (Larus canus). For godwits, higher hatching success comes at the cost of early chick losses to gull predation. We investigated the influence of inter- and intraspecific interactions on chick survival to fledge and space use, and how these interactions changed with population density. We found that risk of mortality (i.e. hazard ratio [HR]) grew with increasing distance to conspecifics (HR=14.8\% per 100m), later hatch dates (10\% per day), and proximity to gulls (6\% per gull). Additionally, while godwit brood clustering was independent of density (-2.7±3.0; [ß±SE]) and developmental stage (difference=-0.08), broods only associated with the gull colony at low densities (19.3±4.5) and during late-development (d=0.32).   This suggests that brood clustering may be an important anti-predator behavior, especially when chicks are still vulnerable to gulls. Our results support the notion that the density of conspecifics can affect the spatial structure and benefits of community interactions. Additionally, these findings highlight the importance of understanding the community-level dynamics in the ecology and long-term viability of declining populations.}

\normaltalk{Global Movement Patterns of Short-Eared Owls}{Chad J Wilhite, Javier Cotin, Jessica L Idle, Laura R Luther, Melissa R Price}{Species may exhibit different spatial ecological strategies throughout their distribution as a mechanism to persist in spatially heterogeneous environments. Thus, studies of spatial ecology should incorporate movement data from across a species' distribution. Short-eared owls (Asio flammeus) are a globally distributed, and declining, species inhabiting continental and island systems spanning polar and tropical climates. In this study we aimed to describe global movement patterns of short-eared owls to infer potential conservation implications. We reviewed 20 publications and consulted with experts relating to short-eared owl movements throughout their distribution. We found that short-eared owls exhibit a spectrum of annual movement patterns ranging from extremely vagrant and migratory to non-migratory. Migratory and vagrant annual movement patterns are observed at high latitudes and are strongest in polar breeding populations. Non-migratory populations are found in tropical and subtropical island systems. We suggest that vagrancy in short-eared owls at high latitudes is likely driven by strong association with grassland systems where prey resources are patchily distributed and dynamic, especially at high latitudes. In contrast, we postulate that non-migratory populations have more flexible diets and more diverse habitat preferences, allowing them to persist in one location without migrating. Our findings provide testable hypotheses for future studies across the global distribution of this species, and suggest that conservation actions must be coordinated on large geographical scales for highly vagrant migratory populations.}

\normaltalk{Nest Success and Beach Restoration: A Comparison of Three Beach-Nesting Birds in Coastal Louisiana}{Kiah M Williams, Erik I Johnson, Caz M Taylor}{Louisiana is losing land at an alarming rate, and beach-nesting birds are on the front lines. The Coastal Protection and Restoration Authority has implemented beach restoration programs to protect communities from the detrimental effects of erosion and flooding as well as create new habitat for wildlife. In order to evaluate the effects of beach restoration on reproductive success, we followed the nests of 239 Wilson's Plover (Charadrius wilsonia), 1373 Least Tern (Sternula antillarum), and 144 Common Nighthawk (Chordeiles minor) to their fates in southeast (SE) and southwest (SW) Louisiana from 2016 to 2019. We calculated daily nest survival (DNS) estimates and compared fates between three restored and six unrestored sites. Nests with unknown fates were excluded from analyses.     For Wilson's Plover, in SW Louisiana during years with storms, DNS at restored sites was higher than at unrestored sites. For Least Tern, in SE Louisiana during years without storms, DNS at unrestored sites was higher than at restored sites. The low number of Least Tern nests at SE unrestored sites, coupled with high predation and low hatch success in SE restored sites supports this result. Lower DNS estimates for Common Nighthawk were still seen in both regions in 2017 and 2019, suggesting the same effect of storms on daily nest survival. These results suggest that beach restoration helps to mitigate the impacts of storms, but in providing more nesting habitat for birds it may also provide habitat for predators.}

\normaltalk{Elevational Niche-Shift Migration and Phenotypic Plasticity of an Austral Migrant Hummingbird}{Jessie L Williamson, Selina M Bauernfeind, Matthew J Baumann, Chauncey R Gadek, Natalia Ricote-Martinez, Peter P Marra, Francisco Bozinovic, Christopher C Witt}{Many of the world's bird species are elevational migrants, yet little attention has been paid to the magnitude of seasonal shift experienced between breeding and non-breeding areas. We describe the phenomenon of elevational niche-shift migration (ENSM) to characterize birds that undertake seasonal elevational shifts of 2,000 meters or more. This degree of shift requires profound flexibility in respiratory physiology to compensate with the ~20-55\% difference in seasonal partial pressure of oxygen and constitutes a seasonal transformation of the elevational dimension of the eco-climatic niche. We analyzed $>$4 million occurrence records to identify 105 taxa spanning 29 families and 10 orders that undergo ENSM. More than $>$90\% breed at high elevations and shift downwards to low-elevation non-breeding grounds. Few ENSM taxa breed at low elevations and shift upwards to high-elevation non-breeding grounds. One such species is the Andean Giant Hummingbird (Patagona gigas). We have tracked the austral migration of 8 Giant Hummingbirds since 2017 and measured blood physiology parameters from three populations to understand metabolic flexibility. Elevational migrant Giant Hummingbirds breeding at sea level in Chile migrate $>$8,000 km round trip to winter in the high Andes of Peru, shifting $>$3,000 m in elevation. They appear to compensate for a reduction in partial pressure of oxygen between seasons via flexible adjustments to blood-oxygen transport. Our work reveals the wintering grounds, migratory route, and stopover sites of austral migrant Giant Hummingbirds and highlights this species to provide perspectives on the physiology, ecology, and evolution of ENSM, an extreme form of elevational migration.}

\normaltalk{Survival by Degrees: The Vulnerabilities and Risks of Climate Change for North American Birds}{Chad Wilsey, Brooke Bateman, Lotem Taylor, Joanna Wu, Geoff LeBaron}{Frameworks of both vulnerability and risk are used to characterize potential future impacts of climate change on natural systems. We first assess how climate change mitigation affects the vulnerability of 604 North American birds by modeling projected future bird ranges and estimating potential range expansion and loss. Our results indicate that over two-thirds of North American birds are moderately or highly vulnerable to climate change under a 3.0°C global warming scenario. Of these climate-vulnerable species, 76\% would have reduced vulnerability and 38\% of those would be considered non-vulnerable if warming were stabilized at 1.5°C. Second, we estimate the risk to 544 birds in the conterminous United States from future climate change-related threats under the same unmitigated 3.0°C global warming scenario and mitigation-dependent 1.5°C scenario. Threats considered included sea level rise, lake level change, human land cover conversion, and extreme weather events. We identified potential impacts to species by overlaying future bird ranges with threats to calculate the proportion of species' ranges affected. Then, we mapped an index of risk. With unmitigated climate change, multiple coincident threats affected over 88\% of the conterminous United States, and 97\% of species could be affected by two or more climate-related threats. However, climate change mitigation would reduce risk to birds across over 90\% of the US. Thus, our work demonstrates how climate change is an existential threat to birds in the related frameworks of both vulnerability and risk, with no species avoiding at least one form of potential impact.}

\normaltalk{Bicknell's Thrush Habitat Use on Commercial Forests in Maine}{Kaitlyn L Wilson, Amber M Roth}{Bicknell's Thrush (Catharus bicknelli; BITH) is a rare, range-restricted habitat specialist occurring in balsam fir-dominated montane forests that are undergoing successional growth. The species historically occurs at elevations above 800 meters in the northeastern U.S., but if suitable habitat is available, the species can occur at lower elevations. The potential for suitable habitat at lower elevations exists in Maine because of changes in forest structure and composition due to forestry practices. The extent to which BITH use these low-elevation regenerating fir stands, however, remains unknown. By means of telemetry, resource selection functions (RSFs), and LiDAR, we aimed to understand the use and availability of breeding habitat for BITH in commercial forestlands in Maine. To accomplish this, individuals were tracked using VHF and GPS telemetry in 2018 and 2019 at two study landscapes: 1) working forest with extensive, recent timber harvesting, and 2) non-harvested forest for 70+ years. Using RSFs, we described habitat selection by BITH at the landscape and the home-range level. In both landscapes, BITH avoided increasing densities of large trees and showed a quadratic relationship with elevation at the landscape level. At the home-range level in the harvested landscape, individuals showed selection for softwoods and increasing densities of small trees, and a quadratic relationship with years since disturbance. At the home-range level in the non-harvested landscape, individuals avoided increasing canopy height, and showed a quadratic relationship with densities of small trees. We conclude that working forests provide important low-elevation habitat for BITH in Maine.}

\normaltalk{Understanding Arctic Tern Migration Strategies to Inform Conservation}{Joanna B Wong, Simeon Lisovski, Ray Alisauskas, Willow English, Marie-Andr\'{e}e Giroux, Autumn-Lynn Harrison, Dana Kellett, Nicolas Lecomte, Mark Maftei, Avery Nagy-MacArthur, Robert Ronconi, Paul Smith}{The Arctic tern (AT) is famous for its annual pole-to-pole migration. However, recent assessment suggests it is one of the most at-risk seabirds to climate change. Given its wide geographical range, knowledge of its movement ecology could provide insight on the remote environments which it inhabits, and help decipher the most at-risk migration pathways. Although AT breed in the Pacific, Atlantic and Arctic coasts of North America, few tracking studies have been conducted on American AT, with no tracking study to date on Canadian AT. Using geolocators, we developed a large-scale tracking study of AT breeding across North America. Throughout their annual migration, we tracked 53 terns across four Canadian colonies and one N. Alaskan colony. We compared routes taken by our colonies to all previously-tracked colonies of Greenland, Netherlands, Sweden, Maine (USA), and S. Alaska (USA). We report the majority of Arctic terns tracked globally are using one of three southbound migration routes: 1) Atlantic West Africa, 2) Atlantic Brazil, 3) Pacific coast, and returning using one of two northbound routes: 1) Mid-ocean Atlantic, 2) Mid-ocean Pacific. Several other trans-equatorial seabirds also share these corridors. However, our results show little overlap between these routes and internationally-recognized Waterbird Flyways. Our research reflects that current management does not favour conservation of AT, and is particularly unrepresentative of their migratory routes, despite highest mortality in this period. We suggest the protection of AT and many seabirds requires a shared effort from multiple countries, which may be facilitated with recognition of these routes.}

\normaltalk{The Importance of Street Trees to Urban Avifauna}{Eric M Wood, Sevan Esaian}{The Los Angeles metropolis (LA) is home to one of the most diverse urban forests on the planet. Approximately 20\% of the urban forest is composed of street trees, which are public resources planted adjacent to sidewalks, median strips of roads, or public-rights of way. Despite the representation of street trees in the LA urban forest, little is known about their distribution and value to wildlife, which presents a gap in our understanding of conservation in the urban ecosystem. Here, we quantified foraging behavior by migratory and resident bird species on native and exotic street trees across a socioeconomic gradient throughout LA. We found a higher diversity, density, and size of street trees in affluent areas than more disadvantaged communities, which in turn, attracted a greater number of foraging migratory birds, and to a lesser extent, non-migratory birds. Foraging migratory and non-migratory birds strongly preferred native street trees such as Coast Live Oak (Quercus agrifolia) and California Sycamore (Platanus racemosa) in greater proportion than they were available on the urban landscape. However, a handful of exotics tree species, including Chinese Elm (Ulmus parvifolia) and Southern Live Oak (Quercus virginiana) were also used in higher proportion than they were available, suggesting some exotic tree species provide valuable foraging resources for native birds. Our findings demonstrate the importance of native street trees, and to a lesser extent, exotic street trees, as important habitat for birds. Further, our results highlight the need for street-tree management in impoverished communities throughout LA.}

\normaltalk{Conservation Genetics of Endangered Parrots}{Timothy F Wright, Carlos I Campos, Michael A Russello}{Genetic studies provide important insights into many key issues for the conservation of threatened species. These include how natural patterns of gene flow and structure are altered by human activity, and how genetic composition changes due to bottlenecks or novel selection pressures. Parrots are one of the most endangered large groups of birds with over 45\% of species listed as near-threatened to critically endangered. Primary threats include habitat loss and capture for the pet trade. This latter threat creates an unusual situation in which threatened species are represented by both wild and captive populations, with captive populations sometimes larger and potentially more genetically diverse. Here we review general patterns from conservation studies of parrots and then focus on two endangered species, the blue-throated macaw of Bolivia, and the thick-billed parrot of Mexico. Both species are highly endangered in the wild but have sizeable captive populations. We used panels of 12 microsatellite loci to genotype 120 blue-throated macaws (BTM) and 112 thick-billed parrots (TBP) from both captive and wild populations to assess the degree of genetic variability within populations and the degree of structuring between them. We found similar patterns for both species, with slightly higher levels of variability in captive than wild populations and F-statistics and Bayesian clustering showing weak but detectable differentiation between captive and wild populations (Fst values: TBP=0.015, BTM=0.020). These results suggest that captive parrot populations represent important resources to augment the genetic variability and demographic robustness of wild populations through captive breeding and reintroductions.}

\normaltalk{An Overview of Hurricane Effects on Birds and Avian Responses, Resilience, and Conservation}{Joseph M Wunderle, Jr, Tom\'{a}s A Carlo-Joglar}{Direct effects of hurricanes on birds are well known, but the indirect effects on birds and their resources remain poorly studied, despite evidence that indirect effects can be substantial and of long duration.  Direct effects include mortality due to exposure to high winds, rain, and storm surges and are often evident shortly after storm passage.  Indirect effects include loss of food resources or foraging substrates, increased risk of predation and parasitism, loss of nests and nest or roost sites.  Short-term responses to hurricane damage include shifts in diet, habitats, and foraging locations. In the aftermath, avian reproduction may be curtailed or nest success decline.  Avian nectarivores and frugivores are especially vulnerable to loss of flowers and fruits, which may be slow to recover prolonging food shortages, some of which may be ameliorated by flowering and fruiting by fast growing pioneer plant species.  Avian insectivores may profit from detritivorous insect outbreaks associated with the pulse of dead vegetation and later from herbivorous insect outbreaks in response to foliage regrowth.  Food resource blooms can increase reproductive success for species which produce multiple clutches extending normal breeding periods.  Predicting avian population recovery times is difficult and may be species-specific and depend on resistance and resilience of their habitats and resources which vary with storm traits, site traits, disturbance history, spatial scale, and successional vagaries. An understanding of hurricane effects on birds and their resources is critically important for conservation as global warming increases the frequency of intense hurricanes and other extreme weather events.}

\normaltalk{Niche Dynamics and Ecological Drivers of Migration in an Insectivorous Owl}{Scott W Yanco, Brian D Linkhart, Peter P Marra, Michael B Wunder}{Many birds migrate seasonally to avail themselves of ephemeral resources and/or to avoid periods of resource scarcity. In contrast to non-migrants, most migratory birds exhibit seasonal niche conservatism at population scales. In fact, niche tracking may be a primary adaptive driver of seasonal migrations. Yet, few studies have examined individual-level niches across a full annual cycle with high spatial resolution or evaluated those results in the context of competing hypotheses about the ultimate drivers of seasonal migration in birds. We sought to understand individual-level niche dynamics across the annual cycle in Flammulated Owls (Psiloscops flammeolus) using high resolution GPS tracking and remote-sensed environmental data. We assessed the constancy and variance of encountered nocturnal temperatures (NT; a proxy for insect activity) and normalized difference vegetation index (NDVI; a proxy for prey biomass) between seasons (breeding, spring and fall migration, and winter). We also compared occupied niche dynamics to counterfactual niches using simulated alternative non-migratory strategies.  Mean NT and NDVI were constant between seasons suggesting strong niche conservatism. Variance in encountered NT and NDVI was higher during migratory periods suggesting more volatile conditions experienced en route between summer and winter grounds. Unsurprisingly, counterfactual simulations showed that year-round residency on northern breeding grounds would result in lower encountered NT and NDVI. Counterintuitively, simulated year-round residents in Mexico would have experienced higher NDVI and NT than migrants. Our findings represent novel ecological support for the ``dispersal-migration" hypothesis wherein winter resource scarcity is the primary driver of migration rather than summer resource abundances.}

\normaltalk{Spatial and Taxonomic Variation in Phenological Dynamics of North American Birds}{Casey Youngflesh, Jacob Socolar, Morgan Tingley}{In seasonal environments, the timing of ecological events (known as phenology) plays a critical role in ecosystem functionality. Phenological changes have been observed across a number of ecological systems in recent years, yet much remains unknown, particularly with regard to how these changes vary over space and across species. Using large-scale observational data from both eBird, a community science platform, and MAPS, a continental-scale bird banding program, we estimated both arrival and breeding phenology for nearly 100 forest dwelling passerine birds across eastern North America over the last several decades. Hierarchical Bayesian spatial autoregressive models were used to evaluate phenological change over time and space, and the degree to which these changes are due to earlier spring arrival, as defined by vegetation phenology. We found that across the species assemblage, both bird arrival and breeding are advancing in response to earlier vegetation `green-up'. This phenological sensitivity however varies among species (i.e., being more pronounced for species that overwinter further south) and across species' ranges. We also find that migration phenology is changing at a different rate from breeding phenology, which has effectively led to a compression of the interval between these events over time. This work highlights both the within-species and cross-trophic nature of phenological change, and has implications for understanding phenological mismatch, whereby species are temporally out of sync with favorable environmental conditions, in the context of global change.}

\normaltalk{Bergmann's Rule is Followed at Multiple Stages of Post-Embryonic Development in a Long-Distance Migratory Songbird}{Joseph Youtz, Kelly D Miller, Keith Bowers, Samantha Rogers, Lesley P Bulluck, Matthew Johnson, Brian D Peer, Katie L Percy, Erik Johnson, Elizabeth M Ames, Christopher M Tonra, Than J Boves}{Bergmann's rule is a well-established, ecogeographical principle which states that body size varies positively with latitude and is explained by the thermoregulatory benefits of larger bodies as temperatures decline. However, this principle does not seem to easily apply to migratory species that are able to avoid the extreme cold associated with higher latitudes. Further, little is known about the ontogeny of this relationship across life stages or how it is has changed over time. To address these knowledge gaps, we assessed this relationship between latitude and body size of a model migratory species, the prothonotary warbler (Protonotaria citrea) across life stages on their breeding grounds. We also compared historic (1865-1988) and contemporary (2018-19) egg sizes to assess if patterns during this critical stage have changed over time. In accordance with Bergmann's rule, we found a positive relationship between breeding latitude and body mass during all post-embryonic life stages, from early nestling stage through adulthood. We observed this same pattern with historic eggs, but contemporary eggs exhibited a reverse (negative) relationship. We speculate that selection for larger sizes in altricial nestlings as latitude increases may help explain the existence of this pattern in migratory species as even rare extreme cold weather events may cause mortality during this sensitive stage. Although this pattern was present in all contemporary post-embryonic stages and historic eggs, the reversal observed in contemporary eggs may be related to complicated (and dynamic) selective landscapes that could be associated with ongoing global climate change, but more research is necessary.}

\normaltalk{The Critical Ecosystem Partnership Fund in the Caribbean Islands Biodiversity Hotspot -- Programmatic Highlights and Future Investment Plans}{Michele Zador}{Since 2010, the Critical Ecosystem Partnership Fund (CEPF) has funded conservation projects in the KBAs of the Caribbean Islands Biodiversity Hotspot.  In its first investment phase, from 2010 to 2016, CEPF awarded \$6.4 million through 68 civil society organizations to conserve 25 KBAs in eight countries. In December 2019, CEPF updated the Caribbean Islands Hotspot Ecosystem Profile, which identified 324 KBAs and presented a new investment strategy.  Starting in 2021, CEPF plans to open a new investment phase to fund conservation projects in priority KBAs. This presentation will highlight CEPF's results in its first investment phase, highlight key findings from the ecosystem profile update, and introduce its plans for future funding for conservation.}

\normaltalk{Low Adaptive and Neutral Genetic Diversity in the Endangered Antioquia Wren}{Danny Zapata, H\'{e}ctor F Rivera-Guti\'{e}rrez, Juan L Parra, Catalina Gonz\'{a}lez-Quevedo}{Determining the amount and distribution of genetic variation represents a fundamental step for the inference of evolutionary processes acting on populations of endangered species. For these, adaptive evolutionary potential is highly reduced by the loss of genetic variation due to drift. Here, we examined genetic diversity at the mitochondrial control region (CR), 17 microsatellites, and six Toll-like receptor (TLR) genes, known to play a key role on the innate immune response of vertebrates in the endemic Antioquia wren (AW), an endangered passerine bird inhabiting the remnant fragments of tropical dry forest in the northernmost part of the Cauca river canyon, Colombia. We found very low variation for the three types of markers. The CR showed scarce variability, while only six of the 17 microsatellites analyzed and two of the six TLRs sequenced were polymorphic. Population structure analyses suggest the occurrence of one single population in the area. Codon-specific selection tests detected no signatures of positive selection for both polymorphic TLR loci. Comparisons with species showing widespread and restricted distribution supports a low TLR variability typical of small populations.  These results suggest current low evolutionary potential for the species, as its reduced genetic diversity is expected to increase extinction risk by limiting the ability to cope with environmental changes. Our study supports current evidence suggesting drift as the main driver shaping TLR variation, whereas balancing selection might only play a minor role that  barely attenuates the effect of drift.}

\normaltalk{Understanding How Stopover Habitats Function During Spring and Autumn Migration along the Northern Gulf of Mexico Coast}{Theodore J Zenzal, Jeffrey J Buler, Liliana Calderon, Lori A Randall, Wylie C Barrow, James P Cronin, Barry C Wilson, William Vermillion}{Sound and effective conservation strategies for declining migratory landbirds require a better understanding of how migrants respond to different habitat patches they encounter en route. Towards this goal, we established six landbird migration monitoring stations in southern Alabama and Louisiana in forested habitats that differ in their patch size, distance from the coast, and surrounding landscape composition. We measured bird abundance and changes in bird physiological condition during stopover, as well as food availability and habitat structure and composition. We found phenology of habitat use, stopover duration, and fuel deposition rate differed between seasons and sites. During spring, use of large inland patches peaked earlier than small inland patches or coastal sites, whereas in autumn use of small inland patches peaked first. Across seasons, stopover duration was longest at large inland patches in autumn and shortest at coastal patches in spring. Fuel deposition rate was highest at the coastal and large inland patches during both seasons. Stopover use during autumn was influenced by factors both intrinsic and extrinsic to habitat, while only external factors influenced use during spring. Our results illustrate differential habitat use during spring and autumn as well as between habitat types within a season. While all forests examined serve as resting areas, the best land management practice would be to enhance refueling opportunities for migratory landbirds in large inland and all coastal habitat patches.}

\normaltalk{Effects of Resource Availability and Climate on Elevational Movements of Passerine Species Within the Breeding Season}{Martha W Zillig, Erica Fleishman}{A dominant paradigm in ornithology is that passerines arrive on their breeding grounds, establish a territory, and remain on that territory for the majority of the breeding season. Our research on passerines in the mountains of the Great Basin, USA, challenges that paradigm. Our models of abundance, which were based on 19 years of data, indicate that some species consistently move upslope or downslope during their breeding season. Microclimate, availability of insects, fruits, and seeds, primary productivity (measured as the normalized difference vegetation index), and snowmelt timing appear to drive such movements in a subset of these taxa. Consistent, directional elevational movement during the breeding season may change understanding of reproductive ecology, effects of climate change on avian populations, and single-species occupancy and abundance modeling.}

\normaltalk{Impacts of Hurricanes on Forest Structure in Puerto Rico:  a Review}{Jess K Zimmerman}{Cyclones and other windstorms are a dominant component of the disturbance regime in many forests. How trees respond to the immediate impacts of storms and the way in which they recover from damage strongly influences forest structure and the responses of consumer populations such as birds. Long-term studies of hurricanes in Puerto Rico in several forest environments offers considerable insight into how hurricanes shape the response trajectories to wind disturbance in forest environments, both literally and figuratively. I review available information on studies in tabonuco, palo colorado, and elfin forests (wet to rainforest Holdridge zones) in the Luquillo Mountains and in dry forest in Guanica Forest Preserve.  I also draw on studies of secondary forests in humid and wet forests in eastern Puerto Rico. Several general conclusions emerge from these comparisons.  First, overall forest stature is a primary proximate determinant of forest damage, taller forests suffering more damage than short forests.  Second, an evolutionary history of exposure to hurricane disturbance has been shown to select for wind resistance when species or seed provenances from regions of contrasting hurricane disturbance are grown in common garden and exposed to a hurricane.  Third, site productivity (positively correlated with forest stature) strongly influences the rate of recovery, more productive sites recovering more quickly towards pre-storm reference points than less productive sites.  Finally, the ability of trees to resprout following damage is a key component of individual species' responses to hurricane disturbance, and these responses have strong feedbacks on forest structural response trajectories.}

\normaltalk{"Returning to the Wild" a Parrot Conundrum}{Bonnie J Zimmermann}{The Wildlife Conservation Society Indonesian Crime Unit, TRAFFIC and other organizations are actively confiscating parrots from the illegal trade and prosecuting perpetrators.  However, that creates a new problem -- the disposition and care of the birds, many of which are endangered species.  Birds can be rehabilitated and returned to their native forests, but there are four critical steps which are likely to determine the success or failure of this procedure:  disease testing, endemism, capacity building at the local level and soft release.   Wild parrot release has been criticized because of the possible presence of latent disease.  Birds confiscated from poachers have been kept in cages and handled by multiple middlemen and may have been exposed to different infectious agents.  When these birds are then released into their homeland they have the potential to bring with them disease to which their compatriots have had no previous exposure.   The Indonesian Parrot Project built our first rescue, rehabilitation and release center on Seram Island in 2006, and are currently consulting and working directly with the Indonesian forestry department and the National Parks to create reintroduction sites in Maluku and Papua.  To date over 2,000 birds have been reintroduced back into the wild.}

\normaltalk{Birds in Urban Greenspaces: Novel Ecosystems for Nesting, Feeding and Eluding Predators}{Iriana Zuria, Jesús Zu\~{n}iga-Palacios, Pilar Carbo-Ramirez}{Greenspaces in cities are considered sites with a potential conservation value because vegetated areas provide habitat for many species of birds. They come in a wide variety of sizes, shapes, vegetation types and management schemes, and therefore are different in terms of conservation. Despite they are present in every Mexican city, very little is known about their potential value for wildlife. Using a medium sized city (ca. 500,000 inhabitants) as study case, we have tried to answer some fundamental questions to understand these important habitats: 1) How large must an urban greenspace be for bird conservation? 2) How does heterogeneity within and around greenspaces affect bird communities? 3) How are greenspaces used for foraging and reproduction? We studied birds in managed greenspaces (parks, gardens, vegetated corridors) and vacant lots of different sizes. Species richness was higher in parks and gardens and lower in road strip corridors, where more noise and traffic was registered. Managed greenspace area was the most important variable that positively influenced species richness, but also important was the amount of area covered by impervious surfaces in the adjacent landscape. Vacant lots also harbored a great diversity of birds and no differences were found in species richness or diversity between small and large vacant lots. We also present data about birds breeding and feeding in greenspaces and about antipredatory behavior. Even small greenspaces can function as bird reservoirs in Mexican urban areas, despite being immersed in intense levels of urbanization, and therefore can improve urban biodiversity and human wellbeing.}

