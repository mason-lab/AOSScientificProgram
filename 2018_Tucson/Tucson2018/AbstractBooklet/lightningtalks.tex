\normaltalk{Ultraconserved elements resolve genus-level relationships in the honeyeaters (Meliphagidae)}{Michael J Andersen\\Nicholas R Friedman\\Leo Joseph\\A T Peterson\\Robert G Moyle\\Arpad S Nyari}{The Indo-Pacific hosts rich fauna and flora that have contributed to the development of speciation and biogeographic theory. The region's avifauna is well-suited to broad comparative studies, and has been used extensively as a model for understanding speciation and adaptation. Honeyeaters are one of the most species-rich and ecologically diverse families endemic to the region, with upwards of 12 species co-occurring at some sites. They have been the subject of countless studies of trait evolution ranging from eco-morphological to behavioral evolution. Past studies on honeyeater systematics have been hampered by sparse sampling of both characters and taxa. More recently, a nine-gene supermatrix was the first to include all {raise.17exhbox{\$scriptstylemathtt{sim}\$}}187 honeyeater species, but was unable to resolve key relationships. To disentangle these challenging relationships of honeyeater diversification, we used sequence capture of ultraconserved elements to sequence and analyze 4,397 loci (4.6 Mb) for all meliphagid genera. We present the first phylogenomic hypothesis of the Meliphagidae, analyzed using the multispecies coalescent. We reevaluate previous phylogenetic hypotheses of the group and provide a novel, robust framework that will inform future revisions of systematics, biogeography, and evolution within the largest avian radiation east of Wallace's Line.}

\normaltalk{Seasonal Shift in Understory Thrushes in a Montane Forest in Western Ecuador}{Dusti Becker}{During annual avian monitoring at Reserva las Tangaras near Mindo, in the Andes of western Ecuador, we noticed a rather dramatic shift in the abundance of two understory thrush species. From June to August our mist net samples were well represented with Andean Solitaire (Myadestes ralloides). In Decembers, this species was absent when mist netting included many Swainson's Thrushes (Catharus ustulatus). Later we determined that the Andean Solitaire were at higher elevations when Swainson's Thrushes were present. These thrush species are among the more common species in the understory bird community representing up to 5\% of the community when they are present. What is behind this seasonal shift in thrush species representation? We don't really know what drives the shift, or exactly when it takes place, but hypothesize that interspecific interactions at time of arrival of the neotropical migratory Swainson's Thrushes are a factor. Here we will quickly share the numbers, some community statistics, and information on variation in habitat use by the two thrush species during what is typically considered summer and winter for North Americans.}

\normaltalk{Fine-scale mapping of coastal plant communities in the northeastern USA: measuring habitat of a declining marsh passerine}{Maureen D Correll\\Wouter Hantson\\Thomas P Hodgman\\Christopher S Elphick\\W G Shriver\\Brian J Olsen}{Salt marshes of the northeastern United States are dynamic landscapes where the tidal flooding regime creates patterns of plant zonation through differences in elevation, salinity, and local hydrology. These patterns of zonation can change quickly due to both natural and anthropogenic stressors, making tidal marshes vulnerable to degradation and loss. Many wildlife species, particularly birds, rely particularly on the high-marsh zone for breeding and foraging habitat. We compared several commonly-used remote sensing methods to develop a tool that accurately maps high- and low-marsh zonation for use in management and conservation planning in the northeast USA, particularly for the rapidly-declining and high-marsh obligate Saltmarsh Sparrow (Ammodramus caudacutus). We found that random forests (RF) outperformed other classifiers when applied to the most recent National Agricultural Imagery Program (NAIP) imagery, NAIP derivatives, and elevation data between coastal Maine and Virginia, USA. We then used RF methods to classify plant zonation within a 500-m buffer around coastal marsh delineated in the National Wetland Inventory. We found mean classification accuracies of 94\% for high marsh, 76\% for low marsh zones, and 90\% overall map accuracy. The detailed output is a 3-m resolution continuous map of tidal marsh vegetation communities and cover classes that can be used in habitat modeling for marsh-obligate species or to monitor changes in marsh plant communities over time.}

\normaltalk{Citizen science data reveals hidden migratory movements of the Common Potoo, Nyctibius griseus, in Brazil}{Lucas W DeGroote\\Erika Hingst-Zaher\\Luciano M Lima\\Jacob Slyder\\John Wenzel}{The Common Potoo (Nyctibius griseus) is abundant, charismatic, and generally considered to be sedentary across its range. Using citizen science data from eBird and WikiAves, we demonstrate that the Common Potoo is a partial migrant whose breeding populations depart southeastern Brazil for more equatorial localities in the winter. Overlaying seasonal temperature data, we find that Common Potoos vacate regions that drop below a minimum temperature of 15° C. The migration of southern populations to equatorial regions also results in Common Potoos residing in winter habitats that are on average wetter. Checklists submitted to eBird were spatially biased between seasons while observations submitted to WikiAves were seasonally consistent. By using a mechanistic approach, we expose the potential value WikiAves data offers towards elucidating seasonal movements of South American birds.}

\normaltalk{Impacts of environmental heterogeneity and migration strategies on population genetic structure across an extensively hybridizing songbird genus}{Cooper A French\\John T Klicka}{We identify major genetic variation across the North American Towhees (Pipilo) jointly influenced by introgression, geographic isolation, and migratory strategy. This variation indicates an evolutionary history of acquired migratory and physiological adaptations that contributes to broad phenotypic diversity across the genus. We utilized ddRAD-Seq data from a broad geographic sample of individuals and analyzed phylogeny and population structure/introgression using maximum likelihood and Bayesian methods. These results, combined with demographic modeling and morphological analysis were used to assess the validity of transcontinental gene flow.}

\normaltalk{Countershading as a potential flight-enhancing adaptation in seabirds}{Peter Houde\\Mostafa Hassanalian\\Samah Ben Ayed\\Mohamed Ali\\Peter Houde\\Christopher M Hocut\\Abdessattar Abdelkefi}{Countershading (plumage that is dark above and light below) is pervasive among seabirds. Previous study has shown it to be weakly correlated with multiple potential selective agents, as is true of many adaptations. It is hypothesized to be an adaptation for crypsis and feather durability. White underparts of birds in flight are less visible from underwater than dark underparts, and dark upperparts might be more difficult to spot from above by predators or kleptoparasites. Dorsal eumelaninization may resist feather deterioration due to mechanical abrasion and UV irradiation. It is also correlated with feeding behavior and foraging group size, as well as by phylogeny. We present yet another potential adaptive function for countershading in seabirds: improved flight efficiency by drag reduction. We obtained wing measurements and migratory routes and times of wandering albatross, manx shearwater, and sooty tern from literature. We modeled thermal analysis of all combinations of solid black and white wing color dorsally and ventrally considering ambient temperature, sky temperature, sea surface temperature, wind speed, solar irradiation, humidity, and cloud factor specified hourly for each migratory route. Applying the Blasius solution for heated boundary layers, we found black over white countershading minimizes daily net skin drag, even though dark upperwing drag exceeds that of a light underwing at night. It also outperformed all other color combinations, day and night, although the specific outcomes were temperature dependent. This drag reduction may contribute to the energetic efficiency and unequaled endurance of low altitude dynamically soaring seabirds.}

\normaltalk{A phylogenomic investigation of the family Ardeidae (herons)}{Jack P Hruska\\Jack P Hruska\\Fred Sheldon\\Kevin McCracken\\Robert Moyle}{Although the generic composition of the family Ardeidae (herons) is not controversial, many relationships within the family are still debated. Generally, five subfamilies, 18–19 genera, and 63–66 species are recognized, although the composition of each of these designations varies by authority. A comprehensive phylogeny for the family is still lacking. Previous studies were either limited in scope or produced phylogenies with poorly supported relationships. As a result, the monophyly of certain subfamilies, tribes and genera are still in question. Our study attempts to address these shortcomings by providing the first robust and thorough phylogeny for the family. Using genomic Ultraconserved Elements (UCEs) from 18 genera and 45 species (roughly 75\% of the total species diversity), we recovered preliminary phylogenies using maximum likelihood (RaXML) and quartet (SVDquartets) methods. Both methods produced highly congruent and robust phylogenies with the majority of nodes recovered being supported by high bootstrap values. Our results support some previously suggested relationships, such as the inferred placement of Agamia agami as sister to a clade comprising of the bitterns (Botaurinae), day herons (Ardeinae) and night herons (Nycticoracinae). We provide evidence that dispute other relationships as well. We suggest that the night-herons, as currently classified, do not form a monophyletic group, with Gorsachius falling outside of the clade (in addition to being polyphyletic). Our preliminary results provide a better understanding of the evolutionary history of the herons and lend support for an updated taxonomy of the family.}

\normaltalk{Estimating population-level reproductive rates from roost counts for Red-crowned Parrots (Amazona viridigenalis)}{Simon R Kiacz\\Simon R Kiacz\\Donald J Brightsmith}{Estimating reproductive success of avian populations is important for making conservation decisions but can be costly and time consuming. In addition, when nest monitoring is used the results may be biased by the inclusion of nests that are more easily found. As a result, there is a need for quick, cheap, and relatively unbiased methods for estimating population-wide reproductive success. Here we report on population-wide estimates of reproduction rates from observations of Red-crowned Parrots (RCPA) (Amazona viridigenalis) in the Lower Rio Grande Valley of Texas (LRGV). RCPAs have been regularly recorded in urban areas of the LRGV since the mid 1980's and the population has been steadily increasing since the mid 1990's. All RCPAs in the LRGV are suspected to utilize one of four current roosting sites every evening. Since 2016, we have collected attendance data of juvenile and adult RCPAs at all known roost locations (181 counts). In 2016, we checked 530 birds of which 131 were clearly juveniles, and during 2017, 1616 checks yielded 245 juveniles suggesting a 25\% and 15\% juvenile to adult percentage, respectively. If data on survival of related taxa are applicable, the RCPA population in the LRGV may be growing at an average rate of nearly 15\% per year. This value corresponds well with growth rates since 1995 according to Christmas Bird Count data, suggesting that this juvenile count method is providing useful estimates of reproductive success. This technique may be very useful for other communal roosting species with visually distinct juveniles.}

\normaltalk{Will Intensive Song Playback Affect Breeding Performance? A field study of Abroscopus albogularis in Taiwan}{Mu-Ming Lin\\Cheng-Te Yao\\Tzung-Su Ding}{Song playback can help birdwatchers to efficiently detect specific species, but it might cause unnecessary stress and disturbance on birds. Previous studies mainly examined playback effects on individual's behavior and physical state. This study was aimed to investigate the influence of intensive song playback on breeding performance of Rufous-faced Warbler during their nestling period. The experiments were conducted in a mid-elevation, native broad-leaved forest in southern Taiwan from April to July, 2015. We introduced three different intensities of playback to 35 nest boxes of Rufous-faced Warbler: control (no playback), mid-intensity during early stage of nestling period (310 mins/day, 5 days, day 1–10) and high-intensity during late stage of nestling period (610 mins/day, 5 days, day 9–13). Parents' responses in the first 25 minutes of playbacks were recorded. The nest boxes were checked every 1-2 days and the nestlings were weighted on day 10, 12 and 14. Both mid- and high-intensities playbacks significantly increased alarming, singing and flying frequencies of parents but did not affect homing frequency. There were no significant effects on reproductive success and nestling survival rate. However, the nestlings exposed to high-intensity playbacks had significantly lighter weights on the 12th day of nestling period. We suggest that song playback increases Rufous-faced Warblers' behavioral responses, which could lead to extra energy cost during breeding season. Although song playback did not affect breeding performance in this study, very intensive playback could reduce nestling quality.}

\normaltalk{Niche differentiation and genomic admixture of Free Flying Los Angeles Parrots (FFLAP)}{James M Maley\\Ryan S Terrill\\Whitney LE Tsai\\John E McCormack}{Southern California is home to a large population of naturalized parrots, all former pets or descendants of pets. There are at least six species of Amazona parrots that are known or suspected of breeding in Los Angeles County. The most abundant species is Red-crowned Parrot, numbering in the thousands, followed by Lilac-crowned Parrots. These two close relatives are allopatric in their native ranges, but breed in the same areas of Los Angeles. Lilac-crowned Parrots are often found at higher elevations in their native range, and have been found breeding in the mountains above Los Angeles. We wanted to determine if the two species occupy different niches in Southern California, and to determine if they are hybridizing. We created niche models for the native ranges of both species using specimen and eBird data. We also created niche models for both species in Los Angeles using eBird data. We found that they occupy different niches in their native ranges, but they largely overlapped in Southern California. To determine the level of genetic admixture we sequenced ultraconserved elements for 6 Lilac-crowned and 7 Red-crowned parrots collected between 1934 and 1955 from their native ranges. We also sequenced UCEs for 22 Amazona parrots salvaged in Los Angeles and San Diego. It remains to be seen if species boundaries will remain, or if the parrot flock of Southern California will become a huge hybrid swarm.}

\normaltalk{Movement Ecology and Diet of a Culturally Important Frugivorous Bird Species}{Bret N Mossman\\Bret N Mossman\\Kristina Paxton\\Patrick J Hart}{Avian frugivores are important dispersers of plant seeds and differences in movement patterns can influence the spread of seeds throughout an ecosystem. Oma'o (Myadestes obscurus) are the last native frugivore in the Hawaiian Islands that occur in large enough populations to have a functional influence in forest communities and are highly sedentary. Understanding how age and fragmentation impact movement of important seed dispersers is helpful for understanding how plant communities maintain connectivity on the landscape and it will provide novel information on the life history of the Oma'o. Juvenile movement is a very important aspect of bird behavior, but it has received little detailed study. This project can aid in the understanding of juvenile avian dispersal through fragmented ecosystems which can provide important insight into life history traits that limit the Oma'o's survival and expose patterns that will be relevant to the conservation of additional bird species.}

\normaltalk{Temporal scale of drought response in Great Plains birds}{Timothy J O'Connell\\Samantha M Cady\\Timothy J O'Connell}{A predicted effect of anthropogenic climate change is an increase in frequency, duration, and magnitude of weather extremes, including drought events. Drought can be assessed at multiple temporal scales, each of which relates to a different water source. For example, drought quantified at an annual scale corresponds with groundwater availability while drought quantified at a monthly scale relates to current precipitation and surface flow. Birds present a unique opportunity to assess wildlife responses to drought-induced stress because they are mobile and easily surveyed. This study leverages Breeding Bird Survey (BBS) to determine whether there are detectable songbird responses during drought conditions. To account for observer bias and differing land cover at route locations, we used generalized linear mixed modeling with a fixed effect of drought magnitude at differing temporal scales, and random effects of observer identification and route location. We applied AIC model ranking to determine which drought scale best explained the distribution of birds at the species level. Drought conditions were quantified using PRISM data and a standardized precipitation evapotranspiration index. Results show a mixed response among species with the strongest signal at an annual or near-annual scale. Some birds did not show a response to drought conditions at any scale. Results indicate a complex relationship between drought conditions and songbird distribution. We conclude by recommending that, for precise assessment of the effects of drought on bird communities, species should be analyzed separately, due to variation among them.}

\normaltalk{Population genomics of birds in the North American warm deserts}{Kaiya L Provost\\Brian T Smith}{Bird populations are often geographically structured by biogeographic barriers, which prevent individuals from crossing them via ecological or physical constraints. However, not all bird populations or species become structured by barriers, as some successfully disperse across. In the warm deserts of North America, the Cochise Filter Barrier serves as semi-permeable filter for bird communities in the Sonoran and Chihuahuan deserts. Previous work has shown genetic isolation across the barrier in some species, while others are unstructured or appear to maintain gene flow among deserts. Here we use whole-genome-resequencing to investigate four species across the region (Cardinalis sinuatus, Campylorhynchus brunneicapillus, Melozone fusca, and Amphispiza bilineata), evaluate their genomic structure, and discuss how variation in species' phenotypes and environmental preferences influence their ability to disperse through the barrier. Our preliminary findings indicate that there are differences in genomic structure across the region within these species and within the community as a whole. These differences may be associated with variation in morphological phenotype and habitat preference. Organismal traits could be key in determining whether species can maintain gene flow successfully across the Cochise Filter Barrier.}

\normaltalk{A Novel Approach to Diet Analysis: Next-Generation Sequencing of Raptor Pellets and Fecal Material}{Marilyn E Wright\\Becky Williams}{Northern Goshawks (Accipiter gentilis) are important apex predators and management indicator species. Concerns over potential declines in Goshawk populations have led researchers to discover that some of the main limiting factors for Goshawks are prey abundance and availability. Despite the importance of these factors, the methods for studying prey selection by Goshawks, which currently rely heavily on nest site cameras, may be insufficient for accurately assessing diet composition. We are using a novel approach to diet analysis in raptors, next-generation sequencing of pellets and fecal matter collected at nest sites to characterize diet. It is easy and minimally invasive to collect pellets and fecal material from nest sites, and next-generation sequencing provides the opportunity to quantify diet using the unique genetic markers of prey species. Already a popular method for diet analysis in insectivorous bats, next-generation sequencing provides the potential to assess the diet of raptors and other birds at a finer scale than other methods currently used.}

