\normaltalk{Avian diversity above the clouds: comparing elevational patterns in North and South temperate mountains}{Tomas A Altamirano\\Devin R de Zwaan\\Jose T Ibarra\\Kathy Martin}{Mountain ecosystems comprise 24\% of the land-base of the Americas, and are critical components of the life-cycle for 35\% of North American bird species. Globally, however, high elevation avian communities are poorly monitored, with taxonomic and functional diversity receiving little attention. In 2017, we initiated a temperate montane bird study, comparing: i) taxonomic and functional turnover across three habitats increasing in elevation: upper montane ($>$50\% tree cover), subalpine (5-50\% tree cover), and alpine ($<$5\% tree cover), and ii) the breeding strategies of alpine birds in North and South America. We conducted 600 diurnal bird point counts in six North temperate (British Columbia, Canada; 54°N latitude) and 10 South temperate mountains (Chile; 39°S latitude). Species richness was comparable between sites (49-54 species), but decreased with elevation in Canada (upper montane: 31, alpine: 16) compared to remaining consistent across elevation in the Chilean Andes ({raise.17exhbox{\$scriptstylemathtt{sim}\$}}33 in each habitat). Species composition among mountains was highly variable in both hemispheres (Jaccard index $<$ 0.5) indicating potential variation in drivers such as habitat structure. We found contrasting patterns in alpine breeding strategies, as open-cup nesters made up 88\% (14/16 species) of the Canadian alpine community, while 100\% of the 37 nests (6 species) found in the Chilean alpine were cavity nesters (rock crevices or ground excavations). Comparing high elevation temperate ecosystems in the Americas provides a strong framework to address avian breeding dynamics in stochastic environments, as well as understand the potential conservation value of mountain habitats as refugia for avian populations with environmental change.}

\normaltalk{Phylogeography of Estrildinae Finches from Africa, South Asia and Australia: African and Australian Convergent Evolution Species}{Antonio Arnaiz-Villena\\Valentin Ruiz-del-Valle\\Jose Palacio-Gruber\\Cristina Campos\\Ester Mu\~{n}iz}{Estrildid finches are distributed throughout Africa, South Asia, Australia and Indian and Pacific Ocean islands. Specific phylogenetic questions have been clarified in this study by analyses of 64 species of estrildids through cytochrome b DNA sequencing and Bayesian Inference. Our results support that estrildids are a monophyletic group with polytomies that may have started evolving by Middle Miocene Epoch (about 16.5 million years ago). This date is coincidental with the Fringillinae finches' radiation starting time and also with the biggest Himalayan and Tibetan Plateau uplift and moonson climate changes. The most basal estrildid clade comprises African, Indian and Australian birds, suggesting that the whole estrildids radiation might have originated around India. It is shown that : 1) Gouldian Finch (Chloebia / Erythrura gouldiae) is definitively included within genus Erythrura, 2) the oldest Estrildinae evolutive radiation group seems to be the Indian silverbill (Lonchura malabarica) together with African silverbill (Lonchura cantans) and the phenetically distinct Diamond Firetail (Stagonopleura guttata) from Australia, 3) the enigmatic Java sparrow (Padda / Lonchura oryzivora) is a Lonchura species, 4) African munias (“Spermestes”) form a distinct phylogenetic cluster (within genus Lonchura) with respect to Asian and Australian munias.Finally,phenotypic convergent evolution is found between African and Australian species. Email:arnaizantonio@gmail.com}

\normaltalk{Effectiveness of birds' protection law at Pakistan}{Mahmood A Awan\\Bushra N Khan}{Pakistan is spread over magnificent geographical ranges started from marine areas to second best peak in the world that is inhabited by numerous resident and migratory birds. The present research work was planned to access the existing laws and their efficacy for conservation and management of avian diversity. Last ten years data of avifauna occurrence, abundance and status along with threats was analyzed for this purpose. Field trips, meeting with local community and concerned departments were also arranged. Findings revealed that annually more than 100,000 of 350 avian species visited wetlands due to availability of fish as source of food. These birds included some threatened species like Oxyura leucocephala, Chlamydotis undulata, Anhinga melanogaster, Prinia burnesii, Aythya nyroca and Grus leucogeranus. Different threats like degradation of habitat, land retrieval, illegal hunting, dearth and poor law enforcement were also noticed. Different organization for conservation and management of birds like IUCN, RAMSAR, Wildlife Acts, National Conservation Strategy (NCS) and WWF-P were also working here. The Pakistan government is signatory to numerous worldwide initiatives and is particularly doing concerted schemes to save avian diversify. Noncompliance of existing law for avian diversity protection and management is a key driver for declining of these flying creatures.}

\normaltalk{Habitat preferences of overwintering Baird's and Grasshopper sparrows in relation to seed availability in the soil seed bank}{Fabiola Baeza-Tarin\\Mieke Titulaer\\Denis J Perez-Ordonez\\Erin H Strasser\\Russel Martin\\Louis A Harveson}{Baird's Sparrow (Ammodramus bairdii) and Grasshopper Sparrow (Ammodramus savannarum) have lost between 70–80\% of their total population since 1966. During the wintering months, these grasslands birds depend almost entirely on granivory, suggesting that seed availability in the soil has a direct impact on their populations. Our objectives are to determine the effects of seed availability in the soil seed bank on habitat use by Baird's and Grasshopper sparrows, and compare two different grazing systems (moderate density rotational grazing vs. low density continuous grazing). From December 2017 to March 2018 we deployed over 60 Baird's and Grasshopper sparrows with VHF transmitters and tracked them once per day. We collected vegetation data for 20 points per bird and across a grid of points spaced every 100 m using ocular estimates of ground cover in 5-m circular radius plots. We collected soil seed bank samples for 60 bird points and 60 randomly selected grid points, half in the rotational grazing site and half in the continuous grazing site. We counted and weighed seeds in the soil samples to determine seed biomass availability. We present results on the relationship of seed biomass availability in the soil seed bank to habitat use of Baird's and Grasshopper sparrows in two different grazing systems. The results will allow us to better understand the relationship between grassland birds and their habitat in order to inform land managers and researchers about the ecological conditions needed to support thriving grassland bird populations.}

\normaltalk{Comparison among sky-island haemosporidian communities reveals dynamics of parasite-species turnover and host-switching}{Selina M Bauernfeind\\Lisa N Barrow\\Christopher C Witt\\Matthew J Baumann\\Serina S Brady\\Andrea N Chavez}{Birds and their haemosporidian parasites (Haemoproteus, Plasmodium, and Leucocytozoon) comprise a diverse multi-host, multi-parasite system that is uniquely suited for studying evolutionary dynamics of host-parasite relationships. However, because of the challenges of surveying entire bird communities for parasites, few studies to date have been able to fully characterize parasite diversity, rates of species-turnover, and variation in host-parasite relationships across geographic space. To address these questions, we conducted community-level surveys of haemosporidians within a narrow elevational zone (2100–2500 m) in three adjacent mountain ranges in New Mexico. We screened 776 specimens using PCR and sequencing of an mtDNA 'barcode' (cytb), as well as microscopy of thin blood smears. We used phylogenies of parasite and host lineages, respectively, to compare community composition and host-parasite relationships among the three mountain ranges. We found 269 infections (34.7\% of birds). The majority were Haemoproteus (50.0\%), followed by Leucocytozoon (38.7\%), and Plasmodium (24.2\%). Compared to previously published sequences, 60\% were novel. We found conserved host-specificity at the avian clade and species level, but host-range varied strikingly by parasite genus. Host-switching by haemosporidian lineages was common, particularly between closely related hosts. Infection rates also varied strikingly among hosts, with vireos and nuthatches exhibiting nearly universal infection and non-infection, respectively. Haemosporidian species turnover among mountain ranges was modest overall, but high in specific host species. Host specificity appears to be fleeting over evolutionary time and variable across space, although certain hosts may be critical to the maintenance of overall parasite diversity.}

\normaltalk{Investigating Western Yellow-billed Cuckoo Breeding Status In The Mountains Of Southeastern Arizona; Implications For Recovery}{Nicholas D Beauregard}{The western population of the Yellow-billed Cuckoo (Coccyzus americanus) was federally listed as a threatened species in 2014, with declines across the western states attributed to loss and fragmentation of riparian habitat. Although most western Yellow-billed Cuckoos occur in riparian areas, they have recently been found during the breeding season in many ephemeral Madrean-evergreen woodland drainages in the Sky Islands of southeastern Arizona. If most of these cuckoos are breeding, this region provides a previously unknown population that may have implications for recovery of the species given increasing degradation of southwestern riparian habitat. My research aims to confirm the breeding status and distribution of cuckoos in the Madrean Sky Islands and determine the habitat characteristics associated with breeding. Furthermore, I am evaluating the use of autonomous recording units to monitor cuckoos. Preliminary results indicate that western Yellow-billed Cuckoos are widespread in the Sky Islands, with localized populations in specific mountain ranges. More than half (n=56) of drainages surveyed between 2013 and 2017 were occupied, and more than one quarter (n=29) have probable or confirmed breeding. These results have implications for future research, and field work is planned for 2018 and 2019. This will include intensive nest searching in addition to playback surveys, habitat analysis, and deployment of recording units. Preliminary results and methods for this further research will be presented.}

\normaltalk{An evaluation of land restoration effects on northern bobwhite survival in north-central Texas}{Danielle E Belleny\\Heather A Mathewson\\Jeff B Breeden\\John Tomecek\\Thomas W Schwertner\\Jim Giocomo}{Northern bobwhite (Colinus virginanus) populations have declined 4.5\% in the Oaks and Prairie region during 1966–2015 (BBS 2015). Land restoration practices offer opportunities to mitigate the declines in bobwhites. Our study evaluates the effects of land restoration practices (i.e. brush removal, native grass reseeding, and prescribed grazing) to bobwhite demographics. We expected the application of restoration practices to create bobwhite habitats and predicted increased reproductive success and survival. We radio-collared 182 individuals from 2016-2017 to monitor nest survival and breeding season survival. In 2016, nests in non-restored landscapes had a higher chance of success, 36\% (n = 10), compared to those in treated landscapes, 26\% (n = 4). In 2017, nests in treated landscapes had a higher chance of success, 32\% (n=14), compared to those in non-restored landscapes, 19\% (n= 9). For both years, AIC model selection strongly supported vegetation characteristics as predictors of nest success. Models indicated that increased herbaceous and decreased litter ground cover predicted higher success. In addition, nest success in 2017 increased as visual obstruction increased. Our Welch's t-tests comparing 2017 nest site cover differed in grass (x = 53.0, SD = 24.01; x = 71.5, SD = 21.55, P = 0.04), and litter (x = 33.5, SD = 10.4, x = 8.0, SD = 7.07, P = 0.005), between treated and non-restored landscapes, respectively. Breeding season survival in 2017 was higher for bobwhites in treated landscapes (51\%, n = 85) than non-restored landscapes (44\%, n = 68). Depredation was considered the primary cause of mortality.}

\normaltalk{Tree species preference for foraging by insectivorous birds in a woodland dominated by alien tree species in Puerto Rico}{William Beltr\'{a}n\\Joseph M Wunderle\\Jr}{We tested Russell Greenberg's foliage palatability hypothesis that predicts avian insectivores will preferentially forage in tree species with the greatest abundance of their arthropod prey, which in turn are associated with the tree's foliage nutrition and palatability. The hypothesis was tested in a novel Prosopis-Leucaena woodland in Puerto Rico by determining foraging preferences of 5 insectivorous bird species (2 resident flycatchers, 1 resident warbler, 2 winter resident warblers) for 6 tree species (5 alien, 1 native) and relating preferences to foliage arthropod biomass and leaf chemistry. The most frequently preferred tree species for foraging were the alien Prosopis juliflora (preferred by 5 bird species) and alien Pithecellobium dulce (preferred by 4 bird species). Both species had high arthropod biomass, high N content, low lignin/N ratios, and low hemicellulose content. Compounds, previously known to affect herbivore responses to Albizia lebbeck and Leucaena leucocephala, may explain low arthropod biomass despite high N content in Albizia and avoidance of Leucaena by 4 bird species despite its high arthropod biomass. The native Bucida buceras had tough leaves with low N content, low arthropod biomass, and only 1 bird species showed a weak preference for foraging in it. Some alien tree species had highly palatable foliage with high arthropod biomass and hence were preferred for foraging by avian insectivores as predicted. High foliage palatability of some alien tree species may weaken the effect of enemy release in some novel plant communities.}

\normaltalk{Novel object exploration in a highly curious and social raptor}{Breanna L Bennett\\Keith L Bildstein\\Robert L Curry}{Exploration behavior consists of how an individual explores a novel environment, in terms of both spatial exploration of a new area and manipulation of objects in a new area. Neophobia, a related concept, describes an individual's willingness to approach and engage with novel objects. Recent studies of Chimango Caracaras (Mivalgo chimango) have shown that age class influences degree of object manipulation, though the literature lacks further studies of object exploration in birds of prey, including those with purportedly high levels of curiosity. Further investigation of object manipulation was carried out on a related and reportedly more curious bird of prey, the Striated Caracara (Phalcoboenus australis). From 2015 – 2017, novel object exploration observations of Striated Caracaras in the Falkland Islands were recorded as they interacted with a series of six anthropogenic objects. These anthropogenic objects all resembled each other in terms of color and length and differed only in shape. Interactions were consistently measured for 20 minutes once a test was initiated. A total of 68 tests were completed in the three-year period on three different islands in the Falklands, including Saunders Island, Sealion Island, and Carcass Island. While all tests were initiated while in sight of at least one caracara, between 10-15\% of tests did not see any engagement by individuals. All other tests included engagement by individuals in juvenile, sub-adult, and adult age classes. Further analysis will be conducted to examine whether age class, sex, or other variables have a significant effect on interaction with novel objects.}

\normaltalk{The vocal-flight-respiratory axis: origins of avian vocal learning?}{Karl S Berg}{Vocal learning is only found in three, albeit important, lineages of birds: hummingbirds, parrots and oscine songbirds. Its evolutionary origins remain mysterious. The Motor Theory for the Origins of Vocal Learning posits that expansion on ancestral neural pathways in the forebrain that control non-vocal motor activities, led to neural circuits that support vocal production learning. However, vocal and non-vocal motor activities have seldom been compared using phylogenetic comparative analysis, a cornerstone of evolutionary biology. Because many species vocalize during flapping flight, and flight and vocal production are assumed to have each co-opted underlying respiratory physiology, we tested for evolutionary relationships between body size, wingbeat period and vocal signal duration during flapping flight in 150 species spanning 12 (of 34) extant avian orders. Bayesian phylogenetic analysis demonstrated a tight evolutionary triad between body size, wingbeat period and vocal signal duration. Non-vocal learners had phylogenetically corrected mean ½ wingbeat periods roughly equal to their signal durations, predicted if birds amplify vocal production via the expiratory (power) stroke. Surprisingly, vocal learners' signal durations on average exceeded the entire wingbeat cycle. However, vocal learners also had on average smaller bodies and shorter wingbeat periods than non-vocal learners. Results provide general support for the Motor Theory in that non-vocal and vocal behaviors co-evolved, leading us to speculate that reductions in body size and selection for increased information content of acoustic signals liberated vocal systems from wingbeat and respiratory constraints, resulting in increased vocal plasticity early in the evolution of avian vocal learning.}

\normaltalk{From the Ground Up: Identifying Drivers of Nesting Success in Grassland Songbirds of the Northern Great Plains}{Jacy S Bernath-Plaisted\\Maureen D Correll\\Arvind O Panjabi}{Grassland songbirds of North America have declined rapidly as a group over the last half-century. Although these declines are generally understood to be a consequence of landscape-scale habitat loss and degradation, knowledge of regional and habitat-scale factors affecting grassland songbird demographics is often incomplete. An understanding of these demographic baselines and the covariates that influence them is critical to prescribing effective management regimes on remaining grasslands. Nesting success is an important demographic rate that may play a role in the long-term maintenance of avian populations. We monitored the nesting success of grassland songbirds in the Northern Great Plains at sites in western North Dakota (2015-2017) and eastern Montana (2016-2017). We located and monitored nests of four grassland specialist species, including Baird's sparrow (Ammodramus bairdii; n= 128), grasshopper sparrow (Ammodramus savannarum; n= 169), chestnut-collared longspur (Calcarious ornatus; n= 271, and Sprague's pipit (Anthus spragueii; n= 34). We analyzed the nesting success of these species using logistic exposure glm regression and found that baseline estimates fell within typical ranges for all four species; mean (CI range) nesting success rates across years and sites for these species were 34\% (16-50), 16\% (7-28), 36\% (28-43), and 33\% (14-53), respectively. To determine the drivers of these baselines, we also investigated the effects of regional climate and vegetation covariates on the nesting success of these species.}

\normaltalk{Habitat analysis of Black-capped Vireos at Salt Creek Canyon, Blaine County, Oklahoma}{Elizabeth M Besozzi\\John A Muller\\Joseph A Grzybowski\\Jeremy D Ross}{The Black-capped Vireo (Vireo atricapilla) is a small migratory songbird, listed as endangered by the U.S. Fish and Wildlife Service under the Endangered Species Act (1987). Habitat conversion to rangeland and concurrent increases in Brown-headed Cowbird (Molothrus ater) prevalence has restricted the historical range of V. atricapilla, establishing a modern northern margin in north-central Oklahoma. Salt Creek Canyon in Blaine Co., Oklahoma, contains the northernmost known modern population of V. atricapilla, which is relatively small and isolated, and has been monitored inconsistently since its discovery (the last available data is derived from a project ending in 2004). We spent May-July of 2017 surveying V. atricapilla territories, monitoring nests, and managing cowbird traps in Salt Creek Canyon. Approximately 30 males were identified, roughly half of which were accompanied by females. We used these territories and nest sites to conduct habitat analyses using GIS, incorporating various landscape features (e.g., canopy cover, slope, soil, habitat type). We extrapolated these associated landscape features to create a habitat suitability model and discover potentially unknown V. atricapilla habitat on nearby private lands. Encouragingly, our population estimates constitute an increase in the number of individuals present at Salt Creek Canyon over previous years; with the implementation of appropriate conservation strategies and more efficient management, Salt Creek Canyon may represent a valuable opportunity for V. atricapilla range expansion in Oklahoma.}

\normaltalk{Breeding biology and ectoparasites of Great Horned Owls in desert fragmented landscape of Baja California peninsula}{Raquel Bola\~{n}os-Garc\'{i}a\\Ricardo Rodr\'{i}guez-Estrella}{Habitat loss and fragmentation produce a decrease in prey availability and structures for breeding that can change the abundance and distribution of raptors. Also, parasites prevalence can change in fragmented habitat. In this work, we present our results on the breeding performance and ectoparasites prevalence in Great Horned Owl fledglings in fragmented and continuous desert habitat during three breeding seasons, 2014 to 2017. The density of active nests was one pair/25 km2 in fragmented and one pair/66 km2 in natural area. Nests were primarily located in giant cacti (Pachycereus pringlei). In fragmented area, breeding pairs started their reproduction earlier and were more successful than late breeders. Productivity was higher in fragmented area. An ectoparasites' prevalence of 53\% of nestlings was recorded in fragmented area, with the louse fly Icosta americana as a concern parasite. This is the first record of Great Horned Owl being a host for louse Colpocephalum pectinatum. The productivity of the Great Horned Owl in fragmented habitat is the highest reported for the species in desert ecosystems. We propose small fragments are important for maintaining populations for top order predators as the Great Horned Owl. However, caution should be taken in regard to health conditions because a high diversity of ectoparasites was recorded in this highly fragmented habitat.}

\normaltalk{Effects of beach renourishment on nest survival of Least Terns in Southeast Louisiana}{Sarah M Bolinger\\Melinda D Averhart\\Scott M Duke-Sylvester\\Erik I Johnson}{Beach renourishment projects, such as those sponsored by Louisiana's Coastal Protection and Restoration Authority, are expected to provide attractive nesting habitat for Least Terns because they increase the area of sparsely vegetated open beach. However, our previous monitoring efforts suggest that beach restoration has a mixed effect on nest success for these birds. Increased presence of mesopredators associated with restoration may lead to lower hatching success, and may offset the lower risk of nests being destroyed by floods and storm surge associated with increased beach elevation. We tested these predictions on Elmer's Island in Louisiana, in which the Caminada Headlands Phase II dune renourishment project was completed in late 2016. To quantify the effects of restoration on nesting success, we used program MARK to calculate daily survival rate (DSR) for 287 nests monitored in summer 2017, and modeled nest success with DSR evaluated at different intervals. Nest elevation, predation rates, and percent cover of vegetation surrounding nests were used as covariates, showing the complex and interrelated effects of restoration on nest survival. Such analyses are important in understanding whether attractive nesting areas on restored beaches serve as a population source or sink for nesting birds.}

\normaltalk{Variation in nest success and offspring development in a migratory songbird using natural and artificial nest-sites}{Andy J Boyce\\Harold F Greeney\\Charles van Riper III}{Artificial nest-sites are often used as conservation tools and can have positive effects on breeding density and population size, particularly in habitats where the density of natural nest sites has been reduced due to land-use change. Nest predation and overall nest success are often higher in birds using artificial nests, further highlighting their efficacy as a conservation tool. However, artificial nest sites also have different thermal properties than natural sites, including reduced insulation compared with natural sites. Nest site microclimate can have strong influences on development time, ectoparasite prevalence and fledgling mass, which all directly impact fitness. Therefore, understanding the developmental and demographic consequences of artificial nest-site use is critical to evaluating their utility as a conservation tool. Here, we explored differences in development time, nest success, ectoparasite prevalence and fledgling mass in Cordilleran Flycatchers (Empidonax occidentalis) using natural and artificial nest sites at two locations with distinctly different climatic regimes; Dolores, CO (37°N, 7500') and Summerhaven, AZ (32°N, 7500'). We predict reduced nest predation rates but prolonged development time and reduced mass at fledge at artificial nest sites, with greater differences between natural and artificial nests at our colder site in Colorado. Data were collected in Summer of 2017 and are currently being analyzed.}

\normaltalk{A camera-trap survey of avifauna in a Bornean peat-swamp forest}{Daniel M Brooks\\Adul \\Susan M Cheyne}{As part of an ongoing camera trapping project in the Sabangau Peat-swamp forest (Central  Kalimantan, Indonesia), we present camera trap data on the avifauna. Since 2008 the camera traps have captured  images/videos of a total of 224 individual bird photos represented by 142  unique events of 25 species, 12 of which are Endangered, Vulnerable, or Near-Threatened, all of which are described in species accounts. We describe natural history using quantitative data for Ciconia stormi, Spilornis cheela, and Centropus sinensis, and notes on  the virtually unknown Caprimulgus concretus. We also discuss important breeding and behavior observations for Spilornis cheela, Lophura pyronota, Blythipicus rubiginosus, and Trichixos pyrropygus. Expansions for range, habitat and migrant dates are discussed for Accipiter trivirgatus, Melanoperdix niger, and Pitta nympha. We discuss conservation implications, as  well as the usefulness of using long-term camera trap data to monitor birds.}

\normaltalk{OpenWings: Collaborative construction of a fossil-calibrated species-level bird phylogeny}{Robb T Brumfield\\Brant C Faircloth\\Brian T Smith\\Rebecca T Kimball\\Edward L Braun\\F K Barker}{Class Aves is the only extant lineage of theropod dinosaurs and the single most diverse clade of amniotes. Birds occur in almost every terrestrial environment, and as highly visually- and auditorily-oriented organisms with complex behaviors, elaborate morphologies, and impressive physiological adaptations, birds have been a major focus of modern scientific research. Consequently, birds are emerging as a leading model system for comparative biology, but still lack an accurate and complete phylogeny calibrated to their increasingly rich fossil record. We announce a new NSF-funded initiative, OpenWings, with the goal of producing a time-calibrated, species-level phylogeny of all birds by integrated analysis of phylogenomic and fossil data. This highly collaborative project will make all data publicly available through the project website (openwings.org) as they are generated and validated. We welcome inquiries from potential partners interested in sharing samples, data, or expertise.}

\normaltalk{Avian use of mature and revegetated riparian woodlands in the Lower Rio Grande Valley of Texas}{Timothy Brush\\John S Brush}{In the Lower Rio Grande Valley of Texas, remaining wooded riparian tracts with subtropical bird communities are small and often isolated in a matrix of agricultural and expanding urban areas. The US Fish and Wildlife Service and other agencies have revegetated extensive areas, but little evaluative analysis has been done. We studied bird use of existing mature riparian corridor forest habitats with revegetated habitats. In May-June 2013 and 2014, we did 47 point-counts in mature habitat and 45 in revegetated habitat generally ranging from 15-25 years in age. There was extensive overlap in the breeding bird community, with most species occurring at similar frequencies in mature and revegetated habitat. Species associated with closed-canopy riparian forest, such as Plain Chachalaca and Long-billed Thrasher, and those associated with mid-stature thorn forest, such as Green Jay and Altamira Oriole, also used the more successful revegetated tracts. Cavity-nesting species such as Golden-fronted Woodpecker and Black-crested Titmouse were strongly associated with dead trees and tall-canopy forest, more regularly in mature habitat. Less successful revegetated tracts with patches of woody plants and more open understory supported species like Common Ground-Dove and Buff-bellied Hummingbird. Mature and revegetated habitats were affected by an historic flood event in 2010, which reset plant and bird communities to a more open, early-successional condition. Overall, revegetation has allowed many riparian bird species to expand into areas of formerly unsuitable habitat, but studies of breeding success and site fidelity are needed to confirm this apparent success.}

\normaltalk{Pattern and structure of a suburban bird community in subtropical South Texas}{John S Brush\\Alexis Racelis\\Timothy Brush}{The effects of urbanization on avian community pattern and structure remains understudied in subtropical regions where bird diversity is relatively high. We explored the effects of local land-cover and habitat features on the bird community of suburban habitats in the subtropical Lower Rio Grande Valley of southern Texas. We documented 39 species during point-counts in residential neighborhoods and small urban woodlands. Bird diversity was positively related to increased tree cover, and negatively related to increased grass and impermeable surface cover. Granivorous and ground-foraging birds had the greatest relative abundances (53\% and 82\%), with most individuals representing only a few species. Insectivores made up 33\% of the species, but only represented 6.4\% of total relative abundance. Redundancy analysis suggested that native trees were positively related to more bird species than non-native trees were, and that trees had a greater effect than shrubs. The majority of insectivorous species were correlated with increased tree (particularly native) cover. The two most abundant urban exploiter species (House Sparrow and Great-tailed Grackle) were positively related to increased grass and impermeable surface cover. Of the regionally-distinct species we observed, White-tipped Dove, Long-billed Thrasher, and Olive Sparrow were limited to densely-vegetated urban woodlands, while others, such as Buff-bellied Hummingbird, Couch's Kingbird, Clay-colored Thrush, and Golden-fronted Woodpecker were also seen in suburban habitats. We conclude that suburban areas with greater tree cover and small urban woodlands are of significant conservation value for regionally-distinct forest birds, particularly in rapidly urbanizing areas in agriculture-dominated regions.}

\normaltalk{Double-digest RAD sequencing outperforms microsatellite loci at assigning paternity and estimating relatedness: a poof of concept in a highly promiscuous bird}{Bronwyn G Butcher\\Derrick J Thrasher\\Leonardo Campagna\\Michael S Webster\\Irby J Lovette}{Information on genetic relationships among individuals is essential to many studies of the behavior and ecology of wild organisms. Parentage and relatedness assays based on large numbers of SNP loci hold substantial advantages over the microsatellite markers traditionally used for these purposes. We present a double-digest restriction site-associated DNA sequencing (ddRAD-seq) analysis pipeline that, as such, simultaneously achieves the SNP discovery and genotyping steps and which is optimized to return a statistically powerful set of SNP markers (typically 150-600 after stringent filtering) from large numbers of individuals (up to 240 per run). We explore the tradeoffs inherent in this approach through a set of experiments in a species with a complex social system, the variegated fairy-wren (Malurus lamberti), and further validate it in a phylogenetically broad set of other bird species. Through direct comparisons with a parallel dataset from a robust panel of highly variable microsatellite markers, we show that this ddRAD-seq approach results in substantially improved power to discriminate among potential relatives and considerably more precise estimates of relatedness coefficients. The pipeline is designed to be universally applicable to all bird species (and with minor modifications to many other taxa), to be cost- and time-efficient, and to be replicable across independent runs such that genotype data from different study periods can be combined and analyzed as field samples are accumulated.}

\normaltalk{Conservation Through the Lives of Adirondack Loons: 20 Years of Collaborative Research and Outreach}{Valerie L Buxton\\Nina Schoch\\John Ozard\\Daniel Rosenblatt\\David C Evers\\Michale Glennon}{Research efforts on Common Loons (Gavia immer) in New York's Adirondack Park were initiated in 1998, borne out of concern for the toxic effects of mercury exposure on loon populations and the aquatic ecosystems they inhabit. Since this time, Adirondack loon research has expanded to encompass studies on multiple aspects of loon ecology, behavior, health, and conservation in the Park. Each breeding season, we monitor $>$ 90 Adirondack lakes for more than 250 uniquely color-banded loons and capture, band, and sample ≥ 10-15 loons on our study lakes to assess loon productivity, health, and mercury exposure. Additionally, we have used satellite telemetry and geolocators to identify migratory patterns of Adirondack loons, placed trail cameras at loon nesting sites to better understand incubation behavior and factors affecting nesting success, and rescued debilitated loons. Here we present a brief summary of this work, and highlight how our research has informed loon conservation and management in the Adirondack Park, and contributed to enhancing public awareness and understanding of wildlife conservation, utilizing the Common Loon as an indicator species.}

\normaltalk{Investigating the relationship between avian microbiota and stress physiology along an urban-rural gradient}{Sophia S Carryl\\Rachel Santymire\\Jack A Gilbert}{Deforestation, agricultural land use, climate change, rapid human population growth, and continued movement to urban centers force wild animals to adapt or face extinction. We seek to understand how anthropogenic stressors drive selection on wild hosts microbiome, which plays an integral role in establishing and maintaining host health. Specifically, whether avian hosts exhibit identifiable microbial plasticity when adjusting to an urbanizing world. We argue that selection on the avian microbiome may be attributed to host stress hormone levels, given (1) the survival and reproductive costs of elevated stress hormone levels, and (2) the bidirectional relationship between microorganisms and host endocrinology – microbial metabolites can influence host stress responses, and stress hormones alter the microbiome. We sampled house sparrow (Passer domesticus) body sites (eye, mouth, uropygial gland, and cloaca) and collected feathers for corticosterone analysis in populations living along an urban-rural gradient in Midwest, USA. Three study sites are defined by housing units per acre (Hu/acre): urban: $>$1000 Hu/acre; suburban: $<$1000 – 149 Hu/acre; rural: $<$149 Hu/acre. We profiled the microbiome using 16S rRNA amplicon sequencing and measured stress hormones using a corticosterone enzyme immunoassay. Subsets of samples with different microbial profiles will be further characterized using shotgun metagenomics to identify functional pathways that associate with host-stress, such as gamma-aminobutyric acid production. These results will address whether wildlife hosts are under strong natural selection to favor symbionts that improve resilience to stressors found in an urban world or indicate that the host and its microbial community are under divergent selective pressures.}

\normaltalk{Patrones de variación geográfica en algunas especies de aves asociadas al bosque seco tropical caducifolio del Pacífico de Mesoamérica}{Vicente J Castillo\\Adolfo G Navarro\\Luis A S\'{a}nchez-Gonz\'{a}lez}{In Central America, the seasonally tropical dry forest (STDF) has a continuous distribution along the Pacific Coast from Guanacaste, Costa Rica to Sonora, Mexico. In Mexico, represent the fourth largest biome and harbor more than 20 \% of birds species. Recent studies revealed that different species have high phylogeographic structure and monophyletic genealogies, however, the relative importance of the most accepted explanation (Pleistocene forest refugia hypothesis) to account for these patterns of genetic diversity and diversity of neotropical species is not yet conclusive. Our objective is evaluate a hypothesis of simultaneous divergence for four birds species associated with STDF, by means of comparative phylogeography and evaluate if time of divergence is coinciding with the location of stability-areas of the historical distribution, through Ecological Niche Models (ENM). We use mitochondrial DNA sequences to obtain haplotype-networks, phylogenetic trees and structure analysis and test 12 parameters combinations for each specie and transfer the best performed models to: 6,000, 22,000 and 120,000 - 140,000 years ago, for two different general circulation models. The results of the niche models suggest location of stability-areas are coincident with high genetic structure in three of the four species. The haplotype-networks and phylogenetic trees are congruent in exhibit high genetic structure and reciprocal monophyly.}

\normaltalk{Characterizing the Microbiota of the Avian Nest}{Michael A Castro\\Rachel L Nelson\\John Constable\\Tricia Van Laar}{Bird nests can serve as reservoirs for potentially harmful bacteria that can infect fledglings and their parents. Nests are often found within urban communities due to decreases in natural habitats and adaptation to foraging among urban settlements, but it is unknown how this anthropogenic association may impact the nest microbiota. The major goal of our study is to characterize the microbiota of American Bushtit (Psaltriparius minimus) nests found in and around Fresno, CA. We chose bushtits as they make a unique enclosed, hanging nest. We hypothesize the majority of species will be non-pathogenic and species composition will differ depending on the region of the nest sampled. Post-fledging, we used sterile swabs to sample nest openings and brood areas. DNA was extracted from swabs and the 16S rRNA gene was amplified and sequenced. Sequencing data was analyzed using QIIME2. During year 1, we sampled two nests, which demonstrated significant differences in the microbiota of openings compared to brood areas. However, we hypothesized these data may be confounded by the presence of three dead nestlings in one of the nests. In this brood area, we identified pathogenic and antimicrobial resistant bacteria. During year 2, the data did not support our hypothesis of the opening being different from the brood area. Instead, parts of an individual nest were more closely related to one another than the same part in another nest. We are continuing to locate and sample nests in order to generate a more complete profile of the bushtit nest microbiota.}

\normaltalk{Post-breeding period movement and behavior of adult black-capped vireos (Vireo atricapilla)}{Paula M Cimprich\\Joseph A Grzybowski\\Jeffrey F Kelly}{Movement and behavior are not well documented for most adult migratory birds during the post-breeding period, the time between the end of nesting and fall migration. However, this period may provide opportunistic social cues and lack of breeding obligations for individuals that unsuccessfully produced offspring during the main breeding period and seek to increase their fitness. We examined black-capped vireos (Vireo atricapilla), with emphasis on males in their first breeding cycle, at Fort Sill, Oklahoma from April to September 2017 to study movement and behavior in the post-breeding period. Previous anecdotal information suggests higher movement rates of males in the post-breeding period compared to restricted territories during the main breeding period. We monitored 55 territorial males using a combination of re-sighting color-marked individuals and radio telemetry to monitor productivity and quantify territory sizes and movement rates in during the main breeding and post-breeding periods. Vegetation and arthropod surveys were also conducted to ascertain habitat selection. Our results suggest age and reproductive status are correlated with territory size and movement rates but also indicated a temporal component. These data suggest the post-breeding period is being used for more than survival and molt for birds of a certain age and breeding status.}

\normaltalk{Exploring a Potential Hybrid Zone of Blue-Capped Motmots and Lesson's Motmots}{Miles D Collins\\Olivia A Stevens\\John E McCormack\\James M Maley}{Two species that were recently split from the Blue-crowned Motmot complex occur in northeastern Mexico. Blue-capped Motmots (Momotus coeruliceps) and Lesson's Motmots (Momotus lessonii) both occur in Veracruz, but the boundaries of their respective ranges are not clear. We analyzed 116 specimens in the Moore Laboratory of Zoology collection of both species. We conducted a morphological analysis and qualitatively scored the crowns on a one (blue) to five (black). We found apparent hybrids ranging from west of Laguna Tamiahua, Veracruz south over 400 kilometers to Soyaltepec, Oaxaca. Multiple birds had crowns with feathers that were blue at the base but black on the tips. This suggests a rather large hybrid zone between these two species. The hybrid zone requires further study, but apparently, species limits need to be reevaluated between these two lineages.}

\normaltalk{Fifty years of Black-throated Blue Warbler migration in North America}{Kristen M Covino\\Kyle G Horton\\Sara R Morris}{The North American Bird Banding Laboratory maintains an unparalleled dataset useful to investigate numerous aspects of avian biology at a variety of scales. We used these banding records to investigate geographic and temporal migratory patterns of the Black-throated Blue Warbler. In both seasons, young birds represented a higher proportion of individuals captured in the periphery compared to the core of the migratory range. During spring migration, males migrated 5.5 days earlier than females and adults migrated 2.9 days earlier than young individuals. During fall, there was no difference in the passage timing between males and females but young birds migrated 2.3 days earlier than adults. Additionally, migration proceeded more quickly in the spring, at a pace of 1.1 days per degree latitude, compared to the fall, 1.6 days per degree latitude. Analyses across the 50-year period indicated that, in spring, early migrants passed through sites approximately 1.8 days earlier per decade, but there was no change in passage date for late migrants or overall spring migration. During fall, the overall timing of migration stayed consistent across years, but the migratory season is becoming protracted: early migration is 1.8 days earlier per decade and late migration is 0.9 days later per decade. These results indicate a lengthening of the Black-throated Blue Warbler's breeding season. This framework lends to testable system hypotheses, including whether migration distance, foraging guild, body size, migration route, etc. drive differential responses. We must utilize these unique data to understand critical shifts at relevant scales of migration.}

\normaltalk{Electronic Transponder Analysis Gateway (ETAG): An Animal Behavior Observatory for Radio Frequency Identification Technology}{Claire M Curry\\J E Ruyle\\Eli S Bridge}{Radio Frequency Identification Technology (RFID) is a widespread technology, familiar to most from RFID tag use as “microchips” implanted in pets or in electronic “toll tags” on highways. A growing community of researchers uses this short-range wireless technology to track animal movements, particularly birds, but also including mammals, fish, reptiles, and even insects. Use of RFID generates large datasets, often from multiple readers or geographic locations. However, at present most researchers manage their own data. Here we present an informatics observatory, the Electronic Transponder Analysis Gateway (ETAG), to store and visualize RFID tracking data in a standardized format. ETAG will help researchers manage and collaborate with the large volumes of data produced by RFID technology in animal behavior studies by providing a comprehensive cyberinfrastructure support for this otherwise mature technology. The user uploads two metadata files containing reader locations and animal tagging data (including all data that would be collected during banding, as the database supports flexible numbers of fields) plus tag reads from each reader. Visualization on the ETAG website can be filtered by reader, tag, species, and other fields in addition to Application Programming Interface (API) support for featuring maps of your readers and tags on your own website. Adoption of ETAG will allow researchers to focus more on science and less on data management.}

\normaltalk{Song variation between House Wrens living in urban and rural areas on the island of Grenada}{Marie-Eve Cyr\\Nicola Koper}{The House Wren (Troglodytes aedon) has one of the largest habitat range among North American bird species. Although House Wrens has been the subject of much research in North America, little is known on the populations of South and Central America where multiple subspecies and resident populations are found. The House Wren population on the island of Grenada is resident and, due to human activities, urban areas are expanding rapidly. Bird species living in urban areas often suffer negative effects through frequency masking when unable to adapt to their new acoustic environment. To avoid masking Grenada House Wrens should adjust part or all of their song in environments with different levels of noise by increasing the minimum frequency and trill rate in urban (noisier) environments. We recorded 37 birds at 11 sites throughout the island, 6 in urban areas and 5 in rural areas. Spectrogram analyses were performed using RavenPro 1.5. House Wrens should avoid masking because of the importance of songs and calls in their behaviors. House Wrens are known for their complex singing and extensive repertoires, which suggest an inherent song plasticity. Furthermore, Grenada being a small island with high levels of anthropogenic pressure, assessing the potential of its resident populations for adaption is an important step towards better conservation and management strategies.}

\normaltalk{Patterns in habitat selection between breeding and wintering grounds of Neotropical migrant passerines}{Maryanne E Dantzler-Kyer\\Michael A Patten}{Migratory animals must select suitable habitat in different locations to take advantage of abundances of resources and escape harsh climates during seasonal changes. Niche tracking occurs when migratory animals fill the same niche space on their breeding and wintering grounds. Niche space overlap is expected because of limiting ecophysiological or behavioral needs, though resources available on native and migratory ranges may differ. We assessed whether Neotropical migrants tracked niche space between their breeding and wintering grounds. We measured microclimate variables and vegetation structure associated with five species of Neotropical migrant birds in the Ouachita National Forest (breeding) in southeastern Oklahoma and La Milpa Field Station in Orange Walk, Belize (wintering). Using a Bayesian approach, we determined that patterns in habitat selection were apparent between the breeding and wintering grounds for all five focal species. More species tracked temperature and light than other microclimate variables measured, while canopy cover was tracked by more species than other vegetation structure variables measured. Finer scale investigation of niche tracking will further our understanding of which physiological constraints and resources contribute to habitat selection in breeding and wintering grounds of migratory bird species. Furthermore, future studies will allow us to predict future distributions in the face of major environmental changes.}

\normaltalk{Make new fields but keep the old: Nesting success of Dickcissel in Arkansas remnant and restored tallgrass prairies}{Alyssa L DeRubeis\\Kimberly G Smith}{Of all North American bird groups, grassland species have experienced the most precipitous declines in part due to significant habitat loss. As a remedy, some land managers are restoring prairies to increase available grassland bird habitat. While studies have shown that breeding bird populations between remnant and restored prairies are similar, none have specifically analyzed nesting success of grassland birds in these two habitat types. Potential nest success factors like arthropod diversity and percent vegetative cover vary between remnant and restored prairies. How suitable are these newly created sites for nesting grassland birds? In the summer of 2017, field volunteers and I used behavioral clues to detect 40 nests of a locally abundant (but overall declining) grassland bird, the dickcissel (Spiza americana), in two remnant sites and three restored sites in northwest Arkansas. We then equipped 14 of these nests with infrared video surveillance cameras to record nest outcome, predation/abandonment rates, and feeding rates. We also measured vegetative composition, percent visibility, and litter depth at nest and random sites. Other factors I will consider in analysis include nest density, edge habitat, parcel size, and the surrounding landscape. After I analyze the data, my study will show how valuable restored prairies are for nesting dickcissels.}

\normaltalk{Modeling Nest Predation and Cowbird Parasitism Rates of Bell's Vireo in North East Texas}{Hannah S Dill\\Natasha R Lehr\\Johanna Delgado-Acevedo\\Dean Ransom}{Neotropical migrants as a group have experienced long term population declines, which in some species may be due to elevated rates of nest predation and brood parasitism by brown headed cowbirds (Molothrus ater) in human fragmented habitats. The Bell's Vireo (Vireo bellii bellii) is a neotropical migrant that experiences elevated parasitism and nest predation rates. This songbird breeds throughout the central and southwestern US. Unlike many states in this species range, Bell's vireos populations are increasing in Texas. In 2016 and 2017, we studied the nesting ecology of a recently established breeding population of Bell's vireo on two sites in north east Texas. We quantified cowbird parasitism and nest predation rates and modeled these rates using nest site vegetation data using logistic regression in an information-theoretic model selection framework. We described 10 variables associated with the nest site with means ± 95\% CI. We located 107 nests, 77 of which were of known fate. Forty seven nests (61\%) failed; 28 (36\%) and 19 (25\%) nests were depredation and parasitized, respectively. Nest parasitism was best modeled with lateral concealment to the nest, distance of nest to canopy edge of nest tree, and the distance from the ground to rim of the nest (wi = 0.198). Nest predation was best modeled with visual obstruction at the nest using the Robel cover pole technique (wi =0.114).}

\normaltalk{Examining temporal changes in morphology, population dynamics, and wind migration patterns of raptor species migrating through Cape May, New Jersey}{Calum R Dixon\\Nicholas A Baer}{Long-term datasets enable researchers to investigate raptor population and morphological trends over time to better understand potential temporal variation in particular species. The objectives of this research were to assess long-term population trends in endemic raptors in northeastern North America, to understand changes in raptor morphology and to evaluate the influences of wind patterns on annual and interannual species migration through Cape May, NJ. Specifically, this study analyzed 49 years of morphological data collected from the Cape May Raptor Banding Project in Cape May, NJ. Data was analyzed to assess wing chord changes over time for raptor species. Additionally, the study analyzed 40 years of population data compiled by HawkCount.org from Cape May, NJ. Results indicated five species of raptors (Cooper's hawk, Sharp-shinned hawk, American Kestrel, Peregrine falcon, and Norther Harrier) exhibited a significant reduction in wing chord length in both male and females over time. The data illustrated fluctuations in migrating populations. Sharp-Shinned Hawks and American Kestrels showed strong downward trends, whereas Peregrine Falcons and Cooper's Hawks displayed increases in migrating numbers. While our research found significantly higher daily raptor migration through Cape May, NJ during days with north and west winds (p$<$0.0001), wind direction did not account for interannual migration abundances. Temporal changes in raptor morphology and population can be indicative of changes in land cover, climate or human built environments. Further research examining land use changes may help refine our understanding of species specific impacts.}

\normaltalk{Non-breeding ecology of a declining grassland bird in agricultural landscapes: the loggerhead shrike}{Emily R Donahue\\Michael D Collins\\Lee Bryant\\Alexander Worm\\Alix Matthews\\Jacob Wessels}{As native grasslands have been transformed by agricultural development (ranging from family farm mosaics to intensive monocultures) and forest succession, many grassland bird populations in North America are experiencing severe declines. The Loggerhead Shrike is one species that has severely declined since the 1960s and is now listed as endangered, threatened, or special concern in multiple states (and federally endangered in Canada). However, the specific factors and regions most responsible for this decline remain unclear. The landscape of the Lower Mississippi Alluvial Valley (LMAV) is dominated by intensive agriculture but still provides habitat for two Loggerhead Shrike subspecies (one migratory and one sedentary). However, the winter ecology of shrikes in the region is acutely understudied, and virtually nothing is known about important habitat features and/or if these features differ between shrike subspecies during this season of co-existence. To fill these gaps, we conducted occupancy, behavioral, and larder surveys along with banding and monitoring to estimate detectability, survival, site fidelity, and body condition of shrikes in the LMAV of Arkansas from 2016-2018. We estimated relatively high occupancy (23\%), high apparent annual survival ($>$60\%), and site fidelity (47\% within 1 km of previous winter banding site). We documented and estimated foraging attempt and success rates; perch, larder, and hunting substrates; and diet composition (based both on behavioral surveys and larder contents). Our results will aid in understanding how to best manage these agricultural systems to continue providing appropriate features and potentially even improve sustainability of shrikes in this unlikely landscape.}

\normaltalk{Assessment of western Burrowing Owl (Athene cunicularia) translocations from urban development sites to artificial habitat in the Phoenix, Arizona area}{Dejeanne Doublet\\Martha J Desmond\\David H Johnson\\Fitsum A Gebreselassie}{Translocation as a conservation tool is increasingly being utilized to mitigate the effects of land conversion and development on wildlife. The western Burrowing Owl (Athene cunicularia hypugaea) is a declining species that readily nests in anthropogenic landscapes and has been the subject of translocation efforts. We are assessing an ongoing Burrowing Owl translocation program in the Phoenix, Arizona area to answer two key questions: (1) 'How well do translocation programs work for Burrowing Owls?' and (2) 'How can we improve the success of these programs?' From 2017-2019, we will track survival, movement, and nesting of 80 adult owls (40 translocated, 40 non-translocated) with the use of VHF radio-telemetry. In 2017, male survival was lower for translocated (0.34) than non-translocated males (0.66), non-translocated females (0.64), and translocated females (0.69; 43 weeks). We constructed a set of joint live-dead encounter models assessing the importance of translocation and sex on survival and found the top-ranking model to be one in which survival differs for translocated males only. Translocated owls were less likely to establish nests, and those that did had lower nest survival (0.32) than non-translocated nests (0.85; P$<$0.001, Fisher's exact test) and lower productivity (2.00 fledglings/nest) than non-translocated nests (2.85 fledglings/nest; t = 2.5103, P$<$ 0.025). Our results suggest that translocation poses challenges for Burrowing Owls who have limited success for first year releases. Stress from the translocation process (i.e. 8-10 owls held in release tents), site location, and competition may be contributing to these results.}

\normaltalk{Moore Laboratory Specimens and Field Notes as Tools for Citizen Science: Assessing A Century of Environmental Change}{Betty Du\\James M Maley\\Whitney LE Tsai\\John E McCormack}{Natural history collections provide a historical snapshot of human-caused habitat impacts, for these collections contain a wealth of information on past biodiversity patterns. The Moore Laboratory of Zoology is a glimpse of Mexico's bird diversity from 1933 to 1955 as the largest Mexican bird collection in the world and can aid in comparing past and contemporary bird distribution patterns. The major collector of Moore Laboratory's specimens, Chester C. Lamb, wrote extensive field notes that described the locations and the birds he saw and collected. By comparing Lamb's historical field notes with contemporary data from modern eBird records, we can assess bird distribution patterns changes and investigate reasons behind those changes. Data from one site, El Naranjo in San Luis Potosí, Mexico, contain information on birds collected and seen over 19 days in 1942 to present day eBird records. While many more species were noted in modern records, 181 species compared to Lamb's 94, this occurrence is likely due to increases in observations and microhabitats explored. We documented several cases of species losses, especially among raptors and game birds. Lamb recorded sixteen species, including Crested Guan, Bicolored Hawk, Collared Forest-Falcon, Black Swift, Canyon Wren, and Greenish Elaenia, which were not observed modern records. These potential losses are likely attributed to deforestation and landscape alternations for arable land since the 1940s. Analysis at this site provides a framework for using Lamb's specimens and lists and citizen science data to document bird distributional change for all of Mexico.}

\normaltalk{Variations in social network structures between eruption and non-eruption years in two chickadee species}{Matthew N Dula\\Rebecca D Garlinger\\Robert L Curry}{Social networks have recently become an important tool for understanding social behaviors in animals. Social networks are beneficial for understanding how individuals are affected by the social structure of a community and how individual behavioral phenotypes can mediate natural selection. We use social networks to understand winter season interactions between two species of chickadees, Carolina and Black-capped chickadees. In years where supplies are limited in Black-capped Chickadee territories, young birds will disperse southward into Carolina Chickadee territories, called eruption years. We seek to determine the differences in social structures via social network analysis in Carolina Chickadee populations in eruption years and non-eruption years. We have roughly 60,000 data points from 49 birds from the 2016-2017 eruption year and roughly 14,000 data points from 13 birds so far from the 2017-2018 year of only Carolina Chickadees. The social network metrics, strength (0.156 – 4.48 to 0.648 – 2.46) and degree (9 – 35 to 7 – 12) were greater in the eruption year while centrality (3.82e-6 - 0.888 to 0.181 – 1.00) was greater in the non-eruption year. Social networks can help us understand how the social structures of winter flocks change in Carolina Chickadees when Black-capped Chickadee eruption individuals are introduced and how the two species interact.}

\normaltalk{Effects of changing wind speed on the breeding success of an aerial insectivore}{Peter O Dunn\\Linda A Whittingham}{Climate change is often thought of in terms of increasing temperatures; however, wind speed is also changing and impacting the environment. Surface wind speeds have decreased in the Northern Hemisphere over the past 30 years, but little is known about how this impacts wildlife. In this study examined the effect of decreasing wind speed on the breeding success of tree swallows, which feed primarily on flying insects. We analyzed data collected on wind speed, flying insect availability and swallow reproductive success over 21 years (1997-2017) at the University of Wisconsin-Milwaukee Field Station in southeast Wisconsin. We found that cumulative wind speed during the breeding season has decreased over the past 21 years, likely increasing food availability for swallows. Tree swallows laid their eggs earlier when it was less windy and nestlings weighed more in less windy conditions. Thus, wind speed is decreasing at this location and it is affecting the timing of breeding and potentially the reproductive success of tree swallows.}

\normaltalk{Local meteorological conditions reroute a migration}{Joseph M Eisaguirre\\Travis L Booms\\Christopher P Barger\\Carol L McIntyre\\Stephen B Lewis\\Greg A Breed}{Understanding how species respond to changing abiotic conditions is increasingly important as climates change. For migrating animals, realized migration routes and timing emerge from hundreds or thousands of movement decisions made along migration routes. Local weather conditions along migration routes continually influence these decisions, and even relatively small changes in en route weather may cumulatively result in major shifts in migration patterns. Here, we analyzed satellite tracking data to score a discrete navigation decision in a large migratory bird, Golden Eagle Aquila chrysaetos, as it navigated a high-latitude, 4500-m elevation mountain range to understand how those navigational decisions changed under different weather conditions. Using Bayesian hierarchical models, we showed that wind in a particular area along the migration pathway drove this major navigational decision; conditions encountered at the migration route break point predictably resulted in migrants routing either north or south of the mountain range. With abiotic conditions continuing to change globally, simple decisions, such as the one described here, might additively emerge into new, very different migration routes.}

\normaltalk{Modeling pesticide risk to California Gnatcatchers}{Matthew A Etterson\\Nathan Schumaker\\Kristina Garber\\Steven Lennartz\\Andrew Kanarek\\Jennifer Connolly}{Pesticides are used widely in US agriculture and may affect non-target organisms, including birds. Recently, USEPA has worked with other federal agencies, including USFWS and NMFS, to revise and strengthen methods for conducting pesticide risk assessments under section 7 of the U.S. Endangered Species Act. Three existing USEPA models, the Terrestrial Investigation Model (TIM), the Markov Chain Nest Productivity Model (MCnest), and the HexSim modeling environment were modified to develop an integrated model for avian pesticide risk assessment. The model is parameterized using only data currently available under the standard pesticide registration process, together with life history data available in the scientific literature. The model will be demonstrated by simulating potential pesticide effects on the federally threatened California Gnatcatcher (Polioptila californica) in the U.S. portion of the species range. Two pesticides, malathion (organophosphate), and lambda-cyhalothrin (pyrethroid), applied to wheat crops under varying spatially explicit usage consistent with the labeling for the two pesticides were modeled. Model-predicted declines in gnatcatcher abundance and changes in the distribution of the species following applications of each pesticide will be highlighted. The integrated TIM/MCnest/HexSim model should allow chemical risk assessors to evaluate spatial and temporal dynamics that are important to understanding population persistence in complex spatial landscapes with multiple stressors.}

\normaltalk{Influence of behavior, landscape features, and spatial scale on Veery breeding site occupancy and preference in NY, USA}{Elizabeth A Farley-Dawson\\Kenneth A Schmidt}{The Veery (Catharus fuscescens) commonly nests near wetlands, but it is unclear whether this is due to local or macro-scale habitat attributes. The extent that social factors such as conspecific attraction contribute is also uncertain. We assessed whether long-term breeding site preference (LTSP) was influenced by habitat and behavioral factors using a long-term (2006-2015) nesting study from New York. We hypothesized that if LTSP was habitat-driven, then high LTSP sites would be positively associated with the proportion of that site covered by wetlands (WP), and high LTSP sites would cluster around wetlands. If LTSP was socially driven, clusters of high LTSP sites should occur away from wetlands. Spatial autoregressive modeling of 6 variables (elevation, aspect, slope, WP, road length, proportion of successful nests) on focal sites showed that in only one of three study areas did high LTSP sites contain higher WP. High LTSP sites clustered in this area (Moran's I = 0.324, p $<$ 0.001). We examined summarized metrics at the areal, neighborhood, and focal site scales using generalized linear modeling. The neighborhood model was most supported (K = 7, wi =0.89), with high LTSP corresponding with high focal WP, and LTSP of neighbor cells, but not neighbor WP. Thus, we found overall support for the habitat hypothesis at a focal site scale, but at a neighborhood scale, occupancy patterns may be socially driven. Collectively, these spatial patterns expose synergistic effects of behavior and landscape features to shape occupancy and habitat preference at multiple spatial scales.}

\normaltalk{Tracking understudied songbirds in the desert: the post-fledging period of Gray Vireos}{Sarah E Fischer\\Kathy Granillo\\Henry C Stevens\\Claire E Nemes\\Henry M Streby}{Gray Vireos (Vireo vicinior) are short-distance migratory songbirds that breed primarily in piñon (Pinus spp.) –juniper (Juniperus spp.) systems in the southwestern United States and northwestern Mexico. Compared to many songbirds in North America, Gray Vireos are understudied throughout their range, making it challenging to develop and implement conservation plans. Additionally, most studies of Gray Vireo demography focus on the nesting stage (i.e., nesting success as a measure of productivity). However, incorporating data from the post-fledging period is essential to determine full-season productivity and habitat associations. We monitored nests (n=36 in 2016; n=51) at Sevilleta National Wildlife Refuge in central New Mexico, USA. Nest success was 24\% and 25\% in 2016 and 2017, respectively. Brown-headed Cowbirds (Molothrus ater) parasitized 38\% and 43\% of Gray Vireo nests in those years, respectively. In 2017, we used radio telemetry to track the daily movements of 32 fledglings to determine post-fledging space use, survival, and habitat features associated with those parameters. Fledgling survival was 66\% during the dependent stage (i.e., between nest departure and independence from adult care). The mean distance between daily movements was 241.1 m (± 235.4 SD), and the maximum observed daily distance was 1.4 km. These movements increased with age. Like other fledgling studies, the majority of mortality occurred during the first week post-fledging. Our study highlights the importance of monitoring fledgling songbirds when considering full-season productivity and management decisions.}

\normaltalk{Variations of Bird Communities in Early Successional Secondary dry forest versus Late Secondary dry forest in Salinas, Puerto Rico}{Soely E Luyando Flusa\\Waleska Vazquez-Carrero\\Fred C Schaffner Gibbs}{This study investigates how forest composition may affect the abundance and diversity of resident and migratory passerine avian communities on the south coast of Puerto Rico at the Jobos Bay National Estuarine Research Reserve (JBNERR). Site 1, (“La Poza”) consists of early successional secondary dry forest, heavily dominated by a near monoculture of mesquite (Prosopis juliflora), while Site 2 (“El Corredor”) is an interhabitat corridor between mangrove (mostly black mangrove Avicennia germinans) and more mature secondary dry forest. We used mist netting to evaluate the species presence and abundance at each site during the months of January through April (migratory season) and May through July (non-migratory season), using the Shannon-Weiner and Rank-Abundance indices of diversity, and the Simpson Index of Dominance. The Shannon-Weiner Index indicated higher biodiversity at the interhabitat corridor Site 2 (H = 2.2831), with greater species richness and evenness, versus the early successional dry forest Site 1 (H =1.8820); similar results were revealed in the Rank Abundance curves. Contrarily, the Simpson Index indicated higher species dominance in the early successional Site 1 (C = 0.2144) versus the more mature secondary dry forest Site 2 (C=0.1428). With the results shown in Site 2, we consider that the complex vegetative structure and diversity provides more diverse niche opportunities for both nesting resident and intercontinental migratory land birds. Our ongoing research and monitoring efforts provide an opportunity for translational ecology to inform conservation measures and management actions to enhance both vegetative and avian biodiversity.}

\normaltalk{Completeness of Digital Accessible Knowledge of the Birds of Western Africa and Priorities for Survey and Inventory}{Benedictus B Freeman\\A T Peterson}{Primary biodiversity data are widely used to model ecological niches and geographic distributions of species, assess the status of global biodiversity, and inform conservation decisions. However, these data are often unavailable for some taxa or unevenly distributed across regions and landscapes. This study aims to identify current gaps in the current knowledge of West African birds that can guide future avian surveys and inventories across the region, and to inform conservation decisions. We examined Digital Accessible Knowledge (DAK) of the birds of Western Africa, available online from the Global Biodiversity Information Facility (GBIF) and eBird. The data were standardized and cleaned carefully, and inventory completeness indices were calculated at 0.5o, 0.3o and 0.1o spatial resolution were calculated across the region. Well-known sites were identified as those grid squares with completeness indices above 80\% and $>$200 associated DAK records. We identified 81 well-known pixels at 0.1o, 73 at 0.3o, and 63 at 0.5o. Well-known sites were notably clustered around accessible areas (e.g., cities). Countries holding more well-known sites were Ghana, Cameroon, Gambia, and Ivory Coast. Our results show the biases and gaps in West African bird data, and identify areas to be prioritized in future surveys and inventories.}

\normaltalk{Segregation vs overlapping, small raptors diet in a fragmented desert in Baja California Peninsula, Mexico}{Mart\'{i}n G Frixione\\Ricardo Rodriguez-Estrella}{The Burrowing Owl (Athene cunicularia) and the American Kestrel (Falco sparverius) are small size raptors, generalists and widely distributed in the American continent. We collected and analyzed pellets during the breeding season (A. cunicularia, n = 135; F. sparverius, n = 59) from a fragmented habitat with agriculture matrix in the sarcocaulescent scrub vegetation in arid ecosystem of Baja California peninsula. We estimated frequency of occurrence (O\%), items frequency per pellet (N\%) and ingested biomass (B\%). A. cunicularia preyed upon mostly on arthropods (O\%: 94.2\%, mostly orthopterans) and rodents (O\%: 3.9\%), although rodents contributed the most in biomass (36.1\%). F. sparverius preyed upon mostly on arthropods (O\%: 67.8\%, mostly orthopterans) and reptiles (O\%: 15.5\%), and reptiles contributed most of the prey biomass (64.7\%). Vertebrates biomass in Burrowing Owl diet (66.9\%) was lower than in American Kestrel diet (95.5\%). Diversity (A. cunicularia: H' = 1.25, F. sparverius: H' = 2.14) and niche breadth (A. cunicularia: Ba = 0.07, F. sparverius: Ba = 0.25) was lower comparing with other studies. Pianka's index showed an important diet overlap by order (Oij = 0.84) and class (Oij = 0.96) frequencies but decreasing by order biomass (Oij = 0.59). Species found in pellets were mostly the same in this fragmented habitat but different proportions could evidence segregation by day–lapse activity, extending Burrowing owl preying activity during crepuscular time.}

\normaltalk{Genomic characterization of the rosy-finch (Leucosticte spp.) species complex}{Erik R Funk\\Garth M Spellman\\Kevin Winker\\Erika S Zavaleta\\Kristen C Ruegg\\Scott A Taylor}{Alpine and arctic ecosystems are among the most rapidly impacted environments from climate change. Their young age and difficult accessibility impede field work and contribute to their understudied status. Successful conservation of these ecosystems requires accurate documentation of standing biodiversity and an understanding of the processes involved in generating and maintaining it. Rosy-finches are a complex of charismatic bird species that inhabit these extreme environments and exhibit extensive plumage variation coupled with low levels of genetic variation. The number of species within the complex is debated, making it excellent for studying the forces involved in the process of diversification, as well as an important group within which to clarify species boundaries and biodiversity. We are sequencing whole genomes from 68 individuals to 1) document genomic divergence in the rosy-finch species complex and clarify species boundaries and to 2) examine the genetic basis of plumage differences between the taxa within each of these groups, the genetic controls of which likely played important roles in diversification. We will examine genetic variation across species and populations using FST, a measure of genetic dissimilarity, and assess ancestry proportions using NGS Admix. We will also use a Genome-wide Efficient Mixed Model Association (GEMMA) approach to identify genetic variation potentially associated with plumage color differences. Broadly, this research will lend insight into patterns and processes of species divergence at high latitudes and high elevations. Further, evaluating the distinctness of these evolutionary lineages will help inform conservation decisions in environments that are highly affected by climate change.}

\normaltalk{Twenty-Five Years of Change in Spruce Grouse Occupancy at Their Southern Range Margin}{Christopher J Gilbert\\Erik J Blomberg}{Populations along southern range margins are often exposed to fragmented habitat and less suitable climatic factors compared to populations within the center of the species' range. Spruce Grouse (Falcipennis canadensis) are conifer obligates, endemic to the boreal forests of North America. At their southern range margin, Spruce Grouse populations are scattered and at risk of local extinction due to the lack of suitable habitat where coniferous forests interface with deciduous forests, as well as occurring at the southern end of their bio-climatic envelope. Occupancy and abundance can be affected by changes in forest succession and climate. I resurveyed 18 black spruce-tamarack forest stands within Mount Desert Island, Maine, that were originally surveyed during the early 1990s (Whitcomb et al. 1996) to examine changes in spruce grouse population dynamics over 25 years. I measured occupancy and abundance using callback surveys during the breeding season, and I also recorded micro-climate covariates (temperature and humidity) using Thermochron iButtons. I detected 7 Spruce Grouse in 4 stands, whereas 40 individuals were detected in 8 stands during the 1990s. I analyzed the effects of stand size, average stand humidity, and the average daily maximum stand temperature on occupancy using program R. I found that stand size effected occupancy, while relative humidity and average daily maximum temperature had no statistically significant effect. To preserve populations at southern range margins, larger patches of habitat are likely needed to mitigate the effects of habitat loss.}

\normaltalk{Hatching a plan for plover conservation: Predator exclosures increase Wilson's Plover (Charadrius wilsonia) nest success}{Lauren Gingerella\\Abby Sterling\\Scott Coleman\\Timothy Keyes\\Robert Cooper}{The effectiveness of predator exclosures on nest success of Wilson's Plovers (Charadrius wilsonia) was studied on Little St. Simons Island, Georgia. Predation is a leading cause of nest failure among shorebirds, and a threat that limits the distribution of this species of high concern (according to the U.S. Shorebird Conservation Plan). Although studied with other plovers, predator exclosures have been studied minimally with Wilson's Plovers. Exclosures were deployed at nest sites within the first three days of incubation at every other discovered nest. All nests were monitored until they either failed or eggs hatched, and trail cameras monitored exclosed nests as well. Exclosed nests had higher daily survival rate (n=18, DSR=0.984, SE=0.446) than unexclosed nests (n=41, DSR=0.959, SE=0.227) in 2016, and the treatment effect was significant (p=0.032). Results were similar in 2017 as exclosed nests (n=23, DSR=0.993, SE=0.5) again had higher survival than unexclosed nests (n=27, DSR=0.931, SE=0.219). No vertebrate predators depredated exclosed nests, but ghost crabs (Ocypode quadrata) were responsible for partially and fully depredating nests in both treatment groups. No adult mortality was observed during the study. Results show that predator exclosures increase nest success of Wilson's Plovers, but hatchling mortality remains high, so that the method may be of limited value for increasing overall productivity.}

\normaltalk{Grassland community responses to habitat structure and restoration management actions}{Alex J Glass\\Michael W Eichholz\\Stan McTaggart}{The Northern Bobwhite (Colinus virginianus, hereafter bobwhite) has experienced steady population declines across its range since the 1960's, despite being one of the most widely researched and managed species in North America. This study seeks to improve management policy for bobwhites and other grassland-dependent birds by identifying the mechanisms within the grassland community that drive bird productivity. We anticipate that bobwhites and grassland songbirds are influenced by the same mechanisms, thus identifying them will allow us to inform management practices that benefit bobwhites as well as other grassland birds, many of whom have also experienced large population declines in recent decades. We will monitor invertebrate and small mammal abundance, snake and mesopredator activity, songbird nest success and density, and vegetation structure and composition as they respond to grassland restoration efforts in the context of varying habitat structure. Specifically, we will consider the effects of vegetation species composition, patch size, interior-to-edge ratio, surrounding habitat type, and landscape cover on these different facets of the biotic grassland community. This study will take place on restored grassland patches at Burning Star State Fish and Wildlife Area in southern Illinois. We expect that productivity for both bobwhites and grassland songbirds will generally be higher in patches of larger size and interior-to-edge ratio, and smaller percentage of tree cover in the patch and surrounding landscape. We also expect nest predation rates to exhibit a unimodal relationship with small mammal abundance, with lowest predation rates coinciding with a moderate abundance of small mammals.}

\normaltalk{Why Every Ornithologist Needs a Social Scientist Collaborator}{Ashley R Gramza\\Ashley A Dayer}{Bird conservation fundamentally includes humans, and the most successful conservation actions are those aligned with the values, well-being, and perspectives of people. Therefore, studying and understanding the human aspects of bird conservation is essential for developing effective bird conservation strategies that address challenges for birds identified by ornithologists. Human dimensions (HD) is a field of study that applies the social sciences to examine research questions that have implications for wildlife conservation efforts. Combining expertise from the fields of human dimensions and ornithology through interdisciplinary research can improve bird conservation approaches and outcomes, and leverage opportunities for research through a coupled-human natural systems approach. Biologists and ornithologists are increasingly employing social science methods in their projects to address pressing conservation issues; these efforts will be further advanced by collaborations with social scientists. Effective collaboration requires: 1) involving social scientists and ornithologists from the beginning of interdisciplinary projects; 2) integrating both sciences into conservation planning and implementation; and 3) building HD capacity in organizations and agencies. This poster will identify resources and opportunities available to researchers interested in taking such steps. While interdisciplinary research can be challenging and time consuming, only through collaboration will researchers be able to effectively understand and address conservation threats faced by birds.}

\normaltalk{The nestling who lived: predictors of survival during harsh weather events in an aerial insectivore}{Ilsa A Griebel\\Russell D Dawson}{Periods of inclement weather can have devastating effects on aerial insectivorous birds, particularly during brood rearing when sufficient food must be obtained to sustain both young and parents. Here, we examine predictors of survival of whole broods and individual nestlings in tree swallows (Tachycineta bicolor) during an extreme, two-day harsh weather event in central British Columbia, which co-occurred with an experiment reducing nest ectoparasite loads using an anti-parasite drug (ivermectin). We found that middle-aged broods and nestlings were less likely to survive than those that were younger or older in age. Survival of broods and individual nestlings was higher when raised by males with bluer plumage, whereas it was lower when female parents had brighter and more UV-reflective plumage. Within broods, smaller nestlings had a lower chance of surviving the harsh weather event than their larger siblings. Nestlings in broods where half of the offspring received ivermectin injections had significantly higher chances of surviving than nestlings from non-experimental broods, suggesting that parasite loads can influence survival during inclement weather. Our results that identify several factors influencing resiliency of nestlings to harsh weather are particularly relevant in the context of declining bird populations and climate change. As the frequency of extreme weather events are predicted to increase with the progression of global climate change, understanding survival during harsh weather events may be critical for aerial insectivores, one of the fastest declining group of birds in Canada.}

\normaltalk{A novel host: parasites, hybrid fitness, and species boundaries}{Angela N Hansen\\Amanda K Hund\\Scott A Taylor}{Hybrid zones have long been viewed as natural experiments for understanding the evolutionary process. However, despite a long history of study, we still do not have a comprehensive understanding of the underlying mechanisms that maintain or erode species boundaries in the presence of regular hybridization. Previous studies have suggested that species boundaries can be maintained due to hybrid inviability, which has been attributed to a variety of both intrinsic and extrinsic mechanisms. However, a potentially important mechanism that has been less well studied is the role that parasites may play in either maintaining or eroding species boundaries. I will present a comprehensive review and synthesis of the literature regarding our current understanding of host-parasite relationships and hybridization in nature, and will discuss exciting new advances in technique and theory that are poised to revolutionize our understanding of how parasites shape species boundaries between naturally hybridizing taxa. I will also present planned future research focused on understanding the role that nest parasites play in shaping the black-capped (Poecile atricapillus) and Carolina (P. carolinensis) chickadee species boundary. My planned research will involve geographically replicated field-based nest manipulations within and outside of the hybrid zone during which I will examine fledging success and post-fledging survival of parasitized and non-parasitized hybrid and non-hybrid chickadees.}

\normaltalk{Social Affiliation and Call Sharing in Budgerigars (Melopsittacus undulatus)}{Clara A Hansen\\Gregory Kohn\\Timothy Wright}{Vocal learning, the ability to modify acoustic sounds based on social context is a trait important for species living in social groups. One hypothesis for the evolution of this ability is that shared vocalizations facilitate social interactions between group members. We investigated whether captive budgerigars who are more affiliated with each other in a social network share a higher proportion of call types. We examined the relationship for 3 different groups, one all-female group of 5 birds and two mixed-sex groups of 6 male and 6 female birds. Contact call recordings and behavioral observations were conducted for a period of 4 weeks. We predicted that pairs of birds with strong affiliative associations would have more shared contact calls. We found a weak correlation between birds that shared a higher proportion of call types and birds with strong affiliative connections. In all experiments, birds shared calls with 0-2 other birds but there was no strong affiliative association evident between birds that shared more calls with each other. We did find, that in the mixed sex group the strongest affiliative correlations were between birds of the same sex with the strongest relationships between males, weakest between females and intermediate between male-female pairs. This suggests that in captivity male budgerigars form strong affiliative ties with other male birds even when potential mates are present. This study provides insight into the correlation between social networks and call sharing and sets up future experiments looking at this relationship in larger, more complex groups of birds.}

\normaltalk{Factors Affecting Knowledge and Beliefs of College Freshmen towards Climate Change}{Sara E Harrod\\Virginie Rolland}{Despite overwhelming scientific consensus and educational outreach efforts, about 25\% Americans remain unconvinced of anthropogenic climate change (ACC), and significant misconceptions of its causes and effects remain. Although many studies have assessed working adults' views towards ACC, incoming college freshmen remain largely overlooked, yet can provide a gauge for how their generation plans to deal with ACC. This study aimed to determine freshmen attitudes towards and knowledge of ACC, which factors affect these variables, and whether there is a correlation between attitude and knowledge. We issued online surveys at 7 universities across the eastern United States and received responses from 718 first-semester freshmen. The survey contained 3 sections assessing the students' attitude, knowledge, and demographic background. Attitude was students' degree of concern of ACC: Alarmed (26.7\%), Concerned (41.4\%) Cautious (19.6\%), Disengaged (3.34\%), Doubtful (4.04\%), and Dismissive (2.92\%). Knowledge was students' total score on knowledge questions (mean = 79.9\%). We analyzed attitude and knowledge using ordinal and beta logistic regressions, respectively. Gender and religion affected both attitude and knowledge, whereas college major and university attended influenced only knowledge, and political affiliation only explained variation in attitude. Yet knowledge and attitude were significantly correlated, suggesting education is an important tool to mitigate ACC. Though limited to the eastern US, our results may be used by educators to develop more effective study plans and outreach strategies, policymakers to better tailor environmental legislation to their constituents, and non-governmental organizations to refine their message about ACC.}

\normaltalk{Harmful algal blooms, seabirds, and forage fish: assessment of algal toxins during a large-scale Common Murre die-off in Alaska}{Caroline Van Hemert\\Sarah K Schoen\\William C Holland\\John F Piatt\\Mayumi L Arimitsu\\Rance Hardison}{An unprecedented mortality event of Common Murres (Uria aalge) was observed across the Northeast Pacific Ocean in 2015‒2016 in association with a mass of anomalously warm ocean water dubbed “The Blob.” The proximate cause of murre mortality was starvation, which appears to have resulted from a shortage of available forage fish. However, birds may have also been exposed to harmful algal blooms (HABs), raising questions about the possible role of algal toxins in the murre die-off. To address these concerns, we tested for the HAB neurotoxins domoic acid (DA) and saxitoxin (STX) in seabird tissues from deceased Common Murres associated with the 2015‒2016 die-off, as well as from apparently healthy murres, Black-legged Kittiwakes (Rissa tridactyla), and samples of forage fish and invertebrates that were collected during summer 2015 and 2016. DA occurred infrequently and only at trace concentrations in both bird and forage fish samples. In contrast, we detected STX across multiple tissue types in both healthy and die-off birds as well as in forage fish. No reference values currently exist for seabirds, but concentrations of STX in most individuals were relatively low, and we found no evidence that acute exposure to algal toxins was a direct cause of Common Murre mortalities. Nevertheless, our results demonstrate the widespread occurrence of STX in seabirds and forage fish across multiple seasons and geographic locations in Alaska, and suggest that HABs should be considered in future assessments of seabird health, especially given the potential for greater occurrence of HABs in the future.}

\normaltalk{Niche modeling of todies (Todus spp.) in the Greater Antilles}{Kate M Henderson\\Peleg Kremer\\Holly Garrod\\Robert L Curry}{Niche modeling is a powerful predictive tool used to determine habitat suitability and potential ranges for specific species, and can be especially useful for filling in knowledge gaps about little-studied species. We used MAXENT niche modeling software to predict habitat suitability for the five species of Todus in the Greater Antilles based on eBird occurrence data and environmental variables including elevation, land cover, and WorldClim's 19 bioclimatic variables. Land cover and precipitation were important for determining suitable habitat for Todus multicolor in Cuba, Todus todus in Jamaica, and Todus mexicanus in Puerto Rico. On Hispaniola, where Todus angustirostris and Todus subulatus are both endemic, elevation was the most important variable for determining habitat suitability. The model outputs successfully identified the Cordillera Central, Cordillera Septentrional, and several parts of the Sierra de Bahoruco and Massif de la Hotte as potential zones of overlap between the two species on Hispaniola, which is supported by the current literature available on Todus distribution. Based on our suitability models, we can make detailed predictions about Todus ranges in the Greater Antilles and can identify ideal regions to study interactions and potential hybridization between Todus angustirostris and Todus subulatus on Hispaniola.}

\normaltalk{Distribution, abundance, and habitat selection of breeding Mississippi Kites (Ictinia mississippiensis) in southern Illinois}{Benton J Hendrickson\\Michael W Eichholz}{Mississippi kites (Ictinia mississippiensis) experienced a population decline between 1900-1950 due to changing agricultural practices, but have since exhibited range expansion throughout the American Southwest, Great Plains and into urban centers. Although considered a species of continental importance and threatened by the state of Illinois, previous studies suggested that Kite populations in southern Illinois were stable or increasing, and reports from numerous natural heritage biologists, avian conservation organizations, and amateur birders suggest a substantial population increase across southern Illinois and range expansion away from historically utilized riparian areas, which may warrant de-listing the species. Numerous studies have investigated the importance of nest-site selection, habitat quality, and prey abundance on Mississippi kite nesting success, with some suggesting regional variability in extent to which these factors are influential. To determine the status of Mississippi kites in southern Illinois and identify factors influencing potential population growth and range expansion, high-quality kite nesting habitat will be identified through spatial analysis of land cover data in ArcGIS, reports from social media and birding resources such as eBird, and auto, foot, and boat surveys throughout 10 southern Illinois counties. Once located, nests will be monitored throughout the season and nesting habitat quality will be recorded quantified. Diet will be analyzed via manual observations of prey delivery and fecal DNA analysis using next generation sequencing. The results of this study will help determine current population size and distribution of Mississippi kites in southern Illinois and inform management decisions regarding kite nesting habitat and prey resources.}

\normaltalk{Distribution and population size estimates for landbirds breeding in Edéhzhíe; a candidate National Wildlife Area in the Northwest Territories, Canada}{Kristina G Hick\\Samuel A Hache\\P\'{e}ter S\'{o}lymos\\Rhiannon F Pankratz\\Erin M Bayne\\Natasha Annich}{Under the Canada Wildlife Act, Canadian Wildlife Service (Environment and Climate Change Canada; ECCC) establishes and manages National Wildlife Areas (NWAs), for wildlife research, conservation, and interpretation.  Edéhzhíe (i.e. Horn Plateau area) is 14,000 km2 of undisturbed forest being considered for permanent protection by the Decho First Nations and ECCC in the Taiga Plains ecozone of the Northwest Territories, Canada. In 2016, ECCC initiated a landbird monitoring program where autonomous recording units were deployed at 205 sampling locations (within 41 study sites) and recorded the vocalizations of 106 species from early May to the end of July. Count data from the 38 most common species were converted into density estimates using offsets generated by the Boreal Avian Modelling Project (http://www.borealbirds.ca/). We present results from species distribution models, habitat-specific density estimates, and population size estimates for these 38 species, with emphasis on species at risk (SAR) in Canada (i.e. Common Nighthawk, Chordeiles minor; Yellow Rail, Coturnicops noveboracensis; Olive-sided Flycatcher, Contopus cooperii; and Rusty Blackbird, Euphagus carolinus).  Data for these SAR were derived from automated species recognition algorithms, i.e. species recognizers. We report considerably high density estimates given that our study area is at the northern limit of the breeding range for most of these SAR and current national population size estimates.  Our results provide important baseline data for setting conservation objectives for this candidate NWA and for identifying important habitats for SAR to inform recovery strategies and management plans.}

\normaltalk{Spatial and Temporal Patterns of Domestic Trade Pressure on Parrots in Nicaragua}{David C Hille\\Francisco Mu\~{n}oz\\Mart\'{i}n Lezama Lopez\\Donald J Brightsmith\\Michael A Patten}{Commercial trade in wildlife is a negative contributing factor to the conservation of biodiversity across many taxa at a global scale. Parrots (order Psittaciformes) are the most common avian order overexploited by capture for trade, a pattern inflated in the Neotropics. Impacts from the domestic trade of parrots is an important factor to analyze in order to understand the causes of declining populations. Nicaragua provides a study system to address multiple questions surrounding this conservation issue. Parrot population trend data is available in Nicaragua from 1995 to 2013, which provides the opportunity to test trade pressure as a cause of population declines. And a ban on parrot capture and trade was voted into law in 2004, which provides an opportunity to test the efficacy of trade bans at the domestic level. In June 2017, we collected parrot count data at 30 city markets, geographically dispersed throughout Nicaragua to characterize present day trade pressure. Parrots were found at 73\% of public markets surveyed. Ten of the potential sixteen species were recorded, totaling 420 individuals. We will test species presence and count data against predictions made from several variables, including species distribution and road accessibility. Additionally, from March to July 2018, we will return to the primary market, Oriental Market in the capital Managua, to conduct once a week surveys. We will directly compare these counts to a 1997 study in order to understand any potential market impacts from the 2004 trade ban.}

\normaltalk{An indirect mutualism between host-specific insects and Hackberry trees (Celtis spp), mediated by migratory songbirds}{Chance H Hines\\Eric L Walters}{We examined how ephemerally abundant leaf-galling hackberry psyllids (Pachypsylla spp) benefit their obligate hosts, hackberry trees (Celtis spp), by attracting foraging migratory songbirds in a coastal region of eastern Virginia. Standardized foraging bird transects conducted during the 2016 and 2017 autumn migration revealed that hackberry was a preferred songbird foraging tree during both years, but the preference was exhibited before and during the psyllid irruption. Hackberry psyllid abundance was a significant predictor of the timing of hackberry seed dispersal. Interestingly, areas where hackberry was present were also epicenters of seed dispersal for other commonly consumed fruits. The patterns of foraging and seed rain observed were supported by field experiments where hackberry leaves were placed beneath isolated eastern baccharis (Baccharis hamifolia) shrubs. Migrant songbirds spent more time foraging and deposited more seeds in baccharis shrubs where hackberry leaves had been placed relative to controls. To our knowledge, this is the first example of migratory songbirds mediating an indirect mutualism between a parasitoid and its host.}

\normaltalk{Conspicuous plumage acquisition and its behavioral correlates during the non-breeding season in the red-backed fairywren}{Lauren G Hitt\\Sarah Khalil\\Megan L Massa\\Joseph F Welklin\\John P Swaddle\\Michael S Webster}{Individuals bearing conspicuous plumage might behave differently than individuals bearing cryptic plumage as the showy plumage could increase the bearer's predation risk and heighten their vigilance. The red-backed fairywren (Malurus melanocephalus) provides a suitable system to track how vigilance and other behaviors vary with the acquisition of conspicuous ornamental plumage. Males in a population exhibit flexible reproductive phenotypes, wherein some males express conspicuous red-black plumage and others exhibit cryptic brown plumage. During the breeding season, red-black males demonstrate territorial, aggressive, and courtship behaviors more frequently than do brown males. We do not know whether these plumage-associated behavioral differences occur in the non-breeding season, when some males are molting from cryptic brown to conspicuous red-black plumage. Therefore we performed behavioral observations on a population of banded red-backed fairywrens during the non-breeding season. We found that brown males spent the least amount of time on anti-predatory vigilance behavior relative to other males, with mid-molt males intermediate and red-black males spending the greatest amount of time on vigilance. These observations are consistent with the idea that showy plumage increases conspicuousness, but that males may respond to this cost by budgeting more time for vigilance behaviors in the non-breeding season. However, unlike in the breeding season, we found a lack of other behavioral differences between male plumage types in this non-breeding season, suggesting flexibility in the degree to which behavior and plumage are coupled in this system.}

\normaltalk{Preliminary molecular phylogeny of the swifts (Aves: Apodidae)}{Jesse M Holmes\\Robert G Moyle}{Swifts (Aves: Apodidae) are a charismatic group of highly aerial birds within the order Apodiformes. They are divided into four tribes: Cypseloidini ("new world swifts"), Collocalini ("swiftlets"), Chaeturini ("needletails"), and Apodini ("typical swifts"). Though much phylogenetic work has been done on groups within this family (i.e. the swiftlets and typical swifts), there has not been a family wide molecular phylogeny proposed. In this study, we introduce the first molecular phylogeny for the family Apodidae. We sequenced ultraconserved elements (UCEs) from samples and analyzed data using the program PHYLUCE developed by Brant Faircloth. Our results provide a preliminary family-wide molecular phylogeny, including most of the 19 genera currently recognized. Future efforts will be towards adding more samples to complete the genus-level phylogeny for the group.}

\normaltalk{What's on the menu? Using next-gen sequencing to examine diets of Ferruginous Hawks and Golden Eagles in the Uintah Basin, UT}{Dylan J Hopkins\\Kim Sullivan\\Becky Williams\\Natasha Hadden}{As the sagebrush steppe ecosystem faces continued degradation from climate change and land development, researching the response of apex predators may indicate the extent of the impacts to the ecosystem. The Ferruginous Hawk and Golden Eagle are apex predators that rely on the sagebrush steppe and the prey species that inhabit it. As prey prevalence is important to raptor occupancy, we seek to understand what prey species compose the diets of these understudied raptor species. We propose to analyze diet using next-generation sequencing techniques. While this novel method of understanding avian diet is uncommon throughout the literature, there has not yet been any published use of this technique to examine raptor diet. We will assess the efficacy of this technique in estimating raptor prey usage.}

\normaltalk{Using opportunistic citizen science data to estimate avian population trends}{Joshua J Horns\\Frederick R Adler\\Cagan H Sekercioglu}{Determining population trends is critical for conservation. For most bird species, trends are based on count data gathered by institutions with formalized survey protocols. However, limited resources prevent many organizations from maintaining these types of formal surveys, especially in developing countries. Ecotourism growth and subsequent increases in opportunistic data from birdwatching may provide a source of population trend information if analyses control for inter-observer variation. List length analysis (LLA) controls for such variation by using the number of species recorded as a proxy for observer skill and effort. Here, we use LLA on opportunistic data gathered by eBird to estimate population trends for 574 species of birds and compare these estimates to population trends based on both formal breeding bird surveys, as well as to population estimates from eBird data controlled by using more rigorous means of correction. Our analyses show that eBird data produce population trend estimates that differ on average by only 0.4\%/year from formal surveys and do not differ significantly from estimates using more effort control metrics. We find that estimates do not improve appreciably beyond 10,000 checklists, suggesting this figure as the minimum threshold of opportunistic data required for population trend estimation. Lastly, we show that characteristics affecting the ubiquity of a species can affect its population trend estimate. Our results suggest that opportunistic data can be used to approximate species population trends, especially for widespread species. Because our protocol uses information present in all checklists, it can be applied to a diversity of data sources.}

\normaltalk{Anomalous ocean conditions coincide with a lack of nesting activity in Marbled Murrelets in Oregon}{Cheryl A Horton\\Lindsay J Adrean\\S K Nelson\\Daniel D Roby\\Matthew G Betts\\James W Rivers}{The Marbled Murrelet (Brachyramphus marmoratus) is a threatened seabird that requires mature forest and adequate prey in the nearshore environment for successful breeding. Despite being listed under the U.S. Endangered Species Act in Washington, Oregon and California for more than two decades, the population is still declining. We established a large-scale, long-term study to understand space use and nest success of murrelets within coastal forests of Oregon, where nesting data are especially limited yet needed for effective conservation. In 2017, we captured and radio tagged 61 murrelets, and tracked these individuals from April – September to identify nest locations and quantify marine and terrestrial habitat use. Although we expected tagged murrelets to undertaken breeding, none of the tagged individuals exhibited nesting behavior, a first for a study of this magnitude. Instead, a majority of individuals exhibited unprecedented long distance movements spanning $>$1200 km of coastline from northern Washington to San Francisco Bay. We hypothesize that the anomalous ocean conditions present in Oregon during the 2017 breeding season reduced available prey resources, leading to absence of nesting behavior and long-distance movements to locate suitable foraging areas.}

\normaltalk{Geographic variation and color polymorphism in Attila spadiceus (Tyrannidae)}{Jeffrey T Howard\\Rafael S Marcondes\\Robb T Brumfield}{Detailed, quantitative descriptions of phenotypic variation are the inspiration and bedrock of systematic, evolutionary and genetic studies. In birds, two common types of intraspecific phenotypic variation in plumage are geographic variation and discrete color polymorphism. Here, we study both types of variation in the extraordinarily variable Bright-Rumped Attila (Passerifomes: Tyrannidae). We obtained spectrophotometric plumage color data from more than 200 specimens from all of this species' expansive range, from Mexico to Brazil. We show that populations in the lowlands west of the Andes (trans-Andes) present continuous geographic variation with little intrapopulational variation. In contrast, populations east of the Andes (cis-Andes) have three discrete morphs (brown, grey and green) and no geographically-structured variation. The cis-Andean polymorphism includes multiple plumage patches. The shapes of spectral curves indicate variation in pigment types across morphs. Discrete plumage polymorphisms are common in orders such as Strigiformes and Falconiformes but are particularly rare within passerines. To the best of our knowledge, this is the first spectrophotometry-based description of discrete plumage polymorphism in a passerine. The molecular systematics of this wide-ranging species have not been studied in detail, but the stark contrast in pattern of variation across the Andes suggest that they have been an important feature in shaping the Bright-Rumped Attila's evolutionary history. It also opens an intriguing avenue of future investigation into the genetic and ecological bases of plumage variation in this species.}

\normaltalk{Aspects of municipalities associated with occupancy and abundance of Chimney Swifts in Illinois}{Maureen L Hurd\\Thomas J Benson\\Michael P Ward}{Aerial insectivores such as the Chimney Swift (Chaetura pelagica) are experiencing population declines across North America. While causes of these declines are unknown, habitat availability is likely a contributing factor. Chimney Swifts historically nested in tree cavities, but switched to nesting in masonry chimneys as North American settlements expanded. With the growing trends of chimney caps and gas/electric heating, available chimneys are disappearing. We examined the influence of various habitat and landscape factors on Chimney Swift abundance. We conducted surveys for swifts in municipalities of varying size throughout Illinois and recorded the number of visible uncapped chimneys at each survey point. We used these data along with landscape-level data such as proportion of developed land surrounding points, areal extent of municipality, and age distribution of buildings. We found that 97\% of municipalities were occupied by Chimney Swifts, but abundance varied considerably. Our results suggest that although Chimney Swifts are still widespread, contemporary building practices will continue to drive population declines and management should focus on approaches for providing suitable nesting sites.}

\normaltalk{Singing the wrong song: Mismatching plumage and song phenotypes in a hybridizing population of Blue- and Golden-winged Warblers}{Amy E Janik\\Maarten J Vonhof\\Sharon A Gill}{Golden-winged (Vermivora chrysoptera, GWWA) and Blue-winged warblers (Vermivora cyanoptera, BWWA) are sister taxa that differ by only six genomic regions and often hybridize despite their distinct plumages and songs. Although their conservation statuses vary, Blue-winged warblers and even more so Golden-winged warblers, are experiencing population declines. A typical GWWA type I song can be described as zee bee bee bee and a typical BWWA type I song can be a bee buzz. We recorded songs of males during the breeding season in a hybridizing population of GWWA and BWWA in upstate New York. In the field, we classified the plumage phenotype of males as GWWA, BWWA, or Brewster's warbler while recording them and then from a subset of our recordings (n=24) we quantified the number of males that gave species-typical songs (i.e. those that matched their plumage phenotype) and those that deviated from the typical vocalization of their species. Our preliminary data showed that 75\% of males sang their species typical song (i.e. a BWWA singing bee buzz), 12.5\% of males sang the typical song of the other species (i.e. a BWWA singing zee bee bee bee), and 12.5\% sang atypical songs that did not match either of the typical type I or type II songs of the warblers. We hypothesize that song variation may affect species recognition in these warblers and may contribute to hybridization. More research is required to understand the effects of atypical vocalizations on species recognition in Golden-winged and Blue-winged warblers and their hybrids.}

\normaltalk{DNA markers and discovery of a contact zone support lumping White-naped (Atlapetes albinucha) and Yellow-throated (A. gutturalis) Brush-Finches}{Rosa A Jim\'{e}nez\\Tsang T To\\Zachary R Hanna\\Marie Droual\\Rauri CK Bowie}{Atlapetes is a genus with nearly 30 species inhabiting Neotropical mountains. They are well-known for the geographic replacement that different species show, where the replacements usually have strikingly different coloration. Current taxonomy recognizes the White-naped Brush-Finch as a single species with two morphotypes, each of which was previously considered a distinct species. The main difference between the two morphs is that the White-naped Brush-Finch has a yellow throat and underparts, whereas the Yellow-throated Brush-Finch has a yellow throat with whitish underparts. The similar phenotype and the apparent geographic isolation of the two morphs resulted in a proposal to lump the two species in 1978 and its acceptance by AOU in 1997. Recent phylogenies suggest that both morphs are part of the same clade. However, detailed phylogenetic and phylogeographic analyses have not been undertaken to date. Our mitochondrial and nuclear DNA sequence data support the monophyly of the lineage, and suggest that splitting the two morphs would result in paraphyletic groups. The White-naped and Yellow-throated Brush-Finches are genetically structured, but this structure corresponds with geography rather than with phenotype. The Isthmus of Tehuantepec, the Nicaraguan Depression, and the Isthmus of Panama are major geographic barriers for this taxon. The two phenotypes contact in the highlands of Chiapas and Guatemala, and intermediate plumage coloration can be found in the contact zone. Their continued distinction outside of this zone suggests that some selective force may be acting to maintain the two morphs. It seems that Atlapetes albinucha represents a species with intraspecific morphologic variation in which a northern yellow phenotype undergoes geographic replacement by a whitish one in Central and South America.}

\normaltalk{Clutch and egg size in a high-elevation population of Tree Swallows}{L S Johnson\\Katie Iser\\Haley Molnar\\Andre Nguyen\\Chelsea Connor}{High elevation, montane environments are typically characterized by colder, wetter, windier weather. Species or populations breeding at high elevations are expected to adapt to these conditions in terms of their life history strategies. However, researchers have had difficulty theorizing exactly how life histories should change in response to elevation. Perhaps consistent with this, species studied to date have shown a wide range of responses to elevation with respect to life history traits. More studies on a variety of species are needed to understand why birds at high elevations do what they do. To this end, we compared clutch and egg sizes of Tree Swallows (Tachycineta bicolor) breeding at two sites in northern Wyoming, including a lower-elevation prairie site at 1350 m (4450 ft) above sea level and a high elevation site on a plateau in the adjacent Bighorn Mountains at 2500 m (8150 ft) a.s.l. Compared to low-site females, high-site females laid, on average, significantly fewer eggs (on average, 0.4 fewer per clutch) and smaller eggs (on average, 2.4\% smaller in volume). When captured during incubation, high-site females were significantly lighter in weight relative to wing length. This suggests that differences in clutch and especially egg size are the result of energetic constraints and not necessarily adaptive adjustments in life history strategy.}

\normaltalk{Circulating carotenoid concentrations differ among sexes and phenotypes in the red-backed fairywren}{Sarah Khalil\\Joseph F Welklin\\Kevin J McGraw\\Michael S Webster\\Jordan Karubian}{Carotenoid-based signaling is a thought to be a classic example of honest signaling, in that acquiring or metabolizing carotenoids likely imposes costs, such that individuals expressing the most carotenoid rich signals should be in the best condition. Yet the proximate mechanisms of carotenoid production remain poorly resolved, which limits our understanding of the evolutionary constraints and physiological costs associated with this widespread signaling modality. The red-backed fairywren (Malurus melanocephalus) provides a unique opportunity to examine mechanisms underlying expression of carotenoid-based signals: males within a single population exhibit flexible reproductive phenotypes, where some males express red/black plumage and other males exhibit female-like brown plumage. To better understand how individuals regulate signal expression, we asked if circulating carotenoid levels differ as a function of sex and male phenotype, and examined the relationship between circulating carotenoids and hue of the red plumage patch in red/black males. Established theory predicts that red/black males should have the highest concentrations of circulating carotenoids, and that levels should peak during molt. We found that that red/black males did indeed have higher concentrations of circulating carotenoids than did brown males and females, supporting our first prediction. However, carotenoid concentrations were only loosely associated with molt date, and there was no relationship between circulating carotenoids and hue for red/black males. This work highlights the complex ways in which carotenoids may be used to signal quality, and raises important questions concerning the physiological costs associated with production of carotenoid based signals.}

\normaltalk{Avifauna abundance, population trends and threats at Mangla Wetland}{Bushra N Khan\\Zulfiqar A }{Avifauna of Mangla Wetland AJK, Pakistan was observed for three years (2013-2016). The globally important fresh water reservoir was surveyed to observe distinctive and unstudied avian diversity and major threats to its existence. A total of 42 surveys of site were conducted to record avian diversity by point count method. Meetings with bird watchers, hunters and concerned department were also arranged to generate the secondary data. A total of 57,892 birds belongs to 188 species were observed at wetland. According to linear regression estimation 101 birds' species were declined in abundance. The population size of 54 birds' species was increasing while 33 avian species were stable throughout the research work. On the basis of seasonal distribution 72 birds were resident, 64 species of birds were winter migrant, year round residents were 28 and summer breeders were 24. Among observed birds Sterna acuticauda was endangered, Prinia burnesii, Anhinga melanogaster, Aythya nyroca and Circus macrourus were near threatened, and Ficedula subrubra, Saxicola macrorhyncha and Haliaeetus leucoryphus were vulnerable. Various factors that threaten the avian diversity of wetland were water pollution due to municipal and industrial wastes, addition of bicarbonate through agriculture runoff, sedimentation, deforestation, annually illegal hunting of more them 10,000 migratory duck, unsustainable fisheries practices, poverty, illiteracy and poor law enforcement. A management plan has also been designed for conservation and management of unique avifauna of site. The key aspects of proposed management plan are betterment of linked fauna, flora and living quality of connected communities without disturbing the natural equilibrium.}

\normaltalk{Backpack microphone system for individualized vocalization recordings in captive flocks of Budgerigars (Melopsittacus undulatus)}{Gregory M Kohn\\Justin C Apodaca\\Timothy F Wright}{Accurately measuring complex social interactions while minimizing disruption of natural behaviors is an important goal of ethologists. This problem is particularly acute for vocal communication signals, where masking and cocktail-party effects can make it difficult to track individual behavior within a group setting. Advances in chip manufacturing and embedded systems are providing behavioral researchers with novel technological toolkits to measure acoustic signals in naturalistic contexts. Here we introduce a wearable backpack microphone system that allows for continuous recording of individual specific vocalizations in free-flying captive bird flocks. The backpacks use small programmable FM transmitters to transmit vocalizations to a communications receiver. We used a software-defined radio (SDR) to receive and analyze FM signals and to assess signal strength from each unit. These recordings were then compared with standard recordings made in sound attenuation chambers. Behavioral observations were made to see how quickly individuals adapt to the backpack, and how other group members respond to individuals wearing backpacks. Comparisons of social interactions before and after backpack fitting is worn are shown. This system may have general utility for understanding the vocal dynamics in complex bird flocks. Such a system shows promise as a cost-effective way to record detailed vocalizations over time, alongside observations of social interactions without the use of sound attenuation chambers. , and can easily be extended to transmit different signals, such as heart rate, in other species.}

\normaltalk{Movement Patterns, Survivorship, and Home Range Size of LeConte’s Thrasher (Toxostoma Lecontei)}{Christina L Kondrat-Smith\\Shawn F Lowery}{The LeConte’s Thrasher (LCTH) is a species of conservation concern included on the Red Watch List of Partners in Flight. The species distribution range consists of sparsely vegetated Sonoran Desert landscapes (Lower Colorado Subdivision) across the southwest and northwestern Mexico. In southwestern Arizona, the Department of Defense (DoD) manages large tracts of habitat on the Barry M. Goldwater Range (BMGR) and Yuma Proving Grounds (YPG). Given the scale of this area, DoD installations play a major role in the conservation of this ecoregion. The Arizona Game and Fish Department (AGFD) conducted occupancy surveys during the 2011-2013 breeding seasons on the BMGR and YPG to better understand species distribution and identify potential habitat relationships.  Survey results were used to develop detection probabilities across the DOD installations, and a Prediction of Occurrence model was built to provide a predictive index of the species habitat.  In addition to species surveys, active nests were identified to monitor fledglings using VHF telemetry (2013). Fourteen birds were radio-tracked from the nestling to post-fledgling dispersal periods. Tracking results provided an understanding of survival, movement, and home range size of LCTH post-fledging. Results from this study allowed military installations in this ecoregion to adapt their Integrated Natural Resource Management Plans to better conserve LCTH habitat while maintaining military readiness into the future. }

\normaltalk{Nest success and brood parasitism of birds in response to grazing of Conservation Reserve Program grasslands}{Heather M Kraus\\William E Jensen}{Grassland bird populations have experienced declines in recent decades that coincide with fragmentation and loss of prairie habitat. The Conservation Reserve Program (CRP) has benefitted grassland birds through grassland restoration. Although the CRP provides better habitat than row crops, grazing by domestic cattle (Bos taurus), which is currently restricted in CRP, might improve habitat structure for some bird species. However, such changes in habitat structure, and the presence of cattle, might affect nest concealment from predators and attract brood parasitic Brown-headed Cowbirds (Molothrus ater). We investigated daily nest survival of Mourning Doves (Zenaida macroura) and Dickcissels (Spiza americana), and brood parasitism of Dickcissel nests, in response to experimental grazing by cattle across CRP fields in central Kansas. Daily nest survival rates were not strongly related to grazing. Brood parasitism rate (presence or absence of cowbird offspring) was similarly unaffected by grazing but was negatively related to nest concealment. The number of cowbird eggs per parasitized nest (parasitism intensity) was higher in grazed CRP, suggesting the presence of cattle might increase parasitism pressure by cowbirds, at least in years when cattle are present. Although daily nest survival and brood parasitism rate were unaffected by cattle grazing, we may observe lag effects in coming years due to changes in nest concealment from grazing in 2017.}

\normaltalk{Movements and habitat use of golden eagles in Baja California peninsula, Mexico}{Tom\'{a}s L\'{o}pez\\Ricardo Rodr\'{i}guez-Estrella\\Peter H Bloom\\Jeff Tracey\\Robert Fisher}{Golden eagles have long-distance movements during migration and juvenile dispersal. Distances up to 5000 km have been recorded for some Golden eagles in North America, from Alaska to Canada and USA. During dispersal and migration Golden eagles of southwestern USA enter Mexico. No published information exists on habitat use and habitat selection into Baja California peninsula during the first years after fledging. Here, we present the movement patterns and habitat use of golden eagles banded while nestlings in San Diego, California and moving southwards to Baja California peninsula, México. Golden eagles were tagged with PTTs and satellite data of 18 individuals preliminarily show that two females traveled up to 8000 km while two males moved up 5000 and 7000 km into the peninsula. No distance patterns were found by sex and age nor seasonally. Some females used more frequently habitats modified by human activity (grasslands and agricultural areas) than males did, however the proportion of females using modified habitats vs natural vegetation was 50:50. More males used mostly chaparral vegetation, and the proportion of males using modified habitats vs natural vegetation was 30:60. We conclude that there is not a clear pattern of distance movements no related to sex or age, and more eagles used natural vegetation (chaparral) but a moderate proportion used human-modified habitats. Our information has important implications for conservation of the Golden eagle in both countries.}

\normaltalk{Characterizing avian social networks using banding station capture data}{Katie LaBarbera}{Many bird species form flocks for at least part of the year. Membership in a flock may aid in foraging, predator avoidance, and securing a mate; many species form complex dominance hierarchies, and social interactions in the flock may have large effects on fitness. Yet due to the difficulty of long-term studies, research on the duration of social associations and changes in flock social structure over time is scarce. We use 29 years of year-round bird banding data from the San Francisco Bay Bird Observatory's Coyote Creek Field Station to investigate the temporal and demographic patterns of social associations in local species (Bushtit Psaltriparus minimus, Chestnut-backed Chickadee Poecile rufescens, Golden-crowned Sparrow Zonotrichia atricapilla, Ruby-crowned Kinglet Regulus calendula, White-crowned Sparrow Zonotrichia leucophrys, Yellow-rumped Warbler Setophaga coronata). Social associations are inferred from captures at close spatial and temporal proximity, an imperfect but likely conservative metric. We will present results on the shape of intra- and interspecies social networks, their age and sex structures, and the duration of social bonds.}

\normaltalk{Pneumoconiosis in a hurricane-stranded Least Storm-petrel (Hydrobates microsoma)}{Patricia J Latas}{Hurricane Newton swept northward over the Gulf of California during the first week of September 2016. The well-organized remnants of the storm crossed southern Arizona 6-7 September, and deposited numerous and diverse pelagic seabirds. Several species of live storm petrels were presented, by the public, to a local wildlife care center for urgent veterinary attention. Samples from post-mortem examination of deceased specimens were submitted for histologic and other analyses. Findings from an individual Least Storm Petrel Hydrobates microsoma included pathological changes of the lungs indicative of pneumoconiosis, and foreign material associated with the lesions. Pneumoconiosis has not been described in storm-petrels and this account documents the clinical presentation, histological and pathological findings, analysis of the material associated with lesions.}

\normaltalk{Fire-Bird – Gis Tools for Applying Habitat Suitability Models to Inform Forest Management}{Quresh S Latif\\Victoria A Saab\\Jessica R Haas\\Jonathan G Dudley}{Habitat suitability models are used to guide land management for species of conservation concern. Models quantify relationships between known species locations and environmental attributes, which are used to identify areas most likely to support species of concern. Managers can then limit human activities with negative impacts on habitat in areas of high suitability. Application of habitat suitability models, however, typically requires technical expertise not available to most land managers. FIRE-BIRD is a series of ArcGIS tools that facilitates application of habitat suitability models to inform forest management for disturbance-associated woodpeckers of conservation concern. By operating within an ArcGIS environment, FIRE-BIRD is well positioned to integrate with forest planning. Tools are currently developed for black-backed (Picoides arcticus) and white-headed woodpecker (P. albolvartus) in Inland Northwest burned forests; black-backed, white-headed, and hairy woodpecker (P. villosus) in Northern Sierra burned forests; and white-headed woodpecker in Inland Northwest unburned forests. The toolset automates both model application and preliminary data processing to minimize required technical expertise. An accompanying manual describes implementation and interpretation of resulting habitat suitability maps. Tool prototypes have been tested by National Forest biologists, and their feedback has been incorporated. The suite of species currently included makes this toolset best suited to inform post-fire management and restoration treatments in dry mixed conifer forests. Incorporation of additional species and forest conditions in the future could broaden the scope of this toolset.}

\normaltalk{Modes of Sexual Dichromatism in New World Blackbirds (Icteridae)}{Gabriella M LeFevre\\Jordan J Price\\Muir D Eaton}{Since Darwin and Wallace defined natural and sexual selection, research has demonstrated that males and females can undergo different selective pressures, potentially resulting in various levels of sexual dichromatism over evolutionary time. Icteridae is a monophyletic assemblage of 108 species of ‘blackbirds' distributed throughout the Americas. Within this clade, species exhibit a broad range in levels of sexual dichromatism as well as a diversity of life history traits and ecologies. Using a model of avian perceptual color space, we quantified several color variables representing sexual dichromatism from plumage reflectance data for males and females of 89 species. We also scored levels of color pattern differentiation using human visual criteria and attempted to define modes of sexual dichromatism given concurrence between these two methods. We further tested for correlated evolution of quantified measures of sexual dichromatism and various life history traits across the family, given a robust molecular phylogeny of the icterids. We find that clades within Icteridae have largely experienced different selective pressures on male and female color differences during their respective evolutionary histories, reiterating the complexity in broadly explaining how and why sexual dichromatisms evolve across birds.}

\normaltalk{Nesting Ecology of Bell's Vireo (Vireo bellii bellii) in north east Texas}{Natasha R Lehr\\Hannah S Dill\\Johanna Delgado-Acevedo\\Dean Ransom}{The Bell's vireo (Vireo bellii) is a neo-tropical migrant that breeds throughout the central and southwestern US. Unlike many states in this species range, Bell's vireos populations are increasing in Texas. In 2016 and 2017, we studied the nesting ecology of a recently established breeding population of V. b. bellii on two sites in north east Texas. We located 107 nests from which we quantified nest success, daily nest survival (program MARK), and nest site vegetation. We modeled nest site selection using logistic regression models in an information theoretic model selection approach and evaluated best models using AIC and odds ratios. Apparent nest success was 39 \% (n=30); 36 \% and 25\% of nests failed due to nest predation and cowbird parasitism, respectively. Bell's vireo preferentially nested in Honey Locust (Gleditsia tricanthos), in which they realized higher nest success; vireos nested in Osage Orange (Maclura pomifera) in direct proportion to its availability, and Hackberry (Celtis laevigata) less than its availability in which nest success was lowest. Daily nest survival was 0.964±0.005 and did not differ between nest tree species, nest fate, or early versus late season, but was positively associated with distance of the nest to nest tree canopy edge and canopy cover. Nest sites differed from random sites by lower concealment below the nest, greater concealment lateral to the nest, and higher shrub density within a 10 m radius of the nest.}

\normaltalk{From Kingbirds to Kinglets: Where are the Queens of ornithology?}{Susannah B Lerman\\Kimberly Sullivan\\Bonnie S Bowen}{Although for a generation, women have made up 50\% or more of the student membership in the fields of ecology, behavior, and ornithology, few women are found in positions of leadership in these fields. Having visible role models for women during their early career stages can address the attrition, and has the potential to influence impressions on whether women can attain success in their scientific field. We present data on the representation of women as editors (i.e., ‘gatekeepers' of science), society officers, recipients of senior awards, fellows, and plenary and symposium speakers across the ornithological societies, and compare them with the representation of women in behavior and ecological societies. We examine several hypotheses to explain these results including pair formation in early career scientists, persistence and societal expectations of service and leadership.}

\normaltalk{Using Automated Acoustic Recorders to Uncover Patterns of Avian Presence in a Sequoia Forest}{Carlos Garcia Linares\\Megan McKenna\\Jacob Job\\George Wittemyer\\Erik Meyer}{Sequoia forests are some of the oldest forests in the world, housing a unique variety of habitats and species. The oldest Sequoia trees grow to over a hundred meters in height, and can persist through disturbances (e.g., fire) that decimate other tree species within these forests. Many species rely on this unique habitat, yet we lack understanding how species are spatially distributed across sequoia forests and how the combination of wildfire and prescribed burns affect their distribution. We deployed automated acoustic recording devices in Sequoia National Park across different fire treatments and along a vertical gradient within an individual Sequoia tree to uncover patterns of bioacoustical activity, focusing specifically on avian vocalizations. We manually identified species-specific vocalizations from a subsample of acoustic recordings, accounting for nearly 5000 vocalizations. Using these manual detections, we then constructed automatic detectors to analyze the remainder of our data, focusing specifically on American Robins (Turdus migratorius) and Pacific-Slope Flycatchers (Empidonax difficilis), two species associated with fire events. Here we present preliminary results on differences in bird distributions along the different fire treatments, as well as the vertical gradient. We are currently expanding our study to detect a broader suite of species to examine community distribution as a whole, as well as using acoustic indices to more broadly examine ecological patterns in these forests.}

\normaltalk{A bird survey: Assessment of migratory birds on Punta Cucharas Nature Reserve, Ponce, Puerto Rico before and after Hurricane Maria}{Eduardo M Llegus-Santiago\\Ian L Maldonado-Ortiz\\Jos\'{e} G Rodriguez-Ramirez\\Gilmarie Rivera-Castellar\\Cecilia M Jim\'{e}nez-Figueroa}{Our bird survey has been conducted between 2016 and 2018 on Punta Cucharas Natural Reserve on Ponce, Puerto Rico. It serves as habitat and stopover site for more than 70 species of endemic and migratory birds, other fauna and flora. Our aim was to determine values of importance and compare diversity indexes of avifauna in Punta Cucharas, according to their adaptation and food source, in six ecosystems: Mangroove, Secondary Forest, Lagoon, Salitral, Flooded Muddy Bottom, and Coastal Beach. In a monthly basis, 10-minute bird count was carried out by a group of volunteers from July 2016 to 2018 in the six habitats. The statistical analysis included a Student's t-Test, paired two samples assuming equal variances ($\alpha$=0.05) comparing six habitats from years 2016, 2017 and 2018 in Bird Diversity using Inverse Simpson and Simpson Diversity Index. We are currently comparing numbers after Hurricane Maria impacts on Puerto Rico landscapes. A total of 2,374 individuals have been counted, we classified them as aquatic or terrestrial adapted. There was no significant difference (P=0.056) between years 2016 and 2017 in bird diversity composition, we plan to compare with recent bird counts (2018) after a natural disturbance. Our findings have demonstrated high diversity and richness of birds on this community before the hurricane. The data suggests that species diversity increases in the months of migration (September-October) and nesting (May). This place serves as critical habitat for avifauna and our future approaches include: bird nest monitoring, citizen science programs, and compare our bird counts throughout years.}

\normaltalk{Disease Dynamics in Migratory and Resident birds of Eastern Oklahoma and Texas}{Krisangel Lopez\\Matthew J Miller}{Emerging infectious diseases represent a significant threat to global health and security. Wild birds play an integral role in pathogen dispersal dynamics. Oklahoma and Texas serve as breeding locations and migration routes for many bird species, as well as are potential sites for virus amplification, host switching, and geographic expansion. Various studies have demonstrated that wild bird populations are persistently impacted by an assortment of encephalitic viruses, with consequences for both wildlife and human health. For instance, the number of human West Nile Virus cases in Oklahoma has doubled since 2016. This study focuses on surveying viral prevalence in breeding birds from Oklahoma and eastern Texas, and furthering the understanding of the roles both migrant and resident birds play as amplifying hosts for multiple arthropod-borne viruses (arboviruses). We surveyed three locations in Oklahoma and Eastern Texas in 2017. We collected blood serum from 200 wild caught birds and screened for five encephalitic viruses (West Nile Virus, St Louis Encephalitis, Eastern Equine Encephalitis, Western Equine Encephalitis and Highlands J virus) using Hemagglutination Inhibition tests. Our results demonstrate local circulation of all five viruses in both Oklahoma and Texas. These data will serve as a foundation for future work to study the geographic ecology of birds and arboviruses in the southern Great Plains and into tropical and subtropical regions to the south.}

\normaltalk{Facilitating Frugivory by Birds: A Novel Application of Conspecific Attraction}{Sean E MacDonald\\Jinelle H Sperry\\Michael P Ward}{Conspecific attraction is the propensity for members of the same species to settle near one another and is a behavior that has been successfully exploited by conservation practitioners to augment bird populations in target areas. Methods of attracting birds include visual (i.e. decoys) and audio cues (i.e. playback of song). There is a growing body of literature surrounding social information and habitat selection in birds but, to date, all studies in this area focus on the birds themselves and not the ecological relationships they support. Here we explored the effectiveness of enticing birds to targeted fruiting plants using playback experiments, thereby facilitating seed dispersal. Our study was conducted from June 2016 to July 2017 on Oahu, Hawaii. We performed 80 experiments across 25 plant species using vocalizations from four resident bird species. The average number of birds that consumed fruit from a target plant per trial during the control (no playback) was 0.13±0.08 and increased to 1.58±0.5 during the treatment (playback). Strength of response appeared to be plant species-specific with nearly 90\% of observed frugivory events occurring on common plants. 30\% of trials (24/80) resulted in frugivory from a target plant during treatments. All species were significantly attracted to conspecific vocalizations with Japanese white-eye being the only species significantly attracted to heterospecific vocalizations. Our results suggest that certain species may be using social information, both from within their species and species in their dietary guild, when making foraging decisions and that bird behavior can be manipulated to facilitate mutualistic interactions.}

\normaltalk{The effects of land-cover and invasive species on the bird community along the Rio Grande, Big Bend National Park, Texas}{Heather L Mackey\\Eric M Wood\\L Y Pomara\\J E Coffey}{Along the Rio Grande in Big Bend National Park, Texas, land use, such as agriculture and urbanization, have altered natural land cover composition and biodiversity outside the boundaries of the protected area. Further, an invasive grass species, giant cane (Arundo donax) has infested the floodplain, altering natural conditions. Our goal was to determine the effect of broad-scale habitat variables, such as land-cover composition, and local-scale habitat variables, such as cover of invasive species, on the bird community. We quantified the broad-scale habitat variables from satellite imagery and aerial photographs, and they included a measure of greenness (NDVI), habitat structure (image texture), and land cover (e.g. developed land). We collected the local-scale variables at field sites, and they included plant composition and vegetation structure. We used a linear mixed-effects analysis to determine the strength of the relationship among independent variables and bird richness and abundance of breeding bird guilds (e.g. riparian, synanthrope). Additionally, we used an occupancy analysis to model habitat associations of riparian-affiliated birds in relation to broad- and local-scale variables. We found that broad-scale variables had the highest support in models, explaining upwards of 40\% of the variance in bird guild richness and abundance (e.g., habitat structure), whereas local-scale variables explained up to 30\% of the variance (e.g. A. donax). We found that occupancy of the Yellow-billed Cuckoo (Coccyzus americanus) was best explained by tree cover surrounding territories. Our results underscore the importance of both land-cover and invasive species in structuring the avian community along the Rio Grande.}

\normaltalk{Multiple parasitism relaxes egg rejection thresholds in hosts of mimetic avian brood parasites}{Tommy Manna\\Lainga Tong\\Csaba Moskat\\Miklos Ban\\Zachary Aidala\\Mark E Hauber}{A host which has been targeted by an avian brood parasite can recover most of its potential fitness loss by ejecting the foreign egg(s) or offspring from its nest. The propensity for some hosts to engage in this behavior has put selective pressure on their parasites to evolve mimetic eggshells resembling the host's own shell colors and maculation. In turn, hosts have counter-evolved increasingly more sophisticated detection methods such as narrowing visual discrimination thresholds or even using non egg-specific cues. However, multiple sensory and cognitive mechanisms acting simultaneously could theoretically interfere with one another and ultimately decrease egg rejection accuracy. Through an experimental parasitism protocol, we tested a host species of the common cuckoo Cuculus canorus, the great reed warbler Acrocephalus arundinaceus's response to 1, 3, or 5 model foreign eggs of varying color and uniformity. Using reflectance spectra of egg background coloration and avian perceptual modeling, we then estimated the sensory thresholds triggering egg rejection by this host. As previously reported, rejection rates were positively related to the perceptual distance between own and foreign eggs in the nests in all treatments, but rejection thresholds were more permissive (error-prone) both with greater proportions of foreign eggs per clutch and/or also when the suite of foreign eggs were perceptually more variable within the nest. These results suggest that the evolution of host recognition of parasite mimicry can be partially overcome by multiple parasitism and its impact on hosts' egg recognition thresholds, thus making them vulnerable to such specific parasitic strategies.}

\normaltalk{Bright coloration does not incur predation cost in red-backed fairywrens}{Megan L Massa\\Samantha J Hagler\\John Swaddle\\Michael Webster\\Jordan Karubian\\Joe Welklin}{Though many studies have documented fitness costs associated with the expression of secondary sexual traits, there is inconsistent evidence that apparently conspicuous plumage increases the risk of depredation in birds. Therefore, we investigated whether bright, sexually attractive coloration increases predation risk in a dimorphic passerine bird, the red-backed fairywren (Malurus melanocephalus). Males of this species can breed in either a bright red-black plumage or a female-like dull brown plumage. We examined predator preference for 3D-printed models of birds that were painted to simulate bright or dull plumage, with the models affixed to tall poles in grassland habitats. According to the conspicuous-plumage hypothesis, we predicted that predatory birds would preferentially attack the red-black model at each site. We also quantified antipredatory behavior and habitat preferences of free-living bright and dull males in response to experimental playback of predator vocalizations, and predicted that bright males would show relatively increased antipredatory responses and would favor habitats with a greater cover of dense vegetation types. Contrary to our predictions, avian predators did not preferentially attack the bright or the dull models. Likewise, we did not find any differences in antipredatory responses or habitat preferences between males of the two plumage types. We suggest that predation risk does not differ notably between the bright and dull color morphs of male red-backed fairywrens and, further, that these color differences are not strong constraints on the evolution of sexually selected traits in this species.}

\normaltalk{How does non-breeding environment affect sexual signaling? A comparative analysis of two subspecies of Barn Swallow}{Molly T McDermott\\Rebecca J Safran}{Sexually selected traits, such as the colorful plumage of many birds, advertise to potential mates an individual's genetic quality and access to resources. These traits are used in species recognition, and differences can contribute to divergence among populations. Within migratory songbirds that molt on their non-breeding grounds, the relevance of environmental information encoded in sexual signals is particularly puzzling. If females use plumage signals to gain information about a male's access to resources, what information do they gain from feathers developed thousands of miles away? I outline a comparative approach to study environmental effects on sexual signals in two subspecies of Barn Swallow (Hirundo rustica) that differ in sexually selected plumage traits and migratory behavior. I will use miniaturized GPS tags to identify non-breeding grounds used by each subspecies, and global climate databases to ask how environmental parameters affect annual changes in plumage traits. By comparing the same individuals in two consecutive breeding seasons, I will control for genetic differences and quantify the relative influence of environmental conditions on plumage variation. Condition-dependence is an important component of sexually selected traits, therefore, I predict that sexually selected plumage traits will have more environmentally-induced variation than naturally selected plumage traits. Due to occupancy of a single area year-round, I predict that environmental effects will explain a greater proportion of variation in plumage in non-migratory populations. This research will contribute to our understanding of how ecology affects sexual signaling and speciation processes.}

\normaltalk{A comparative genomic study of introgression in secondary contact in lowland Panamanian birds}{Jessica F McLaughlin\\Matthew J Miller}{In the Neotropics, many show strong phylogeographic structure across space. Often geographically-separated lineages do no interact, because of the presence of physical barriers such as mountains, rivers, or strong ecological gradients. However, when they meet, there are multiple potential outcomes, ranging from the reticulation of the lineages in contact to the maintenance of reproductive isolation between the populations. Multiple bird species show abrupt mitochondrial breaks between eastern and western populations in Panama and meet in secondary contact in central Panama, where there is no obvious geographic barrier to gene flow. Previous work demonstrates this abrupt turnover of mitochondrial lineages, but the extent to which this geographic pattern of sharp divergence is also reflected in the nuclear genome is unclear. Here, we use loci associated with ultraconserved elements (UCEs) to measure introgression between eastern and western populations of eight species of Panamanian birds and to begin to characterize the nature of hybridization in secondary contact. We compare patterns of nuclear introgression to those observed in mtDNA. Typically, though levels of UCE introgression are greater than mtDNA, most species show geographically-limited, often asymmetrical, introgression consistent with a pattern of restricted admixture expected with speciation and reproductive isolation. This comparative study lays the foundation for future studies into ecological and evolutionary predictors of the outcome of secondary contact, and provides an initial dataset for going forward to investigate what mechanisms may determine this outcome.}

\normaltalk{Epigenetic variation between urban and rural populations of Darwin's finches}{Sabrina M McNew\\Daniel Beck\\Ingrid Sadler-Riggleman\\Sarah A Knutie\\Jennifer AH Koop\\Dale H Clayton}{The molecular basis of evolutionary change is assumed to be genetic variation. However, growing evidence suggests that epigenetic mechanisms, such as DNA methylation, may also be involved in rapid adaptation to new environments. An important first step in evaluating this hypothesis is to test for the presence of epigenetic variation between natural populations living under different environmental conditions. In the current study we explored variation between populations of Darwin's finches, which comprise one of the best-studied examples of adaptive radiation. We tested for morphological, genetic, and epigenetic differences between adjacent “urban” and “rural” populations of each of two species of ground finches, Geospiza fortis and G. fuliginosa, on Santa Cruz Island in the Galápagos. Using data collected from more than 1000 birds, we found significant morphological differences between populations of G. fortis, but not G. fuliginosa. We did not find large size copy number variation (CNV) genetic differences between populations of either species. However, other genetic variants were not investigated. In contrast, we did find dramatic epigenetic differences between the urban and rural populations of both species, based on DNA methylation analysis. We explored genomic features and gene associations of the differentially DNA methylated regions (DMR), as well as their possible functional significant. In summary, our study documents local population epigenetic variation within each of two species of Darwin's finches.}

\normaltalk{Influence of Recreation on the Occupancy, Abundance, and Breeding Performance of Alpine Tundra birds in Denali National Park}{Avery L Meeker\\John M Marzluff}{In Denali National Park, over 50 years of traditional ecological knowledge suggests that many bird species within the park are being displaced from historical breeding areas. Population declines are most notable for tundra species such as American Golden-Plover (Pluvialis Dominica) and Long-tailed Jaeger (Stercorarius longicaudus). This change has coincided with an increase in tourism, including increased recreational opportunities like hiking trails, back country camping permits, and daily bus trips. Our project investigates the impacts of these recreational activities on birds in alpine tundra habitats. In our first year we repeatedly surveyed birds using 600-m- long line transects at 81 sites using the Land-bird Monitoring Protocol for the Arctic Alaska Network. We found decreasing densities of birds in relation to more back country use and proximity to the park road. We also detected birds at greater distances in areas including trails than in area's without a trails. We also monitored reproductive success for five tundra shorebird species at three sites (29 breeding territories and 17 nests, which produced 22 flying offspring). Our results suggest that birds may be avoiding some locations with higher recreational use within Denali National Park. We will continue the study in 2018.}

\normaltalk{Spatiotemporal patterns of habitat use during incubation by a uniparental shorebird in a heterogeneous landscape}{Tyler J Michels\\Angela M Dwyer\\Kristen Philbrook\\Michael B Wunder}{The mountain plover (Charadrius montanus) is a migratory shorebird that breeds in rangelands of the western Great Plains. Populations have declined sharply since the 1960s and the species is of conservation concern in most states and provinces where it occurs. Nesting behavior involves an unusual split-clutch mating system; females typically lay 3 eggs in a nest tended entirely by the male, before laying another 3 eggs in a different nest for the female. Mated birds do not provision each other during the incubation and brood-rearing periods ({raise.17exhbox{\$scriptstylemathtt{sim}\$}}29 and {raise.17exhbox{\$scriptstylemathtt{sim}\$}}35 days, respectively). Because of this, nesting habitat must also provide efficient foraging for incubating adults. Plovers are known to have nested on fallow cultivated fields since the 1980s, but little information exists about adult plover behavior and habitat use during the incubation period on croplands. To determine how incubating plovers use habitat in a mixed habitat and ownership landscape, we deployed and recovered 11 miniature global positioning system (GPS) dataloggers on adult plovers incubating nests on cultivated fields in Weld County, Colorado. We estimated home-range size while foraging using 95\% kernel density contours and report a median of 259.5 ha (range 20.04-1555). We recorded the proportion of foraging locations per habitat type throughout the deployment period and found plovers foraged the most on cropland. These data will help understand how landscape level habitat fragmentation brought about by changes in farming economics, energy development, and/or climate may impact mountain plover breeding biology.}

\normaltalk{Ecological niches and environmental overlap of the Arremon brunneinucha (Aves: Passerellidae) complex in Mesoamerica}{Israel Moreno-Contreras\\Adolfo G Navarro-Siguenza\\Luis A S\'{a}nchez-Gonz\'{a}lez\\Mar\'{i}a del Coro Arizmendi}{Speciation can either proceed as gradual divergence, mostly in allopatry or parapatry, or may instead consist of, most likely, a series of genomic events, which result in the establishment of new evolutionary lineages. To better predict the widely recorded species geographical and environmental shifts, it is crucial to first define and understand species environmental niches and their variability in light of their biotic contexts. A good way to examine niche shifts and niche partitioning is to measure and test the overlap and the difference in the environmental space occupied (i.e. realized niche) over time and across species. Indeed, the realized niche reflects the action of both abiotic and biotic factors on species distribution. Environmental Niche Models (ENMs) have been used to model and predict species distributions according to changes in climatic and environmental variables, considered to be the main drivers of species distribution at large and small spatial-scales. Here, we investigate the ecological niche of the Arremon brunneinucha complex (Aves: Passerellidae) in Mesoamerica and examine how they overlap in geographical and environmental space. We used a correlative method (Maxent) for the generation of ecological niches in this complex.}

\normaltalk{Structure versus time in the evolutionary diversification of avian carotenoid metabolic networks}{Erin S Morrison\\Alexander V Badyaev}{Historical associations of genes and proteins are thought to delineate pathways available to subsequent evolution, however the effects of past functional involvements on contemporary evolution are rarely quantified. Here we examined the extent to which the structure of a carotenoid metabolic network persists in avian evolution. Specifically, we tested whether the evolution of avian carotenoid networks was most concordant with phylogenetically structured expansion from core reactions of common ancestors or with subsampling of evolutionary conserved biochemical pathway modules. We compared structural and historical associations in 467 carotenoid networks of extant and ancestral species and uncovered the overwhelming effect of preexisting metabolic network structure on carotenoid diversification over the last 50 million years of avian evolution. Over evolutionary time, birds repeatedly subsampled and recombined conserved biochemical modules, which likely preserved access to the same metabolic network during avian evolution. These findings explain the recurrent convergence of evolutionary distant species and weak phylogenetic signal in avian carotenoid evolution. Remarkable retention of an ancient metabolic structure throughout extensive and prolonged ecological diversification in avian carotenoid metabolism illustrates a fundamental requirement of organismal evolution – historical continuity of a deterministic network that links past and present functional associations of its components.}

\normaltalk{Aggressive females trade-off resource defense with parental investment in blue tits (Cyanistes caeruleus)}{Troy G Murphy\\Peter Korsten\\Martje Birker\\Reinaldo Marfull\\Jan Komdeur}{Understanding the causes of sexual differences in aggression is a major goal in studies of evolution. Although males are widely considered the more aggressive sex, females often accrue substantial fitness benefits through competition. However, despite such benefits, substantial costs are associated with aggressive behavior in females, and some of these costs are female-specific. Female costs can arise from the aggression-mediating hormones which can interfere with breeding physiology. Additionally, female aggression may be especially costly because offspring survival and condition is so dependent on maternal investment -- from development through dependency. A tradeoff is therefore expected to exists between the benefits arising from aggression (e.g., resource defense), and the depression of reproductive output. We investigated the costs associated with female aggression in the blue tit (Cyanistes caeruleus). To do this, we elicited female aggression by repeatedly presenting stuffed female taxidermic models on top of nestboxes before egg laying. We additionally ran a control treatment by placing empty cages on top of nestboxes. We measured whether females that expend energy and resources in defending nests (augmented-aggression-females) shifted energy and resources away from producing multiple high-quality offspring. We found that augmented-aggression-females suffered reduced reproductive performance, and that offspring of augmented-aggression-females showed less begging vigor compared to offspring from control females, suggesting that there is an across generational reproductive cost associated with female aggression. Our results provide evidence that there are substantial costs associated with female aggression.}

\normaltalk{Temporal covariation of demographic rates in Lesser Scaup (Aythya affinis) and management implications}{Kelsey L Navarre\\David N Koons}{The continental population of scaup remains well below population objectives. Previous analyses of long-term demographic data have revealed recruitment, which is a combination of multiple vital rates (e.g., nest success, duckling survival, juvenile survival), as the likely driver of the decline in population growth. Vital rates are often assumed to be independent in population models; however, this is a costly assumption because covariation in vital rates can have a greater influence on population dynamics than variation in any vital rate on its own. To study covariation, vital rates of a population must be measured over long periods of time. We are using the long-term dataset at Red Rock Lakes National Wildlife Refuge, Montana and the creation of demographic models to explore the implications of environmental variability on lesser scaup population dynamics. Specifically, we are: continuing to monitor and estimate seasonal survival of adult females, breeding propensity, clutch size, nesting success, duckling survival, and first-winter survival of females, developing models to evaluate the relative contribution of habitat and climate factors to variation and covariation of demographic rates, creating populations models for lesser scaup that incorporate covariation in key demographic rates as a function of environmental variables, and exploring implications of environmental variation on vital rates by examining alternative life-history responses to such variation and consequent impacts on population dynamics. Results from this study will inform which vital rates are most important to population growth rate as well as the best management actions for increasingly variable environmental conditions due to climate change.}

\normaltalk{Characterizing the Microbiota and Resistome of American Crow (Corvus brachyrhynchos) Feces}{R L Nelson\\Jonathan A Eisen\\Madhusudan Katti\\Tricia A Van Laar}{American Crows (Corvus brachyrhynchos) are well adapted to living closely with humans, roost in large numbers, and feed at waste-treatment facilities and on human-generated trash. These characteristics make them likely to encounter antibiotic resistant (AR) and/or enteric pathogens and deposit them in anthropogenic settings. Our study areas are Fresno and Davis (urban) and Critter Creek Wildlife Station (rural). We hypothesize the urban fecal microbiota would share many taxa with the rural (core microbiota), but to also have significant differences due to different food sources (variable microbiota). We also hypothesize the resistomes of the urban crows will be different from rural as the urban crows are exposed to organisms associated with antibiotics used in agriculture and medicine. We amplified and sequenced the 16S rRNA gene of DNA extracted from crow feces. Sequencing data was analyzed using QIIME2. For resistome analysis, DNA from feces at each location was pooled, giving one sample for each location. We are performing metagenomic sequencing of the pooled DNA. We also cultured bacteria from feces and identified AR pathogenic genera including Staphylococcus, Klebsiella, and Enterococcus. Through 16S rRNA sequencing, we identified pathogen-containing taxa like Legionellales, Treponema, Rickettsiaceae, and Clostridia. The diversity of the microbiota within the roosts was not significantly different when comparing roosts, however there was a significant difference between the composition of the microbiota from the urban and rural settings. The data from this study will provide insight into potential health of urban versus rural crows and risks to humans near roosts.}

\normaltalk{Effects of nocturnal luminance and weather on Eastern Screech-Owl vocal activity}{Kimberly M Nijoka\\Leesia C Marshall}{The Eastern Screech-Owl (Megascops asio) is a small owl common throughout eastern North America. Eastern Screech-Owls are nocturnal and cryptic, making usual methods of sampling birds unsuitable. Broadcasts of taped calls have been an effective technique in detection. However, lunar and weather factors and threat of predation by larger owls could influence calling rates. Effects of these variables on calling rates and activities of owls have been studied, but often reach contradictory results within a species. Previous studies have not directly quantified crepuscular and nocturnal luminance. Our study uses a night sky brightness photometer that measures luminance in magnitudes per square arc second to better quantify the brightness of the night as it is experienced by owls within the forest. This measure is taken along with moon-phase, temperature (°C), \% humidity, wind speed (mph), \% cloud cover, and a measure of background noise (dB). Broadcasts of Eastern Screech-Owl vocalizations are followed by broadcasts of Barred Owl (Strix varia) and Great Horned Owls (Bubo virginianus) vocalizations to determine if calling by screech owls is restricted by the presence of a predator. Principal components analysis is used to determine which variables best explain vocal activity of the Eastern Screech-Owl in the pinewoods of central Louisiana.}

\normaltalk{Song Variation of Cerulean Warblers in Northeast Iowa}{EmiLee A Novak\\Jon W Stravers\\Paul DB Skrade}{The Cerulean Warbler (Setophaga cerulea) is a species of conservation concern across its breeding range. However, one of the most western breeding populations of this species located in northeast Iowa near the Mississippi River appears to be doing well, with annual point count surveys finding territories consistently occupied from year to year. Territorial male Cerulean Warblers will sing a three-part song throughout the day. This consists of intro figures, a middle section of shorter figures, and ending with a trill. Previous studies in other parts of their range have found seasonal variation in song. During the 2016 breeding season we made several hundred recordings of territorial males to examine the song pattern for this western population. Mean song length across the season was 1.27 seconds (SE ± 0.04) with early songs generally longer (mean 1.31 secs, SE ± 0.05) and late songs shorter (mean 1.17 secs, SE ± 0.05). This was likely due to additional intro notes sung by early-season birds, as mean trill length was similar early and late in the season. Mean song length was similar in the mornings and afternoons, as was trill length, although birds more frequently had an additional intro figure in the afternoon. The length of time between songs was similar in the mornings and afternoons, but there was much more variation in the amount of time between songs. Overall, song patterns for this population are similar to Cerulean Warblers in other geographic regions.}

\normaltalk{The fledgling doesn't roam far from the nest: Red-headed Woodpecker fledgling movement patterns during parental dependence}{R Kyle Pagel\\Elena H West\\Henry M Streby}{The Red-headed Woodpecker (Melanerpes erythrocephalus) is a flagship species of the oak savanna ecosystem found in small isolated areas throughout the Midwest. Populations have declined by 70\% across the species' breeding range over the past 50 years. To identify drivers of these population declines, monitoring demographic parameters (e.g., nest productivity, fledgling survival, adult survival) is vital. In 2017, during the first year of a long-term study of full annual cycle ecology of red-headed woodpeckers, we used radio telemetry to track daily movements of fledglings at Oak Openings Preserve Metropark in northwestern Ohio, USA (n=9) and Cedar Creek Ecosystem Science Reserve in central Minnesota, USA (n=15). Daily locations, parental care, and habitat use of fledglings were recorded to identify movement patterns and potential factors influencing survival. Our findings suggest that fledglings are dependent upon parental care for 4-8 weeks after leaving the nest cavity and do not range far from the nest during this period (max = 653 m). On average, fledglings ranged 104 m (± 119 SD) and 87 m (± 62 SD) from the nest in Minnesota and Ohio, respectively. Further research is required to understand whether these small-scale fledgling movements are consistent from year to year, or if there are drivers that influence the duration and dispersal range of fledglings during parental dependence (e.g., food availability, habitat availability, population density, etc.). We hypothesize about the potential implications of these fledgling movement patterns in behavioral and conservation contexts.}

\normaltalk{Population genetics of Alaska Common Raven}{Christin L Pruett\\Tianyu Li\\Kevin Winker}{The Common Raven (Corvus corax) is widespread at high latitudes and a noted disperser. Based on their dispersal abilities, we hypothesized that Alaska populations would be genetically similar and maintain genetic diversity across thousands of kilometers. We sampled 134 ravens from 9 areas in Alaska, including six populations we considered as mainland and three as island (Kodiak, Adak, and Attu islands). We examined eight microsatellite loci and found that in most Alaska populations gene flow is high. However, we found that here are at least two raven populations in Alaska. The Attu Island population was quite different from the other sampled populations, and it was the only location that appeared to be completely isolated from other Alaska locations. Prior work suggests that the uniqueness of the Attu Island population could be due to isolation in a glacial refugium. Postglacial dispersal has not erased this signal, suggesting limits to the species' movement abilities. The Attu Island population underwent a substantial decline {raise.17exhbox{\$scriptstylemathtt{sim}\$}} 4 Kya, which roughly coincides with the arrival of humans. Traditional cultural values suggest that any effect of humans would likely be indirect.}

\normaltalk{Using environmental DNA sampling methods to determine cryptic wetland bird occupancy}{Anastasia A Rahlin\\Mark A Davis\\Matthew L Niemiller}{Wetland fragments in Illinois support over 100 bird species, 15 of which are state threatened and endangered. Wetland birds are of particular concern under the Illinois State Wildlife Action Plan due to the lack of information about their population sizes and distributions. Traditional playback methods fall short in elucidating cryptic wetland bird occupancy due to small body sizes, infrequent vocalizations, and unique habitat requirements. In this study, we investigated whether environmental DNA (eDNA) methods could be used to detect multiple rail species, and also hypothesized that taking environmental DNA samples over time would allow us to temporally track rail migration. We collected and filtered water samples from four sites from northern Illinois. We collected and purified DNA from filters using a Qiagen DNeasy kit, and quantified extracts on a Qubit 3.0 fluorometer. We amplified a short fragment of the mitochondrial cytochrome oxidase subunit 1 locus using newly designed degenerate bird primers. We positively detected eDNA in our samples, with Qubit concentrations ranging from 1.30 ng/ml to 80.3 ng/ml. As expected, negative control samples collected in the field yielded no eDNA. Degenerate primers positively detected GBHE and SORA DNA extracted from liver controls, with band fragment sizes of approximately 125 base pairs. Degenerate primers also yielded multiple {raise.17exhbox{\$scriptstylemathtt{sim}\$}}125 bp bands, showing strong evidence for wetland bird DNA detection from eDNA samples. Samples were sequenced on a MiSeq using degenerate bird and vertebrate primers. Illumina sequencing results are pending.}

\normaltalk{Woodpeckers and parasites: Testing the relationship between immune function and carotenoid levels}{Ashleigh M Rhea\\Shawn M Billerman\\Ryan J Weaver\\Douglas K Eddy\\Matthew D Carling}{Many physiological processes must take place in order for an immune response to be activated. Adequate nutrition providing all of the body's essential vitamins, minerals, pigments is a necessary component of normal health responses. Diet-based carotenoids have been linked with activating the immune response. What is unclear is whether the concentration of carotenoids present, as measured in the surface area of carotenoid-based plumage coloration, correlate with the presence of parasitic infection. To answer these questions, we used 65 Red-breasted Sapsucker (Sphyrapicus ruber) and 16 Red-naped Sapsucker (Sphyrapicus nuchalis) specimens to determine carotenoid concentration from red-colored breast and crown feathers via gas chromatography. These data were analyzed in conjunction with avian malaria (Plasmodium spp. and Haemoproteus spp.) infection rates obtained by polymerase chain reaction (PCR). We found that 40\% of our sample size was infected with avian malaria. We also compared standardized measures of the estimated red coloration on the breast and crown using ImageJ. We found that there is a significant (p=0.0116) difference between the red feather proportions on the breasts of infected individuals in comparison with uninfected individuals. The findings of our study will help shed light on how climate change facilitating the northern expansion of Culex mosquitos and avian malaria will impact these woodpeckers. Additionally, these data indicate differences in the reproductive and survival costs associated with avian malaria infection.}

\normaltalk{Haemosporidian Diversity and Prevalence in a Chickadee Hybrid Zone}{Ari A Rice\\Robert L Curry\\Jason D Weckstein}{Avian haemosporidians are single-celled parasites that cause malaria and malaria-like diseases in a wide range of bird taxa throughout the world. However, not all birds are equally susceptible to infection, and these differences in parasite resistance are thought play a major role in shaping the evolutionary processes, geographic distributions, and interspecific competition among certain birds. We are investigating whether host-specific haemosporidians play a role in the ongoing range shifts in Carolina Chickadees (Poecile carolinensis) and Black-capped Chickadees (P. atricapillus). In southeastern Pennsylvania, these species are largely allopatric except for within a narrow hybrid zone. For the past century, this zone has been shifting northward as the more southerly carolinensis seemingly outcompetes atricapillus and displaces them. In this preliminary study, we examined the prevalence of haemosporidians in chickadees from three populations: one south of the hybrid zone (n = 73 individuals tested), one north of it (n = 48), and one in its current center (n = 67). These screenings revealed infection rates of 35\% in the southern (Carolina) population, 35\% within the hybrid zone, and only 15\% in the northern (Black-capped) population. These results suggest the possibility that carolinensis is transferring a novel strain of haemosporidian to atricapillus within the hybrid zone and providing this less resistant species with a competitive disadvantage. Ongoing screenings, sampling from other chickadee populations, and information involving haemosporidian gene sequences will ultimately test whether this phenomenon occurs.}

\normaltalk{The effect of calcium on the life-span of breeding Tree Swallows}{Marina D Rodriguez\\Susan Bailey\\Kathryn P Huyvaert\\Paul F Doherty}{Stress can have various effects on the health and longevity of organisms. One mechanism whereby stress exposure affects health and lifespan is telomere length. Telomeres are non-coding, repetitive DNA sequences that cap the end of eukaryotic chromosomes whose length is highly correlated to lifespan. Telomeres can shorten due to stress and age, and telomere length can be inherited from parents. While many studies focus on factors that cause telomere shortening, very few wild studies have looked at factors that slow down or even reverse the shortening of telomeres. Because calcium has been shown to increase reproductive success in passerines, our study utilized calcium supplementation to determine if the nutrient can act as a buffer to shortening telomeres in female Tree Swallows (Tachycineta bicolor) and their offspring during the breeding season. We took blood samples from females and supplemented nests with calcium (crushed oyster shell) or a control (local soil) at the beginning of nest construction. Once chicks were 12 days old, a second blood sample was taken from mothers as well as nestlings. Other factors, such as clutch size, brood size, and age of the mother, were measured to determine their influence on telomere length. This study gives insight into the factors that influence longevity in breeding Tree Swallows that act via telomere length. By having a better sense of the role that calcium availability has on Tree Swallow telomere length, we can then gain insight on the impact that varying amounts of calcium has on avian health and lifespan.}

\normaltalk{Pathway to a Global Analysis of Geographical Variation in Avian Life History Patterns}{John T Rotenberry\\Priya Balasubramaniam}{Avian life histories are shaped by both the environmental attributes associated with a species' geographical distribution and phylogenetic constraints, the latter associated with the species' evolutionary history. As information concerning all three elements becomes increasingly available it is now possible to imagine a global analysis of life history patterns encompassing the majority of avian species. Identifying sources of data, we outline a path for combining individual species life history attributes with environmental data extracted from individual species breeding distributions, which can then be analyzed in a phylogenetic context to test alternative theories of life history evolution. We illustrate the process using a globally distributed order, Accipitriformes, examining the relative roles of breeding latitude and elevation in explaining clutch size variation.}

\normaltalk{How have ecosystem services such as bird diversity changed since the development of the Schoodic Woods campground in Acadia National Park?}{Katharine J Ruskin\\Alyson M East\\Joseph J Cicero\\Aaron L Strong}{In 2015, Acadia National Park opened the 97-site Schoodic Woods Campground, as well as 7 miles of new gravel bicycle paths, 5 miles of new hiking trails, and a new visitor center. With opportunities for increased tourism, the ecosystem services – the real economic benefits to humans provided by ecosystems – have likely changed on Schoodic Peninsula, with some ecosystem services such as tourism likely increasing, while others such as biodiversity and carbon storage likely decreasing. With this study, we aimed to (1) quantify the valuation of ecosystem services provided by Schoodic Peninsula, focusing on the three greatest contributors: recreation and tourism, biodiversity, and carbon storage, and 2) estimate how these values have changed with the development of Schoodic Woods. We conducted 184 avian point counts and 19 vegetation transects to estimate biodiversity, used these vegetation transects and 8 woody debris transects to calculate carbon storage, and gathered 790 visitor surveys along with 15 semi-structured stakeholder interviews to estimate value of tourism opportunities. Preliminary results suggest that bird species richness is highest near human developments, especially the campground, and decreases moving away from it. The results of this project will inform the evolution of development on Schoodic Peninsula, creating an opportunity for strategic growth that preserves ecosystem services such as biodiversity and carbon storage, while optimizing tourist revenue and benefitting local businesses.}

\normaltalk{Spotted Towhee (Pipilo maculatus) song characteristics at two time points across an urban gradient}{Karina A Sanchez\\Lauryn Benedict\\Carla Cicero\\Kristina Fialko}{Bird songs and calls serve numerous functions including resource defense, species identification, and attracting and retaining mates. As ecosystems are altered due to the rapid growth of human populations, avian species have been observed to alter song behaviors, which are critical to individual fitness. For organisms that rely heavily on acoustic communication, it is important to understand the effects of anthropogenic environmental changes on signaling behavior and outcomes. Studies have shown that birds alter the temporal timing, frequency, amplitude, and duration of their songs to avoid being masked by anthropogenic noise. In this study, we examined the effects of urban environment over time and space on song characteristics of the Spotted Towhee (Pipilo maculatus). Over time and with an increase in urbanization, we should see an increase in minimum frequencies, amplitude, and song length. Recordings of 191 individuals from 1970 and 175 individuals from 2015 were used to analyze the differences in song frequency, amplitude, and song duration between and within populations along an urban to rural gradient stretching from Berkeley to the foothills of the Sierra Nevadas in El Dorado County, CA. The results of this study will shed light on how song characteristics change due to anthropogenic affects over temporal and geographic scales.}

\normaltalk{Frugivorous birds in recovering sagebrush steppe habitat: Frenemies of restoration?}{Gabriel L Sandoval\\Dana M Sanchez\\Jacob W Dittel}{Sagebrush steppe is an imperiled ecosystem due to numerous interacting pressures such as the expansion of western juniper, which exerts strong negative effects on native plants and wildlife. The Phillip W. Schneider Wildlife Area (PWSWA) is a degraded sagebrush steppe ecosystem undergoing management to restore wildlife habitat through reduction (cutting) of juniper. Ironically, through eating and dispersing juniper berries into restoration treatment patches, birds may be countering efforts to restore shrubs and understory plants. Species composition of the area's breeding season bird community had not previously been documented. Furthermore, the relationship among birds, fruits of juniper trees bordering restoration patches, and potential perching (and seed deposit) structures, specifically downed juniper, within the restoration patches was unexplored. Our objective was to understand the role that frugivorous birds might play in this restoration landscape. We conducted point-count surveys during the 2017 breeding season to document the bird community composition. We observed that 14\% of the species were frugivorous. Then during fall behavioral observations we observed four common species of birds consuming juniper berries, with the American Robin and Townsend's solitaire being the most frequently observed. During observations, birds spent an equal time foraging and roosting. Birds, however, spent the most time and defecated most often within living juniper stands and not within cut juniper patches. Many of our observations demonstrate a common pattern in seed dispersal mechanisms, in that most seeds are likely deposited relatively close to the parent tree, suggesting frugivorous birds are not disproportionately dispersing seeds into restoration patches.}

\normaltalk{On the fruit color preferences of the Southern Cassowary (Casuarius casuarius)}{Vinodkumar Saranathan\\Adam HY Goh}{The ratite, Southern Cassowary (Casuarius casuarius), is a ‘keystone' fruit disperser in the Australian rainforests. And yet, currently it is unclear whether they exercise any fruit color preferences while foraging in the wild, and if so, what colors they do prefer. Towards this end, over six non-consecutive days, we presented three female adult captive cassowaries with a total of 45 model fruits each dyed with nine artificial food colors corresponding to three distinct categories: naturally-occurring fruit colors that also occur on the bodies of cassowaries (red, blue, black), naturally-occurring fruit colors that are absent in cassowaries (yellow, purple, pink), and novel hues outside the gamut of available fruit colors (grey, gold, turquoise). We observed their feeding behavior and compared both the order and frequency of each fruit color consumed. The choice of model fruits is non-random with red and pink colors significantly preferred when all three individuals are considered, consistent with prior studies of color choice in frugivorous (passerine) birds, suggesting perhaps a conserved innate fruit color preference in Aves. While there were no significant differences in the choices between the individuals, they did consistently exhibit some idiosyncratic differences in preferences (one individual preferred more yellow and gold fruits over red or pink). Based on tetrachromat visual analyses of the model fruit colors in relation to the visual ecology of these ratites, we conclude with important implications on the evolution of fruit colors in the Australian rainforest ecosystem (e.g. in Elaeocarpaceae).}

\normaltalk{Do Blue-winged and Golden-winged Warblers adjust their songs in response to military noise?}{Joanna M Sblendorio\\Amy E Janik\\Eric R Britzke\\Maarten J Vonhof\\Sharon A Gill}{The military manages large expanses of land across the United States that serve as breeding habitat to threatened species, such as the Golden-winged Warbler. Military training operations on these lands produce anthropogenic noise that can either be chronic or sudden, and disrupt bird breeding activity and communication. Birds depend on acoustic signaling to attract mates, defend territories, and warn neighbors of predators. To minimize acoustic masking by noise, male songbirds can increase the rate, frequency, duration, and amplitude of their songs. Military noise reduces breeding success and habitat occupancy, but the effect of high-intensity, intermittent noise on avian communication is not well understood. Thus, the objective of our study is to analyze the temporal and spectral song adjustments used by males to reduce noise masking and maintain signal transmission across a gradient of military training noise intensity. We recorded the songs of focal male Golden-winged (n=31) and Blue-winged Warblers (n=51) and their hybrids (n=9) across the training areas at Fort Drum, New York. We analyzed song spectral characteristics, and preliminary analysis suggests that both Golden-winged and Blue-winged Warblers occupying noisier territories produce songs with higher peak frequency and of longer duration than warblers from quiet territories. Our study is a critical step in informing habitat management of military lands for species that depend on vocal communication.}

\normaltalk{Southeastern American Kestrel nestling provisioning in the West Gulf Coastal Plain}{Richard R Schaefer\\D C Rudolph}{The Southeastern American Kestrel (Falco sparverius paulus) is a permanent resident of the southeastern United States, and reaches its western range limit in the pine-dominated forests of eastern Texas. This subspecies has declined in many areas due to the loss of nesting and foraging habitat, and there has been little research in the western portion of its range. A majority of kestrel nests we found (83\%) were located in forest stands managed for Red-cockaded Woodpeckers. We recorded 412 prey deliveries (males = 225, females = 187) to nest sites by breeding kestrels in eastern Texas and west-central Louisiana. Females delivered mostly invertebrates (88\% of all female deliveries), and grasshoppers were the most common prey item (n = 55). Other invertebrates occasionally provided to nestlings included dragonflies, cicadas, katydids, crickets, roaches, spiders, and caterpillars. The most significant vertebrate prey delivered by females was the ground skink (Scincella lateralis). However, males delivered primarily vertebrate prey (85\% of all male deliveries), with green anoles (Anolis carolinensis) being the most common prey item (n = 121). Other vertebrate prey provided by males included several other lizard species, and rarely a mouse or bird nestling. The green anole was the single most common prey item delivered to kestrel nestlings by males and females combined, and accounted for 36\% (147 obs.) of prey deliveries. Green anoles are an abundant and relatively conspicuous prey source, favoring shrub-level vegetation within Southeastern American Kestrel nesting territories in the West Gulf Coastal Plain.}

\normaltalk{Abundance and Diversity of Land Birds in Puerto Rican Coastal Dry Forest in the Aftermath of Hurricane María}{Fred C Schaffner\\Mariangely Cruz-Col\'{o}n\\Soely Luyando-Flusa\\Braiam A Rosado-Ramos\\Ian J P\'{e}rez--Cruz\\Linoshka M Flores-Garc\'{i}a}{On September 20, 2017, Hurricane Maria struck the island of Puerto Rico, and is considered the worst natural disaster on record for the island. It was the tenth-most intense Atlantic hurricane on record and the most intense tropical cyclone worldwide of 2017, the thirteenth named storm, eighth consecutive hurricane, fourth major hurricane, second Category 5 hurricane, and the deadliest storm of the hyperactive 2017 Atlantic hurricane season. This storm had devastating effects on habitat throughout the entire island of Puerto Rico, including massive crown loss and defoliation of mangroves and coastal secondary dry forest. Herein we present the results of our mist netting sampling of resident and migratory land birds, a preliminary assessment of hurricane impact in an interhabit corridor connecting mangroves with coastal dry forest at Jobos Bay in Salinas, Puerto Rico (Jobos Bay National Estuarine Research Reserve, JBNERR). This assessment includes sampling just a few days before the storm and from 2 months onwards after the storm, with notable changes in species composition and relative abundance.}

\normaltalk{Recoveries of Archival Light-Level Geolocators from Northern Waterthrushes (Parkesia noveboracensis), in Puerto Rico – Size Matters}{Fred C Schaffner\\Ivelisse Rodr\'{i}guez-Col\'{o}n\\Soely Luyando-Flusa\\Mariangely Cruz-Col\'{o}n\\Braiam Rosado-Ramos\\Ian J P\'{e}rez-Cruz}{The effective conservation of long-distance migratory birds requires the identification and protection of the birds' stopover and wintering habitats, in addition to nesting habitat. We wish to establish the specific intercontinental migratory connectivity of the 15-20 g Northern Waterthrush (Parkesia noveboracensis), a species breeds in North America and passes the winter in the Caribbean, Central and South America. We deployed 40 archival light-level geolocator data logger units of two different designs and total masses on Northern Waterthrushes at two sites at Jobos Bay in Salinas, Puerto Rico (Jobos Bay National Estuarine Research Reserve, JBNERR) in early successional and late successional coastal dry forest habitat from Sep-Dec 2015 (19 units) and Jan-Apr 2016 (21 units). Of the 40 geolocators deployed, overall, 11 (27.5\%) were recovered from September 2016 to April 2017, while 47 of 152 non-tagged birds (30.9\%) were recaptured for these same periods. However, only 3 of 20 birds (15.0\%) carrying the larger units were recaptured, while 8 of 20 birds (40.0\%) carrying the smaller, lighter units were recovered. Despite the potential confounding effects of season and habitat, we conclude that size (mass) is the most important factor influencing the recovery of geolocator-tagged Northern Waterthrushes.}

\normaltalk{Comparison of DNA extraction methods from museum specimens}{Margaret E Schedl\\Whitney LE Tsai\\James M Maley\\John E McCormack}{Next Generation Sequencing has greatly expanded the utility of museum collections by making specimens available as genomic resources. For avian museum specimens, the established method of taking tissue from toe pads works well for most specimens. However, for some types of very small and very large specimens, toe pads are a poor source of DNA. As the field of museum genomics grows, so does the need for extraction methods that maximize DNA yields. In this study, we examined alternative sources of tissue from avian specimens, and compared two methods for DNA extraction. We collected toe pads, skin punches, and bone fragments from 10 bird specimens over a range of species from the teaching collection of the Moore Laboratory of Zoology. For each sample, we extracted DNA using a modified Qiagen DNeasy protocol and a phenol chloroform protocol. Samples extracted using the phenol chloroform protocol yielded more DNA than samples extracted using the modified Qiagen DNeasy protocol. While toe pads yielded more DNA on average than skin punches, skin punches proved to be a reliable alternative source for DNA when toe pad samples fail.}

\normaltalk{Evolution and Taxonomy study of Herons in East Africa}{Golya Shahrokhi\\Michael A Patten}{The herons (Aves: Ardeidae) constitute one of the taxonomic families with many unresolved phylogenetic relationships, both within and among clades. Although some morphological and molecular studies have been done on different species and resolved some basic relationships, several species of this family have not been studied at all. The Little Egret (Egretta garzetta), Western Reef-Heron (Egretta gularis), and Dimorphic Egret (Egretta dimorpha) are three members of Ardeidae with many ambiguities in their taxonomic relationship. Their similar morphology, overlapping ranges, and reports of possible inbreeding has pushed taxonomists and ornithologists into controversial conflicts about how to classify them. The primary objective of this research is to use mitochondrial and nuclear markers to help clarify the phylogenetic relationships of these three taxa. I also will test for the prevalence and origin of hybridization among them using both molecular markers and spatial analyses. I plan to use tissue and blood samples from both preserved (from museums) and wild specimens. After collecting samples, the extracted DNA will be amplified for both mitochondrial (mtDNA) and nuclear markers. The mtDNA markers are cytochrome b (cytb) and cytochrome c oxidase subunit I (COI) and the nuclear markers are the color morph gene (MC1R) and two additional markers. For hybridization, I would screen single nucleotide polymorphisms (SNPs) to obtain abundant data on the extent of interbreeding as well as obtain a picture of dispersal trends. The analyzed data will be joined with geographical and spatial analyses to clarify the history and evolutionary pattern of their speciation and divergence.}

\normaltalk{Secondary “Market” for Bird Feathers: What Happens to Feathers After Birds Shed Them?}{Daniel P Shustack}{Every year billions of birds shed and replace trillions of feathers. Yet these dropped feathers do not build up in the environment. This is a peculiar observation given that feathers are composed primarily of keratin, a tough and decay-resistant protein. I conducted a series of experiments to evaluate the fate of shed feathers under natural conditions. First, I placed flight feathers of Gray Catbirds and Cedar Waxwings in litter bags for {raise.17exhbox{\$scriptstylemathtt{sim}\$}}2 years on the forest floor in Western Massachusetts. Litter bags eliminated feather removal and gnawing by rodents, Trogid and Dermestid beetles, and Teinid moths. While a few feathers were mostly decomposed over these two years, some remained remarkably intact, and overall, mass loss was negligible ({raise.17exhbox{\$scriptstylemathtt{sim}\$}}2\%, p=0.4). Second, I tethered flight feathers from Barred Owls to the forest floor without the protection of litter bags. After 12 months, the feathers remained largely intact, but by 20 months, most of the feather was gone, except the calamus which was tied to the fishing line. Third, I placed trail cameras in front of piles of feathers and individually tethered feathers. Most animals detected on the trail cameras ignored the feathers. However, a few rodents (eastern chipmunk, gray squirrel, and Peromyscus) were detected ($<$5\% of observations) gnawing on the feathers. I failed to detect a single keratin-consuming insect during this study.}

\normaltalk{Habitat-specific variation in bobwhite chick body condition on native prairies and traditionally managed Conservation Areas in Missouri}{Emily A Sinnott\\Frank R Thompson}{Growth rates of precocial chicks reflect variation in local forage condition and habitat suitability of brood-rearing areas. Environments with lower food quality or abundance may result in starvation-induced weakness, reduced body mass, and lower chick survival. Body size is positively associated with speed and endurance and advanced mobility may improve foraging success and predator avoidance of young. Bobwhite chick survival remains low until young reach a mass of 50-g. Habitat conditions that constrain growth rate may be an ultimate factor affecting juvenile survival. Our objective was to assess habitat-specific variation in body condition of bobwhite young. We captured 21 broods on five Conservation Areas in southwest Missouri in 2017. Three sites were native prairie and two were traditionally managed areas with patches of grassland and agriculture. We captured broods at 5–12 and 16–35 days of age and measured mass and tarsus length of all individuals. We used a mixed models with a random brood term to account for intra-brood dependence, a random intercept for each individual to account for repeated measures of subjects, a fixed effect of tarsus to account for structural size, and a fixed effect for We found support for an effect of both tarsus length and dominant cover type. Mass was significantly higher in native-grass plantings and agricultural units, followed by native prairie, mixed grass, and restored prairies. Insight into juvenile physiological responses to habitat type and management will enhance our understanding of mechanisms underlying survival of young.}

\normaltalk{Rain or shine, telomeres are fine: Variable weather influences incubation length, but not telomere length at hatch in house sparrows}{Aubrey E Sirman\\Aurelia C Kucera\\Britt J Heidinger}{In our free-living nest box population of house sparrows (Passer domesticus) in Fargo, North Dakota, incubation length varies considerably. During the breeding season, summer weather is also highly variable. Previous research in birds has demonstrated that telomere length at hatching is predictive of lifespan, and impacts lifetime reproductive success. However, little is known about the factors that influence early life telomere length. Because incubation length is variable it may play a role in telomere length at hatching. Several factors might also influence incubation including weather variables. This is supported by previous data in our population that demonstrated that early life telomere length is impacted by weather. We collected blood samples from nestlings in the summer of 2016 and measured telomeres using quantitative PCR. We hypothesized that incubation length may mediate the relationship between weather and early life telomere length. To explore this relationship, we did a principle components analysis (PCA) to reduce the dimensionality of the weather data and then used path analysis to identify relationships between weather, incubation period, and telomere length. We found that weather components did influence incubation length, supporting previous findings. However, incubation length did not impact telomere length at hatching. Interestingly, nest attempt positively influenced telomere length at hatch, which subsequently positively impacted telomere length near fledging. Together, these results suggest that while weather does play a role in incubation length, the relationship between weather and early life telomere length is more complex than predicted.}

\normaltalk{Mapping patterns of vocal variation in the native range of an invasive parrot}{Grace Smith-Vidaurre\\Tim Wright}{Biological invaders that have repeatedly and independently colonized human-altered habitats provide natural experiments to ask how processes like social learning facilitate rapid adaptation to human-induced environmental change. Monk parakeets (Myiopsitta monachus) have invaded cities across the world through dispersal for the global pet trade. Previous work with various parrot species has shown that geographic variation in call types often results from social learning rather than genetic differentiation. Previous research with invasive monk parakeets in the U.S. described discrete, dialect-like patterns of contact call variation, even over short geographic distances. Here we mapped geographic variation in contact calls in a region of monk parakeets' native South American range that is considered a source for invasive populations. We recorded native monk parakeet populations at over 40 sites across a 400km transect along the coast of Uruguay from May – November 2017. We used Raven and the recently developed warbleR package to select, filter, and visualize calls. We used machine learning approaches in R to classify calls across the geographic transect. Geographic variation patterns in native monk parakeet contact calls are complex, characterized by high levels of variation within sites and clinal changes in acoustic structure across the transect. These patterns are different from previous findings of discrete (rather than clinal) contact call variation in invasive U.S. populations, pointing to changes in social structure and/or social learning processes as likely signatures of invasion. Our findings will provide a foundation for future work investigating how acoustic signals change after invasion of human-altered habitats.}

\normaltalk{Landbird Species at Risk in Forested Wetlands of Nova Scotia}{Cindy A Staicer\\Erika Nissen\\Suchinta Arif}{Forested wetlands are declining, understudied and important ecosystems rich in biodiversity. In Nova Scotia, three landbird species at risk (SAR), the Olive-sided Flycatcher (OSFL; Contopus cooperi), Canada Warbler (CAWA; Cardellina canadensis), and Rusty Blackbird (RUBL; Euphagus carolinus), breed mainly in forested wetlands. This project aims to help conserve these species through research, stewardship, outreach, and volunteer involvement in monitoring. Current research focuses on identification of high quality habitat that supports greater densities, higher reproductive rates, and more stable populations. In 2017, 10-min point counts were conducted at 120 sites to document breeding bird communities across a range of forested wetland types. CAWA (at 19 sites) co-occurred less often than chance with 3 other species, and OSFL (at 17 sites) co-occurred less often than chance with 2 species but more often with 4 species. We obtained densities of breeding birds using the R packages detect, to model detectability using both distance and time to first detection, and cmulti. To increase SAR detection rates, playback surveys (30-s of conspecific songs and calls) were conducted following point counts or listening periods. Playbacks yielded 31\% of CAWA, 24\% of OSFL, and 17\% of RUBL detections in these surveys. Additional SAR occurrence data were obtained through opportunistic encounters and from volunteers, mainly through eBird. We extracted landscape-scale habitat variables using ArcGIS. Detailed vegetation surveys conducted at 37 sites revealed the importance of cinnamon fern, high shrub cover and coarse woody debris for CAWA, and the cover of low shrubs and conifers for OSFL.}

\normaltalk{Continued Decline of the Yellow-billed Cuckoo in California}{John R Stanek}{Thirty years ago cuckoo biologists Laymon and Halterman (1987) asked “can the western Yellow-billed Cuckoo be saved from extinction”. They cite a southward range contraction, extirpations and significant population declines observed in the western U.S. over the last century. California's then estimated cuckoo population of 50-75 pair was largely restricted to two populations found along the Sacramento River and in the Kern River Valley. Despite its listing as a Threatened species in 2014, the downward population trend has continued. The Sacramento River population appears to be locally extinct, and the Kern River Valley population experienced an 85\% decline in the last five years and was down to one pair in 2017. The last known viable population in CA resides on the Lower Colorado River and in 2017 experienced its first population decline in over a decade of monitoring. Across the species range apparently suitable habitat remained unoccupied and while habitat loss has been attributed as the major factor in the species decline, it may not be the immediate factor impeding its recovery. Reasons for the continued decline of the western yellow-billed cuckoo are unclear, and new research into potential causes of decline, including prey availability and wintering habitat conditions, need to be explored if the cuckoo is to be saved from extinction in the U.S.}

\normaltalk{American Crows Roost Near Development and Roads, but in Different Places Each Night During the Winter of 2016-17}{Ben Steele\\Jeremy Johnston}{American Crows (Corvus brachyrhynchos) collect in large roosts at night during the non-breeding season. We observed a roost near Lebanon, NH to determine the location and size of the roost. We located the roost on most nights between Oct 26, 2016 and March 19, 2017. Although it was rare to see all the crows at one time before it was dark, our maximum estimate was 6000 crows. With a few exceptions, the roost was in a different location each night, although always near (within .2 mi. of) roads, parking lots, businesses, and streetlights. The crows spent the night in trees or on the tops of buildings, and occasionally on the ground. Over the course of the winter, the roost moved more than 9 mi., but was always close to the location of the previous night. Our observations were consistent with the hypotheses that roosts serve as protection from predators or as pooled information on food resources, but collective warmth was rejected as a hypothesis because crows never perched more than one foot from each other and roosted in deciduous rather than pine trees even on cold windy nights.}

\normaltalk{Balancing personal maintenance with parental investment in a long-lived seabird, the common murre}{Anne E Storey\\Michelle G Fitzsimmons\\Amy-Lee L Kouwenberg\\Greg J Robertson\\Carolyn J Walsh}{We examined whether variation in several physiological indicators reflects the balance between parental investment and survival in common murres (Uria aalge) under a wide range of foraging conditions. Blood samples were taken from adults during mid chick rearing in eight breeding seasons and analysed for corticosterone (CORT, stress hormone), beta-hydroxybutyrate (BUTY, lipid metabolism reflecting ongoing mass loss), and haematocrit (reflecting blood oxygen capacity). These measures, plus body mass, were related to three levels of food availability (good, intermediate, and poor years) for capelin, the main forage fish for murres in this colony. Results indicate that in good years, murres can delay their mass loss, yet adequately provision their chicks (higher mass, higher BUTY levels, higher chick-feeding rates). In contrast, murres appeared to work harder in intermediate years (higher CORT, higher haematocrits, lower mass), and they shifted to increased personal maintenance in poor years (lower mass, lower chick-feeding rates). Heavier birds had higher current reproductive success and birds with lower CORT fledged more chicks across years. Our multi-year data set provides insight into how decisions about resource allocation reflect physiological indicators under different foraging conditions. This relationship between foraging conditions and physiology may help us understand how seabirds will respond to changes in marine ecosystems as ocean temperatures continue to rise.}

\normaltalk{35 years of avian colonists to disturbed Amazon rainforest: Which birds, and how did they get there?}{Philip C Stouffer\\Cameron L Rutt\\Vitek Jirinec\\Mario Cohn-Haft}{Removal of rainforest inevitably leads to an altered avifauna as deforested areas are colonized by non-forest species. These birds may be dispersing broadly, settling based on their attraction to features of the modified landscape, or they may be passively diffusing along dispersal corridors. The avifauna of the Biological Dynamics of Forest Fragments Project (BDFFP), on three large ranches north of Manaus, Brazil, has been well characterized for over 30 years. Deforestation in the 1980s led to an influx of non-forest species to pastures and second growth. Since then, there has been little additional deforestation, and forest has returned in much of the landscape. Despite the landscape becoming generally more like the original terra firme rainforest, we have gradually recorded 21 new bird species, 75\% of which are species associated with várzea and second growth (i.e. river margins and urban Manaus). These birds appear to represent populations expanding along 80km of modified landscapes on the highways between Manaus and the BDFFP. The roads have been in place since the 1970s, with relatively little but gradually increasing collateral development. Our results suggest that roads may have been less important for early colonization by some widespread generalists (e.g. Columbina spp., Tyrannus melancholicus, and Troglodytes aedon) but have probably facilitated expansion by more localized várzea species (e.g. Megascops choliba, Megarynchus pitangua, and Cacicus cela). Continued expansion of road networks into terra firme rainforest will likely lead to novel bird communities that include species whose central Amazonian distribution was previously restricted to major rivers.}

\normaltalk{Arizona Lovebird Habitat Selection Inquiry Project}{Katherine A Studey}{The rosy-faced, or peach-faced lovebird, (Agapornis roseicollis) is a small parrot native to southern Africa. Lovebirds are also popular pets in the United States. Due to a number of releases of captive birds in the Phoenix Metropolitan area, this particular species of lovebird is in the process of establishing a feral population. Observations of these birds over the past few decades indicates that the population is successfully breeding and increasing in numbers in the Phoenix area. In order to better understand how rosy-faced lovebirds have been able to adapt to their non-native habitat, an analysis of land cover types present at lovebird observation locations was conducted. This was accomplished by using ArcGIS mapping software and observations reported to citizen science websites. I hypothesized that lovebirds would be found more often in urban habitat types rather than non-urban habitats. The results of the study supported my hypothesis, and they indicate that the success of this non-native bird is due to resources available in an urban environment.}

\normaltalk{Land Trusts and Birds: Partners in Strategic Conservation}{Sara Barker Swarthout\\Ronald W Rohrbaugh\\Ashley A Dayer\\Amanda D Rodewald}{More than 60\% of the land area in the United States is privately owned, and more than 100 bird species have $>$50\% of their U.S. breeding distributions on those lands. Unfortunately, conserving private lands is complicated by both individual and institutional barriers, thus leaving birds reliant on private lands with inadequate protection and management. Land trusts are an increasingly popular mechanism to protect private lands and potentially conserve birds and their habitats. In 2013, the Cornell Lab of Ornithology used social science-based methodology to conduct a national, online survey of land trusts and their attitudes toward bird conservation. Results indicated that land trusts, if supported with science and technology, could achieve landscape-scale conservation for birds. To develop mutually beneficial collaborations between land trusts and the bird conservation community, we established the Land Trust Bird Conservation Initiative. The initiative provides: 1) access to science-based information about birds to inform strategic conservation planning, investment decisions, prioritization of easements and acquisitions, grant writing, and landowner engagement; 2) resources and tips on bird-focused funding opportunities; 3) ideas to cultivate new members and volunteers by engaging birdwatchers, bird organizations, and bird conservation advocates; 4) guidance for habitat management on fee-owned lands and resources for landowners holding easements; 5) connections with science-based bird conservation resources and land trust success stories; and 6) assistance with monitoring birds and visualizing data through eBird.org. Our poster will summarize the survey and describe how the Initiative is using science and outreach to conserve bird populations on private lands.}

\normaltalk{Hooded Warbler Habitat Use in the Delaware Water Gap National Recreation Area and an Association with Japanese Barberry}{Brandon W Swayser\\Jennifer A Kutch\\Terry L Master}{Hooded Warblers (HOWA) (Setophaga citrina) have become common in the Delaware Water Gap National Recreation Area (DEWA) of Pennsylvania/New Jersey in the last decade. Point counts from a previous study (2011-2012) in randomly-chosen locations, each containing Japanese Barberry (Berberis thunbergii) dominated, transitional, and native-dominated understory plots, revealed significantly more detections from plots with barberry (barberry-dominated and transitional) vs. native-dominated understory. The objectives of the current study (2016-2017), largely using the same study plots, were to confirm HOWA preference for barberry as a nesting substrate by monitoring territories/nests, determine the age of breeding HOWA (mostly males) in each plot type, and measure site fidelity in the three plot types. Vegetation analyses compared the three plot types and available vs. territorial sites. Territories were significantly more common in barberry-containing plots. Only three nests were monitored due to difficulties visually locating nests. A significantly higher frequency of ASY (older) birds was found in plots containing barberry. Too few HOWA returned to compare site fidelity among plot types. Understory Importance Values for Japanese Barberry, Spicebush (Lindera benzoin), and Highbush Blueberry (Vaccinium corymbosum) were higher than expected in territories compared to available habitat. Three predictors, understory density, barberry coverage, and shrub height, contributed significantly to the response variable of presence/absence of HOWA territories in a binary logistic regression. Territory occurrence and age suggest that HOWA prefer barberry as a nesting substrate. Range expansion of both HOWA and this exotic invasive barberry species in the Northeast coincide anecdotally.}

\normaltalk{Inside the Birds of the World: CT-scanning fluid-preserved bird collections via the oVert Thematic Collections Network}{Oona M Takano\\John M Bates\\John P Dumbacher\\Ben D Marks\\Robert G Moyle\\A T Peterson}{The oVert (openVertebrate) Thematic Collections Network is a new collaborative initiative among museums across the U.S. which aims to CT-scan {raise.17exhbox{\$scriptstylemathtt{sim}\$}}20,000 fluid-preserved vertebrate specimens, representing over 80\% of extant vertebrate genera. This project will generate free, publicly available three-dimensional anatomical data housed in the online MorphoSource database. These data will facilitate research in various branches of biology, including exploration of taxonomic relationships, developmental morphology, and evolution. Additionally, the scans generated by oVert will be made accessible to K-12 students through teacher workshops to create anatomy lesson plans incorporating both digital and 3D printed specimens. Among major taxonomic groups included in the oVert scanning effort will be {raise.17exhbox{\$scriptstylemathtt{sim}\$}}1700 genera of birds representing approximately 70\% of bird diversity. While fluid preservation is an uncommon method for preserving bird specimens, initial searches for this project have uncovered a large number of fluid-preserved birds in U.S. collections allowing this project to tap into a long overlooked resource for ornithological studies. Initial searches for fluid-preserved birds have also highlighted a challenge of using informatics approaches to query existing specimen databases, exemplified by the {raise.17exhbox{\$scriptstylemathtt{sim}\$}}600 different descriptions of preparation types presently used on VertNet to identify bird specimens that are partially or completely fluid-preserved. Applications for present and future research include comparative paleontological studies and functional morphology, but these data will also facilitate teaching and access to anatomy for rare taxa.}

\normaltalk{The link between social/familial relationships and ecto-parasite infection}{Jesse A Taylor\\M M Webster\\J F Welklin\\S Khalil\\J P Swaddle\\Jordan Karubian}{Physical interactions between animals increase their risk of a host of negative consequences from spreading pathogens to being the victim of aggressive behavior. This makes understanding the patterns of when and with whom animals choose to interact an interesting subject; it implies that whatever is gained from these social interactions outweighs the potential risks. The purpose of this study was 1) to determine if the removal of ectoparasites from the receiving bird was the reason that allopreening is such a common behavior and 2) what social factors influence patterns of allopreening. Through behavioral observations of color banded birds, I was able to determine the proportion of time each bird spent allopreening with the individuals in their group. By comparing the amount of time a bird received allopreening to the number of parasites present on its wing I saw that increased allopreening was loosely correlated to a decrease in an individual's ectoparasite infection. In fairy-wrens some males have brightly colored red and black plumage and they were the ones most commonly allopreened. Bright males were preened most by females but also by the younger, dull brown males in their group. There was also evidence that allopreening was highly reciprocal, with dull birds who allopreened the bright male the most also received the most allopreening from him. The difference in allopreening based on identity supports the idea that allopreening is a parasite removal technique but also could serve as a method for strengthening social relationships and/or increasing the fitness of related individuals.}

\normaltalk{Monitoring the Response of the Bird Community to Restoration of the Owens River, California}{Robert D Taylor\\Eric M Wood\\Sacha K Heath\\Deborah House}{The vast majority of river systems in the US have been altered by humans, especially through water abstraction for domestic and industrial use. A striking example is the Owens River in California, which was diverted from its natural channel in 1913 to provide water for the Los Angeles metropolitan region. Diversions left the river dry and drastically reduced the availability of riparian habitat throughout the Owens Valley. In 2006, a restoration plan was implemented to restore water to the natural channel of the Owens River. I used a ‘before-and-after' survey design to quantify the responses of bird communities and vegetation to rewatering. Prior to rewatering, Point Blue Conservation Science (formerly PRBO) established 173 point count stations along 62 miles of the Owens River and surveyed bird communities and vegetation in 2002 and 2003. After rewatering, the County of Inyo and the Los Angeles Department of Water and Power completed additional surveys in 2010 and 2015. I completed surveys at existing point count stations in 2017. Using a generalized linear modeling approach, I found that riparian-affiliated bird species increased in abundance following rewatering, likely due to increased riparian vegetation cover. For example, Red-winged Blackbird (Agelaius phoeniceus), Marsh Wren (Cistothorus palustris) and Common Yellowthroat (Geothlypis trichas) increased in abundance by 185\%, 126\% and 311\%, respectively. In general, bird community composition has shifted towards dominance by riparian-affiliated bird species. These results indicate a gradual recolonization of the riparian zone by a riparian-affiliated bird community following rewatering of the Owens River.}

\normaltalk{Avian Malaria and Vertebrate Host Population Dynamics in Southern Louisiana}{Eric J Tobin\\Elizabeth C Heintz\\Scott M Duke-Sylvester}{Avian malaria (Genera Plasmodium \& Haemoproteus) is a cosmopolitan infection, found in nearly every bird species surveyed for the Apicomplexan parasite. Despite this, few studies have assessed the fitness effects of chronic avian malaria infections in natural populations. In conjunction with Louisiana Bird Observatory, we have collected demography data and blood samples from April 2013 to present, giving us a long-term dataset with which to model reductions in fitness on two resident Passerine species, Northern Cardinal (Cardinalis cardinalis) and Carolina Wren (Thryothorus ludovicianus). We have developed mark-recapture models and epidemiological models to test hypotheses to explain the patterns of malaria diversity and infection prevalence in our systems. Our study populations have estimates of infection prevalence over 90\%. In addition we have used molecular techniques to determine sex of T. ludovicianus which allow us to estimate differential survivorship of males and female, providing a more detailed view in to the ecology of this species.}

\normaltalk{Increasing survival of wild macaw chicks using foster parents}{Gabriela Vigo Trauco\\Donald J Brightsmith}{The use of foster parents in avian population management is a technique with great potential to aid in the recovery of highly endangered species. However, few studies have studied how to accomplish this successfully. Our research shows that Scarlet Macaws in southeastern Peru hatch 2-4 chicks per nest but just 1.3 of them fledge. Here about 22\% of all hatched chicks die of starvation and starvation is the most common cause of chick death. Parents always raise the first chick that hatches, but 45\% of second chicks, 97\% all of third and 100\% of all fourth chicks are left to starve to death by their parents. Our goal was to develop and test new techniques to increase survival of wild Scarlet Macaw chicks by reducing chick starvation. We hypothesized that we could pull chicks at risk of starvation, raise them in captivity to about 18 days of age then move them to nests with only one chick to increase their chances of survival. Our results show that all translocated macaw chicks were successfully accepted by their foster parents (N=15 chicks, 2 consecutive breeding seasons) and 93\% of the translocated chicks fledged successfully. Overall we increased fledging success per available nest from 18\% (1999 – 2016 average) to 29\% (2017 and 2018) and decreased chick death by starvation from 19\% to 4\%. These findings show that the use of foster parents in the wild is a promising management tool to aid wild parrot population recovery in areas with low reproductive success.}

\normaltalk{A complete evolutionary history of New World Jays}{Whitney L Tsai\\Elisa Bonaccorso\\Emiko M Schwab\\Benjamin Scott\\James M Maley\\John E McCormack}{New World Jays are a clade within the broader Corvidae, comprising a radiation of 38 species in North and South America. The group has been an interesting system in which to study the biogeography of the Americas and its colonization by endemic radiations, as well as the evolution of cooperative breeding behavior. We present the first complete dated phylogeny of New World Jays including 83 New World Jay taxa (all species and many subspecies) and 4 outgroups built from 3,142 UCE loci. In addition to providing new resolution on the timing of divergences and taxonomic recommendations for this group, we also assess its implications for biogeography and phenotypic evolution.}

\normaltalk{Distribution and Ecology of the Birds of the Sierra Cacachilas, Baja California Sur}{Philip Unitt\\Lori Hargrove\\Gorgonio Ruiz-Campos\\Daniel Galindo\\Lea Squires\\Kevin Clark}{Previously unexplored biologically, the Sierra Cacachilas, just east of La Paz, Mexico, reaches a maximum elevation of 1277 m. It is an outlying range of the well-explored Sierra de la Laguna, home to many endemic montane taxa. We documented its biota with respect to elevation and habitat, and to ascertain whether the Sierra Cacachilas supports outlying populations of the endemics of the Sierra de la Laguna. We investigated birds' use of the elephant tree (Bursera microphylla), which Bates hypothesized had coevolved with the Gray Vireo (Vireo vicinior). Our seven surveys, 2013–2016, included 181 point counts during fall/winter with distance sampling along transects at elevations 152–1277 m. We used “Unmarked” to model bird abundance and probability of detection in relation to habitat, elevation, and temporal variables. Of 27 species modeled, 5 increased in abundance with elevation; 13 decreased. The fruit of Bursera microphylla was eaten by 15 species, including Vireo vicinior, which defended territories and contested fruit with other fruit-eaters. Cover of Bursera microphylla was important in models for Vireo vicinior, Melozone crissalis, and Amphispiza bilineata but negatively correlated with some other fruit-eaters. We identified 102 species of birds, but of 13 possible montane endemics, found only Aimophila ruficeps sororia. Evidently the sparse oak woodland in the Sierra Cacachilas is insufficient for the montane forest birds endemic to the Cape Region. However, we suggest that Bursera microphylla is a keystone species, and if Vireo vicinior is its primary seed disperser, the vireo's continuing decline may alter the ecosystem broadly.}

\normaltalk{Demography of sooty fox sparrows following a shift from a migratory to resident life history}{Hannah E Visty\\Scott Wilson\\Ryan Germain\\Jessica Krippel\\Peter Arcese\\N/A }{Identifying causes and consequences of variation in species life history has the potential to improve predictions about how climate and land use change may affect the demography and distribution of species in future. Sooty fox sparrows (Passerella unalaschcensis J.F. Gmelin, 1789; or commonly grouped within Passerella iliaca B. Merrem, 1786) were migrants that rarely bred in the Georgia Basin of British Columbia prior to {raise.17exhbox{\$scriptstylemathtt{sim}\$}}1950 but have since established resident populations. Data on 270 color-banded birds and 54 nests on Mandarte Is., BC, allowed us to estimate demographic vital rates and population growth in one recently established population. Annual fecundity (F), estimated as the product of the number of broods initiated (1.5 ± 0.01; mean ± sd), clutch size (2.82 ± 0.44), and probability of survival to fledging (0.68 ± 0.02), exceeded values reported for migrants, supporting the hypothesis that residents invest more in reproduction on average than migrants within species. Estimating juvenile and adult overwinter survival (Sj = 0.32 ± 0.06, and Sa = 0.69 ± 0.05) next allowed us to simulate an expected distribution of population growth rates as: λexp = Sa + (Sj × F), given parameter error. Our estimate of λexp (1.61 ± 0.57) implies expeditious population growth, consistent with the species' recent colonization of the region.}

\normaltalk{Corvid Response to Forest Thinning in the Willamette National Forest: Implications for the Conservation of the Marbled Murrelet}{Lorraine K Waianuhea\\Joan C Hagar}{The marbled murrelet (Brachyramphus marmoratus) is federally listed as threatened under the Endangered Species Act in Washington, Oregon, and California. The loss and fragmentation of historical old-growth forest nesting habitat over the last two centuries has been the greatest threat to the marbled murrelet. One consequence of forest fragmentation is an increase in nest predation rates. Corvid predators are the leading cause of marbled murrelet nest failures and the second greatest threat to murrelets after loss of habitat. For this project, we analyzed bird survey data that were collected between 1992 and 2007 as a part of the Young Stand Thinning and Diversity Study (YSTDS) which investigated the effects of the forest management practice of thinning on songbird communities. Our research questions focused on the abundance of corvids in forest stands of varying levels of thinning intensity: (1) Was there a difference in the number of corvids observed before and after thinning treatments? and (2) Did the trends in corvid numbers for each treatment change over time? Preliminary results indicate that there were statistically significant differences in the number of corvids observed before and after thinning treatments and that the response of corvids over time varied by treatment. These findings add to recent studies which suggest that forest harvest may increase the abundance and activity of corvids. Overall, the results of this analysis will contribute to the knowledge of corvid response to forest thinning in the Pacific Northwest, and may inform decisions related to murrelet conservation and forest management.}

\normaltalk{Newfoundland Gray-cheeked Thrush migration and wintering - from oceanic island to habitat island}{Ian G Warkentin\\Darroch M Whitaker}{Gray-cheeked Thrushes (Catharus minimus) fitted with GPS tags during summer 2016 were subsequently recaptured during summer 2017 at a breeding site in western Newfoundland occupied by a distinct subspecies (C. m. minimus) which is imperiled. Although damaged, the four tags retrieved provided a total of 15 locations. These locations indicated similar migratory routes along the eastern seaboard of the U.S. but subsequent divergence to either fly directly across the Caribbean to South America or travel via the Caribbean islands and east coast of Central America. Final winter locations for the three birds with readings in South America suggest a high degree of migratory connectivity for the Newfoundland population. All three were located in the vicinity of the Sierra Nevada de Santa Marta, an isolated mountain range that runs from the Caribbean coast of Colombia southwards and from which several historical specimens in museum collections have been identified as being C. m. minimus. These highland areas are surrounded by lands which have been extensively converted from forest to agricultural cover types, while forests of the highlands themselves are also being converted to shade coffee which appears to be low quality habitat for Gray-cheeked Thrush. Routes of travel and wintering will be compared to that known for other breeding populations and the potential impact of habitat degradation in wintering areas will be discussed in the context of dramatically reduced Newfoundland Gray-cheeked Thrush populations.}

\normaltalk{Effects of artificial light at night on carotenoid distribution patterns in king quail}{Emily A Webb\\Pierce Hutton\\Kevin J McGraw}{With the increased prevalence of urbanized landscapes, there is also an increase in spread and intensity of artificial light at night (ALAN). Currently, there are very few experiments that manipulate ALAN as a separate stimulus from other urban-associated stimuli (e.g. anthropogenic noise, heat islands, etc.). Carotenoids in animals, since they cannot be produced de novo, must be consumed and are usually in limited dietary and physiological supply. The ability to absorb and distribute carotenoids in tissues is known to be sensitive to changes in individual condition and the environment. Because ALAN may alter stress, nutrition, and other physiological variables (e.g. production of reactive oxygen species), we hypothesized that experimental exposure to ALAN would alter carotenoid accumulation in the body and distribution among tissues. To explore this, we raised king quail (Excalfactoria chinensis) from hatch under two different light conditions. Both were on a 18h:6h light-dark cycle, but the experimental group received exposure to weak blue LED night lights from weeks 2-8 of development. Throughout the study, plasma levels of carotenoids were monitored. At the end of the study, tissues including eyes, liver, adipose, spleen, gonads, and legs were collected for carotenoid analyses. We found that plasma carotenoids were not significantly different in quail exposed to ALAN than in control quail. Further analyses will reveal if there are any differences in carotenoid distribution patterns between ALAN-exposed and control birds.}

\normaltalk{Influence Of Environmental Factors On Vital Rates Of Rufous Hummingbirds Breeding In British Columbia}{Susan M Wethington\\Kira A Monroe\\Patrick Jantz\\Alison J Moran}{The hummingbird species of highest conservation concern in the USA and Canada is the Rufous Hummingbird (Selasphorus rufus). Based upon data from the Breeding Bird Survey (BBS) since the mid-1960s, Rufous Hummingbird has an estimated 63\% population loss and is considered a common species in steep decline by both Partners In Flight and National Audubon Society. Understanding the drivers underlying these steep declines requires an understanding of how vital rates such as survivorship are affected by environmental conditions and a species' life cycle. To investigate factors influencing survivorship, we combined capture/mark/recapture (CMR) data from seven breeding sites in British Columbia with remote-sensing and climate data and used Cormack-Jolly-Seber (CJS) open population models with predictions that migration strategy, sex, and environmental conditions such as precipitation, temperature, and vegetation productivity will affect survivorship in Rufous Hummingbirds. Resulting survivorship estimates suggest populations are stable and that male survivorship has consistently increased during the time of the study (2002-2017). Decreasing minimum temperatures and increasing precipitation are the environmental factors having the greatest negative effect on Rufous survivorship. These results suggest that the BBS data may not represent trends in Rufous populations and these populations may benefit from predicted climate changes.}

\normaltalk{The paradoxical Giant Hummingbird: comparison of Andean and coastal subspecies with respect to blood, migration, and genes}{Jessie L Williamson\\S M Bauernfeind\\C R Gadek\\N Ricote-Martinez\\F Bozinovic\\Christopher C Witt}{The Giant Hummingbird (Patagona gigas) is twice as large as the second largest hummingbird species and has long been considered paradoxical with respect to flight biomechanics. It is also an extreme outlier in other respects. For example, it is the only hummingbird species that breeds above 4000 m elevation and also along the beaches of the Pacific Ocean. The high Andean populations of Giant Hummingbird (P. g. peruviana) that we have studied previously have a beta-hemoglobin genotype (serine at beta-hemoglobin A positions 13 and 83) that is characterized by high O2-affinity and is only shared with four unrelated hummingbird taxa that are also restricted to extreme high altitudes. Here we report that lowland-breeding populations of Giant Hummingbird (P. g. gigas) are genetically highly similar to their high-elevation counterparts; they even share the same beta-hemoglobin genotype, a unique characteristic among lowland hummingbirds. We found that hemoglobin concentration is lower and red blood cell volume is higher in the lowland P. g. gigas compared to their high Andean relatives. Complicating this comparison is the possibility that coastal P. g. gigas may be a seasonal elevational migrant, but neither the geographic range nor elevation of non-breeding P. g. gigas are known at present. We describe our efforts to describe its migratory behavior using geolocators.}

\normaltalk{Response of bird communities to cattle grazing and plant diversity in Conservation Reserve Program grasslands}{Benjamin S Wilson\\William E Jensen}{Grassland bird populations have been in decline, partially due to the loss of contiguous grassland habitat to row-crop agriculture. However, the Conservation Reserve Program (CRP) is helping to restore grassland habitat in the United States. Conservation practices (CP) for CRP grasslands include varying levels of plant diversity in seed mixes (e.g., higher in CP25 vs. CP2); however, there are disincentives for cattle grazing in CRP, which limits an important management option for improving habitat and avian diversity in these grasslands. We hypothesize that increased plant diversity and conservative grazing will promote higher species diversity of grassland birds in CRP. We are testing this hypothesis by doing line-transect surveys of birds on 108 CRP fields in Kansas that were treated in a factorial design: fields were either CP2 or CP25 plantings and were grazed or ungrazed by cattle in 2017. The preliminary results of our 3-year study show that species diversity of birds in CRP were not substantially affected by CP (CP2 vs. CP25) or grazing. Ordination analyses (NMDS) also showed that bird communities are indistinguishable between CP2 and CP25 fields and between grazed and ungrazed fields. Analyses from the next two years of data collection might reveal lag effects of grazing in 2017 on grassland bird communities.}

\normaltalk{Analysis of an Avian Disease Network in the Greater Yellowstone Ecosystem: Opportunities for Undergraduate Students}{Chloe D Winkler\\Kayla M Harakal\\Eric C Atkinson}{From an ongoing project, we are testing avian species in the Greater Yellowstone Ecosystem for malaria, West Nile Virus (WNV) and irides somatic variability. This study started in 2013 and over this time frame we sampled 471 birds of 33 species, covering several altitudes at six sites. In the summer of 2017 alone, we caught over 200 birds within the Greater Yellowstone Ecosystem. Each individual researcher has chosen an area of research: disease comparison between House Sparrows (Passer domesticus) and House Finches (Haemorhous mexicanus), and color variability of the Common Grackle (Quiscalus quiscula) iris. In the disease comparison of malaria and WNV loads, their sample size consists of 170 different House Sparrows and House Finches. Screening for malaria provided evidence that 18 HOFI and 14 HOSP were suspected to have malaria. We have also identified both conjunctivitis and Avipoxvirus in a small proportion of sampled House Finches. Recently, we initiated metagenomic analysis of the fecal microbiome from 94 different samples employing HiSeq protocols. It is our hope that we can further our research in the different diseases present in avian species. We hope to test our hypotheses this upcoming summer on the abnormalities found within the female Common Grackle iris as well as the somatic variability. Through this study we could gain further knowledge of where this abnormality could have originated. Our poster provides an overview of our continued research throughout the upcoming summer to further our knowledge on avian diseases present in the Greater Yellowstone Ecosystem.}

\normaltalk{Swainson's Hawk (Buteo swainsoni) hunting behavior at an urban population of Free-tailed Bats (Tadarida brasiliensis)}{Katherine M Winston\\Daniel M Brooks\\Jeremy R Winston}{Large colonies of bats that emerge from roosts en masse can provide unique opportunities and challenges for predators. Several species of raptors have been documented preying on bats. However, sustained predation by raptors on large bat colonies and the factors contributing to predation success or failure have rarely been studied. We investigated the sustained predation of free-tailed bats (Tadarida brasiliensis) by Swainson's hawks (Buteo swainsoni) in an urban setting and sought to determine factors that correlate with hawk predation attempts and successes. Additionally, we documented the hunting and post-catching techniques utilized by hawks. We collected observational data on Swainson's hawks hunting and feeding on free-tailed bats in urban Houston, Texas during the summers of 2014 and 2015. Hawk hunting attempts significantly correlated with date (cumulative experience) during 2015 (r = 0.83, P = 0.04, N = 6) but not 2014, as well as temperature (r = 0.54, P = 0.02, N = 14) and relative darkness during hawk arrival (r = 0.47, P = 0.05, N = 14). We observed aerial hunting behavior in three general patterns: diving (recorded during 93\% of all sampling sessions, n = 13), in-flight pursuit (71\%, n = 10) and circling (50\%, n = 7). The hawks consumed bats both in-flight (43\%, n = 6) and while perched in a tree after catching (50\%, n = 7). Our work might help to explain the mechanisms of novel prey selection by hawks in an unusual habitat.}

