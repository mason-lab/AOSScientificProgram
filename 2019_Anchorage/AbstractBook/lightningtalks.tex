\normaltalk{\capitalisewords{Connecting the Dots: Identifying Melanistic Pathways in a Polymorphic Raptor, Buteo jamaicensis harlani}}{\underline{Jacqueline L Barry}}{The Red-tailed Hawk (Buteo jamaicensis) is one of the most widespread raptors, breeding throughout North and Central America and exhibiting extensive phenotypic variation across a range of geographic landscapes. One subspecies in particular, Harlan's Hawk (B.j. harlani), displays the most extreme plumage variation, despite documented interbreeding with other subspecies. Harlan's Hawk occurs in light, intermediate, and dark morphs, with the dark morph is far more common, and has a multi-colored, variably marked tail, a trait unique to Harlan's Hawk alone. Genetic analyses have attempted to resolve the evolutionary relationships of Red-tailed Hawk subspecies, including Harlan's Hawk, but the genetic mechanism(s) driving melanistic plumage in this bird remains unknown. Whereas the melanocortin-1 receptor (MC1R) has been associated with plumage variation in other birds, no such correlation has been observed in Harlan's Hawk. Here, I test the associations of several candidate genes known to regulate melanin with the high phenotypic variation observed in Red-tailed Hawk plumage. MC1R, CORIN, and TYRP1 were not associated with melanistic variation in B.j. harlani, B.j. calurus and B.j. borealis. ASIP, POMC, and SLC24A5 are currently under study.}

\normaltalk{\capitalisewords{Long-term data show shifts in urban bird communities by land use type in Arizona}}{Daniel C Allen, \underline{Heather L Bateman}, Paige S Warren}{Long-term studies are needed to assess the potential for species persistence in the face of expanding urbanization. We investigated trends in bird data from an urban LTER (long-term ecology research) project to explore how communities of urban birds changed over time at the regional, landscape, and local spatial scales. At 43 sites during winter and spring from 2001-2002 and 2004-2016 birds were surveyed using 10-minute point counts. Sites occurred in five land use categories (agriculture, desert, desert-riparian, urban, urban-riparian). We examined changes in abundance related to functional group or guild. We used ordinations to assess differences in communities and linear regressions to investigate guild abundance change over time. Winter bird diversity (gamma) was consistent over time; however, spring diversity decreased. Trends of bird abundances showed interesting patterns of winners and losers by land use and traits. For example, winter nonbreeding migratory species decreased in desert, urban, and urban riparian areas. Summer migratory breeders were stable and increased in urban areas. Birds feeding on plants generally increased in urban and urban-riparian areas; whereas, predators decreased in desert-riparian and urban-riparian areas. Nest parasites (cowbirds) decreased in deserts and increased in urban-riparian areas. Overall, desert-riparian areas seems to have the fewest changes in bird communities, and urban-riparian areas and urban areas experienced the most shifts (seeing both increases and decreases in specific bird guilds). Bird species needing water and tree habitats appear to be losing ground in desert and urban areas and increasing in agriculture areas.}

\normaltalk{\capitalisewords{Diversity of grassland bird communities as a function of grazer species on the Northern Great Plains}}{\underline{Andy J Boyce}, Hila Shamoon, William Mcshea}{Large ungulates can strongly influence grassland bird communities through their impact on structure and composition of grassland vegetation. Understanding the biodiversity effects of grazer type, as well as removal of grazers from grassland landscapes has important consequences for landscape scale conservation in globally-threatened temperate grassland ecosystems. North American grassland bird communities co-evolved with Plains Bison (Bison bison bison), and there is evidence that bison grazing behavior differs significantly from domestic cattle and can increase vegetative heterogeneity in grassland systems. If true, we predict increased diversity of grassland birds in areas grazed by bison, particularly in large pastures. To test this hypothesis, we sampled grassland bird communities on grasslands on and surrounding the American Prairie Reserve in NE Montana, on lands grazed by cattle, bison, or no large bovids. We sampled 49 80-hectare plots, within larger fenced pastures using 10-minute point counts. We conducted 6 counts within each 80-hectare cell to assess both overall diversity and heterogeneity in bird communities within a cell. Based on preliminary analysis of species richness and functional diversity of grassland-obligate songbirds, we found that species richness and functional diversity were similar between areas grazed by Bison and cattle, and lower in areas without bovine grazers. Analyses of this dataset are ongoing and will include data from a second field season.}

\normaltalk{\capitalisewords{Nesting sites and declining snowpack co-limit habitat of North Americas highest-elevation breeding bird}}{\underline{Tim M Brown}, Karim Hanna, Erika Zavaleta}{Diminishing snowpack in mountains throughout the world is expected to negatively impact many alpine organisms. In western North America snow pack is predicted to decline as much as 80\% by 2100. The Sierra Nevada Grey-crowned Rosy-finch (Leucosticte tephrocotis dawsoni) breeds at the highest elevations within its narrow geographic range of the Sierra Nevada and White Mountains.  The Rosy-finch is an omnivore, foraging on aeolian invertebrates and plant seeds found on and around snowfields in the alpine ecosystem. Based on predicted habitat suitability we chose 15 sites above tree line throughout the Sierra Nevada and White Mountain ranges of California.  We conducted point count surveys at 250m intervals along 2km transects to assess Rosy-finch occupancy and collected data on environmental variables including invertebrate abundance, distance to snowpack and cliffs along each transect.  Within 200m of cliffs, we observed an increased probability ($>$30\% detection) and 100m of snow (~40\% detection) of detecting a Rosy-finch. We found that the probability of detecting Rosy Finches decreased with distance away from both cliffs (logistic regression, p=$<$0.0001) and snow (p =$<$0.0001), indicating that the Rosy-finch could depend on sites where snowpack, observed to be a primary foraging habitat, and cliffs, used for nest sites, co-occur at high elevations.  These findings inform development of a framework to understand Rosy-finch response to rapid climate change.}

\normaltalk{\capitalisewords{Cross --Species Responsiveness to Auditory Stimuli:  Do Birds and Humans React to Similar Sounds in Similar Ways?}}{\underline{Janice R Crook-Hill}}{Research comparing acoustic communication in distantly related vertebrates is limited. Our study is motivated by the question, "Do avian vocalizations and human music have common evolutionary origins?" If so, then we suggest there may be similar elements in avian vocalizations and human music which elicit similar responses across species. To explore this, we used recorded segments of alarm calls and contact calls of wild Tufted Titmice and "alarming" and "calming" human music. We played all four segments to both wild birds and human volunteers and compared responses. Avian behavioral responses were observed through playback at feeders. Human participants' reactions were assessed through a survey given after hearing the recordings. Our data indicate similar responses from birds and humans in that alarm calls and "alarming music" elicited avoidance behavior in birds and were interpreted by humans as frightening, whereas contact calls and "calming" music elicited no change of behavior in birds and were interpreted as calm by humans. Next, we plan to isolate similar acoustic parameters within the "alarming" and "calming" segments of each species' recordings, playback the isolated segments, and again document responses of both birds and humans, in an attempt to find commonalities in music and avian vocalizations that elicit similar responses across species.}

\normaltalk{\capitalisewords{Fast Relief: Immediate Benefits From Reducing Invasive,  Non-native Overpopulations Of Canada Goose Via Addling Their Eggs}}{\underline{Daniel A Edelstein}}{Throughout many metropolitan portions of the USA, Canada Goose (Branta canadensis) populations have dramatically risen in recent years. Negative impacts resulting from the presence of a western USA Canada Goose subspecies — B. c. moffitti — have included water quality degradation from large deposits of goose feces; the potential spread of bird flu disease to other birds and to people; and reduced biological diversity as the large, naturally-aggressive geese outcompete native bird species for food and nesting spaces. Costly treatment of goose-polluted water is also necessary so that it can be used again for recycled water on landscapes, car washes and other non-potable uses.  In response to more than 250 geese in 2010 residing at a sanitary district's reclamation ponds adjacent to the San Francisco Bay, the Las Gallinas Valley Sanitary District's initiated an addling program as part of an Integrated Goose Management Plan. One year after addling goose eggs so they ceased hatching, geese numbers dropped 64 percent. By 2018, merely seven newborn geese hatched at the site, compared to 74 before the now-ongoing, annual addling effort began.  Consequently, the District's believes its addling program is 1) a low-cost environmentally elegant way to maintain the quality of its treated wastewater for later treatment and reuse; 2) provides several native wildlife species suitable habitat for foraging and raising young; 3) preserves native plant species from foraging geese; and 4) offers a safe place for visitors to hike without experiencing altercations with periodically aggressive geese that previously crowded portions of popular trails.   }

\normaltalk{\capitalisewords{Colorful and Calm or Angry and Ornamented: Correlated Nest Defense and Uv and Melanin Ornaments in Eastern Bluebirds}}{\underline{Lauren M Gillespie}, Lynn Siefferman}{In times of changing climate and shifting environmental conditions, understanding individual variability and flexibility of relationships between physiology and certain behaviors crucial to survival and reproduction are more important than ever. In bluebirds, ultra-violet (UV) blue plumage brightness and chroma are sexually selected ornamental traits indicating male competitive ability among conspecifics. Defense against predators may be just as important, and research here investigates ornament signaling in relation to predator defense behavior. Here, we present data from a Mississippi population of bluebirds inhabiting two sites where mated pairs were exposed to simulated predator intrusions in 2013 and 2014 using an American crow decoy during the nestling phase, and aggressive and vigilance behaviors were recorded. Plumage samples were collected and analyzed for UV reflectance and melanin ornament intensity. We found darker melanic colored and more UV ornamented males are less aggressive at both sites, and relationships are stronger at one site. More aggressive females at another site are more UV ornamented and more melanin pigmented, while less aggressive females at the other site are more UV ornamented and less melanin pigmented. Plumage does not correlate with habitat features of either site while defense behaviors do. Results here prompt discussion of how differing environmental or sexual selection pressures may act more strongly on females presenting unique reproductive issues regarding functional hormonal mechanisms in having or maintaining flexibility in sexually selected plumage traits, especially when correlating with aggression.}

\normaltalk{\capitalisewords{Phylogeny and evolution of the Gallopheasants}}{\underline{Peter A Hosner}, Hannah L Owens, Sarah Hyde, Edward L Braun, Rebecca T Kimball}{The gallopheasants comprise a clade of 23 species, including some of the most elaborately-plumaged and highly ornamented birds in the world. They also occupy a remarkable breath of environments and habitats, ranging from lowland rainforests to high grasslands and steppes of the Tibetan plateau. Here, we provide the first well-resolved species phylogeny of this charismatic group, inferred from ultraconserved element sequences. Unlike previous studies, which supported a rapid initial burst of diversification and unresolvable relationships, we identified a steady accrual of lineages through time. Reconstructions of morphological traits demonstrated strong phylogenetic signal, not only for highly ornamented males, but more cryptic females as well. Degree of ornamentation, plumage elaboration, and sexual dimorphism may be related to breeding system, with more monogamous species tending to exhibit reduced sexual dimorphism. However, breeding biology requires clarification for several poorly known species. Environmental niche also exhibited strong phylogenetic signal, yet with no evidence of niche filling.}

\normaltalk{\capitalisewords{Social polygyny and genetic confirmation using minION DNA sequencing in the Eastern Bluebird}}{Katherine M King, \underline{Memuna Z Khan}}{Orian's polygyny threshold model states that polygyny occurs when a female enjoys greater reproductive success with a high quality male in a socially polygynous arrangement than a low quality male in monogamy.  Assumptions of this model state that polygynous females bear the cost of sharing resources on a territory.  Although Eastern Bluebirds (Sialia sialis) commonly exhibit genetic polygyny in the form of extra pair fertilizations, reports of social polygyny are rare.  We report a serendipitous observation of social polygyny by a known age male on separate territories.  This male was observed tending a nest with 14 day old nestlings and tending a nest box 200m away that housed a different female incubating eggs during the same time period.  After the older nestlings had fledged, the male continued to tend young at the second nest. The context of this these observations suggest that the polygyny threshold model is not an adequate explanation.  Furthermore, behavioral observations were later genetically tested using the minION Nanopore gene sequencing tool on blood samples from the male breeder, and offspring from each his two nests.}

\normaltalk{\capitalisewords{Quantifying snag resources for cavity-nesting birds and Red-cockaded Woodpecker cavity usurpation in old-growth longleaf pine}}{\underline{Heather E Levy}, Robert J Cooper, James A Cox}{Cavity-nesting bird communities (CNBs) interact through a complex web based on the creation of and competition for cavities as nest and roost resources. CNB communities have been studied in restored longleaf pine forests, but few studies have occurred in old-growth stands, which occupy only 0.014\% of their former range. The Red Hills region of the southeast contains some of the last remaining old-growth longleaf pine forests and supports one of the largest remaining populations of the endangered Red-cockaded Woodpecker (RCW). RCWs excavate exclusively in living pines, and their cavities are often usurped by other CNBs, but mechanisms that drive this interaction remain understudied in old-growth systems. We quantified snag abundances, CNB abundances, and nest productivity across 8 old-growth sites in the Red Hills. Sites were expected to contain varying snag abundances due to differences in management. Our goals were to (1) examine the relationship between CNB abundance and snag abundance and (2) to compare the proportion of RCW cavities usurped in relation to snag abundances to determine whether snag limitation affects frequency of interspecific competition. We found that CNB abundance is positively correlated with snag abundance, suggesting that snag retention/creation promotes high abundances of cavity-nesting birds. We also found that RCW cavities were usurped despite local snag abundances, suggesting that CNBs may prefer RCW cavities despite abundant snag resources due to their sturdiness and persistence on the landscape, or that there is a surplus of natural RCW cavities in old-growth longleaf pine forests that results in reduced interspecific competition.}

\normaltalk{\capitalisewords{Hybridization between Red-crowned and Lilac-crowned parrots in Southern California}}{\underline{James M Maley}, Margaret E Schedl, Brooke Durham, Whitney LE Tsai, Ryan S Terrill, Kimball L Garrett, John E McCormack}{Southern California is home to a large population of naturalized parrots, all former pets or descendants of pets. There are at least six species of Amazona parrots that are known or suspected of breeding in Los Angeles County. The most abundant species is Red-crowned Parrot, numbering in the thousands, followed by Lilac-crowned Parrots. These two close relatives are allopatric in their native ranges, but breed in the same areas of Los Angeles. Lilac-crowned Parrots are often found at higher elevations in their native range, and have been found breeding in the mountains above Los Angeles. Phenotypic hybrids have been suspected to occur but we wanted to determine if the two species are hybridizing based on genomic data. To determine the level of genetic admixture we sequenced ultraconserved elements for 4 Lilac-crowned and 4 Red-crowned parrots collected between 1934 and 1955 from their native ranges. We also sequenced UCEs for 22 Amazona parrots salvaged in Los Angeles and San Diego. Based on 25,647 single nucleotide polymorphisms we found evidence of several hybrids between the two species. There appears to be limited introgression from Lilac-crowned Parrots into Red-crowned Parrots, the extent of which requires further study to determine. It remains to be seen if species boundaries will remain, or if the parrot flock of Southern California will become a huge hybrid swarm.}

\normaltalk{\capitalisewords{Mexican hummingbird nesting}}{\underline{Laura E Nu\~{n}ez-Rosas}, Mar\'{i}a dC Arizmendi}{Nesting is a sensitive period that presents a high mortality rate because of many ecological factors. Hummingbirds are altricial birds which need parental care. Female carries all the nesting process. Although, Trochilidae family is a diverse group in terms of species number (330), the knowledge about their reproductive aspects is scarce. For this study, we made a review based on published data and personal observations. We collected data from nests and eggs measures, incubation and fledging periods, lichen use, site and vegetation where nest was located. For statistical analysis, we used PGLS, which incorporates a phylogenetic correction. We also used simple linear regressions. We found that nest shape is a cup, built at 41.6 ± 12.39 mm of high, 40.21 ± 13.94 mm of diameter and 18.59 ± 5.06 mm of deep. Eggs were 13.19 ± 1.04 mm of length and 8.65 ± 0.67 mm of width. We couldn´t find evolutionary relationship in lichen use, however, it shows a tendency of recently clades using more lichens. Additionally, both incubation period (16.17 ± 1.16 day) and fledging period (22.4 ± 2.25 day) were not related with female size or clade. Finally, most of the nest has been found in pine forest (50\%). Available information for reproductive aspects in hummingbird is limited, and much work is needed to fully understand the process. Although some data for exist for some species, most of these are not distributed in Mexico.}

\normaltalk{\capitalisewords{Is Kluane Lake in the Yukon Territory a superhighway for Alaska's migrating Golden Eagles?}}{\underline{Neil A Paprocki}, Travis L Booms, Robert J Spaul, Jesse L Watson, Dave Oleyar}{The identification of critical migratory corridors is one of many key insights to come out of the recent explosion of data generated from remotely tracked migratory birds. Remote-tracking data from Golden Eagles (Aquila chrysaetos) originating in northwestern North America suggests the Kluane Lake region in southwestern Yukon Territory, Canada is a major migratory flyway as birds fly south to wintering grounds in Canada and the contiguous United States. Somewhat surprisingly, relatively little standardized raptor migration count data exist for this region. The primary objectives of this study were to conduct fall raptor migration counts in the Kluane Lake region to 1) inform population estimates of Golden Eagles summering in Alaska and northwest Canada, and 2) identify the best count location for future migration monitoring. We conducted raptor migration counts from 24 September 2017 – 22 October 2017 at twelve different locations based on a combination of remotely tracked eagle flight-paths, local knowledge of the area, and on-the-ground observations by migration observers. We counted a total of 1,842 migrating raptors over 26 days of effort. This includes a total of 1,397 Golden Eagles with a daily high count of 281 individuals on 20 October. Our results add to the mounting weight of evidence suggesting the population size of Golden Eagles summering in northwestern North America is considerably higher than current estimates suggest. We successfully identified a count location for future monitoring, setting the stage for a full-season of raptor migration counts in fall of 2019.}

\normaltalk{\capitalisewords{Differential effects of landscape composition and patch size on avian habitat use of restored fields in agriculturally fragmented landscapes}}{\underline{Bryan M Reiley}, Thomas J Benson}{Existing private-land conservation programs play an important role in restoring wildlife habitat in agriculturally fragmented landscapes. These conservation programs are generally implemented for a variety of environmental purposes in addition to providing wildlife habitat and as a result private-land programs are often implemented without identifying target species. Conservation benefits of private-land conservation programs could be improved if the needs of declining species were used to guide future enrollments and management of habitat created through these programs. To improve implementation and management of these programs, we examined what within-field, patch, and landscape features affected habitat selection by 6 conservation priority bird species. Specifically, we were interested in understanding how species responded to patch size, surrounding privateland conservation habitat, and the degree to which the amount of surrounding cropland moderated the response to the amount of conservation habitat. To do this, we conducted bird surveys at 172 private-land fields during 2012–2015. While most species were positively associated with patch size, only the Dickcissel (Spiza americana) was strongly positively associated with this variable. Dickcissel (+) and the Field Sparrow (Spizella pusilla) (-) had strong relationships with the amount of surrounding private-land conservation habitat, although we found no evidence that benefits were greater in landscapes with higher amounts of row-crop agriculture. Contrary to previous studies, our results suggest that for most of our focal species, targeting fields in areas with existing private-land conservation habitat and creating larger patches does not improve the benefit of these programs, although maintaining and placing them in grass-dominated landscapes will.}

\normaltalk{\capitalisewords{Genetic, morphological and color variation in Piranga bidentata (Aves:Cardinalidae)}}{\underline{Sahid M Robles Bello}, Blanca E Hernandez Ba\~{n}os}{Describing and explaining the origins of biodiversity are two of the chief goals of Systematics. As the biodiversity crisis worsens, establishing species limits accurately can help us make better-informed conservation decisions.   In this ongoing study we aim to establish species limits and explain the causes of color differences in Piranga bidentata, a polytypic songbird that inhabits both Mexican Sierras as well as Central America up to west Panama, with 4 described subspecies. While females through its entire distribution are yellow, the males on West Mexico are bright orange, and the males on East Mexico and Central America are blood red. Using genetic, morphometric, colorimetric and environmental data from museum specimens, we quantified variation and divergence between subspecies. Our prediction is that if these differences in color are caused primarily by random variation accumulated in geographic isolation, phylogenetic relations should explain most of the color variation. If these differences in color better are explained by climatic variables, this is better interpreted as the result ecological or adaptive processes.  Our preliminary results show that P. bidentata populations sort into at least two well-supported lineages based on mitochondrial DNA data. This corresponds to a West-East split in Mexico. We found that there's significant color and morphometric divergence between subspecies, and that this might be linked to environmental differences.}

\normaltalk{\capitalisewords{Patterns of change within the avian community of Acadia National Park: long-term changes and a case study in human development}}{\underline{Katharine J Ruskin}, Alyson M East, Marie I Ring}{Protected lands such as national parks are intended to conserve natural resources such as biodiversity.  Despite their protected status, however, wildlife in national parks is changing due to a number of drivers, including climate change, addition of invasive species, and land use change.  Quantifying rates of change of biodiversity on protected lands is critical for conserving this resource.  We examined whether the avian community of Acadia National Park (ANP) has changed in the past fifty years.  Using Christmas bird count data collected from 1960-2017, we found that the avian community of ANP is changing; each year, the avian community becomes more dissimilar to the baseline community of 1960.  In addition, we used a case study to quantify the effects of a single driver of environmental change in this system.  In 2015, ANP opened the Schoodic Woods Campground which hosts 97 campsites, accompanied by 7 miles of new gravel bicycle paths, 5 miles of hiking trails, a visitor center, and parking for 100 cars.  During the breeding season of 2017, we conducted avian point count surveys on transects beginning from the edge of developed areas and radiating into the forest.  Across 184 point count surveys, we observed that species richness at each survey location was best predicted by distance from the campground.  Species richness was highest in and near the campground, and declined moving away from it.  The avian community observed at survey locations near the campground clustered together and included several species classified as synanthropic.  Taken together, our results suggest that ANP is experiencing biotic homogenization and illustrate that human development is contributing to these changes in the avian community.}

\normaltalk{\capitalisewords{Where did Cuban bobwhites come from?}}{\underline{Jessie F Salter}, Robb T Brumfield, Brant C Faircloth}{The evolutionary history of Cuban bobwhites (Colinus virginianus cubanensis) has been a source of controversy among ornithologists for nearly two-hundred years. Since they were first noted on the island in 1839, natural historians have debated whether Cuban bobwhites are endemic to the island or were more recently introduced to Cuba from the southeastern United States, as many Cuban naturalists have recounted. Adding to the confusion, bobwhites from the U.S. have been periodically introduced to Cuba since the late nineteenth century, resulting in birds of intermediate phenotypes. Genetic analyses of a few individuals have also produced conflicting results: a recent study of mtDNA showed no differentiation in haplotypes between Cuban and southeastern U.S. populations of bobwhite, while our preliminary data using thousands of genome-wide loci suggest that Cuban bobwhites are most closely related to bobwhites in southern Mexico. To better address this question, we collected genomic and phenotypic data from fifty historical bobwhite specimens collected from Cuba, Florida, and other Caribbean islands that included specimens from Cuban populations where there have been no documented introductions, purported hybrids between Cuban and Florida bobwhites, and specimens from known introduced populations, and we analyzed these data using a variety of approaches to test competing hypotheses regarding the evolutionary history of this enigmatic Cuban bobwhite population.}

\normaltalk{\capitalisewords{Bill Divergence and Speciation in the New World Jays (Corvidae)}}{\underline{Benjamin F Scott}, John McCormack, James Maley, Whitney T Nakashima}{Evolutionary changes in bill morphology have been noted as powerful drivers of diversification. While previous work has focused on how bill trait specialization has influenced speciation, few studies have analyzed how bill traits can impact divergence in a widely dispersed, omnivorous clade. The New World Jay family (Corvidae), occupy a variety of habitats across the montane backbone of the Americas. Bill morphology has remained similar across the family despite significant differences in body size and habitat preference. In this study, I investigate the allometric relationship between body size and bill size, and test how the rate of morphological trait evolution has influenced the speciation rate. Standard linear bill measurements were collected from museum specimens and reduced into a phylogenetic PCA to reconstruct ancestral bill states. I found relatively little change in overall bill size, but a surprising amount of difference in bill shape between genera. To quantify rates of trait evolution and diversification, I calculated species-specific rates of trait evolution using a phylogenetic ridge regression and compared them to tip-specific diversification rates. I show the rate of trait evolution is not associated with the rate of speciation, and investigate the other factors which drive bill shape in Corvidae.}

\normaltalk{\capitalisewords{Earlier spring arrival of migrant birds at the Aras Ornithological Research Station, Igdir, Turkey}}{Joshua J Horns, Michael Ford, Kayahan Agirkaya, Emrah Coban, Berkan Demir, Lale Aktay, \underline{Cagan H Sekercioglu}}{As climate change causes earlier spring onset, many species of migratory bird must advance the timing of their arrival in temperate regions to continue to take advantage of seasonal food peaks. However, the biological characteristics associated with migratory plasticity remain poorly understood, as do the potential health consequences for both individual birds as well as breeding populations. In this study, we look at changes in the timing of spring migration over 12 years for 71 species of bird migrating through eastern Turkey, a region largely lacking in long-term ecological studies. We also investigate whether certain species characteristics, including migratory distance, primary diet, primary habitat, and specialization, are associated with changes in migratory phenology. Finally, we look to see if changes in spring arrival are associated with differing levels of fat reserves in individual migrants as well as overall changes in species abundance. Our results suggest that 22 species have significantly advanced the date of their spring arrival, 5 have significantly delayed spring arrival and 44 have shown no significant change. Across all 71 species, birds are arriving on average 1.8 days earlier/decade.}

\normaltalk{\capitalisewords{Individual distinctiveness in vocalizations of a suboscine songbird}}{\underline{Josephine T Tagestad}, Daniel J Albrecht-Mallinger, Corey E Tarwater}{Assessment of organisms' vocal traits can lead to insights regarding behavior, age, health, and identity. Studies of bird song and its functions focus almost exclusively on the learned songs of oscine birds, while the innate songs of suboscine birds have received less attention. Suboscine songbirds are assumed to be less vocally complex than their oscine counterparts, resulting in lack of individual distinctiveness in their vocalizations. Here we test whether vocalizations of the suboscine bird of Ceratopipra mentalis, the red-capped manakin, are individually distinctive. We examined recordings of 20 males at 10 leks across the Isthmus of Panama, extracting cadence and frequency data from whistle-peeps, a contact call males give throughout the breeding season. We used a flexible discriminant function analysis (FDFA) in Program R to test for individual identity coded in vocalizations. Based on this analysis, individual variation is present, with the FDFA assigning identity to the calls with 87\% accuracy. Furthermore, individual identity explains more variation than local dialect. As C. mentalis is an important seed disperser and is predicted to decline with continued changes in climate, the use of vocal recordings to estimate abundance in a rapid, non-invasive way would aid in studying them in a changing world. More generally, we argue that further studies of suboscines are needed to test long-held assumptions about individual distinctiveness.}

\normaltalk{\capitalisewords{Dynamics of remigial molt of captive spectacled (Somateria fischeri) and Steller's eiders (Polysticta stelleri)}}{\underline{Sadie E Ulman}, Tuula Hollmen, Dan Esler}{Remigial molt in waterfowl is an energetically costly and critical time in the annual life cycle. In waterfowl in general, and sea ducks in particular, information about the ecology of this time period is sparse. The Alaska SeaLife Center captive spectacled and Steller's eiders were studied to provide data on the dynamics of this physiologically demanding period. Data on captive birds will provide beneficial information for interpretation of data collected on these federally threatened species of eiders in the wild.  During the 2017 and 2018 remigial molt, we looked at the pre-emergence interval, timing of molt initiation, primary feather growth rate (mm/day), and duration of flightlessness of captive spectacled and Steller's eiders. Body mass and 9th primary feather measurements were taken on spectacled (n=14, in 2018) and Steller's eiders (n=41, in both 2017 and 2018) at 1-2 week intervals from end of July through October. Of those birds, we used our newly developed feather color marking method to uniquely color mark the 9th primary feather of 9 spectacled and 9 Steller's eiders. Data collected using this marking method can be used to determine molt initiation and lag time between drop data and start of growth, which will improve estimates of growth rate and molt duration.}

\normaltalk{\capitalisewords{Phylogenomic of Euphoniinae subfamily (Aves:Fringillidae): a taxonomic and biogeographic review}}{\underline{Alma M Vazquez Lopez}, Blanca E Hernandez Ba\~{n}os}{Euphoniinae is a subdivision of the Fringillidae family; it is formed by the genera Euphonia and Chlorophonia. Chlorophonia genus has five species and with a green pattern coloration. In contrast, Euphonia genus is a large group with different coloration patterns and with 27 species. The relationship between Euphoniinae members is not determinate. A Phylogeny of Fringillidae showed that the blue-crowned Euphonias are closer related with Chlorophonia genus than with the Euphonia species. Based on that the IOC proposed to reorder the subfamily in Chlorophonia, Euphonia and Cyanophonia (for blue-crowned Euphonias) genera. The goals of this work are to obtain a phylogenetic of Euphoniinae and propose a new taxonomic arrangement. We obtained large sampling tissues for Euphoniinae thanks to many Scientific Collections. The DNA was extracted and sent to SNPsaurus Company to obtain a nextRAD genotyping. The genotyping analysis was done with BBMaptools (IT 0.88) and the genotyping calling with Samtools. The partitions and GTR models were obtained with PartitionFinder2. The phylogeny was done by Maximum Likelihood with RaxML. The phylogeny has two principal clades Chlorophonia-Cyanophonias and "true" Euphonias. These clades are divided into them. The first one is divided into a Chlorophonia and Cyanophonias clade. The second clade is divided also for two clades, into these clades is not possible to match a pattern coloration, but personal specimens reviews and colorimetry measurements suggest an apomorphic melanic orange in one of them. This preview results partially agree with IOC taxonomic and suggested a fourth genus into the Euphoniinae.}

\normaltalk{\capitalisewords{Painting the Bunting: The molecular and structural basis for coloration of Painted Bunting (Passerina ciris)}}{Nicholas M Justyn, \underline{Ryan J Weaver}}{Conspicuous feather coloration is fairly widespread in birds and the most common colors are attributed to various combinations of melanins, carotenoids, and spatial arrangements of the feather microstructures. Male Painted Bunting (Passerina ciris) are perhaps the most colorful songbird in North America, displaying vibrant red, orange, pink, yellow, green, and blue feather coloration. Surprisingly, the pigments and or feather structures that give rise to the multitude of colors of Painted Buntings has not been determined. Here, we use digital photographs, spectrophotometry, high-performance liquid chromatography, and scanning electron microscopy to elucidate the mechanisms of feather color production in six body (color) regions: Crown (blue), chest (red), belly (orange), rump (pink), back (yellow), and wing (green). We found that carotenoids and structural components are responsible for the observed coloration of each region. Red ketolated carotenoids were found in blue, red, orange, and pink feathers. Blue coloration comes from light interacting with specifically arranged feather microstructures, but Painted Bunting blue feathers also contained red ketocarotenoids, albeit at low concentration. The shift from red to orange to pink was dependent on the concentration of red ketocarotenoids deposited, but the orange to pink shift occurred by tuning from the underlying blue structural coloration. Yellow and green feathers contained the same canary xanthophyll carotenoids, but yellow feathers had higher concentrations. To our knowledge, this report is the first to elucidate the molecular and structural basis for the diverse array of conspicuous coloration in male Painted Buntings.}

\normaltalk{\capitalisewords{Habitat Use of Hawaiian Short-eared Owls (Asio flammeus sandwichensis) in Hawai'i}}{\underline{Chad J Wilhite}, Javier Cotin, Laura R Luther, Melissa R Price}{Pueo, or Hawaiian Short-eared Owls (Asio flammeus sandwichensis), play an important role in the cultural and ecological landscape of Hawai'i as the only widespread native apex terrestrial predator. Globally Short-eared Owl populations are experiencing declines, likely due to a combination of factors including predation, food availability, disease, and anthropogenic mortality. Despite population declines very little is known about the biology of Pueo or Short-eared Owls in general. In temperate and continental systems Short-eared Owls are typically considered nomadic grassland specialists. In this study we deployed VHF transmitters to track Pueo on the island of O'ahu and identified habitat use and movement patterns in Hawai'i. Pueo tend to exhibit higher site fidelity with less nomadic movements and utilize a broader range of habitat types than continental and temperate populations. Our increased understanding of Pueo biology will also help inform land managers how to limit anthropogenic activities in high quality habitat and during peak breeding seasons, potentially resulting in substantial reductions of human-caused mortality.}

\normaltalk{\capitalisewords{Nocturnal flight calls influence vulnerability to artificial light and building collision}}{\underline{Benjamin M Winger}, Brian Weeks, Andrew Farnsworth, Andrew Jones, Mary Hennen, David Willard}{Understanding interactions between biota and the built environment is increasingly important as human modification of the landscape expands in extent and intensity. For migratory birds, collisions with lighted structures are a major cause of mortality, but the mechanisms behind these collisions are poorly understood. Using 40 years of collision records of passerine birds, we investigated the importance of species' behavioral ecologies in predicting rates of building collisions during nocturnal migration through Chicago, IL and Cleveland, OH, USA. We found that use of nocturnal flight calls is an important determinant of collision risk in nocturnally migrating passerine birds. Species that use flight calls during nocturnal migration collided with buildings more than expected given their local abundance, whereas those that do not use such communication collided much less frequently. Our results suggest that a stronger attraction response to artificial light at night in species that produce flight calls may mediate these differences in collision rates. Nocturnal flight calls likely evolved to facilitate collective decision-making during navigation, but this same social behavior now exacerbates species' vulnerability to a widespread anthropogenic disturbance. Our results also suggest that social behavior during migration may reflect poorly-understood differences in navigational mechanisms across lineages of birds.}

\normaltalk{\capitalisewords{Winter Movement of Brown-capped Rosy-Finches (Leucosticte australis) between Feeder Sites in Colorado, USA}}{Aaron A Yappert, \underline{Erika Zavaleta}, Amy E Seglund, Kathryn Bernier, Gregory J Levandoski, Bruce Ackerman, Susan Werner}{Brown-capped Rosy-Finches (Leucosticte australis) are poorly studied passerines that are nearly endemic to Colorado and are of conservation concern because of limited data on population dynamics and the projected impacts of climate change on alpine breeding sites and winter foraging areas. In winter, Rosy-Finches form large flocks that can number several hundred individuals but local and landscape-level movement by individuals and flocks is poorly understood. We used capture-mark-recapture techniques at six feeder sites across Colorado to quantify movements during three winters (2017 – 2019). 6.5\% of birds recaptured during their original banding winter moved between sites (N = 92). However, 20.7\% of birds recaptured in subsequent winters moved between sites (N = 217), and males tended to move more than females (Pmale=28.1\%, Pfemale=8.8\%, X2=5.75, p=0.057). 88\% of movements between the six sites were short-range ($<$25 km, N = 33); however, three individuals, encompassing both sexes, were recaptured as far as 315 km from their original capture site, which suggests that long-range movement does occur at the level of the individual and involves both males and females. The large proportion of stationary individuals indicates that large-scale flock mixing and movement between regions is uncommon within the same winter. Though long-range individual movements may be rare, they could enable gene flow between distinct breeding regions. Additional work should focus on linking small- and large-scale winter movements amongst age and sex classes with summer breeding site selection to determine the degree to which winter movements influence breeding location and gene flow.}

