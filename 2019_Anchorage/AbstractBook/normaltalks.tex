\normaltalk{\capitalisewords{Applying an Ensemble of Small Models to Predict Breeding Distribution of Virginia's Warbler in South Dakota}}{\underline{Reza Goljani Amirkhiz}, Mark D Dixon, Jeff S Palmer, Daivid L Swanson}{Defining the spatial distribution of populations is a primary step in conservation. However, the lack of sufficient occurrence records for rare populations is a major obstacle in precisely defining their distributions. The northeastern limit of the breeding range of Virginia's Warbler (Oreothlypis virginiae, VW) occurs in the southern Black Hills of South Dakota, where it is a species of conservation concern. Ensemble of Small Models (ESM) has been proposed as a technique to overcome the problem of few occurrence points in species distribution modeling. An ESM averages prediction of bivariate models across all possible combinations of predictor variables to form an ensemble weighted by cross-validated AUC scores. The goal of this study was to compare the performance of the ESM with a suite of commonly used species distribution modeling techniques since an ESM approach has not been tested for birds. We obtained 51 presence and 65 absence points from recent surveys in the southern Black Hills and considered 21 biologically relevant predictors to build our models. An ESM of artificial neural networks performed better than other modeling techniques. ESM allowed us to explore more dimensions of the Virginia's warbler's ecological niche. The mean importance value for each variable considered across all individual bivariate models showed that terrain ruggedness index, tree canopy height, and shrub density were the most important variables associated with VW occurrence. These results can guide future surveys and inform management actions for this species of conservation concern.}

\normaltalk{\capitalisewords{eBird Status and Trends: Species Distributions and Population Trends with Citizen Science Data}}{\underline{Daniel Fink}, Tom Auer, Alison Johnston, Matt Strimas-Mackey, Marshall Iliff, Orin Robinson, Brian Petersen, Steve Kelling}{The eBird Status and Trend products provide ecological information for over 100 species in North America, describing their ranges, relative abundance, environmental associations, and population trends at high spatial resolution. To generate these products, we use statistical and machine learning analyses combining eBird data with environmental data from NASA. The analyses are used to estimate the occurrence and abundance of species across North America at weekly intervals. These estimates are summarized to produce a suite of visualizations and data products available at https://ebird.org/science/status-and-trends. In this presentation we will provide an overview of the Status and Trend products, which include weekly estimates of population ranges and relative abundances at a resolution of 2.8km x 2.8km across terrestrial North America, estimates of the associations between species and features of their local environment, and season-specific trends in relative abundance. We will also briefly describe some aspects of the modeling framework used to estimate these metrics, and introduce the ebirdst R package to access and analyze the data.}

\normaltalk{\capitalisewords{Geolocators and Stable Isotopes Reveal the Migratory Route and Overwintering Locations of an Arctic-Breeding Passerine}}{Heather R McFarland, Alexis Will, \underline{Abby Powell}}{Critical migratory passerine habitats have historically been identified by sightings of birds during the non-breeding season. These observations may overlook habitats that are remote or difficult to access, and, for arctic-breeding passerines, may provide only a patchy understanding of non-breeding movements. We deployed light-based geolocators and used stable isotope analysis to track the migration and identify the overwintering locations of male Smith's Longspurs breeding in the Alaskan Arctic in 2013. From the five tags recovered we learned that birds spent almost a month in the Alaskan and Canadian Low Arctic after the completion of breeding and prior to migrating south through central North America. All five males overwintered in the coastal grasslands of Texas and western Louisiana. In March individuals headed north where they staged in the Hudson Plains and Taiga Shield ecozones of central Canada prior to completing their migration and returning to their breeding grounds in late May. The stable isotope values for hydrogen and oxygen of winter-grown head feathers collected in 2012, 2013 and 2014 from two tracked birds and four additional individuals suggest males return to the same region year after year. These data reveal that the coastal grasslands of Texas and Louisiana may be an important overwintering habitat for Alaskan-breeding Smith's Longspurs. In addition, our small dataset suggest that Smith's Longspurs may demonstrate a leap frog migration pattern with birds from the western portion of the breeding range traveling further to winter south of the previously described wintering areas in Oklahoma and Arkansas.}

\normaltalk{\capitalisewords{Landscape-Level Information Use Shapes Brown Pelican (Pelecanus Occidentalis) Foraging Behaviors}}{\underline{Brock Geary}, W R James, Jordan Karubian, James A Nelson, Paul L Leberg}{Optimal foraging theory states that animals should maximize resource acquisition rates with respect to energy expenditure, which may involve alteration of strategies in response to changes in resource availability and energetic needs. However, field-based studies of changes in foraging efficiency at fine spatial and temporal scales are rare, particularly among species that feed on highly mobile prey across broad landscapes. To better understand how foraging behaviors of breeding brown pelicans (Pelecanus occidentalis) are altered over time, and to infer how coastal restoration practices may influence these patterns, we used GPS telemetry and multiple characterizations of prey availability to relate bird movements to environmental variation in the northern Gulf of Mexico. Previous projections of Gulf menhaden (Brevoortia patronus) distribution models onto pelican foraging data have shown pelican populations occupying progressively higher-quality foraging grounds as energetic demands from chicks increase. Our current work demonstrates how the broader energetic landscape, as evidenced by isotopic data, shape foraging decisions as well as the paths traversed by pelicans when locating prey. By integrating data on environment, prey, and predators at multiple spatial and temporal scales, we provide a more comprehensive understanding of how top predators make complex foraging decisions, and improve our understanding of how dynamic landscape-level features can impact individual behavior and the demography of important animal populations.}

\normaltalk{\capitalisewords{Does Object Neophobia Predict Exploratory Behavior in a Free-Living Wild Bird?}}{\underline{Angela Tringali}, Sara Prussing, David Sherer, Rebecca Windsor, Reed Bowman}{Exploratory behavior is of interest to ecologists studying a variety of phenomena including dispersal, range expansion, invasion, migration, and adaptation to climate change. Object neophobia, frequently measured as the latency to approach and amount of time spent near a novel object, is sometimes used as a proxy for exploratory behavior. However, the ecological context of these behaviors may be quite different. Thus, response to novelty may not be a good surrogate for exploration. We use a cohort of wild, free-living Florida Scrub-Jays to evaluate the relationship between neophobia and exploration. Florida Scrub-Jays breed cooperatively and remain on their natal territory for 1 or 2 years, making occasional forays away to search for breeding opportunities. We measured object neophobia of juveniles using a modification of a typical assay. The next year, during the pre-breeding season when forays are at their peak, we used a series of point counts stratified across our study to measure exploration. Males who spent more time near the novel object were detected at more unique points, but this was not true for females. Males who were slow to approach the novel object forayed farther from their resident territory, and those who approached quickly stayed close to home. Females showed the opposite pattern: those that approached the novel object quickly forayed farther than those who were slow to approach. Life history differences between the sexes influence the costs and benefits of exploratory forays and may have implications for the utility of object neophobia assays.}

\normaltalk{\capitalisewords{Natural Selection on Body Size During Range Expansion in Red-Bellied Woodpeckers}}{\underline{Joel Ralston}, Erika Mayette-Draper, Shanan Hamilton, Jeremy J Kirchman}{Ecological niche models (ENMs) have traditionally been used to predict species-wide responses to changing climates. However, there is likely variation among individuals within natural population in the ability to cope with changing climates or extreme weather, for example as a function of body size or other morphological characteristics. Indeed, several studies have shown that extreme weather events are important times of increased natural selection that can have lasting impacts on morphology in bird populations. Here we propose that ENMs can be used to study natural selection in wild population by interpreting times of low climate suitability as times of stronger selection. We demonstrate this approach in the study of Red-bellied Woodpecker (Melanerpes carolinus), a species that has undergone a recent range expansion and breakdown of Bergmann's Rule. We modeled annual climate suitability from 1901-2015 as a function of variables that may impact body size, for example mean and extreme winter temperatures. We found periods of low suitability (strong selection) were more frequent and severe at higher latitudes, but that with climate change this pattern weakened over the study period. Additionally, we measured 534 Red-bellied Woodpecker specimens collected within the study period, and found that birds collected during or following years of low suitability tended to be bigger than those collected during periods of high suitability. These results support a role of episodic natural selection in shaping geographic patterns in morphology, and demonstrates a new application of ENMs in studying selection in wild populations.}

\normaltalk{\capitalisewords{Is Specialism an Evolutionary Dead-End? Estimates of Transition Rates Between Generalism and Specialism in a New World Clade}}{\underline{Meaghan Conway}, Brian J Olsen}{Evolutionary theory suggests that diversification proceeds from generalist ancestors to specialist descendants.  However, exceptions to this rule have led to alternate hypotheses that predict the evolution of specialized populations from generalist ancestor is interrupted by periods of niche-breadth expansions.  We tested predictions of the directionality of the evolution of specialization by estimating character state transition rates between generalists and specialists using a novel index of specialization in bill morphology in a diverse clade of passerine birds. We found higher transition rates from specialist to generalist, suggesting that niche expansion to more generalist morphologies is an important driver of diversity in this clade.  We did not find support for irreversibility of specialization in avian bill morphology.  It is clear, therefore, that specialization in a feeding apparatus does not always limit further niche evolution.}

\normaltalk{\capitalisewords{Cryptic Bird Species on a Remote Archipelago of the Pacific Ocean}}{\underline{Alice Cibois}, Jean-Claude Thibault, Nicholas R Friedman, Kevin E Omland, Laure Desutter-Grandcolas, Tony Robillard, Eric Pasquet}{The Pacific Ocean represents the most important diversification of insular reed warblers, all in the genus Acrocephalus. In the Marquesas Islands (Eastern Polynesia), previous studies have shown that reed warblers belong to two distinct lineages that have reached the archipelago independently. However these molecular phylogenies were based solely on mitochondrial genes and on a limited number of individuals. We increased the sampling and we included a combination of mitochondrial and nuclear markers. We found evidence of gene flow between the two lineages, mostly in the nuclear genome. One island in particular, Nuku Hiva, seems to act as the main contact zone between the two taxa. Because of their very uniform plumage pattern, all populations in the Marquesas were originally treated as a single species. We used reflectance spectrometry to evaluate the variation of carotenoid coloration among all populations. We found that the two lineages exhibit convergent evolution in carotenoid-based plumage coloration, but stochastic variation among islands was important. Finally, we showed that vocalizations differ slightly between taxa, significant differences between islands having led to different dialects. These vocal differences may act as the main pre-zygotic barriers when birds from different lineages come into contact. This study provides another example of convergence in plumage evolution in insular bird species. Furthermore, it highlights the importance of integrative analyses, using both genetic and phenotypical characters, for the study of the evolutionary history of the Pacific Ocean avifauna.}

\normaltalk{\capitalisewords{Tiger Sharks Eat Passerines: Exploring the Prevalence of a Unique Cross-System Pulsed Subsidy}}{\underline{Marcus Drymon}, Kevin Feldheim, Auriel Fournier, Emily Seubert, Andrea Kroetz, Sean Powers}{As apex predators, sharks can couple energy pathways from disparate foodwebs. This is particularly true of tiger sharks (Galeocerdo cuvier), notorious for their dietary breadth. In addition to invertebrates, fishes, sea snakes, marine mammals, and seabirds, tiger sharks also consume terrestrial birds. We investigated the prevalence of this unique cross-system subsidy using a combination of monitoring programs, traditional stomach content analysis, and DNA barcoding. Tiger sharks were sampled during bottom longline surveys off the coast of Alabama from 2010-2018. Stomach contents were examined opportunistically and identified to the lowest possible taxon. Avian remains were genetically identified. Avian species composition from tiger shark stomachs was then compared to bird sighting data from eBird from that same time period. From 2010-2018, 774 bottom longline sets were performed and 292 tiger sharks were encountered. Opportunistic analysis of tiger shark stomach contents revealed remains from 11 bird species: 8 passerines (barn swallow Hirundo rustica, eastern kingbird Tyrannus tyrannus, house wren Troglodytes aedon, common yellowthroat Geothlypis trichas, marsh wren Cistothorus palustris, eastern meadowlark Strunella magna, swamp sparrow Melospiza georgiana, and brown thrasher); 2 near passerines (white-winged dove Zenaida asiatica and yellow-bellied sapsucker Sphyrapicus varius); and 1 waterbird (American coot Fulica Americana). Peaks in eBird sightings data for the 11 species we identified showed remarkable alignment with individual tiger shark/bird interactions. Using this combination of data, we were able to evaluate the frequency of a cross-system pulsed subsidy to a marine predator, thereby furthering our understanding of this unique trophic interaction.}

\normaltalk{\capitalisewords{Dominance in Hummingbird Communities: An Alternative  Way to Visually Represent it}}{\underline{Raul Ortiz-Pulido}}{Dominance hierarchies have been visually represented in several ways, such as in a matrix or diagram, but it is difficult to quickly understand visually from them the interactions carried out between the interacting entities in a community, overall when there are complex interactions between multiple entities. Here I propose a new way to visually represent the hierarchy of dominance between entities in such systems. I demonstrate this method using data from nectar feeding bird species interacting agonistically in two ecosystems- one in Australia (honeyeaters) and other in America (hummingbirds). The best aspects of showing antagonistic interactions in the new way are: (1) that the graphs can be compared visually with other interaction diagrams in related fields, and (2) that the analytical tools produced in other fields can be used to assess antagonistic networks. With the new visual approach, it is clear that agonistic and mutualistic networks are similar in several ways; they can be showed with nodes and lines between or within nodes. Thus, antagonistic networks can be quantified in ways not used previously, such that it is possible to obtain, with relatively minor changes, automated graphs from computational programs and several ecological metrics used currently to understand mutualistic interactions. This is the case for metrics of nestedness, modularity and robustness, or the identity of core and peripheral species and the effects of extinction on networks, among other information.}

\normaltalk{\capitalisewords{Cross-Scale Interaction Effects of Climate and Forest Patch Size on the Persistence of Forest-Interior Birds}}{\underline{Anand Chaudhary}, Kevin J Gutzwiller}{Climate and habitat are among the most important factors influencing avian survival and reproduction.  These environmental factors may be operating at different spatial scales, and the multiplicative influence of these factors on avian species may be much greater than their additive effects alone.  Little is known about how interactions, especially across spatial scales, affect birds.  We explored cross-scale interaction (CSI) effects of sub-regional climate (breeding-season temperature and precipitation, spring precipitation, and winter temperature) and landscape-scale deciduous-forest patch size on four declining forest-interior bird species in the eastern United States.  For each of the four species, we compared twenty-one a priori ordinary least-squares models using AIC statistics.  Persistence of all four species was influenced by forest patch size and two or more climate variables.  One of the four species, the Scarlet Tanager (Piranga olivacea), was influenced by a CSI involving maximum breeding-season temperature and forest patch size.  In sub-regions with higher breeding-season temperatures, Scarlet Tanager persistence increased with an increase in forest patch size, but in sub-regions with lower breeding-season temperatures, the species' persistence decreased with an increase in forest patch size.  Had we not considered interactions, our best model would have shown that Scarlet Tanager persistence declined with increasing breeding-season temperature and with decreasing forest patch size.  These results suggest that the influence of climate warming on Scarlet Tanager persistence may be reduced if adequate-sized forest patches are maintained, and that failure to consider CSIs may result in misleading information that could reduce the effectiveness of land-management efforts for birds.}

\normaltalk{\capitalisewords{Forest Management, Habitat Conservation and Monitoring Northern Goshawks in Southeast Alaska}}{\underline{Kimberly Titus}}{Northern Goshawks (Accipiter gentilis) were identified in Tongass National Forest planning as a species associated with old growth forest.  Various Endangered Species Act (ESA) petitions and lawsuits became entangled in Tongass forest planning associated with goshawks, wolves and other wildlife.  This resulted in the Tongass Conservation Strategy that included a network of old-growth reserves and other management guidelines within the matrix of managed (logged) landscapes.  The result was to maintain habitats for old-growth associated wildlife – goshawks, wolves, flying squirrels and assure species viability.  A study of radiotagged (n = 68 adult) goshawks found that this bird was strongly associated with old-growth forest and that nesting adults also have some of the largest home ranges described for the species.  45\% of these adult female goshawks moved $>$6km between years to different nesting areas and 0\% of the adult males moved to different home ranges/territories.  We also found that goshawks rely on a few prey such as red squirrels, Sooty Grouse, and mid-size passerines.  Results suggest that goshawks are likely conserved as a nesting bird by the conservation strategy.  However, monitoring goshawk nest stands and temporarily protecting known nest trees as part of timber sale planning may not maintain those nest stands because the birds are so difficult to detect and their movements further complicate monitoring.}

\normaltalk{\capitalisewords{Testing the Proposed IUCN Green List Categories and Criteria}}{\underline{Garima Gupta}, Philip McGowan, Jonathon Dunn, Roy Sanderson}{Prevention of a species from going extinct is a basic pillar of conservation biology. Conservation action has had success in the field of biodiversity conservation, however these actions take time in achieving its objective and it is imperative to evaluate the outcomes of these actions. In spite of the fact that IUCN Red List has been remarkably effective in catalysing conservation efforts, a robust method is needed to assess species recovery owing to positive conservation actions and to celebrate the conservation victory. The IUCN Green List of Species has been proposed with an aim to complement the IUCN Red List by providing a tool for assessing the recovery of a species population and for measuring conservation success. I have assessed the implementation of the IUCN Green List methodology with different spatial units for Galliformes in the Himalaya to understand what challenges and limitations exists for each of the different spatial units. Countries are assessed to be the most feasible spatial unit for Himalayan Galliformes. Determining ecological "functionality" of the species was found challenging. Information on other variables such as indigenous (natural) range (spatial baseline) and "Past" definition (temporal baseline used for counterfactual state) were based on experts' opinion. Once implemented, the IUCN Green List will help the conservationists and decision makers to focus on those species that are greatly reliant on conservation actions and have high potential to gain from these conservation actions.}

\normaltalk{\capitalisewords{Extensive in Situ Radiation of Feather Lice on Tinamous}}{\underline{Stephany Virrueta Herrera}, Jason D Weckstein, Andrew D Sweet, Julia M Allen, Kimberly K Walden, Kevin P Johnson}{Tinamous host the highest generic diversity of lice of any group of birds, including four avian feather louse ecomorphs. While tinamou feather louse generic diversity is well documented, few attempts have been made to address the phylogeny of these lice. To address whether tinamou feather lice form a monophyletic group, we estimated a higher level phylogeny of tinamou feather lice and relatives using phylogenomic data derived from genome sequencing. Data from 48 genera of avian lice and over 1000 genes were analyzed using both concatenated (RaxML) and coalescent (ASTRAL) approaches. We found that tinamou feather lice are not a monophyletic group as a whole, and genera from other avian feather lice were recovered in a clade together with genera of feather lice from tinamous. Many of the genera that are nested phylogenetically within tinamou lice occur in South America, the main center of diversity for tinamous. These results suggest a possible in situ radiation of these parasites in South America. Using the same genome sequences, we have identified the bacterial symbionts found within these lice, which are thought to provide nutritional benefits to the lice. These data indicate the symbionts of these lice have been lost and replaced several times over evolutionary time. These data provide the first molecular phylogeny for this highly diverse group of avian feather lice as well as novel information regarding their bacterial endosymbionts.}

\normaltalk{\capitalisewords{Out in the Cold: a Novel Approach to Understanding the Physiological Drivers of Phenotypic Flexibility in Dark-Eyed Juncos}}{\underline{Maria Stager}, Nathan R Senner, Zachary A Cheviron}{Organisms maintain dynamic regulatory systems that can confer the flexibility to reversibly match their phenotype(s) to fluctuating environmental conditions. This process often involves the dramatic modification of multiple subordinate traits. However, the relative influence of these component traits on whole-organism performance is poorly understood in natural systems. As a case study, we explore the contribution of subordinate phenotypes to body temperature regulation in the Dark-eyed Junco (Junco hyemalis), by combining assays of gene expression, tissue-level- and whole-animal physiology in a novel network analytic framework. This work indicates that organismal performance is disproportionally influenced by a few subordinate traits. We replicate this approach across multiple Junco populations that vary in the natural thermal regimes they experience and interpret these patterns in light of historical demographic processes. Our results shed light on the mechanisms underlying seasonal phenotypic flexibility, and provide a general analytical framework for other evolutionary studies of similarly complex physiological traits.}

\normaltalk{\capitalisewords{From Shallow to Deep: A Torpor Spectrum in Hummingbirds Uncovered using Thermal Imaging}}{\underline{Anusha Shankar}, Isabelle N Cisneros, Sarah Thompson, Catherine H Graham, Donald R Powers}{Daily torpor—a controlled reduction in body temperature and metabolic rate—is a key energy saving strategy for small endotherms. In hummingbirds, past studies show the exclusive use of deep torpor, where body temperature drops by 23-34oC down to a species-specific minimum. As hummingbirds normally have some of the highest mass-specific vertebrate metabolic rates, using deep torpor to maximize nighttime energy savings seems an efficient energy management strategy. However, deep torpor presents a predation risk, likely prevents restorative sleep and inhibits immune function, and incurs steep rewarming costs during arousal. Therefore, a shallower form of torpor (i.e., a decrease of 3-10oC) might sometimes be a better energetic strategy to balance energy savings. The occurrence of both deep and shallow torpor in a single species is known in mammals, but has not been reported in any bird species. Using infrared imagery in Arizona under natural temperature and light cycles, we found that all three hummingbird species we studied used shallow torpor, and it often occurred with deep torpor on the same night. Blue-throated hummingbirds (8.4g) spent an average of 25\% of the night in shallow torpor, while black-chinned hummingbirds (2.9g) spent only 5\% in shallow torpor. Rivoli's hummingbirds (7.6g) spent the most time in shallow torpor (35\%) and appeared the most flexible at managing nighttime metabolism. Hummingbirds appear to have much finer control of their torpid metabolism than previously thought. Since hummingbirds can use both shallow and deep torpor, it is possible that torpor is on a physiological continuum with sleep.}

\normaltalk{\capitalisewords{Factors Mediating Reproductive Isolation Between Related Species at Contact Zones}}{Alexander NG Kirschel, Emmanuel C Nwankwo, Ara Monadjem, Gregory F Grether, \underline{Alan Brelsford}}{When related species come into contact their interactions may include competition for resources, territory, even mates. Such interactions can influence species range limits and drive phenotypic evolution. But little is known about the extent of variation in phenotypic or ecological traits that might lead to a breakdown in assortative mating. We investigated the extent of genomic and phenotypic variation across independent contact zones between closely related red-fronted and yellow-fronted tinkerbirds in sub-Saharan Africa, and performed a genomewide association study to determine genes associated with forecrown coloration. Each contact zone involves interactions between a different pair of subspecies and variation in ecological gradient steepness. We found evidence that the extent to which songs differ mediates the extent of reproductive isolation. Where songs differed more in allopatry, they remained distinct where the two species came into contact. By contrast, where songs were more similar in allopatry, we found a pattern of convergence in song towards the contact zone; with playback experiments showing songs are distinguished only where differences are greatest. Genomic data reveal extensive hybridization at the contact zones where songs converge in spite of over 4 Ma of divergence in mitochondrial DNA, but no evidence of hybridization where songs remain distinct. Genes associated with forecrown colour are found to introgress asymmetrically from red-fronted tinkerbirds into the genomes of yellow-fronted tinkerbirds, suggesting a possible sexual preference for the red plumage. Our work suggests song plays a vital role in mediating when species interbreed, irrespective of the timespan of genetic divergence between them, while a preference for red feather coloration may influence the direction of introgression between the species.}

\normaltalk{\capitalisewords{Restoration Success: Avian Richness and Abundance Among 1,000 Acres of Restored San Francisco Bay Tidal Marsh Habitat, Based on Point Count Surveys that Employ Citizen Scientists}}{\underline{Daniel Edelstein}}{To document avian richness and abundance at a restored tidal marsh adjacent to San Francisco Bay previously devoid of suitable habitat for waterfowl and shorebirds, Sonoma Land Trust initiated point count surveys. Restoration design added physiographic and vegetation features at the site to promote roosting, foraging, and nesting success. Eight annual point count surveys at 12 prescribed stations included data collection support from more than 40 citizen scientists under the direction of a consulting Avian Biologist. Survey dates/times occurred during high and low tides to ensure representative avian observations.  Results from the 2016-2018 surveys yielded sightings of eight avian guilds among tidal marsh, tidal mud flat, upland levee, beach, rocky shoreline, open water, and tidal panne habitats. Shorebird family members represented 36 \% of total avian species observed, followed by diving ducks, 17 \%; dabbling ducks, 14 \%; gulls, 11 \%; raptors, 11 \%; grebes, five \%; rails, three \%; and American Avocet/Black-necked Stilt, three \%.  Two years of survey data suggests that the restoration is a success with the site now providing: 1) essential "wayside" habitat for migrating and overwintering avian species;  2) increased nesting niches for several waterfowl species;  3) robust plant succession from hand plantings that are likely to soon attract additional common and rare nesting avian species (including listed species such as Ridgway's Rail and California Black Rail); and  4) the California Avian Database Center with valuable information to assist the planning and implementation of management actions at other nearby restoration projects.  }

\normaltalk{\capitalisewords{Winter Movement of Brown-Capped Rosy-Finches (Leucosticte Australis) Between Feeder Sites in Colorado, USA}}{Aaron A Yappert, \underline{Erika Zavaleta}, Amy E Seglund, Kathryn Bernier, Gregory J Levandoski, Bruce Ackerman, Susan Werner}{Brown-capped Rosy-Finches (Leucosticte australis) are poorly studied passerines that are nearly endemic to Colorado and are of conservation concern because of limited data on population dynamics and the projected impacts of climate change on alpine breeding sites and winter foraging areas. In winter, Rosy-Finches form large flocks that can number several hundred individuals but local and landscape-level movement by individuals and flocks is poorly understood. We used capture-mark-recapture techniques at six feeder sites across Colorado to quantify movements during three winters (2017 – 2019). 6.5\% of birds recaptured during their original banding winter moved between sites (N = 92). However, 20.7\% of birds recaptured in subsequent winters moved between sites (N = 217), and males tended to move more than females (Pmale=28.1\%, Pfemale=8.8\%, X2=5.75, p=0.057). 88\% of movements between the six sites were short-range ($<$25 km, N = 33); however, three individuals, encompassing both sexes, were recaptured as far as 315 km from their original capture site, which suggests that long-range movement does occur at the level of the individual and involves both males and females. The large proportion of stationary individuals indicates that large-scale flock mixing and movement between regions is uncommon within the same winter. Though long-range individual movements may be rare, they could enable gene flow between distinct breeding regions. Additional work should focus on linking small- and large-scale winter movements amongst age and sex classes with summer breeding site selection to determine the degree to which winter movements influence breeding location and gene flow.}

\normaltalk{\capitalisewords{Evolution of Molt-Migration: a Large-Scale Phylogenetic Approach}}{\underline{Claudie Pageau}, Matthew W Reudink}{Each year, birds complete a full-bodymolt in order to exchange old feathers for new, high-quality ones. For birds, it is important to have feathers of top quality because they are necessary for flight performance, insulation, and communication. Molt requires substantial energy, making the timing of molt and the availability of resources critical. Balancing the demands of molt with other energetically expensive tasks, such as breeding and migration, has resulted in the evolution of various molt strategies. For example, most birds completely molt after breeding on the breeding grounds. However, some birds who breed in the arid West of North America begin migration shortly after breeding, but stop to molt in Northwest Mexico/Southwest United States during the late summer when monsoon season happens. This strategy appears to have evolved due to a lack of food resources on the breeding grounds during late summer compared to the monsoon region where there is an abundance of fruits and insects, perfect conditions for molt. Interestingly, birds using the molt-migration strategy are from at least seven different families, which suggests convergent evolution (i.e., the strategy has evolved multiple times). Thus, the objective of my research is to understand the role that ecological pressure or/and life history characteristics play in the evolution of molt-migration. Using phylogenetic comparative methods, I will test for correlations between molt strategies and life history characteristics, morphological characters, and ecological characteristics.}

\normaltalk{\capitalisewords{The Role of Diet and Nesting in Bill Shape Diversification of Frugivorous Asian Barbets}}{\underline{Anand Krishnan}, Nicholas Souza, Sushma Reddy}{Avian bills are adaptive traits that reflect their ecological function while often being constrained by their evolutionary history.  Bills, which consist of jaw bones covered by a keratinous rhamphotheca, perform a variety of tasks. Few studies, however, have examined how evolutionary changes in the shape of the outer bill covering are connected to their inner bony structures. Our study sought to examine the integration of these features in the Asian barbets, a clade of tropical frugivores. Asian barbets consist of 30-35 species (taxonomy-dependent) that sort into three distinct size classes, primarily feed on soft fruit, and excavate holes in trees for roosting and nesting. We obtained 3D computed tomography scans of the bills of the Asian barbets from museum study skins, which allowed us to segment and study both the outer rhamphotheca and the inner bony structures independently and in relation to each other. These scans revealed a striking difference between the extent of skull and corresponding rhamphotheca in all species of barbet. We employed landmark-based geometric morphometrics to quantify shape differences between species to examine the association of shape changes with ecological niches. Finally, we related bill shape evolution in a phylogenetic context to examine the functional underpinnings of bill diversity.}

\normaltalk{\capitalisewords{Agent-Based Modeling and Simulation -- What's in it for Ornithologists?}}{\underline{Thomas Clemen}, Ulfia A Lenfers, Karen Bradshaw}{The agent-based paradigm originated as a concept in computer science back in the 1990s. Recent advances in computing power have led to increased utilization of this powerful concept in a wide variety of disciplines such as sociology, ecology and many others. General application in the field of ornithology is, however, still lacking. Agents are logical representatives of individuals such as birds, humans, elephants etc. Basically, they are implemented as autonomous software processes, which can interact with one another and their environment using sensors. These sensors are virtual interfaces that mimic eyes, ears, haptics and other senses, allowing the software agents to react to sensory triggers in the same way as individuals in the real world. By transferring ornithological knowledge into behavioral rules in a model, bird agents can adapt realistically to landscape changes, disturbances or other species, e.g. predators. This knowledge-based approach allows the modeling of individual birds and social grouping in flocks even when data availability is sparse.   The authors have conducted short courses on social-ecological modeling and simulation in different locations in Africa and the U.S. since 2012. In our talk, we will present some of the advantages of agent-based modeling by describing a bird-related case study. We will show how to build a conceptual model from the research question and how to transform that into an agent-based design. Our aim is to show ornithologists some of the benefits of utilizing this powerful method in their future research.}

\normaltalk{\capitalisewords{Alternative Pathways to Success: Transient Dispersers in a Cooperative Breeding Corvid}}{\underline{Young Ha Suh}, Michelle L Dent, John W Fitzpatrick, Reed Bowman}{Natal dispersal, the process of moving from natal to breeding territory, is integral for species distribution and population dynamics. Despite its importance as a crucial life-history trait, dispersal is often simplified as an unconditional, single parameter while it is more likely to be condition-dependent with three distinct stages: departure, transience, and settlement. We focused on the transient stage of dispersal and its potential drivers in a wild population of Florida Scrub-Jays Aphelocoma coerulescens, a cooperatively breeding corvid. Using 34 years of demography information, we found that a third of successful dispersers were nonbreeding helpers at other territories before they ultimately acquired a breeding territory. We hypothesized that these "transient" dispersers were more likely to have dispersed from natal territories of low quality or when these birds were subordinate to their siblings. We used generalized linear mixed models to test natal territory attributes and within-group dominance associated with the transient strategy. Natal territory quality had little effect on dispersal strategy; however, socially dominant jays within their group were less likely to be transient dispersers and more likely to disperse directly. Transient dispersers dispersed earlier and ultimately settled farther from their natal territory than direct dispersers, but lifetime reproductive success did not differ based on dispersal strategy. We compared this population residing in contiguous habitat to one in a fragmented habitat to test whether elevated risks in the unsuitable habitat matrix affected dispersal decisions. Our results suggest that Florida Scrub-Jays have condition-dependent dispersal strategies that assist in successful natal dispersal.}

\normaltalk{\capitalisewords{The Allometry of Movement and Connectivity of Bird Communities}}{\underline{Jack Hartfelder}, Robert J Fletcher, Chevonne Reynolds, Ara Monadjem, Robert McCleery}{Predicting connectivity is necessary for understanding how species may persist in the face of environmental change. Despite this importance, understanding connectivity is challenging because it requires information on how organisms successfully move across complex and dynamic landscapes. Furthermore, extending our knowledge of connectivity to entire communities remains a frontier for ecology and conservation. We harness the allometric scaling of movement to derive a community-level framework for connectivity using a multilayer network approach and apply this framework to bird communities in southern Africa. First, we used field translocation experiments on 49 savanna bird species, where we released bird into open sugarcane land-uses and measured flight distances and velocities. These responses identified a positive allometric relationship of flight distances with body size. Then, we related this relationship to dispersal information taken from mark-resight data across the region. Allometric relationships explained 47\% of the variation in observed dispersal distances. Finally, we apply this allometric scaling to network analysis to derive community-level metrics and mapping of network connectivity for birds in the region. Our framework provides a novel and tractable approach for interpreting dispersal and metacommunity dynamics across landscapes and provides a foundation for understanding and managing for connectivity for entire communities.}

\normaltalk{\capitalisewords{Wild Turkey Speciation Seen Through Full Genomes}}{\underline{Hernan Vazquez-Miranda}, Robert M Zink}{Genomic data have changed our perspective on speciation and evolution, bringing unparalleled resolution power to research. Wild Turkeys are charismatic and well-known, however we ignore much of their genetic diversity and diversification despite marked geographic variation. Here, using full genomes from multiple individuals across the range in the US and Mexico, we found that there are multiple diagnosable lineages revealed in the mitochondrial and nuclear genomes in spite of historical and recent reintroduction efforts. These genomic speciation signatures are also reflected in plumage traits traits and are consistent with turkey diversity in the fossil record. We propose that Wild Turkey taxonomy should be revised for conservation and managing purposes.}

\normaltalk{\capitalisewords{Molecular Phylogeny of Whistling Ducks (Dendrocygninae)}}{\underline{John Harshman}}{Congruent phylogenies of the whistling ducks from one nuclear and one mitochondrial gene show that 1) the white-backed duck Thalassornis leuconotus is confirmed as the sister taxon of Dendrocygna; 2) the Cuban whistling duck D. arborea is sister to the black-bellied whistling duck D. autumnalis and those two are sister to the remainder of the genus; 3) the wandering whistling duck D. arcuata is sister to the lesser whistling duck D. javanica; 4) the white-faced, spotted, and plumed whistling ducks D. viduata, D. guttata, and D. eytoni form a clade and D. viduata is the sister of D. eytoni. The sole disagreement among data sets is the position of the  fulvous whistling duck D. bicolor. I consider implications for biogeography and morphological evolution.}

\normaltalk{\capitalisewords{Comparative Phylogeography of Amazonian Riverine Island Birds}}{\underline{Oscar Johnson}, Robb T Brumfield}{The riverine island avian community of the Amazon basin is characterized by decreased alpha diversity and increased abundance compared to bird communities of other widespread Amazonian habitats. The island habitats on which these birds specialize are both highly ephemeral and linearly distributed across the Amazon Basin. As a consequence, island specialist species would be predicted to have greater dispersal abilities and smaller population sizes relative to related species in upland forest. Here we aim to assess how habitat specialization to a highly ephemeral habitat affects population genetic structuring in river island specialist species. We utilize population-level sampling of a selection of Amazonian riverine scrub forest specialist birds and sequence capture of ultraconserved elements to assess the phylogenetic patterns of this bird community. Applying a suite of population genetic and phylogenetic analyses to these data, we infer the population structure and evolutionary history of these species. To test the above predictions we compare our results to the bird communities of two other dominant Amazonian forest types - upland terra firme forest and flooded varzea. The results from this study allow better characterization of the evolutionary patterns of Amazonian river island birds and provide a more complete picture of the avian evolutionary history of Amazonia.}

\normaltalk{\capitalisewords{Long-Term Change in the Intrinsic and Extrinsic Drivers of Individual Fitness}}{\underline{Peter Arcese}, Ryan Germain}{Despite much evidence that climate change, competitors, and predators can limit the distribution and abundance of native species, few studies estimate the relative influence of multiple intrinsic and extrinsic factors on individual fitness or identify those factors with chronic versus episodic effects over decades.  We tested existing hypotheses on the influence of 9 intrinsic and exogenous limiting factors on fitness in Song Sparrows (Melospiza melodia). Annual and lifetime fitness were estimated over 42 yrs (1975 – 2018), a period that varied greatly in climatic conditions, coincided with colonization by a dominant competitor (Fox Sparrow, Passerella iliaca), and declining nest depredation by cowbirds (Molothrus ater). Variation in climate, population, Fox Sparrow abundance, and the number of eggs laid by cowbirds (extrinsic factors), and variation in inbreeding, individual quality, lifespan, and age (intrinsic factors) together accounted for substantial variation in fitness. However, despite long-term declines in severe weather and cowbird abundance, Song Sparrows declined 75\% as Fox Sparrows increased 3000\%. Fox Sparrow abundance was positively related to annual reproductive rate in Song Sparrows, but negatively related to lifetime fitness, because juvenile survival also declined by 50\%. Because the pace of human-induced climate and land use change is accelerating, long-term studies are now offering key lesson on the ways in which the predictions of classical demographic models may fail, despite being based on temporally precise estimates of key demographic rates.}

\normaltalk{\capitalisewords{Associations of the Cloacal Microbiome with Sexual Ornamentation, Condition, Sex and Habitat in Molting House Finches}}{\underline{Pierce Hutton}, Mathieu Giraudeau, Tuul Sepp, Kevin J McGraw}{The recent flourish of studies suggests that microbiomes have co-evolved with avian host ecology, physiology, and behavior. However, few studies have tested how microbiomes might be under the influence of sexual selection, such as how the microbiome relates to signals used by choosy sexes during mate choice. We studied molting male and female House Finches (Haemorhous mexicanus) from both desert and urban populations near Phoenix, AZ, USA. Specifically, we tested how cloacal microbiome diversity and assemblage relates to developing ornamental carotenoid-based plumage color, body condition, and intestinal parasitism. Additionally, we tested how microbiome diversity and assemblage varies between sexes, and between urban and rural habitats.}

\normaltalk{\capitalisewords{Who is There? Response to Different Type of Territorial Intruders}}{\underline{Luis Sandoval}, Tihana Hamzaj, Kathy Bonilla, Ignacio Gutierrez}{The majority of tropical bird species show territorial behavior throughout the year. To avoid unnecessary energy expenditures in territory defense, territorial individual developed capabilities to distinguish between strangers and neighbors; and when showing less aggression towards neighbors, the response behavior is called "dear enemy effect".  In this study, we analyze if White-eared Ground-Sparrows (Melozone leucotis) use duets to recognize neighbors and reduce territorial interactions. We use this species as a model-species because duets are used in territorial interaction throughout the year and are produced with a vocalization different from male solo-song. We conducted 32 playbacks in three populations of Costa Rica. In each experiment, we reproduced a neighbor, same population pair but non-neighbor, other population non-neighbor pair, and a control duet inside each pair territory. We measured five behavioral responses: latency of the first vocalization and approaching within 3 m of the speaker, duration close the speaker, number of individuals approaching, and number of vocalizations. Contrary to the majority of previous studies our results showed more aggressiveness response against neighbors than non-neighbors from the same or another population. This may be result of a higher probability of encountering a neighbor in the territory threatening territory size, mate or resources in comparison to non-neighbors. How common is this phenomenon inside tropical bird species, is poorly known. Therefore, to develop a better understanding of territorial behaviors in this large group of species it is worthwhile to conduct more studies on this topic including the large variety of territorial behaviors and socials interactions.}

\normaltalk{\capitalisewords{Understanding Cumulative Effects for Land Management in Alberta: Models and Applications}}{\underline{Peter Solymos}, Judith D Toms, Erin M Bayne}{Transformation of native habitat by human activity is the main cause of biodiversity loss. Humans have visibly transformed 27\% of Alberta, Canada, to date. We compiled all available point-count data and applied statistical techniques for data harmonization and population density estimation to understand the effects of anthropogenic disturbances on bird species. We used our model estimates to predict species distributions in northern Alberta under current landscape conditions, and compared that to a reference landscape where human footprints have been removed and 'backfilled' by the most likely natural vegetation. We assessed how different industrial sectors, such as forestry, agriculture, energy, transportation, and urban/rural development has impacted habitat supply for different bird species. Some of the results from our efforts are being used for scenario based land-use planning. As we are refining our approach, we are also evaluating the relative contributions of local (direct) and landscape level (indirect) drivers of habitat change and assessing whether the cumulative effects of multiple industrial sectors are additive.}

\normaltalk{\capitalisewords{How can Habitat Degradation Push Extreme Migrants over the Edge?}}{\underline{Ying-Chi Chan}, T L Tibbitts, Chris J Hassell, He-Bo Peng, Theunis Piersma}{Long-distance migrations are a major part of the annual cycle of many shorebird species, and often involve multi-day non-stop flights over geographic barriers. These extremely long flights are operated close to the individual's physiological limits, and are prone to rapid changes in the environmental circumstances that they evolved. This may now be occurring in the East Asian-Australasian Flyway, where the staging mudflats of migratory shorebird, especially those in the Yellow Sea, are rapidly being degraded by human activities. To understand the effects of habitat degradation to the annual routines of great knots, an 'Endangered' species listed in the IUCN Red List, we tracked individual migration by satellite telemetry in 2015-2017. We also measured prey availability at major staging mudflats along the Chinese coast by grid sampling. We detected shifts to other staging sites during the northward migration, which could be a plastic response to the drastic declines in local food availability and quality. Moreover, given the current poor food situation in the Yellow Sea, we suggest that it is becoming more crucial that great knots made short fueling stops before reaching the Yellow Sea. This strategy reduces starvation risk upon arrival at the Yellow Sea, and allow time to adjust physiologically to consume prey of lower quality. We discuss how individual flexibility and variations of migration strategies within the population allow adjustments to worsening conditions at staging habitats. Understanding these adaptations and their constraints enable us to define the 'edge' of a population collapse.}

\normaltalk{\capitalisewords{Net Effects of Birds in Agroecosystems}}{\underline{Liba Pejchar}, Yann Clough, Johan Ekroos, Kimberly Nicholas, Ola Olsson, Dafne Ram, Mattias Tschumi, Henrik Smith}{Incorporating both ecosystem services and disservices into land-use decisions is essential for meeting conservation and livelihood goals. This concept, termed net effects, has both merits and challenges for birds in agroecosystems. Although birds have widely documented positive (e.g., pest control) and negative (e.g., crop damage) impacts on agriculture, the net effects of such activities are rarely quantified. This could be attributed to the complexity of measuring direct and indirect trophic interactions, and the necessity of cross-disciplinary collaboration to value biophysical outcomes in economic and other relevant terms. In a recent forum article, we suggest that the direction and magnitude of net effects is driven by biotic, farm-level and landscape factors. These factors, some within and others beyond farmer control, interact in potentially predictable ways. We propose a framework for making policy decisions about farming practices and land-use planning informed by net effects to help farmers and society achieve positive outcomes for biodiversity and agricultural production.}

\normaltalk{\capitalisewords{Foraging Selectivity of Insectivorous Birds in Kenyan Shade Coffee Systems}}{\underline{Deven Kammerichs-Berke}, Matthew Johnson, Tim Bean}{Shade coffee has been recognized as a well-studied example of a land-sharing strategy that both creates habitat for birds while also maintaining agricultural yield. While there is a general consensus that a shade strategy is more "bird-friendly" than a sun monoculture strategy, little work has been done to investigate the effects of specific shade tree species on bird diversity, and the capacity for different tree species to help deliver ecosystem services. Previous studies in temperate regions have demonstrated that, due to shared evolutionary histories, native plant species are better at promoting native arthropod numbers, which in turn support a greater number of birds in an area. Using avian foraging observations, mist netting data, and arthropod sampling, I studied the effects of two shade tree taxa - indigenous Cordia sp. and introduced Grevillea robusta - on bird foraging behavior and arthropod abundances on central Kenyan shade coffee farms. Initial analyses indicate that arthropod biomass and bird foraging was higher on Cordia than Grevillea, with ongoing analyses examining occupancy and the relationship between canopy-foraging and the presence of insectivorous birds in the understory where they potentially deliver pest control services. Identifying differences in prey abundance and preferences in bird foraging behavior not only fills basic gaps in our understanding of the ecology of East African coffee farms, but also aids in developing region-specific information to optimize functional diversity, ecosystem services, and the conservation of birds in agricultural landscapes.}

\normaltalk{\capitalisewords{What is Multiple-Benefit Conservation and Why Care?}}{\underline{Thomas Gardali}, Nathaniel E Seavy, Kristen E Dybala}{Despite abundant evidence of the importance of the environment to human health and well-being,  earth's biomes continue to be lost and degraded at alarming rates. Addressing this problem requires increasing the pace and scale of conservation, and explicitly incorporating human health and well-being into conservation strategies.  Multiple-benefit conservation is a promising framework for this approach, and can be defined as projects designed to meet societal needs, enhance ecological function and improve habitat quality for fish and wildlife. Multiple-benefit conservation is a variation on the idea of ecosystem services, but its promise is that it is easily understood and can include benefits that are not easy to monetize. Multiple-benefit conservation also appeals to our human desire to do or to get more than one good thing at the same time, especially things that are not usually possible to have together (e.g., stable bird populations and economic growth). Bird conservation fits well into the multiple-benefit framework because birds are recognized as excellent environmental indicators, are highly visible, are desirable for recreational purposes, and their populations can be relatively easily described. As ornithologists, we need to establish credible links between bird conservation activities and additional benefits such as climate stabilization, clean water, and local economies.}

\normaltalk{\capitalisewords{The Influence of Climatic Dipoles on Irruption Dynamics in a Warming World}}{\underline{Benjamin Zuckerberg}, Court Strong, Jalene LaMontagne}{Ecologists have long marveled when, in some years, large numbers of birds appear far beyond their usual breeding and wintering grounds. These avian irruptions are witnessed by millions of people who frequently record the arrival of unfamiliar species to their feeders and backyards. In North America, the most well-known examples of this phenomenon are seed-eating birds that typically overwinter in the boreal forests of Canada, but in some years, irrupt south as far as Mexico. Using data generated from Project FeederWatch (PFW), an international citizen science program, we have documented how Pine Siskin (Carduelis pinus) irruptions are entrained by continent-wide dipoles in climate. These dipoles influenced siskin irruptions at time lags of two years, suggesting an indirect effect of climate on seed production in boreal forests. We have now extended these analyses to a suite of 12 additional species to explore the existence of multi-species dipoles in irruption and how these are influenced by climatic drivers of temperature and precipitation. Using data from PFW combined with empirical orthogonal functions, we found that irruptive species share similarities in ecological zones of irruption and community-wide synchrony driven by similar patterns of temperature and precipitation anomalies at continental scales. Future warming will likely influence the variability and geography of these ecological dipoles and their climate drivers. The fusion of citizen science and methods of climate science have discovered new spatiotemporal characteristics of irruptive migration and the synoptic collection biological and climate observations is a clear advancement in our understanding of avian ecology at continental scales.}

\normaltalk{\capitalisewords{Conservation Lessons from the Study of Boreal Birds at their Southern Periphery}}{\underline{Joel Ralston}, William V DeLuca}{Many boreal forest birds reach the southern periphery of their distributions in the montane spruce-fir forests of New York and New England. Because the southern periphery may be the first to be impacted by warming climates, these populations provide a unique opportunity to examine several factors that will influence the conservation of this threatened group under climate change. We discuss recent research on boreal birds in Northeastern US and in Maritime Canada related to population trends in abundance, niche tracking in response to climate change, habitat selection and community composition, threats from shifting nest predators, and advances in predictive distribution modeling. We discuss how results from these studies may inform the conservation of boreal birds in a warming world.}

\normaltalk{\capitalisewords{Conservation Planning for Boreal Birds in a Changing Climate: A Framework for Action}}{\underline{Diana Stralberg}, Dominique Berteaux, Mark Drever, Ronnie Drever, Ilona Naujokaitis-Lewis, Fiona Schmiegelow, Junior Tremblay}{The boreal forests of North America support billions of birds of over 300 species. The region remains mostly intact, but is expected to undergo major changes due to anthropogenic climate change over the next century. This warming and resulting changes in moisture regimes are altering vegetation and disturbance dynamics, and will likely result in expansion of grasslands and deciduous forests, which should strongly challenge bird species to keep pace. We propose a vulnerability-adaptation framework to guide bird conservation based on species' individual vulnerability and exposure to climate change.  For sensitive species with declining populations, conservation should focus on management of current threats and species recovery in situ to improve adaptive capacity and facilitate future shifts in distribution. Sensitive species with high exposure to climate change will warrant more extreme intervention, such as translocation or habitat manipulation. For species with lower sensitivity and stable populations, but high climate change exposure, long-term investments in protecting refugia and stepping stones will be most effective.  In general, across all species, land-based approaches that "conserve nature's stage" by promoting geophysical diversity and habitat connectivity, maintaining natural disturbance dynamics, and facilitating broad shifts in bird distribution may prove most effective in maintaining species diversity. Implementation of this framework will require large-scale, inter-agency coordination on recovery plans, as well as adaptive forest management, designation of critical habitat, and land protection.  Challenges include data gaps, uncertainty about future conditions, coordination of conservation actions during the non-breeding periods, and the region's vast scale.  However, given the region's continental importance, successful implementation of this framework could benefit birds throughout the western hemisphere.}

\normaltalk{\capitalisewords{What can We Do with Survey Design Specific Biases in Point-Count Data? Integrating Roadside Surveys and New Technologies}}{\underline{Peter Solymos}, Steven M Matsuoka, Steven Van Wilgenburg, Diana Stralberg, Steve G Cumming, Erin M Bayne}{Not all point-counts are created equal: increasing duration and radius leads to increasing expected counts. Such differences related to sampling effort produce monotonic relationships with time and area and are relatively easy to control for given some ancillary information. Other characteristics of point-counts, however, can create less predictable variation in expected counts. The first class of such characteristics relates to differences in sensor sensitivity: this includes differences among observers or different types of recording devices, and leads to variation in the effective area sampled. The second class of characteristics relates to directional differences in signal transmission as a function of habitat, i.e. along linear features or in edge habitats: these characteristics lead to a mixture of biological (numeric and behavioural responses) and non-biological effects that complicate attribution. We present simulation results to highlight the magnitude of possible biases and review a range of design- and model-based options (including calibration and paired sampling) to adjust for these types of differences when combining point-count data sets for integrated analyses.}

\normaltalk{\capitalisewords{Estimation of Occupancy Parameters for Breeding Birds on the Basis of Single-Visit and Multiple-Visits Point-Count Designs}}{\underline{Matthias Leu}, Rick D Scherer, Vitek Jirinec, Erica Fleishman}{Occupancy estimates adjusted for imperfect detection commonly are used to investigate persistence of bird species and to identify factors associated with bird distributions. To estimate probabilities of occupancy and detection, sites must be visited multiple times within a given period, with surveys conducted either consecutively on a single day (the single-visit design) or on multiple days (the multiple-days design). We evaluated the two designs with respect to potential differences in estimates of occupancy for 26 species that breed in eastern deciduous forests. We visited 131 sites 3 times during each of 2 breeding seasons. We used single-season occupancy models to derive model-averaged estimates of occupancy on the basis of 6 models of detection probability that included different covariates. Estimates of occupancy that were based on the multiple-days design consistently were higher than those based on the single-visit design. We then evaluated potential biases in each design by simulating data in which we kept occupancy and detection constant but varied availability (i.e., a species is locally present and available for detection). Availability induced a negative bias in occupancy estimates in the single-day design but a positive and smaller bias in the multiple-days design. In contrast, availability induced a negative bias in the probability that a given species was detected at least once during three surveys in the multiple-days design but bias was negligible in the single-day design. Our results suggest that the multiple-visits design is preferable because the occupancy estimates it yields are more robust to changes in availability.}

\normaltalk{\capitalisewords{Migratory Patterns of Sooty Shearwaters (Ardenna Grisea) Across the Pacific Ocean}}{Scott A Shaffer, \underline{Autumn-Lynn Harrison}}{As a follow-up to our initial study that revealed the migratory patterns of sooty shearwaters (Ardenna grisea) across the Pacific Ocean, we synthesize subsequent studies on the species to make predictions about possible changes in habitat use associated with climate change. We also place these results in the context of other trans-Pacific seabird migrations we have studied, including Laysan albatross, black-footed albatross, long-tailed jaeger, and glaucous gull. On average, sooty shearwaters conduct a trans-equatorial migration lasting 198±17 days while traveling 64,037±9779 km. Breeding pairs often migrate to opposite sides of the North Pacific and there is a distinct repeatability for travel to similar regions of the North Pacific in consecutive migrations. Sooty shearwaters are prolific divers, routinely reaching depths of 15.9±10.8 meters but depths to 69.9 meters have been recorded. While migrating, sooty shearwaters pass through the EEZ's of no less than 30 countries but spend overall, 65.9\% on the high seas. Climate models predict substantive changes in the North Pacific oceanography, yet sooty shearwaters may actually benefit (e.g. 20-30\% increase in core habitat). However, we caution that global wind patterns may alter travel times and trajectories as shearwaters transit between hemispheres. Clearly additional modeling is required to test this but higher shearwater mortality and poor breeding performance have been linked to North Pacific ENSO events suggesting that migration costs are high enough to impact sooty shearwater populations.}

\normaltalk{\capitalisewords{Demographic and Distributional Responses by Long-Distance Migratory Shorebirds to the Rapid Loss of Staging Habitat}}{\underline{Ying-Chi Chan}, Tamar Lok, Suet-Wah S Chung, T L Tibbitts, Chris J Hassell, Theunis Piersma, Shen Zhang}{Habitat destruction is a major threat to bird populations. Whether this leads to population declines depends on the ability of birds to cope by behavioral flexibility and redistribution. We studied this process in a long-distance migratory shorebird, the bar-tailed godwit (Limosa lapponica menzbieri), facing rapid habitat loss at their main refueling area during their migration, which is the mudflats in the Yellow Sea. Analyses of satellite imagery showed that between 2007 and 2018, mudflats were destroyed in nine out of our 12 study sites, mainly being converted to aquaculture ponds and industrial uses, led to a total of about 40\% loss of mudflat area. We compared migration route and timing of godwits satellite-tracked in 2015-2018 to those in 2008. The tracked godwits did not change their staging duration and behavior in the Yellow Sea. However, sites where almost all the mudflat was reclaimed were no longer visited in 2015-2018. At sites where substantial amounts of mudflats remained after reclamation, godwits shifted seawards into more offshore mudflats, which are exposed for shorter times allowing less time to forage. As survival rates estimated from mark-resighting data of this population declined since 2011, the shift to deeper mudflats could not fully compensate for the loss of foraging habitat. A severe population decline is expected if there is any further loss in mudflats in the Yellow Sea.}

\normaltalk{\capitalisewords{Cross-Seasonal Interactions: The Pacific as the World's Greatest Theater of Bird Migration}}{\underline{Theunis Piersma}}{In the consciousness of Pacific seafarers of the past were migratory birds that carried physical items from the High Arctic to the south Pacific. More recently, we have learned about the nature of these birds, and the enormity of their individual non-stop flights. No longer is the Pacific seen as just a formidable barrier between habitats in the north and the south, but rather as a gateway allowing specialized birds to make a living on seasonal resources around the globe. This recent change in perspective is dramatic, and the research that underpins it has posed many new questions that again challenge our sense of 'normal'. How do birds manage many days of nonstop high-level exercise with the lack of sleep? What explains birds that act as if they possess GPS?  What is the role of learning in the establishment of trans-Pacific seasonal migrations? How is it that some shorebirds are better wind forecasters than the best meteorologists? What does all the ecological upheaval in East Asia mean for the Pacific migrations? How do such seemingly extreme migrations evolve? With ever better tools at their fingertips, migration biologists are now poised to greatly expand the horizons of biology as we know it.}

\normaltalk{\capitalisewords{Evolutionary Genomics of Variable Carotenoid-Based Ornamentation in the Red-Backed Fairywren}}{\underline{Sarah Khalil}, Jennifer Walsh, Erik D Enbody, Daniel T Baldassarre, Michael S Webster, Jordan Karubian}{Carotenoid-based pigments provide most of the vivid red, orange, and yellow colors we see in vertebrates. Though carotenoid plumage is an important mediator of inter- and intrasexual social interactions, we lack an understanding of the underlying genetic mechanisms that control their production. Genomic studies of hybrid zones can help elucidate these mechanisms because the combined natural variation and reduced background divergence from admixture offers unique opportunities to identify genomic regions associated with phenotype differences. In the red-backed fairywren (Malurus melanocephalus), two subspecies differ drastically in the hue of the male red plumage patch: the western crimson-backed M.m.creuntatus and the eastern orange-backed M.m.melanocephalus. There is no relationship between abiotic environment and hue, but there is experimental evidence for strong female preference for crimson backs, suggesting this is a sexually-selected trait. Moreover, there is evidence of plumage hue introgression between the two subspecies across a hybrid zone, with the crimson color introgressing into orange populations. Hybridization is extensive in this system, and hybrids exhibit variation in the hue of their red patch. Here, we leverage this natural hybridization by using genomic data and admixture mapping to identify genes that are associated with variation in carotenoid-based plumage. We sequenced whole genomes of the 10 most "crimson" and 10 most "orange" males from a single population within the hybrid zone, identifying divergent genomic regions associated with plumage variation. Identifying pigmentation genes improves our understanding of how these phenotypes are produced and maintained, which is fundamental to our understanding of the evolution of these signals.}

\normaltalk{\capitalisewords{Using Nestboxes to Understand if Landscape Influences American Kestrel Reproduction and Movements along an Urban Gradient in Northern Utah}}{\underline{Jesse Watson}, Dave Oleyar}{Understanding how American Kestrels (Falco sparverius) respond to different human-modified landscapes could help explain documented declines of this formerly widespread North American species.  Annually since 2014 we've utilized an army of 40-60 citizen scientists to monitor a network of 200-400 nest boxes distributed in different landscapes along the greater Salt Lake City metropolitan area, an area with a human population in excess of 1.2 million.  We document clutch size, brood size, and number of fledglings produced by nesting Kestrels in wildland, agricultural, and urban landscapes.  From 2014-2018 we monitored 421 Kestrel nesting attempts and found no difference in nest success (fledging ≥ 1) between nests in wildland, agricultural, and urban landscapes, although nest success in wildland landscapes appeared more stable than that of nests in agricultural or urban landscapes.  Kestrels nesting in urban landscapes had slightly smaller clutches than their counterparts in ag or wildlands.  We also band and colorband nestling and adult American Kestrels in an effort to monitor movement, and eventually estimate survival in different landscapes.  To date we have banded 1,475 Kestrels in our study area, and colorbanded 634 of these birds.  Nestbox monitoring paired with banding efforts provide important insights into site fidelity, survival, movements, and recruitment of breeders into local populations—this information requires a commitment to long-term monitoring studies.}

\normaltalk{\capitalisewords{The Nidobiome as a Framework to Understand Microbiome Assembly}}{\underline{Felipe Campos-Cerda}, Brendan JM Bohannan}{Animals host entire microbial communities (i.e. microbiome) that influence their anatomy and physiology. The use of molecular techniques have allowed the characterization of these communities, as previous attempts at culturing microbes only recollected 1-10\% of them. In spite of its importance, we are still beginning to understand microbiome assembly. Microbial associations develop early in life and are highly shaped by microbial exposure. The nest is the immediate environment that a bird will face after hatching, making it crucial for microbiome assembly. We propose a new ecological unit where parents, nest and nestlings interact during initial microbiome assembly: the Nidobiome. This concept provides a clear distinction of the interacting elements on microbial assembly and layout their expected roles through development. Following the nidobiome framework, we constructed nest cavities in tree holes and arboreal termite colonies in the tropical rain forest of Southern Mexico to explore if nest substrate was shaping the microbial communities inside the nest. We found that each substrate had a different microbiome profile. Community similarity from both cavity types seemed to converge through time, with termite colonies being more similar between them. Our results suggest that termite colonies provide a regulated microbiome compared to tree holes, implying an ecological advantage for their use as nesting site. Using molecular techniques we were able to characterize the microbiome of two different nest substrates and track population changes of individual microbial taxa through time. This opens a new frontier on the study of nest evolution with microbes as an evolutionary factor.}

\normaltalk{\capitalisewords{Assessing the Relationship Between Multiple Paternity and the Cloacal Microbiome in Female Tree Swallows}}{\underline{Jessica Hernandez}, Joel W McGlothlin, Lisa K Belden, Ignacio T Moore}{Until relatively recently, most avian species were considered to be truly monogamous. We now know that the vast majority of avian species engage in fertile copulations outside the social pair. Socially monogamous females that mate with multiple males could obtain potential fitness benefits including acquisition of good genes for offspring and viable sperm, but also face potential costs such as loss of paternal care or increased harassment from their social partner. The acquisition of sexually transmitted (pathogenic) bacteria has also been suggested to be a cost of multiple mating to females but has yet to be adequately tested. To begin to determine how multiple mating is related to the acquisition of sexually transmitted (pathogenic) bacteria, we studied free-living female tree swallows (Tachycineta bicolor) during the breeding season in southwestern Virginia. We collected cloacal swabs and used 16S rRNA gene amplicon sequencing to characterize the bacterial communities present in the cloacae of female birds. We also took blood samples from nestlings and both social parents, and performed microsatellite marker analyses to assess nestling paternity. We used nestling paternity as a conservative proxy to estimate the frequency and success of extra-pair copulations, and to determine the minimum number of sexual partners per female. By using modern molecular techniques, this study broadens our understanding of the potential costs of engaging in multiple mating by focusing on host-associated bacterial communities.}

\normaltalk{\capitalisewords{Building Social Science Capacity in the Bird Conservation Community: Nationwide Priorities and Practices}}{\underline{Judith Scarl}, Jessica C Barnes, Tammy VerCauteren}{This presentation will describe efforts within the North American Bird Conservation Initiative (NABCI) and other national-level bird teams to integrate human dimensions into bird conservation. These ongoing efforts include (1) the identification of urgent and feasible priority actions to improve the status of bird populations; (2) original social science research to inform public engagement in bird conservation; (3) initiatives to build social science understanding and capacity among bird scientists; and (4) the integration of social science research and knowledge in the planning, design, and delivery of conservation programs for migratory and other birds. This presentation will also include a short exercise in identifying the human dimensions of the socio-ecological systems in which birds are situated.}

\normaltalk{\capitalisewords{Preferences of Birdwatchers and Waterfowl Hunters in the United States and Canada: Results of Discrete Choice Experiments (DCEs)}}{David C Fulton, \underline{Alia Dietsch}, Howie Harshaw, Andy Raedeke, Rudy S , Ashley Dayer, Jennie Duberstein}{We used DCEs across the US and Canada to understand the preferences of waterfowl hunters (n = 8,123 US; n = 2,000 CAN) and birdwatchers (n = 33,071 US; n = 3,837 CAN) for different combinations of hunting and viewing experiences. Choice experiments present hypothetical scenarios to respondents to derive preferences for alternatives composed of multiple resource and management attributes. Alternatives presented in these DCEs included five hunting related attributes: 1) harvest; 2) access effort; 3) length of travel; 4) quantity of waterfowl; and 5) potential for interference/competition. Birdwatching alternatives included seven attributes: 1) diversity; 2) rarity of birds; 3) number of birds; 4) ease of access; 5) wetlands presence; 6) naturalness; and 7) travel distance. In summary, the order of importance of the waterfowl hunting attributes was: 1) potential for interference; 2) length of travel; 3) harvest; 4) quantity of waterfowl; and 5) access effort.  The most important birdwatching attributes were: 1) travel distance; 2) chance to see rare or unusual bird species; 3) the naturalness of the area; and 4) presence of wetlands.   We applied latent class analysis and market simulation techniques to understand the influence of heterogeneity within the respondent sample on choice preferences.}

\normaltalk{\capitalisewords{Minimizing All-Terrain Vehicle Impact in Bird Habitat: Reconciling Harvest and Conservation on Private Land in Western Alaska}}{\underline{William Naneng}, Ryan L Maroney, Myron Naneng, William Tinker}{Some of the highest densities of breeding birds in the world occur near the community of Hooper Bay, in the Yukon-Kuskokwim Delta. These birds and their environment are important food and cultural resources to local indigenous people, who want to maintain these assets as part of key conditions for their community's well-being. However, use of all-terrain vehicles (ATV) to access harvesting areas has caused habitat degradation and disturbance to breeding birds. The Sea Lion Corporation, the largest local private land owner, has worked with its partners to address this issue. These efforts have included: 1) locally-developed outreach products encouraging habitat protection; 2) a field-based outreach program to encourage best ATV use practices; and 3) building a hardened trail as a durable travel surface to direct ATV traffic. Community engagement and compliance with these efforts have been positive, but further work is needed to reinforce newly developed social norms for appropriate ATV use. More proactive habitat management guidelines to reduce disturbance may also be necessary. To successfully reconcile local harvest needs and habitat conservation, community leaders recognize the importance of developing and implementing land management plans rooted in a participatory process and based on local culture and traditions.}

\normaltalk{\capitalisewords{Sustainability in Long-Term Collaborations: The Harvest Assessment Program of the Alaska Migratory Bird Co-Management Council}}{Liliana C Naves, James Fall, \underline{Christopher Tulik}, Mike Pederson, Jacqueline M Keating, Julian Fischer, Patty Schwalenberg}{Harvest data are needed to document uses of biological resources, inform harvest management and resource conservation, manage allocation among user groups, and support sustainable harvest opportunities. Harvest management of migratory birds in Alaska has a complex history linking indigenous users with other users along the Pacific Flyway and beyond. In this talk, we share our experience building and maintaining a collaborative long-term harvest research program including challenges and effective approaches to navigate limited funding, trust issues, statistical robustness, cross-cultural partnerships, stakeholder engagement, and applications of harvest data in bird harvest management and conservation.}

\normaltalk{\capitalisewords{Integrative Approaches to Species Delimitation in Birds}}{\underline{Carla Cicero}, Nicholas A Mason, Rosa A Jimenez, Daniel R Wait, Cynthia Y Wang-Claypool, Rauri CK Bowie}{Species delimitation is a cornerstone of systematic biology and conservation. How species are delimited guides research and affects management decisions. Conversely, methods used to study geographic variation and diversity in populations are fundamental to delineating species boundaries. Historically, species delimitation has focused on phenotypic differences. Alongside the advent of DNA sequencing techniques, molecular variation has been increasingly incorporated into species delimitation decisions. Likewise, ecological niche models have been used recently to infer processes of diversification into different environments. Optimally, rigorous species delimitation should integrate of all of these methods, as well as behavioral study such as song variation and playback experiments, across the geographic range of populations to fully understand the nature of diversification. Furthermore, sampling must incorporate putative or potential contact zones to address how individuals in different environments interact with each other. We illustrate these points with specific examples from birds in western North America, where the physiographic and ecological complexity has resulted in varying levels of differentiation and complex suture zones. In particular, we present data for several species (Cyanocitta stelleri, Pipilo maculatus, Artemisiospiza belli/nevadensis) with especially extensive sampling and integration of diverse suites of characters to address questions of species status. We also discuss examples from recent AOS Checklist proposals where evidence and/or sampling were considered unconvincing for species delimitation, especially in critical areas such as contact zones. Finally, we discuss challenges in conducting species delimitation research, including the time, cost, and permitting required to adequately sample and capture diversification across species boundaries.}

\normaltalk{\capitalisewords{The Genomic Revolution and Species Delimitation in Birds: Gene Flow Matters, but What About Other Evolutionary Forces?}}{\underline{Carlos Daniel Cadena}}{With adoption of the notion that species are population-level lineages and the availability of genetic and genomic data to identify separately evolving populations, researchers often establish species limits based on gene flow or lack thereof. This is explicitly embodied in the increasingly popular use of the multispecies coalescent and similar approaches as criteria for species delimitation. With the increasing amount and quality of genomic data employed to address questions about species limits, however, it has become clear that species readily diagnosable based on phenotypic traits typically used by biologists to distinguish species do not always form distinct genomic clusters. There are two possible interpretations of the emerging pattern of phenotypically distinct yet genomically admixed and thereby seemingly undiagnosable "species". One is that traditional means to identify evolutionary lineages often fail, mistreating within-species phenotypic variation as if it were evidence of species limits; under this view, evidence of gene flow revealed by genomic analyses is evidence of conspecifity. Alternatively, equating species with lineages one can identify using coalescent analysis of genomic variation may be problematic because it fails to recognize the role of other evolutionary forces (i.e. various forms of selection) in defining evolutionary lineages. Using examples from my research on various groups of Neotropical birds, I will argue that genomics will likely not solve many issues in avian species-level taxonomy and will call for the importance of considering evolutionary forces other than gene flow in species delimitation. Such a task requires detailed attention to phenotypes firmly grounded on evolutionary theory.}

\normaltalk{\capitalisewords{Sex Biases in Natural History Collections and Consequences for Bird Trait Studies}}{\underline{Natalie Cooper}}{Natural history collections are treasure troves of historical distributional and phenotypic data, but also have several major biases. One bias that has not been fully assessed is the sex ratio of specimens. Various active (e.g. choosing to collect the most colourful individuals) and passive (e.g. ease of capture) factors, make male bias in collections likely. Here we use catalog data from five major natural history collections of birds. Using a dataset of over 1 million records, representing over 5,000 species, we calculate the proportion of female specimens per species, and how this varies through time and across taxonomic groups. We also investigate correlations with sexual size dimorphism, and ornamentation (e.g. plumage colouration). Our results have implications for global bird trait studies.}

\normaltalk{\capitalisewords{Responses of Bird and Plant Communities to Global Change: Coupled or Decoupled?}}{\underline{Matthias Schleuning}}{Birds and plants are intricately linked in ecological communities, e.g. through mutualistic interactions in pollination and seed-dispersal networks. Although these interactions are ubiquitous in nature and critically important for ecosystem dynamics and functions, interaction-mediated responses to global change are little studied, due to the lack of data and integrative concepts. Here I present how the increasing availability of trait data provides new avenues for quantifying linkages between birds and plants in ecological communities. First, I show how we can identify species traits that determine the matching of species in mutualistic networks and how the functional trait diversity of birds and plants reciprocally impacts their bird and plant partners, respectively. Despite their coupling in mutualistic networks, there is a high risk of decoupled responses of bird and plant communities to global change as bird and plant functional diversity are associated with different abiotic factors. Second, I will propose a trait-based framework that could help to predict the (de)coupled responses of interacting bird and plant species to global change. A refocused collection of trait data and their incorporation into this framework would allow us to model and predict how bird-plant interactions could modulate biodiversity responses to global change.}

\normaltalk{\capitalisewords{How Trait Data Help Improve Avian Biodiversity Forecasts under Global Change}}{\underline{Damaris Zurell}}{Under the burden of increasing biodiversity loss, quantitative predictions of potential species and community response to global change are important for informed species management and policy support. Over the recent years, new biodiversity modelling frameworks have been developed and proposed with the promise of more realistically capturing and predicting niche and range shaping processes. However, the success of such models relies to a large degree on the availability of detailed information, not the least on species' traits related to phenology, morphology, reproduction, physiology and movement. Although birds constitute one of the best-studied species groups worldwide, data integration into large-scale biodiversity models is still poor. In this talk, I briefly outline recent theoretical and technical advances in predictive biodiversity models for single species and species communities. I then provide perspectives how trait data can help to inform these models and their predictions. For example, I will show that incorporating more detailed information about dispersal traits, life-history and phenological traits as well as trophic traits could strongly affect model complexity and future projections. Stronger integration of avian trait data into biodiversity models and more targeted collection of additional trait data will greatly improve our understanding and ability to predict biodiversity response to global change.}

\normaltalk{\capitalisewords{A Comparison Between Traditional and Next Generation Screening Approaches to Characterize Avian Haemosporidian Parasites}}{\underline{Suravi Ray}, Marissa Henry, Spencer Galen, Emily N Ostrow, Janice Dispoto, Lisa Kiziuk, Jason D Weckstein}{Parasites are known to influence the behavior, ecology, and evolution of their avian hosts and many studies of avian parasites have aimed to assess the effects of parasite infection prevalence and intensity on these aspects of host biology.  In particular, avian haemosporidian parasites in the genera Leucocytozoon, Haemoproteus and Plasmodium have been studied for two centuries and have served as models for testing hypotheses about disease and parasite ecology and evolution.  Since the development of nested PCR-based methods for screening avian blood samples, numerous studies have been published on these extremely diverse and widespread avian parasites.  However, Sanger sequencing methods are not ideal for detecting and characterizing mixed infections of haemosporidian parasites, which turns out to be relatively common.  Here we apply a modified approach to the nested PCR screening method that employs a combination of a recently developed qPCR detection assay and cytochrome-b iTru dual indexed amplicon libraries (TaggiMatrix) sequenced using paired end 300 base pair reads on an Illumina MiSeq.  We applied this new amplicon sequencing approach and the commonly used nested PCR and Sanger sequencing approach to a series of DNA extracts from resident and migrant bird blood samples collected at Rushton Woods Preserve banding station in Newtown Square, Pennsylvania to assess patterns of diversity and prevalence of haemospridians infecting birds in this avian community. Comparison of sequenced haemosporidian data with avian ethological data and the improved ability to detect mixed infections could further our understanding of these parasites.}

\normaltalk{\capitalisewords{Spatial Patterns of Habitat Use of the Sierra Madre Sparrow, Mexican and Endemic Mountain Bird, in Milpa Alta, Mexico City}}{\underline{Annamaria Savarino-Drago}, Irene Ruvalcaba-Ortega, Misael Martinez, Her'{o}n Martinez, Ulises Martinez-Molina, Yuridia Martinez-Molina, Sarai Garcia-Loeza, Vicente Rodriguez-Contreras, Gabriel Martinez-Molina}{Understanding movement patterns and home range of species is critical to define conservation strategies, particularly for species at risk of extinction. The Sierra Madre Sparrow (SMS) Xenospiza baileyi is an endangered and Mexican endemic, whose distribution is restricted to a few known localities of highland bunchgrasses.  Research on this species is scarce and limited to the breeding season.  We aimed to define its spatial use of the habitat through home range estimation and habitat characterization during the breeding and post-breeding seasons in a diverse landscape. Between June and September 2017, we color-banded 84 individuals and estimated 13 reproductive territories; and placed 20 radio-transmitters and defined 15 post-breeding home ranges using 95\% probability contour from fixed kernel density estimators. It was previously assumed that its secretive behavior and avoidance of conifer forests would result in limited spatial use of the habitat throughout the year; however, we found that during the post-breading season, it used larger areas (Mean=94,320 m2, SE=26,050) than during breeding season (Mean=766 m2, SE=123).  We also observed that half of the monitored sparrows abandoned grasslands and moved to oatmeal and potato croplands. Sparrows used areas with significantly lower grass cover, dead vegetation and litter cover and shorter grasses (p $<$0.05), than available. This research provides key information to be considered when proposing conservation and management strategies for the species and its habitat.}

\normaltalk{\capitalisewords{Volunteer-Led Surveys Document 40-Years of Shorebird Declines Across North America}}{Paul A Smith, Adam C Smith, Stephen Brown, Christian Friis, Julie Paquet, Brad Winn, \underline{Amelia MacDonald}}{Shorebirds have been monitored annually at migratory stopover locations since 1974 as part of the International Shorebird Survey, the Atlantic Canada Shorebird Survey and the Ontario Shorebird Survey.  With over 100 million shorebirds counted during over 100,000 surveys, these largely volunteer-based programs offer the most comprehensive source of trend information for shorebirds breeding in North America.  Here, we report the results of a recent analysis of data from 1974-2017 using hierarchical Bayesian models.  These results suggest continued declines of shorebirds, placing them alongside aerial insectivores and grassland birds as a taxon of conservation concern in North America.  The number and severity of declines is greater for Arctic breeding shorebirds than for temperate breeders.  The recent analysis also sheds light on finer-scale geographic patterns in trend.  Population trends were more negative in Atlantic Canada and Ontario than along the Atlantic and Gulf Coasts of the United States, but these regional patterns are complicated by differences among species.  Moreover, regional patterns could reflect long-term shifts in migration behaviour, especially temporal changes in routes and rates of migration.  While time-series of data to assess behavioural changes such as these do not yet exist, recent advances in tracking technology now offer the promise of such assessments in the future.  We briefly review some of the ongoing efforts to track migrant shorebirds and discuss how these will improve future assessments of status from migration monitoring data.}

\normaltalk{\capitalisewords{Antibiotic Resistance Dissemination Through Wild Birds}}{\underline{Jonas Bonnedahl}}{There are estimates that deaths worldwide caused by antibiotic resistant bacteria (ARB) will exceed 10 million in 2050. The mechanisms of ARB dispersal are not limited to health care or human settings and the spread and maintenance of ARB in natural environments are gaining more and more attention. Wild birds, and in particular gulls, often carry a plethora of clinically relevant ARB often with prevalence rates that exceed those in humans from the same geographical area, including resistance to last line antibiotics as the carbapenems.  Some studies show that there are more genetic similarities between ARB isolated from humans and wild birds than between wild birds and farm animals. This suggests that ARB in wild birds are mirroring environmental pollution from humans rather than from farm animals. This support that gulls can function as reservoirs and bio-indicators of human associated antibiotic resistance genes in the environment.  The spread of ARB through long distance migration are probably limited but regular short distance movements of gulls between wastewater treatment plants and garbage dumps to farmed fields and recreational areas could pose a risk of ARB transmission to humans with birds as vectors.  These findings highlight the need for standardized long-term ARB-screening of selected bird populations by an international network of ecologists and microbiologists. Such a network can produce baseline data to inform and minimize the risk of exposing wildlife to human waste/sewage and prevent further contamination and dissemination of ARB by identifying important point sources and routes of antibiotic resistance dissemination.}

\normaltalk{\capitalisewords{Investigating the Influence of Polarized Light and Artificial Night Lighting on Bird-Building Collisions}}{\underline{Sirena Lao}, Abigail W Anderson, Robert B Blair, Joanna W Eckles, Bruce A Robertson, Reed J Turner, Scott R Loss}{Collisions with buildings annually kill hundreds of millions of birds in the United States. Bird-building collisions are primarily associated with windows: during the day, birds often fail to perceive glass as a barrier when it is transparent or reflects surrounding habitat, and at night, birds are thought to be attracted to artificial light emitted from and near windows. However, birds perceive and respond to light differently than humans, and some aspects of avian vision are poorly understood, including how bird responses to different types of light influence building collisions. Some evidence suggests birds can detect polarized light, which may serve as a cue to assist with migration orientation and/or detect bodies of water. Dark, reflective surfaces, including glass, reflect high degrees of polarized light, causing polarized light pollution. However, no studies have analyzed whether there is a relationship between bird collisions and artificially polarized light reflected from buildings. Additionally, while night lighting is frequently implicated as a major contributing factor to bird-building collisions, few studies have formally analyzed the relationship between collisions and artificial night lightings. We analyzed both types of light pollution—polarized light pollution and artificial night lighting—and their relationship with collisions at different building facades in downtown Minneapolis, Minnesota, USA. Our results indicate that large expanses of glass area ultimately are the most important factor influencing collisions. However, designing new glass surfaces and retrofitting existing glass in ways that reduce both types of light pollution should contribute to significantly reducing bird-building collisions.}

\normaltalk{\capitalisewords{Trends in Abundance and Habitat Associations of Forest Birds on Southern National Forests 1992-2017}}{\underline{Elizabeth A Matseur}, Thomas W Bonnot, Frank R Thompson, Micah G Thorning}{The Southern Region of the USDA Forest Service adopted a migrant and resident landbird conservation strategy in 1992 to address conservation concerns arising from long-term population declines in many birds. The strategy implemented a region wide program to improve monitoring, research, and management of avian populations and their habitats. Our objective was to analyze 25 years of monitoring data collected by this program to assess population trends and habitat associations. Ten-minute point counts were conducted from early-April to late-June across 15 national forests from 1992–2017. There were 82,367 point counts completed and 992,287 birds detected across 15 national forests and 25 years. We modeled abundance and population trends of 152 individual bird species at points over time, by forest, using time-removal models within a hierarchical Bayesian model framework. We were able to fit models for 117 – 58 species per national forest. We plotted abundance over time and estimated trends for species on each national forest. Seventy-five species had a majority of positive annual trends and 68 species had a majority of negative annual trends across all national forests. Northern Parula, Yellow-throated Vireo, Black-throated Green Warbler, Blue-gray Gnatcatcher, and Blue-headed Vireo were among the species that frequently had large positive trends and American Crow, Blue Jay, Summer Tanager, Northern Bobwhite, and Kentucky Warbler frequently had large negative trends. Future analyses will relate abundance to major vegetation types and successional stages. This monitoring will help inform management and conservation for migrant and resident birds in the Southern Region.}

\normaltalk{\capitalisewords{Bias in Estimated Breeding-Bird Abundance from Individual Movement}}{\underline{Frank A Fogarty}, Erica Fleishman}{The closure assumption of detection-based abundance models, that individual animals are present throughout the survey season, is violated in many field studies of breeding birds. The effects of these violations on estimates of abundance and their associations with covariates are not well understood. We used simulations to explore how violations, in the form of movement of individual birds during the breeding season, biased modeled estimates of abundance and environmental covariates. We also collected point-count data on bird species in the Great Basin to test whether within-season movement is frequent and whether the magnitude of closure-assumption violations in our simulations was similar to that in field data. Analysis of our field data suggested that the magnitudes of assumption violations tested in our simulations (movement equal to 5-50\% of abundance) were realistic. Individual movement (availability) strongly influenced detection probability of nearly all examined species. Availability of 11 of 27 species was temporally heterogeneous, suggesting that closure assumption violations are common in our study system. In our simulations, even relatively small violations of the closure assumption biased abundance estimates by $>$10\% in most scenarios. Larger violations induced biases $>$100\%. Estimates of the association between abundance and an environmental covariate were less biased in most scenarios. Our results suggest that within-breeding season movement is common for individual birds in the Great Basin, that the magnitude of this movement likely biases estimates of abundance that do not account for movement, and that common assumptions about season-long occupancy of breeding territories in birds are overly simplistic.}

\normaltalk{\capitalisewords{Population Trends and Habitat Associations of Conservation Priority Bird Species in Iowa}}{\underline{Rachel A Vanausdall}, Stephen J Dinsmore, Karen E Kinkead, Paul W Frese}{Accurately estimating population trends of avian Species of Greatest Conservation Need (SGCN) and determining the effects of habitat variables on these changes are important steps to informing the management and conservation of avian species. Using data collected between 2015 and 2018 from Iowa's Multiple Species and Inventory Monitoring (MSIM) program we estimated the population trends and detection probabilities of nine breeding birds identified as SGCN in Iowa across 49 public properties. We used a hierarchical modeling approach (N-mixture) to estimate abundance trends for each species and to assess the influence of habitat variables on abundance. We completed a total of 5,423 surveys, and between 8\% and 34\% of sites were occupied by these species. Several species did not show a significant change in abundance over time, but we found that the Sedge Wren (β=0.35, SD=0.03), Bobolink (β=0.11, SD=0.05), and Field Sparrow (β=0.25, SD=0.05) showed a positive trend in abundance. The Grasshopper Sparrow (β=-0.21, SD=0.07) and Northern Flicker (β=-0.28, SD=0.07) showed a negative trend. Several habitat covariates influenced abundance, and for species showing changes in abundance these covariates included vegetation height, litter depth, and presence of trees or shrubs. We hope these models can be used to assess potential trends of these species across other areas throughout the Midwest and help identify specific management and conservation needs.}

\normaltalk{\capitalisewords{Manipulated Sex Ratios Alter Group Structure and Cooperation in the Brown-Headed Nuthatch}}{\underline{James A Cox}, Jessica A Cusick, Emily H DuVal}{A biased adult sex ratio (ASR) can influence cooperative breeding behavior if the bias limits mating opportunities for the more abundant sex. We tested predictions associated with the ASR-cooperation hypothesis in the Brown-headed Nuthatch (Sitta pusilla).  We manipulated ASR by cross-fostering known-sex nestlings within 2 large (≥100 ha) experimental plots for 5 years using a crossover design where each plot received an opposing male- or female-biased treatment for 2 consecutive years.  A year with no manipulations followed before the bias was reversed on each plot for 2 additional years.  Variation in ASR (adult males/total adults) was pronounced compared to background proportions (0.55) and ranged from a female bias in female-biased plots (0.47) to a strong male bias in male-biased plots (0.71).  Sex ratios during the post-breeding period ranged more broadly (0.33 in female-biased plots vs. 0.74 in male-biased plots).  Territory densities did not change significantly and allowed six predictions to be assessed.  Consistent with predictions, the prevalence of cooperative breeding groups doubled under male-biased treatments and large cooperative groups appeared (≥ 2 male helpers vs. the single male helper most common prior to the experiment). These changes occurred despite increased dispersal of cross-fostered males in male-biased plots.  Most juvenile females dispersed, but, consistent with predictions, the prevalence of female helpers increased under female-biased treatments.  Manipulations did not alter the sex of nestlings produced nor extend the time that males served as helpers. Taken collectively, results support the ASR-cooperation hypothesis and the role that mate limitations play in cooperative breeding behavior.}

\normaltalk{\capitalisewords{Deep Snow and Dense Cover Influence Habitat Selection in a Winter-Adapted Bird}}{\underline{Amy A Shipley}, Benjamin Zuckerberg}{Birds in temperate northern regions employ a variety of strategies to cope with the energetic demands of winter. The ability to select habitat that provides suitable protection from the cold is important, especially when key habitat features are dynamic throughout the winter season.  Previously, we found that deep snow, which is thermally insulative, mediates the negative effect of cold temperatures on physiological stress in a winter-adapted bird. However, it is unknown whether snow depth varies spatially across areas available to individual birds, and whether birds actively select habitats with deeper snow. We conducted a fine scale microclimate study of 151 roost sites used by Ruffed Grouse (Bonasa umbellus) during the winter of 2015-2016. We measured snow depth at roost sites, and interpolated snow depth and minimum daily temperature based on data collected from 58 weather stations across our study site. We used a resource selection function to compare environmental conditions and forest cover types between grouse roost sites and unused but available locations within a buffer centered on each roost site. We found that grouse strongly selected for areas with deep snow, and preferred dense cover over open areas and mature forest. Temperature did not appear to influence roost site selection.  This strong preference for deep snow highlights its importance, but loss of snow cover due to climate change may limit the ability of winter-adapted species to access this spatially and temporally dynamic refuge.}

\normaltalk{\capitalisewords{Dominance Rank and Sex Predict Social Network Position in Winter Social Groups in a Passerine Bird}}{\underline{Theadora A Block}, Daizaburo Shizuka, Alexis Chaine, Bruce E Lyon}{Social group structure is more complex than simply to 'group or not group', with many different potential types of social cohesion.  Organisms have groups ranging from a few individuals to many, and the types of associations range from stable, long-term associations to fluid short-term interactions.  Social network analysis techniques are now revealing the intricate and complex patterns of social organization in many organisms. Individuals in social groups often vary in the nature of their social relationships, but the drivers of this variation are poorly understood. Here, we ask if behavioral and morphological traits in golden-crowned sparrows (Zonotrichia atricapilla) can predict individual variation in social relationships, assessed using social network metrics. These sparrows form complex groups in winter with stable communities, high local philopatry and strong individual associations lasting multiple years. We assessed the relationship between several key behavioral traits—personality, sex, dominance rank, and plumage badge —and various social network metrics, including number of associations (degree), the strength of these associations (edge strength) and the variation of the edge strength (edge heterogeneity).  Dominance rank and sex predict a bird's social network measures. Specifically, females and dominant birds tended to have higher degree and edge strengths, while only dominance predicted edge heterogeneity. Our results suggest that birds with different dominance levels may be adopting alternate social strategies, with dominant birds potentially needing to interact with a greater number of birds to maintain their dominance status.}

\normaltalk{\capitalisewords{Investigating the Role of Avian Vocalizations as 'Magic Traits' using on-Bird Sensors}}{\underline{Christopher A Tarango}, T S Sillett, Anthony J Gomez, John W Fitzpatrick}{What mechanisms underlie ecological speciation? The homogenizing effects of gene flow have long been hypothesized to overcome local adaptation and prevent speciation in sympatry. However, recent evidence suggests that for so-called 'magic traits', adaptive divergences, driven by ecological selection, are reinforced by associated divergences in mating signals, resulting in a clustering of genotypes in continuous landscapes. Yet there remains a lack of direct evidence of sexual or social selection acting on these divergent signals. Here, we investigate a proposed magic trait, the female-only rattle call of the Island Scrub-Jay (Aphelocoma insularis), an island-endemic species that exhibits microgeographic adaptations in bill morphology across two habitat types. By coupling remote and on-bird audio recorders with direct observations, we aim to understand how correlated, population-specific variation in bill morphology and bioacoustic features of the rattle call impact mate choice. In a first step, we describe and quantify the context and individual repeatability of rattle call features and their variation. Furthermore, we describe the benefits and challenges of using innovative on-animal sensors in the wild and discuss the broader application of this potential major advancement in ethological methodology.}

\normaltalk{\capitalisewords{Impacts of the December 2016 to August 2017 Bogoslof Island Eruption Event on a Major Alaskan Seabird Colony}}{\underline{Nora A Rojek}, Gary S Drew}{Bogoslof Island, managed by the Alaska Maritime National Wildlife Refuge, is an important breeding site for seabirds and marine mammals in the southeastern Bering Sea. Importantly, it includes one of only four major breeding sites for red-legged kittiwakes (Rissa brevirostris). Periodic eruptions of Bogoslof volcano result in a changing landscape and alters seabird nesting habitat. The most recent event of ≥64 eruptions over eight months increased island area by 300\% and buried vegetation and soil. Continuous eruptive activity in 2017 likely prevented seabirds from successfully breeding that season. In August 2018, a site visit detected the presence of most seabird species known to previously occupy the island but with little evidence of successful breeding except by glaucous-winged gulls (Larus glaucescens). Nest building species, such as kittiwakes (Rissa spp.) and cormorants (Phalacrocorax spp.), lost access to nest materials. The main pre-eruption breeding area for murres (Uria spp.) was altered and only a small number of eggs and chicks were observed. Tufted puffins (Fratercula cirrhata), which typically rely on soil and vegetation for burrows, lost all nesting habitat; birds attempted to dig new burrows but could not excavate far into hard surface layers and unconsolidated subsurface layers. Because most long-lived seabirds exhibit colony site fidelity, we expected them to return with limited impact on population numbers.  However, given that no fine ash was deposited during the eruptions, soil development is expected to be prolonged. Species relying on vegetation for nests and soil for burrows may be constrained from breeding until soil develops.}

\normaltalk{\capitalisewords{The Role of eBird in Fine-Scale Distribution and Density Modeling}}{\underline{Tyler A Hallman}, W Douglas Robinson}{The eBird database is a rich source of information on avian occurrence and abundance. The inclusive nature of eBird has brought in more than 400,000 contributors and 32 million checklists, but the quality of eBird data for complex fine-scale analyses remains unclear. Independent of data quality, eBird surveys do not employ the methods necessary to account for imperfect detection. In this study, we compared eBird data from two Oregon counties to extensive avian survey data from the Oregon 2020 project gathered by professionals using modern counting protocols. For each dataset, we used boosted regression trees to build species distribution models for sixteen species of songbird. We used the professional survey data to estimate detection probability within zero-inflated boosted regression tree models of spatially explicit density. Finally, we used the results of density models to estimate population. We found that for most species, distribution models were similar between the two datasets. When eBird data were adjusted to account for Oregon 2020 modeled imperfect detection, population estimates were similar between datasets for most species. Despite similar population estimates, spatially explicit densities differed greatly between datasets for most species. Density models built on eBird data frequently failed to stabilize and environment-abundance relationships remained unclear. Our results show that the quality of eBird occurrence and abundance data are not equivalent. Research into how best to analyze eBird abundance data at a fine-scale is much needed.}

\normaltalk{\capitalisewords{Birds Across Borders: Uniting Curriculum, Students, Cultures, and Generations}}{\underline{Jennifer A Campbell-Smith}, Gregg Cannady}{Birds do not recognize human political and cultural borders, they migrate freely, interacting with individuals of their own and other species. Taking this example from birds, we are building a K-12 program to link students across the globe using birds as a common focus and passion-point. The K-12 model of teaching is improving, with schools moving in the direction of hands-on, problem based learning, character development, global citizenship, and a push to unite and interlink curriculum. The Birds Across Borders program is a student-driven initiative collaborating with schools, community centers, local governments, creative studios, independent researchers, museums, and other organizations to empower students around the world to connect with one another, professionals, and their environment. Our initiative proposes to use studies of avian form, function, ecology, behavior, and the roles they play in human art, writing, and history represent a unique, highly accessible way to unite all curriculum at all levels. For example, while reading about ravens in Native American and Norse mythology in English or history class, they will learn about the bird's intelligence and behavior, or the physics of flight, in science class, and compare what they've learned and created with others, outside their school. Ornithology provides a unique opportunity for us to teach real-world skills and help our youth form valuable, world-changing connections. In this talk, I will outline the Birds Across Borders project initiatives, our vision, what we've already accomplished, and how the broader ornithological community can get involved in this avian-based educational innovation.}

\normaltalk{\capitalisewords{Climate, Storm-Petrels, and Chick Growth}}{\underline{Robert A Mauck}, Charles E Huntington, Mark F Haussmann, Donald C Dearborn, Robert E Ricklefs}{Climate change in the form of rising temperatures imposes both direct and indirect effects on avian reproduction. Pelagic seabirds, such as Leach's storm-petrels, breed on land and forage at sea and are, therefore, susceptible to change in both local air temperatures (LAT) and regional sea surface temperatures (SST). Variation in air temperature at the nesting island has direct effects on thermoregulation in developing chicks. Variation in SST has a myriad of indirect effects manifested in the ability of seabird parents to deliver food to chicks. We monitored chick growth as a proxy for both direct and indirect effects of variation in temperature on storm-petrel reproduction across 12 breeding seasons spanning a 27-year period (1983-2010) during which Mean Annual Global Temperature (GMT) increased by more than 0.5°C. To index chick growth, we measured wing growth rate during the linear phase of wing growth in this species. We used a multi-step modeling approach to examine chick growth as a function SST, LAT, AGT, as well as the North Atlantic Oscillation (NAO). GMT had a strong quadratic effect on wing growth rate. LAT showed a steep increase in growth rate with increasing temperature and declined somewhat at higher temperatures. SST on the foraging grounds had a weak negative effect. We detected no effect of NAO on chick growth. These results agree with previous findings on the effect of GMT on hatching success in this species, with somewhat stronger evidence for the direct effect of terrestrial air temperatures on storm-petrel reproductive success.}

\normaltalk{\capitalisewords{The Endangered Black-Capped Vireo (Vireo Atricapilla): Status after 30 Years of Recovery Management in Oklahoma, USA}}{\underline{Joseph A Grzybowski}, Daniel T McDonald}{The Black-capped Vireo (Vireo atricapilla) was listed as endangered by the U.S. Fish Wildlife Service in 1987.  Threats were loss of early-succession scrub habitat largely to fire-suppression, and brood parasitism by Brown-headed Cowbirds (Molothrus ater).  Searches in the mid-1980's estimated fewer than 80 pairs in their historic range in Oklahoma, U.S.A.  Management efforts (vegetative and cowbird control) were undertaken on public and private lands.  Efforts on public lands of the Wichita Mountains increased populations from some 50-70 pair in 1987-88 to a peak estimate of $>$5000 by 2010.  Recent declines from a drought-cycle and erratic seasonal weather patterns has occurred.  On private lands, efforts to increase numbers in small groups of $<$20 pairs had limited success, with mostly local extirpation occurring.   Two isolated small clusters persist in Oklahoma of $<$ 30 pairs each.  The histories and circumstances for these groups will be contrasted.   Isolated individuals were seldom recorded.  The large increase in numbers of vireos on public lands in Oklahoma and a few other sites elsewhere has justified their delisting as endangered, a pragmatic success story.}

\normaltalk{\capitalisewords{Life on the Edge: Density and Demography of Marsh Wrens and Seaside Sparrows in Northeastern Florida}}{Amy C Schwarzer, \underline{W A Cox}}{The Worthington's Marsh Wren (Cistothorus palustris griseus) and MacGillivary's Seaside Sparrow (Ammospiza maritima macgillivraii) are saltmarsh specialists that have experienced severe range contractions and population declines in Florida over the past 50 years.  We studied habitat selection and demographic rates of both species during 2014–2017 in northeastern Florida saltmarshes dominated by tall spartina (Spartina alterniflora) and black needlerush (Juncus roemarianus) to better understand factors contributing to population declines.  Nest survival for both species was low (wrens: 0.10, 95\%CI: 0.04–0.18; sparrows: 0.03, 95\%CI: 0.00–0.10), with high tides associated with the lowest survival rates.  We observed little evidence of direct nest loss via flooding and it appears high tides may concentrate predators in nesting habitat.  Marsh wren post-fledging survival was also low (0.10, 95\%CI: 0.01–0.37) which when combined with nest survival estimates for the species suggests low overall productivity for the population.  Nevertheless, the density and range of occupancy for both species were similar to those reported for the same population in 2000–2001, which indicates that either adult survival for these nonmigratory populations is high, or that immigration from outside the study area is substantial.  Tall spartina was particularly important to the birds: wren occupancy, abundance, nest-site selection, fledgling habitat selection, and post-fledging survival were all positively associated with tall spartina, as was sparrow abundance and nest-site selection.  Our results suggest that protection and creation of saltmarsh with a high proportion of tall spartina will benefit both species.}

\normaltalk{\capitalisewords{Patterns of Northern Bobwhite Juvenile Survival on Extensive Prairies and Mixed Agricultural, Grassland, and Wooded Landscapes in Missouri}}{\underline{Emily A Sinnott}, Frank R Thompson, Mitch D Weegman, Tom Thompson}{Northern bobwhite are a short-lived species whose populations have exhibited long-term declines.  Bobwhite population growth is sensitive to juvenile survival, yet very little is known about brood demographics.  We determined habitat factors related to juvenile survival from hatch to 90 days of age on five conservation areas managed traditionally (i.e., encouraging woody edge and strip crops) or extensively (i.e., patch-burn-grazing to maintain remnant native grasslands).  We analyzed fates of 429 individuals from 46 broods on 2 traditionally-managed areas and 3 extensively-managed native prairies in 2017 and 2018.  We tracked broods and individual radio-tagged juveniles daily and recorded cover type at each location as native grassland, mixed- or cool-season grasslands, agriculture, or woodlands.  We examined the influence of age and cover type on survival using an information-theoretic approach.  The best-supported model included age and cover type effects. Daily survival rates increased with age and were significantly higher on native prairies compared to mixed- and cool-season grasslands, agricultural fields and woodlands.  Ninety-day interval survival (+SE) was 0.37 (0.13), 0.11 (0.08), 0.12 (0.06), 0.07 (0.05) in native grassland, mixed grasslands, agricultural fields, and woodlands, respectively.  Extensively-managed native prairies may improve juvenile survival and productivity over traditionally-managed mixed-grass pastures and agricultural crops, thereby contributing to bobwhite population stability. Future analyses will examine more detailed relationships between survival and landscape composition.}

\normaltalk{\capitalisewords{Where in the World are Migratory Birds Declining During the Annual Cycle? the Winter Survival of a Near Threatened Shorebird in South America}}{\underline{John A Herbert}, David Mizrahi, Caz Taylor}{Neotropical migratory shorebirds have been declining over the past 40 years, yet significant knowledge gaps exist throughout their annual cycle, particularly during the winter in South America. Wintering semipalmated sandpiper (Calidris pusilla) populations have decreased by 80\% since 1982, and they are now considered a near threatened species. The winter survival of semipalmated sandpipers is virtually unknown, but a current hypothesis is that low survivability during the winter is contributing to their overall decline. In this study, we are using autonomous tracking systems (Motus towers) in conjunction with uniquely coded transmitters (nanotags) to measure the winter survival of semipalmated sandpipers. From 2016-18, we deployed 315 nanotags onto semipalmated sandpipers, and installed Motus towers at seven sites in Suriname, French Guiana and Brazil. For individuals tagged in 2016 at three sites in Suriname, we found similar apparent winter survival, and higher survival among adults (after hatch year) than juveniles (hatch year). This is the first study to investigate the winter survival of semipalmated sandpipers, and the first study to use nanotags to investigate winter survival of shorebirds in South America. Our findings will increase the understanding of the semipalmated sandpiper annual cycle and improve conservation efforts for this species and other neotropical migratory shorebirds.}

\normaltalk{\capitalisewords{When Prioritizing Regions for Migratory Bird Habitat Conservation, Quantifying Costs can Supersede Parametrizing a Population Model}}{\underline{Richard A Stanton Jr}, Conor P McGowan, Orin J Robinson}{How should we allocate habitat improvements for migratory species when we do not know which locations and periods govern population dynamics? Conservationists strive to resolve this uncertainty to make better decisions. However, costs constrain conservation actions and are difficult to estimate, so quantifying costs may improve the quality of conservation decisions. Conservationists direct little attention to this uncertainty. We examined three habitat allocation scenarios (breeding regions, wintering grounds, or range-wide) for a migratory duck using value-of-information analyses to compare the benefits of perfecting 1) the parametrization of a full-annual-cycle population model, 2) cost estimates, and 3) weights on different management objectives (maximize abundance and harvest while minimizing costs). Simulating 50,000 alternative parametrizations of the full-annual-cycle population model, 100-fold variation in relative costs, and all possible objectives weightings, we learned that investing in the breeding grounds had the highest expected value under most parameterizations, costs, and objective weights. Further, the best decision was most sensitive to costs yet nearly invariant across the full range of population model parametrizations. Perfectly quantifying costs would improve performance by 21\% but resolving parametric uncertainty in the population model and objectives weights on costs would each improve performance by $<$1\%. Our findings show that getting precise cost estimates will best inform habitat improvement for this species and illustrate that resolving costs can have more value than resolving ecological uncertainty for a common class of conservation decision.}

\normaltalk{\capitalisewords{Road Mortality in Barn Owls: Identifying Temporal and Spatial Hotspots in the Fraser Valley of British Columbia}}{\underline{Elizabeth A Krebs}, Sofi R Hindmarch, Gerry Powers}{For species intersecting with human impacted landscapes, the relative importance of direct mortality effects versus the indirect loss or degradation of habitat are poorly understood.  Recent work suggests that direct mortality of birds in human-altered landscapes has been greatly underestimated and can lead to local population effects. For example in Canada, collisions with buildings, transmission lines and vehicles are estimated to kill more than 30 million birds, and represent three of the four top sources of human related mortality.  To better understand the magnitude and mitigations for road mortality, we examine the impacts on Barn Owls (Tyto alba), a Threatened Species under the Species at Risk Act, in Western Canada. Barn owls are especially vulnerable to road mortality because they co-occur in agricultural landscapes and use grassy habitats to hunt their favoured prey, voles. To evaluate the factors predicting road mortality, we analyse 21 years (1998-2018) of systematic road surveys across a 60 km section of highway in the Fraser Valley of British Columbia. Overall, 0.64 Barn Owls were killed/year/km of highway, with road mortality increasing in the cold months .  Barn owls were more frequently killed in areas with highway-associated grass, such as grassy verges, rather than in areas containing adjacent suitable habitat .  It is not clear if highway mortality influenced the distribution of breeding pairs, the productivity of barn owls in the area was not correlated to land use or distance to highway s.}

\normaltalk{\capitalisewords{Haemosporidians Show Seasonal Patterns but Little Host Specificity Within a Chickadee Hybrid Zone}}{\underline{Ari A Rice}, Robert L Curry, Jason D Weckstein}{Haemosporidians cause malaria and malaria-like diseases in birds throughout the world, though little is known about their distributions, effects on wild bird populations, and levels of host specificity. Avian hybrid zones present unique opportunities for studying how these parasites transfer between closely-related hosts, how their pathogenicity differs between hosts, and how parasites may influence the breakdown of species barriers among certain birds. Black-capped and Carolina chickadees form one such hybrid zone in Southeastern Pennsylvania, and our ongoing study has focused on whether these species harbor different haemosporidians, whether hybrid chickadees are disproportionately infected, and whether parasite communities have changed in accordance with recent chickadee range shifts. We tested 724 chickadees for parasites across four sites supporting Black-capped, Carolina, and mixed populations that included hybrids. We detected 30 haemosporidian lineages, with two lineages unique to Black-caps; the proportion of infected birds was highest in the center of the hybrid zone. Additionally, an unexpected seasonal pattern among Haemoproteus infections emerged: across all sites, Haemoproteus was absent in samples collected from Apr-Jun (n = 504) but present from Sep-Feb (n = 220). This difference suggests that this parasite's insect vectors may not emerge until mid-late summer, and that chickadees may rid themselves of detectable infections by spring. Moreover, these patterns suggest that some haemosporidians may be underrepresented in studies that exclusively sample bird populations during breeding seasons. Thus, our study reveals complex trends among chickadee haemosporidians and highlights the importance of year-round sampling when assessing entire parasite communities within bird species.}

\normaltalk{\capitalisewords{Demographic and Seasonal Variation in Biofilm Consumption by Migratory Western Sandpipers (Calidris Mauri)}}{\underline{Laurie A Hall}, Susan EW De La Cruz, John Y Takekawa}{Migratory shorebirds rely on adequate prey resources at stopover sites to maintain their body condition during migration and prepare for breeding. As their primary prey become depleted, shorebirds may utilize alternative sources to meet their nutritional demands. We quantified the contribution of biofilm to the diets of Western Sandpipers (Calidris mauri) foraging on a tidal mudflat in San Francisco Bay, CA during mid-winter (January/February) and at the initiation of northward migration (mid-April) when their invertebrate prey may become depleted. Stable isotopes of δ13C and δ15N measured in Western Sandpiper plasma during mid-winter did not differ significantly between sexes (p=0.11 and 0.20 for δ13C and δ15N, respectively) or between juveniles and adults (p=0.53 and 0.70 for δ13C and δ15N, respectively). However, in mid-April, δ15N values were significantly more depleted in juveniles compared to adults suggesting that juveniles may have consumed a greater proportion of biofilm (δ15N mean ± standard error: Juveniles=19.57±0.24, Adults=20.40±0.10, p=0.04). Stable isotope mixing models that incorporated δ13C and δ15N measured in Western Sandpiper plasma, invertebrates (Crustacea, Mollusca, Oligochaeta, and Polychaeta), and biofilm indicated that consumption of biofilm generally increased in mid-April relative to mid-winter, especially in juveniles. Further, mixing models indicated that biofilm composed greater than 25\% of the total diet of three juveniles captured in mid-April. Our results align with previous studies suggesting that migratory Western Sandpipers utilize biofilm as a supplemental food source. Thus, future efforts to understand seasonal dynamics and enhance biofilm production at stopover sites along the Pacific Flyway could benefit this species.}

\normaltalk{\capitalisewords{Comparative Genomic Structure Within Alaskan Galliforms}}{\underline{Sarah A Sonsthagen}, Robert E Wilson, Richard A Merizon, Sandra L Talbot}{In Alaska, resident avian species such as ptarmigan and grouse possess unique ecological traits and likely corresponding genomic variation, allowing them to thrive year-round in Arctic and sub-Arctic ecosystems.  Specialization to synergistic effects of thermal and day-light regimes typified by the Arctic may have resulted in narrow response windows in resident species, potentially rendering them more (or differentially) impacted than migratory species to environmental change.  We collected reduced representation genomic (ddRAD) data for ptarmigan (rock, willow, white-tailed) and grouse species (spruce, ruff, and sharp-tailed) sampled along a latitudinal gradient (60–70°N) within Alaska.  Our goals were to examine levels of genomic diversity and gene flow, and simultaneously identify loci in that may signal local adaptation to Arctic environments.  Genomic diversity within Alaskan galliforms is arrayed differentially: weak to no genetic structure in rock ptarmigan (ΦST $<$ 0.024; $<$ 0.009 within subspecies) and sharp-tailed grouse (ΦST $<$ 0.012); isolation by distance in willow (ΦST $<$ 0.064) and white-tailed (ΦST $<$ 0.025) ptarmigan; and northern and southern clusters in spruce (ΦST $<$ 0.071, with substructure within clusters) and ruff (ΦST $<$ 0.042) grouse.  Patterns of genomic diversity appear to be influenced by topographic features (e.g., mountain ranges), though long distance dispersal events were detected.  Our comparative approach across closely related species and different ecoregions allowed us to uncover unique and common evolutionary patterns that correspond to extreme environments and that would not have been evident if focusing on a single species.}

\normaltalk{\capitalisewords{Innovations in Measuring Habitat: Nest-Site Selection of Grassland Songbirds in Northern Mixed-Grass Prairies}}{\underline{Nicole A Guido}, Maureen D Correll, Katharine Ruskin, Brian J Olsen, Jacy Bernath-Plaisted}{To effectively conserve grassland bird populations that are currently in decline, we must understand habitat selection and its relationship with reproductive success.  Typically, habitat selection for nest sites (e.g., vegetative composition, height) is measured on the ground.  This requires many boots on the ground, driving up the expense of research and disturbance to the birds.  Unmanned aircraft systems (UASs), or drones, can be used remotely and inexpensively to gather data on grassland habitats over large areas. However, few studies have assessed whether drones can characterize grassland habitat in a way that is meaningful for habitat selection in birds. We combined drone-collected and ground-collected habitat data at 343 nest sites to examine nest-site selection in four grassland bird species in the mixed-grass prairies of the Northern Great Plains: Baird's Sparrow, Grasshopper Sparrow, Sprague's Pipit, and Chestnut-collared Longspur. We calculated the normalized difference vegetation index (NDVI) to quantify vegetative biomass, a potentially important factor driving nest-site selection that is not easily estimated from ground measurements. We found nest partitioning between the species along a gradient of NDVI values.  Sprague's Pipits were disproportionately found in low vegetative biomass and NDVI values, while Baird's Sparrows were found in sites with higher values. The predictive ability of NDVI was similar to ground-collected vegetation data. Our results thus support that drone-collected spectral data represents habitat conditions representative of ground-collected data and can assess habitat that is important for nest-site selection by grassland birds.}

\normaltalk{\capitalisewords{Woodpecker Nest Survival and Density in Relation to a Pine Beetle Outbreak}}{\underline{Victoria A Saab}, Quresh Latif, Matthew Dresser, Jonathan Dudley}{Mountain pine beetle (Dendroctonus ponderosae; MPB) outbreaks in western North American coniferous forests are increasing in size and severity. An understanding of wildlife population responses to MPB outbreaks is needed to inform habitat conservation strategies. We monitored 355 nests of 5 woodpecker species before and after a MPB outbreak in dry mixed conifer forest of Montana, U.S.A, over an 11-year period (2003–2006 before MPB; 2009–2014 after MPB). Three of 5 woodpecker species represented the beetle-foraging group; American three-toed woodpecker (Picoides dorsalis) and 2 Dryobates spp., Hairy (D. villosus) and Downy (D. pubescens) woodpeckers. The other 2 species studied were Northern Flicker (Colaptes auratus), a foraging/habitat generalist, and Red-naped Sapsucker (Sphyrapicus nuchalis), a sap forager and bark-gleaning insectivore.  We analyzed daily survival rate (DSR) of nests in relation to MPB outbreak severity and timing, along with covariates unrelated to the outbreak (temperature, nest height, and nest tree diameter). We found stronger evidence for nest survival relationships with the non-outbreak variables, identified in previous studies as important to woodpecker nest survival. Our results indicated limited support for relationships with beetle severity (annual and cumulative pine tree mortality at 0.81- and 314-ha spatial scales). Nevertheless, we observed a significant increase in densities of hatched nests for beetle-foraging woodpeckers following the outbreak.  Our results suggest that woodpeckers, particularly beetle foragers, respond numerically to MPB outbreaks through increased nesting densities more so than functionally via nest survival.}

\normaltalk{\capitalisewords{Quantifying Trade-Offs in Migration and Reproduction in Long-Distance Migratory Birds of Contrasting Life History}}{\underline{Stephanie A Cunningham}, Toryn L Schafer, Christopher K Wikle, Bart M Ballard, Jay A VonBank, Mitch D Weegman}{The North American mid-continent and Greenland populations of greater white-fronted geese (Anser albifrons) exhibit disparities in life history, primarily migration strategy, as well as population trajectory. While the mid-continent population exceeds 1 million birds, the Greenland population is estimated at 20,000 individuals and has declined 43\% over the past 25 years due to low productivity. To examine the differences in decision making and possible trade-offs between individuals that result in differences in productivity, we fit 23 geese with neck collar GPS/accelerometer tracking devices between January 2016 and February 2018 (11 and 12 on mid-continent and Greenland birds, respectively). We identified 2 successful breeders and 9 non-successful breeders/birds that did not attempt in the mid-continent population, and 2 successful breeders and 10 non-successful breeders/birds that did not attempt in the Greenland population. We converted acceleration values to energy expenditure and time activity budgets, and used permutation analyses to compare like-for-like rates between successful and non-successful breeding birds throughout spring migration because we expected that successful breeding birds would prepare for a reproductive attempt. We found no difference between mid-continent and Greenland birds in their proportion of time feeding during any time period prior to the reproductive attempt. Mid-continent birds showed a slight difference in energy expenditure during the spring, with non-breeders expending more energy. Further analyses will explore links between environmental patterns and individual decisions, which could differentially explain successful and non-successful breeding. Quantifying decisions that promote productivity will be critical to improving conservation plans for these and similar migratory birds.}

\normaltalk{\capitalisewords{Fast Relief: Immediate Benefits from Reducing Invasive,  Non-Native Overpopulations of Canada Goose via Addling Their Eggs}}{\underline{Daniel A Edelstein}}{Throughout many metropolitan portions of the USA, Canada Goose (Branta canadensis) populations have dramatically risen in recent years. Negative impacts resulting from the presence of a western USA Canada Goose subspecies — B. c. moffitti — have included water quality degradation from large deposits of goose feces; the potential spread of bird flu disease to other birds and to people; and reduced biological diversity as the large, naturally-aggressive geese outcompete native bird species for food and nesting spaces. Costly treatment of goose-polluted water is also necessary so that it can be used again for recycled water on landscapes, car washes and other non-potable uses.  In response to more than 250 geese in 2010 residing at a sanitary district's reclamation ponds adjacent to the San Francisco Bay, the Las Gallinas Valley Sanitary District's initiated an addling program as part of an Integrated Goose Management Plan. One year after addling goose eggs so they ceased hatching, geese numbers dropped 64 percent. By 2018, merely seven newborn geese hatched at the site, compared to 74 before the now-ongoing, annual addling effort began.  Consequently, the District's believes its addling program is 1) a low-cost environmentally elegant way to maintain the quality of its treated wastewater for later treatment and reuse; 2) provides several native wildlife species suitable habitat for foraging and raising young; 3) preserves native plant species from foraging geese; and 4) offers a safe place for visitors to hike without experiencing altercations with periodically aggressive geese that previously crowded portions of popular trails.   }

\normaltalk{\capitalisewords{Is Kluane Lake in the Yukon Territory a Superhighway for Alaska's Migrating Golden Eagles?}}{\underline{Neil A Paprocki}, Travis L Booms, Robert J Spaul, Jesse L Watson, Dave Oleyar}{The identification of critical migratory corridors is one of many key insights to come out of the recent explosion of data generated from remotely tracked migratory birds. Remote-tracking data from Golden Eagles (Aquila chrysaetos) originating in northwestern North America suggests the Kluane Lake region in southwestern Yukon Territory, Canada is a major migratory flyway as birds fly south to wintering grounds in Canada and the contiguous United States. Somewhat surprisingly, relatively little standardized raptor migration count data exist for this region. The primary objectives of this study were to conduct fall raptor migration counts in the Kluane Lake region to 1) inform population estimates of Golden Eagles summering in Alaska and northwest Canada, and 2) identify the best count location for future migration monitoring. We conducted raptor migration counts from 24 September 2017 – 22 October 2017 at twelve different locations based on a combination of remotely tracked eagle flight-paths, local knowledge of the area, and on-the-ground observations by migration observers. We counted a total of 1,842 migrating raptors over 26 days of effort. This includes a total of 1,397 Golden Eagles with a daily high count of 281 individuals on 20 October. Our results add to the mounting weight of evidence suggesting the population size of Golden Eagles summering in northwestern North America is considerably higher than current estimates suggest. We successfully identified a count location for future monitoring, setting the stage for a full-season of raptor migration counts in fall of 2019.}

\normaltalk{\capitalisewords{Phylogeny and Evolution of the Gallopheasants}}{\underline{Peter A Hosner}, Hannah L Owens, Sarah Hyde, Edward L Braun, Rebecca T Kimball}{The gallopheasants comprise a clade of 23 species, including some of the most elaborately-plumaged and highly ornamented birds in the world. They also occupy a remarkable breath of environments and habitats, ranging from lowland rainforests to high grasslands and steppes of the Tibetan plateau. Here, we provide the first well-resolved species phylogeny of this charismatic group, inferred from ultraconserved element sequences. Unlike previous studies, which supported a rapid initial burst of diversification and unresolvable relationships, we identified a steady accrual of lineages through time. Reconstructions of morphological traits demonstrated strong phylogenetic signal, not only for highly ornamented males, but more cryptic females as well. Degree of ornamentation, plumage elaboration, and sexual dimorphism may be related to breeding system, with more monogamous species tending to exhibit reduced sexual dimorphism. However, breeding biology requires clarification for several poorly known species. Environmental niche also exhibited strong phylogenetic signal, yet with no evidence of niche filling.}

\normaltalk{\capitalisewords{Landscape Composition Explains High Rates of Dispersal in Translocated Lesser Prairie-Chickens}}{\underline{Liam A Berigan}, Carly Aulicky, Daniel Sullins, David Haukos, Kent Fricke, Jonathan Reitz, Liza Rossi, Kraig Schultz}{Since 2016, a multi-agency effort has been augmenting lesser prairie-chicken (Tympanuchus pallidicinctus) populations within the Sand Sagebrush Ecoregion of southwestern Kansas and southeastern Colorado. To date, 254 lesser prairie-chickens have been captured in the Short-Grass Prairie/CRP Mosaic Ecoregion of northwestern Kansas, equipped with VHF radio-collars or SAT-PTT GPS transmitters, and translocated to the Cimarron and Comanche National Grasslands. An impediment to successful lesser prairie-chicken translocation is the tendency for translocated females to disperse incredible distances (up to 400 km) following release. Minimizing post-release dispersal is critical to ensure translocated individuals join the target population. To determine why these birds are dispersing, we used Markov movement models to segment individual movement trajectories into dispersing and settled modes. We then quantified how landscape features and individual interactions affected the likelihood that a lesser prairie-chicken would switch from a dispersing mode to a settled mode. Preliminary results show that entering at least a quarter section (65 ha) of Conservation Reserve Program (CRP) grassland increases a bird's chance of settling down by 33\%, with larger areas of unbroken CRP grassland increasing the likelihood of settling down further. There was little evidence that conspecific attraction (including individual interactions and the presence of a nearby lek) encouraged lesser prairie-chickens to settle down. By prioritizing release sites near large tracts of unfragmented grassland, future lesser prairie-chicken translocation projects may be able to reduce the frequency of long-distance dispersal and increase the efficacy of translocation in restoring this vulnerable species.}

\normaltalk{\capitalisewords{Movement Patterns and Habitat Use of Endangered Whooping Cranes During Migration Through Canada's Oil Sands Mining Region}}{Mark T Bidwell, \underline{John A Conkin}, Jeff Ball, David A Brandt, Wade Harrell, Rhona Kindopp, Kris Metzger, Aaron T Pearse, Richard Wiacek}{Management of migratory birds depends on knowledge of movement patterns and habitat use throughout the annual cycle, but for many species this information is lacking during migration. The last remaining wild, self-sustaining population of endangered whooping cranes (Grus americana) breeds in northcentral Canada and winters in coastal Texas. The corridor connecting these endpoints transits the oil sands region (OSR) in northeastern Alberta. Oil sands development could affect cranes directly and slow recovery of the population if individuals are injured or killed by contacting oil sands infrastructure, or indirectly if quality of preferred habitat is degraded from impacts of exploration or extraction. From 2009 to 2018, we marked and monitored 71 cranes with GPS tracking devices to quantify movement and habitat use in the OSR. Most cranes (98 ± 3\%; mean ± SD) migrated through the OSR and many landed or stopped overnight (51 ± 18\%), sometimes near tailings ponds. However, apparent survival of cranes in the OSR was high; we did not record any mortalities during 260 individual migrations by marked individuals. Cranes used areas with more fen, open water, marsh and mudflat habitat, and less bog, upland and anthropogenic habitat than regionally available. Among anthropogenic classes, cranes used areas with more mine/well sites than available. Results will improve our understanding of how cranes move through and use this landscape during migration and of risks from oil sands development to cranes and their stopover habitats, and inform potential opportunities for regulators and industry to mitigate those risks.}

\normaltalk{\capitalisewords{Revisiting Red-Winged Blackbird Migration: Fine-Scale Temporal and Spatial Movements of Individuals Across Annual Migration}}{\underline{Michelle A Eshleman}, Page E Klug, Timothy J Greives, Erin H Gillam}{Humans have long marveled at large flocks of birds as they migrate across the sky and endeavored to understand their origin and purpose. Previous knowledge about migration patterns of red-winged blackbirds (Agelaius phoeniceus) comes from banding and recapture efforts in the middle of the 20th century. While useful, these methods are biased by the locations of banding stations and dedicated researchers capable of capturing the birds. They provide a single snapshot of where a bird was recaptured. With recent technology, researchers have the opportunity to investigate the migratory decisions of individuals throughout the year and to look at these movements on a finer scale. Our team asked whether males from two distinct breeding locations, separated by approximately 70 kilometers, share migratory pathways, stopover sites, and movement schedules with individuals from the same broad geographic area or if there is increased synchrony within a breeding population. To this end, we attached light-level geolocators, which provide daily location estimates, to blackbirds from two sites in eastern North Dakota. Six birds were recaptured at our eastern location and seven from the western location. Our results suggest that individuals traveled along the border between the Mississippi and Central Flyways and spent the winter in states ranging in latitude from Missouri to Louisiana. Although the two breeding sites are only separated by about 70 kilometers, we see longitudinal separation in overwinter sites for these two populations. Our data also suggests that this species may not be a diurnal migrant as previously reported in the literature.}

\normaltalk{\capitalisewords{Assessing the Potential for using Weather Radar to Predict Bird Collisions}}{\underline{Jared A Elmore}, Kyle G Horton, Corey S Riding, Timothy J O'Connell, Scott R Loss}{Up to 1 billion birds die per year from colliding with buildings in the United States. Many of these collisions occur at night when artificial lighting disorients nocturnally migrating birds. Radar aeroecology is a discipline used to study behavior and ecology of migrating birds that may help link migration characteristics (e.g., density, migration traffic rate, altitude, heading, and direction) to numbers of collisions. We collected bird collision data at buildings in Stillwater, Oklahoma during spring and fall migration from spring 2015 to spring 2017 using a standardized methodology and data was post-corrected for scavenger removal and survey effort biases. We used scans from three Next Generation Radar (NEXRAD) sites triangulated around Stillwater, Oklahoma to quantify sums, averages, and maxes of nightly flight characteristics (density, migration traffic rate, altitude, heading, and direction). Linear models were developed to assess whether observed numbers of nightly bird collision fatalities were predicted by radar derived migration characteristics. Models were ranked using Akaike's information criterion (AIC) and considered competitive with ∆AIC values of 0-2 and at least 2 greater than the null model. While our study was based on a small spatial scale area with little variation in radar derived migration characteristics, the same approach can be extrapolated to a network of radars to analyze broad, continent wide spatial scales with higher amounts of variation. Bird migration can be accurately predicted at broad spatial scales using radar and predicting collisions at such scales has important implications for reducing and mitigating bird collision risk.}

\normaltalk{\capitalisewords{Winter Ecology and Spring Migration Timing of Swainson's Thrush using Forest and Shade Coffee Plantations in Colombia}}{\underline{Keith A Hobson}, Ana M Gonzalez Prieto, Sean McElaney, Nick Bayly}{Shade coffee plantations in the Colombian Andes are considered de facto habitat for several Neotropical migrant birds. However, few studies have contrasted use of natural forest and shade coffee habitats and how habitat occupancy influences spring departure time and migration speed. We evaluated density, diet, roost site use and departure times and speed of migration of Swainson's Thrush (Catharus ustulatus) using three shade coffee plantations and adjacent forest in Colombia.   Despite diurnal habitat use, thrushes roosted in forest. Forest birds also consumed more fruit than those in coffee. Using a MOTUS tracking system, we found a strong effect of habitat on departure date (n= 166). Birds in forest departed an average of four days later than birds in coffee. However, birds that departed later migrated faster and the pattern was maintained along the migration path (n=44). Since migration speed is correlated with fuel load, birds wintering in forest likely reached higher fuel loads than birds wintering in shade-grown coffee before migration. Birds overwintering in forest can apparently compensate for late departure while decreasing risks associated with a slower migration and more frequent stops. Such facultative adjustment to winter conditions reveals previously unappreciated complexity in migratory decisions by long-distance migrants.}

\normaltalk{\capitalisewords{Investigating Aleutian Tern Breeding Season Movements Using Satellite Telemetry}}{\underline{Kelly A Nesvacil}, Donald E Lyons, Susan Oehlers, John Skinner, Jeff Mondragon}{The Aleutian tern (Onychoprion aleuticus) is a Beringian seabird with a small global population estimated at around 30,000 individuals. While populations are apparently stable or increasing in Russia, populations within Alaska at known breeding sites appear to be in decline. However, Aleutian tern population trends are difficult to assess due to lack of breeding site fidelity, gaps in colony counts, variability in colony attendance within and among years, and potential for high rates of inter-colony movement. As such, we investigated the use of satellite telemetry to observe breeding season movement in the vicinities of Dillingham and Yakutat, Alaska. Local scale movements were noted in both years and were primarily within 150 km of the tagging site, although locations extended up to 300 km away. Furthermore, we found previously undocumented Aleutian tern colonies in both 2017 and 2018 using location data from tagged individuals. For the Dillingham deployments, we conducted opportunistic flights to all colonies found in 2017, and in 2018, flights to colonies found in either year. Nesting attempts at the capture site near Dillingham were short-lived, but tagged individuals led us to colonies elsewhere that did fledge young, suggesting that later initiated nesting efforts and/or renesting may be important contributors to the population's productivity in this region. Identifying patterns of colony occupancy within and between seasons can help interpret historical Aleutian tern colony data, as well as assist in planning a future state-wide monitoring effort.}

\normaltalk{\capitalisewords{Habitat Use and Foraging Flights of Roseate Spoonbills in Florida Bay}}{\underline{Elizabeth A Lago}, John D Baldwin, Jerry J Lorenz}{Hydrological changes to the Everglades system has had negative impacts on the foraging habitats and nest success of roseate spoonbills (Platalea ajaja) that rely on distant coastal ephemeral wetlands in the southern edge of the Everglades to forage. In the early 1990s, roseate spoonbills traveled an average of 12 km to foraging sites that were proximate to their nesting colonies, using areas further away from their nest, therefore inland, as the breeding season progressed. Increasing water levels has left historical foraging habitats unsuitable for roseate spoonbills and we predict they are nesting further inland and it is unknown if there is relationship between spoonbill foraging flight distance and duration to nest success. This study will examine the 2005 through 2008 foraging habitat use and foraging flight behavior of twenty satellite tracked roseate spoonbills and compare them to historic observations. We will also demonstrate data collected from spoonbills during the 2018-19 breeding season. Initial observations are variable environmental conditions from 2005 to 2008 including a 2006-07 draw down reversal causing 46\% nest failure in Florida Bay and nest success decreasing with season in Tern Key leading to colony abandonment in 2008. Kernel density maps of past foraging areas demonstrated spoonbills foraging along the entire Everglades coast during the 2006-07 breeding season, including western foraging habitats such as Cape Sable but did not forage in these areas during the other two breeding seasons. Five spoonbills demonstrated nesting behavior during the past tracking study and three of them traveled the most total successive distances of over 1000 miles. The relationship between foraging behavior and nest success will inform future management decisions and give insight to the adaptive capability of roseate spoonbills with current sea-level rise.}

\normaltalk{\capitalisewords{Using Ancient DNA from Fossils to Place an Extinct Flightless Bird Genus in a Phylogenetic Context}}{\underline{Jessica A Oswald}, Julia M Allen, Ryan Folk, Brian Stucky, David W Steadman, Robert Guralnick}{Islands are home to morphologically, phylogenetically, and behaviorally unique lineages due to unoccupied niche space and ecological release not experienced by continental relatives. Many insular species of birds become flightless and, in many instances, much larger than related lineages. Extinct examples of this include the dodo, elephant birds, many rail species, and some species of ducks and geese in Hawaii. The extinction of these birds and many other species occurred when humans colonized and exploited islands during the Holocene. To understand the drivers of island biogeography and morphological shifts, it is critical to place these extinct species in a phylogenetic context. The Caribbean archipelago follows the same pattern with numerous Quaternary extinctions that include a large number of morphologically odd species including the flightless Club-winged Ibis (Xenicibis xympithecus) and the flightless and large rail genus Nesotrochis. Nesotrochis consists of three extinct species that were found on Cuba (N. picapicensis), Puerto Rico and the Virgin Islands (N. debooyi), and Hispaniola (N. steganinos). Based on morphology it was suggested that Nesotrochis is closely related to the small-bodied, volant African flufftails (Sarothruridae), surprising due to the wide difference in size. Here we explore this hypothesis by sequencing ancient DNA from Nesotrochis steganinos from a 7,000 year old toe bone. Our mtDNA provide support for a close evolutionary relationship between Nesotrochis and the Sarothruridae. These results have multiple implications for elucidating morphological evolution and flightlessness among rails, and historical dispersal events from Africa into the Caribbean.}

\normaltalk{\capitalisewords{Habitat- And Sex-Specific Parental Care in the Monomorphic Red-Headed Woodpecker}}{\underline{L A Walter}, Catherine Viverette, Lesley P Bulluck}{Red-headed Woodpecker (RHWO, Melanerpes erythrocephalus) populations are declining throughout most of their range, yet they occupy a variety of habitat types from golf courses to beaver wetlands. Understanding how habitat influences parental care and subsequently, fledging success, is important for RHWO conservation. Moreover, sex-specific parental care differences are not well understood in this monomorphic species. To see whether habitat type influences parental care, and if male and female provisioning efforts differ, we recorded and analyzed 142 hours of video across 21 broods in two important breeding habitats (savanna and closed canopy forest) at Fort A.P. Hill, Virginia. We captured and color-banded RHWO, taking feather samples for genetic sexing, and determined brood size and chick age with an extendable camera. Using generalized linear mixed models, we found the best predictors of provisioning to be habitat and an interaction between chick age and parent sex. Chicks are fed more often in closed canopy forests (p $<$ 0.01) compared to savanna. The interaction between chick age and sex (p = 0.05) shows that males increase their provisioning rate as chicks mature, but females do not. Specifically, males feed more than females after day 8, a developmental benchmark associated with pinfeather emergence. Additionally, males almost exclusively remove fecal sacs from nests (p $<$ 0.001), highlighting a potential method to determine sex of breeding adults in the field. Future analyses using our videos will assess if prey types differ by habitat to ascertain if higher provisioning rate is indicative of habitat quality.}

\normaltalk{\capitalisewords{Light Pollution is Greatest Within Migration Passage Areas and is Related to Flight Altitudes of Nocturnal Migrants}}{\underline{Sergio A Cabrera-Cruz}, Jaclyn A Smolinsky, Kyle P McCarthy, Jeffrey J Buler}{Artificial light at night (ALAN) produces light pollution that influences the biology and ecology of birds. Nocturnally-migrating birds traverse large expanses of land twice every year at night when ALAN illuminates the sky. We evaluated the association of several factors with the annual mean ALAN intensity over land within the geographic ranges of 298 nocturnally-migrating bird species. Light pollution within geographic ranges was relatively greater 1) during the migration season, 2) for shorter-distance migrants, 3) for species with smaller ranges, and 4) for species in the western hemisphere. Thus, any negative effects of light pollution may be particularly pronounced for long-distance migratory birds in the Western hemisphere during migration over bright light polluted areas. Hence, we evaluated the effect of ALAN on migrant flight altitudes across broad extents, predicting that birds would fly lower over bright (urban) areas than over dark (non-urban) areas. We used data from 9 weather surveillance radars in the US to estimate altitudes of birds migrating at night over bright and dark areas during 5 consecutive spring and autumn migration seasons. After controlling for weather, we found that flight altitudes are relatively higher over bright compared to dark areas. There remains an ongoing challenge to identify the mechanism for this relationship and to discern the influence of ALAN per se from other factors related to urbanization. We suggest that the effects of urbanization on wildlife extend into the aerosphere, stressing the need of understanding the influence of anthropogenic factors on bird use of airspace habitat.}

\normaltalk{\capitalisewords{Seasonal Associations with Urban Light Pollution for Nocturnally Migrating Bird Populations}}{\underline{Frank A La Sorte}}{The spatial extent and intensity of artificial light at night (ALAN) has increased worldwide through the growth of urban environments. There is evidence that nocturnally migrating birds are attracted to ALAN, and there is evidence that nocturnally migrating bird populations are more likely to occur in urban areas during migration, especially in the autumn. We use weekly estimates of diurnal occurrence and relative abundance for 40 nocturnally migrating bird species that breed in forested environments in North America to assess how associations with distance to urban areas and ALAN are defined across the annual cycle. Migratory bird populations presented stronger than expected associations with shorter distances to urban areas during migration, and stronger than expected association with higher levels of ALAN outside and especially within urban areas during migration. These patterns were more pronounced during autumn migration, especially within urban areas. Outside of the two migration periods, migratory bird populations presented stronger than expected associations with longer distances to urban areas, especially during the nonbreeding season, and weaker than expected associations with the highest levels of ALAN outside and especially within urban areas. These findings suggest that ALAN is associated with higher levels of diurnal abundance along the boundaries and within the interior of urban areas during migration, especially in the autumn when juveniles are undertaking their first migration journey. These findings support the conclusion that urban sources of ALAN can broadly effect migratory behavior, emphasizing the need to better understand the implications of ALAN for migratory bird populations.}

\normaltalk{\capitalisewords{Adverse Effects of Climate Change on Boreal Bird Communities Accentuated by Natural and Anthropogenic Disturbances}}{\underline{Junior A Tremblay}, Philippe Cadieux, Yan Boulanger, Dominic Cyr, Anthony R Taylor, David T Price, Peter Solymos, Diana Stralberg}{Climate change is expected to influence boreal bird communities significantly through changes in boreal habitat (forest composition and age structure) in coming decades. We used the LANDIS-II forest landscape model to project impacts of climate change and anthropogenic disturbance (forest harvesting) on boreal bird communities in northeastern Alberta. Changes in forest landscapes, and associated bird population trends, were projected according to four climatic forcing scenarios (Baseline, RCP 2.6, RCP 4.5, RCP 8.5), and three forest management scenarios. Both forest harvesting and climate-related drivers were projected to have large impacts on bird communities in this region. An important proportion of forests will transition to grass-, or shrub- dominated vegetation; many conifer-dominated stands will likely be replaced by younger deciduous tree cover. Consequently, the density of bird species associated with these habitats will largely increase. However, sharp declines in abundance of mature and overmature forest stands due to a warmer, drier climate would drive major declines in associated bird species, especially in conifer-dominated forest habitats. Under the most severe climate scenario, our simulations suggest that modified forest management could play an important role in the conservation of vulnerable bird species. Intensive forest harvesting (clearcutting 0.6\% of forest area annually) was projected to accelerate declines in bird density compared to a "no harvest" scenario (37\% and 22\% declines, respectively by 2100). Further, our study indicates that forest management practices will need to be adapted if old boreal mixedwood and coniferous forest habitats are to be conserved successfully as change s in climate unfold.}

\normaltalk{\capitalisewords{Connectivity and Staging Site Pollution Effects in a Long-Distance Migratory Shorebird}}{\underline{Christy A Morrissey}, Kristin Bianchini, Jessica H , Ann McKellar, David Newstead}{Many shorebird species are long distance migrants and are currently suffering from steep population declines. We hypothesize that contaminants may be impacting fuelling and migration timing at staging sites throughout the migratory flyways. We assessed the migration routes and staging locations of Sanderling (Calidris alba) captured from 2012-2018 using band resights (n=89/884) and Motus automated radio telemetry (n=238). The Gulf of Mexico (GOM) is a known hotspot for contamination from oil and its toxic constituent, PAHs. Therefore, we also collected data on the sediment PAH levels, fuelling and migration timing at 3 staging areas in the GOM. We found many Sanderling using the Midcontinent flyway in spring were also later sighted or detected via telemetry in the coastal Atlantic Flyway in fall, suggesting an elliptical migration. Winter resights were as far south as Peru, and a large concentration occurred in the GOM (e.g., Florida and Texas). There was strong spring connectivity between the GOM and the northern staging site of Chaplin Lake, Saskatchewan, such that 94\% (n=33) of GOM-tagged individuals successfully tracked were detected in Chaplin. PAH concentrations in sediment were highest in Louisiana and birds from this site had lower fuelling rates and departed later than the study average.  Overall, we found birds departing later from the GOM also arrived later in Chaplin. However, these birds caught up by reducing their staging duration at the northern site, which may have affected their departure fuel loads. We suggest that contaminants have the potential to alter fuelling and migration timing in shorebirds, and there is an urgent need to assess the carry-over effects of contaminants that may manifest at disparate staging sites.}

\normaltalk{\capitalisewords{Traveling as a Newbie:  Migratory Strategies Linked to Survivorship in Juvenile Shorebirds in Peru}}{\underline{Eveling A Tavera}, David B Lank, Mark C Drever, Viviana Gutierrez-Ruiz}{Selection of non-breeding sites by long-distance birds should involve a balance between local survival rate, the relative costs of migration, and the effects on other components of life history strategies. We use Western and Semipalmated Sandpipers at Paracas, Peru, as models to test survival differences among populations with different migration distances to travel between breeding and non-breeding grounds. Westerns breed in Alaska, whereas Semipalmated populations at Paracas are a mixture of short-billed birds from western Arctic breeding sites ~11,000km from Paracas, plus long-billed birds from eastern sites, migrating ~8000km. Western sandpiper juveniles confirmed the oversummering strategy of more southerly non-breeding populations, with no individuals preparing for northward migration in their first year. These non-migrants had significantly higher annual survival than adults (Juveniles: 0.93 ±0.06 (SE) to 0.76±0.04 (SE)/Adults: 0.86 ±0.09 (SE) to 0.60±0.04 (SE)). The high survivorship of non-migrants highlights the the life history advantage of missing a first potential breeding season and provides an estimate of the survival cost of migration and breeding attempts. On the other hand, Semipalmated juveniles showed a bimodality strategy. Long-billed eastern birds showed a higher propensity to migrate north in the first year of life. However, these long-billed birds had a significant lower survival than short-billed ones (-0.059±0.02 (SE) per mm), thus explaining why only a lower proportion (~31\%) attempt to perform this migration as a juvenile. Our results help us explain how these birds may offset the real cost of migration by selecting the best migratory strategy with no life experience at all.}

\normaltalk{\capitalisewords{Birds in Fruit Production Systems: Contexts, Outcomes, Costs, and Benefits}}{\underline{Catherine A Lindell}}{Birds provide ecosystem services and disservices in agroecosystems. Fruit-eating birds cause significant losses in fruit production systems. From a survey of 1500 North American growers, we estimated that birds cause over \$100 million in damage annually to five fruit crops in five states. Birds also provide benefits to fruit producers. We installed nest boxes for American kestrels, Falco sparverius, in sweet cherry orchards in northern Michigan and documented reduced fruit-eating bird abundance in orchards with active kestrel boxes compared to orchards without kestrels. If all sweet cherry growers in Michigan installed kestrel boxes and boxes had similar occupancy rates to those we observed, we estimated that Michigan would have more than \$2 million in extra revenue and approximately 50 jobs would be created over a five-year period. Kestrel boxes that we installed in western Michigan blueberry fields had lower rates of kestrel occupancy than those in sweet cherries and higher rates of starling, Sturnus vulgaris, occupancy, likely leading to lower benefits for blueberry growers than those experienced by sweet cherry growers. Some bird species, like American robins, Turdus migratorius, provide both ecosystem services and disservices. Although American robins commonly eat fruit from plants, they also eat fruit from the ground, helping to remove resources for insect and fungal pests of cultivated fruit. Challenges for the future include determining which fruit management practices result in an optimal ratio of ecosystem services to disservices provided by birds. This ratio undoubtedly varies by crop and region, necessitating local-scale studies.}

\normaltalk{\capitalisewords{Insectivorous Birds Foraging in Temperate and Tropical Agricultural Lands}}{\underline{Julie A Jedlicka}}{Determining the ecosystem function of high-order predators is critical for evaluation of food web interactions. Insectivorous birds are abundant predators in many ecosystems yet because they forage upon such small taxa, it remains largely unknown whether birds are providing ecosystem services in the form of pest control or disservices by preying upon predaceous arthropod species, functioning as intraguild predators. I provide a look into various methods used to quantify the ecosystem services provided by insectivorous birds including exclosures, sentinel pest experiments, and more recent molecular scatology methods. Case studies implementing these techniques in agricultural lands will be presented ranging from temperate vineyards to tropical coffee plantations. Logistical problems implementing these techniques will be discussed to assist other researchers. Overall, adult and nestling Western Bluebirds (Sialia mexicana) were found to consume mostly herbivorous insects, primarily from the orders Hemiptera and Lepidoptera, representing over half (56\%) of the prey items from foraging in California vineyards. Intraguild predation (of predator or parasitoid arthropods) represented only 3\% of adult and nestling dietary items. As high-throughput Illumina sequencing becomes more accessible, the powerfully informative techniques molecular scatology offers can be more widely applied to reveal the ecosystem services and potential disservices provided by abundant yet cryptic foragers.}

\normaltalk{\capitalisewords{Waterbird Habitat Quality and Groundwater Recharge Potential of Alfalfa in the San Joaquin Valley}}{Khara M Strum, \underline{Kristin A Sesser}, W D Shuford, Gregory H Golet}{Alfalfa (Medicago sativa) provides important habitat for birds and other wildlife throughout western North America. In California's San Joaquin Valley, alfalfa provides foraging habitat for several of the Central Valley Joint Venture's (CVJV) focal species, such as herons, egrets, White-faced Ibis, and Swainson's Hawk, and a recent update for the CVJV Implementation Plan states the habitat objective for flood-irrigated alfalfa is to maintain its current extent in this region. With implementation of California's Sustainable Groundwater Management Act, flooding of alfalfa fields is being considered as a method for groundwater recharge. Given this interest, we designed a study to evaluate 1) the relationships between habitat value for the CVJV's focal bird species and groundwater recharge potential of alfalfa fields, and 2) the landscape characteristics associated with habitat value of alfalfa fields. Our preliminary results support the hypothesis that multiple-benefit investments in bird habitat and groundwater recharge can be developed. The results of landscape analyses can be used to develop strategies for locating alfalfa fields in places that will provide the greatest benefits for both birds and groundwater recharge. This work highlights an example of how ornithologists can design field studies to inform the implementation of multiple-benefit conservation strategies.}

\normaltalk{\capitalisewords{Conservation of Kirtland's Warbler in a Changing World}}{\underline{Christine A Ribic}, Donald J Brown, Deahn M Donner, Carol I Bocetti}{Identifying management strategies to maximize probability of long-term persistence requires an understanding of the vulnerability of species to projected environmental changes. Species that undertake annual long distance migrations may be particularly vulnerable to rapid climate change because they interact with multiple spatially discrete environments. We sought to quantify the vulnerability of Kirtland's Warbler, a rare Neotropical migratory songbird that breeds almost exclusively in the Lower Peninsula of Michigan and winters in the Bahamian Archipelago, to projected environmental changes on both the breeding and wintering grounds. We developed a population-level model that simulates the full annual cycle, and parameterized the model using previously established empirical relationships. We then simulated independent and interactive effects of reduced breeding grounds habitat quantity and quality, and wintering grounds habitat quality on the four densely occupied central islands (using precipitation as a proxy), on long-term population viability. Our results indicated the Kirtland's Warbler population is stable under current environmental and management conditions. Reducing breeding grounds habitat quantity resulted in reductions of the stable population size, but did not cause extinction. In contrast, projected moderate to large reductions in wintering grounds precipitation caused the population to decline, with risk of extinction magnified when habitat quantity also decreased. Potential shifts to alternative wintering locations could mitigate our projected declines. Our study indicates that probability of long-term persistence for Kirtland's Warbler will depend heavily on wintering grounds habitat quality, and underscores the importance of considering the full annual cycle for understanding impacts of changing environmental conditions on migratory species.}

\normaltalk{\capitalisewords{Simulating the Effects of Climate on Fire Regime \& Vegetation: Implications for Woodland Caribou \& Boreal Landbird Communities}}{\underline{Samuel A Hache}, Tati Micheletti, Diana Stralberg, Steve Cumming, Eliot McIntire, Junior Tremblay, Mathieu Leblond, Jean Marchal, Angeles Ana Paula Raymundo Sanchez, Mario van Telgen,  H ,  P }{Projections from climatically based models suggest dramatic changes in the potential vegetation, natural disturbance regimes, and fauna of Canada's ecosystems. These changes will be especially severe and rapid for Canada's western boreal forests, where polar amplification and generally flat terrain will promote rapid change, including an increased frequency and severity of drought. We present the findings of an integrated simulation of wildfire, anthropic disturbances and vegetation dynamics to evaluate the consequences of 21st century climate warming over BCR6 in the Northwest Territories, Canada. These process simulations were combined with statistical of predictions of existing national Boreal Caribou demographic and resource selection function models, and new abundance models for 80 migratory bird species to: 1) inform multi-species conservation planning by determining whether conservation actions to protect Boreal Caribou would result in disproportionate protection benefits for migratory birds, including  species at risk (SAR); 2) evaluate the potential conservation contributions of the newly established Edéhzhíe National Wildlife; and 3) determine how these results might be altered under climate warming. The process and statistical models were implemented and integrated in SpaDES, a suite or R packages for spatial simulation and reproducible science. Results from studies such as this are required to develop conservation measures and sector-specific mitigation to benefit Boreal and Southern Mountain Caribou, migratory birds, and other SAR.}

\normaltalk{\capitalisewords{A Synthesis of Climate Change Impacts on Boreal Bird Communities in Managed Landscapes of Canada}}{\underline{Junior A Tremblay}, Philippe Cadieux, Yan Boulanger, Dominic Cyr, Anthony R Taylor, David T Price}{Climate change is expected to strongly influence boreal bird communities through changes in boreal forest composition and age structure in the coming decades. In this study, we use the LANDIS-II forest landscape model to project the impacts of climate change and anthropogenic disturbance (forest harvesting) on species of interest (Black-backed Woodpecker, Bicknell's Thrush) and boreal bird communities in two provinces of Canada (Alberta and Québec). More specifically, we project changes in forest landscapes, and associated bird population trends according to climatic forcing scenarios (Baseline, RCP 4.5, and RCP 8.5), and different forest management scenarios. Both forest harvesting and climate-related drivers are projected to have great impacts on bird communities in these regions. Simulations projected significant changes in dominant tree species biomass, from coniferous to broadleaved deciduous, and reduction of mature / old forest stands. Consequently, important declines are projected for bird species associated with old forests, especially coniferous dominated forests with climate change, amongst others the Black-backed Woodpecker. Our projections also demonstrated the key role played by elevation, suggesting upslope migration of deciduous and temperate tree species will occur in regions with elevation gradient. However, significant changes in species composition are projected to generally occur after 2080 for the Canadian boreal zone. Such successional lags in vegetation responses to climate change contrast with results from bioclimatic models that do not capture the expected lags in vegetation responses to changing climate. Our results also suggest that innovative forest management strategies could help maintain boreal bird habitat abundance under climate forcing scenarios. Such results can provide guidance for considering effective long-term conservation of habitat for threatened species in a changing world.}

\normaltalk{\capitalisewords{On a Simulated Day, you can See Forever: Comparing Analytical Methods using Simulated Point Counts}}{\underline{Elizabeth A Rigby}, Douglas H Johnson}{I created a computer simulation of point counts, including variation in detection due to spatial arrangement (including distance from observer and movement of birds) availability (song rate), and perceptibility (observer ability, extraneous noise, etc).  I then analyzed the simulated data using counts as an index of abundance, distance sampling, double observer methods, n-mixture models, and removal sampling.  By using simulated data, I was able to compare those results to the true population, which is almost never known in field studies. While biased, indices performed well, and were highly correlated with true abundance.  Some "adjustment methods" that account for imperfect detection probability performed well in certain circumstances (removal methods) and some often performed poorly (n-mixture models), but adjustment methods did not overwhelmingly or consistently out-perform indices.}

\normaltalk{\capitalisewords{The Potential of Citizen Science Data for Population Monitoring}}{\underline{Alison A Johnston}, Matt E Strimas-Mackey, Wesley M Hochachka, Orin Robinson, Viviana Ruiz Gutierrez, Eliot Miller, M T Auer, Daniel Fink}{Citizen science data on the occurrence and abundance of birds are now an integral part of many biological monitoring projects. However, citizen science projects without defined sampling design or rigorous protocols can produce data that are not suitable for population monitoring. There are two important tools that can be used to process and analyse citizen science data to estimate robust species distributions. First, opportunistic citizen science data can be filtered in order to produce data with a more standardised set of survey characteristics. This can involve spatial subsampling and selecting data with more consistent survey effort. Second, including covariates of effort in the analyses can account for variation in detectability, enabling the model to more accurately estimate relationships between the species distribution and environmental covariates. We assess the impact of these two tools to leverage the power of citizen science data to estimate species distributions in models of species encounter rate, occupancy, and relative abundance. We find that for all ecological metrics, the estimated species distributions improve when using these analytical tools of filtering and accounting for variation in detectability. We discuss scenarios for which citizen science data may provide a useful dataset to monitor species populations or to augment more standardized surveys. Overall, we describe the potential to use citizen science data for population monitoring, when combined with appropriate and careful data processing and modelling.}

\normaltalk{\capitalisewords{Genetic Architecture and Asymmetric Introgression of Plumage-Related Reproductive Barriers in Hybridizing Birds}}{\underline{Georgy A Semenov}, David R Khaydarov, Rebecca J Safran, Scott A Taylor}{Uncovering the link between natural selection, the phenotypes it acts upon, and the genetic architecture of phenotypic traits is key for advancing our understanding of the speciation process. Hybrid zones provide a unique opportunity to address these questions because ongoing admixture facilitates our ability both to identify traits relevant for reproductive isolation and to determine their genetic basis. In addition, the evolutionary mechanisms that promote non-equilibrium introgression of such traits are of a broad interest. Motacilla alba alba and M. a. personata are two subspecies of the white wagtail that hybridize in Asia. Previous studies revealed that coloration of the head and neck regions are important mating signals associated with reproductive barriers in the hybrid zone; however, personata-like plumage traits have advanced via introgression into alba genomic background and territory by ~300 km. Using whole genome sequencing of alba (n = 10) and personata (n = 10) that exhibited subspecies-specific plumage traits, but that originated from the hybrid zone center, we used naturally occurring admixture to reveal that the two plumage types can be distinguished by two narrow ($<$400kb total) highly divergent genomic regions. These regions are located on chromosomes 1A and 20, and together contain ~10 genes involved in melanogenesis, cell signaling, and developmental processes. One of these regions harbors Agouti Signaling Protein and exhibits variation patterns consistent with dominant inheritance of the alba-allele. We discuss how these findings establish a putative link between plumage-mediated positive assortative mating, the underlying genetic architecture, and the mechanism promoting their asymmetric introgression.}

\normaltalk{\capitalisewords{Insights into Adaptation and Population Change from Long-Term Studies of Cavity-Nesting Birds}}{\underline{Renee A Duckworth}}{Cavity‐nesting birds that breed in artificial nestboxes have long been the focus of studies worldwide. Nestbox studies have not only provided insight into essentially every aspect of avian ecology including reproduction, environmental stress, parasitism, and behavior, they have also been at the forefront of work on how contaminants and other environmental stresses affect individuals and populations. Long term studies of birds are increasingly important for establishing and monitoring the impacts of climate change and we suggest that because of the rich history of detailed monitoring of nestbox populations, these systems are uniquely positioned to provide insight into how birds will adapt. Running parallel to studies of nestbox populations are numerous long‐term studies of cavity nesting birds breeding in natural populations. These studies provide crucial insight into population responses to environmental change because they highlight the historical selection pressures that shaped cavity nesting species and so may shed light on how both natural cavity and nestbox populations will respond to environmental change. I will highlight work on both nestbox and natural cavity systems to share the unique insights that can only be gained from multi‐year datasets and to highlight the benefits of studying cavity nesting species in both their man‐made and natural habitats.}

\normaltalk{\capitalisewords{Migration Patterns of Three Loon Species (Gavidae) Breeding in the Arctic}}{\underline{Joel A Schmutz}, Brian D Uher-Koch, Autumn-Lynn Harrison, Daniel M Mulcahy, Jeffrey S Fair, Kenneth G Wright, Christopher R DeSorbo}{Three loon species breed in low densities across much of the North American Arctic. Despite conspicuous ecology and conservation concerns, they remain little studied. Beginning in 2000, we began tracking loons with satellite transmitters to understand migration behavior, geographic extent, and contrasts among species. We deployed transmitters on 38 Red-throated Loons (Gavia stellata) across four populations in Alaska, 85 transmitters on Yellow-billed Loons (G. adamsii) across four populations in Alaska and northern Canada, and 30 transmitters on Pacific Loons (G. pacifica) between two sites in Alaska. We found differences in migration patterns, body sizes, contaminant exposure and genetic distinctions among breeding populations. Populations of all three species breeding in northern Alaska migrated to east Asia whereas more southerly populations stayed within North America, wintering along the Pacific coast, some as far south as the Baja Peninsula. Migration to Asia was substantially longer (6000 km) than migration within North America (1500 km). Migrants to Asia incurred greater contaminants exposure. Toxicity of PCBs in eggs of Red-throated Loons was much higher in Asian migrants. For Yellow-billed Loons, exposure to mercury was also greater for birds wintering in Asia. Yellow-billed Loons breeding in interior Canada exhibited plasticity in migration route and distance. Most migrated to marine wintering habitats via short migrations ($<$ 1800 km) over mountains with few lakes, whereas others migrated via a marine route that tripled travel distance. Short-distance migrants were 20\% smaller and genetically different from long-distance coastal migrants.}

\normaltalk{\capitalisewords{An Ornithologist's Toolkit for Molecular Community Ecology}}{\underline{Michelle A Jusino}}{Molecular-based community sampling techniques such as high-throughput amplicon sequencing (HTS) have exploded in many fields of study. By providing unprecedented volumes and quality of community data, these techniques are revolutionizing our understanding of biodiversity and the essential but complex interactions between organisms. Fields such as bacteriology and mycology have long embraced molecular tools for understanding the ecologies and natural histories of microorganisms – and these tools are now being used by the ornithological community to answer similar avian natural history questions. However, HTS output from environmental samples requires careful interpretation and consistent controls. I will discuss newly available tools that have streamlined the analysis and interpretation of ecological data generated by HTS, and will validate these tools using HTS data from a diet study of Red-cockaded Woodpeckers. I will demonstrate the importance of careful primer choice and validation with sequencing data from multiple platforms. Several critical decisions must be made during the bioinformatics process and I will provide suggestions to help researchers get the most out of their data. Using "mock community" positive controls for fungi and arthropods, I will demonstrate how positive sequencing controls can be used to parameterize bioinformatics pipelines and mitigate known issues with PCR bias and index bleed. Mock communities can be used to produce vastly more accurate community characterization and are an important component of every HTS run. With care taken at a few vital stages, HTS will continue to vastly improve our understating of avian ecology and the complex communities of organisms associated with birds.}

\normaltalk{\capitalisewords{Detecting Ecosystem Services from Insectivorous Birds}}{\underline{Julie A Jedlicka}}{The field of molecular scatology has grown immensely in the past decade. Most of the literature, however, is dominated by complex, lab and computationally intensive methods yielding a plethora of data that can be difficult to analyze. This includes using next-generation sequencing to reveal previously unknown dietary items and elucidating the ecosystem function of such predators such as Western Bluebirds (Sialia mexicana). After a brief summary and case study of how next-generation sequencing can be used in this way, I will offer up alternative methods for more straightforward and accessible approaches. I present a case study of recent work analyzing the arthropod diets from insectivorous birds foraging in Kenyan coffee farms.  The focus is to determine whether bird foraging is providing avian-induced ecosystem services by testing whether three different pest species are consumed from a variety of different bird species.}

\normaltalk{\capitalisewords{Using Box-Nesting Birds to Explore Methods in Studying Developmental Immunity and Gut Microbiota}}{\underline{Sarah A Knutie}}{Parasites can affect the health of their bird hosts. Consequently, birds have evolved defense strategies, such as resistance, to combat their parasites. An example of a resistance mechanism is the immune response, which has been studied extensively in birds. Recent evidence suggests that the gut microbiota of the host affects aspects of the immune response, which unlocks an exciting direction of research in avian host-parasite evolutionary ecology. My talk will first address why it is important to consider parasites when studying bird health. I present affordable methods to quantify the parasite-specific immune response in wild birds and use box-nesting bird species, such as tree swallows (Tachycineta bicolor) and eastern bluebirds (Sialia sialis), as examples. I will also present current methods in characterizing the wild bird microbiota and results on how the gut microbiota of birds could be affecting their resistance to parasites. Lastly, I will discuss challenges in studying the gut microbiota and immune system of wild birds.}

\normaltalk{\capitalisewords{Why Do Bird Conservation and Ornithology Need Social Science?}}{\underline{Ashley A Dayer}}{Human dimensions research, conducted through conservation social sciences, can be applied to improve conservation outcomes for the benefit of birds, habitats, and people. It can inform how to design effective conservation programs, how to change people's behavior, how to support human livelihoods, and how to overcome challenges to program success.  Conservation social science can also be accomplished in socio-ecological frameworks to better our understanding of complex systems. To illustrate these roles, I will use diverse examples from my research program: barriers to bird-friendly coffee consumption by birdwatchers; experiences and future behavior of participants in a private lands conservation program; the socio-ecological services of working wet meadows produced and experienced by ranchers and waterbirds in the Intermountain West; behavior change campaigns to protect beach-nesting shorebirds; and the role of citizen scientists who feed birds in coupled-human natural systems. My goal will be to illustrate how collaborations between conservation social scientists, biologists, and practitioners can lead to conservation success.}

\normaltalk{\capitalisewords{Distribution and Use of Freshwater Habitats by Kittlitz's Murrelets on Lakes in the Lower Wood River Lake System, Alaska}}{\underline{Kelly A Nesvacil}, Grey Pendleton, Michael Swaim, Patrick Walsh}{The Kittlitz's murrelet (Brachyramphus brevirostris) is a small seabird known only to breed in Alaska and the Russian Far East. Substantial knowledge gaps exist for this species, especially its potential use of freshwater systems during the breeding season. Incidental observations regarding use of the freshwater systems in the Bristol Bay region of Alaska by Kittlitz's murrelet had occurred prior to 2013. In order to further document use of these systems, from 2013-2015 we determined the occurrence or distribution of Kittletz's murrelets in the lower lakes of the Wood River lake system outside of Dillingham, Alaska. Transect surveys or  distance sampling were conducted on Lakes Aleknagik, Nerka, Beverly, Togiak Lake, and Little Togiak Lake. No Kittlitz's murrelets were observed during these surveys on Togiak Lake or Lake Beverly, however occurrence was documented throughout Lake Aleknagik, Little Togiak Lake, and Lake Nerka. Moreover fish holding was observed across the years of the project. The use of freshwater lakes during the breeding season has been documented for the congener species, marbled murrelet (B. marmoratus). Comparisons and hypothesized similarities of Kittlitz's ecology and use of freshwater systems to that of marbled murrelets will be discussed, as will recommendations for future monitoring.}

\normaltalk{\capitalisewords{Seasonal Offshore Distribution and Habitat Use of Brachyramphus Murrelets in Alaska}}{\underline{Elizabeth A Labunski}, Kathy Kuletz, Erik Osnas}{Marbled murrelets (Brachyramphus marmoratus) and Kittlitz's murrelets (B. brevirostris) are distributed across marine waters in Alaska extending from the Gulf of Alaska (GOA), the Bering, Chukchi and Beaufort seas. We used visual at-sea survey data from 2004 – 2018 to examine seasonal distribution and habitat use of Brachyramphus murrelets in Alaska. In summer marbled murrelets are typically near shore in the GOA adjacent coastal breeding habitats, although they also occur offshore in the southern Bering Sea, widely distributed in low numbers. During fall marbled murrelets increase in number across the southeastern Bering Sea (SEBS) with high densities near Bristol Bay, where prey is abundant; this may also be an important molting area. Kittlitz's murrelets are typically near shore during the breeding season from southeast Alaska to the Aleutian Islands, with few individual sightings near the Seward Peninsula in the Chukchi Sea. In fall post-breeding Kittlitz's murrelets appear to disperse west and north, with low densities in the SEBS and higher densities in the Chukchi and western Beaufort seas, particularly on-shelf near underwater canyon edges or shelf breaks. Birds may be drawn to the Arctic due to currents and upwelling sites that concentrate prey, such as large copepods, euphausiids, and Age-0 arctic cod.  During winter/early spring, Kittlitz's are mainly observed in the Bering Sea north of 60˚N between Nunivak and St. Lawrence Island, where they occupy ice edge habitats, open water leads, and polynyas. As the Arctic warms, changes in sea ice extent and duration will likely impact Kittlitz's murrelet overwintering habitat}

\normaltalk{\capitalisewords{The Past and Future of 'benchmark' Systems for Delimiting Bird Species}}{\underline{Joseph A Tobias}}{A major challenge to setting species limits under any form of the Biological Species Concept is the problem of assessing the degree of reproductive isolation between allopatric forms where interbreeding is impossible simply because of geography. One longstanding approach---often referred to as 'yardstick' or 'benchmark' methods---is to quantify divergence in undisputed sympatric species and use this as an objective framework to set a divergence threshold for delimiting allopatric species. Recent guidelines for applying this method in birds are controversial and unpopular among many regional and global taxonomic committees. Nonetheless, their obvious weaknesses are arguably offset by important strengths, and they are currently shaping taxonomic practices worldwide, driving rapid changes to some influential global treatments of avian diversity, including the IUCN Red List. In this talk, I will summarise the history and rationale for benchmark criteria, outlining the main criticisms they face, and discussing possible solutions.}

\normaltalk{\capitalisewords{Coalescent-Based Species Delimitation in Avian Taxonomy: History, Promise, and Pitfalls}}{\underline{Nicholas A Mason}, Nicholas K Fletcher, Brian Gill, Chris Funk, Kelly Zamudio}{Species are a fundamental unit of biodiversity that are challenging to define. Increasingly, genetic data and coalescent-based methods are used to identify and delimit species, often based on large genome-scale datasets. Despite their widespread use and potential relevance to avian taxonomy, we lack a comprehensive understanding of the sensitivity of these methods to potential sources of bias. Here, we provide an overview of the development of these methods, and explore the performance and interpretation of coalescent-based species delimitation in light of various forms of missing data. We highlight an array of case studies from the literature that illustrate different findings and examine how they were subsequently used (or not) in decisions related to species limits. We discuss the impact that incomplete data matrices have on species delimitation, especially how variation in geographic sampling and missing genetic data impact coalescent-based inferences. We find that removing sites with genetically admixed individuals incorrectly favors multi-species delimitation scenarios, especially when isolation by distance is strong, but even when gene flow among neighboring populations is relatively high. Furthermore, we find that species delimitation scenarios are impacted by the number of SNPs and individuals sampled. Coalescent-based species delimitation methods provide an appealing way to quantify support for alternative delimitation scenarios, but as highlighted here, users and interpreters should be aware of potential sources of bias. By providing a brief history and my perspective on coalescent-based species delimitation scenarios, we hope this talk will engender discussion on their use and implementation in alpha taxonomy in ornithology.}

\normaltalk{\capitalisewords{Phylogenetic Relationships Within the Sharp-Shinned Hawk Complex with a Focus on Caribbean Populations}}{\underline{Therese A Catanach}, Jeff A Johnson, Russell Thorstrom, Matthew R Halley, Samantha Palhano, Jason D Weckstein}{The Sharp-shinned Hawk (Accipiter striatus) is widely distributed across the New World, breeding from western Alaska to Argentina.  Currently, 10 subspecies are recognized, although based on size and plumage variation some authorities have suggested that the South and Central American taxa are distinct species.  Using DNA sequences of the flanking regions of ultraconserved elements and full mitochondrial genomes we inferred a phylogeny for the Accipiter striatus complex.  We then coupled our phylogeny with a suite of morphological characters to identify species limits within this clade.  Based on these results, we determined that multiple lineages within the complex warrant species status.  Three Accipiter striatus subspecies occur on Caribbean Islands and a fourth occurs on Haida Gwaii, British Columbia.  The Caribbean subspecies in particular have undergone drastic population declines, and one, the Puerto Rican Sharp-shinned Hawk (A. s. venator) is federally listed as endangered.  Within each island a number of disjunct montane forest populations exist and, based on band recoveries, individuals do not appear to travel between populations.  However, we do not know if this is modern isolation due to habitat fragmentation or whether these populations are truly isolated montane specialists without historical gene flow between them.  Therefore, we sequenced multiple individuals from across the Caribbean, including virtually all distinct collecting localities to identify population connectivity both between islands and among forest patches on individual islands.}

\normaltalk{\capitalisewords{Exploring the Links Between Morphological Form and Ecological Function in Birds}}{\underline{Joseph A Tobias}}{The connection between morphological form and ecological function is a central organizing principle in nature, but a lack of data has prevented a general understanding of how organismal traits are connected to ecological function. Using comprehensive beak and body measurements for $>$99\% of bird species, I show that avian ecological niches are inherently multidimensional, being minimally described by a phenotypic trait space with six dimensions. Moreover, the position of species within this trait hypervolume maps onto niche axes with remarkable fidelity, often predicting variation in resource type, substrate use and acquisition mode with over 70\% accuracy. I also show that this fundamental association arises through the convergence of ecologically similar lineages toward the same regions of phenotypic space, and that similar evolutionary outcomes have recurred independently across multiple biogeographic realms. These analyses suggest that ecological adaptation generates a predictable structure to avian diversity, and that a restricted set of trait dimensions can provide information about ecological function at a range of spatial scales.}

\normaltalk{\capitalisewords{Creating a Win-Win for Ranchers and Waterbirds Through Co-Produced Conservation (Social) Science}}{\underline{Ashley A Dayer}, Mary Sketch, Laurel Anders, Alex Metcalf}{Participatory approaches to natural resource research allow for co-production of science across communities of local stakeholders, conservation practitioners and conservation scientists. We used a novel participatory approach -- landowner-led workshops -- to understand ranchers' experience with flood irrigation in the Intermountain West, which is an essential practice for maintaining waterbird habitat. We present results about the effectiveness of our approach from a mixed-methods, empirical evaluation including participant observation, qualitative analysis of the workshop transcript, and a post-workshop survey. We found that landowner-led workshops provided a unique space for informal, open conversation, with both conservation practitioners and landowners benefiting from candid social learning. However, workshops did not perform consistently on fostering the diversity of participation. Based on our two applications, we conclude landowner-led workshops show promise for enhancing participatory research within the realm of natural resource management, empowering community members, and bridging the gap between conservation practitioners, landowners, and conservation scientists.  We also reflect on lessons learned from such an approach.}

\normaltalk{\capitalisewords{Reducing the Impacts of Cats on Birds in Canada Through Collaborative Efforts with Multiple Stakeholders}}{\underline{Elizabeth A Gow}}{Cats (Felus catus) are one of the most beloved pets in North America. In Canada, cats are estimated to kill 100–350 million birds annually and are the number one direct killer of birds in Canada and the USA. Despite the highly significant impacts that cats likely have on bird populations, cat management is a hotly contested and controversial issue due in part to the love and personal connections people have to cats. At the root of this conservation problem is the overpopulation of cats, which also contributes to poor cat welfare, unnecessary cat mortality or injury, and transmission of diseases to other wildlife and humans. For these reasons numerous stakeholders have interests in reducing outdoor cat numbers, including conservationists (e.g. ornithologists), animal welfare workers, veterinarians, cat owners, ordinary citizens, and policy makers. "Bridging the gaps" among stakeholders and people with divergent views and opinions involves getting all parties to sit down at the same table, listening and then discussing possible solutions based on the best available science. It is this critical step that has the biggest impact on creating change and is necessary to begin to solve this wicked conservation issue. But such collaboration and openness to the sharing of insight and perspectives is often met with resistance and numerous challenges. By adjusting my research to meet the needs of several stakeholders this has facilitated efforts to find solutions that are beneficial for cats and birds.}

\normaltalk{\capitalisewords{An Assessment of Ecological Value and Vulnerability in the Bering, Chukchi, and Beaufort Seas}}{\underline{Melanie A Smith}, Max Goldman, Erika Knight, Benjamin Sullender}{Tasked with the responsibility of ensuring the resilience of Arctic ecosystems, managers and decision-makers must often prioritize their efforts. An integrative analysis of extensive ecological spatial data can serve to identify areas of disproportionate value in an ecosystem for consideration when management decisions are made. Our assessment of ecological value and vulnerability had two primary objectives: 1) use publicly available mapping of species patterns in the Ecological Atlas of the Bering, Chukchi, and Beaufort Seas to identify areas of especially high importance to the function and productivity of this ecosystem, and 2) to apply a spatial representation of ecosystem stressors to evaluate the vulnerability of these areas to anthropogenic impacts, namely: vessel traffic, commercial fishing, energy exploration and extraction, and climate change.   Spatial patterns in these data were analyzed using the Important Ecological Area methodology, which has previously been applied to spatial ecological information in Alaska's Arctic waters. We modified the method to measure value relative to a regional neighborhood, which balanced differences in species diversity across the study area and resulted in more well-distributed and geographically diverse importance values. We assessed ecological value across hexagonal bins and ecological units for four taxa groups: lower trophic organisms, fishes, birds, and marine mammals; as well as a combined ecosystem assessment. We then combined relative measures of stressor exposure with the ecological values to produce a prioritized index of the most vulnerable and valuable areas, offering insight into the interface between ecosystems and human activity to inform management and further research.}

\normaltalk{\capitalisewords{Use of Thermal Imaging for the Study of Seabirds in the Gulf of Alaska: Implications for Management and Conservation}}{\underline{Emmylou A Kidder}, Richard Sherwin}{The interaction between seabirds and their habitat has become increasingly important due to evidence of population declines associated with rising temperatures and habitat loss. Shifts in the spatial ecology of the pelagic and coastal habitats used during stages of seabird life-cycles determines the timing of breeding and reproductive success of many seabird species. Fluctuating ocean temperatures, prey abundance, and habitat availability has led to multiple die-offs and nesting failures occurring along the Gulf of Alaska such as Black-legged Kittiwakes in 2017 and Common and Thick-billed Murres in 2018. This study focuses on utilizing infrared thermal technology to investigate the correlation of habitat use and behavior on the thermoregulatory needs of seabirds. Boat-based surveys were conducted to collect thermal data of seabirds in the Kenai Fjords NP and the Alaska Maritime NWR using a thermal camera (FLUKE Ti50) and a thermal video camera (FLIR Rev 110 H-series) in summer 2018. Thermal video footage was collected of seabirds engaged in various activities such as resting, swimming, flying, foraging, incubating and predator avoidance to study energetic costs of these activities. In addition, thermal images were collected of birds utilizing diverse substrates such as ice, water, rock and plants in contrasting weather conditions. Study species include a wide variety of arctic and sub-arctic seabirds encompassing alcids, gulls, sea ducks, cormorants, and tube-nosed seabirds.   These data may elucidate how habitat and microclimate influence thermoregulation of seabirds and can contribute to the management and conservation efforts of at-risk seabirds in the Gulf of Alaska.}

\normaltalk{\capitalisewords{Integrating Machine Learning and Citizen Science Data to Generate High-Performing Species Distribution Models for the Globally Vulnerable Gray Tinamou (Tinamus Tao)}}{\underline{Robert A Wiebe}}{Species distribution models integrate distributional information and environmental data to model the geographic ranges of target species. Citizen science data is an impressive and growing source of distributional information that can be used for these models, but pose some difficulties, including in data quality and spatial sampling bias. Various methods have been proposed to mitigate the effects of spatial bias in response data on modeling, but it is of interest to see how these methods perform with real world data. I used random forest machine learning models with data from eBird to investigate methods to maximize distribution model performance for bird species in the Amazon Basin, focusing on the widespread but globally vulnerable Gray Tinamou (Tinamus tao). I tested the effects of sub-selection schemes for presence and absence points on model performance. My results show random forest machine learning methods can improve the accuracy of species distribution models, generating the most informative models for T. tao to date in spite of strong spatial sampling bias. I identified optimal methods to reduce the negative effects of spatial bias in presence and absence data on distribution models, and identified areas of persistent difficulty in using citizen science data with these models.}

\normaltalk{\capitalisewords{Pesticide Exposure and Effects on Migratory Songbirds: Past and Current Threats}}{\underline{Christy A Morrissey}, Margaret Eng, Bridget Stutchbury, Rebecca Stanton, Robert Clark}{Over the last 60 years, agriculture has undergone dramatic changes worldwide, with consequences for many farmland-associated migratory birds. The relationship between agricultural intensification and avian declines has been established in Europe, however links and mechanistic studies for North American migrants have received less study. Through analysis of trends in bird populations associated with farmlands and grasslands in North America, we found declines occurred in 57 of 77 (74\%) species from 1966 to 2013. The greatest concurrence in multiple species declines occurred during 1960s-1980s, a period with high agrochemical use and widespread conversion of grasslands to cropland. Farmland bird declines persist today and evidence suggests pesticides are one of the strongest drivers of farmland bird declines through mortality, sublethal effects and indirectly through reduced food supplies. Our field studies of Tree swallows reveal how intensive agriculture can alter insect food supply causing changes in foraging behaviour, physiology, mass and return rates. Captive and field studies of seed eating White-crowned sparrows or Red winged blackbirds exposed to common and widely used insecticides imidacloprid (neonicotinoid) can reduce fat loads and alter orientation, departure timing and activity.  Farmland landscapes make up a significant proportion of the land cover and bird habitat in North America where there is an ongoing need to evaluate the magnitude of pesticide effects to migratory birds across their range and implement management practices that facilitate reductions in pesticide use.}

\normaltalk{\capitalisewords{Evolutionary Changes Following a Successful Host Shift: The Case of Mycoplasma Gallisepticum}}{\underline{Andr\'{e} A Dhondt}, Keila V Dhondt, Andy P Dobson, Steven J Geary, Dana M Hawley, Wesley M Hochachka, David H Ley, Katie Pflaum, Maritere Reinoso-Peres, Edan R Tulman}{Twenty-five years ago poultry bacterium Mycoplasma gallisepticum successfully shifted to house finches and started an epidemic that rapidly spread across eastern North America, and after a 5-year pause also made it across to the West. We have been able to study this system for the last twenty years and collected bacterial samples from infected finches throughout this epidemic. We found (1) a gradual  increase in virulence once the pathogen had become established in an area; (2) a change in initial pathogen survival in that eastern isolates barely survive in the host after inoculation during the first few  days after inoculation (after which they do grow), while the western isolates survive quite well following inoculation; and (3) a change in severity of  disease caused by a given bacterial load between eastern and western strains, in that eastern strains are more pathogenic than western strains. (4) The adaptation to the new host, however, did not come at no cost. While the MG house finch strain in 1994 survived well in poultry, more recent strains isolated from house finches no longer survive in poultry. We will also report on recent results regarding the competence of house finches as reservoirs for poultry MG strains.}

\normaltalk{\capitalisewords{Whole Genome Sequencing Provides Inference on the Epidemiology of Antibiotic Resistant Bacteria in Wild Birds}}{\underline{Christina A Ahlstrom}, Jonas Bonnedahl, Andrew Ramey}{Antimicrobial resistant (AMR) bacteria are a threat to global health, with an increasing number of bacterial infections no longer responding to once-standard treatments. Although AMR bacteria are naturally occurring, antimicrobials and other anthropogenic exposures have increasingly selected for resistance in commensal and pathogenic bacteria in diverse hosts and environments. Previous surveys suggest that some types of birds, such as landfill foraging gulls (Larus spp.) and scavenging birds of prey, may be useful indicators of AMR in the environment. However the pathways governing the acquisition and dispersal of such bacteria by birds are not well-described. We investigated the role of gulls and bald eagles (Haliaeetus leucopehalus) in the epidemiology of AMR bacteria in Alaska through whole genome sequencing of AMR E. coli. We found extensive genetic diversity of E. coli isolates and antimicrobial resistance genes, but also evidence for strain sharing among bald eagles and gulls through time and space. Furthermore, satellite tracking of gulls marked at the Soldotna, Alaska landfill revealed that the genetic population structure of AMR E. coli originating from locations on the Kenai Peninsula generally mirrored gull movements among locations at which gull feces were sampled. Our findings support complex epidemiological interactions, including bacterial strain sharing between birds, horizontal gene transfer among E. coli harbored by birds, and maintenance and dispersal of AMR E. coli of public health importance by wild bird populations in Alaska.}

\normaltalk{\capitalisewords{Testing the Social Intelligence Hypothesis in Wild Birds}}{\underline{Kelsey B McCune}, Piotr Jablonski, Sangim Lee, Renee R Ha}{Social learning allows naïve birds to learn a behavior by observation and can be more efficient and safer than asocial learning.  The social intelligence hypothesis (SIH) posits that life in complex social groups led to the evolution of cognitive abilities like social learning.  Consistent support for this hypothesis is lacking, potentially because most research has evaluated cognitive abilities of animals in captivity where artificial social environments may prevent the use of optimal social information.  Furthermore, very little research testing the SIH has been conducted in birds.  Previous studies documented social learning in birds in the lab, but little evidence for social learning exists from experiments on wild birds.  We tested the SIH by comparing performance of social Mexican jays to asocial California scrub-jays on a puzzle box task in the wild.  Neither species copied the behavior of the demonstrators, but rather all naïve jays used social information to avoid specific puzzle box locations where demonstrators interacted.  In Mexican jays, observation of trained demonstrators had a strong, positive effect on motivating interactions with the puzzle box in general.  These results provide mixed support for the SIH because social information facilitated participation in the task for the social species, but individuals from both the social and asocial species used social information to inform foraging decisions.  Avian conservation and management strategies can be improved by manipulating the presence of social information, but research on wild birds is critical to predict individual responses and the ecological variables that select for social information use.}

\normaltalk{\capitalisewords{To Burn or not to Burn? Breeding Success of King Rails in a Fire-Managed Coastal Marsh}}{\underline{Susan B McRae}, Katie M Schroeder}{Due to loss of wetland habitat, the King Rail Rallus elegans is now largely restricted to the coastal margins of its historical eastern US range. Considered globally Near Threatened, and a species of priority concern by the U.S. Fish and Wildlife Service, there is considerable interest in managing habitat to enhance population growth. We studied the reproductive ecology of breeding King Rails at a national wildlife refuge on the Atlantic coast where the marsh is managed using rotational prescribed burns. We found and monitored nests over a 7-year period, and related success rates to the timing since burn. King Rails were as likely to defend territories and build nests in habitat that was recently burned as in unburned habitat. Time since burn did not significantly affect hatching success or predation rate. Nests in recently burned areas were slightly more likely to be deserted; having less cover may increase vulnerability to heat stress. Effectiveness of prescribed burns is commonly uneven producing a mosaic of patches of reeds with shallow open water between them that is favorable to rails. When timed appropriately, prescribed burns are an effective means of keeping marsh habitat at an early successional stage and can be an important management tool for marshbirds.}

\normaltalk{\capitalisewords{Functional Changes in Assemblages of Avian Frugivores Following Extinction and Invasion}}{\underline{Samuel B Case}, Corey E Tarwater}{Extinction and invasion can alter ecological processes owing to the loss and gain of functional roles. In vertebrate-dispersed plants, successful dispersal of seeds requires the matching of functional traits between dispersers and plants. For example, avian disperser traits, such as gape width, body mass, and wing shape, influence frugivory and dispersal distance. Thus, shifts in these traits may impact seed dispersal, a critical ecological process. We investigated how avian disperser traits have shifted in the Hawaiian Islands with the loss of most native frugivores and the introduction of a suite of invasive frugivores. Despite the high turnover in frugivore species, the impacts of invasion and extinction on seed dispersal remain unclear. Using museum specimens, we found significant declines in gape width and body mass in the modern assemblage, affecting both ground and above-ground foraging guilds. Hand-wing index did not shift among volant taxa. Maximum gape width values, which set the upper limit on seed size for dispersal, significantly shifted downward among above-ground foragers on the five largest islands, but not among ground foragers. Our results suggest that the pronounced shifting of gape width has likely caused dispersal limitation of large-seeded plants across forest strata. Shifts in body mass have likely reduced frugivory rates and shortened seed dispersal distances. Overall, shifts in functional traits between historic and modern assemblages of avian frugivores have likely narrowed the possibility of trait-matching between birds and fruiting plants, and we postulate that large-seeded native plants are at risk of extinction due to dispersal limitation.}

\normaltalk{\capitalisewords{Using Remote Sensing of Avian Habitat to Direct Bird Conservation}}{\underline{Susan B Bonfield}, David Schimel, Natasha Stavros, Erin Questad, Fabian Schneider, Antonio Ferraz}{Restoring habitat after fire for birds is complicated by lack of information about the plant communities on which diverse bird species depend. On the Angeles National Forest in Los Angeles, California, 60\% of the area has experienced fire within the past 24 years. Forest managers are working to direct their restoration efforts for the benefit of the landscape, e.g. soil stabilization, as well as for native wildlife, but have few tools to help them select plants and identify target areas. Using both field vegetation and point count surveys in conjunction with airborne spectrometry and Lidar measurements, we examined the relationships between post-fire vegetation and bird communities to inform restoration. The avian community studies showed significant associations between certain bird species and particular plants. Field vegetation surveys showed these species to have well-defined elevational limits and remote sensing was able to map the extent of some key species, or correlated plant communities. A particular focus of the study examines revegetation post-fire for the benefit of hummingbirds, which depend on floral nectar resources throughout their life cycles. We found that 50\% of hummingbird foraging by four species was on two plant species.  Results like this help land managers determine if they can depend on spectrometry and Lidar measurements to direct their restoration efforts. We conclude that new technologies to map plant communities and, specifically, bird habitats, may benefit restoration efforts that contribute to bird conservation and management and provide rapid updates in dynamic landscapes and comprehensive data in challenging terrain.}

\normaltalk{\capitalisewords{Pleistocene Glacial Cycles and the Relative Ease of Speciation with Periodic Gene Flow}}{\underline{Ethan B Linck}, CJ Battey}{Pleistocene glacial cycles affected the distribution of species and ecological communities from the poles to the equator, and have been hypothesized to play a major role in generating both extant avian species richness and population genetic structure. Yet the majority of studies testing for Pleistocene effects within or among species using genetic data have focused solely on the expected consequences of the Last Glacial Maximum, ignoring previous periods of isolation and secondary contact. Here, we ask if these dynamics can be detected with population genomic data, and explore its influence on the relative difficulty of speciation. We use empirical atmospheric CO2 concentration data from the Vostok Ice Cores to parameterize biologically realistic simulations of the demographic history of diverging bird populations in the Pleistocene. We further develop a simple analytical model of speciation with periodic (or cyclical) gene flow, and validate it with forward time simulations. We find glacially-mediated cycles of gene flow and isolation leave a detectable signature on numerous common population genetic summary statistics, but cannot be distinguished using allele frequency spectrum-based demographic inference. We also find that even brief lulls in migration dramatically decrease the expected waiting time to speciation. Our results provide a new approach to link Earth history to contemporary patterns of genetic variation, and highlight an important limit of contemporary speciation research.}

\normaltalk{\capitalisewords{Genomics and Color Morphs: Comparison of Whole Genome Sequence Data Between a Melanin-Based and a Carotenoid-Based Species}}{\underline{Subir B Shakya}, Robb T Brumfield, Frederick H Sheldon}{The current feasibility of whole genome sequencing (WGS) and the availability of well-annotated bird genomes allow for investigation of potential genes responsible for avian morphology and behavior. One area that has yielded many positive results is the genomic basis of pigmentation in birds. This is especially true in the case of genes involved in melanin differentiation. Recently, however, other studies have started identifying genes potentially responsible for carotenoid as well as psittacofulvin-mediated coloration in birds. In this study, we apply WGS and resequencing to identify potential genes responsible for melanin-based color morphs in the Great Blue Heron (Ardea herodias herodias/wardi) versus the Great White Heron (A. h. occidentalis) and for carotenoid-based color morphs in the Black-headed Bulbul (P. atriceps atriceps) versus the Maratua Black-headed Bulbul (P. a. hodiernus). We discuss how the abundance of information on the melanin pathway facilitated the discovery of a candidate gene in the case of the herons, whereas a dearth of information on the carotenoid pathway led to our inability to pinpoint potential genes responsible for the carotenoid-based color morph in the bulbuls. We also discuss how population dynamics of the species being studied influenced the efficiency of our WGS studies.}

\normaltalk{\capitalisewords{Comparative Demography at the Core and Southern Edge of the Black-Throated Blue Warbler Breeding Range}}{\underline{Richard B Chandler}, Ryan Chitwood, Will Lewis, T S Sillett, Robert J Cooper}{Climate change is causing the ranges of many species to shift poleward and upward in elevation. However, the effects of climate change on demography are poorly understood, with little information available about the contributions of changes in survival and recruitment to range shifts. We used mark-recapture data collected on black-throated blue warblers from 2002 to 2016 to assess the hypothesis that climate change on the breeding grounds is causing the species' range to shift via population declines at the southern edge. Data were collected at six plots, with three plots distributed over an elevation gradient at the core of the species' range in New Hampshire and at the southern edge of the range in North Carolina. We evaluated the following predictions: (1) density, recruitment, and growth rate should be positively correlated with elevation and latitude, (2) recruitment should decline over time at the southern sites, (3) survival should not be correlated with elevation or latitude because it is primarily influenced by non-breeding season events. Our results were largely consistent with the first and third predictions, with the steepest population decline occurring at the lowest elevation site in North Carolina. Although we found no evidence of trends in either recruitment or survival, recruitment was too low to offset apparent mortality at the low elevation site in North Carolina. Our results generally agree with predictions of climate-induced range shifts, and they suggest that more research is needed to understand the factors influencing recruitment near the southern range margin.}

\normaltalk{\capitalisewords{Behavioral and Spectrographic Analysis of Chatter: A Novel Vocalization in Northern Mockingbirds}}{\underline{Brishauna B Brown}, Christine Stracey}{Northern Mockingbirds are best known for their extensive vocal repertoire. They have the ability to mimic other bird songs as well as man-made sounds. Despite their voice being the basis of numerous studies, our research focuses on a novel vocalization that has yet, to our knowledge, been defined or described.  Chatter, which we define as a vocal exchange between a mated pair at the nest,  seems to be unique from previously documented calls and song. To investigate chatter events, we obtained continuous video footage of mockingbird nests from a prior study. Videos were screened for chatter events and from 15 nests, we collected chatter sound samples and behavioral data from the event. Chatter vocalizations were analyzed using Raven Pro and spectrographs of chatter were compared visually and acoustically to spectrographs of known calls. Chatter syllables differed from feeding prompts directed at nestlings and alarm calls. For the behavioral data we noted duration of the bout, stage of the nest, exchange of food, and mate arrival and departure. The majority of incidents started with the arrival of the male at the nest and involved food exchange from the male to the female who then passed it to the nestlings. In addition to describing a novel vocalization in the Northern Mockingbird, this study provides insight into intra-pair communication at the nest and the potential function of chatter.}

\normaltalk{\capitalisewords{Deep Convolutional Neural Network to Detect Forest Owls of the Pacific Northwest}}{\underline{Damon B Lesmeister}, Zachary J Ruff, Christopher M Sullivan}{Passive acoustic monitoring is an emerging approach for studying avian populations that leverages recent improvements in autonomous recording units that can be deployed for long periods. A central challenge of this approach is the task of processing and identifying target species vocalizations in large volumes of audio data. To address this issue, we developed an efficient data processing pipeline using a deep convolutional neural network (CNN) to automate the detection of owl vocalizations in unprocessed field recordings. The initial motivation for the project was to develop alternative approaches to study Northern Spotted Owl populations, but we also trained the network to recognize Barred Owl, Northern Saw-whet Owl, Great Horned Owl, Northern Pygmy-owl, and Western Screech-owl. Although the network's performance varies across species, initial results are promising. Recall, the proportion of calls in the dataset that are detected and correctly identified, ranged from 52\% to 96\% based on raw network output. Precision, the rate of true positives among apparent detections, varied from 0.02\% to 59.6\% and correlated strongly with commonness of the species. For example, precision was higher for Saw-whet Owls than Spotted Owls. In comparison tests for detecting owl calls, the CNN performed better than human observers using commercial audio software. Once validated, our model output is suitable for developing species encounter histories for occupancy models and other analyses. We have deployed the CNN to support long-term, large-scale monitoring of owl populations, and we are expanding our target species list to include other birds as well as several mammalian species.}

\normaltalk{\capitalisewords{Migratory Dynamics of Beringian Dunlin on the East Asian-Australasian Flyway}}{\underline{Richard B Lanctot}, Benjamin J Lagasse, Stephen Yezerinac, Stephen Brown, Steve Kendall, Christopher Latty, Joe Liebezeit, Rebecca L McGuire, Martin Robards, Sarah T Saalfeld, Jon Slaght, Michael Wunder}{Understanding when and where birds migrate is important to identify critical areas for species (or populations) of conservation concern. This is particularly true for the four subspecies of Beringian Dunlin (Calidris alpina) that breed in the Arctic and migrate along and winter in the East Asian-Australasian Flyway (EAAF). Extensive degradation of intertidal habitats in the EAAF are likely negatively affecting these birds, and could explain the low adult survival and long-term population decline of the Alaska (arcticola) subspecies of Dunlin. We analyzed geolocator data belonging to 91 individuals from four subspecies of Dunlin breeding at eight Arctic locations to determine their spatiotemporal distribution during the nonbreeding period. Subspecific migratory networks indicate that the Okhota River region and central Japan exclusively support arcticola Dunlin while the Penzhina River region, the northwest Yellow Sea, and inland China predominantly support sakhalina Dunlin. However, all four subspecies occurred in Sakhalin Island, the East China Sea, and the Yellow Sea. Temporally the Dunlin subspecies generally did not co-occur during south migration (kistchinski migrated first, then sakhalina, and then arcticola), showed variable patterns of co-occurrence in the winter, and co-occurred throughout north migration. These results indicate that subspecific considerations are necessary when determining flyway conservation priorities for Dunlin in the EAAF. Additional efforts involving the capture and subspecific assignment of Dunlin are needed on the nonbreeding grounds to help clarify migratory dynamics at finer geographic scales. Our use of small archival GPS tags will also be presented but worked poorly overall.}

\normaltalk{\capitalisewords{Shared Stewardship for Bird Conservation: Linking Backyard Habitats, Citizen Science and Yard Management}}{\underline{Susannah B Lerman}, Peter P Marra, Robert Reitsma, Brian S Evans, Desiree L Narango, Hannah Clipp}{Although habitat degradation resulting from residential development is recognized as one of the greatest threats to bird populations on a global scale, the resulting yards embedded within the matrix have the potential to contribute to bird conservation. Yards exemplify a coupled natural and human system, and thus present an ideal platform for applying the art and science of translational ecology. From a human perspective, yards represent an extension of the home and thus managed to adhere to social norms. From a bird perspective, management decisions (e.g., inclusion of native plants and nest boxes, application of pesticides, owning outdoor pets) have profound implications on the fate of birds within human-dominated environments. Since avian territories span multiple yards, whereas households manage individual parcels, a mismatch in scales creates a challenge for successful conservation of bird populations in residential landscapes. As part of Neighborhood Nestwatch (NN), a yard-based citizen science program, we investigate how yard management decisions influence vital rates, what social processes shape these decisions, and the consequences of feedback loops arising from connecting people to birds through citizen science. NN scientists mentor participants during annual yard visits which enables us to forge new conservation partnerships, strengthen ecological connections, and co-produce knowledge. These interactions shape the levers for socio-behavioral change; specifically identifying how information flows from participant to neighbors, how management decisions ultimately enhance the quality of bird habitat at both yard and neighborhood scales, and how to create greater reciprocity between birds and people.}

\normaltalk{\capitalisewords{Long-Term Avifaunal Change in Undisturbed Amazonian Rainforest:  Pervasive Declines and Shifting Baselines}}{\underline{Philip C Stouffer}, Cameron L Rutt, Vitek Jirinec, Stephen R Midway}{How are rainforest birds faring in the face of habitat loss and climate change?  Inferences are typically drawn from birds sampled across landscape elements in a narrow temporal window.  In other cases, temporal sequences reveal longer-term patterns, but without controlling for landscape change.  We used  $>$35 years of bird captures from replicated continuous forest sites within a vast area of intact Amazonian rainforest to document declining abundance of terrestrial and near-ground insectivores in the absence of landscape change.  Army-ant followers, which use similar resources, were more stable, as were most midstory insectivores that use ant swarms or participate in mixed-species flocks.  Understory frugivores and midstory omnivores appeared stable or increasing.  Available data provide little insight into mechanisms driving these results, but forest structural change may be implicated.  We applied abundance metrics from these 1980s samples to modern results from fragments and second growth.  This comparison revealed that for many species, abundance in modern disturbed forest differs more strongly from the continuous forest baseline than we had previously described.  The current landscape includes far fewer terrestrial and near-ground insectivores and more midstory omnivores than it did historically, but not just in fragments and second growth.   Any goals for bird community recovery in Amazonian second growth should recognize that a modern bird community will inevitably differ from a baseline from $>$35 years ago.  Changes driven by landscape change may be the easiest to document, but factors affecting continuous forest may have already shifted the 'undisturbed' baseline.}

\normaltalk{\capitalisewords{Juniper Expansion in the Sagebrush Ecosystem: Avian Predator Occupancy Patterns and Implications for Sage-Grouse Habitat Restoration}}{\underline{Aaron C Young C Young}, Tracy N Johnson}{Understanding the implications for altered habitat structure on community composition and species interactions is vital to conservation of multiple populations.  Juniper (Juniperus occidentalis) expansion in the sagebrush ecosystem is one factor associated with declines for Greater Sage-Grouse (Centrocercus urophasianus) populations. The effects of juniper expansion on other groups associated with sagebrush and juniper woodland ecosystems are less well understood. Changes in habitat structure may influence predator-prey interactions, and increased abundance of avian predators for Sage-Grouse, including Common Raven (Corvus corax) and Red-Tailed Hawk (Buteo jamaicensis), have been linked to declines in Sage-Grouse populations. Increased predator abundance as a result of juniper expansion has been one justification for habitat restoration projects including removal of juniper in areas historically dominated by sagebrush shrubs. However, uncertainty remains over how juniper expansion affects occupancy of avian predators. The effects of juniper removal on avian predator occupancy and behavior are unexamined.   In 2017 and 2018 we conducted occupancy surveys (n=188) for the avian predator community in southwest Idaho. Surveys were conducted across 30,000 acres within an area where juniper will be removed in order to support Sage-Grouse conservation. Avian predator occupancy was estimated at multiple spatial scales as function of juniper cover composition and other habitat variables. Our data will help to inform future habitat restoration efforts by more explicitly quantifying the effects of habitat gradients on community composition and individual species occupancy. These data will allow habitat restoration efforts to be targeted so that conservation outcomes for multiple species and communities can be maximized.}

\normaltalk{\capitalisewords{Large-Scale Phenological Dynamics of North American Passerines}}{\underline{Casey C Youngflesh}, Jacob B Socolar, Morgan W Tingley}{In seasonal environments, the timing of ecological events (known as phenology) plays a critical role in ecosystem functionality. Phenological changes have been observed across a number of bird species in recent years, however much of this work has focused on European taxa. Large-scale changes in the breeding phenology of North American birds has not heretofore been assessed. Using observational data from the eBird citizen science platform, we estimated the timing of both arrival and breeding for forest dwelling passerines across eastern North America from 2002 to present. We derived spatially-smoothed phenological estimates using hierarchical spatial autoregressive models for each species and modeled phenological change over both space and time. We investigated the relationship between bird arrival phenology and breeding phenology, to evaluate to what degree arrival might be an honest indicator of breeding phenology. We further assessed how the estimated time interval between arrival and breeding changes through space and time. Results from this work have implications for better understanding the phenological responses of North American birds under global change and what this might mean for phenological mismatch dynamics, whereby species are temporally out of sync with favorable environmental conditions.}

\normaltalk{\capitalisewords{The Importance of Geography and Environment for Diversification: a Case Study in the Northern Double-Collared Sunbird Cinnyris Reichenowi}}{\underline{Jacob C Cooper}, Dylan Maddox, Kellie McKague, John Bates}{The African highlands are known for their both their high biological diversity and for their geographic isolation across the continent. Despite the distances between major highland regions (up to ~2,300 km), there is a large overlap in species composition between many of these regions, suggesting high interconnectivity during the last Ice Age. Many species that occur across the African highlands exhibit limited morphologic or phenotypic divergence, with some of the most widespread and fragmented species being considered monotypic. We analyzed one of these widespread taxa, the Northern Double-collared Sunbird Cinnyris reichenowi, that currently has one western and one eastern subspecies recognized. We collected morphological data from all named taxa (including several that have been synonymized), and we gathered genetic sequences from tissue samples and specimen toepads from major populations via anchor-enrichment of ultraconserved elements (UCEs). Using these UCEs, we created matrices of single-nucleotide polymorphisms to study population genetics. We found previously undocumented patterns of genetic diversification within Cinnyris reichenowi despite a lack of complete morphological or ecological separation between genetically divergent populations. Pronounced geographic barriers were not always the greatest predictors for genetic clustering, suggesting that other factors can lead to isolation and divergence within Cinnyris lineages. We discuss the implications of these results for our understanding of ecomorphological niche divergence and for the patterns of speciation within and across the African highlands.}

\normaltalk{\capitalisewords{Estimating the Effect of Prairie Strips on Grassland Birds}}{\underline{Jordan C Giese}, Lisa A Schulte, Robert W Klaver}{The STRIPS project (Science-based Trials of Rowcrops Integrated with Prairie Strips) is a long-term, interdisciplinary agricultural research venture led by Iowa State University. The project aspires to determine how integration of strips of native prairie vegetation into agricultural landscapes can provide benefits to agriculture and wildlife. To investigate breeding bird use of prairie strips in row crops, we conducted point count surveys at 12 study sites throughout Iowa during May-July, 2015-2018. Each site was divided into fields representing three land cover types: conventional crops, crops with low diversity grass strips, and crops with high diversity prairie strips. We calculated avian abundance, richness, and Shannon-Weaver diversity of each field. Birds were 71\% more abundant in fields with prairie strips than conventional fields. However, diversity and richness were similar across all treatments. This is likely due to the strong response of a few species [e.g., Red-winged Blackbirds (Agelaius poeniceus), Dickcissels (Spiza americana), and Common Yellowthroats (Geothylpis trichas)] to prairie strip installations. Dickcissels responded the strongest and were three times more abundant in fields with strips. Establishment of prairie strips within agricultural fields does not favor area-sensitive species like Bobolinks (Dolichonyx oryzivorous) and Henslow's Sparrows (Ammodramus henslowii) but can benefit other declining grassland species. Through increases in biodiversity and reductions in soil erosion, water runoff, and nutrient loss, prairie strips offer substantial improvements to the long-term sustainability of agriculture in the Midwest.}

\normaltalk{\capitalisewords{Nest Initiation Time and Distance to Water Predict Nest Success of the Hawaiian Stilt in Wetlands on O'ahu}}{\underline{Kristen C Harmon}, Nathaniel H Wehr, Melissa R Price}{In temperate waterbirds, early nesters often have higher nest success than late nesters, but because nesting seasons are typically longer in tropical regions and many birds are year-round residents, tropical waterbirds may lack an early-bird advantage. The Hawaiian stilt (Himantopus mexicanus knudensi), an endangered Hawaiian waterbird, nests from December to August in wetlands across the Hawaiian Islands. The objective of this study was to determine the relationship between nest initiation time and nest success of stilts in wetlands on the island of O'ahu. Weekly field surveys and game cameras were used to monitor nests. Nest status, approximate nest initiation time, and nest-site characteristics, such as distance to water, vegetation height, and vegetation cover, were recorded for each nest. Nest success was not significantly different between early and late nesters. However, early nests were located closer to water than late nests (ꭕ2= 6.38, p=0.01) and distance to water was positively correlated with nest success (z=1.96, p=0.04). Correspondingly, nest initiation time was a significant predictor of nest success (z= -2.39, p=0.01). Water level may be predictive of other factors, such as food availability, and therefore, parental involvement or nest-site competition. Further investigation is needed to better understand the impact of water level, as it corresponds to distance from nests to water, on nest success, as well as to identify additional factors that are predictive of nest success. The results of our study will inform decisions regarding management of Hawaiian stilt nesting habitat statewide.}

\normaltalk{\capitalisewords{Not so Secretive Marshbirds: Using Technology to Research a Cryptic Species, the Clapper Rail}}{\underline{Elisa C Elizondo}, W G Shriver}{The Clapper Rail (Rallus crepitans) is a secretive saltmarsh bird which occurs across the east coast of the United States.  While the sound of their territorial calls is often heard during the breeding season, visually sighting them is more difficult to achieve. The saltmarsh is a difficult habitat in which to conduct avian research; it can be treacherous to navigate, but more importantly, human observers traversing the soft terrain can seriously damage the integrity of the habitat. The logistical difficulties of saltmarsh research coupled with the secretive nature of rails has resulted in few data on their populations. Auditory surveys in the northeastern United States indicate that Clapper Rail populations are declining, but vital rates and land use have yet to be established.  In an effort to study Clapper Rail home range sizes adult breeding season survival, and productivity we deployed both VHF and GPS tags in Delaware, USA.  Preliminary data from 2018 found a broad variation in home range size with 75\% kernel estimates ranging from 1.67 ha to 54.79 ha (n = 4).  In 2019, the incorporation of an Unmanned Aerial Vehicle (UAV) equipped with a Forward-Looking Infrared camera (FLIR) will be used to detect heat signatures at nest sites.  UAVs with FLIR have been used successfully in other avian species, most notably waterfowl, and preliminary flights show promising results with rails. These technologies will likely provide previously unattainable vital rate and habitat use data for cryptic species.}

\normaltalk{\capitalisewords{Geographic Variation in Host Immunogenetics and Malarial Parasite Infection in a Catharus Thrush Species Complex}}{\underline{Naima C Starkloff}, Morgan A Sammons, Helena Westerdahl, Wendy C Turner, Jeremy J Kirchman}{Unlike free-living organisms that generally follow the latitudinal diversity gradient, different avian malaria parasite lineages exhibit contrasting diversity and prevalence gradients with latitude. This may be due to ecological differences among their vector organisms but may also reflect differences in host-parasite interactions mediated by evolved host immunity. Parasites and hosts engage in a co-evolutionary "arms race", resulting in a tug of war of evolved traits for parasite persistence and host resistance. Host genes involved in immune response are under strong selective pressure to mediate infections that have negative fitness consequences for birds. A classic case is the selection for high copy number and high polymorphism in the Major Histocompatibility Complex (MHC), in which greater allelic diversity enables the recognition of a greater diversity of infective microorganisms. Using next generation sequencing, we identify the diversity of MHC alleles in a large sample of individuals (n=476) of the closely-related Gray-cheeked Thrush (Catharus minimus), Bicknell's Thrush (C. bicknelli) and Veery (C. fuscescens) from 20 localities spanning $>$17 degrees of latitude in eastern North America. We find that the diversity of MHC alleles among the three species of thrushes varies significantly but find no linear relationship between MHC allele diversity and the diversity or prevalence of malaria parasite infections. We also investigate the interaction of avian malaria and the MHC by testing whether the presence or absence of specific MHC alleles affect the likelihood of infection by different malaria parasite lineages.}

\normaltalk{\capitalisewords{Role of Food Availability on a Bird's Response to a West Nile Virus Infection}}{\underline{Jen C Owen}, Alan P Dupuis, Hannah Landwerlen, Laura Kramer}{Studies in ecological immunology suggest that migrating birds, when faced with limited resources and energy, may alter or suppress their immune system. Yet, we do not know whether this change in immune function negatively affects a bird's ability to effectively respond to an invading pathogen. Further, in the case of zoonotic pathogens for which birds are the primary host, we do not know if this suppressed immune system may enhance their ability to become a 'superspreader' of pathogens.  We investigated whether short-term food deprivation affects a bird's ability to resist a viral pathogen using wild-caught, captive American robins (Migratorius turdus) and West Nile virus.  Hatch-year robins, naïve for WNV, were captured during fall migration and housed in a biosafety level three facility.  We found that birds that experienced a short-term food deprivation immediately prior to being exposed to WNV, had higher virus titers than birds fed on a diet to maintain their body mass. Furthermore, the sum virus of all the food-deprived birds represented 95\% of the cumulative virus titer for all the WNV infected birds in both groups. While all the birds survived the infection, two of the food-deprived robins exhibited clinical signs of WNV infection for a few days and then recovered. These results suggest that lack of food can affect a bird's susceptibility to a pathogen as well shape their ability to serve as a reservoir for a zoonotic pathogen.}

\normaltalk{\capitalisewords{Avian Genomic 'bycatch' Facilitates the Study of Symbiont Ecology and Evolution}}{\underline{Spencer C Galen}, Susan Perkins}{In recent years there has been an explosion of avian genomic data as sequencing bird genomes and transcriptomes has become routine for studies of avian ecology and evolution. However, it is becoming increasingly recognized that most, if not all, avian genomes and transcriptomes contain a diversity of non-target symbionts (parasites, mutualists, and commensals). Identifying symbionts within genomic datasets is important not only for analysis of the host species under study, but also because these symbiont genetic resources have the potential to contribute to our understanding of the ecology and evolution of the diverse organisms that many birds interact with. Here, we demonstrate the importance of considering symbionts in avian genomic research using a dataset of 25 transcriptomes generated from blood samples of wild birds. In each sample we detected at least one, and as many as six, distinct infections of malaria parasites (order Haemosporida). An average of 6.5\% of all assembled transcripts were determined to be derived from malaria parasites, though this number was as high as 16.8\% in the transcriptome of Vireo plumbeus. Importantly, we were able to assemble malaria parasite transcripts from samples that had no detectable infections using traditional screening approaches. In total, we identified at least 45 individual malaria parasite infections from 25 sampled hosts, from which we were able to generate multi-species alignments for over 3,000 protein coding genes. In sum, we show that avian genomic resources are capable of providing unprecedented insight into the evolutionary history of a globally important group of avian parasites.}

\normaltalk{\capitalisewords{Unidirectional Introgression Between Rhegmatorhina Hoffmannsi and R. Berlepschi in the Amazon Forest}}{\underline{Glaucia C Del-Rio}, Marco A Rego, Bret M Whitney, Fabio Schunck, Lu\'{i}s F Silveira, Brant C Faircloth, Robb T Brumfield}{At hybrid zones, different species meet, mate and generate individuals of mixed ancestry. Because hybrid zones are areas where reproductive isolation is not complete, they are natural laboratories for understanding how species boundaries are maintained in the face of hybridization. Here we study a hybrid zone in the most biodiverse area in the planet, the Amazon Basin. At the Sucunduri River headwaters, the distributions of two antbirds (Thamnophilidae), Rhegmatorhina hoffmannsi (with white chest) and R. berlepschi (with brown chest), meet at a hybrid zone containing individuals with recombinant plumage patterns. To investigate the processes involved in the maintenance of the hybrid zone, we used geographic and genomic cline analyses to characterize genetic and morphological transitions across the zone. We sequenced the whole-genome of one individual of R. hoffmannsi, and generated RADseq and ND2 data for 212 samples spanning the zone. We characterized plumage color of 212 museum specimens using reflectance spectrometry. We found that: (1) the mitochondrial cline is shifted 140 km south of the autosomal cline center, in the direction of hoffmannsi (2); the cline for chest color is shifted 110 km north of the autosomal cline center, in the direction of berlepschi; (3) the majority of hoffmannsi alleles (93\%) are found on the genomic background of berlepschi. These results suggest a pattern of unidirectional introgression in which hoffmannsi nuclear genotype and white plumage phenotype are moving towards berlepschi populations.}

\normaltalk{\capitalisewords{Early Stage of the Speciation Process: Two Coloration Genes Underlie Species Differences in a Pair of Capuchino Seedeaters (Sporophila)"}}{\underline{Mar\'{i}a C Estalles}, Leonardo Campagna, Mar\'{i}a J Rodriguez Cajarville, Pablo L Tubaro, Irby Lovette, Dar\'{i}o A Lijtmaer}{The southern capuchinos constitute a recent radiation of ten sympatric South American seedeaters in the genus Sporophila, which differ phenotypically (color and song) but do not show neutral genetic differentiation. Recent analyses showed that some of these species differ in areas of the genome related to the melanogenesis pathway, presumably as a consequence of sexual selection. While capuchinos tend to differ in various plumage patches, S. hypoxantha and S. ruficollis only differ phenotypically in the presence of a black plumage patch on the throat of the latter species. We analyzed the complete genome of 15 S. ruficollis and 12 S. hypoxantha individuals. Only two areas of the genome presented significant divergence between these species, and all the SNPs with Fst ≥ 0.85 were placed in these areas, showing that S. ruficollis and S. hypoxantha are the least genetically differentiated pair of capuchinos. The differentiated areas include the genes HERC2 and TYRP1, both of which are involved in the melanogenesis pathway. These results confirm that these two species are in an early stage of the speciation process, still lack neutral genomic differentiation and are only separated by two genes involved in melanogenesis that could be responsible for the different color of their throats. Behavioral experiments linking these findings with species recognition and mate choice are underway to further assess the role of sexual selection in the speciation process.}

\normaltalk{\capitalisewords{Genomic Differentiation along the Speciation Continuum in Three Hummingbird Species Pairs}}{\underline{Elisa C Henderson}, Alan Brelsford}{Identifying the genetic basis of speciation is a prominent goal of evolutionary biology. Previous work has found heterogeneous patterns of genomic differentiation in many species pairs, characterized by regions of elevated differentiation (Fst peaks) throughout the genome. These regions were originally thought to indicate genomic barriers to reproduction, but this initial hypothesis has fallen out of favor: under this hypothesis Fst peaks are expected to coincide with increased absolute divergence (dxy), but the opposite pattern is emerging from studies of multiple species pairs. Two other models, termed "recurrent selection" and "sweep-before-differentiation", have been proposed to explain this pattern. Here we aim to distinguish between these models of speciation using whole-genome resequencing data from pairs of hybridizing species in three hummingbird genera: Calypte, Archilochus, and Selasphorus. Each of these species pairs has a geographic range of overlap, but they represent different points on the speciation continuum based on divergence time and frequency of hybridization. Using this dataset, we will look for signatures of selective sweeps within species and across species pairs. This study will contribute to our understanding of heterogeneous genomic divergence and the role of hybridization in speciation.}

\normaltalk{\capitalisewords{A Genomic Comparison Between Flycatchers Only Distinguishable by Innate Song Differences}}{\underline{Natalia C Garcia}, Leonardo Campagna, Andrew C Rush, Rauri CK Bowie, Irby J Lovette}{Comparing the genomes of different species can help identify genomic regions linked to variation in phenotype, one of the great challenges of evolutionary biology. Here, we applied a whole-genome sequencing and re-sequencing approach to analyze the levels of differentiation between the Alder and Willow flycatchers (Empidonax alnorum and E. traillii) and between the Pacific Slope and Cordilleran flycatchers (Empidonax difficilis and E. occidentalis). Each pair of sister species was taxonomically separated less than 50 years ago, mainly due to differences in vocalizations, as they are mostly indistinguishable based on plumage coloration and morphology. Analyses of mitochondrial DNA suggest that the Alder and Willow flycatchers have been diverging from each other for a longer period (2.3 millions of years) than the Pacific Slope and Cordilleran flycatchers (350,000 years), which still hybridize in areas of sympatry, resulting in individuals with intermediate songs. We recovered ~ 21 million SNPs (single nucleotide polymorphisms) between the Alder and Willow flycatchers, with a mean Fst of 0.25, and a large proportion of SNPs with very high Fst values. Between the Pacific Slope and Cordilleran flycatcher we found ~ 7 million SNPs, and a lower mean Fst of 0.06, with only a small number of SNPs with Fst $>$ 0.7. These results are congruent with the previously estimated divergence times for these sister taxa. The shallow genomic differentiation between the Pacific Slope and Cordilleran flycatchers opens the possibility of detecting candidate areas of the genome related to innate song variation, the primary mean of mate recognition between these taxa.}

\normaltalk{\capitalisewords{Phylogeny of the Hawaiian Thrushes (Myadestes) Based on UCE and Mitochondrial DNA Sequences}}{\underline{Robert C Fleischer}, Nancy McInerney, Carl E McIntosh, Sarah Sonsthagen, Storrs Olson, Helen James}{The Hawaiian thrushes include two morphologically differentiated lineages that have been allied with the New World genus Myadestes.  One lineage (the "large thrushes") is represented by extinct and extant taxa (species, subspecies or populations) from each of six main islands (Kauai, Oahu, Maui, Molokai, Lanai, and Hawaii) while the second lineage contains a single extant, but endangered, species on Kauai (the Puaiohi, Myadestes palmeri).  Previous molecular analyses of the extant Omao (M. obscurus) of Hawaii supported the placement of the Hawaiian thrushes within Myadestes.  We obtained mtDNA sequences from most of the continental and all HawaiianMyadestes, from either fresh tissue, museum specimens, or fossil bones (in the case of the form on Maui).  We also obtained roughly 2.6 million bp of nuclear DNA sequences via capture of UCEs from the extant and some of the extinct taxa.  mtDNA sequence analyses of the extinct forms show that they form a clade with M. obscurus, but are moderately differentiated from it.  These data also support the species level status of four taxa within the large thrush lineage, corresponding to forms on Kauai, Oahu, Maui-nui, and Hawaii, in spite of their morphological uniformity.  Myadestes palmeri, while also falling within Myadestes, is divergent from the large Hawaiian thrush lineage.  Resolution with mtDNA alone was poor, and it was possibly the result of a separate colonization of the archipelago.  However, the UCE sequences provide extremely high resolution (100\% bootstrap support of all nodes), and analyses strongly support a single clade and colonization with a rather rapidly ensuing split into the large and small thrush lineages.}

\normaltalk{\capitalisewords{Influence of Reproductive Status of Greater Sage-Grouse on Habitat Selection in a Juniper Dominated Landscape}}{\underline{Jordan C Rabon}, Peter S Coates, Mark A Ricca, Tracey N Johnson}{Greater sage-grouse (Centrocercus urophasianus, hereafter, sage-grouse) within the Great Basin have experienced substantial loss of habitat from the expansion of conifer woodlands into sagebrush ecosystems.  Individual sage-grouse vary in habitat selection patterns as related to conifer expansion in areas dominated by sagebrush, yet a previously unexplored source of variation is female reproductive status.  Thus, we quantified habitat selection patterns for hens during different periods of the breeding season across a gradient of western juniper (Juniperus occidentalis, hereafter, juniper) cover in the Owyhee Mountains of southwestern Idaho during 2017-18.  We collected multi-scale habitat data at known-use locations of hens with (n = 11) and without (n = 16) broods and classified locations as early brood ($<$25 days old; n = 21), late brood (≥25 days old; n = 23), early non-brood (April 29-June 14; n = 23), and late non-brood (June 15-July 31; n = 25).  At the micro-habitat scale, we observed no difference in perennial forb cover or juniper stem density between brood and non-brood locations for either time period.  Shrub height did not differ between early brood and non-brood locations but shrubs at late brood locations (x ̅  = 59.1cm ± 1.2, SE) were taller than non-brood locations (46.1cm ± 0.8; P=$<$0.01).  We will evaluate additional habitat variables at larger spatial scales to further investigate multi-scale habitat relationships.  Understanding variation in habitat selection associated with reproductive status will likely guide more effective conifer treatment planning to improve sage-grouse habitat.  Findings are preliminary and provided for timely best science.}

\normaltalk{\capitalisewords{Multi-Scale Approach to Evaluating Space Use in Invasive Avian Dispersers and the Implications for Seed Dispersal}}{\underline{Rebecca C Wilcox}, Corey E Tarwater}{Spatial overlap between fruiting plants and dispersers is required for seed dispersal in vertebrate-dispersed plants. Spatial mismatches between plants and dispersers may result in reduced frugivory or recruitment owing to limited encounters or seeds being deposited in inhospitable areas. Seed dispersal of native plants is threatened by the introduction of new dispersers which may vary in their diet and space use compared to native dispersers. The first step in evaluating whether spatial matches or mismatches are occurring is to examine space use within avian dispersers. Here we examined variation in space use in four invasive birds at the understory and landscape scales on Oahu, Hawaii. We selected habitat features that influence avian habitat selection and plant recruitment, including ones associated with topography, light environment, plant community composition, and habitat complexity. We found that landscape variables explained more variation in avian space use than understory variables. Thus, future analyses evaluating spatial matches/mismatches should be performed at the landscape scale. We also found that although all four species are successful invaders, they are not homogenously spread across the landscape and they differ in their space use. Thus, both spatial matches and mismatches are likely occurring and may be a mechanism contributing to dispersal facilitation or limitation depending upon the plant species. Lastly, we found that patterns were not consistent across scales. This study highlights the importance of taking a multi-scale approach and suggests that further studies are needed to examine whether spatial matches/mismatches alter seed dispersal in a novel ecosystem.}

\normaltalk{\capitalisewords{Are We Sure About Shearing? Evaluating the Impact of a Widely Used but Poorly Assessed Habitat Management Practice}}{\underline{Anna C Buckardt Thomas}, Amber M Roth, Darin J McNeil Jr, Kirsten Johnson, Amanda D Rodewald, Jeffery L Larkin}{Many North American bird species associated with young forests and shrublands have experienced population declines due to loss of breeding habitat. Successful conservation of these birds relies, in part, on restoring or mimicking natural disturbances to create habitat. Periodic mowing or shearing of shrubs and saplings is a strategy used by managers to create young forest or shrublands for a variety of wildlife species. However, little research has focused on evaluating avian response to shearing. We used Golden-winged Warbler (GWWA, Vermivora chrysoptera) as a model species to measure the impact of shearing on male abundance. We conducted breeding season point counts in 2012, 2013, and 2015-2018 at 252 sites across Minnesota and Wisconsin in three habitat-management types: mature alder shrublands, sheared alder shrublands, and sheared aspen saplings. In 2015-2018, we estimated habitat metrics at 231 of the sites. We used integrated-likelihood modeling in a distance sampling framework to assess the importance of 1) management variables (like habitat-management type, time since shearing, and cover type), and 2) vegetation metrics (such as woody stem cover, canopy cover, and grass cover) on male GWWA abundance. Model selection suggested that habitat-management type and time since shearing are important predictors of male abundance, with the lowest abundance in mature alder shrublands and the first season after shearing. Vegetation models showed a positive linear effect of grass cover and a negative quadratic effect of 1-2 m tall woody regeneration on male density. We found shearing to be a viable management strategy for increasing male GWWA abundance.}

\normaltalk{\capitalisewords{Duration of Corvid Response to Forest Restoration Thinning: Implications for Conservation of the Threatened Marbled Murrelet}}{\underline{Joan C Hagar}, Lorraine K Waianuhea}{The silvicultural practice of thinning to increase timber yield has been adapted to address contemporary ecological goals of forest management on public lands under the hypothesis that it will hasten development of old forest habitat for threatened species such as the marbled murrelet (Brachyramphus marmoratus). However, short-term effects of thinning for old-growth associated species may be negative, including an increased risk of nest predation. Predation by corvids is the greatest cause of marbled murrelet nest failure and corvids are attracted to disturbance. Although thinning is common adjacent to old-growth reserves, corvid response to thinning has not been well studied. To address this information gap, we used point count survey data that were collected over 15 years in a long-term silviculture experiment in the Oregon Cascades. We used a Before-After-Control-Impact (BACI) design to: (1) quantify differences in corvid detection rates among treatments representing variation in thinning intensity; and (2) determine the influence of thinning intensity on the duration of corvid response over more than a decade following thinning. Our preliminary results indicate that corvid detection rates increased immediately after thinning, but generally decreased over time with expansion of tree canopies. Preliminary findings also suggested that the duration of the period of potentially increased risk of nest predation following thinning adjacent to murrelet nesting habitat is a function of thinning intensity. A better understanding of the duration of the effects of thinning can help managers balance short- and long-term goals for conservation of species associated with old-growth forests.}

\normaltalk{\capitalisewords{Conservation Implications of a Migratory Network for the Common Nighthawk}}{\underline{Elly C Knight}, Autumn-Lynn Harrison, Amy L Scarpignato, Steven L Van Wilgenburg, Erin M Bayne, Peter P Marra}{Conservation of migratory species is complicated by the varying environmental conditions that populations experience across the annual cycle. Describing migratory connectivity across the range of a species, or the migratory network, can provide insight into when and where within the annual cycle conservation efforts are required. First, the degree of migratory connectivity can help identify vulnerability of individual populations and important areas during migration. Second, conditions experienced across the annual cycle can be correlated with population-specific attributes such as population trend to understand potential drivers of those trends. The Common Nighthawk (Chordeiles minor) is a long-distance migratory bird with one of the largest breeding ranges in North America. Available data suggest the Common Nighthawk is declining, along with other aerial insectivore species in North America. We described the Common Nighthawk migratory network by deploying Argos-GPS tags on individuals from 12 populations across Canada and the US and tracking them across their annual cycle. All tagged individuals appeared to migrate to the Mississippi flyway and then used a single migration route to South America. We thus found little migratory connectivity between populations during migration. The connectivity on the wintering grounds was also low, but individuals returned to the same breeding territory the following year. Our results suggest that differential population trends for this species may be due to limitations on the breeding grounds or interactions between population-specific attributes and environmental conditions at other stages of the annual cycle.}

\normaltalk{\capitalisewords{Preferred Orientation of Vagrant Passerines During Autumn Migration}}{\underline{Lucinda C Zawadzki}, Philip D Taylor, Tim Guilford}{Vagrancy, the phenomenon by which passerines travel to areas far beyond the boundaries of their known species' range, remains largely unexplored due to the seemingly unpredictable nature of these events. Though various theories have been postulated as to how and why this behavior occurs each autumn, few studies have directly quantified the occurrence of this behavior in the field. We directly assessed the preferred migratory bearing of vagrant passerines in the field to determine whether vagrants orient in a different direction from what is expected given their historic migratory routes. We also tested whether vagrants exhibit different orientation behavior than non-vagrant individuals. We hypothesized that non-vagrant individuals would orient along their normal migratory pathway, while vagrants may exhibit one of three behaviors: 1) continued orientation along the migratory path needed to reach the study site, 2) re-orientation towards the historic migratory pathway, or 3) orientation towards the nearest land mass. Vagrant and non-vagrant birds were captured on Bon Portage Island, Nova Scotia during autumn migration, and their orientation behavior was recorded in outdoor video-based orientation cages at sunset. Individuals exhibited a preferred migratory orientation. This study highlights the importance of studying vagrancy directly in the field.}

\normaltalk{\capitalisewords{Genomic Signatures of Pleistocene Dry Forest Connections in a Widespread South American Songbird}}{\underline{Eamon C Corbett}, Gustavo A Bravo, F\'{a}bio Schunck, Luciano N Naka, Lu\'{i}s F Silveira, Scott V Edwards}{South American dry forests are an important and understudied biome, with a complex and poorly understood biogeographic history. Based on the fragmented distribution of many Neotropical dry forest species, Prado and Gibbs (1993) suggested that this habitat was more widely distributed under drier climate conditions in the Late Pleistocene. But this "Pleistocene Arc Hypothesis" has rarely been tested genetically, much less with genome-wide data. We used ddRADseq in combination with mtDNA sequencing to sample 7,167 genome-wide single-nucleotide polymorphisms from 74 individuals across the range of the Rufous-fronted Thornbird (Phacellodomus rufifrons), a widespread dry forest bird with a disjunct distribution closely matching that of the biome itself. We found low genetic differentiation across two prominent geographic breaks – including, most strikingly, a 1,000 km gap between populations in Bolivia and Northern Peru. Using demographic analyses of the joint site frequency spectrum, we also found evidence of recent divergence without subsequent gene flow across those breaks. This pattern is highly consistent with the idea that currently disjunct patches of dry forest were more connected in the recent past, likely during the Middle and Late Pleistocene. This scenario was also supported by paleoclimate species distribution modeling. By contrast, we found a deep genetic break with recent introgression between morphologically distinct parapatric populations in Brazil, suggesting that there may be a previously unrecognized hybrid zone between them. Our results support the major predictions of the Pleistocene Arc Hypothesis and illustrate the importance of genome-wide sampling of widespread species for examining biogeographic and evolutionary questions.}

\normaltalk{\capitalisewords{Population Genomics of Hawaiian House Finches (Haemorhous Mexicanus): Investigating Genetic Variation of an Introduced Population}}{\underline{Amelia-Juliette C Demery}, Scott Edwards, Allison J Shultz}{Exploring the patterns and possible causations of temporal and spatial genetic variation in invasive species provides a window in which to glimpse the first steps of speciation due to novel environmental pressures and demographic history. The House Finch (Haemorhous mexicanus) is native to the western United States but dispersed to the East Coast and Hawaiian islands via two separate anthropogenic introductions. Given the results of past research identifying unique genetic divergence between the mainland populations, the Hawaiian House Finches present a unique opportunity to compare how colonization drives genetic variation across different geographies. Using whole-genome resequencing of 24 individuals from four islands (Hawaii, Kauai, Maui, Oahu), we explored how numerous microevolutionary processes can drive rapid population divergence in an introduced species immediately following the founder event. Validating past analyses of population structure within the system, we found that all four islands clustered separately from the mainland populations. Among the islands, the populations were structured by island geography but clustered Oahu and Kauai separately from the other islands. Oahu had the highest number of rare alleles while Kauai had the lowest, suggesting that Oahu went through a population expansion since the colonization. We also identified genomic overlap between individuals from Maui and Hawaii, suggesting that these islands could have shared original founder event diversity or are experiencing recent or ongoing gene flow. Continued work will identify which demographic patterns best explain the unique population structure of the Hawaiian House Finches and whether their divergence from the mainland populations holds biological significance.}

\normaltalk{\capitalisewords{Behavioral Landscape of 'Oma'o Vocalizations in a Naturally Fragmented Habitat}}{\underline{Erin C Netoskie}, Kristina L Paxton, Eben H Paxton, Patrick J Hart}{Habitat use has long been studied by ecologists to understand how organisms utilize their environment, but they often do not evaluate the diversity of behaviors that coincide with resource selection. We combined movement and vocalization data to create a behavioral landscape model that identifies where different types of vocalizations (i.e. song, call, whisper song) most frequently occur across the landscape, while linking these locations with their underlying habitat features. Using automated telemetry towers, we collected precise location data from radio-transmittered '\={O}ma'o (Myadestes obscurus), a species of thrush endemic to Hawai'i Island, to establish core zones and movements across the landscape. In addition, we recorded individual bird vocalizations using a new transmitter technology, which broadcasts all sound emitted from an individual through a miniature microphone. We found calls were more strongly associated with shrub habitat in the matrix while whisper songs were associated more often with the forested kīpuka. These results indicate that behaviors such as calling while foraging in the matrix or the use of whisper songs for courtship may occur within different habitat types. Furthermore, as distance from a bird's core zone increases, total time spent vocalizing decreased. Determining the effects of habitat type, composition, and distance from the core zone may clarify context specific uses of sound across the landscape. Moreover, this study provides a comprehensive methodology to document the relationship between habitat use and vocalizations, which can be applied to many taxa across different ecological landscapes.}

\normaltalk{\capitalisewords{Sampling Design Considerations for Making Strong Inference from Large-Scale Monitoring Programs}}{\underline{David C Pavlacky}}{Study designs for avian point counts are rooted in data standardization to account for observation processes.  Over recent decades, model-based inference has become a strong attractor in ornithological research.  Current trends in large-scale population monitoring increasingly focuses on model-based inference with less attention to design-based sampling theory.  Paradoxically, an increase in the complexity of model-based inference parallels a decline in the presentation of data collection methods that are fundamental to repeatability and replicability in science.  Both design-based and model-based paradigms have strong theoretical underpinnings, but there is considerable confusion regarding applicability of the inferential approaches to sampling and estimation.  It is important to understand the assumptions, advantages and disadvantages of the inferential approaches to effectively address various monitoring objectives.  For example, many assume that probabilistic sampling designs are infeasible for large-scale population monitoring, yet discussion of trade-offs in strength of inference along a continuum of experimental design is often overlooked.  I will discuss the applicability of design- and model-based inference for addressing various monitoring objectives, catalog important assumptions and outline steps in the sampling process using examples from the Integrated Monitoring in Bird Conservation Regions program.  The applications of model-based inference to avian conservation will likely increase with the quantity of opportunistic data, but because of growing concerns about landscape and climate change, as well as uncertainties in population responses to management, the integration of design- and model-based approaches will likely improve strength of inference and provide better outcomes for large-scale bird conservation.}

\normaltalk{\capitalisewords{Cardinals are Red, Buntings are Blue; Because They Hybridize, They're Interesting Too!}}{\underline{Libby C Megna}, Matthew D Carling}{Understanding how reproductive isolation arises between populations is critical to understanding the process by which species diverge. However, when two species are completely reproductively isolated, it is difficult to assess which isolating mechanisms are operating. Hybridization between species reveals information on which isolating mechanisms have failed upon secondary contact. Furthermore, one of the most salient questions in evolutionary biology is whether we can predict evolutionary processes; can we find overarching patterns in how reproductive isolation is maintained among passerine species? Here, we determine if there are repeated patterns of trait divergence for species in the family Cardinalidae, and if those patterns can predict hybridization between species. Twelve of the 48 species in Cardinalidae hybridize. We quantified several traits for each species: plumage coloration, climatic niche, bill morphology, and song. We used phylogenetically informed GLMMs to examine the association between trait values and hybridization. Our preliminary results indicate that Neotropical migrant species are more likely to hybridize than tropical resident species; that species with greater frequency bandwidth songs are more likely to hybridize; and that species with redder plumage are more likely to hybridize, whereas species with more blue are less likely to hybridize. It is generally thought that each instance of hybridization between species has its own evolutionarily unique history, and thus dynamics of hybridization between two species pairs cannot be generalized to other species pairs. However, our results suggest that there are overarching patterns in how reproductive isolation arose within Cardinalidae.}

\normaltalk{\capitalisewords{Long-Term Responses to Environmental Change Across Multiple Scales in the Great Tit}}{\underline{Ben C Sheldon}}{Long-term studies of birds have provided extensive evidence of changes in phenology correlated with large-scale changes in climate. In this talk, I use evidence from the long-term study of the great tit Parus major in Wytham Woods, near Oxford, UK, from 1960 to date, to explore how our understanding of this process has been limited by a limited focus on the scale of response. I will first describe work that reveals the importance of phenotypic plasticity in driving population-level responses to climate change. I will then describe work that shows that important components of seasonal timing are also driven by processes at very small spatial scales, implying that there are individual optima for timing of reproduction. I will also show that there may be considerable within-population variation in the rate of change in phenology over time. Taken together, these perspectives imply a complex set of processes occurring over a range of different spatial scales, and I will discuss the implications of this complexity for understanding resilience to changing environments.}

\normaltalk{\capitalisewords{Revealing the Migratory Path and Wintering Areas of Olive-Sided Flycatchers that Breed in Alaska}}{\underline{Julie C Hagelin}, James A Johnson, Michael T Hallworth}{The Olive-sided Flycatcher (OSFL; Contopus cooperi) is a neotropical migrant of long-standing concern, due to a 76\% decline in North America over the last 40 years.  Documenting the annual movements, including important stopover and wintering locations, of OSFL is fundamental to understanding threats and taking appropriate conservation actions.  We report on a multi-year effort (2013-2017) to deploy light-level geolocators on adult OSFL breeding in central and southcentral Alaska and a pilot effort to deploy GPS tags.  We retrieved geolocators from 14 breeders (12 male, 2 females), 3 of which provided data over two consecutive years.  We identified 13 stopover locations that Alaska birds used over their ~22,000 km (~13,500 mi) annual migrations.  During both fall and spring, individuals stopped in eastern coastal and southern Mexico; only 43\% of birds crossed the Gulf of Mexico. Northwestern Columbia/Panama and the Pacific Northwest, U.S., both stand out as important stops during spring only. Wintering occurred in two general areas: (1) western Columbia/Ecuador/northern Peru, and (2) southern Peru/western Brazil.  Subsets of our data suggest wintering patterns worthy of further investigation.  First, consecutive years of geolocation data revealed high inter-annual winter site fidelity (n = 3 birds).  Second, pilot GPS data suggest a discrete wintering territory (e.g. "cluster" of points between Nov-Mar, all falling within 150 m; n = 1 bird).  Future GPS deployments are advised to consider technological issues or possibly species-specific interference (e.g. biting at units), as 7 of 8 GPS tags we recovered failed to collect data.}

\normaltalk{\capitalisewords{Cross-Cultural Considerations in Wildlife Management and Conservation}}{\underline{Crystal C Leonetti}}{Indigenous peoples from five main cultural groups have lived in Alaska for thousands of years. By living in close contact with nature, these peoples have developed extensive traditional knowledge on birds, other wildlife, and ecosystems in their regions. In Alaska and elsewhere, indigenous peoples nowadays coexist with western cultures, share biological and natural resources, and are stakeholders and partners in research and resource management and conservation. This presentation will draw from experience working with indigenous peoples in Alaska to share key concepts important to understand different communication styles; learn from other cultures; incorporate traditional ecological knowledge in research, management, and conservation; and identify opportunities to work with local Indigenous peoples. Effective cross-cultural communication and partnerships are key to support meaningful participation of stakeholders from diverse cultural backgrounds and achieve shared conservation and management goals.}

\normaltalk{\capitalisewords{Genoscapes \& Species Limits - Lessons from the Last 5 Years}}{\underline{Kristen C Ruegg}, Thomas Smith, Christen Bossu, Jasmine Rajbandary}{Most populations of migratory birds are now threatened. It is estimated that the populations of 1 out of every 2 songbirds are declining in the Western Hemisphere with impacts predicted to worsen with climate change. However, because migratory birds have both breeding and wintering areas and may move vast distances between them, understanding where the steepest population declines are occurring has been difficult or impossible. To address this critical issue, researchers at Colorado State University and the University of California, Los Angeles have launched the Bird Genoscape Project, an effort to map the population-specific migratory routes of migratory songbirds by harnessing the power of genomics. Over the course of the last five years, the Bird Genoscape Project has worked with 100's of collaborators and volunteers across the country to map migratory flyways in $>$10 species of migratory birds, spanning multiple families and life histories.  In this presentation, I will review lessons learned regarding the advantages and potential pitfalls associated with a genomic approach to migratory flyway mapping and discuss future directions for the project based upon our results to date.}

\normaltalk{\capitalisewords{Endless Forms less Beautiful: Relevance of Morphological Analyses to Integrative Taxonomy of Asian Prinia Species Complexes}}{\underline{Pamela C Rasmussen}, Per Alstrom, Urban Olsson}{Although most Asian Prinia species are common, widespread, and often vocal and conspicuous, their species limits have received comparatively little recent study.  Despite being less morphologically uniform than some other Old World warbler groups, most prinias exhibit significant confounding geographic, sexual, seasonal, and age-related variation. We examined species limits within several long-recognized Prinia species complexes—Striated Prinia P. crinigera, Brown Prinia P. polychroa, Yellow-bellied Prinia P. flaviventris, Graceful Prinia P. gracilis, Plain Prinia P. inornata, and Hill Prinia P. atrogularis—using genetics, vocalizations, and morphological analyses. These analyses congruently show that each taxon long recognized as a single species is comprised of at least two well-defined clades, most of them allopatric, that we consider specifically distinct. Morphological study allowed us to detect a zone of sympatry between two highly genetically diverged clades in the P. crinigera species complex, and to show that two taxa (P. crinigera and P. polychroa), long considered separate species on the basis of sympatry, actually do not overlap geographically. While morphologically distinctive taxa also usually exhibit congruent levels of vocal and genetic divergence in Asian Prinia, in at least one case (Bornean race latrunculus of P. flaviventris) a morphologically distinctive taxon lacks significant divergence in other character sets. Conversely, in numerous other cases, morphologically very similar Prinia taxa are more highly diverged vocally and/or genetically, usually being strongly structured geographically. We discuss strategies for, and lessons learned during, study of large series of such morphologically challenging taxa.}

\normaltalk{\capitalisewords{The Impacts of Diversity and Assembly History on Community Vulnerability}}{\underline{Brian C Weeks}, Joseph Tobias}{Biodiversity is positively associated with a broad range of ecosystem functions and properties; some of these properties, for example resistance to invasion, are expected to impact extinction. As such, understanding the persistent impacts of the relationship between biodiversity and ecosystem functioning (BEF) across evolutionary time is critical to applying BEF theory to natural systems. Using functional trait data from $>$99\% of bird species, we characterize the relationship between diversity and assemblage vulnerability, or the collective vulnerability of species within an assemblage, for the world's birds. Further, while differences in community assembly history, for example the order of arrival of species into a community, can change BEF relationships at ecological timescales, it is unknown whether the impacts of assembly history persist across evolutionary time. We demonstrate how a global scale analysis, and the complementary information within three dimensions of diversity (functional, taxonomic, and phylogenetic), allow us to characterize a gradient of community assembly history, and to test whether differences in assembly history have impacted the relationship between diversity and assemblage vulnerability. Our findings have implications for conservation interventions, explore the outcomes of BEF relationships as they play out over evolutionary time scales and in natural systems, and present a novel framework for characterizing community assembly history that can be used broadly to test whether differences in assembly history have had persistent impacts on contemporary ecology.}

\normaltalk{\capitalisewords{Nest Survival, Predator Assemblage, and Patterns of Predation at Veery (Catharus Fuscescens) Nests in a Mature Forest}}{Christopher B Goguen, \underline{Les D Murray}}{Predation is typically the primary cause of nest failure for forest songbirds, however, surprisingly few video camera studies have been performed in forests of northeastern North America. Over 5 years, we used video cameras at Veery nests in a mature forest landscape in northeastern Pennsylvania to identify the predator assemblage and to describe relevant patterns of predation. Overall, we located and monitored 289 nests and recorded 67 of these nests with video cameras. Nest survival varied substantially across years (0.17-0.44) with this variation driven primarily by the dominant source of nest mortality, predation. Video cameras recorded 40 predation events at 36 nests by at least 12 predator species. Seven mammal species accounted for 75\% of the predation events recorded, 4 species of birds comprised 20\% of predation events, and a single predation by a snake was recorded. Depredation events occurred during the day more often than expected by chance, but there was no pattern of differences in predation rates among nesting stages. Logistic exposure analyses indicated that both camera presence and nest age affected nest survival by influencing predation rates, but date did not. The diverse predator community observed depredating Veery nests led to low nesting success in some years, but also demonstrated that causes of nest failure vary by year and cannot be attributed to any single predator species or group.  Difference in predation rates at nests with cameras, however, suggests some bias in the documented predators and annual nest survival estimates in some years due to cameras.}

\normaltalk{\capitalisewords{Extinction Debt and Species Credit in Tropical Forest Bird Communities: Conservation Implications for Two Biodiversity Hotspots}}{\underline{William D Newmark}}{Most species extinctions following habitat loss are delayed. Extinction debt is the number or proportion of species that will eventually become extinction as a community reaches a new equilibrium.  Conversely, the number or proportion of species that can be conserved through habitat restoration is termed species credit.   Using an empirical relaxation time-area relationship for tropical bird communities, we estimate the relaxation half-life or time that it will take to lose one-half of all species in tropical forest fragments. Using this same relationship, we estimate the enhanced persistence time of tropical bird communities or species credit in forest fragments by restoring continuous forest among forest fragments in two highly fragmented tropical biodiversity hotspots. In the Eastern Arc Mountains in Tanzania, regenerating 8,474 ha among the largest and closest fragments at nine location would enhance persistence time for species by a factor of 6.6 per location or ~2,293 years, on average, relative to individual fragments. In the Atlantic Forest, regenerating 6,452 ha of forest at two locations would enhance species persistence time by a factor of 13.0 per location or ~5,102 years, on average, relative to individual fragments.}

\normaltalk{\capitalisewords{Evolutionary Genomics of the Brown Creeper}}{\underline{Joseph D Manthey}, John Klicka, Garth M Spellman}{The Brown Creeper (Certhia americana) is a widespread, forest-associated North American songbird. Previous work has shown the species to include several phylogeographic clades associated with known regional forest assemblages. The two major lineages diverged around two million years ago. We assembled a high-quality genome of the Brown Creeper (N50 $>$ 30 Mbp) and re-sequenced several individuals of the two major lineages, including individuals from pure populations (allopatric) and contact zone populations (sympatric) in southern Arizona. We found that genetic structure between the two lineages is largely shaped by variation in chromosome size and locally shaped by recombination rate. We investigated patterns of genomic variation between the two lineages in several genomic marker types, including single nucleotide polymorphisms, indels, inversions, and transposable elements. Using these genomic variants, we looked for signatures of natural selection across the genome and across marker types. Lastly, we used several methods to test for introgression between the two lineages in the contact zone in southern Arizona.}

\normaltalk{\capitalisewords{Migratory Pathways, Stopover Locations, and Wintering Destinations of Flammulated Owls Breeding in Colorado}}{\underline{Brian D Linkhart}, Scott W Yanco, James W Fox}{Population dynamics are strongly influenced by patterns in annual movements, particularly for long-distance migrants, where choices of migratory destinations and stopover habitats directly and indirectly affect individual performance on breeding grounds. We documented patterns in migratory movements of Flammulated Owls (Psiloscops flammeolus) using light-level geolocators (12 males) and GPS loggers (one male) from 2009-present. Males departed Colorado breeding areas for autumn migration from $<$5-21 Oct, arrived on wintering areas in Mexico from 3 Oct-3 Nov, departed wintering areas from $<$6-21 Apr, and returned to Colorado between 3-21 May. Core wintering areas for most males were located in central and southern Mexico, where they were primarily associated with the Trans-Mexican Volcanic Belt and Sierra Madre del Sur Mountains. Fall migrations were characterized by two flight pathways, with most males taking a southern path to central Mexico and some males taking a southeastern path to eastern Mexico, while spring migrations were characterized by a singular pathway along the Sierra Madre Oriental Mountains to the Rio Grande River, and north through New Mexico. The most prolonged stopovers by males during both migrations occurred in New Mexico, 300 km from breeding areas. Two males tracked over two years showed similar patterns in migration routes and wintering areas between years. Vegetation within core wintering areas, and at stopovers, were primarily associated with evergreen forest/woodland, suggesting conifer forests are important throughout the owl's annual cycle.}

\normaltalk{\capitalisewords{Idiosyncratic Changes in Spring Arrival Dates of Pacific Northwest Migratory Birds}}{\underline{W D Robinson}, Christina Partipilo, Tyler A Hallman, Karan Fairchild, Jim Fairchild}{Shifts in the timing of bird migration have been associated with climatic change. However, climatic change does not affect all species or geographic locations equally. Climate in the Pacific Northwest has shifted during the last century with mean temperatures increasing by 1C but little change in total annual precipitation. Few long-term data on migration phenology of birds are available in the Pacific Northwest. We analyzed trends in spring arrival dates from a site in the Oregon Coast Range where nearly daily inventories of birds were conducted across 25 years. Several species showed statistically significant shifts in timing of first spring arrivals. Six of 18 species occur significantly earlier now than during the initial phase of the study. One species arrives significantly later. Eleven show no significant shifts in timing. We associated trends in spring migration phenology with weather (precipitation and temperature), traits of species such as migration strategy, foraging behavior, diet, and habitat use, and regional trends in abundance as indexed by Breeding Bird Survey data. We found no set of variables consistently correlated with avian phenological changes. Only variables associated with occurrence of warm early springs, probably indicative of fewer moisture-laden systems that might slow northward progress, appeared regularly in models, but even those variables were relatively uninfluential in the arrival trends of many species. We conclude that either some previously unrecognized variable or combination of variables has affected the timing of migration in some species but not others, or statistically significant directional changes over time can occur without being driven by consistent environmental or species-specific factors.}

\normaltalk{\capitalisewords{Fine-Scale Genetic Population Structure of an Understory Bird (Thamnophilus Atrinucha) Across the Panama Canal Zone}}{\underline{J D Maddox}, J P Kelley, Corey E Tarwater}{Birds are frequently defined by their ability to move across the landscape, but studies have demonstrated that many tropical birds are poor dispersers. Consequently, landscape characteristics (e.g., extent of fragmentation, intervening habitat type, fragment size) are likely to affect how individuals move about the landscape, which in turn may alter the genetic structure within and between populations. Here we determine how landscape connectivity or lack thereof affects fine-scale population genetics of a small understory bird (black-crowned antshrike, Thamnophilus atrinucha) across multiple spatial scales. We obtained samples from approximately 400 individual antshrikes located at 20 sites that varied in size (range: 6 to $>$1000 ha), degree and type of fragmentation across the entire Panama Canal Zone (≈ 65 km). To increase our ability to detect genetic differences we produced thousands of genetic markers by preparing and sequencing ddRADseq libraries. Our initial results are consistent with previous experimental manipulations that indicate antshrikes do not traverse water barriers greater than 300 m as pairwise Fst values were significantly different between Barro Colorado Island and Parque Nacional (PN) Soberania. We also discuss how gene flow among the 20 sites is affected by landscape characteristics and whether populations in small fragments have experienced drift or inbreeding.}

\normaltalk{\capitalisewords{Using Integrated Models to Identify Ecology and Population Dynamics of a Threatened Arctic Species}}{\underline{Kylee D Dunham}, Asheber Abebe, Stephen F Dobson, David N Koons, James B Grand}{Recent developments in statistical modeling have provided a framework to integrate multiple data sources to provide better estimates of population size, trend, and vital rates, as well as formally describe important ecological relationships. Integrated population models (IPMs) are a method developed to analyze two or more data sets on population demography (capture-recapture data, fecundity data, and population counts or indices) in a single modeling framework. Spectacled eiders (Somateria fischeri) are an excellent model for studies of population dynamics in changing environments because they are inextricably linked to their marine habitats and exposed to significant threats under global change. Multiple studies monitoring demographic parameters (survival, productivity, abundance) have produced long-term data sets. We leverage these data sets to estimate vital rates including those that are unobservable, estimate population size and trend, and quantify ecological relationships. Abundance has increased since listing in the 1990s and population growth is stable. We found strong support for a nonlinear relationship between extreme sea ice days in the core wintering area and adult survival. Additionally, we found evidence of a nonlinear relationship with extreme sea ice days and hatch year survival. Nest success was strongly negatively correlated with fox abundance potentially creating a threat to future viability and recovery of the species. The global population of spectacled eiders winters together and ice conditions are projected to become anomalous in the core wintering area. It is likely that extreme sea ice conditions would negatively impact successive breeding population dynamics through decreased survival.}

\normaltalk{\capitalisewords{Frontline Defenses Against a Purported Mafia Brood Parasite}}{\underline{Brian D Peer}, Patrick Page}{Hosts of obligate avian brood parasites are expected to evolve defenses to minimize the costs of parasitism. Small hosts incapable of ejecting parasitic eggs should invest more in frontline defenses to prevent parasitism from occurring in the first place. Likewise, hosts subjected to mafia tactics employed by brood parasites should aggressively defend their nests to avoid being forced to accept parasitism. In North America only a single host species, Prothonotary Warbler (Protonotaria citrea), has been reported being subjected to mafia enforcement by the parasitic Brown-headed Cowbird (Molothrus ater). In 2018 we tested the frontline defenses of Prothonotary Warblers by presenting taxidermically-prepared mounts at 37 nests. These mounts included a Brown-headed Cowbird, a competitor for cavity nests (House Wren, Troglodytes aedon), and a non-threatening control species (Northern Cardinal, Cardinalis cardinalis). We predicted that warbler responses toward the brood parasite and nest predator would be similar, and more aggressive than their responses toward the control species. Our study is important in determining whether frontline defenses are an effective strategy against mafia behavior by brood parasites.}

\normaltalk{\capitalisewords{It's the Same Old Song: Benefits of Song Dialects in the Lapland Longspur}}{\underline{Robert D Montgomerie}, Amelia Cox, Adele Mullie}{Based on analysis of two years of song recordings at our study site on the Melville Peninsula, Nunavut, we found that most male Lapland Lonspurs sang remarkably similar songs with respect to syllable length and structure, number of syllables and the order in which syllables were sung. At this site, the typical local song type also differed markedly from the songs that males sang at a site only 50 km away. As well, a sample of songs from around the world suggests that there are many local dialects but we cannot tell whether they are distinctive or grade into one another with no dialect boundaries. In one year at our site, 16 males that sang the typical local song were more likely to get mates than 5 males that sang foreign songs, and their overwinter return rate was higher than that of foreign males. Adult breeding philopatry was high, and there was no detectable change in the typical local song over at least 8 years. Using this study as a basis we develop a conceptual cost-benefit model for dialect formation in birds (and humans) that considers the conditions under which dialects should evolve and be maintained. We test this model with a comparative study of dialects in birds.}

\normaltalk{\capitalisewords{Multiple Reproductive Costs Due to Light Pollution: Insights from Manipulative Experiments \& Continental-Scale Data Synthesis}}{\underline{Clinton D Francis}, Masayuki Senzaki, Danielle M Ferraro, Jesse R Barber}{Variation in photoperiod, which is the length of exposure to illumination, serves as a primary, and evolutionarily stable, cue for controlling circadian rhythms and phenology. Yet global growth in artificial-night lighting has the potential to not only create mismatches between cues and highly conserved physiological and behavioral responses, but impair sensory capabilities via several mechanisms. Here, we use manipulative field experiments and continental-scale analyses of NestWatch data and control for other anthropogenic stressors, such as presence of humans, landcover changes or anthropogenic noise, either experimentally or statistically. We demonstrate that artificial-night lighting can strongly advance clutch initiation and decrease clutch size and reproductive success, especially among birds that inhabit open environments where vegetation does not limit the propagation of light pollution. Importantly, effects of artificial lighting on nesting phenology are dependent on natural photoperiod – birds at lower latitudes with shorter breeding-season daylengths experience stronger advancement in clutch initiation due to artificial-night lighting. Additionally, manipulative experiments where photoperiod was lengthened by several hours reveal that exposure to artificial-night lighting decreases the condition of Western bluebird nestlings relative to those in dark nests, but the decrease in condition was especially acute when exposed to both night lighting and anthropogenic noise. This insight is especially important given that these stimuli often co-occur across the landscape. Provided that artificial-night lighting is increasing by 2\% per year globally, our findings suggest that a bright future may present serious challenges for birds and understanding the full costs of this novel sensory stimulus is paramount.}

\normaltalk{\capitalisewords{Community-Level Response to Cumulative Effects of Forestry and Energy Development}}{\underline{Judith D Toms}, Thea M Carpenter}{In the past few decades, the boreal forest of northern Alberta has been rapidly altered by the extensive footprints of forestry and energy development. We provide a broad summary of the impacts of this development on the boreal bird community, including new work testing the hypothesis that the bird community is beginning to undergo a process of biotic homogenization. Human footprints associated with forestry and energy sector development have significant impacts ($>$20\% changes in population densities) on a substantial proportion (30-45\%) of the bird species present, with populations of some species increasing and others decreasing in response to these changes. Within the boreal bird community, generalists tend to be 'winners' (increasing), while specialists and conifer-associated species tend to be 'losers' (decreasing). Preliminary results suggest that, as a result, the bird community is beginning to undergo a process of biotic homogenization, with a loss of species and functional diversity.}

\normaltalk{\capitalisewords{Genomic and Morphological Approaches to Understanding Speciation in Jamaican-Endemic Streamertail Hummingbirds}}{\underline{Caroline D Judy}, Robb T Brumfield, Gary R Graves}{The Jamaican streamertail hummingbirds (Trochilus polytmus and T. scitulus) represent a potential rare example of avian in-situ speciation to occur on a small oceanic island. These spectacularly plumed island endemics form a narrow hybrid zone where their ranges meet in eastern Jamaica. Streamertail hummingbirds are highly volant, yet bill color, the secondary sexual ornament that distinguishes the taxa, changes from ruby red  (T. polytmus) to jet black (T. scitulus) over less than a few kilometers. Here, I use 6,451 genome-wide SNPs from a genotyping-by-sequencing dataset to resolve the structure of the hybrid zone and discriminate the species. Using a combination of multivariate and population genetic techniques, I recovered signal for two genetic clusters that match species definitions and identified 33 SNPs that were informative for species discrimination. The range of admixture assignments (0.03 - 93\%) in the hybrid zone populations supports the hypothesis that hybrid zone individuals represent a spectrum of parental types, F1 hybrids, and advanced backcrosses. Geographic cline models fit to the subset of informative SNPs (n = 33) mirror the narrow clinal patterns seen in two diagnostic morphological traits: bill color and bill width. I discuss these genomic and morphological patterns in the broader context of hybrid zone dynamics and speciation.}

\normaltalk{\capitalisewords{How Do Individual Attributes Predict Bird Conservation Behaviors Among Birders in the United States?}}{\underline{Terrell D Rich}}{There is a serious disconnect between birder interests, needs for bird conservation, and public policies for bird conservation in the United States. An estimated 45 million Americans feed and/or watch birds. At the same time, 233 species of birds (22\% of all native species) need conservation action. Numerous studies have shown long-term declines in the number of hunters, fishers, and other consumptive users. Despite these megatrends, the policies of public agencies – most importantly state wildlife agencies – are fixated on the past. To better respond to evolving public values and bird conservation needs, it is necessary to gain a better understanding of the values, motivations, and behaviors of birders. Such an understanding can be used to craft public policies that not only better serve people and birds, but also enhance the long-term relevance of the agencies themselves. A survey of 2139 of the most dedicated birders in the U.S. revealed that, 1) 42\% of respondents scored bird conservation second only to acquiring more skills in identification among their top priorities, 2) very few ($<$ 5\%) were members of major ornithological societies, 3) age and interest in bird conservation were inversely correlated, 4) females were more interested in bird conservation than males and were more inclined to social birding events, and 5) all age groups would like to get more information about bird conservation. These and many other links between the attributes of all types of birders, not just the dedicated, and their bird conservation behaviors will be examined.}

\normaltalk{\capitalisewords{Land Trusts and Birds: Partners in Strategic Conservation}}{\underline{Amanda D Rodewald}, Sara B Swarthout, Ashley A Dayer, Ron W Rohrbaugh}{More than 60\% of the land area in the United States is privately owned, and more than 100 bird species have $>$50\% of their U.S. breeding distributions on those lands. Unfortunately, conserving private lands is complicated by both individual and institutional barriers, thus leaving birds reliant on private lands with inadequate protection and management. Land trusts are an increasingly popular mechanism to protect private lands and potentially conserve birds and their habitats. In 2013, the Cornell Lab of Ornithology used social science-based methodology to conduct a national, online survey of land trusts and their attitudes toward bird conservation. Results indicated that land trusts, if supported with science and technology, could achieve landscape-scale conservation for birds. To develop mutually beneficial collaborations between land trusts and the bird conservation community, we established the Land Trust Bird Conservation Initiative. The initiative provides: 1) access to science-based information about birds to inform strategic conservation planning, investment decisions, prioritization of easements and acquisitions, grant writing, and landowner engagement; 2) resources and tips on bird-focused funding opportunities; 3) ideas to cultivate new members and volunteers by engaging birdwatchers, bird organizations, and bird conservation advocates; 4) guidance for habitat management on fee-owned lands and resources for landowners holding easements; 5) connections with science-based bird conservation resources and land trust success stories; and 6) assistance with monitoring birds and visualizing data through eBird.org. We will summarize the survey and describe how the Initiative is using science and outreach to conserve bird populations on private lands.}

\normaltalk{\capitalisewords{Learning to Join the Information Web}}{\underline{Robert D Magrath}, Chaminda Ratnayake, Andy Radford, Jessica McLachlan, Branislav Igic, Dominique Potvin}{Many birds and mammals gain useful information about predators by eavesdropping on the alarm calls of other species. This raises the puzzle of how individuals recognize these calls, because alarm calls vary widely in acoustic structure among species. Recognition might be possible because alarm calls of different species share key acoustic properties, or because individuals learn to recognize other species' calls. I consider these possibilities in our research on Australian birds, which reveals that while call structure can affect response, learning is critical in enabling individuals to tailor responses to the local community and to explain taxonomically widespread eavesdropping. Our work includes geographic comparisons and learning experiments carried out on wild birds. We show overall that individuals gain valuable information about danger from heterospecifics, and that learning–including social learning–helps individuals join the ecological "information web".}

\normaltalk{\capitalisewords{The Utility and Significance of Population Genomic Data in Assessing Avian Species Limits}}{\underline{Michael D Sorenson}}{Large multi-locus genetic data sets generated using genotype-by-sequencing (GBS) methods such as restriction site associated DNA sequencing (RAD-Seq) are becoming increasingly common in analyses of avian speciation. Data sets comprising thousands of loci offer the power to detect subtle patterns of population structure, and to correctly assign individuals to populations even in the absence of any fixed genetic differences. This raises the question of whether it is legitimate or meaningful to "diagnose" species (or subspecies) based on the successful assignment of individuals to distinct groups based on multi-locus genotypes. To inform discussion of this question, I present both empirical data sets and simulations to examine the power of multi-locus data to distinguish individuals from different populations, and how this changes with divergence time, gene flow, effective population size and the number of loci analyzed. I propose that "multi-locus diagnosis" should be viewed as a necessary but not sufficient criterion for recognizing species when the context is allopatry, whereas it is potentially decisive when correlated with phenotypic divergence in the context of sympatry. In contrast, empirical examples demonstrate that divergence and/or reciprocal monophyly in mitochondrial DNA (mtDNA), long touted as the best single locus for detecting historical isolation, is neither necessary nor sufficient as a criterion for delineating avian species.}

\normaltalk{\capitalisewords{Machine Annotation of Puaiohi Calls in Environmental Acoustic Recordings}}{Grady S Weyenberg, \underline{Madori D Rumpungworn}, Lindsey O Howells, Kristina L Paxton, Lisa H Crampton, Patrick J Hart}{Falling hardware costs has made large-scale passive acoustic monitoring of wildlife increasingly affordable in recent years. Wildlife conservationists and scientists have a variety of potential uses for bioacoustic data, but usually require accurate annotations as a first processing step. Manual methods for producing annotations of these recordings (using Raven, for example) are time consuming, tedious, and prone to human error. Using machine learning techniques to assist in this task has the potential to greatly reduce the costs associated with producing these annotations.  In this study, we employ a deep neural network to produce an automated annotation pipeline targeting the Puaiohi (M. palmeri). Puaiohi are a critically endangered songbird endemic to the Alakai plateau on the island of Kaua'i, Hawaii, USA, with an estimated population of only 500 remaining in the wild. They are threatened by multiple factors, such as climate change, habitat loss, and mosquitos carrying avian malaria. The call of the Puaiohi is particularly amenable to automatic detection, due to its simplicity and regularity. Using a human-annotated dataset consisting of approximately 30 hours of forest environmental recordings, we trained a deep convolutional network to annotate environmental recordings. The automatically generated annotations feature accuracy and recall which are similar to a first-pass manual annotation. Using retail commodity hardware,  the network can be trained in approximately 15 minutes and can annotate a new 15-minute sample in under a minute. This demonstrates the utility of machine learning approaches for targeted annotation of environmental recordings.}

\normaltalk{\capitalisewords{Conservation of Boreal Birds: Status, Trends, and Data Gaps}}{\underline{Judith D Toms}, Peter Solymos, Diana Stralberg, Nicole KS Barker, Tati Micheletti, Lionel Leston, Samuel Hache, Steve Cumming, Samantha Song, Fiona Schmiegelow,  B }{We summarize the current state of knowledge to identify recent advances in our understanding of boreal bird ecology and populations. We summarize our understanding based on the spatially extensive dataset compiled by the Boreal Avian Modelling Project. We present methods to estimate population status and trends from ad-hoc data, and link population density to the increasing extent and number of anthropogenic threats to boreal bird populations, both at regional and at continental scales. We will discuss local and landscape level impacts of anthropogenic disturbances and their potential implications for future habitat supply for boreal species. We review these findings in the context of historic population declines, inter-annual variation considering full life-cycles, and projected climate change effects. We conclude by presenting key gaps in knowledge required for effective management and conservation of these species.}

\normaltalk{\capitalisewords{Songbirds of the Sagebrush Sea: Declines and Mechanisms}}{\underline{Anna D Chalfoun}, Lindsey Sanders, Tayler Scherr, Matthew Hethcoat}{Migratory birds that breed in aridlands continue to compose one of the fastest declining groups of birds in North America. The three species of sagebrush-obligate passerines (Brewer's sparrow, sagebrush sparrow, and sage thrasher) fall within this category and have decreased in abundance with concomitant sagebrush habitat loss and alteration. In many cases, the specific mechanisms underlying population declines of avian species are unclear.  Our long-term research in Wyoming, USA has clarified why habitat loss caused by energy development, a major source of sagebrush habitat change in the intermountain west, results in lower reproductive success and smaller local populations of sagebrush songbirds. After energy companies construct new well pads and pipelines on federal land, they are required to reclaim disturbed areas via re-seeding. Reclaimed areas, however, very rarely resemble the composition and structure of undisturbed sagebrush patches. The abundance of deer mice, the most common nest predator of sagebrush songbirds in the area, increases with surrounding reseeded area which leads to higher nest predation rates.  Deer mice make forays into re-seeded areas and consume both the native and non-native grasses and forbs found there. Sagebrush songbirds, moreover, do not settle preferentially in areas with less energy development or surface disturbance, which results in an ecological trap. Management actions that minimize the amount of initial surface disturbance during new development and that improve the restoration efficacy post-development will likely increase the nesting productivity of declining sagebrush birds.}

\normaltalk{\capitalisewords{Hatching Date Influences Winter Habitat Quality: Examining Seasonal Interactions Across the Full Annual Cycle in Prairie Warblers}}{\underline{Michael E Akresh}, David I King, Peter P Marra}{Previous studies on carry-over effects in migratory birds have focused on how non-breeding season events influence breeding season performance. Here, using stable carbon isotopes, we examined carry-over effects initiated during the breeding season on subsequent life history stages in prairie warblers (Setophaga discolor), a declining Nearctic-Neotropical migratory bird. We found a 'silver spoon' effect, in that juvenile males that hatched earlier at our study site in Massachusetts, USA acquired wetter, better-quality winter habitat in the Caribbean. These individuals also exhibited high site fidelity to the same winter habitat over time, as indicated by high repeatability in isotope signatures for individuals sampled over multiple years. Thus, hatching date presumably influenced lifetime winter habitat quality, which may impact adult survival. Adult males that successfully reproduced did not obtain poorer-quality habitat in the following winter compared to unsuccessful males, indicating no apparent cost of caring for young on winter habitat acquisition. We found few significant full annual cycle (breeding to wintering to breeding) or intergenerational carry-over effects. Overall, our study is one of the first to show that developmental processes such as hatching date can carry-over to influence winter habitat acquisition. Our results contribute to the growing body of literature that emphasizes the importance of seasonal interactions in migratory birds.}

\normaltalk{\capitalisewords{Using Playback Experiments to Investigate Species Discrimination in a Hybridizing Population of Blue-Winged and Golden-Winged Warblers}}{\underline{Amy E Janik}, Maarten J Vonhof, Sharon A Gill}{Species recognition is crucial for reproduction in birds and is usually aided by species-specific vocalizations or songs. When ranges of closely related species overlap, cross-species learning may occur, eliminating the associations with signals necessary for species discrimination. In 2017 and 2018, we recorded mismatched and atypical songs in Blue-winged (Vermivora cyanoptera, BWWA) and Golden-winged Warblers (V. chrysoptera, GWWA) in a hybrid contact zone at Fort Drum Military Base in upstate New York. Here, we investigate species discrimination in that hybrid contact zone. Given the intensity of introgression in this region, we hypothesized that GWWA and BWWA show limited or no species discrimination by song. We used playback experiments to test the responses of focal males identified by song phenotype to simulated conspecific and heterospecific male intruders. We presented pre-recorded songs of GWWA and BWWA, as well as atypical songs, in randomized order to males of both species. We ran a principal component analysis on vocalizations, closest approach, and number of perch changes to find overall response scores and then ran linear mixed models with song phenotype and playback treatments as predictors, PC1 and PC2 as overall responses, and male identity as a random effect. BWWA and GWWA responded to playbacks of both conspecific and heterospecifc songs, but whereas BWWA responded similarly to all playbacks, GWWA responded less to BWWA song. Our study suggests weak or limited species discrimination by song in winged warblers.}

\normaltalk{\capitalisewords{Timing of Breeding in American Coots: Bet-Hedging Against Nest Predation when Reproductive Performance Declines with Season}}{\underline{Bruce E Lyon}}{Changes in reproductive performance with laying date are ubiquitous in birds. Most attention has focused on seasonal declines in performance but hump-shaped relationships also occur, particularly in multi-brooded species. Experiments are needed to distinguish between two key mechanisms that drive these temporal patterns: changes in food supply over time versus the fact that higher quality individuals breed earlier. In a study of American coots, nest predation forced many birds to nest later than their original choice and provides a natural experiment to explore mechanisms of temporal change. Comparing the performance of birds forced to renested, both to nests that matched the original nesting date and birds that matched the renesting date, allowed me to assess the relative importance of bird quality and season change in resources as drivers of seasonal patterns of performance. These comparisons revealed a cryptic humped pattern of resource change across time, a pattern not apparent in the observational data because it was masked by the effects of bird quality. The existence of a humped resource pattern raises the questions as to why birds would ever nest before the optimal time with respect to resources. I suggest that nesting before the optimum is an adaptive bet-hedging mechanism when nest predation forces some birds to renest after the maximum when resources are declining.  A simple optimal breeding time model with a humped fitness relationship with time confirms that birds should always nest before the maximum if there is a risk of renesting forced by nest predation.}

\normaltalk{\capitalisewords{Montane Biogeography and the Evolution of Disjunct Ranges: a Case Study in the Diglossa Flowerpiercers}}{\underline{Anna E Hiller}, Robb T Brumfield, Brant C Faircloth}{Among montane ecosystems, the Andes are some of the most species rich, high-elevation avian communities in the world, yet the processes involved in generating this spectacular diversity remain poorly understood relative to our understanding of avian speciation in temperate regions. In particular, few studies have (1) characterized relationships among populations within species across the entire Neotropics, and (2) investigated the role that gene flow plays in population divergence within these montane systems.  Within this framework, we examine the biogeographic history of flowerpiercers in the genus Diglossa using an unprecedented genome-wide dataset collected from population samples spanning species diversity across the entire genus. Diglossa are montane-restricted, inhabit a range of elevations from ~400m to 4300m, and occur across the Neotropics from Mexico to Argentina. Morphological studies on Andean Diglossa have suggested that the diverse plumage patterns found in this group have been influenced by hybridization events, making Diglossa a compelling group in which to examine the role of gene flow in Andean speciation.  Our analyses resolved the phylogenetic placement of the two Tepui-endemic flowerpiercers, D. duidae and D. major, as well as relationships within the D. lafresnayii complex. However, we found that relationships within the D. carbonaria complex remain difficult to disentangle, with the placement of the isolated population D. brunneiventris vuilleumieri remaining particularly elusive. We will discuss the potential causes of this finding, including the role of hybridization in the contact zones in this group and the potential of extinction driving the evolution of disjunct ranges in Diglossa.}

\normaltalk{\capitalisewords{Impact of Wildfires on Barn Owl Habitat Selection in a Vineyard Agroecosystem in Napa Valley}}{\underline{Allison E Huysman}, Matt Johnson}{In the world-renowned wine growing region of Napa Valley, California, wine producers install nest boxes to attract barn owls (Tyto alba) which may reduce rodent crop damage. Annual monitoring of 273 nest boxes began in 2015, and devastating wildfires burned approximately 60,000 hectares in the region in 2017, including homes and businesses, as well as some vineyards and uncultivated habitat. Little is known about how barn owls respond to drastic landscape changes such as wildfires. Occupancy surveys and GPS tagging before and after these wildfires reveal changes in habitat selection at the nest and hunting scales. Owls were found breeding in recently burned areas that were previously unoccupied, suggesting that wildfires may change the landscape in a way that encourages nest box occupancy. Results from occupancy modeling show that nest box occupancy had a positive relationship with burned habitat. Data from GPS transmitters provide further insight into the effects of the fires on hunting habitat selection. These results have implications for the potential of barn owls to provide rodent pest control as vineyard owners increasingly install nest boxes and as wildfires increase in the western United States.}

\normaltalk{\capitalisewords{Quantifying Multiple-Benefits of Bird Conservation in the Agricultural Mosaic of California's Central Valley}}{\underline{Nathaniel E Seavy}, Diana Humple}{In California's Central Valley, achieving bird conservation objectives requires habitat generated from both natural areas (riparian forest and managed wetlands) and wildlife-friendly agriculture.  One strategy we are exploring is the potential for information about the multiple benefits of wildlife-friendly agriculture and natural areas to generate increased support for the conservation of bird habitat in the Central Valley. To this end, we have developed a framework that includes four broad categories of benefits: public health and safety, ecosystem vitality, thriving economy, and enriching experiences. Drawing on published literature, reports, and unpublished data, we have compiled information for more than a dozen metrics across three of these categories. Our suite of metrics included water use, greenhouse gas emissions, soil health, and economic profit. This initial information allows us to identify crops that provide bird habitat and simultaneously generate other societal values, such as protecting water quality and soil health.  This information also helps identify challenges. For example, wildlife-friendly crops tend to require more water and be less economically profitable compared to other crops in the valley.  This efforts illustrates a coarse-scale approach to quantifying multiple-benefits that can be used to establish conservation strategies or generate hypotheses to be tested with field studies.}

\normaltalk{\capitalisewords{Viewing Habitat Through Another Lens: Bird Nest-Site Selection and Productivity Across the Beach Thermal Landscape}}{\underline{Lauren E Schaale}, J B Baxley, Narcisa G Pricope, Raymond M Danner}{Little is known about the fine-scale temperature variation of the coastal landscape (i.e. the thermal landscape), which limits our ability to assess its effects on animal productivity. We can now map thermal landscapes in high spatial resolution using thermal images collected from unoccupied aerial vehicles (UAVs). The images are captured remotely, thus limiting habitat and wildlife disturbance. We studied a Least Tern (Sternula antillarum) breeding colony with respect to its thermal landscape on Lea-Hutaff Island, NC. We monitored 157 nests for hatching or failure. We recorded surface temperatures with a thermoMAP thermal camera mounted on a fixed-wing mapping UAV. We developed methods for UAV design and flight paths to minimize bird disturbance. We recorded ambient environmental conditions at ground-level to help determine which variables influence sand surface temperature and nest success. Our results show a wide range of temperatures across the landscape (25–68.5°C) and that temperature differs with sand type. Tern nest locations varied in surface temperature (42–54.8°C in June) and were a subset of the range of potential temperatures (35–63°C) suggesting that birds did not select the highest or lowest temperatures. Surface temperature at the nest site did not differ significantly between hatched and failed nests (p-value=0.29, t-test). The variation in surface temperature observed in this study leads to ±6°C difference in calculated environmental (i.e. perceived) temperature, which could influence bird physiology, cognition, and behavior.}

\normaltalk{\capitalisewords{Macroevolutionary Dynamics of Egg Colour and Patterning}}{\underline{Catherine E Sheard}, Sally E Street, Camille Troisi, Andrew Clark, Susan D Healy, Kevin N Laland}{The magnificent diversity of egg colouration and patterning has long fascinated biologists, and a wide array of data from local geographic and taxonomic scales point to multiple causes. Here we present a global survey of avian egg colour and patterning, where we have tested five key ecological and social hypotheses across 4,762 species. We first characterised the macroevolutionary trends of avian egg colour and patterning, reconstructing the ancestral egg and finding that the rates of change of both colour and patterning have accelerated substantially toward the present day. We then used phylogenetic comparative models to investigate potential drivers of this acceleration. We found that patterning is correlated only with temperature, but that colour can be associated with nest structure, environmental context, social signalling, brood parasitism threat, and diet. As many of these variables have themselves changed substantially over the Paleogene, our results suggest that this acceleration can at least partially be explained by long-term climate trends and evolutionary innovations in avian behaviour.}

\normaltalk{\capitalisewords{Webcams as an Untapped Opportunity to Conduct Citizen Science: Six Years of the American Kestrel Partnership's KestrelCam}}{\underline{Sarah E Schulwitz}, D P Spurling, Tala S Davis, Christopher J McClure}{Hundreds of zoo-based or wildlife webcams have become available during the past twenty years, mostly with the goal of educating the public. However, there has been virtually no peer-reviewed research that evaluates the education, conservation, or scientific impact of webcams. Here, we provide one of the few examples of a webcam used for citizen science, and the only test of efficacy for crowd-sourced data collection using webcams. The Peregrine Fund streamed six seasons of American Kestrel (Falco sparverius) nests using the same nest box from 2012 through 2017 and viewers input observations into an online portal. We analyze trends in participant and kestrel behavior and test for sources of bias in this citizen scientist-generated dataset by independently reviewing a subset of recordings to determine accuracy of viewer-logged data. Citizen scientists logged a maximum of approximately 5.25\% of all footage, but with an accuracy of 88\%. Although number of participants declined yearly, on average, participants became more engaged. Sources of bias were related to people's daily activity periods (i.e., less participation at night) and activity within the nest box (i.e., less participation when there were no birds in the box). This citizen scientist-generated dataset generally corroborated the literature regarding American Kestrel biology. Researchers may be cautiously optimistic that datasets generated by citizen scientists can provide valuable information on a given system or study species. Given the ubiquity of webcams and their potential competition for conservation dollars, more research evaluating any aspect of their impact or application is sorely needed.}

\normaltalk{\capitalisewords{Extreme Weather Lengthens Incubation Periods and Lowers Productivity for a Sagebrush-Obligate Songbird}}{\underline{Lindsey E Sanders}, L E Hall, Anna D Chalfoun}{Weather patterns are increasingly being altered by contemporary climate change, with temperate regions regularly experiencing more temperature extremes and storm events.  Understanding of how avian species are affected by climatic changes under different contexts, however, is still in its infancy. Songbirds that breed in exposed, arid systems, such as the Brewer's sparrow (Spizella breweri) continue to exhibit some of the steepest population declines and may be particularly vulnerable to extreme weather conditions.  Environmental factors, for example, that result in lengthened periods of parental care can lead to increased nest losses to predation and physiological stress to parents. We collected fitness and phenological information and concomitant daily temperature and precipitation values for 3,998 Brewer's sparrow nests across four study areas in the intermountain west, USA during 2002−2018.  We were interested in how temperature and precipitation affected 1) breeding season lengths, 2) nesting period (incubation and nestling) lengths and 3) number of young fledged. Brewer's sparrow breeding seasons began earlier with warmer spring temperatures, which tended to lengthen the breeding season.  Nests experiencing extreme temperatures (both hot and cold) resulted in longer incubation periods, but no change in nestling period lengths.  Brewer's sparrows also fledged fewer young when experiencing extreme temperatures and precipitation events.  Understanding the effect of novel climatic conditions on vulnerable species during key life stages will be critical as more extreme weather patterns continue to manifest.}

\normaltalk{\capitalisewords{Intraspecific Variation in Morphological Traits of an Understory Insectivore Across a Precipitation Gradient in the Tropics}}{\underline{Corey E Tarwater}, Patrick Kelley}{Studies across environmental gradients often focus on changes in species richness. Nevertheless, even for species that can be found across a range of conditions, individual traits (e.g., morphological, physiological) may vary and influence ecological and evolutionary processes. Studies along gradients tend to focus on altitudinal gradients, where changes in precipitation and temperature may co-vary. Nevertheless, isolating the effects of precipitation is critical, particularly in tropical regions where precipitation influences both demography and species distributions and is predicted to shift with climate change. Here we examined how individual traits of one tropical forest understory insectivore, Thamnophilus atrinucha (black-crowned antshrike), varied along a precipitation gradient in Panama. We captured 400 antshrikes across the short (65km) and steep precipitation gradient (1700 - 3500 mm annual rainfall) over a two-year period. We found that antshrikes in wetter forests were bigger than antshrikes in drier forests; they had wider and deeper bills, longer tarsi, larger body masses, and were in better condition. Therefore, even within interior tropical forest habitat, precipitation can be a strong selective force, leading to differences in food resources, predators, and other selection pressures. In particular, our results suggest variation in food resources across the gradient may lead to selection on bill size in insectivores. Although further work is needed to address the underlying causes of intraspecific variation, our work indicates that precipitation could affect not only ecological processes as currently recognized, but also evolutionary processes.}

\normaltalk{\capitalisewords{Contrasting Demographic Responses of Eastern Bluebird Populations to Climatic Variability}}{\underline{Sara E Harrod}, Virginie Rolland}{As climate change alters temperature and precipitation patterns, numerous avian species have declined. However, populations of some species that show responses to climate change, such as Eastern Bluebirds (Sialia sialis), have increased or remained stable nationwide. Therefore, using nesting and banding data collected 2003–2018 from 4 populations in 2 regions of contrasting climate, we aimed to (1) estimate bluebird demographic parameters and (2) quantify the seasonal effects of 15 local climatic variables and 2 global climatic indices on these parameters. We adopted an information-theoretic approach on generalized linear mixed models of breeding parameters and Cormack-Jolly-Seber models of survival. At our easternmost populations, estimates (± SE) of hatching success, fledging success, and adult survival varied from 0.26 (± 0.10) to 0.97 (± 0.03), 0.39 (± 0.11) to 0.96 (± 0.04), and 0.16 (± 0.05) to 0.52 (± 0.15), respectively. At the westernmost populations, estimates ranged from 0.78 (± 0.05) to 0.96 (± 0.02), 0.73 (± 0.10) to 0.98 (± 0.01), and 0.12 (± 0.07) to 0.55 (± 0.12), respectively. Excluding clutch size, all breeding parameters (i.e., hatching success, hatchability, and fledging success) and adult survival were affected by climatic variables during breeding and non-breeding seasons. However, climatic variables of importance differed among populations. For example, hatching success was affected by variability in snowfall in the easternmost populations, but not in the westernmost populations. We conclude that bluebird populations respond to climate variability in complex ways, both at site and regional levels. Additional research is needed to predict long-term population projections.}

\normaltalk{\capitalisewords{How Do Seabirds See Light? Spectral Effects on the Temporal Sensitivity of Hawaiian Seabirds}}{\underline{Hannah E Moon}, Tracy Anderson, Marc Travers, Ellis Loew, Megan Porter}{Anthropogenic lights at night cause high mortality in fledgling seabirds due to attraction and subsequent grounding. The seabirds most affected by grounding are Procellariformes. Of concern on Kaua'i are the Hawaiian Petrel (Pterodroma sandwichensis), Newell's shearwater (Puffinus newelli), and the Wedge-tailed shearwater (Ardenna pacifica). Newell's Shearwater fledglings have the highest susceptibility, which suggests differences in behavior and/or vision between species as well as between fledglings and adults. Previous studies on migratory birds suggest that the color of light can affect attraction. To better understand visual perception in seabirds and drivers of fatal light attraction, the temporal sensitivity of the target species was tested in different spectra and intensities of light. Flashing LED lights were used in up to five intensities in three wavelengths- violet (385nm), blue (450nm), and white light (peak at 594nm). Electroretinography was used to measure the magnitude of response in the eye to each intensity and identify the flicker fusion frequency. Juveniles of all three species and one adult A. pacifica have been tested by the end of the first field season. All species were sensitive to 385nm light at bright intensities, suggesting a violet sensitive visual system. Preliminary analyses suggest a flicker fusion frequency of about 30Hz in the violet-sensitive photoreceptor of P. newelli. Preliminary results also suggest variation in response of the eye to different colors of light between species and life stages. Ultimately, a greater understanding of the physiology of seabird visual systems may help reduce anthropogenic light attraction and groundings in the future.}

\normaltalk{\capitalisewords{Mapping Wetland Habitat using Remote Sensing Data for the Critically Endangered White-Bellied Cinclodes (Cinclodes Palliatus)}}{\underline{Richard E Gibbons}, Marco Otto}{The White-bellied Cinclodes (Cinclodes palliatus) is a critically endangered resident of the Peruvian high Andes occurring above 4400 m in patchily distributed wetlands sustained by wet season precipitation and dry season snowmelt. The population is estimated to be less than 300 individuals with much of the range being in remote areas with challenging access. Continuing threats to the White-bellied Cinclodes include habitat loss and degradation. We identified a signal range of Sentinel 2 remotely sensed data (10 m) using several ground-based habitat surveys to identify high Andean wetlands within the geographic range. Using this technique, we learned that these wetlands comprise approximately 4\% of the land cover. Recent sight records for this species occurred in the medium to large wetland areas. Several priority areas were identified for surveys.}

\normaltalk{\capitalisewords{Response of a Southern California Riparian Bird Community to Habitat Destruction by the Invasive Kuroshio Shot Hole Borer}}{\underline{Barbara E Kus}}{A serious emerging threat to southern California riparian ecosystems is the exotic Kuroshio shot hole borer (KSHB), an invasive beetle that cultivates a pathogenic fungus that kills trees in as little as 6 months. We examined the response of the bird community at the Tijuana River to a massive KSHB infestation in 2016 by comparing data from a MAPS (Monitoring Avian Productivity and Survival) station operated during 7 pre-infestation (2009-15) and 2 post-infestation (2017-18) years. Species richness did not change, but average annual captures declined by 35\%. Among the species making up ≥ 5\% of the total individuals caught in any one year (n=13), abundance declined by up to 76\% in 7 species, including those most abundant at the station (Bushtit, Song Sparrow, Common Yellowthroat, Orange-crowned Warbler and Wilson's Warbler). Mean annual abundance was unchanged among 4 species, including the endangered Least Bell's Vireo, and doubled for House Finch and Warbling Vireo post-KSHB. Although historically not among the most abundant species at the site, cavity nesters including Downy Woodpecker, Nuttall's Woodpecker and Ash-throated Flycatcher increased post-KSHB, possibly reflecting the increased availability of dead trees. Post-KSHB recovery of the bird community will depend on currently unknown interactions between the effects of KSHB re-invasion and weather on vegetation recovery, and further research is required to guide development of effective management of this threat to regional biodiversity.}

\normaltalk{\capitalisewords{Assessing Halogenated Marine Persistent Organic Pollutants in the Critically Endangered California Condor and their Marine Mammal Prey}}{\underline{Margaret E Stack}, Eunha Hoh, Nathan Dodder, Christopher Tubbs, Ignacio Vilchis, Rachel Felton, Jade Johnson}{After nearing extinction in the 1980s, California condors have rebounded in the past decades due to breeding and conservation efforts. These reintroduction efforts have established two distinct, wild populations: inland and coastal groups. Coastal habitats are advantageous for the condors because it allows for their independence from anthropogenic food sources. Inland populations rely on a human-provided diet, such as livestock and hunted carrion, but coastal groups have access to marine mammal carcasses. However, marine mammals are highly susceptible to the bioaccumulation of persistent organic pollutants (POPs) that can biomagnify in the condors. Recent studies have indicated the occurrence of eggshell thinning in the condors as a potential result of POP consumption. If such thinning continues, coastal populations will be inhibited in becoming self-sustaining. Our study aims to identify the contaminant profiles of inland and coastal condors, along with the profiles of the marine mammal prey, through the use of novel non-targeted chemical analysis using two-dimensional gas chromatography coupled to time-of-flight mass spectrometry (GCxGC/TOF-MS). The results provide dietary exposure assessments of the two populations and has the potential to identify previously unknown contaminants that may be affecting coastal wildlife more broadly than only the condors. Preliminary results of the project examining 3 coastal and 3 inland condors showed that coastal condors contained significantly higher ranges of halogenated contaminants than their inland counterparts (52-83 contaminants/sample vs. 9-10 contaminants/sample). An additional 17 samples are currently being analyzed that will provide more data to inform conservation and management decisions to aid in condor conservation.}

\normaltalk{\capitalisewords{Population Genomics of Greater White-Fronted Goose}}{\underline{Robert E Wilson}, Sarah A Sonsthagen, Craig R Ely, Sandra L Talbot, Jeffrey M DaCosta, Michael D Sorenson, Kim T Scribner, Melanie Weaver, Dan Skalos}{Dispersal and migratory behavior can greatly influence how genetic diversity is apportioned geographically. In migratory species, genetic structure can be promoted via several mechanisms including fidelity to distinct migratory routes. Particularly within North America, waterfowl management units have been delineated according to distinct longitudinal migratory flyways supported by banding data. The greater white-fronted goose (Anser albifrons) is a migratory waterfowl species with a largely circumpolar distribution consisting of up to six subspecies roughly corresponding to phenotypic variation. We examined the range-wide population genetic structure of greater white-fronted geese using mtDNA control region sequence data, microsatellite loci, and double-digest restriction associated DNA sequences (ddRADseq) from 23 locales across North America and Eurasia. We found significant differentiation in mtDNA between sampling locales, with flyway delineation explaining a significant portion of the observed genetic variation. This is concordant with band recovery data showing little inter-flyway or intercontinental movements. Although the microsatellite data showed low resolution, we uncovered structure across the nuclear genome that corresponded to regional groupings with 3,888 ddRAD loci.  Our results support the distinctness of the two isolated breeding populations within Cook Inlet, Alaska and Greenland, consistent with their subspecies status as well as the subspecific classification of a North American and Asian subspecies.  Overall, the degree of observed genetic structuring aligns well with current understanding of population differences in seasonal movements, social structure, pairing behavior, and resource partitioning.}

\normaltalk{\capitalisewords{Reconciling the Mitonuclear Compatibility Species Concept with Rampant Mitochondrial Introgression}}{\underline{Geoffrey E Hill}}{The mitonuclear compatibility species concept defines a species as a population that is genetically isolated from other populations by uniquely coadapted mitochondrial and nuclear genes.  A key prediction of this hypothesis is that the mitochondrial genotype of each species will be functionally distinct and that introgression of mitochondrial genomes will be prevented by mitonuclear incompatibilities that arise when heterospecific mitochondrial and nuclear genes attempt to cofunction to enable aerobic respiration.  It has been proposed, therefore, that the observation of rampant introgression of mitochondrial genotypes from one species to another constitutes a strong refutation of the universality of mitonuclear speciation.  The displacement of a mitochondrial genotype from a nuclear background with which it co-evolved to a foreign nuclear background will necessarily lead to fitness loss due to mitonuclear incompatibilities.  Here I consider two potential benefits of mitochondrial introgression between species that may, in some cases, overcome fitness losses arising from mitonuclear incompatibilities.  First, the introgressed mitochondrial genotype may be better adapted to the local environment than the native mitochondrial genotype such that higher fitness is achieved through improved adaptation via introgression.  Second, if the mitochondria of the recipient taxa carry a high mutational load, then introgression of a foreign, less corrupt mitochondrial genome may enable the recipient taxa to escape its mutational load and gain a fitness advantage.  Under both scenarios, fitness gains from novel mitochondrial genotypes could theoretically compensate for the fitness that is lost via mitonuclear incompatibility.  Rampant mitochondrial introgression will typically lead to erasure of species but in some cases could lead to hybrid speciation.}

\normaltalk{\capitalisewords{Identifying Critical Nesting Habitat for Ducks in Alberta's Western Boreal Forest}}{\underline{Matthew E Dyson}, Stuart M Slattery, Bradley C Fedy}{The western boreal forest (WBF) is an important breeding area for North American ducks, second only to the Canadian Prairies. The WBF is under intensive industrial development, causing habitat loss and fragmentation. Land use change can have profound effects on predator-prey interactions and influence population dynamics. In most avian species, nest success is critical to population persistence. Therefore, species are under intense selective pressure in choosing a safe nest site. Currently, we have limited knowledge of duck nesting ecology in the WBF, including the influence of changing land use practices on how ducks select nest sites. We investigated nest-site selection of ground nesting ducks in the WBF of Alberta at multiple spatial scales using logistic regression-based resource selection functions. We located 167 duck nests of 8 different species between 2016 and 2018 by nest searching across a gradient of industrial development. We hypothesized nest-site selection strategies associated with habitat and land use variables would vary by species and with spatial scale. We also hypothesized that industrial development increased predation and expected ducks to avoid highly disturbed habitats. In addition to quantifying the nesting-ecology relationships for multiple species and scales, we spatially predicted our best models to identify nesting habitat for boreal ducks. These maps can help prioritize habitat conservation and represent the first step towards understanding the nesting ecology of ducks in this important region.}

\normaltalk{\capitalisewords{Patterns and Causes of Tropical Montane Life Histories: An Observational and Experimental Study in Malaysian Borneo}}{\underline{Adam E Mitchell}, Thomas E Martin}{Species living at higher elevations often have slower life histories than relatives at lower elevations. This pattern mimics the wide-spread and well-studied pattern across latitudes, but hypotheses for latitudinal life history variation do not explain the variation observed across elevations. We provide data verifying this pattern in two communities of birds at two field sites (1500m asl and 3200m asl) in Malaysian Borneo. High elevation relatives lay fewer eggs per clutch, and have longer incubation and nestling periods on average. We tested a novel hypothesis that may explain this pattern. The harsh weather hypothesis states that slow life history traits at high elevations are caused by constrained parental behavior due to cool ambient temperatures exacerbated by rainfall. These harsh montane conditions require parents to spend more time warming nestlings, resulting in less time available for food provisioning. Reduced provisioning rates may proximately cause slower post-natal development and ultimately select for reduced clutch sizes. We tested this hypothesis by adding supplemental heat and a rain cover to nests of Mountain Blackeyes (Chlorocharis emiliae) at 3200 m asl. Our results show that nestlings in heated and covered nests grew faster and left the nest sooner than unmanipulated nests. Furthermore, parents from heated nests spent less time keeping the young warm and brought food back to the young at a higher rate. Thus, our results explain one mechanism by which life history traits of high elevation birds fall closer to the slow end of the life history continuum than those at lower elevations.}

\normaltalk{\capitalisewords{Quantifying Snag Resources for Cavity-Nesting Birds and Red-Cockaded Woodpecker Cavity Usurpation in Old-Growth Longleaf Pine}}{\underline{Heather E Levy}, Robert J Cooper, James A Cox}{Cavity-nesting bird communities (CNBs) interact through a complex web based on the creation of and competition for cavities as nest and roost resources. CNB communities have been studied in restored longleaf pine forests, but few studies have occurred in old-growth stands, which occupy only 0.014\% of their former range. The Red Hills region of the southeast contains some of the last remaining old-growth longleaf pine forests and supports one of the largest remaining populations of the endangered Red-cockaded Woodpecker (RCW). RCWs excavate exclusively in living pines, and their cavities are often usurped by other CNBs, but mechanisms that drive this interaction remain understudied in old-growth systems. We quantified snag abundances, CNB abundances, and nest productivity across 8 old-growth sites in the Red Hills. Sites were expected to contain varying snag abundances due to differences in management. Our goals were to (1) examine the relationship between CNB abundance and snag abundance and (2) to compare the proportion of RCW cavities usurped in relation to snag abundances to determine whether snag limitation affects frequency of interspecific competition. We found that CNB abundance is positively correlated with snag abundance, suggesting that snag retention/creation promotes high abundances of cavity-nesting birds. We also found that RCW cavities were usurped despite local snag abundances, suggesting that CNBs may prefer RCW cavities despite abundant snag resources due to their sturdiness and persistence on the landscape, or that there is a surplus of natural RCW cavities in old-growth longleaf pine forests that results in reduced interspecific competition.}

\normaltalk{\capitalisewords{Mexican Hummingbird Nesting}}{\underline{Laura E Nu\~{n}ez-Rosas}, Mar\'{i}a dC Arizmendi}{Nesting is a sensitive period that presents a high mortality rate because of many ecological factors. Hummingbirds are altricial birds which need parental care. Female carries all the nesting process. Although, Trochilidae family is a diverse group in terms of species number (330), the knowledge about their reproductive aspects is scarce. For this study, we made a review based on published data and personal observations. We collected data from nests and eggs measures, incubation and fledging periods, lichen use, site and vegetation where nest was located. For statistical analysis, we used PGLS, which incorporates a phylogenetic correction. We also used simple linear regressions. We found that nest shape is a cup, built at 41.6 ± 12.39 mm of high, 40.21 ± 13.94 mm of diameter and 18.59 ± 5.06 mm of deep. Eggs were 13.19 ± 1.04 mm of length and 8.65 ± 0.67 mm of width. We couldn´t find evolutionary relationship in lichen use, however, it shows a tendency of recently clades using more lichens. Additionally, both incubation period (16.17 ± 1.16 day) and fledging period (22.4 ± 2.25 day) were not related with female size or clade. Finally, most of the nest has been found in pine forest (50\%). Available information for reproductive aspects in hummingbird is limited, and much work is needed to fully understand the process. Although some data for exist for some species, most of these are not distributed in Mexico.}

\normaltalk{\capitalisewords{Dynamics of Remigial Molt of Captive Spectacled (Somateria Fischeri) and Steller's Eiders (Polysticta Stelleri)}}{\underline{Sadie E Ulman}, Tuula Hollmen, Dan Esler}{Remigial molt in waterfowl is an energetically costly and critical time in the annual life cycle. In waterfowl in general, and sea ducks in particular, information about the ecology of this time period is sparse. The Alaska SeaLife Center captive spectacled and Steller's eiders were studied to provide data on the dynamics of this physiologically demanding period. Data on captive birds will provide beneficial information for interpretation of data collected on these federally threatened species of eiders in the wild.  During the 2017 and 2018 remigial molt, we looked at the pre-emergence interval, timing of molt initiation, primary feather growth rate (mm/day), and duration of flightlessness of captive spectacled and Steller's eiders. Body mass and 9th primary feather measurements were taken on spectacled (n=14, in 2018) and Steller's eiders (n=41, in both 2017 and 2018) at 1-2 week intervals from end of July through October. Of those birds, we used our newly developed feather color marking method to uniquely color mark the 9th primary feather of 9 spectacled and 9 Steller's eiders. Data collected using this marking method can be used to determine molt initiation and lag time between drop data and start of growth, which will improve estimates of growth rate and molt duration.}

\normaltalk{\capitalisewords{Migration of Buff-Breasted Sandpipers at Two Key Regions in the Mid-Continent: Northern Gulf of Mexico and the Flint Hills}}{\underline{James E Lyons}, Brad A Andres, Robert L Penner, Kelli Stone, L L Wolfenbarger}{Buff-breasted Sandpipers migrate through the mid-continent of North America as they move from wintering areas in southern South America to breeding areas in the Arctic. Loss of grasslands in the Central and Mississippi flyways, as well as on the wintering grounds in South America, may limit population growth of this once numerous species. Additional information is needed on stopover abundance in the mid-continent to plan conservation and management. The first stopover region after crossing the Gulf of Mexico is the Western Gulf Coastal Plain ecoregion of Texas and Louisiana. Continuing north, many birds also stopover in the Flint Hills ecoregion of Kansas and Oklahoma. With nearly 2.8 million ha, the Flint Hills includes approximately three quarters of all remaining tall-grass prairie in North America. We conducted surveys using line transects and distance sampling in Kansas and Oklahoma in 2014 and 2015. We also conducted distance sampling at roadside point transects in Texas and Louisiana from 2016 to 2018. Our results suggest that approximately 14,300 birds stop in the Flint Hills, with birds concentrated in the southern parts of the region where there is more grassland and less development. Our results from Texas and Louisiana further document the extensive use of commercial turf (grass) agriculture, areas that may have once been coastal prairie or riparian forest. These surveys provide important information about abundance and distribution that will be useful for conservation planning at two key stopover regions in the mid-continent, the Western Gulf Coastal Plain and the Flint Hills.}

\normaltalk{\capitalisewords{Movement and Distribution of Red Knots (Calidris Canutus) in the Southeastern U.S}}{\underline{Molly E Tuma}, Abby N Powell}{The Southeastern U.S. is an important nonbreeding area for Red Knots, supporting migratory and overwintering populations. Studies indicate possible regional declines and shifts in distribution in the knot population in the Southeast; however, most studies have focused on subregions of the Southeast, leaving questions about use and connectivity across the region as a whole. Using over a decade of data from ebird and band resightings from bandedbirds.org, we examined: (1) movement strategies of knots in coastal areas from Texas to North Carolina, and (2) the distribution of knots across that range and what drives that distribution. First, we used band resightings to characterize movement between sites and we quantified connectivity between these locations using an observed movement network. We then estimated knot distribution using ebird data and species distribution models. Preliminary analyses conducted for Florida show some knots using both Atlantic and Gulf coast locations in a year, typically moving from the Atlantic to the Gulf coast in fall and from the Gulf to the Atlantic in spring. Distribution mirrored historical counts in Florida and emphasized the importance of the Tampa Bay area for nonbreeding knots. Results from Florida suggest a possible hop or skip migration strategy by knots in the Southeast. We expect to see further evidence of this when we expand our analyses to the whole region. A hop or skip strategy has not been documented in Red Knots, and could indicate the importance of lesser stopover locations along the Atlantic and Gulf coasts.}

\normaltalk{\capitalisewords{Population-Level Importance and Migratory Connectivity of a Shorebird Staging Site in the Midcontinental Flyway}}{\underline{Jessica E Howell}, Ann E McKellar, Richard HM Espie, Kristin Bianchini, Christy A Morrissey}{Arctic-breeding shorebirds undergo some of the longest migrations in the animal kingdom, relying on staging sites to rest and replenish fat stores between migratory flights. Although staging sites are a significant determinant of migratory success, migration chronology and population data from many midcontinental sites are lacking, while human development encroaches into shorebird habitat across their range. Chaplin and Reed Lakes, a saline wetland complex in southcentral Saskatchewan, Canada, is a site of hemispheric importance but had not been surveyed since the 1990's. Using data from point count surveys (2014-2017) and nano-tagged individuals tracked with the Motus automated radio telemetry array (2015-2018), we examined migration chronology of arctic-breeding shorebirds as well as population size and migratory movements of Sanderlings (Calidris alba). We found that these interior lakes support 32 shorebird species including 17 arctic breeding species with ~25\% of North America's Sanderling population. Of tagged Sanderlings with detections during both northward and southward migration, 63\% of individuals detected in the Midcontinental Flyway in spring were detected in the Atlantic Flyway in the fall. Our results highlight the significance of Chaplin and Reed Lakes as a critical staging site and its importance for migratory connectivity to prioritize shorebird conservation across North America.}

\normaltalk{\capitalisewords{Does Range Expansion or Ongoing Dispersal Explain Apparent Panmixia in Bachman's Sparrow? a Temporal Study using Museum DNA}}{\underline{Amie E Settlecowski}, Brant C Faircloth, Jeremy M Brown, James A Cox, James W Tucker, John P Carpenter, Sabrina S Taylor}{Bachman's Sparrow (Peucaea aestivalis) is uncommon throughout its wide distribution across the southeastern US, primarily persisting in declining populations in remnant fragments of longleaf pine forest. Despite these declines and the extreme loss and fragmentation of its habitat, a previous study indicated Bachman's Sparrow is panmictic across its distribution. Eighteen microsatellite and two mitchondrial loci showed limited genetic differentiation even among distantly separated populations, such as their southern limit in Florida and west of the Mississippi River in Louisiana. However, the extensive contemporary gene flow necessary for connectivity among such distant populations is improbable in Bachman's Sparrow, given the extent of loss and fragmentation of their habitat. Instead, apparent panmixia may be a signature of past gene flow resulting from a recent range expansion and subsequent retraction. To evaluate this hypothesis, we are reassessing Bachman's Sparrow population structure with time-stratified genetic data from five regions and four time periods: (i) before and (ii) during range expansion, (iii) during retraction, and (iv) modern day. If the panmictic signal is the result of gene flow associated with the range expansion, the five sampled regions should show differences in population structure and connectivity before and after the range expansion. Alternatively, if dispersal ability maintains connectivity among Bachman's Sparrow populations, populations should maintain a consistent pattern of structure and connectivity through time. We collected 3,000 single nucleotide polymorphisms from 318 modern blood samples and 282 museum toepads to distinguish between these scenarios.}

\normaltalk{\capitalisewords{Effects of Source Population and Release Strategy on Reintroduced Scaled Quail Mortality and Dispersal}}{\underline{Rebekah E Ruzicka}, Paul F Doherty, Jr, Dale Rollins}{Declines of scaled quail (Callipepla squamata) as a result of habitat fragmentation have been well documented over the last 25 years and translocation has been increasingly implemented as a means to reestablish populations. Yet translocation success remains variable and factors contributing to success are often poorly understood. We sought to test hypotheses concerning the influence of source population and variation in delayed release strategy (1-9 weeks) on mortality and dispersal of wild-caught, translocated quail. We trapped and translocated quail over two years (2016-2017) from source populations in the Edwards Plateau and Rolling Plains ecoregions to a large contiguous ($>$40,000 ha) release site in Knox County, Texas. We evaluated hen survival and dispersal as a function of holding time prior release, source population, age, release location, and year using a multi-state model that incorporated state uncertainty. We found no effect of source population on mortality and dispersal. Quail with longer holding times prior to release had higher mortality, but lower dispersal rates. Future reintroduction efforts should be cognizant of source population and the effect of release strategy on mortality and dispersal.}

\normaltalk{\capitalisewords{Multiple Hybrid Zones in a Widespread Amazonian Bird Reveal Different Evolutionary Processes}}{\underline{Andre E Moncrieff}, Robb T Brumfield}{Hybrid zones are composed of distinct populations that are still connected by gene flow, and these zones thus offer a unique opportunity to investigate intermediate stages of the speciation process. An avian hybrid zone between green and black plumage morphs of adult male Blue-crowned Manakins (Lepidothrix coronata) extends for 1500 km from the base of the Andes northeast to the Madeira River. Discordance between the geographic centers of the genetic and plumage color clines suggest a role for sexual selection on plumage color. In addition to the plumage color hybrid zone, we hypothesize that hybrid zones between genetically divergent populations occur where river widths narrow towards the headwaters of several major river systems. To further clarify the evolutionary processes shaping gene flow and hybrid zone location, we obtained genome wide sequence data from individuals across the Amazon Basin. Using this dataset, we estimated geographic patterns of population structure and constructed a comprehensive phylogenetic hypothesis for Blue-crowned Manakin populations. A lack of geographic structuring of the genetic variation from samples spanning the plumage color hybrid zone suggests the plumage divergence is restricted to just a few underlying plumage color loci. Headwater hybrid zones, in contrast, revealed variation evenly distributed across the genome consistent with neutral processes.}

\normaltalk{\capitalisewords{Growth of Two Atlantic Coast Piping Plover Populations}}{\underline{Chelsea E Weithman}, Samantha G Robinson, Kelsi L Hunt, Jon Altman, Henrietta A Bellman, Audrey L DeRose-Wilson, Katie M Walker, James D Fraser, Sarah M Karpanty, Daniel H Catlin}{Species listed under the U.S. Endangered Species Act are required to meet stated recovery goals for delisting. These goals often are developed early in the species' conservation history and may need to be refined as new information becomes available. Initial recovery goals for the Atlantic Coast Piping Plover (Charadrius melodus) included maintaining a reproductive output of 1.5 chicks/pair for 5 years. This goal was based on modelling that assumed equal survival rates throughout the range, but recent studies suggest that demographic rates may vary with latitude. To investigate latitudinal variation, we developed demographic and population growth estimates for two breeding populations of Piping Plovers on Fire Island, New York (Latitude ~40.7°N) and on the Outer Banks, North Carolina (Latitude ~35.3°N) 2013–2017. Reproductive output varied annually but was lower in NC than NY. Average adult true survival in NY (0.73 ± 0.04) was similar to rates reported previously for this species, but average survival in NC was lower (0.69 ± 0.07). Annual post-fledging survival for both sites was variable and often higher than had been previously reported for Atlantic Coast Piping Plovers (0.43–0.66 for NY; 0.31 and 1.0 for NC). While the estimated reproductive output needed for a stationary population for both sites was similar (1.10 ± 0.73 chicks/pair for NY; 1.07 ± 0.69 for NC), only the NY population achieved or exceeded these values during our study. Our findings suggest that understanding the variability of population demographic rates would be helpful in refining recovery goals.}

\normaltalk{\capitalisewords{Can Male Zebra Finches Perceive their Own Quality and Adjust Mate Choice Accordingly?}}{John O Martin, Emilie Chien, \underline{Katherine E Henson}, Nancy T Burley}{While there is growing evidence for male mate choice in birds, little is known about male selective behavior in monogamous species like zebra finches (Taeniopygia guttata). Most studies have focused on female choice, and the work that has been conducted on male choice has used methods that limit free interaction among trial birds. Here, we aimed to mimic a more natural setting for courtship to take place. One important factor that may influence male choosiness is their ability to assess their own quality. If it is possible, they should adjust their mate preferences to maximize their mating success. Therefore, we hypothesize that low quality males will prefer low quality females in order to reduce the probability of not obtaining a mate. Male quality was manipulated by clipping flight feathers, and female quality by altering the early life conditions of brood size and diet. Preliminary results show that low quality males significantly prefer high quality females (p=0.012), which suggests that some other variables are involved in quality-dependent mate choice.}

\normaltalk{\capitalisewords{Social Versus Genetic Mating Systems: The Complex Mating System of the Lark Bunting}}{\underline{Bruce E Lyon}, Alexis S Chaine}{Social and genetic mating systems are often uncoupled in a species. For example, many socially monogamous birds have high levels of genetic polygamy due to extra-pair paternity (EPP). However, the cryptic nature of extra-pair copulations in most species makes it difficult to determine their social and evolutionary causes and consequences, and they are therefore often relegated to a minor role in the social mating system. In the socially monogamous lark bunting we found that behaviors associated with acquisition of extra-pair copulations are a major driver of the social system during the breeding season. Male buntings show classic territoriality until they attract a female, at which point they completely cease defending their territory and switch to mate guarding. Prior to clutch completion, most females are closely attended by readily observable 'mobs' of up to five additional males seeking extra-pair copulations, with little overt physical aggression shown by the social mate. Behavioral observations revealed complex social interactions between males in mobs, the existence of subtle tactics for enhancing success at mobbing, and suggest that EPP in lark buntings are often forced by males rather than driven by female choice. Genetic analysis confirms high rates of EPP in our population and selection analyses on this component of fitness indicate that success in EPP is non-random with respect to male plumage and size traits. Mobbing is a key component of the mating system of the lark bunting and plays a significant role in the evolution of their social behavior and plumage signaling.}

\normaltalk{\capitalisewords{What Drives Flexible Signaling? Anthropogenic Noise and Social Context Affect Male House Wren Vocal Behavior}}{\underline{Erin E Grabarczyk}, Maarten J Vonhof, Sharon A Gill}{Vocal communication shapes avian social networks, connecting multiple individuals over space and time via information, facilitating mate attraction and resource defense. Despite evidence that both the physical and social environment affect signaling behavior, few studies consider variation in individual responses to rapid environmental change within a social context. For example, in response to anthropogenic noise, males adjust their signal structure. But change in signaling behavior may not be driven by the noise disturbance alone, and social factors may also play an important role in song adjustments. We test the hypothesis that male House wrens (Troglodytes aedon) flexibly adjust their songs in response to their immediate noise environment, but that adjustments may depend on social context. We recorded paired males prior to clutch initiation, quantified ambient noise in the moments before signaling, and define social context within pairs as female fertile status and between males as number of conspecific neighbors. Both social factors and noise affected temporal structure of songs, whereas frequency adjustments were driven by the social environment. Males sang shorter songs if they had neighbors and flexibly shortened song components used for short-distance communication with increasing noise. Spectral adjustments were predicted by social context; males sang songs at lower minimum frequencies when their mates were fertile. Our study is an essential step towards understanding drivers of flexibility in male signaling behavior, and expands our understanding on how both the social and physical environment may drive selection on male vocal signals.}

\normaltalk{\capitalisewords{Repertoire Size and Individual Variation in Vocalizations of '\={o}ma'o (Myadestes Obscurus) an Endemic Hawaiian Thrush}}{\underline{Eirlys E Tysall}, Kristina L Paxton, Erin Netoskie, Eben Paxton, Patrick J Hart}{Acoustic signalling is the predominant form of communication in birds, and there can be considerable variability in the vocal complexity and repertoire size among species. Population-level vocal complexity is well studied, however methodological challenges have limited the number of studies focused on the individual-level, particularly for tropical species. Here we examined individual repertoire size and variability in a member of the thrush family; the 'ōma'o (Myadestes obscurus). We used a novel transmitter with a miniature microphone attachment to locate and record individuals. '\={O}ma'o vocalizations were spectrographically imaged and a catalogue of all unique syllables and songs was created for each individual. Individual repertoire size ranged from 99 to 200 syllables and 84 to 303 songs, though neither syllable or song accumulation curves reached an asymptote even with over 13 hours of recording. Forty-two to 55\% of syllables were recorder for each bird only once, and less than 3\% (±2\%) of syllables were shared between individuals. We also found extremely high song versatility index values which indicate a high level of song complexity. Overall, we found individual 'ōma'o have a highly variable song and possess a large repertoire of unique syllables. The unusually large repertoire and low level of repertoire sharing between individuals suggests syllable invention may be present in this species. This study represents the first comprehensive analysis of 'ōma'o vocal repertoire, and at a broader level provides insight into whether a snapshot of song recorded in bird species with complex songs is a true representation of an individual's song.}

\normaltalk{\capitalisewords{Effects of Spectral Composition of Nighttime Lighting on Host Competence to West Nile Virus in a Passerine Reservoir}}{\underline{Meredith E Kernbach}, Vincent M Cassone, Thomas R Unnasch, Lynn B Martin}{Artificial light at night (ALAN) has become one of the most pervasive but least understood anthropogenic stressors on the planet. Exposure to such light pollution invokes a myriad of negative impacts on the health and fitness of wildlife including metabolic dysregulation, exaggerated immunopathology, and mistimed behavior. In a recent study, exposure to ALAN enhanced the infectiousness of West Nile virus (WNV) infected house sparrows without affecting mortality rates, which increased population outbreak potential by 41\%. The recent push to switch to energy efficient LED lighting poses new threats regarding short wavelength-rich spectral composition of nighttime lighting. Non-visual opsin receptors (i.e. melanopsin) responsible for synchronizing circadian rhythms via melatonin regulation are maximally sensitive to short wavelength light, which may be why cool white LED lighting exacerbates health consequences of ALAN. Here, we explored the effects of spectral composition of nighttime lighting (5000K, 3000K, 1800K) on WNV infection outcomes in house sparrows, and whether the observed effects are mediated by melatonin suppression. Exposure to low intensity 5000K ALAN increased WNV-induced mortality, whereas exposure to low intensity 1800K ALAN improved the maintenance body condition throughout the course of infection. Exposure to 5 lux of 3000K ALAN suppressed melatonin secretion by an order of magnitude; however, the implications for melatonin suppression during WNV infection are still unclear. These results emphasize the need to consider how nighttime lighting intensity and spectral composition can affect infectious disease dynamics across levels of biological organization.}

\normaltalk{\capitalisewords{Trade-Offs Between Ecosystem Services and Disservices Provided by Birds Covary with Agricultural Intensification in the US}}{\underline{Christopher E Latimer}, Olivia M Smith, Jeb Owen, William Snyder, Christina M Kennedy}{Given that agriculture is the dominant human land use worldwide, maintaining and enhancing biodiversity in agroecosystems is vital for conservation. Birds are an important component of biodiversity in agroecosystems and provide a range of ecosystem services that farmers value, including pollination, seed dispersal, and pest control. At the same time, birds can impose disservices in the form of crop damage and pathogen transmission. These ecosystem service trade-offs, while recognized in a general sense by the scientific community, have not been quantitatively evaluated across broad spatial extents that are needed to inform farm management or policy recommendations. Agricultural intensification – one component of farm management – can generate potential ecosystem service trade-offs by substantially and differentially altering the ability of landscapes to support both service- and disservice-providing species. Here, we conducted a literature review of North American birds and identified species as either service or disservice providers based on the empirical evidence. We then evaluated the traits most strongly associated with species in each group. Combining spatially-explicit land cover and agricultural census data with citizen-science breeding bird survey data, we modeled the extent to which agricultural intensification filters species' traits (functional diversity) and alters the spatial distribution of service and disservice providers across regions within the U.S. Lastly, we evaluated the degree to which land cover composition and configuration mediate the effects of agricultural intensification on functional diversity. We will discuss the socio-ecological implications and management recommendations of this work, as well as future research needs.}

\normaltalk{\capitalisewords{Trade-Offs Between Carbon Storage and Biodiversity Conservation in Reforested Riparian Zones}}{\underline{Kristen E Dybala}, Kristin Steger, Robert G Walsh, David R Smart, Thomas Gardali, Nathaniel E Seavy}{Restoration of riparian forests is expected to provide multiple benefits simultaneously, including important habitat for wildlife and increased carbon storage. But achieving multiple goals simultaneously requires understanding any potential trade-offs among them, so that, for example, carbon storage isn't maximized at the expense of biodiversity conservation. In remnant and restored riparian forest stands in California's Central Valley, we examined the distribution of bird abundance, bird diversity, and carbon stocks to identify their relationships to each other and to forest stand features. Overall, reforested areas successfully provided multiple benefits: significantly greater carbon storage and avian biodiversity compared to baseline conditions. However, we also found substantial variation in these benefits with forest stand features within study areas, indicating an opportunity to further increase the magnitude of these benefits. Soil carbon storage and bird metrics increased with understory shrub cover, but we also identified a trade-off between biomass carbon storage and bird metrics in their relationship to forest stand density. Biomass carbon stocks were strongly positively related to stand density, while bird density and diversity suffered at the highest stand densities. The ability of conservation projects to successfully achieve multiple goals requires: (1) clearly defining those goals and objectives, (2) interdisciplinary research that can identify synergies and trade-offs among those goals, and (3) an openness to testing new restoration designs aimed at optimizing across those goals.}

\normaltalk{\capitalisewords{Migratory Landbird Conservation Compromised by Failure to Address Sexual Segregation}}{\underline{Ruth E Bennett}, Amanda D Rodewald, Kenneth V Rosenberg}{Many conservation efforts fail to consider the possibility that male and female animals spatially segregate among habitats or landscapes. Overlooking this phenomenon may result in conservation plans that do not meet the needs of both sexes. When resources or threats vary spatially, sexual segregation can profoundly affect sex-specific survival rates to the point of influencing population persistence. We assessed the degree to which current conservation efforts account for sexual segregation with a review of primary and conservation planning literature for 66 North American migratory landbirds of conservation concern. Sexual segregation was reported for two-thirds of species with published nonbreeding sex ratios, and the pattern of segregation exposed females to higher rates of land use change than male for multiple Neotropical migrants. Despite this, only 7\% of conservation plans accounted for sexual segregation, and only one plan explicitly called for protection of female-dominated habitats. Next, we used the declining Golden-winged Warbler (Vermivora chrysoptera) as a case study to test for sex-bias in habitat prioritized by current conservation efforts. By modeling nonbreeding occupancy and assessing forest cover loss for males and females, we show that females lost twice as much nonbreeding habitat as males from 2000-2017, yet existing conservation focal areas remain heavily biased towards male-dominated landscapes. Our work reveals a systemic failure to address sexual segregation that compromises the effectiveness of migratory landbird conservation.}

\normaltalk{\capitalisewords{Elaborate Female Coloration and Song: Diverse Perspectives Lead to a More Comprehensive Understanding of Avian Biology}}{\underline{Kevin E Omland}, Evangeline M Rose, Casey D Haines}{Diverse perspectives on topics ranging from behavior to climate change can lead to a more informed understanding of many ornithological questions. Geographical and gender biases in the history of ornithology have caused missed opportunities and in some cases a backwards understanding of key evolutionary and behavioral processes. Female birds have historically been neglected in studies of elaborate coloration and song. The main framework for thinking about the evolution of elaborate traits has been sexual selection acting on males. Females have either been ignored or assumed not to change. Our work on plumage coloration in orioles indicates that tropical females have elaborate coloration that has been lost in temperate lineages. The ancestral oriole was likely a tropical resident with year-round territoriality and female plumage not diagnosably different from males. More recently, our work on vocalizations demonstrated that female song is much more common than previously thought. Furthermore, the common ancestor of all songbirds likely had female song, suggesting that that social selection acting on both sexes may have caused the original evolution of bird song. We have analyzed author numbers for papers on female song versus bird song in general. Over the last 20 years, significantly more women than men have published on female song. Moving forward, ornithologists from diverse backgrounds who account for female display traits will lead to a clearer understanding of the evolution of coloration and song. This will also facilitate a broader understanding of the full scope of topics in avian ecology, evolution and behavior.}

\normaltalk{\capitalisewords{Present and Future Distribution and Abundance of Aerial Insectivores in the Northwest using Climate and Landcover}}{\underline{Tara E Stehelin}, Fiona KA Schmiegelow}{Northwestern North America is predicted to experience some of the greatest human-caused changes to climate, making study of northern-breeding birds imperative for conservation and management. Populations of long-distance migrants and aerial insectivores such as the Olive-sided Flycatcher (Contopus cooperi) (OSFL) and the Western Wood-Pewee (C. sordidulus) (WEWP) may be impacted disproportionately, due to additional ex situ stressors such as habitat loss. We used 15,637 records of these species from 1049 regional locations sampled between 1992-2014 to generate models of abundance over multiple climate scenarios and time periods. Boosted regression tree models included species-specific detectability offsets and a suite of 37 climate, landcover, topographical and disturbance covariates at a 4 km scale. A second set of models used climate, landcover, one topographical and one disturbance layer with climatic conditions from a baseline time period of 1981 – 2010 and two future time periods: 2041-2070 and 2071-2100 under a comparatively high (RCP8.5) and low (RCP4.5) greenhouse gas emission trajectory. Mean projected outputs revealed high relative abundances in the far northwest, especially in riparian areas. Mean abundances either remained the same or declined across scenarios for OSFL but increased for WEWP.  Accounting for probable time lags in vegetation change, the difference in predicted abundance between time periods was positive for OSFL (i.e. more gains than losses) and negative for WEWP, underscoring the importance of planning proactively for future species-specific needs. Most gains in distribution were in the northern portions of current range and most losses were in the southern part of the region.}

\normaltalk{\capitalisewords{Tropical Convection, Atmospheric Teleconnections, and Possible Links to Wind-Assisted Bird Migration over the North Pacific}}{Steven B Feldstein, \underline{Robert E Gill}, David C Douglas}{This study examines linkages between El Nino/Southern Oscillation (ENSO) and Madden-Julian Oscillation (MJO) tropical convection, North Pacific teleconnection patterns, and wind-assisted bird migration.  The PNA, EP-NP, and WP are the dominant North Pacific patterns during spring, summer, and fall, being weakest in summer.  The PNA and EP-NP (WP) are mostly confined to east (west) of the Dateline, with a zonal (meridional) length scale of about 4,000 (2,000) km.  These patterns undergo little propagation, grow and decay over a 10-14 day period, and are active 2-4 times per season. The PNA is excited by both ENSO and MJO convection, the WP mostly by MJO convection, and the EP-NP is active when both ENSO and MJO convection are suppressed.  Atmospheric teleconnection patterns can assist or hinder migrating birds directly through the direction and strength of their anomalous wind and indirectly through their influence on the location of synoptic-scale storms. Therefore, the presence or absence of ENSO and MJO convection can benefit trans-Pacific migration by generating localized regions of wind assistance. For example, composite analyses based on Bar-tailed Godwit departure dates show that southward bound Bar-tailed Godwits leave Alaska during periods of persistent (3-5 day) northerly winds when the EP-NP pattern is in its negative phase and the PNA is suppressed, which enhances wind assistance over the North Pacific. This alludes to Bar-tailed Godwits, and presumably many other shorebirds and procellarids, using wind direction, strength, and especially persistence at their departure location as cues to maximize wind assistance along their entire migratory route.}

\normaltalk{\capitalisewords{Wind Assistance along the Trans-Pacific Migration Corridor of the Bar-Tailed Godwit: Past, Present, and Future}}{\underline{Robert E Gill}, Jr, David C Douglas, Steven B Feldstein, Catrin M Mills, Phil F Battley, Colleen M Handel}{Wind is likely the most important selection factor in long-distance bird migration, but the behavioral flexibility of birds to respond to future changes in atmospheric circulation that may accompany climate change has received little attention. During autumn, Bar-tailed Godwits (Limosa lapponica baueri) undertake a nonstop, wind-assisted, transoceanic migration of ~10,000 km from Alaska to Australasia. Projections of future climate by GCMs indicate an overall weakening in atmospheric circulation globally, possibly altering the energy expenditures of birds during migrations that already push their physiological limits. Here we assess past and present wind patterns to assess the behavioral flexibility that may be required of birds to migrate under future circulation regimes. We do this using a tailwind index that integrates near-surface winds experienced during simulated migratory flights, and we assess the frequency and duration of beneficial tailwind episodes between present climate, paleoclimate, and future climate projections. In-situ observations show that godwits select advantageous dates to initiate migration, typically affording them net wind assistance for the first half of their southward migration. Under future climates, shifts in winds and storm track will likely create more difficult migrations on average, but optimal departure opportunities will still be available, albeit fewer in number. Continued variability in projected wind conditions in a future climate and the godwits' ability to choose optimal departure days will likely afford the species sufficient resilience to the model-projected declines in average net tailwind assistance along their trans-Pacific autumn migration corridor between Alaska and Australasia.}

\normaltalk{\capitalisewords{Consequences of Long-Term Habitat Change and Resource Selection for Population Limitation in Cavity-Nesting Birds}}{\underline{Thomas E Martin}}{One environmental change that may negatively impact some populations of cavity-nesting organisms is the geographically broad decline of quaking aspen Populus tremuloides, a commonly preferred nest tree. I counted densities of woody plants and eight cavity-nesting bird species over 29 years in 15 high elevation riparian drainages in Arizona. I also studied nest tree use over time based on 4946 nests. Aspen suffered a severe decline in availability over time, while understory woody plants and canopy deciduous trees also declined due to increased elk Cervus canadensis browsing linked to declining snowfall. Six woodpecker species all exhibited very high specialization ($>$95\% of nests) on aspen for nesting, and densities of all species declined with aspen over time. House wren Troglodytes aedon densities declined modestly over time, but nest box addition experiments demonstrated that nest-site availability was not limiting their populations counter to expectations for cavity-nesting birds. Indeed, house wrens increasingly switched to nest trees other than aspen over time. An increase in their densities along with understory vegetation on long-term (10 year) elk exclosures demonstrated that the decline in habitat quality (understory vegetation) on the broader landscape was the cause of their population decline. Resource use plasticity can allow a species, like the wren, to take advantage of a habitat refuge, such as provided by the elk exclosures. Yet, when habitat on the broader landscape is declining in quality, then resource generalization cannot offset the broader negative changes, as demonstrated by the decline in wrens on the broader landscape.}

\normaltalk{\capitalisewords{Heterospecific Sociality: How Cooperation, Competition, and Communication Across Species Shape Avian Behavior}}{\underline{Allison E Johnson}}{Heterospecific interactions play an important role in shaping avian communities. Cooperative or competitive, associations across species can alter foraging, vigilance, patterns of habitat use, and nest site choice. These behaviors can be influenced by Information eavesdropped from the broader heterospecific community or through direct interactions such as mixed species foraging flocks which may allow birds to expand their foraging niche or territorial disputes which result in competitive exclusion. In this introduction to the symposium, I discuss examples of heterospecific associations and discuss broadly how such associations may impact avian behavior and evolution. How are long-term heterospecific associations mediated? How do patterns of habitat use and recruitment in one population impact another? How do heterospecific social interactions differ from conspecific associations? I also discuss briefly how the field is advancing, touching on new techniques—such as social network analysis—and new hypotheses that are helping us to clarify why heterospecific associations occur and with whom.}

\normaltalk{\capitalisewords{Life History Trait Variation Across a Tropical Elevational Gradient}}{\underline{Jill E Jankowski}, Micah N Scholer, Gustavo A Londono}{Variation in life history traits has been demonstrated to occur across broad environmental gradients such as latitude, with the well-documented pattern that the pace of life is 'slower' in the tropics, compared to the temperate zone. Tropical species are expected to show traits consistent with this 'slow pace of life' (small clutch size, long incubation periods, slower development, lower basal metabolic rates, higher survival and longevity). However, tropical regions harbor strong environmental gradients of their own, such as elevation, and it remains to be shown whether similar trade-offs between physiology and species vital rates are reflected among tropical birds of the same latitude along mountainsides. We examine variation in two key life history traits, basal metabolic rate (BMR) and apparent annual survival, for Neotropical birds along a 2600-m elevational gradient in the Peruvian Andes to address how these traits are correlated and how they change from low to high elevations. We found a negative relationship between BMR and survival and no indirect effect of elevation on BMR, suggesting that high BMR in Neotropical birds is associated with low survival, regardless of a species' native altitude. Elevation had a weak, but significant negative relationship with survival. Thus, tropical montane species may be characterized by a unique suite of traits in their 'pace of life,' in which BMR does not differ from lowland species, but survival does. We discuss future avenues for research of high elevation bird communities to improve our understanding of the links between life history, energy metabolism and environment.}

\normaltalk{\capitalisewords{Global Patterns and Predictors of Dispersal-Trait Variation in Birds}}{\underline{Catherine E Sheard}, Monte Neate-Clegg, Nico Alioravainen, Samuel EI Jones, Claire Vincent, Hannah EA MacGregor, Tom P Bregman, Nathalie Seddon, Joseph A Tobias}{An organism's ability to disperse constrains many fundamental ecological and evolutionary processes. Though dispersal is often difficult to measure directly, physical properties of the avian wing can be used as a proxy for flight ability. Here we present the hand-wing index (HWI), a measure of wingtip pointiness, for 10,391 species of birds. Across all birds, HWI is correlated with range size and migratory behaviour, and is most strongly predicted by environmental uncertainty (breeding range temperature and precipitation variability) and by social strategy (territory defence and pair bonding). Passerines and non-passerines, however, exhibit different secondary spatial and temporal correlates of HWI, as do migratory and non-migratory species, indicating several potential trajectories of flight evolution depending on the species' ecological and social niche. HWI evolution most closely resembles an Ornstein-Uhlenbeck process, with higher optima for migratory and/or non-passerine birds than non-migratory and/or passerine birds, and shows high levels of phylogenetic conservatism. The early history of avian wing evolution displays the double-peaked disparity previously observed in other avian traits, with the second peak approximately around the time of the K-Pg extinction. This, coupled with a remarkably consistent HWI throughout the early evolutionary history of birds, could suggest that only certain flight abilities (HWIs approximately equivalent to the modern European Goldfinch) were able to survive the mass extinction event, followed by an adaptive radiation in the wing morphologies of surviving lineages.}

\normaltalk{\capitalisewords{Co-Production of a Continental-Scale Movement and Energetics-Based Framework for Collaborative Management of Waterbird Populations}}{\underline{Wayne E Thogmartin}}{The U.S. Fish and Wildlife Service's Integrated Waterbird Management and Monitoring Program uses two approaches for understanding and managing waterbirds during the non-breeding period. Refuge-level collection of waterbird abundance during the non-breeding period informs chronology and intensity of use as well as the efficacy of management actions. A generalized energetics and movement model, based on a first-principle understanding of avian energetics, behavior, and flight physiology, describes general principles organizing the occurrence and movement of waterbirds during the non-breeding period, including migration. The former is data-driven, the latter based on nothing more than theoretical understanding informed by literature. Together, these two elements are meant to combine to provide a scalable framework for decision making - to understand the regional and continental consequences of local management actions and to identify pinchpoints in the migrational process impinging on waterbird populations and the means to relieve them. We describe the evolution of these two co-produced investigative endeavors, impediments to their integration, and challenges to translating scientific information for sound management action.}

\normaltalk{\capitalisewords{Audubon's Vision: Restoring the Gulf of Mexico for Birds and People}}{\underline{Kara E Lankford}}{The 2010 BP oil disaster in the Gulf of Mexico was the largest oil spill in U.S. history, and it is estimated that as many as one million birds were killed as a direct result of the spill. In 2016, a settlement was reached and restoration funding became available in 2017, enabling a once-in-a-generation opportunity to restore the Gulf ecosystem. The National Audubon Society and its vast network of partners are seizing this moment by developing our Gulf Conservation Plan, Restoring the Gulf of Mexico for Birds and People. With this three-year plan, Audubon is working to produce actionable science by collaborating with scientists, decision-makers, and other stakeholders to design restoration and conservation plans aiming to restore habitat for, and populations of, Audubon's 11 flagship bird species. To achieve these goals, Audubon developed habitat suitability models that incorporate both scientific and local knowledge to identify the highest-priority habitat for breeding and wintering waterbirds. This analysis identified over 8 million acres of highly suitable habitat for Audubon's flagship birds, allowing managers to pinpoint areas of greatest restoration potential in the Gulf. As a result, scientists and stakeholders have worked together to establish a suite of 29 projects that will protect over 136,000 total acres of habitat for birds and people, and represent a \$1.7 billion investment. This is a transformative plan that is made possible through a multi-stakeholder process that addresses the sociological, ecological, and political contexts of an environmental problem, with the goal of restoring the health of the Gulf.}

\normaltalk{\capitalisewords{The Role of Song in Reproductive Isolation in Two Newly Overlapping White-Crowned Sparrow (Zonotrichia Leucophrys) Subspecies}}{\underline{William E Brooks}, Peter H Wimberger, Melvin L Rouse}{Secondary contact zones can be used to examine the mechanisms affecting reproductive isolation. In the North Cascade mountains (Washington state), the first known area of sympatry between the Pacific (pugetensis)and Interior (gambelii) populations of white-crowned sparrows (Zonotrichia leucophrys) has emerged within the past thirty years. White-crowned sparrows are a model species for studying how song-can act as a reproductive barrier. We performed playback experiments to assess male recognition between the populations, and examined the contact zone for indications of hybridization. . The sparrows have an asymmetric response pattern to playback.. Pacific population birds had a significantly greater response to their own compared to Interior songs, while Interior sparrows did not differentiate between song type. This pattern obtained in both behavioral data and acoustic data from the birds' songs. This difference in response correlates with the dialect structure of the two populations. The Pacific population has a complex dialect structure, whereas the Interior population has little to no dialect structure and recognizes multiple song-types within an area. The observed asymmetric response may suggest that gene flow is asymmetric as well, with song acting as a reproductive barrier for one population but not the other.}

\normaltalk{\capitalisewords{Migratory Bird Policies: Changes, Threats and Opportunities for Action}}{\underline{Stanley E Senner}}{Population declines across much of the North American avifauna should prompt reevaluation of the basic laws and policies protecting birds and the habitats on which they rely. Are sufficient laws and policies in place and are they effectively implemented and properly funded? Recent actions and proposed changes—for example to the Migratory Bird Treaty Act and Endangered Species Act--lend urgency to these questions. In this presentation, I will survey the legal framework for protecting birds and their habitats, describe recent and potential changes in law and policy, and highlight opportunities to improve bird conservation through action at the federal level. Of special note is the recent change in interpretation and application of the Migratory Bird Treaty Act with respect to incidental taking of birds. A recent solicitor's opinion at the U.S. Department of the Interior limits prosecutions to cases in which the death of a migratory bird is intentional. This reverses decades of practice in law enforcement and eliminates an important incentive for industries to actively minimize or eliminate bird kills incidental to their structures, activities and wastes. Although the basic laws and policies protecting birds and their habitats are, in and of themselves, not sufficient to protect bird populations at scale, they are the foundations on which many important bird conservation efforts are based and are part of a larger fabric of essential environmental protection.}

\normaltalk{\capitalisewords{Can Stable Isotopes Help Identify Links Between Wintering, Migration, and Breeding Areas of Semipalmated Sandpiper?}}{\underline{Kirsty EB Gurney}, David S Mizrahi, Ray T Alisauskas, The Arctic Shorebird Demographics Network }{To help develop and implement a conservation strategy for semipalmated sandpiper (Calidris pusilla; hereafter sandpipers) and to evaluate interactive effects of environmental change and life history strategies on contaminant levels in Arctic-nesting shorebirds, we are currently collecting and analyzing biological samples (feces, blood, eggs, feathers) from birds at multiple locations across the species' breeding and wintering ranges. Difficulties in tracking sandpipers over their annual cycle and lack of information on migratory connectivity, however, make it challenging to link specific threats – whether related to contaminants or other stressors – to breeding and wintering ground populations. To explore whether the stable isotopic composition of feathers can be used to connect sandpipers across their annual cycle, we are developing and testing an isotopic approach, using feather samples from multiple locations that represent wintering ground signals. For birds with known wintering locations (based on geolocator data and location of capture), we compared regional signals for four stable isotopes (δ13C, δ15N, δ2H, and δ34S) and found a significant amount of isotopic variation across regions that was primarily driven by differences in δ15N, potentially reflecting dietary differences across different areas. Results suggest that a stable isotope approach could improve our understanding of how population threats for avian species vary over the course of their annual cycles.}

\normaltalk{\capitalisewords{Distribution Update, Habitat Use and Conservation Status Assessment of the Grey-Bellied Comet (Taphrolesbia Griseiventris)}}{\underline{Sandra F Cuadros}}{The Grey-bellied Comet is a hummingbird endemic to northern Peru, considered Endangered by IUCN, Critically Endangered by the Peruvian government and "rare" throughout its distribution due to its low numbers. The current Grey-bellied Comet's distribution comprises two Endemic Bird Areas recognized by BirdLife International, and at least one Important Bird Area, highlighting the importance of habitat-based conservation. While the species has been historically recorded in three regions, the region with the highest record density is severely threatened due to slash-and-burn agriculture and a recent proposal to build a dam. In the other two regions the records have been scarce and it is critical to determine whether these are stable populations or if they no longer exist.  To address these questions, I used Maxent to propose an updated distribution map of the species based on recent records. The map was processed in Google Earth Pro and used to identify areas with potential habitat. The selected areas were explored to assess the presence of the species. Where the species was present, a behavioral and habitat characterization was done to identify important ecological requirements and potential threats. Preliminary results show that the species prefers shrubby vegetation in lower valleys with preference for Delostoma integrofolium flowers in one region. While it seems to tolerate some degree of disturbance, how this affect the species presence needs to be further examined. These results will be an important input to re-assess its Red List Category and address appropriate conservation strategies.}

\normaltalk{\capitalisewords{Effects of Non-Breeding Rainfall on Signal Development, Social Behavior, and Reproductive Success in Red-Backed Fairywrens}}{\underline{Joseph F Welklin}, Samantha M Lantz, Sarah Khalil, Jordan Karubian, Michael S Webster}{Ecological variables such as weather and climate play important roles in determining an organism's phenotype and behavior. While most research into the role of ecology in shaping the evolution of birds occurs during the breeding season, the effects of weather and environment are well-known to operate across seasons. Non-breeding habitat quality on the wintering grounds of migratory warblers is known to influence the quality of sexual signals developed prior to migration, which can then impact reproductive success in the following breeding season. In non-migratory species, similar effects of ecology can influence signal quality, but the year-round residency of non-migratory species means the timing of signal development can also become a signal of male quality. Currently missing from both migratory and non-migratory studies is an understanding of the role a dynamic social landscape plays in bridging the gap between ecology and eventual breeding consequences of signal development. Here we present an across-year comparison showing that differences in non-breeding rainfall lead to variation in Red-backed Fairywren social network dynamics through rainfall's influence on sexual signal development. We show that for old males, timing of molt into nuptial plumage is governed by rainfall, but for young males, timing of molt into nuptial plumage is mainly driven by social interactions, which differ across years depending on rainfall and the number of old males in bright plumage. By comparing these findings to reproductive success, these results will help us understand the mechanisms that connect ecology and evolution.}

\normaltalk{\capitalisewords{Integrating Genetic, Population Monitoring, and Climate Data to Inform Conservation of a Declining Migratory Bird}}{\underline{James F Saracco}, Kristen C Ruegg, Ryan J Harrigan, Caz M Taylor, Madeleine Rubenstein}{Effective conservation for declining migratory bird species requires identification of meaningful population boundaries, understanding of migratory connectivity, and development of population models that integrate broad-scale population monitoring and environmental data. Here we present a network connectivity model for Wilson's Warbler (Cardellina pusilla) based on genetically distinct breeding populations that integrates seasonal relative abundance estimates from citizen science data and genetic assignment of overwintering birds to breeding populations. The model showed strong connectivity between breeding genetic lineages in the Pacific Northwest, coastal California, and the Sierra Nevada and wintering grounds in southern Baja California and western Mexico. We then developed an integrated population model that incorporated count (North American Breeding Bird Survey), capture-recapture (Monitoring Avian Productivity and Survivorship), and climate (ClimateNA) data to assess environmental drivers of vital rates and population change in these three breeding genetic groups. Spring temperatures on the breeding grounds were positively related to productivity in the Sierra Nevada and Pacific Northwest breeding regions, but not in coastal California. Adult apparent survival was negatively related to winter drought conditions. Although all three populations have tended to decline, populations were relatively large in the Pacific Northwest. Some conservation actions (e.g., management of winter habitats resilient to drought) will benefit multiple genetic groups. However, others will be population-specific and may thus require choosing between contrasting conservation goals (e.g., preserving the largest number of individuals v. preserving unique genetic lineages).}

\normaltalk{\capitalisewords{Bill Divergence and Speciation in the New World Jays (Corvidae)}}{\underline{Benjamin F Scott}, John McCormack, James Maley, Whitney T Nakashima}{Evolutionary changes in bill morphology have been noted as powerful drivers of diversification. While previous work has focused on how bill trait specialization has influenced speciation, few studies have analyzed how bill traits can impact divergence in a widely dispersed, omnivorous clade. The New World Jay family (Corvidae), occupy a variety of habitats across the montane backbone of the Americas. Bill morphology has remained similar across the family despite significant differences in body size and habitat preference. In this study, I investigate the allometric relationship between body size and bill size, and test how the rate of morphological trait evolution has influenced the speciation rate. Standard linear bill measurements were collected from museum specimens and reduced into a phylogenetic PCA to reconstruct ancestral bill states. I found relatively little change in overall bill size, but a surprising amount of difference in bill shape between genera. To quantify rates of trait evolution and diversification, I calculated species-specific rates of trait evolution using a phylogenetic ridge regression and compared them to tip-specific diversification rates. I show the rate of trait evolution is not associated with the rate of speciation, and investigate the other factors which drive bill shape in Corvidae.}

\normaltalk{\capitalisewords{Where Did Cuban Bobwhites Come From?}}{\underline{Jessie F Salter}, Robb T Brumfield, Brant C Faircloth}{The evolutionary history of Cuban bobwhites (Colinus virginianus cubanensis) has been a source of controversy among ornithologists for nearly two-hundred years. Since they were first noted on the island in 1839, natural historians have debated whether Cuban bobwhites are endemic to the island or were more recently introduced to Cuba from the southeastern United States, as many Cuban naturalists have recounted. Adding to the confusion, bobwhites from the U.S. have been periodically introduced to Cuba since the late nineteenth century, resulting in birds of intermediate phenotypes. Genetic analyses of a few individuals have also produced conflicting results: a recent study of mtDNA showed no differentiation in haplotypes between Cuban and southeastern U.S. populations of bobwhite, while our preliminary data using thousands of genome-wide loci suggest that Cuban bobwhites are most closely related to bobwhites in southern Mexico. To better address this question, we collected genomic and phenotypic data from fifty historical bobwhite specimens collected from Cuba, Florida, and other Caribbean islands that included specimens from Cuban populations where there have been no documented introductions, purported hybrids between Cuban and Florida bobwhites, and specimens from known introduced populations, and we analyzed these data using a variety of approaches to test competing hypotheses regarding the evolutionary history of this enigmatic Cuban bobwhite population.}

\normaltalk{\capitalisewords{Phenology of the Purple Martin Annual Cycle}}{\underline{Jeffrey F Kelly}, Kyle Horton, Phillip Stepanian}{It has been difficult to study the spatial and temporal scaling of phenological change because it requires data at both a fine temporal and spatial grain and a regional to continental spatial extent. We explore these questions by focusing on the Purple Martin (Progne subis). The Purple Martin is an abundant migrant that travels from wintering locations in Brazil to breeding locations in the eastern USA every year. We use data from the eBird citizen science network with those from the US weather surveillance radar network to examine spatial and temporal pattern in first-of-season arrival date, post-breeding roost formation date, end of roost date, and last-of season departure date over an 8 year period. We also examine the variation in these phenologies relative to annual environmental conditions.}

\normaltalk{\capitalisewords{Asymmetric Gene Flow in an Understory Fijian Bird}}{\underline{Ethan F Gyllenhaal}, Xena M Mapel, Michael J Andersen}{Islands have provided a disproportionate amount to the development of allopatric speciation theory, because myriad populations are separated by open-water gaps. These isolated populations are often found to be divergent, but little is known about the extent of gene flow between them. The Fiji Bush Warbler Horornis ruficapilla is a furtive, understory passerine found on the four largest Fijian islands. These populations are described as subspecies based on slight, but noticeable, vocal and plumage differences. We sequenced thousands of ultraconserved elements to infer population structure and patterns of gene flow within this species. We found that the four island populations are reciprocally monophyletic and formed distinct genetic clusters. Each pair of adjacent islands—Vanua Levu and Taveuni in the north and Viti Levu and Kadavu in the south—were close sisters to each other, with deeper divergences between these two pairs. However, our demographic analyses inferred low levels of asymmetric gene flow, not just between each member of a given pair, but also between the pairs of northern and southern islands. For each sister pair, the population on the larger island had much higher levels of gene flow into the smaller islands than vice-versa. The discovery of gene flow between non-sister lineages confirms that despite moderate genetic divergence, these allopatric populations will still interbreed after dispersing over miles of open ocean. Additionally, our finding of uneven gene flow between large and small islands offers a hypothesis to test for inter-island population structure in other Fijian taxa.}

\normaltalk{\capitalisewords{Using Comparative Genomics to Investigate Non-Geographic Drivers of Divergence and Introgression in Panamanian Birds}}{\underline{Jessica F McLaughlin}}{Many studies emphasize the landscape as a principal driver of avian population divergence. However, many key insights into the genomic basis of reproductive isolation may be found in systems where speciation appears to be occurring absent obvious geographic barriers to gene flow, highlighting other influential factors. Of particular interest is which conditions maintain reproductive isolation in secondary contact. In Panama, many species have deep mitochondrial splits between eastern and western populations which come into contact in the absence of clear geographic barriers, yet show abrupt mtDNA turnover, suggesting that limited gene flow occurs across this contact zone. We used genome-wide markers in eight of these species to investigate whether the extent of nuclear introgression is predicted by the level of mitochondrial divergence and to characterize patterns of introgression and divergence across the genome. We also incorporated whole genome sequence from parental populations to map the genomic landscape of divergence across species. In most, introgression occurs to an unexpectedly large degree, and is not correlated to time since initial divergence, instead potentially being influenced by species ecology. We will focus future work in this natural laboratory in Panama to investigate how non-geographic factors impact divergence, how this plays out across the genomes of a broader suite of species, and how this impacts the outcomes of secondary contact.}

\normaltalk{\capitalisewords{Consistent Post-Breeding Movements of Marbled Murrelet from British Columbia to Alaska (2014-2016)}}{\underline{Douglas F Bertram}, Christie A MacDonald, Patrick D O'Hara, Jenna L Cragg, Robin Corcoran, Randal Greene, Kerry Woo}{Knowledge of patterns of seasonal movements are required to inform examinations of the distribution and abundance of marine birds to support of conservation efforts.  We deployed solar satellite transmitters to track the movements of Marbled Murrelet (Brachyramphus marmoratus) during the breeding and post breeding periods.  We tagged birds in British Columbia, Canada, over three years (2014-2016, n=27 ), from three different Marbled Murrelet conservation regions.  The tags were most useful in revealing long distance movements of birds following reproductive attempts.  In all three years birds from B.C. moved to Alaska in the post breeding period.  In 2014, a murrelet from the Northern Mainland Coast conservation region (Kitimat) travelled to waters near Kodiak Island in August.  In July 2015, a bird from the Southern Mainland Coast conservation region  (Desolation Sound) travelled to Glacier Bay.  In early August 2016, a bird from the West and North Vancouver Island conservation region (Clayoquot Sound) moved to the Alexander Archipelago in the Alaskan panhandle.  Historical and ongoing time series counts of murrelets in Alaska show a swelling of numbers in July and August, consistent with the timing of arrival of BC birds, and indicate that our results likely reflect  annual movement patterns.  The arrival of BC birds in Alaska also coincides with the timing of salmon gill net fisheries, which presents a mortality risk to murrelets in areas of known overlap, such as Kodiak Island.  Our tracking work demonstrates strong connectivity between Alaskan and Canadian populations}

\normaltalk{\capitalisewords{Intraspecific Variation and Adaptive Significance of a Little-Known Life-History Trait, Laying Interval Length, along a Latitudinal Gradient}}{\underline{Carl G Lundblad}, Courtney J Conway}{Intraspecific variation in life-history traits like clutch size has been widely documented, but variation in other life history traits remains poorly known. The length of the interval between the laying of sequential eggs within a clutch has rarely been documented or reported. Physiological constraints put a lower bound on the length of egg-laying intervals, but strong directional selection imposed by nest predation should favor the shortest laying intervals possible. Hence, laying intervals are often assumed to be fixed in many species, but these data are extremely difficult to obtain. We used motion-activated nest cameras to non-invasively measure laying intervals of Burrowing Owls at five study sites along a latitudinal transect spanning the intermountain western United States. Laying interval length varied widely within and across clutches and within and across study sites, suggesting that interval length is subject to selection and trade-offs with other life history traits. Interval length declined seasonally, declined across the laying order, and increased with average burrow temperature during the laying period. Laying interval length was negatively associated with clutch size, hatching success, nestling size, and nestling condition on the 10th day post-hatching. Nestling growth over the next 10 days, however, was positively associated with laying interval length, suggesting a possible tradeoff between short laying intervals and nestling growth. The detection of trade-offs between laying interval length and other life-history traits may be precluded by variation in individual quality or condition, accounting for the observed correlations between long laying intervals, reduced hatching success, and small nestling size.}

\normaltalk{\capitalisewords{Investigation of Population Change in a Hurricane-Affected Piping Plover Population}}{\underline{Samantha G Robinson}, Daniel Gibson, James D Fraser, Henrietta A Bellman, Audrey L DeRose-Wilson, Sarah M Karpanty, Katie M Walker, Daniel H Catlin}{Understanding the effects of large-scale disturbances and anthropogenic changes on imperiled species can be important to improving management actions. In October 2012, Hurricane Sandy storm surges cleared vegetation throughout a barrier island, Fire Island, New York, creating piping plover (Charadrius melodus; plover) nesting habitat. Storm effects also prompted an island-wide stabilization project. Given that island stabilization may negatively affect new plover habitat, restoration areas were created to mitigate that loss. Since plovers in NY appear to be habitat limited, we expected to see positive population growth following habitat creation. We developed an integrated population model that incorporated breeding pair counts, nest success, and mark-recapture data to assess the effect of restoration areas on the constituent demographic processes (e.g., reproductive output, adult survival, site fidelity) that determined plover population growth during 2013–2018. We observed positive population growth in three of five years (λ ̅=1.11), and reproductive output parameters were highly correlated (r=0.80–0.87) with population growth. Chick survival was higher in restoration areas, although nest survival was lower at these sites. When restoration area-specific parameters were used to estimate population growth, the combination of higher chick survival and lower nest survival in restoration areas resulted in population growth similar to the whole study area (λ ̅=1.13). Efforts to improve overall reproductive output are likely to have the greatest local effect on population growth if there is suitable habitat to support recruits. In the short term, restoration areas seem to mimic natural plover habitat, but vegetation removal may be needed to maintain habitat suitability.}

\normaltalk{\capitalisewords{The Genomic Landscape of Microgeographic Adaptive Divergence in an Island Endemic}}{\underline{Rebecca G Cheek}, W C Funk, T S Sillett, Cameron Ghalambor}{Fine-scale genetic structure in natural populations is becoming increasingly ubiquitous in the fields of population and landscape genetics. Two main processes underlie these observed patterns: 1) isolation by distance (neutral genetic divergence with increasing geographic distance; IBD), and 2) isolation by environment (adaptive genetic differences associated with environmental differences; IBE). Both processes suggest neutral and adaptive genetic variation can be maintained at small spatial scales (e.g. "microgeographic adaptation", adaptive differences among groups without barriers to gene flow), but few studies have explored the relative contribution of both natural selection and gene flow within continuously distributed populations. To do this, we must disentangle effects of IBD and IBE in the wild and ask how these evolutionary processes drive genetic divergence at smaller scales. The Island Scrub-jay (Aphelocoma insularis) endemic to 250km2 Santa Cruz Island, exhibits a repeated pattern of adaptive bill morphology in three separate pine stands that mirrors the species' mainland congener. Here, we use thousands of single nucleotide polymorphisms (SNPs) to determine the genetic basis of bill morphology and the relative roles of selection and gene flow in shaping adaptive phenotypic divergence across a heterogenous landscape. By relating morphology, genomics and habitat, we have the unique opportunity to assess how gene flow and natural selection shape genetic divergence (IBD and IBE) within a single, range restricted population of birds.}

\normaltalk{\capitalisewords{Life History Structures Phenotypic Variation in Avian Functional Traits}}{\underline{Shane G DuBay}}{Functional demands can vary markedly by species, age, and sex, which together define how traits vary and evolve in bird populations. Studies of avian functional traits, however, are largely restricted to species-level comparisons, ignoring critical variation that exists within species. Here, we leverage within- and between-species comparisons in a clade of small passerines (Tarsiger bush-robins) from the Qinghai-Tibet Plateau to ask how age- and sex-specific demands on function are reconciled in the avian flight muscle, an essential multifunctional organ in birds. The flight muscle is inextricably linked to life history and ecology, mediating a remarkable range of functions. By integrating measurements of flight muscle physiology, morphology, behavior, phenology, and environmental data, we test how age- and sex-specific demands on function are balanced in the flight muscle. We analyze trait data within a context of two widespread, adaptive life-history strategies, age and sex-structured migration and delayed plumage maturation, which provides the necessary framework to uncover the selective forces that shape phenotypic variation within and between species. We find that age- and sex-specific demands on flight muscle function are met through adjustments in muscle size and muscle fiber composition. Within each species, variation in flight muscle phenotypes is structured by age and sex, reflecting discrete life-history stages and the pressures that individuals face as a result of their multi-class identity (i.e. species, age, and sex). Our findings reveal how functional phenotypic variation is maintained and structured within species at the nexus of life history and multi-class identity.}

\normaltalk{\capitalisewords{Does Experimental Local Extinction of a Flowering Plant Affect Hummingbird Movement Patterns?}}{\underline{Kara G Leimberger}, Adam S Hadley, Matthew G Betts}{Plants and pollinators are declining worldwide, in part due to landscape-level processes of habitat loss and fragmentation. In an agricultural landscape in southern Costa Rica, habitat fragmentation has reduced seedling recruitment of the common flowering plant Heliconia tortuosa, putting populations in small, isolated forest fragments at risk of local extinction. However, large-bodied, trap-lining hummingbirds—such as green hermits (Phaethornis guy) and violet sabrewings (Campylopterus hemileucurus)—frequently visit Heliconia flowers, especially during the dry season when other resources are scarce. To explore how local Heliconia extinction influences movement patterns and foraging behavior of these two hummingbird species, we conducted field experiments that temporarily removed H. tortuosa from premontane forest fragments in southern Costa. In the dry seasons of 2016-2018, we attached miniaturized radio transmitters to 49 hummingbirds and tracked their movements over $>$700 hours of radio telemetry. We hypothesized that if hummingbirds are behaviorally flexible enough to feed from other flowering plant species—and these alternative floral resources can supply enough calories—then they should continue to visit focal fragments following Heliconia removal. Indeed, although individual hummingbird responses were highly variable, space use shifted away from focal fragments after we removed Heliconia flowers. These results suggest that large-bodied hummingbirds may not simply 're-wire' to alternative resources if Heliconia populations continue to decline. Future work should thus explore the extent to which loss of these hummingbird pollinators affects reproductive success of the broader plant community.}

\normaltalk{\capitalisewords{Experimentally Increasing Perceived Competition for Nests Reduces Female Reproductive Performance and Extra-Pair Offspring}}{\underline{Troy G Murphy}, Peter Korsten, Martje Birker, Reinaldo Marfull, Jan Komdeur}{Females often accrue fitness benefits through competition. However, substantial costs can be associated with aggressive behavior in females, and some of these costs are sex-specific maternal effects. Because offspring condition is tightly linked to maternal investment -- from development through dependency, a female's allocation to aggression during the breeding season is expected to affect its fitness. We investigated the costs associated with increased female aggression in the blue tit (Cyanistes caeruleus). To do this, we created an experimental treatment where we elicited female aggression by repeatedly presenting stuffed female taxidermic models on top of nest boxes (one at a time). We additionally ran a control treatment by placing empty cages on top of nest boxes. All simulated territorial intrusions occurred after nest construction and defense had begun, but before eggs were laid. We measured whether females that were forced to increase defense in nests shifted energy and resources away from producing multiple high-quality offspring. We found that females with augmented aggression suffered reduced reproductive performance, laying at a later lay date, and that these females produced offspring that showed less begging vigor compared to offspring from control females. We also found that females with augmented aggression had fewer extra-pair copulations (EPCs), which may have arisen for multiple reasons, including that these females may have spent less time seeking EPCs, or because these females may have become so aggressive that they no longer allowed courting males to approach. These results indicate that costs associated with increased female aggression can occur across the breeding cycle, and that there is a substantial cross-generational reproductive cost when females experience multiple aggressive encounters.}

\normaltalk{\capitalisewords{Predation-Mediated Ecosystem Services and Disservices in Agricultural Landscapes}}{\underline{Henrik G Smith}, Matthias Tschumi, Klaus Birkhofer, Johan Ekroos, Cecilia Hjort}{Birds may affect ecological processes in farmland in ways both benefitting and hampering the crop production, but the relative importance of such ecosystem services and disservices are seldom quantified. We studied the contribution of birds and other animals to yield-related ecosystem services (predation of weed seeds or invertebrate pests) and disservices (predation of crop seeds or beneficial invertebrates) in Sweden. We used wildlife cameras to study predation in experimental arenas containing weed/crop seeds or pest/beneficial invertebrates in a design were landscape composition varied. In addition we estimated the relative contribution to predation by excluding vertebrates and/or invertebrates from arenas. Beneficial invertebrates and crop seeds were consumed to a similar or even higher extent than harmful invertebrates or weed seeds. Predation levels were modified by landscape context. Seed predation was dominated by vertebrates, while vertebrates and invertebrates contributed equally to predation of invertebrate prey. Against expectations, the role of bird predation was small compared to that of mammals and did not differ between resource types. Our study suggests that it is crucial to consider trade‐offs between ecosystem services and disservices, but that the role of birds in generating these trade-offs may be limited. However, the value of predation may be context-dependent and the interpretation of results affected by experimental setups used. We review potential bird-related ecosystem services and disservices to agriculture and alternative methods to capture these, and suggest a framework for future studies.}

\normaltalk{\capitalisewords{What Nonstop Trans-Pacific Flights of Shorebirds Mean for Our Understanding of the Energetic and Physiological Limits to Endurance Flight}}{\underline{Christopher G Guglielmo},  {A}ke Lindstr"{o}m}{Historically, predictions from flight aerodynamic theory and known maximal fuel loads of migrant birds suggested that maximal flight distances attainable by birds flying in still air should be 4000 – 5000 km. Longer flight ranges were speculated to be only possible with wind assistance. However, there was evidence that some species could fly much longer and farther than models predicted. The spectacular long-distance flights of many species, particularly shorebirds, that were subsequently documented with new tracking technologies have forced a reckoning between theory and data. Several analyses have attempted to show how, with careful fuel budgeting and an assumption of extremely low body drag, continuous flights of 10,000 km lasting 8 – 9 days are obtainable. These analyses have also led to the idea that most empirical measurements of flight energy costs made in wind tunnels and by other means are far overestimated. From the energetics standpoint, flight distance and ground speed are fairly irrelevant. Only exercise duration and rate of energy expenditure are important. Using empirically derived measurements or allometric predictions of flight energy costs indicates that for birds to fly for 8 – 9 days continuously would require almost twice the amount of fat they are known to carry, or conversely that actual flight energy expenditure is one half or less of that expected in these species. We will discuss these issues, as well as the physiological and biochemical tricks that extreme long fliers may use to meet their needs for energy, nutrients and water during these flights.}

\normaltalk{\capitalisewords{Quantifying the Stability of Birds' Cultural Niches: Changing Public Perceptions of the North American Avifauna}}{\underline{Justin G Schuetz}, Alison Johnston}{Bird conservation efforts are constrained by poor understanding of why species are viewed differently within contemporary culture and to what degree perceptions of birds change over time. In previous analyses, we used Google Trends data and eBird data to characterize the magnitude and geographic specificity of public interest in 621 North American bird species averaged over a 10-year period. The resulting metrics of popularity and congruence enabled us to define the positions of species in "cultural niche space". We did not explore whether niche positions shifted through time, however, as species became relatively more or less popular or as the distribution of public interest shifted in relation to the geographies occupied by species. Here, we address this issue by tracking species' movements through cultural niche space between sequential 5-year periods and exploring potential causes of those movements. On average, public interest in introduced, endangered, and large-bodied species increased over time and geographic specificity of public interest in introduced taxa decreased between periods. Among species that shifted the most in cultural niche space, apparent causes of shifts were extremely variable. We attributed some changes to gradual shifts in search volumes relative to encounter rates over time, some to spikes of interest correlated with news events, and some to changes in how Google summarizes queries. By characterizing the positions of birds in contemporary culture, and assessing the degree to which those positions are conserved or flexible, we hope to provide further leverage for understanding and shaping public perceptions of wildlife populations.}

\normaltalk{\capitalisewords{Australo-Papuan Perspectives on Unravelling Tangled Knots on Branches of Species Trees?}}{\underline{Leo G Joseph}}{Molecular approaches to species-level systematics and population structuring in Australo-Papuan birds have recently surged. This work builds on an extensive body of morphology-based taxonomy, which provides a wealth of taxonomic hypotheses for testing. Similarly, the region's complex geological history provides a dramatic palaeoenvironmental backdrop. I show strengths of various molecular datasets and analytical pitfalls that can derail accurate taxonomic interpretation of results in the Australo-Papuan setting. A first area is that of mitochondrial captures, illustrated by the Copper-backed Quail-thrush (Cinclosoma clarum), butcherbirds (Cracticus), rosellas (Platycercus) and others as well as demonstrating how to recognize and test for these captures when they are in progress or complete. Understanding these captures critically informs how they can support or reject alternative taxonomic hypotheses. Selection at the DNA level, especially in mtDNA work, if not properly expected and assessed, could mislead taxonomic interpretation as the Eastern Yellow Robin Eopsaltria australis shows. 'Leaky' introgression might be seen to challenge species-level taxonomy; examples from honeyeaters and fairy-wrens (Gavicalis and Malurus spp, respectively) are informative here. Occasional hybridization still delivers interesting challenges and an example from New Guinean orioles (Oriolus spp) nicely illustrates this. Finally, two examples from teal (Anas) and woodswallows (Artamus) illustrate how we need to consider autosomal and sex-linked loci in understanding some of the remaining challenges. And then there's the Pacific Islands…}

\normaltalk{\capitalisewords{Are Rates of Divergent Evolution in "Biotic Interaction" Traits Faster in the Tropics? a Test using Bird Beaks}}{\underline{Benjamin G Freeman}, Dolph Schluter, Joseph Tobias}{The "biotic interactions" hypothesis proposes that strong biotic interactions in the tropics explain in part the latitudinal diversity gradient. The assumption is that strong biotic interactions drive faster divergent evolution--and hence higher rates of speciation. An increasing number of studies attempt to measure whether biotic interactions are indeed stronger in the tropics as proposed. However, the evolutionary consequences of strong interactions have been largely ignored. Here we address this data gap by measuring evolutionary rates in divergence of bird beaks across latitudes at the global scale. Beak morphology is a phenotypic trait that is tightly linked to resource use (i.e., a biotic interaction), and this dataset is therefore suitable to address the prediction that divergent evolution in "biotic interaction" traits is faster in the tropics than the temperate zone. Alternately, greater ecological opportunity in the species-poor temperate zone may spur greater divergence at high latitudes. We explore these competing predictions by analyzing evolutionary rates of divergence in bird beaks for $>$1,000 pairs of sister taxa of birds.}

\normaltalk{\capitalisewords{Effects of Lead Exposure on Reproductive Success and Extra-Pair Paternity in a Common Urban Songbird}}{\underline{Lauren G Hitt}, Sarah Khalil, Annelise Blanchette, Myra E Finkelstein, Renata D Ribeiro, Erik N Iverson, Stephanie C McClelland, Jordan Karubian}{Chronic, sublethal exposure to lead, a common urban contaminant due to historic use in paint and gasoline, causes behavioral and physiological problems in humans living in cities, but little is known about the effects of lead exposure on urban wildlife. We studied the northern mockingbird (Mimus polyglottos) in three neighborhoods with variable habitat parameters and background lead levels in New Orleans, LA, USA, where lead exposure remains a notable public health concern. Our previous work in this system suggests that mockingbirds in high-lead neighborhoods are more aggressive and have higher concentrations of lead in their blood and feathers. In this study, we examine how the reproductive output and cuckoldry rates of mockingbirds may be impacted by lead exposure. On the one hand, more aggressive, high lead mockingbirds may obtain higher quality territories and resources, and mate guard more aggressively, leading to high reproductive output and reduced rates of extra-pair paternity among their offspring. On the other hand, high lead mockingbirds may experience reduced reproductive output on account of the impacts of lead exposure on parental behavior and/or the survival of eggs and nestlings, and increased cuckoldry through infertility or excessive aggressive behavior. Our findings provide insights into how lead exposure may mediate fitness via tradeoffs between aggressive behavior, cuckoldry rates, and offspring survival.}

\normaltalk{\capitalisewords{Waterfowl in North America: A Successful Story of Partner-Based Conservation and Regulated Harvest}}{\underline{Michael G Brasher}, Thomas E Moorman}{At least 45 species of waterfowl regularly breed in North America, occupying habitats as diverse as the marine waters and tundra of the high Arctic to tropical mangrove swamps of Mexico. Collectively, these species compose arguably the most well-studied and heavily harvested group of wildlife on the planet. Since the turn of the 20th Century, these species have experienced remarkable fluctuations in population size and status, with several species once threatened with extirpation since recovering to historic levels. Additionally, total breeding population sizes across 17 of the most common duck species were at their 2nd lowest level ever recorded as recent as 1990, yet many of these species have now rebounded to near record highs. Although these recoveries are partially attributable to the great ecological plasticity and resiliency of most waterfowl species, sharing in these successes are the concerted efforts of an invested constituency, dedicated Federal and State funding, international science-based conservation partnerships, and an information-based, multi-jurisdictional system of harvest management. Despite these notable accomplishments, several species remain below objective levels established by the North American Waterfowl Management Plan and its partners, while others are now viewed as "overabundant" and contributing to ecological damage of sensitive breeding habitats. Waterfowl populations have benefitted greatly from diverse partnerships that have delivered habitat conservation at landscape scales across the continent, yet accelerating and emerging threats to these habitats will demand greater and more innovative conservation approaches if waterfowl populations are to remain as abundant and resilient as observed in recent decades.}

\normaltalk{\capitalisewords{Songbirds Alter their Use of Bird Feeders in Response to Vocalization Playback}}{\underline{Douglas G Barron}, Shelby D Sarna, Robin L Middleton}{Bird vocalizations transmit information that may attract or repel other individuals. Basic research aimed to identify the function of vocalizations has given rise to applications utilizing playback to attract birds to colonize new areas or repel birds from undesired locations. Comparatively less is known about how birds respond to vocalizations in the context of supplemental food; that is, do vocalizations act to advertise the presence of (attraction) or to defend (repulsion) supplemental food sources? Here we focused on the most widespread form of supplemental wildlife food, bird feeders, to test the hypothesis that birds modify their usage of bird feeders based on vocalizations from nearby birds. We tested this hypothesis by comparing bird visitation rates and durations at feeders in the presence or absence of vocalization playback from a nearby speaker. The vocalizations were obtained from actively foraging individuals of six species that commonly visit bird feeders. We found that birds were equally likely to visit feeders in the presence and absence of playback, though they spent half as much time at feeders when playback was present. Birds appeared to habituate to this stimulus, as this pattern of shorter visit duration disappeared with prolonged playback. This experiment demonstrates that birds do modify their use of bird feeders based on vocalizations from nearby birds and implies that foraging vocalizations may function to repel competitors. Although further research is needed, our findings also suggest playback could be applied to repel avian pests from unwelcomed food sources such as agricultural crops and feeds.}

\normaltalk{\capitalisewords{Contrasting Long-Term Population Trends of Beach-Nesting Shorebirds under Shared Environmental Conditions}}{Eunbi B Kwon, Samantha Robinson, Chelsea Weithman, James D Fraser, Sarah M Karpanty, \underline{Daniel H Catlin}}{Identifying the drivers of long-term population change is complicated by a number of extrinsic factors that often covary with time and by intrinsic factors that operate on a time lag. For migratory shorebirds that breed on the barrier islands of eastern North America, populations may be limited by the anthropogenic, climatic, or physical environments that they encounter throughout the annual cycle. Using three-decades (1985–2017) of population monitoring data collected by the National Park Service at two national seashores in North Carolina, we examined the potential drivers of Piping Plover (Charadrius melodus) and American Oystercatcher (Haematopus palliatus) populations nesting there. Our modelling revealed a strong effect of human activity and subsequent protection efforts to reduce disturbance on the population trajectory of plovers, but only a weak effect of breeding and wintering climatic conditions, population productivity, and nesting habitat availability. Thus, a decade-long decline in plover numbers at both seashores starting in the mid-90s was reversed as the parks' visitor counts decreased and they implemented stricter protections from potential disturbance. However, sympatric populations of oystercatchers showed the opposite population trends at the two seashores, increasing after a hurricane improved habitat and reproductive output increased. Our study demonstrates a strong effect of the anthropogenic environment on a threatened species and, simultaneously, the important role of stochastic events in shaping populations of long-lived shorebird species.}

\normaltalk{\capitalisewords{Is Mutation Rate Linked to Diversification in Birds?}}{\underline{Carl H Oliveros}, Brant C Faircloth}{Studying how rates of molecular evolution is associated with net diversification is important in understanding how diversity is generated across the tree of life.  Previous work using a sister-pair approach based on protein-coding sequence data from 19 genes and 64 avian families found a significant but weak correlation between mutation rates and net diversification in birds. In addition, improved knowledge of avian higher level relationships in the last decade shows that 28\% of the sister pairs used are not sister clades, which likely led to underestimates of species diversity in each of the pairs.  We re-test this finding using phylogenomic data from 4303 genome-wide loci from all 237 avian families. We find a weak and insignificant correlation between mutation rate in ultraconserved element loci and net diversification in birds. Further study examining this relationship using other types of genome-wide loci are recommended.}

\normaltalk{\capitalisewords{The Importance of Dynamic Conservation Programs to Shorebirds in California's Central Valley}}{\underline{Gregory H Golet}, Kristen Dybala, Matthew Reiter, Kristin Sesser, Mark Reynolds, Julia Barfield, Paul Spraycar}{To meet the habitat needs of migratory species, active management of limited resources is often required. Such is the case in California's Central Valley where the allocation of water is typically needed to create habitat for migratory waterbirds. Being efficient in these allocations requires applying the right amount of water at the right places, in a manner that is responsive to changing conditions. To help increase efficiency of habitat incentive programs that deliver temporary shorebird habitat on agricultural lands, we studied inter- and intra-annual variability in habitat deficits, and bird response to experimentally flooded fields, from 2014-2017, when California experienced both extreme drought and floods. Deficits were identified by combining satellite imagery of surface water with an existing bioenergetics model that estimates whether there is enough habitat to meet the daily energy requirements of the shorebird community. With this information we identified when shorebird habitat needs were consistently met, consistently unmet, or when there was high variability in the deficit from year to year depending on surface water availability. We then estimated the annual contribution of two habitat incentive programs in reducing the energy shortfalls: The Nature Conservancy's Bird Returns program and the Natural Resource Conservation Service's Waterbird Habitat Enhancement Program. Our results suggest that to address periods of consistent inter-annual habitat deficit, typical static conservation approaches that involve consistent flooding of managed lands may be adequate. However, to address highly variable deficit periods, a dynamic approach that matches the allocation of limited resources to conservation needs is far superior.}

\normaltalk{\capitalisewords{Earlier Spring Arrival of Migrant Birds at the Aras Ornithological Research Station, Igdir, Turkey}}{Joshua J Horns, Michael Ford, Kayahan Agirkaya, Emrah Coban, Berkan Demir, Lale Aktay, \underline{Cagan H Sekercioglu}}{As climate change causes earlier spring onset, many species of migratory bird must advance the timing of their arrival in temperate regions to continue to take advantage of seasonal food peaks. However, the biological characteristics associated with migratory plasticity remain poorly understood, as do the potential health consequences for both individual birds as well as breeding populations. In this study, we look at changes in the timing of spring migration over 12 years for 71 species of bird migrating through eastern Turkey, a region largely lacking in long-term ecological studies. We also investigate whether certain species characteristics, including migratory distance, primary diet, primary habitat, and specialization, are associated with changes in migratory phenology. Finally, we look to see if changes in spring arrival are associated with differing levels of fat reserves in individual migrants as well as overall changes in species abundance. Our results suggest that 22 species have significantly advanced the date of their spring arrival, 5 have significantly delayed spring arrival and 44 have shown no significant change. Across all 71 species, birds are arriving on average 1.8 days earlier/decade.}

\normaltalk{\capitalisewords{Nocturnal Bird Migration and Light: Midnight Sun, a Solar Eclipse and Artificial Light}}{\underline{Cecilia H Nilsson}, Kyle G Horton, Adriaan Dokter, Benjamin Van Doren, Andrew Farnsworth}{Almost all life on earth uses the daily cycles of light and darkness to entrain activity patterns. Not least so the billions of, otherwise diurnal, birds that migrate under the cover of darkness, making up massive nocturnal movements all across the globe. So, what happens when birds are faced with unexpected interruptions to the daily light cycle? By using continental networks of weather radars and small-scale avian radars we have investigated several aspects of what role light plays in nocturnal migration. At high latitude sites there is no nighttime darkness for large parts of the migratory season. Yet, we can show that birds largely keep their nocturnal rhythm, also under midnight sun conditions. Birds also face even more unpredictable light situations, such as the drastically changing light levels of a solar eclipse. The 2017 American eclipse took place midday during the autumn migration season, and we investigated how the amount of biological targets in the air changed during the eclipse, compared to the normal migratory activity at sunset. Apart from these natural disruptions to the light cycle, nocturnal migrants now also have to deal with a new kind of light disturbance: artificial light at night. However, the risks artificial light poses to migratory birds is not uniform in space or time. We quantified where and when the highest number of migrants are exposed to artificial light across continental US to help guide mitigation actions to minimize the impact of artificial light on migratory birds.}

\normaltalk{\capitalisewords{Satellite Tracking a Wide-Ranging Endangered Vulture Species to Target Conservation Actions in the Middle East and East Africa}}{Evan R Buechley, Michael J McGrady, Emrah Coban, \underline{Cagan H Sekercioglu}}{Vultures comprise the most endangered avian foraging guild (obligate scavengers) and their loss from ecosystems can trigger trophic cascades, mesopredator release, and human rabies epidemics, indicating their keystone species status. Vultures' extremely large home ranges, which often cross international borders of countries that have differing laws and capacity for wildlife conservation, makes conserving them challenging. However, satellite-tracking data can be used to identify habitat preferences and critical sites to target conservation actions. We tracked 16 Egyptian Vultures, Neophron percnopterus, in the Middle East and East Africa. We used dynamic Brownian bridge movement models to calculate home ranges and core-use areas, and we analyzed habitat use in a resource selection framework. Combined summer and winter ranges (99\% utilization distributions) of all birds covered 209,800 and 274,300 km2, respectively. However, the core-use areas (50\% utilization distributions) in the summer and winter ranges, accounted for only 0.4–1.1\% of this area (900 and 3100 km2, respectively). These core-use areas are where the home ranges of multiple individuals overlapped and/or where individuals spent a lot of time, such as feeding and roosting sites, and are places where conservation actions could focus. Resource selection models predicted Egyptian Vulture occurrence throughout little-studied parts of the species' range in the Middle East and East Africa, and revealed strong selection for proximity to highways, power distribution lines, and towns. While providing roosts (e.g. power pylons) and food (e.g. garbage dumps), anthropogenic features may also function as ecological traps by increasing exposure to electrocution and dietary toxins.}

\normaltalk{\capitalisewords{Stranger Things: Elaboration and Innovation of Island Avifaunas}}{\underline{Gavin H Thomas}, Chris Cooney, Andrew Beckerman, Jen Bright}{Islands are widely considered as laboratories of evolution harbouring many endemic species with diverse and novel phenotypes. However, whether these novel phenotypes are simply elaborations of mainland forms or are innovations, with species evolving in new directions, is unresolved. We define elaboration as divergence along a common major axis of trait change and innovation as divergence orthogonal to the major axis. Innovation may therefore imply an increase in potential evolvability.  Here, we ask how the strength and stability of correlations among avian ecomorphological traits (beak size, beak shape, and body mass) changes among island species, relative to their mainland relatives. We do this by estimating phylogenetically controlled phenotypic divergence matrices for island and mainland species in multiple clades of birds. We adopt quantitative genetic methods developed for estimating phenotypic variance-covariance matrices of life-history traits to visualise and compare the evolutionary mode of trait correlations. Our results suggest that although island novelty is not ubiquitous both elaboration and innovation contribute to the phenotypic novelty of island avifaunas, and that their relative importance may be contingent on phylogeny.}

\normaltalk{\capitalisewords{Global Raptor Research and Conservation Priorities: Tropical Raptors Fall Prey to Knowledge Gaps}}{Evan R Buechley, Andrea Santangeli, Marco Girardello, Montague H Neate-Clegg, Dave Oleyar, Christopher J McClure, \underline{Cagan H Sekercioglu}}{Raptors serve critical ecological functions, are particularly extinction‐prone and are often used as environmental indicators and flagship species. Yet, there is no global framework to prioritize research and conservation actions on them. We identify for the first time the factors driving extinction risk and scientific attention on raptors and develop a novel research and conservation priority index (RCPI) to identify global research and conservation priorities. We use random forest models based on ecological traits and extrinsic data to identify the drivers of risk and scientific attention in world's raptor species. We then map global research and conservation priorities. Lastly, we model where priorities fall relative to country‐level human social indicators. Raptors with small geographic ranges, scavengers, forest‐dependent species and those with slow life histories are particularly extinction‐prone. Research is extremely biased towards a small fraction of raptor species: 10 species (1.8\% of all raptors) account for one‐third of all research, while one‐fifth of species have no publications. Species with small geographic ranges and those inhabiting less developed countries are greatly understudied. Regions of Latin America, Africa and Southeast Asia are identified as particularly high priority for raptor research and conservation. These priorities are highly concentrated in developing countries, indicating a global mismatch between priorities and capacity for research and conservation. A redistribution of scientific attention and conservation efforts towards developing tropical countries and the least‐studied, extinction‐prone species is critical to conserve raptors and their ecological functions worldwide. We identify clear taxonomic and geographic research and conservation priorities for all raptors, and our methodology can be applied across other taxa to prioritize scientific investment.}

\normaltalk{\capitalisewords{Climate Warming Reduces Population Growth and Recruitment Rates over 30 Years in an Afrotropical Bird Community}}{\underline{Montague HC Neate-Clegg}, Thomas R Stanley, Cagan H Sekercioglu, William D Newmark}{Tropical mountains are centres of biodiversity and endemism and contain many species that are believed to be sensitive to climate change.  Yet, hitherto most research assessing the impact of climate change on tropical species has focused on range shifts.  Here we report on the demographic response of 35 Afrotropical montane bird species in Tanzania over 30 years to changes in temperature and precipitation.  More than 80\% of focal species exhibited population growth rates $<$ 1. For one-third of species, declining realised population growth and recruitment rates were associated with increases in temperature. The association of these same demographic parameters with precipitation was, however, less pronounced and consistent among species. Apparent survival was also relatively unaffected by either temperate or precipitation.  While this is the first study to demonstrate temperature-related demographic impacts across a tropical bird community, it also highlights the need to understand mechanisms underpinning these responses.}

\normaltalk{\capitalisewords{Effects of Habitat Type on Provisioning Florida Burrowing Owl Space Use and Movements}}{\underline{Elizabeth HW Rose}, Raoul K Boughton}{Some wildlife can inhabit urban and suburban areas within their native range, and to effectively manage these species, we must understand how the urban landscape impacts their behavior and ecology. The Florida Burrowing Owl (Athene cunicularia floridana) is one such species and is found in both developed areas and rural grassland habitats. Most previous research on this urban-exploiting subspecies has been conducted on populations in residential coastal cities, and there is no information on their home range requirements or adult movements in either urban or rural landscapes. We applied backpack-style GPS loggers to male Burrowing Owls, set to record overnight locations over two-week periods when they had young in the nest to determine how habitat type influences the amount of space required during chick-rearing along with the spatial behaviors of provisioning males. Results from the 2017 and 2018 breeding seasons (n = 47 males) indicate that while the average urban male uses less space (measured as 95\% UD), this difference is not significant. However, there is significantly greater variation in the amount of space that rural males utilized, which suggests that resources in urban sites are less spatially variable than in rural landscapes. Additionally, we will determine the effects of habitat type on the timing of forays away from the nest burrow and on the maximum distances that provisioning males travel nightly to evaluate how they alter their behavior in the urban landscape (analysis in progress).}

\normaltalk{\capitalisewords{Validating the Barcode-Biomass Match: can DNA Barcoding Determine the Contribution of Arthropod Taxa to the Diet of Birds?}}{\underline{Yvonne I Verkuil}, Richard Ubels, Jelmer Samplonius, Marion Nicolaus, Maurine Dietz, Annabet Galema, Kim Kiekebos, Peter de Knijff, Christiaan Both}{The effect of fluctuations in the emergence and abundance of arthropod species on diets of insectivorous birds is largely unknown. DNA barcoding of fecal samples is a promising technique to track diet composition across habitats and time, but how accurately can we estimate the biomass contribution of prey species in fecal samples using amplification-based barcoding? As a validation study, we COI-barcoded 63 fecal samples of Pied Flycatcher (Ficedula hypoleuca) chicks, whose diet was determined by camera observations of prey items and sizes delivered to the brood. We optimized DNA extraction, PCR conditions and sequencing depth until 90\% of the samples yielded on average 14,351 reads of which 96.3\% were assigned to Animalia. We modified published COI primers (LCO1490 and HCO1777) to reduce the reads "lost" to bird DNA (from 6\% to 0.001 \%) and to improve  amplification of spiders. Among the 27 most common orders within Arachnida, Insecta, Chilopoda, Diplopoda, and Malacostraca (representing 98\% of all Animalia reads), the average proportion of reads in the fecal samples was strongly correlated with the relative biomass scored in the camera sessions (R2 = 0.82). Both datasets identified Lepidoptera, Diptera, and Coleoptera as the most important prey taxa. But does the match between barcodes and camera observations hold at lower taxonomic levels? We will present detailed analyses at the family and genus level. A teaser: as expected, Hymenoptera was underrepresented on the cameras, as we missed parasitic wasps, but both methods identified the same Lepidoptera species as most commonly eaten.}

\normaltalk{\capitalisewords{A Comparison of Spring and Fall Body Condition and Hematological Parameters in Two Migratory Songbirds}}{\underline{Margret I Hatch}, Robert J Smith}{A variety of physiological measures such as body condition and hematocrit have been used to assess habitat quality and health in migratory birds. Often these measures are taken only once, but several studies have found seasonal differences in both migratory and resident species. Based on our long-term study in Northeastern Pennsylvania, we were interested in comparing a suite of physiological measures between fall and spring captures for Gray Catbirds (Dumetella carolinesis) and White-throated Sparrows (Zonotrichia albicollis). Measures included a scaled body mass index, hematocrit, total plasma protein, and total white blood cell count. Initial results show higher scaled body mass in the spring than fall for both species, as well as higher hematocrit and total protein in the spring. If total white blood cell count is also higher in the spring, this would suggest individuals are in overall better condition in the spring than fall. Seasonal differences could indicate different strategies for fall and spring migration and/or differences in quality of the habitat. Differences may also be due to changes in demographics of the birds captured (e.g. all spring birds are adults while many fall birds are juveniles). To help account for this, we also plan to compare individual Gray Catbirds captured during spring and fall within a year to examine if individuals show the same seasonal patterns. Documenting seasonal variation is an important first step, but further research would be necessary to determine causes of seasonal changes in body condition and hematological parameters.}

\normaltalk{\capitalisewords{Results from a Decade of Banding Yellow Rails in Oklahoma and Texas}}{\underline{Christopher J Butler}}{Yellow Rails (Coturnicops noveboracensis) winter from North Carolina to Texas and inland to southeastern Oklahoma. Although listed as a species of special concern, little is known about the winter ecology or population sizes of this secretive species. Yellow Rails have been banded in southeastern Oklahoma since 2008 and, since 2009, in conjunction with a collaborator in Texas. The project had three goals; 1.) estimate the population density in Oklahoma and Texas; 2.) determine the habitat requirements; and 3.) identify where these birds bred. During the winter of 2009-10, 56 Yellow Rails were banded and the population density was estimated to be 5.3 ± 1.5 rails/ha, similar to the estimates obtained for coastal Texas during the same time (5.2 ± 1.3 rails/ha). However, since 2011, no more than four rails have banded during each winter in Oklahoma while the numbers of birds detected in Texas remains consistent. Occupancy modeling suggests that occupancy is related to the amount of grasses and forbs, and that the number of stems between 10-20 and 20-30 cm in height are particularly important. Rectrices from Texas (n = 40) and Oklahoma (n = 32) had similar δD values, suggesting that birds in Oklahoma and Texas breed in the prairie provinces of Canada. We suspect that the decline in Oklahoma is due to an increase in sumpweed (Iva annua), broomsedge (Andropogon virginicus), and saltmarsh aster (Symphyotrichum divaricatum) and suggest that a more active management regime will be required to maintain suitable habitat for this species in Oklahoma.}

\normaltalk{\capitalisewords{New Technologies and Analytical Opportunities for Improved Inference About Avian Social Structure}}{\underline{Sarah J Clements}, Matthew J Silk, David H Hodgson, Mitch D Weegman}{Social structure is widespread in birds, and is a key mechanism influencing population dynamics in many species due to influences on survival and reproduction. There is a rich history of research focused on the mechanisms and drivers of fitness resulting from social structure, but new technological and analytical opportunities provide unprecedented scope for evaluating complex hypotheses via multiple data sets and linking individual and population processes in a unified framework. We review research on social structure in the context of fitness and describe knowledge gaps, such as the carry-over effects of social structure and its influence on demographic patterns. We demonstrate how new tracking technology, which collects positional and behavioral data, can yield improved understanding of spatio-temporal interactions among individuals and the consequences of decision-making. We use an emerging statistical tool, integrated population modeling (IPM), to simultaneously incorporate multiple data sources that link social structure and demography. We calculate social network metrics based on tracking data, and suggest that they can be used to explain variation in demography via regressions in the IPM framework, thus linking individual and population processes. Flexible data combinations and joint analyses improve inference about the implications of social structure and subsequent demography, equipping practitioners with holistic estimation for broad application in avian ecology, conservation and management.}

\normaltalk{\capitalisewords{Age-Related Variation in Bite Force in a Long-Lived Seabird in the Galapagos}}{\underline{Erynn J Rebol}, David J Anderson}{Connecting declining performance in old age at the demographic level (reproductive and actuarial senescence) to underlying physiological causes is challenging in wild populations.  This applies especially to long-lived species that must be marked around birth in order to recognize elderly individuals perhaps decades later.  I used age-related variation in bite force of known-age, long-lived Nazca boobies (Sula granti) from a 37-year longitudinal study as an in vivo indicator of muscle function.  Nazca boobies show large declines in survival and reproductive success in old age, and I predicted that age-related decline in muscle function contributes to this demographic senescence.  The ability to obtain a larger sample size of bite force measurements from this species provides a solid opportunity for signal detection.  Does bite force vary with age, controlling sex and size, and thus contribute to senescent decline in demographic variables?  I predicted weaker bite force in old-age birds than in middle-age.  I also predicted stronger bites from females than males (given sexual size dimorphism) and from larger birds (controlling sex).  Data were collected from 360 adults using a force transducer during the breeding season in November 2017 on Isla Española, Galápagos.  Preliminary results support an age-related trend in bite force production.  This study will provide information on senescent trends in physiological factors, merging the topic of senescence with a novel application of bite force.}

\normaltalk{\capitalisewords{Variability in the Use of Acoustic Space Between Two Tropical Forest Bird Communities}}{\underline{Patrick J Hart}, Kristina Paxton, Grace Tredinnick}{When acoustic signals sent from individuals overlap in frequency or time, acoustic interference and signal masking occurs, which may reduce the receiver's ability to discriminate information from the signal. Under the acoustic niche hypothesis (ANH), acoustic space is a resource that organisms may compete for, and signaling behavior has evolved to minimize overlap with heterospecific calling individuals. Because tropical wet forests have such high bird species diversity and abundance, and thus high potential for competition for acoustic niche space, they are good places to examine the way acoustic space is partitioned. Here we examine the fine-scale use of acoustic space and the relevance of the ANH in two montane tropical wet forest bird communities that vary significantly in bird species richness but have similar overall bird abundance. The first location was within a large forest fragment near the Las Cruces Field Station in southern Costa Rica and the second in Hakalau Forest National Wildlife Refuge on the island of Hawaii. Using automated sound recorders at replicate locations in each area, we found that bird vocalizations in species-poor Hawaii had significantly more temporal overlap than species-rich Costa Rica. Similarly, we found that vocalizations of bird species in Hawaii contained a significantly higher percentage of overlapping frequencies than those in Costa Rica. This different use of acoustic space between the two bird communities, with partitioning being greater in the species-rich community, provides support for ANH in Costa Rica but not Hawaii and is likely due to selection to reduce heterospecific signal overlap.}

\normaltalk{\capitalisewords{Incubation Behavior of Trumpeter Swans Measured by Continuous Infrared Videography}}{\underline{David J Delehanty}, David A Bush}{The Trumpeter Swan (Cygnus buccinator), North America's largest waterfowl, was threatened with extinction during the 20th century when the formerly widespread species had declined to small, migratory populations breeding in remote Alaska and NW Canada, and one non-migratory population of fewer than 100 adults breeding in the Greater Yellowstone Ecosystem (GYE) of Montana, Wyoming, and Idaho. Population restoration has been sufficient such that "expansion flocks" within the GYE now can be studied in greater detail. We used around-the-clock infrared videography to conduct a fine-scale examination Trumpeter Swan incubation behavior at Grays Lake National Wildlife Refuge (GLNWR), a historic breeding site that was repopulated by translocating wild swans from from the core GYE population to GLNWR in 1988-91. Based on 11 nests across 4 years of study, incubation was characterized by continuous nocturnal incubation (~100\%) and high diurnal incubation constancy (~75\%) predictably interrupted by 1-4 recesses/d. Incubation recess occurrence peaked bimodally 6 h after sunrise and 6 h prior to sunset. Diurnal recess duration varied by date and time interval within day, but averaged ~ 76 min/recess. Importantly, we did not witness predatory or other territorial intrusions that disrupted incubation.  We interpreted incubation behavior to be sensitive to prevailing environmental temperature based on timing of recesses and to be insensitive to predation risk based on the absence of overt cryptic behavior by incubating adults. Nest security may be aided by large size of eggs, high territorial vigilance, and the vastness of the marsh within which nesting occurred.}

\normaltalk{\capitalisewords{Modelling the Niche in Present and Past Climatic Scenarios for the Great Tit}}{\underline{Fernando J Machado-Stredel}, Gang Song, Ruiying Zhang, Per Alstr"{o}m, Yanhua Qu, Huijie Qiao, Herman Mays, Per G Ericson, Jon Fjelds {a}, A T Peterson,  L Chinese Academy of Sciences}{The Great Tit (Parus major) is a charismatic bird found in most of the Old World. This widespread species is well known by birdwatchers and has been the subject of a plethora of research studies; however, its evolutionary history and systematics are still in debate. To understand some of the differentiation patterns recovered in a newly well-resolved multi-locus phylogeny, we undertook an ecological niche modeling approach, exploring climatic change during the Pleistocene. We generated an area of calibration (M) from genotyped populations across Eurasia, based on spatial buffers per location. Inside M, we reduced a dataset of 121,041 records (GBIF) through spatial filters to avoid autocorrelation (1,112 records). Additionally, as explanatory factors we used the PCA scores of 15 bioclimatic variables for present climate (WorldClim), and their projections in a Last Glacial Maximum scenario (i.e., 22,000 years into the past). Our model selection criteria were the omission rate, a test of significance (partial ROC), and AICc. From 3,451 explored models (kuenm package), we obtained a subset of 13 best models, all of which showed consistency in losing environmental continuity and suitable areas from the Last Glacial Maximum to the present. These climatic changes might explain the observed patterns of phylogenetic diversification and genetic diversity in Great Tits. Considering our results, we expect the present work to represent a step forward in the understanding of the evolutionary history of other widespread Eurasian bird species.}

\normaltalk{\capitalisewords{Comparing Land-Sparing and Land-Sharing Conservation Approaches in a Coffee-Growing Region with High Avian Biodiversity}}{\underline{Jonathon J Valente}, T B Ryder, Robert A Rice, Camila G'{o}mez, Nicholas J Bayly, T S Sillett, Peter P Marra}{Agricultural expansion is a leading cause of biodiversity loss in the tropics.  There are few places where balancing economic sustainability and biological integrity is more important than in the Sierra Nevada de Santa Marta region of Colombia, which is home to 628 bird species, but also more than 5,000 families dependent on cultivation of coffee.  Providing economic incentives to promote sustainable agriculture may be a viable conservation strategy here, but the benefits of contrasting agroecology models (e.g., land-sparing vs. land-sharing) have not been tested.  To evaluate how land management practices influence avian biodiversity in this landscape we conducted point counts and vegetation surveys at 214 locations on coffee farms and remnant forest tracts.  Results show that a land-sharing model is quite viable as coffee plantations with high concentrations of shade trees ($>$ 75 per ha) supported a greater density and diversity of breeding bird species than native forest.  On the other hand, comparable diversity could be protected under a land-sparing model by setting aside 7.37 ha (95\% CI = 3.12, 12.98) of native forest for every 20 ha of sun-grown coffee.  Given that 31 species found in forests were never detected on coffee farms, and that 10\% of those species are listed as vulnerable on the IUCN Red List, a land-sparing model may be critical for protecting species with strong fidelity to forested habitats.  Our findings suggest that a hybrid approach of encouraging both land-sparing and land-sharing is necessary to protect avian communities in this highly diverse landscape mosaic.}

\normaltalk{\capitalisewords{Can Renesting be an Effective Reproductive Strategy and How Does it Influence Conservation of a Threatened Shorebird Species?}}{\underline{Rose J Swift}, Michael J Anteau, Megan M Ring, Dustin L Toy, Hillary R Wagner, Mark H Sherfy}{Upon reproductive failure, many bird species may make a secondary attempt at nesting, hereafter renesting. Renesting may be an effective strategy to maximize current and lifetime reproductive success, but individuals face uncertainty in the probability of success because nests initiated later in the breeding season often have reduced nest or brood survival. We evaluated renesting propensity, renesting intervals, and renest reproductive success of Piping Plovers (Charadrius melodus) by following 1,922 nests and 1,785 unique breeding adults from 2014 – 2016 in North and South Dakota, USA.  The apparent renesting rate was 20\%. Renesting propensity declined if reproductive attempts failed during the brood-rearing stage, nests were depredated, reproductive failure occurred later in the breeding season, or individuals had previously renested that year. Additionally, plovers were less likely to renest on reservoirs compared to other habitats. Renesting intervals declined when individuals had not already renested, were after second-year adults without prior breeding experience, and moved short distances between nest attempts. Renesting intervals also decreased if the attempt failed later in the season. Lastly, overall reproductive success and daily nest survival were lower for renests than first nest attempts throughout the breeding season. Furthermore, renests on reservoirs had reduced apparent breeding success unless the predicted amount of habitat on reservoirs increased within the breeding season. Our results provide important demographic measures for this threatened species as well as implications for predation- and water-management strategies. Altogether, renesting may not be an effective reproductive strategy for Piping Plovers breeding in the Northern Great Plains.}

\normaltalk{\capitalisewords{Consequences of Nest Outcome on Investment into Later Broods by Gray Catbirds}}{\underline{Lucas J Redmond}, Jordan I Haq, Duncan Noble, Donovan Ricco}{Birds generally exhibit low levels of fecundity while heavily investing into individual offspring, examples of which include incubation, nest defense, and prolonged periods of provisioning dependent young. Allocation of resources into the yolk of eggs represents an important energetic investment, especially in smaller species in which a clutch of eggs could represent half or more of female body mass. Not surprisingly, variation, both within- and among-species, in clutch size and egg mass has received much attention. We studied seasonal variation in clutch size and egg mass of Gray Catbirds breeding on and around the Penn State Schuylkill campus in eastern Pennsylvania. We found that clutch size, but not egg mass, declined across the breeding season and was larger in first broods compared to later broods. Within later broods, females whose first nest successfully fledged young laid larger eggs than females whose first nest failed. Although the interval between laying first and later clutches for females that were initially successful was longer than that for females who initially failed, there was no difference in the number of days between the end of the first nesting attempt and the start of the second for the two groups of females. Nest failure would certainly reduce fitness, but our results suggest there may be carry-over effects of failure onto future breeding attempts that could result in further fitness costs. Future directions will include evaluation of possible physiological mechanisms explaining the reduced mass of eggs laid by females that initially failed.}

\normaltalk{\capitalisewords{Patterns of Nestling Growth Rates in Tidal Marsh Sparrows Mirror Colonization of Specialist Habitat}}{\underline{Katharine J Ruskin}, Valerie K Watson, Brian J Olsen}{Tidal marshes are geologically ephemeral habitat that change greatly in extent and location during glaciation cycles.  They are also home to high rates of endemism, likely the result of ecological specialization by colonizing species, the product of adaptation to this novel habitat.  Under this hypothesis of evolutionary succession, interspecific competitive ability and fitness within the novel habitat should be positively correlated with time since colonization.  We tested this prediction of evolutionary succession by comparing three tidal marsh sparrows that are sister species and are believed to have colonized tidal marshes from non-tidal habitats in successive interglacial periods: Seaside (Ammospiza maritima), Saltmarsh (A. caudacuta), and Nelson's (A. nelsoni) Sparrows.  We compared nestling growth rates among these species, as well as along natural gradients of flooding risk within species.  Seaside Sparrows, the earliest colonist, consistently grow the largest and fastest.  Nelson's Sparrows, the most recent colonist, generally grow to the smallest sizes and slowest, with one notable exception; wing growth of Nelson's Sparrow nestlings outpaces that of Saltmarsh Sparrows though Nelson's Sparrow adults have smaller wings.  This snapshot in evolutionary time suggests a trajectory in which wing growth trades off with other parts to maximize fitness.  This result is consistent with the observed patterns of nest survival in these species, and the notable behavior of Saltmarsh Sparrow nestlings to climb vegetation to escape flood tides.}

\normaltalk{\capitalisewords{Forest Area Required to Support Landbird Population Goals for the Mississippi Alluvial Valley}}{\underline{Daniel J Twedt}, Anne Mini}{Because most birds of conservation concern in the Mississippi Alluvial Valley are dependent on forested wetlands, bottomland hardwood forest is this region's habitat of greatest conservation concern. Past conservation planning for forest-dwelling birds in this region has focused on habitat objectives with presumptions regarding bird population goals being met via habitat provision. Herein, we estimated current populations of silvicolous birds based on detections during North American Breeding Bird Surveys (2007 – 2016) and used these population estimates along with historical change in their relative abundance to establish regional population goals. We used the variance associated with historical BBS trends to estimate the minimum forest area required to sustain a breeding population and combined this with predicted probability of occupancy to identify forest patches capable of sustaining the species. Using species-specific, empirical density estimates, as influenced by forest management, we estimated the proportion of each species' population objective provided within sustainable forest patches. Population sustaining habitat, under existing forest management, appears sufficient to support population objectives for 25 species. Target populations of 16 additional species could be supported with changes in forest management, or within non-sustaining forest, or non-forest habitat. However, population increases are needed to achieve stated population goals for another 13 species. Additional habitat needed to support these population increases varied widely among species, but we surmised that judicious forest restoration on 300,000 ha (~4\% of restorable land) should be sufficient to support population objectives within sustainable forest patches for silvicolous birds of conservation concern.}

\normaltalk{\capitalisewords{Do Piping Plovers in the Northern Great Plains Function as a Metapopulation \& Why Does it Matter for Conservation?}}{\underline{Michael J Anteau}, Rose J Swift, Mark H Sherfy, Dustin L Toy, Megan M Ring}{Piping Plovers (Charadrius melodus) in the Northern Great Plains (NGP) breed extensively on shorelines and islands of the Missouri River and wetlands of the Prairie Pothole Region. Given the birds' threatened status, the US Army Corps of Engineers (USACE) annually spends millions of dollars managing habitat for plovers on the Missouri River, under the direction of an adaptive management process. However, these adaptive-management models are sensitive to two key uncertainties:  rates of dispersal (on/off the river) and variation in annual survival. Previous research suggests that NGP plovers function within meta-population principles and it was posited that annual movement rates among sub-populations were about 2\%. During 2012, we initiated an 8-year mark-recapture study in North and South Dakota to address the USACE's uncertainties; namely, to evaluate variation in annual survival and estimate dispersal rates among breeding areas on the Upper Missouri River and the U.S. Prairie Pothole Region. Our preliminary results indicate that dispersal among breeding areas was much greater than anticipated (overall mean estimates: 47\% natal and 14\% adult dispersal). Natal and adult dispersal rates varied greatly among breeding areas and years, which is likely attributable to habitat dynamics. Annual survival estimates (apparent and true) varied by year and were dependent upon age class. Our findings will advance current adaptive management models and call to question the hypothesis that this population functions as a meta-population. Framing management decisions in the context of a single population may provide new and more flexible opportunities for conservation of plovers in this region.}

\normaltalk{\capitalisewords{Comparison of Abundance and Annual Return Rates of Wintering Piping Plovers on Natural and Man-Made Beaches in the Mississippi Sound}}{\underline{Abigail J Darrah}}{The Mississippi coastline contains a mixture of heavily-managed sand beaches on the mainland and natural barrier islands forming the southern border of the Mississippi Sound. These areas are used by threatened and endangered populations of Piping Plover, and in the face of ongoing and proposed restoration projects, it is important to understand the population trends and habitat needs of Piping Plovers in this region. Audubon staff and volunteers conducted repeated shorebird surveys at 25 sites in Mississippi from August 2014-February 2019 during designated fall, winter, and spring periods. I used N-mixture models to estimate plover abundance at each site and year and to relate abundance to habitat type. I used resights of banded birds to estimate annual recruitment and persistence rates of Piping Plovers among habitat types. Piping Plovers were most abundant on the barrier islands (mean N = 16.5, 9.7-27.0 Bayesian Credible Interval [BCI]), followed by nearshore islands and man-made beaches (mean N = 5.0, 2.7-9.3 BCI). Mean annual persistence was 0.6 (0.4-0.8 BCI) for all habitat types. Abundance during the 5-year period was stable, with a dip in 2015-2016 due to decreased recruitment. Undeveloped barrier islands are therefore critical for Piping Plovers wintering in Mississippi, though the equivalent persistence rates for mainland sites suggest that man-made beaches have the potential to provide quality habitat. Future research should assess whether disturbance, food resources, or some other factor drives the preference for barrier islands; this information could be used to improve management of mainland beaches for Piping Plover use.}

\normaltalk{\capitalisewords{Responses of State-Endangered Common Terns (Sterna Hirundo) in Ohio to the Use of Artificial Nesting Platforms}}{\underline{Laura J Kearns}, Brendan Shirkey, Mike Picciuto, Anna Crouser, Emily Scott}{Many agencies, including the Ohio Department of Natural Resources – Division of Wildlife (ODNR), have been using artificial nesting platforms (ANPs) to create nesting habitat for declining populations of common terns in the Great Lakes region.  The ODNR has supported breeding tern colonies at two locations with ANPs in Ohio since the early 1990s, but excessive predation has been a recurrent problem.  Shifting the locations of the colonies has been helpful in dealing with excessive predation by fox snakes and great horned owls, and promoting better reproductive output of the colonies.  The current Willow Point colony, that was shifted in 2007, has grown from 16 pairs in 2007 to an average of 89 pairs over the last 12 years.  The colony has fledged an average of 162 young since 2007, compared to a pre-2007 average of 35 young per year.  The current Cedar Point colony, that was shifted in 2015, has maintained an average of 73 pairs in the 4 years since the shift.  The colony has fledged an average of 127 young since 2015, compared to a pre-2015 average of 80 young.   While there is some risk to relocating tern colonies with ANPs, and this technique is limited by the availability of suitable locations within a suitable range, these data indicate that relocation can be done successfully with established colonies that use ANPs.}

\normaltalk{\capitalisewords{Grassland Bird Response to Spring Cover Crops in an Agricultural Landscape}}{\underline{Alixandra J Godar}, Adela Annis, David Haukos, Jeff Prendergast}{As people continue to manipulate landscapes, we are faced with the daunting challenge of balancing resource needs of humans and wildlife across altered landscapes. In western Kansas, native grassland bird populations declined after the conversion of natural prairies into expansive crop fields. A potential, agriculture-friendly source of grassland bird habitat is planting spring cover crops to convert barren, chemical fallow fields into alternative sources of food and cover during the breeding season. We selected ring-necked pheasant (Phasianus colchicus) as our focal species due to declining populations in Kansas, popularity with landowners, and timing of breeding season. Cover crops were planted mid-March and terminated mid-June. Seed mixes included Chick Magnet, a broad-leafed warm-season forb mix; GreenSpring, a mix of oats and cool-season peas; and a Custom mix containing 10 different species. Vegetation structure and available food resources differed among mixes, annual environmental conditions, and across the growing season. Using a resource selection function, we compared points used by female pheasants fitted with VHF transmitters to available points (n = 908 paired vegetation surveys). Pheasants selected locations providing cover over specific vegetation types (AICc weight = 1.0) Cover crop use was influenced more by relative placement within the landscape and land cover composition of the surrounding area rather than cover crop type (n = 45 fields). Pheasants did not use chemical fallow fields, instead relying on short-term crop growth and permanent cover. Interspersion of cover crops with permanent cover may provide additional resources for avian species but will not replace natural grasslands.}

\normaltalk{\capitalisewords{Ecological Drivers of Backyard Bird Feeding and Consequences for Parasite Transmission}}{\underline{Richard J Hall}}{Backyard bird-feeding is a widespread and globally popular form of food subsidy for birds, but the ensuing aggregation of birds around food sources can increase their exposure to pathogens. Theory and manipulative experiments that investigate the consequences of food subsidies for parasite transmission typically assume that food is provided at a constant rate. However, the amount and frequency of food provided by backyard bird feeders may be influenced by observations of the natural system, including changes in abundance of the focal fed species, and visible signs of disease. Here I develop a simple mathematical model for coupling the abundance of birds and their contact-transmitted parasites with the intensity of food provisioning, contrasting scenarios where the rate of food provisioning is independent of, or depends on, components of the natural system. Unlike constant food provisioning, which usually results in a small outbreak followed by a smooth approach to an equilibrium infection prevalence, coupling food provisioning rates to bird abundance and infection can result in more complex emergent dynamics, including larger outbreaks and sustained host-pathogen cycles. Accounting for this coupling of human activity to ecological dynamics could inform development of best practice guidelines for intentional feeding of birds and other wildlife that minimize its unintended negative consequences.}

\normaltalk{\capitalisewords{Testing for Adaptive Introgression Across Elevation in a Rapid Radiation, the South American Siskins (Spinus)}}{\underline{Elizabeth J Beckman}, Christopher C Witt}{Interspecific gene flow may be an important source of genetic variation during environmental adaptation. However, clear examples of adaptive introgression are rare across vertebrates. Here, we investigated the significance of interspecific introgression to adaptation to hypoxia, a chronic, abiotic factor at high elevation, in a rapid radiation of finches, the South American siskins (Spinus). The Spinus radiation is well-suited to this question since (1) Spinus high elevation colonization was complex and included at least two independent arrivals, and (2) we previously showed extensive historic introgression among sympatric Andean species. To elucidate the role of adaptive introgression in Spinus, we tested the association between elevational range and the alleles at 36 high-elevation candidate genes, including all seven coding genes of the oxygen-transport protein hemoglobin. If introgression shaped the distribution of alleles at high-elevation candidate genes across Spinus, we expected high elevation species to share alleles more often than predicted based on 500 random exons distributed across the genome. Alternatively, variation at candidate loci may reflect divergence history and incomplete lineage sorting. We found evidence for adaptive introgression in at least two critical hypoxia-associated genes, hemoglobin alphaA and alphaD, among sympatric Andean Spinus, and modest support for the preferential introgression of alleles at hypoxia candidate genes across the genome. Our study supports an outsized role for hemoglobin in understanding avian adaptation to high elevation and highlights introgression as an important source of adaptive variation.}

\normaltalk{\capitalisewords{The Long and the Short of It: Linking Genome-Wide Signatures of Selection Across Evolutionary Timescales in Birds}}{\underline{Allison J Shultz}, Brian Arnold, Timothy B Sackton}{A challenge of evolutionary biology is linking short-term evolutionary processes to long-term patterns of biodiversity. Many studies focus on a single evolutionary timescale,  rather than studying the same question across timescales, inhibiting the ability to link evolution through time. In this study, we compare population-level genome-wide signatures of positive selection to a recently published set of genes under positive selection across the bird tree of life. We obtained genome-wide signatures of selection across populations of birds from over 30 species, first, by building and executing a standardized pipeline to process publicly-available whole-genome resequencing data using a version of the GATK pipeline optimized for non-model organisms. Then, we used a variation of the McDonald-Kreitman test to detect signatures of selection in coding regions of each species. Preliminary results in several species of songbirds suggest that there is a significantly greater overlap than expected in genes under selection at both short (across populations) and long (across Aves) evolutionary timescales. Genes under positive selection in populations of multiple species have immune, recombination, or reproductive functions, strengthening the hypothesis that a host-pathogen arms race may be one of the most consistent selective pressures across different bird species.}

\normaltalk{\capitalisewords{Loggers in Artificial Eggs Reveal Scarlet Macaw Incubation in Natural and Artificial Nests}}{\underline{Constance J Woodman}}{The use of loggers inside artificial eggs has offered new knowledge about how birds incubate and attend to their eggs. Newer technology offers true 3D position recording to more fully describe behaviors. This detailed data can inform artificial incubation practices for intensive species management programs. In the Peruvian Amazon, at the Tambopata Macaw Project, loggers were added to clutches in 21 nests after the first egg was laid. There were three nest types, natural, artificial wood, and artificial PVC boxes. The majority of hens had known nesting histories. Success of wild incubation was compared to two captive breeding incubator regimes. There are species-specific challenges to deploying these technologies.  Scarlet macaws are large birds with long incubation time (battery issues), nests which are high in trees and physically far apart (access issues). They are very powerful, able to significantly damage equipment (engineering issues) and can discriminate between artificial and real eggs (methodology issues). Macaws offer a case study to show how adapting equipment to a challenging species is possible.}

\normaltalk{\capitalisewords{Diversity of Grassland Bird Communities as a Function of Grazer Species on the Northern Great Plains}}{\underline{Andy J Boyce}, Hila Shamoon, William Mcshea}{Large ungulates can strongly influence grassland bird communities through their impact on structure and composition of grassland vegetation. Understanding the biodiversity effects of grazer type, as well as removal of grazers from grassland landscapes has important consequences for landscape scale conservation in globally-threatened temperate grassland ecosystems. North American grassland bird communities co-evolved with Plains Bison (Bison bison bison), and there is evidence that bison grazing behavior differs significantly from domestic cattle and can increase vegetative heterogeneity in grassland systems. If true, we predict increased diversity of grassland birds in areas grazed by bison, particularly in large pastures. To test this hypothesis, we sampled grassland bird communities on grasslands on and surrounding the American Prairie Reserve in NE Montana, on lands grazed by cattle, bison, or no large bovids. We sampled 49 80-hectare plots, within larger fenced pastures using 10-minute point counts. We conducted 6 counts within each 80-hectare cell to assess both overall diversity and heterogeneity in bird communities within a cell. Based on preliminary analysis of species richness and functional diversity of grassland-obligate songbirds, we found that species richness and functional diversity were similar between areas grazed by Bison and cattle, and lower in areas without bovine grazers. Analyses of this dataset are ongoing and will include data from a second field season.}

\normaltalk{\capitalisewords{Patterns of Change Within the Avian Community of Acadia National Park: Long-Term Changes and a Case Study in Human Development}}{\underline{Katharine J Ruskin}, Alyson M East, Marie I Ring}{Protected lands such as national parks are intended to conserve natural resources such as biodiversity.  Despite their protected status, however, wildlife in national parks is changing due to a number of drivers, including climate change, addition of invasive species, and land use change.  Quantifying rates of change of biodiversity on protected lands is critical for conserving this resource.  We examined whether the avian community of Acadia National Park (ANP) has changed in the past fifty years.  Using Christmas bird count data collected from 1960-2017, we found that the avian community of ANP is changing; each year, the avian community becomes more dissimilar to the baseline community of 1960.  In addition, we used a case study to quantify the effects of a single driver of environmental change in this system.  In 2015, ANP opened the Schoodic Woods Campground which hosts 97 campsites, accompanied by 7 miles of new gravel bicycle paths, 5 miles of hiking trails, a visitor center, and parking for 100 cars.  During the breeding season of 2017, we conducted avian point count surveys on transects beginning from the edge of developed areas and radiating into the forest.  Across 184 point count surveys, we observed that species richness at each survey location was best predicted by distance from the campground.  Species richness was highest in and near the campground, and declined moving away from it.  The avian community observed at survey locations near the campground clustered together and included several species classified as synanthropic.  Taken together, our results suggest that ANP is experiencing biotic homogenization and illustrate that human development is contributing to these changes in the avian community.}

\normaltalk{\capitalisewords{Habitat Use of Hawaiian Short-Eared Owls (Asio Flammeus Sandwichensis) in Hawai'i}}{\underline{Chad J Wilhite}, Javier Cotin, Laura R Luther, Melissa R Price}{Pueo, or Hawaiian Short-eared Owls (Asio flammeus sandwichensis), play an important role in the cultural and ecological landscape of Hawai'i as the only widespread native apex terrestrial predator. Globally Short-eared Owl populations are experiencing declines, likely due to a combination of factors including predation, food availability, disease, and anthropogenic mortality. Despite population declines very little is known about the biology of Pueo or Short-eared Owls in general. In temperate and continental systems Short-eared Owls are typically considered nomadic grassland specialists. In this study we deployed VHF transmitters to track Pueo on the island of O'ahu and identified habitat use and movement patterns in Hawai'i. Pueo tend to exhibit higher site fidelity with less nomadic movements and utilize a broader range of habitat types than continental and temperate populations. Our increased understanding of Pueo biology will also help inform land managers how to limit anthropogenic activities in high quality habitat and during peak breeding seasons, potentially resulting in substantial reductions of human-caused mortality.}

\normaltalk{\capitalisewords{Painting the Bunting: The Molecular and Structural Basis for Coloration of Painted Bunting (Passerina Ciris)}}{Nicholas M Justyn, \underline{Ryan J Weaver}}{Conspicuous feather coloration is fairly widespread in birds and the most common colors are attributed to various combinations of melanins, carotenoids, and spatial arrangements of the feather microstructures. Male Painted Bunting (Passerina ciris) are perhaps the most colorful songbird in North America, displaying vibrant red, orange, pink, yellow, green, and blue feather coloration. Surprisingly, the pigments and or feather structures that give rise to the multitude of colors of Painted Buntings has not been determined. Here, we use digital photographs, spectrophotometry, high-performance liquid chromatography, and scanning electron microscopy to elucidate the mechanisms of feather color production in six body (color) regions: Crown (blue), chest (red), belly (orange), rump (pink), back (yellow), and wing (green). We found that carotenoids and structural components are responsible for the observed coloration of each region. Red ketolated carotenoids were found in blue, red, orange, and pink feathers. Blue coloration comes from light interacting with specifically arranged feather microstructures, but Painted Bunting blue feathers also contained red ketocarotenoids, albeit at low concentration. The shift from red to orange to pink was dependent on the concentration of red ketocarotenoids deposited, but the orange to pink shift occurred by tuning from the underlying blue structural coloration. Yellow and green feathers contained the same canary xanthophyll carotenoids, but yellow feathers had higher concentrations. To our knowledge, this report is the first to elucidate the molecular and structural basis for the diverse array of conspicuous coloration in male Painted Buntings.}

\normaltalk{\capitalisewords{Migratory Bird Twilight Ascent and Descent Rates along the Southwestern Shoreline of Lake Erie}}{\underline{Michael J Wellik}, Eileen M Kirsch, Mark Sheildcastle}{When nightly migration starts or ends, birds transition vertically through the airspace where there is the potential for collision with anthropogenic structures. Factors that increase collision risk include reduced visibility because of bad weather, size or orientation of the object (eg., powerlines), time spent in the collision zone and proximity to large concentrations of birds or important habitat areas. The forested wetlands along the southwestern shore of Lake Erie support important concentrations of nocturnal migrant songbirds during spring and fall migration. Inland woodlots are also important to migrant birds in this largely agricultural landscape. We estimated rates of ascent and descent of (presumed) birds as low as 40m above ground level during spring and fall migration using an X-band marine radar with a vertically rotating open array antenna. Using the program radR, nightly track profiles around dusk and dawn were developed and used to calculate the average rates of ascent and descent, along with confidence intervals. Using these rates, we modeled the horizontal distance to achieve altitudes above a 1.5MW wind turbine. As an exercise to demonstrate locations where turbines might more safely be placed in this landscape we use GIS to place buffers around wooded areas based on the modeled horizontal distances. These buffers revealed the extent of landscape from wooded areas which might be of collision risk for nocturnal migrant birds stopping over in woodlots and forests.}

\normaltalk{\capitalisewords{Migratory Behavior of the Western Burrowing Owl: Connecting Canada, United States, and Mexico}}{\underline{Courtney J Conway}, Troy I Wellicome, David H Johnson, Julie L Conley, Carl G Lundblad, David R Bruinsma, Ryan J Fisher}{Identifying causes of declines in long-distance migrants is often hampered because we know little about migratory routes, destinations, and annual dispersal. Western burrowing owls (Athene cunicularia hypugaea) are a species of national conservation concern in the U.S. and are federally endangered in Canada. One obvious knowledge gap for burrowing owls is our limited information regarding their migratory patterns. To address this gap, we deployed light-level geolocators on 367 adult owls, breeding in 8 states and 3 provinces in the U.S. and Canada. We obtained winter site locations from 51 of these owls recaptured in subsequent years. From 2013-2017, we also deployed solar-powered satellite transmitters (PTTs) on 78 adult owls, in 9 states and 3 provinces, from which we obtained data for 34 complete southward migrations to winter sites. All owls that bred in British Columbia, Washington, and Oregon wintered in the U.S. and migratory behavior differed between sexes: most females wintered to California and most males wintered in Washington. In contrast, most burrowing owls that bred in the interior states and provinces wintered in Mexico. Owls varied greatly in the time spent migrating, but we detected no difference between time spent during spring and fall migration: 6-57 days on fall southbound migration (mean 22.9 days) and 6-51 days on spring northbound migration (mean 21.6 days). Most owls that bred on the Great Plains remained east of the Rocky Mountains and funneled through western Texas when migrating southward. Over 90\% of the owls that nested in (and migrated from) Colorado, Wyoming, South Dakota, Nebraska, and Montana took a multi-day break from migration in northwestern Texas (with most of these stopovers near Lubbock, Texas). Owls had stronger site fidelity to their wintering locations than to their breeding sites. Mexico and California are important wintering areas for Western Burrowing Owls and our results identify important stop-over and wintering locations where conservation efforts might be focused.}

\normaltalk{\capitalisewords{Quantifying Passage Population Size and Migration Phenology of Endangered Rufa Red Knots Staging in Southwestern James Bay, Ontario, Canada}}{\underline{Amelia J MacDonald}, Paul A Smith, Christian A Friis, James E Lyons, Erica Nol}{The endangered rufa Red Knot (Calidris canutus rufa) is an extreme long-distance migrant shorebird that breeds in the Canadian Arctic and migrates to non-breeding grounds as distant as Tierra del Fuego. To aid population recovery for this species, we need full annual cycle analyses that examine population trends and dynamics across their range. The southwestern coast of James Bay is one of a few key staging sites that Red Knots use during southbound migration, but we lack a quantitative estimate of how many birds use this remote area. We surveyed Red Knots in James Bay from mid-July to mid-September 2018 and recorded over 1500 resights of birds with uniquely coded leg flags. We generated a passage population estimate for the southwestern James Bay coast using a mark-resight based Bayesian analysis of an integrated population model. We estimated that nearly 13,500 Red Knots used our James Bay study site during migration in 2018. Birds stayed an average of 12 days and peak numbers occurred in early August. We used a novel analysis to determine that juveniles comprised 24\% of the population; the first estimate of juvenile proportion for this staging site immediately south of the breeding grounds. While much of southwestern James Bay is currently intact wilderness, most of the region is not formally protected. Our results indicate that the coast may support up to 23\% of the total rufa population. This suggests that James Bay provides important shorebird habitat and should be included in conservation planning for this endangered shorebird.}

\normaltalk{\capitalisewords{What Causes Vagrancy in some New World  Warblers along the Pacific Coast?}}{\underline{C J Ralph}, Jared D Wolfe}{Birds found outside their typical range, or vagrants, have fascinated naturalists  for decades. Despite broad interest in vagrancy, few attempts have been made  to statistically examine the explanatory variables potentially responsible for  the phenomenon. In this study, we used multiple linear regression to model  the occurrence of 28 rare warbler species (family Parulidae) in autumn in northern  California and southern Oregon as a function of migration distance, continental  population size, distance, and bearing to both closest breeding population and  breeding population center. In addition to our predictive model, we used capture  data from the California coast to 300 km inland to examine relationships between the  presence of vagrant warblers, regional warbler species richness and age class  distribution. Our study yielded three important results: (1) vagrancy is strongly  correlated with larger North American population size and secondarily by longer  migration distance; (2) vagrants are more common at some coastal sites; and  (3) where young birds are over-represented, vagrants tend to occur—such as  on the coast and at far inland sites. Of the many explanations of rare and vagrant  individuals, we feel that the most likely is that these birds represent the ends of  the distributions of a normal curve of migration direction, bringing some  few migrants to locations out of their normal migratory range as vagrants.  We also examine the underrepresented species that, according to our model,  are overdue for being recorded in our study area.}

\normaltalk{\capitalisewords{What can Weather Surveillance Radar Data Tell Us About Stopover Duration of Migrating Land Birds?}}{\underline{Jeffrey J Buler}, Theodore J Zenzal, Jr, Frank R Moore, Liliana N Calderόn, Wylie Barrow, Jr, Barry Wilson}{Coupling weather surveillance radar measures of migrating birds leaving stopover sites across nights (i.e., index of number of birds) with repeated ground surveys of daily migrant density (i.e., bird use days) of unmarked individuals offers an alternative approach to estimating seasonal mean stopover duration with less effort and expense than conventional mark-recapture methods. We tested the temporal validity of this approach by comparing radar-derived stopover duration index measures with observed minimum stopover duration measures based on capture-recapture data of birds at a single passive mist-netting migration monitoring site among 14 autumn seasons. We found a positive correlation between derived and observed seasonal mean stopover duration when there was a sample size of at least six days of radar data per season (r = 0.78, n = 10 years). The correlation strengthened as sample size increased. We also tested the spatial validity of radar-derived stopover duration among 6 mist-netting sites spread across Louisiana and Alabama during spring and autumn 2018. We found a positive correlation between derived and observed seasonal mean stopover duration across sites during spring, but not during autumn. We will discuss how seasonal differences in the behavior and detectability of migrant birds during ground surveys may confound the spatial (across site) validity of radar-derived stopover duration estimates in autumn. Our findings also raise general concerns about comparisons of absolute migrant bird density from ground surveys among seasons, even when using distance-sampling techniques.}

\normaltalk{\capitalisewords{Pacific Arctic Seabird Communities: a Decade of Change Viewed Through the Lens of the Distributed Biological Observatory's at-Sea Surveys}}{\underline{Kathy J Kuletz}, Daniel Cushing, Erik Osnas, Elizabeth Labunski, Adrian Gall}{In the Pacific Arctic, international science programs collaborate to study offshore ecosystems by sampling the same eight Distributed Biological Observatory (DBO) sites, which cover three marine regions: the northern Bering, Chukchi, and Beaufort seas. We used at-sea visual surveys totaling 144,920 km of transects from 2007–2018 to compare seabird species richness, community composition, and relative abundance in these regions over time. Seabird communities within DBO sites were representative of their respective region, although each site captured different taxa. Species richness (total = 59 species), as with abundance, was highest in the northern Bering Sea and peaked near Bering Strait, with a steep drop in the Beaufort Sea and north of 72°N. We identified six seabird community clusters, which were roughly associated with current systems (e.g., Alaska Coastal Current, Anadyr Current) or oceanographic domains, and influenced by locations of breeding colonies. For most taxa, variance in abundance was greater among DBO sites than among years, suggesting that persistent habitat (site) features were more influential than annual fluctuations in prey or environmental conditions. However, in 2017 and 2018, with record-breaking warm seas and changes in prey, we found low abundance of key locally-breeding seabird species. Those taxa for which abundance varied more among years than among sites were shearwaters, murrelets, and phalaropes – all late summer migrants to the area which presumably have more flexibility in timing and distribution. We suggest that with predicted increases in ocean temperatures, there will be greater variability in seabird abundance and distribution, particularly for migrant species.}

\normaltalk{\capitalisewords{Testing the Causes of Speciation with Gene Flow in a Neotropical Avian Species Pair (Pachyramphus Aglaiae)}}{\underline{Lukas J Musher}, Peter Galante, Mary Blair, Jerry Huntley, Gregory Thom}{Isolation with migration is a frequent mode of speciation. For this to occur, it is typically argued that disruptive selection caused by pronounced environmental gradients must act in order to counter the homogenizing effects of gene flow, a process termed, ecological speciation. However, many species have diverged in the absence of environmental clines and instead differentiate across biogeographic barriers. As an alternative to ecological speciation, Pleistocene glacial cycles may have promoted speciation with gene flow by facilitating both increased isolation during times of habitat fragmentation and interpopulation genetic exchange during times of habitat expansion. In the Neotropics, the Isthmus of Panama has acted both as a barrier and corridor for species during the Pleistocene. Using demographic model selection under both approximate Bayesian computation and composite likelihood frameworks on thousands of genome-wide markers, we test the drivers of speciation with gene flow for two sister-species that have diverged across the Isthmus of Panama, Pachyramphus aglaiae and P. homochrous. We find a scenario of isolation with migration has shaped genetic structure in these species. Additionally, using niche modeling, we show that the two species have not significantly diverged in climatic niche, and instead suitable habitat overlap at the Isthmus of Panama for the two species has episodically increased and decreased over time. By making explicit predictions about competing hypotheses, we show that our findings are inconsistent with ecological speciation driven by environmental gradients. Instead, our results highlight the importance of Pleistocene glacial cycles on both differentiation and genetic exchange in Neotropical organisms.}

\normaltalk{\capitalisewords{When the Freezer Breaks: will Climate Warming Impact the Persistence of Resident Birds in Alaska?}}{\underline{Emily J Williams}, Laura M Phillips, Tucker Grigsby, Ryan Norris, Roxan Chicalo, John Marzluff, Kaeli Swift}{Food caching, or the storage of food for later use, is a behavior common to many taxa that has important implications for survival and demography. Many animals have adapted this behavior to survive harsh winters where food is scarce.  Unlike most hoarding species, Canada Jays (Perisoreus canadensis) are unusual in that they store perishable food for long periods of time during the winter. Under a changing climate, growing evidence suggests that perishable food caches are at risk of rotting during increasing warming regimes. This problem is exacerbated at higher latitudes, where Canada Jay populations rely on cached food for prolonged winter seasons. Food spoilage and reduced availability of cached food may negatively impact Canada Jay survival and fitness, as individuals in poorer condition may invest less in reproduction. To examine whether Canada Jays at higher latitudes are more susceptible to a changing climate, we initiated studies on a population of Canada Jays in Denali National Park and Preserve, Alaska in 2017-2018. These studies investigated foraging, caching, and incubation behavior as it relates to reproductive success. We regularly monitored 170 color-banded individuals belonging to 29 family groups and monitored the fate of 53 nests. Preliminary results indicate high adult survival (~95\% raw survival) but low nest survival (~55\% apparent success). Nest attentiveness was lower than expected (~85\% compared to 98\% reported in literature).  Ongoing studies quantifying reproductive performance and juvenile and adult survival will help determine the demographic consequences of a warming climate on Canada Jay persistence in Alaska's boreal forests.}

\normaltalk{\capitalisewords{Population Ecology of Spruce Grouse in Commercially-Managed Forests}}{\underline{Erik J Blomberg}, Joel Tebbenkamp, Stephen Dunham, Daniel Harrison}{Spruce Grouse (Falcipennis canadensis) are obligate residents of conifer-dominated forests throughout the Northern U.S., and Canada. The effect of timber harvest on Spruce Grouse populations is unclear.  We investigated Spruce Grouse population ecology and habitat relationships in commercially-managed forests of northern Maine, USA, from 2012 to 2017.  In this system, an outbreak of Spruce Budworm (Choristoneura fumiferana) in the 1970s resulted in widespread clearcutting, with subsequent post-harvest treatment (herbicide and thinning) of many stands to promote conifer growth. We used radio-telemetry to monitor Spruce Grouse habitat use, survival, and reproductive output, and evaluated the relationship between Spruce Grouse demographics and the legacy of forest management within the system.  Based on data from 150 radio-marked birds, we found that response of Spruce Grouse vital rates to forest treatments was variable. Brood survival was greatest when broods inhabited residual forest stands lacking recent (post-1981) harvest history, however following independence from broods juvenile Spruce Grouse experienced lower survival during their first fall and winter in this same stand type. Daily nest survival was lowest in clearcut stands that received post-harvest treatments to suppress woody growth, while adult spruce grouse exhibited the highest monthly survival when they made frequent use of regenerating clearcuts without post-harvest treatment. We used vital rate estimates to construct a stage-based population model, which predicts future decline of spruce grouse in this system.  These declines will likely be exacerbated by future loss of conifer-dominated forest resulting from anticipated land-use and climate change.}

\normaltalk{\capitalisewords{Population Demographics, Breeding Ecology, and Responses to Grazing of Montana Sagebrush Steppe Songbirds}}{\underline{Victoria J Dreitz}, Kayla Ruth, Lorelle Berkeley}{Sagebrush steppe is one of the most threatened ecosystems in North America. Avian adult density estimates are often used to assess conservation actions given the relative ease in collecting data to inform these estimates. However, information on how conservation actions influence life histories such as nest density and nest success are lacking, despite the fact that life histories inform abundance. We investigated songbird adult densities, nest densities, and nest success over multiple breeding seasons in central Montana. Our goal is to understand the relationship between adult abundance, nest density, and nest success, as well as how land management practices, in the form of grazing, influence those patterns. Two grazing systems were compared in our study: a system using a combination of rest and deferment (hereafter rest-rotation) and traditional grazing. For the purposes of our study, we define rest-rotation grazing as changing the timing of grazing in pastures each year, with some pastures alternately rested every few years. Traditional grazing is defined as grazing a pasture at the same annual season each year or all season. Recently, rest-rotation systems have been used as a conservation management tool by the Natural Resource Conservation Service's (NRCS) Sage Grouse Initiative (SGI). Their goal is to encourage private landowners to graze their livestock more sustainably in order to maintain or improve habitat for greater sage-grouse (Centrocercus urophasianus), as well as to improve rangeland productivity. We explore the effects of rest-rotation compared to traditional grazing on songbird population breeding demographics.}

\normaltalk{\capitalisewords{Using Museum Specimens to Investigate Fitness Variation Across Avian Hybrid Zones}}{\underline{Paul J Dougherty}, Matthew D Carling, Sydnie L Fossberg}{When prezygotic isolating mechanisms between discrete populations are sufficiently weak to allow for hybridization, the two parental populations will not merge if hybrid fitness is low. Although many hybrid zones are so narrow as to suggest low hybrid fitness, definitive evidence of selection against admixed offspring is rare. Museum collections present a cost-effective opportunity to investigate variation in fitness across avian hybrid zones. We collected a series of measurements known to correlate with developmental stress and body condition on specimens sampled from multiple North American avian hybrid systems. While these measurements can in no way describe an individual's lifetime reproductive output, they do allow for inferences regarding fitness by serving as proxies for overall physical condition. As hybrids may suffer genetic incompatibilities that elevate metabolic rate or otherwise impose nutritional stress, we predicted that admixed individuals would display consistent signatures of poor condition in all of the hybrid systems that we examined. Our results support variation in the condition of hybrids relative to non-admixed individuals among the systems that we studied. We attribute this pattern to the assumption that the severity of genetic incompatibilities in hybrid offspring depends on the genetic distance between parental taxa. By demonstrating differences in fitness across avian hybrid zones, our study supports the need to identify precise mechanisms of postzygotic reproductive isolation between hybridizing taxa.}

\normaltalk{\capitalisewords{Sex and Species Divergence in Vocalizations Between Role-Reversed Shorebirds, Jacana Spinosa and Jacana Jacana}}{\underline{Evan J Buck}, Sara E Lipshutz, Elizabeth P Derryberry}{Species-specific vocalizations can act as a reproductive isolating mechanism between closely related populations. We analyzed vocal divergence between two hybridizing species of sex-role reversed polyandrous shorebirds, the Northern Jacana (Jacana spinosa) and Wattled Jacana (Jacana jacana), as well as hybrids. We found that J. spinosa calls have higher peak frequency and fundamental frequency than J. jacana calls. Additionally, a female hybrid exhibited intermediate characteristics between the two species, while a male hybrid displayed characteristics more similar to J. jacana. We also compared calls between males and females for both species, as jacana body size is sexually dimorphic.  Males tend to emit a greater number of notes within a calling bout, exhibit shorter note lengths than females, and have a higher peak frequency in their calls. We found no differences among inter-note intervals when comparing species or sexes. These results indicate that vocal divergence could act as a behavioral barrier to limit hybridization between the species, and that vocalizations may function differently between male and female jacanas. Future studies using playback experiments could test these hypotheses.}

\normaltalk{\capitalisewords{Physiological and Genetic Mechanisms Underlying Organismal Response to Artificial Light at Night}}{\underline{Valentina J Alaasam}, Yong Zang, Ye Niu, Brad Ferguson, Simon Pieraut, Brett Seymoure, Stefania Casagrande, Jenny Q Ouyang}{Light pollution is quickly becoming a pervasive, global concern. Since the invention and proliferation of light-emitting diodes (LED), consumers can select from a range of color temperatures of light with varying spectral compositions, however the biological impacts of these different spectra on organisms and the mechanistic pathways underlying physiological and behavioral effects remain unclear. We tested if nighttime illumination of LEDs, at two commercially available color temperatures (3000K and 5000K) and at dim, ecologically relevant illumination levels (0.3 lx) affected body condition, food intake, nighttime activity and glucocorticoid levels in zebra finches (Taeniopygia guttata). We found that individuals exposed to 5000K light had higher rates of nighttime activity and increased glucocorticoid levels from pre-treatment levels compared to 3000K and control groups, with no changes in body condition or food intake. We then measured biomarkers of long-term pathology (oxidative stress and mitogen-activated protein kinases) under 5000K light (1.5 lx). Finally, we measured the circadian expression of plasma melatonin and five major pace-making genes involved in regulating endogenous circadian rhythms in order to unravel the mechanistic pathways explaining downstream behavioral and physiological change. This study links molecular, physiological, and behavioral impacts of light pollution and adds to the growing evidence that the spectrum and intensity of night light is important for understanding organismal responses.}

\normaltalk{\capitalisewords{Stopover Ecology of Semipalmated Sandpipers in Maritime Canada}}{\underline{Diana J Hamilton}, Sydney Bliss, Erica Geldart, Rebeca Linhart, Sarah G Neima, Julie Paquet, Cheri L Gratto-Trevor}{Semipalmated Sandpipers (Calidris pusilla) rely heavily on Maritime Canada, especially the upper Bay of Fundy, as a post-breeding migration stopover site. However, abundance of birds using the region has declined in recent decades, leading to questions about site quality, stopover activities, and the source of this decline. We have examined timing of migration as well as movements, diet, habitat use, and stopover duration during their time in the region. The median migratory passage is now nearly a week later than it was in the 1980s, and this appears to be linked to changing climate in the Arctic. With the help of radiotracking technology, we have determined that SESA remain longer in the region than previously understood, and that they exhibit very high within and among year regional fidelity. Their diets are also much more variable than previously thought, reflecting local availability of prey and opportunism rather than selectivity. We have identified variability in migratory strategy that can affect how long individuals remain in particular areas, and their overall stopover duration. Birds tagged outside the Bay of Fundy sometimes move into the Bay to complete their stopover, but the reverse is not observed. This multi-stop approach leads to longer overall stays in the region, and may be related to foraging and weight gain. We conclude that SESA use a wider range of strategies than previously understood, and this, together with a potentially expanded stopover period, should be considered in conservation planning. The Bay of Fundy remains critical to this species.}

\normaltalk{\capitalisewords{Multiple Benefits from a Migratory Bird Species: Where They Originate and How it Matters to Management}}{\underline{Darius J Semmens}, Kenneth Bagstad, Jay Diffendorfer, Brady Mattsson, James Dubovsky, Wayne Thogmartin, Ruscena Wiederholt, John Loomis, Joanna Bieri, Christine Sample, Joshua Goldstein, Laura L'{o}pez-Hoffman}{The conservation of migratory species poses major challenges; it requires cooperation and coordination across borders and can involve expensive changes in management and disputes over accountability and payment responsibilities. An ecological and socioeconomic accounting of the multiple benefits they provide to people and where those benefits come from can help address this challenge. The movements of migratory species are supported by habitat and distribute benefits to people throughout their range. By quantifying the relative ecological importance of habitat supporting migratory species population dynamics and the economic value of ecosystem services they provide to people over the course of an annual cycle, it is possible to delineate flows of value between different parts of their range. Using a case study of northern pintail ducks, we demonstrate how values can be estimated geographically for multiple benefits and how these values can be combined with ecological information to delineate flows. Pintails support over \$101 million USD annually in recreational hunting and viewing and subsistence hunting in the U.S. and Canada. Pintail breeding regions provide nearly \$30 million in subsidies to wintering regions, with the ''Prairie Pothole'' region supplying over \$24 million in annual benefits to other regions. We describe how this approach can be used to assign value to habitat and how this value can be combined across species and/or with additional economic benefit and cost data. Data requirements, transferability, opportunities for simplification, and management implications are also discussed.}

\normaltalk{\capitalisewords{Extra-Pair Copulations in Passerines: Why Females Matter}}{\underline{Bridget J Stutchbury}}{The question of what role males versus females play in extra-pair copulations (EPCs) has important implications for understanding how the frequency of extra-pair paternity should between species.  Some of the first papers on extra-pair paternity focused on male-male competition and mate guarding and viewed the female as being a victim. Perhaps females bred synchronously to reduce their suffering from EPCs, since males who must mate guard closely cannot also seek EPCs.  Notably, Susan Smith's 1988 paper on Black-capped Chickadees documented female trespass onto adjacent territories to seek out EPCs from socially high-ranking males. Over the years, researchers showed clear evidence that female passerines could resist unwanted copulations, made EPC forays, and advertised their fertility to prospective EPC males. The Breeding Synchrony hypothesis focused on the role of female choice in EPCs and suggested that synchronous breeding, imposed by ecological factors, is associated with higher rates of extra-pair paternity because females can better compare male quality. Where do we stand 20+ years after this idea was proposed?}

\normaltalk{\capitalisewords{The Impacts of Temporary Emigration on Colonization and Extinction Estimates in Dynamic Occupancy Models}}{\underline{Jonathon J Valente}, Rebecca A Hutchinson, Vitek Jirinec, Matthias Leu, Matthew G Betts}{Occupancy models are commonly used to analyze point count data because they account for imperfect detection and improve distribution estimates.  Yet these models assume sites are closed to occupancy changes between repeated visits, which may be invalid over time scales (~1 month) typically used in point count sampling schemes.  Recent studies have purportedly demonstrated such violations by comparing dynamic occupancy models with very short primary periods (~10 minutes) to static occupancy models that assume closure over all visits.  However, the degree to which temporary emigration (TE; reversible changes in sampling availability) impacts these findings remains untested.  We investigated the impact of TE on the colonization and extinction parameter estimates in dynamic occupancy models using simulated data and empirical point count datasets (19 species) with varying sampling intervals.  Simulations indicated that when secondary periods were open to TE, presence of distribution dynamics was correctly identified ≥ 95.1\% of the time and estimated accurately.  However, dynamic rate estimates were biased when secondary periods were closed to TE.  In empirical datasets, dynamic occupancy models had greater support than closed models for all species when secondary sampling periods occurred in immediate succession (i.e., 3 samples within 10 minutes); however, our results suggest that this is because dynamic estimates were heavily influenced by TE.  While dynamic occupancy models provide estimates of "colonization" and "extinction," these values do not inherently represent distribution dynamics unless TE has been modeled or accounted for with sampling design.  Naivete to this fact can result in incorrect conclusions about biological processes.}

\normaltalk{\capitalisewords{Using Molecular Methods to Identify Fungi at Woodpecker Excavated Cavities in Western Coniferous Forests}}{\underline{Teresa J Lorenz}, Michelle A Jusino, Jonathan S Schilling, Erin Andrews, Jerred Seveyka, Philip C Fischer}{Woodpeckers are indicator species for many land management agencies because of their keystone role as cavity excavators. In western coniferous forests, past efforts to create snags and increase woodpecker excavations have not been very successful in attracting woodpeckers.  However, fungal communities in woodpecker excavations are understudied and the causative decay agents are not known with certainty.  To inform snag creation efforts, we used high-throughput amplicon sequencing of the fungal ITS2 region to characterize fungal communities in naturally occurring woodpecker cavities in Washington and Oregon. We sampled 100 woodpecker cavities in ponderosa pine (Pinus ponderosa) and Douglas-fir (Pseudotsuga menziesii) snags in 2017.  For comparison we also sampled 80 non-excavated snags.  Fungal communities differed between excavated and non-excavated sites (adonis P $<$ 0.0001, r2 = 0.16, F = 21.6) and between Washington and Oregon (adonis P $<$ 0.0001, r2 = 0.07, F = 10.9).  The most prevalent wood decay fungus in Oregon was the brown-rot fungus Fomitopsis pinicola (58\% of nests) and it occurred predominately in large diameter, recently fire-killed ponderosa pine snags.  In Washington Rhinocladiella atrovirens, a soft rot fungus, was the most common taxa and was detected in 38\% of nests.  Fomitopsis pinicola occurred in only 17\% of Washington nests.  None of the sampled snags showed visible signs of fungal colonization indicating that molecular methods are necessary at this time to identify whether trees are infected with decay fungi.}

\normaltalk{\capitalisewords{Anthelminthic Drugs Modulate the Acute Phase Immune Response but not the Microbiome in Wild Song Sparrows}}{\underline{Grace J Vaziri}, Michelle A Jusino, Matthew T Brewer, James S Adelman}{Helminths and microbial communities share space within their hosts. Studies in model organisms have shown that gut helminths can bias immune responses away from inflammatory  processes both independently, and via interactions with the host-associated microbiota. To investigate the role of helminth-induced immunomodulation during co-infection, we tested interactions among gut helminths, the microbiota, and host immunity in wild song sparrows (Melospiza melodia). We used anthelminthic drugs to treat helminth infections in song sparrows from a population with a historically high prevalence of helminth parasitism, and measured markers of systemic inflammation (heterothermia and locomotor activity) in response to an immune challenge with lipopolysaccharide (LPS; a way to mimic bacterial co-infection). Additionally, we characterized the cloacal microbiota of anthelminthic-treated and control birds to determine whether drug treatment changed the cloacal microbiota, and whether the structure of the cloacal microbiota influenced the severity of birds' immune responses. We predicted that treated birds would display higher fevers and reduced locomoter activity after LPS injection. As expected, anthelminthic-treated birds expressed higher fevers in response to immune challenge. However, treatment did not affect birds' locomotor activity. We observed wide variation in microbiota diversity among individuals, but did not observe changes to the microbiota due to treatment with anthelminthic drugs, nor was a relationship between microbiota and the severity of the acute phase response detectable. We found that natural helminth infection can reduce the severity of birds' thermoregulatory response during an immune challenge in songbirds, without impacting behavioral responses to infection, or the cloacal microbial community.}

\normaltalk{\capitalisewords{Prioritizing Regional Landscapes to Achieve Biological and Social Objectives Through Bird Habitat Conservation}}{\underline{Gregory J Soulliere}, Mohammed A Al-Saffer}{Targeting conservation to achieve biological objectives for waterfowl and social objectives for people is an emerging priority for bird conservation Joint Ventures implementing the North American Waterfowl Management Plan (NAWMP). We developed a process to integrate objectives relevant to the NAWMP and the Upper Mississippi/Great Lakes Joint Venture (JV). Objectives related to waterfowl population demography, conservation supporters (hunters and other outdoor recreationists), and wetlands' ecological services important to society. This effort resulted in a Decision Support Tool (DST) to guide regional waterfowl habitat retention and restoration. Starting with a table of conservation issues, we transformed biological and social data into a set of six spatially explicit model-based maps designed to achieve individual objectives. Importance weights for each objective were established by regional decision makers (JV Management Board) and these weights were applied to output maps. The aggregate map product directs waterfowl habitat conservation to areas that should also enhance hunting and outdoor-recreation opportunity, while addressing watershed impairments. This spatially explicit DST emphasized conservation delivery in central and southern portions of the region, locations currently with relatively low conservation land holdings. The DST-development process also clarified the importance of strategically targeting habitat conservation that effectively achieves biological and social objectives. Moreover, the DST is adaptable; objectives and their weights may be adjusted and subsequent output maps customized depending on changing stakeholder priorities.}

\normaltalk{\capitalisewords{Breeding Propensity and Causes of Nest Failure for Marbled Murrelets in Washington State}}{\underline{Teresa J Lorenz}, Martin G Raphael, Thomas D Bloxton}{Marbled Murrelets (Brachyramphus marmoratus) are threatened seabirds from British Columbia to California.  Rates of breeding propensity and causes of nest failure are known for few populations because of the difficulties of monitoring this cryptic species.  We used radio telemetry (n = 157) to locate Marbled Murrelet nests from 2004 to 2008 in northwestern Washington and southwestern British Columbia.  To determine causes of nest failure, we monitored nests with video cameras (n = 4), monitored murrelet nest visits with telemetry (n = 20), and climbed nest trees after the nesting season to view nest contents (n = 15).  Twenty radio-tagged murrelets attempted to breed in our study (12.7\%) and we estimated breeding propensity averaged 13.1 to 20.0\% across five years.  This is the lowest breeding propensity reported for a population of Marbled Murrelets to date.  Among 20 nests, four successfully fledged young and 16 failed. Among failed nests, ten failed from unknown causes and the remaining six from non-predatory causes. This was surprising because the main cause of nest failure reported in past studies was predation by corvids. Nests in our study failed from egg abandonment, eggs failing to hatch, and chicks falling from nest platforms.  Our results indicate non-predatory factors are influential in Marbled Murrelet reproductive failure in Washington.}

\normaltalk{\capitalisewords{Wide-Ranging Movements and Long Nest-Sea Commutes by Marbled Murrelets in Washington}}{\underline{Teresa J Lorenz}, Martin G Raphael, Thomas D Bloxton}{The Marbled Murrelet (Brachyramphus marmoratus) is an alcid that forages in marine waters near shore but nests inland, commonly in old-growth forests.  Over five years we measured ranging behavior of Marbled Murrelets during the breeding season in Washington State where murrelets are threatened and declining.  Marine 95\% kernel ranges averaged 938 km2 (SD ±1348) and were larger than those reported for non-threatened populations in Alaska in other studies.  Ranges computed from MCPs (mean = 708, SD ±847) were similar to those reported for threatened populations in California, although larger than those reported in Alaska.  Least cost paths (LCPs) computed between consecutive marine telemetry locations were greater than reported previously in Alaska.  Variation in LCPs in our study were not associated with oceanographic conditions although appeared greater for murrelets captured along the Pacific Coast, Washington, compared to those occupying interior marine waters in Puget Sound. For breeders, nest-sea commuting distances were also greater than reported previously with four breeders (20\%) traveling further than the previously reported maximum of 125 km (one-way) between nest and forage locations.  The large marine ranges and long nest-sea commutes in this study point to poor-quality terrestrial and marine habitat in Washington compared to other parts of the Marbled Murrelet's range.}

\normaltalk{\capitalisewords{Dispersal and Marine Habitat Use During the Breeding Season by the Marbled Murrelet in Oregon}}{Marie-Sophie Garcia-Heras, \underline{Lindsay J Adrean}, S K Nelson, Daniel D Roby, Matthew G Betts, James W Rivers}{The threatened Marbled Murrelet (Brachyramphus marmoratus) utilizes nearshore marine habitats for foraging and flies inland to nest in mature coastal forests, often crossing long distances. Because of the difficulty in studying this species, few conclusions have been made about this species' movement patterns and marine habitat use in Oregon. We captured and radio tagged 137 adult Marbled Murrelets at-sea during the 2017 and 2018 breeding seasons along the central Oregon Coast. Ground and aerial telemetry surveys were conducted from May-August to relocate individuals at foraging sites. Across the two seasons we collected $>$1100 GPS locations from aerial telemetry and $>$7300 bearings from ground-based telemetry stations. In 2017, no birds were detected moving inland to nest and the majority of individuals dispersed long distances outside of the 120 km core study area. In the following year, 10 individuals attempted to nest and most tagged birds clustered around marine areas and moved shorter distances. The ocean conditions varied considerably in these two years, likely affecting food availability for Marbled Murrelets. These results provide a better understanding of overall dispersal movements and foraging habitat use of Marbled Murrelets in Oregon. This has implications for the population estimation methods currently used for this species and for conservation measures related to marine habitat.}

\normaltalk{\capitalisewords{Sister, Sister: Song Recognition and the Role of Early Experience in Sympatric Zonotrichia}}{\underline{Emily J Hudson}, Dai Shizuka}{Understanding the effects of early experience with closely-related heterospecifics on learned species recognition is important to understanding pre-mating barriers between recently diverged groups. Songbirds have long been used to study the role of experience in shaping behavior (songs) that are used for territory defense and attracting mates. The white-crowned sparrow (Zonotrichia leucophrys) was used to pioneer the auditory template model, in which young birds begin memorizing conspecific sounds soon after fledging; sounds heard before this age do not appear in adult song. Even prior to learning, however, some nestlings show the ability to discriminate heterospecific from conspecific song in the nest, as shown in golden-crowned sparrows (Z. atricapilla), the sister species to white-crowned sparrows. This species pair overlaps broadly in breeding distribution, such that young birds in some populations hear both species' songs in the nest, and during the sensitive period for song learning as fledglings. We took advantage of a population with dense golden-crowned sparrow breeding territories, and a smaller number of patchily-distributed white-crowned sparrow pairs, to test whether acoustic exposure to heterospecific songs in the nest affected the discrimination abilities of nestlings in a natural environment. Using autonomous recording units, we quantified the amount of white-crowned and golden-crowned song audible at each golden-crowned sparrow nest, then tested the nestlings for their behavioral response to each song type. Our behavioral results will be discussed in the context of ongoing genetic work to determine the history of gene flow in this species pair.}

\normaltalk{\capitalisewords{Phylogenomic Analysis of Species Limits and Hybridization in the Red-Bellied/Golden-Fronted Woodpecker Clade (Melanerpes)}}{\underline{Matthew J Miller}, Phillip Lavretsky}{TBD}

\normaltalk{\capitalisewords{Phylogenetic and Functional Structure of Avian Communities along a Tropical Elevational Gradient}}{\underline{Andy J Boyce}, Subir S Shakya, Frederick H Sheldon, Robert G Moyle, Thomas E Martin}{Understanding how biotic and abiotic interactions influence community diversity and structure is a fundamental goal in community ecology. Addressing this issue is particularly tractable on tropical mountains due to high rates of species turnover and strong abiotic gradients across elevations. Here, we examined elevational patterns of avian community structure in tropical Malaysia to assess changes in the relative strength of biotic and abiotic interactions. We used metrics based on phylogenetic relatedness and functional traits to identify patterns and drivers of community structure. High elevation communities were composed of more phylogenetically and functionally similar species than would be expected by chance. Resource acquisition traits, in particular, were clustered at high elevations, suggesting low resource and habitat diversity were important drivers of those communities. Traits associated with cold tolerance and low atmospheric pressure showed no elevational patterns. All traits were neutral or over-dispersed at low elevations suggesting an absence of strong abiotic filters or an increased influence of interspecific competition. However, relative bill size, which is important for thermoregulation, was larger in low elevation communities, suggesting abiotic factors may also influential there. Regardless of metric, clustered and neutral communities were more frequent than over-dispersed communities, implying that interspecific competition among close relatives may not be a pervasive driver of community structure in tropical birds. Overall, our data reveal that a diverse set of biotic and abiotic factors underlie elevational variation in community structure and that both biotic and abiotic forces are likely important in structuring communities at low and high elevations.}

\normaltalk{\capitalisewords{Moving Bird Conservation Forward Through Collaborative Action Targeting Grasslands and Wetlands in the Chicago Wilderness Region}}{\underline{Stephanie J Beilke}, Nathaniel Miller, Daniel Suarez, Bradford Kasberg}{Although heavily urbanized, the Chicago Wilderness region contains a patchwork of natural areas that provide important habitat for a variety of bird species of conservation concern. Many wetland and grassland bird populations are declining throughout the Midwest region, yet the Chicago area remains a stronghold for grassland species such as Bobolink and Henslow's Sparrow and wetland species such as Common Gallinule and Least Bittern due to investment from a variety of dedicated conservation partners. Audubon Great Lakes has become a crucial partner in the Chicago Wilderness region by coordinating conservation actions, leading community science-driven bird monitoring and guiding decision-making across the region. Convening conservation partners to form common goals and communicate management impacts on bird populations has proven critical for moving bird conservation forward in an urban fragmented landscape. Over the past five years, Audubon Great Lakes has led the Chicago Wilderness Grassland Bird Task Force and Calumet Wetland Conservation Working Group, which have collectively engaged over 15 landowners, at least eight other stakeholders and dozens of volunteers. Looking forward, these partnerships continue to adapt and evolve as we recognize the need to explore and coordinate the use of new technologies, such as drone imagery and Motus wildlife tracking, and emphasize climate resiliency in our conservation strategies. The result of these efforts is successful partnership building that influences Audubon's growing conservation work across the Great Lakes area and beyond.}

\normaltalk{\capitalisewords{Addressing Population Declines with Full Annual Cycle Models:  Current Advances and Data Needs}}{\underline{Cara J Joos}, Clark S Rushing}{The Partners In Flight (PIF) 2016 Landbird Conservation Plan indicates that almost 20\% of Canadian and U.S. landbird species are in danger of becoming extinct or endangered. Many species remain common but are in steep decline. Given the migratory nature of many of these species, effective conservation actions must be targeted where and when migratory birds are most limited during their life-cycle. At present, researchers lack an in depth understanding of the full annual cycle for all but a handful of species and there is tremendous need to increase our knowledge. In this talk, we will provide an overview of advances in the field of full-annual-cycle population modeling and how these methods can be used to advance conservation of migratory birds. We will also describe the results of a recent effort by the PIF Science Committee to identify groups of species with similar combinations of breeding and wintering geographies and habitats. These focal species are representative of particular winter and breeding geographies. As a result, focused work on these species will allow us to systematically fill data gaps for full annual cycle modeling in a way that maximizes our ability to extrapolate among species in similar geographies and habitats.}

\normaltalk{\capitalisewords{Helpers Mitigate Negative Effects of Predation Threat in the Cooperatively Breeding Purple-Backed Fairy Wren}}{\underline{Dylan K Meyer}, Eva Kinnebrew, Stephen Pruett-Jones, Allison E Johnson}{The benefits of cooperative breeding have proven difficult to quantify given the varied selection pressures acting in different ways on each group member. In the purple-backed fairy wren (Malurus assimilis) it has been shown that groups with helpers have an increased probability of successfully fledging young despite no increase in provisioning rates observed in association with group size. We set out to examine how helpers can increase fledging probability through mitigating the increased stress a predation threat has on nestlings. Over the course of three breeding seasons (Oct-Dec) from 2015-2017 we conducted paired, randomized playback experiments simulating predation threat to groups of purple-backed fairy wrens at Brookfield Conservation Park in South Australia. We hypothesized that larger groups would be able to maintain a more 'normal' provisioning rate throughout a predation threat because of increased helper effort as compared to two-bird groups. We confirmed that provisioning is stable across group size under control conditions, however under treatment conditions, smaller groups saw a decrease in provisioning not found in their larger counterparts. Under threat conditions, large groups were more vigilant than smaller groups, something not seen under control conditions.  Additionally, although helpers accounted for nearly 1/3 of provisioning in large groups their feeding share did not increase under threat conditions. Rather, the dominant female provisioned more while the dominant male provisioned less, the same as in small groups. This study sheds light on how helpers can mitigate negative consequences of stress and contribute to a higher likelihood of reproductive success.}

\normaltalk{\capitalisewords{Social Network Metrics Predict Future Social Status in Male Lance-Tailed Manakins}}{\underline{Amanda K Beckman}, Carla C Vanderbilt, Emily H DuVal}{Social interactions can be important for information transfer between group members, especially for young individuals. Early social experience has been demonstrated to affect social rise as an adult later in life in male long-tailed manakins (Chiroxiphia linearis). In the closely related lance-tailed manakin (Chiroxiphia lanceolata), males frequently perform displays similar to courtship displays without females present. These interactions may be important for forming future partnerships and learning complex displays, but this hypothesis remains to be tested. The goals of this project were to 1) quantify how males differ in their social interactions, and 2) determine how these differences in young males relate to future social status. We quantified male interactions from videos at nine known displays sites (N = 63 males and 566 interactions) during the beginning of the 2013 breeding season (Feb-Mar). We used social network metrics and GLMMs to quantify differences in social interactions based on display sites, plumage, and social status. We used social status at last sighting from 2014-2017 was used to determine if previous social interactions could predict later rise in social status. We found that individual social interactions varied based on plumage, social status, and display site. Additionally, we found that two social network metrics explained later rise in social status. These results indicate that early social interactions are important for yet another species in the genus Chiroxiphia; supporting the theory that early experiences are essential for understanding social behaviors.}

\normaltalk{\capitalisewords{Does a "Sonic Net" Protect Sunflower from Damage by Blackbirds?}}{\underline{Amanda K Werrell}, Page E Klug, John P Swaddle}{Red-winged Blackbirds cause millions of dollars of damage to North Dakota's sunflower crops each year. My research explores the use of sound that is designed to mask communication among birds (termed a "Sonic Net") to deter red-winged blackbirds from two key habitat types: fields of growing sunflowers where they damage the crop and nearby roosting sites in cattail marshes. The Sonic Net works by masking communication of a target species producing "pink noise" overlapping the frequencies in which a species communicates. If birds can't listen out for predators or conspecific warning calls they are predicted to leave the area. Working with local sunflower producers in North Dakota, we set up experimental sites in 5 sunflower fields and 2 cattail marsh roost sites that were being actively used by large flocks of blackbirds. Preliminary results suggest different responses in the different habitat types. The Sonic Net was effective in reducing bird damage to sunflower plants, but anticipate more subtle the effects in roosting habitat.}

\normaltalk{\capitalisewords{Saving the Red Siskin (Spinus Cucullatus): Managing New Flock Member Introductions in Captivity}}{\underline{Stacy K Hill}}{The Red Siskin (Spinus cucullatus) is a small passerine bird from Venezuela that is endangered from over a century of illegal pet trade to breed Red-Factor Canaries (Serinus canaria domestica). The Red Siskin Initiative, in partnership with the Smithsonian Conservation Biology Institute (SCBI), has established a captive flock to develop husbandry best practices for a breeding and reintroduction program. This study focused on dominance and hierarchy structures within the flock at SCBI, and the effects on the established order when new flock members were added. It is important to assess the receptivity of Red Siskins to new individuals, since a captive breeding program will require regular transfers of birds between facilities to maintain genetic diversity. Two techniques were used to monitor behavior – focal observations, and scan sampling of video recordings. Six cameras were used within the enclosures to capture behavior. About half of the focal observations occurred during recording sessions to assess possible changes in flock behavior in the presence of an observer. The results show that birds significantly increased their aggressive behavior after the introduction of new flock members. However, this increased aggression largely stems from original flock member interactions with other original flock members. This held across focal observation and scan sampling data sets. Furthermore, researcher presence did not seem to significantly change the birds' behavior. This suggests other factors beyond the presence of new birds could be influencing their behavior, and further research questions are proposed to aid ongoing conservation planning.}

\normaltalk{\capitalisewords{Cross-Ecosystem Subsidies in Acadia National Park: How Invertebrates can Help Us Understand Mercury Exposure in Songbirds}}{\underline{Allyson K Jackson}, Leo Frampton, Matthew Garafalo, Stephen Harris, Batya Nightingale, Alex Youre-Moses}{As they emerge from freshwater ecosystems and take flight, aquatic insects play an important role for the surrounding terrestrial ecosystem. This pulse of resources out of the aquatic system and into the surrounding terrestrial ecosystem creates a rich food source for birds living near the water. Furthermore, aquatic emergence can occur at a critical time in the spring before terrestrial invertebrate prey become plentiful. With these important cross ecosystem subsidies, unfortunately, also come aquatic contaminants. Mercury, in particular, moves through emergent aquatic insects to riparian birds. In June and July 2018, we collected aquatic, terrestrial and emerged aquatic insects at four sites in Acadia National Park (Maine, USA) to quantify the movement of resources and mercury out of freshwater systems and determine how they impact the surrounding bird community. Our objectives were to 1) quantify abundance, diversity, and biomass of emergent aquatic insects throughout the spring and summer, 2) calculate temporal changes in invertebrate diversity and abundance in riparian zones during the critical songbird breeding season, 3) correlate riparian songbird mercury contamination with emergent insect biomass. Additionally, we taught park visitors about the connection between emergent aquatic insects, contaminants and riparian areas, by recruiting help from citizen scientists for the invertebrate collections. Documenting, understanding, and protecting cross-ecosystem subsidies now is critical as we manage for resilient ecosystems in the second century of stewardship at Acadia.}

\normaltalk{\capitalisewords{Long-Term Effects of Early-Life Stress on the HPA Axis in a Short- And Long-Lived Bird}}{\underline{Jacquelyn K Grace}, David J Anderson, Frederic Angelier}{Acute, short-term effects of early-life stressor experience and associated glucocorticoid upregulation on physiology and survival are widely documented across vertebrates. However, long-term effects of early-life stress are less well understood, especially for wild species through adulthood. We evaluate effects of early-life stress on baseline and stress-induced corticosterone in two wild birds:  free-living Nazca boobies (Sula granti) and captive House sparrows (Passer domesticus).  Nazca booby adults that experienced maltreatment as nestlings, a wholly natural stressor, exhibited depressed baseline corticosterone in females, and elevated stress-induced corticosterone in males. House sparrow adults that experienced an experimental increase in circulating corticosterone as nestlings also displayed depressed baseline corticosterone , and in contrast to Nazca boobies, depressed stress-induced corticosterone.  Our results for Nazca boobies correspond to an emerging trend of HPA-axis hypersensitivity following early-life stressor experience, while those for house sparrows deviate from this trend.  These results highlight the sex- and species-specific nature of long-term effects of early-life corticosterone elevation.}

\normaltalk{\capitalisewords{Persistence in a Changing Landscape: Recurring Bird Concentration Areas on Alaska's Arctic Coastal Plain}}{\underline{Benjamin K Sullender}, Melanie A Smith}{Alaska's Arctic Coastal Plain is recognized for internationally significant biological values, including key bird nesting, stopover, and molting habitat. At the same time, the Arctic Coastal Plain is being markedly impacted by climate change, and species distributions may shift in response to a rapidly changing environment. Long-term observation data can help identify areas that have provided and will continue to provide important habitat, even in the midst of ecological changes. Our goal with this analysis was to identify areas that support recurring high concentrations of Arctic waterbirds. We analyzed 20 years of aerial survey data from the US Fish \& Wildlife Service's Division of Migratory Bird Management. We aggregated these surveys into four five-year time periods: 1996-2000, 2001-2005, 2006-2010, and 2011-2015. We divided survey transects into a series of cells and, for each cell, quantified survey-effort weighted bird observations for 13 bird taxa. For each taxon and each time period, we converted effort-weighted observations into percentiles. We identified cells within the top 90th percentile for each species as high concentration areas. We defined high concentration areas for at least three of the four time periods as persistently important areas.  All species studied had at least a few persistently important areas, distributed widely from Dease Inlet to the Colville River delta. Although many rapidly progressing impacts of climate change are focused on the coastline and coastal habitats, these same areas have and continue to support high concentration areas for a wide range of birds.}

\normaltalk{\capitalisewords{Glare Reduction Properties of Dark Avian Facial Markings}}{\underline{Clara K Lebow}, D B Burt}{Avian facial plumage and bill coloration may serve one or more adaptive functions.  Several researchers have proposed that dark eyestripes, bills, and facial masks aid in reducing glare, however, there have been relatively few tests of this hypothesis. Dark facial markings have been shown to have an adaptive glare-reduction function in recent field studies of a few species, but this hypothesis has never been tested in a broad multi-species analysis. We tested the hypothesized link between dark facial markings and glare reduction, under natural lighting conditions, in several bird species using a spectrometer probe placed in the eye-position of museum specimens. Reductions in irradiance in full, natural sunlight, used as a measure of glare, was quantified for specimens varying in bill and head plumage coloration and pattern.  Each specimen was tested with the head held at various angles to mimic natural foraging positions. We also quantified the brightness of bills and plumage surrounding the eye of these same specimens using reflectance spectroscopy. Correlations between irradiance measurements and the bill and plumage brightness were analyzed. We show a significant relationship with average head darkness and reductions in irradiance values when the head was held at 45 degrees from horizontal.  Dark patches in the anterior and posterior dorsal quadrants are most important in this reduction in irradiance. Bill brightness showed no significant effect on changes to irradiance. We discuss our findings with an eye towards the influence of this characteristic on selection of foraging habitat.}

\normaltalk{\capitalisewords{Influence of both Male and Female Attractiveness on Reproductive Success, Offspring Phenotype, and Nestling Sex Ratio}}{\underline{Joey K Hubbard}, G P Moranville}{Sexual selection has primarily focused on male signals, however in species with socially monogamous mating systems and biparental care, both sexes may benefit from evaluating the quality of potential mates. Recently, there has been a growing interest in female sexual signal traits resulting in a richer understanding of how traits that are expressed in both males and females are shaped by sexual selection and how signals in both sexes relate to reproductive success of the pair. In North American barn swallows (Hirundo rustica erythrogaster), both sexes express rufous-colored plumage that covaries with aspects of fitness; in males darker individuals are allocated more paternity and in females darker individuals are more likely to lay a second clutch. Here, we asked (i) whether barn swallows mate assortatively by color and (ii) how the combined and relative quality of the male and female in a pair affect reproductive success and sex allocation of the offspring. We expected pairs with higher quality individuals to produce more offspring, offspring in better condition, and offspring with darker juvenal plumage. Additionally, we predicted that the sex ratio of the brood would skew toward the sex of the parent expressing darker sex-specific color. Results from this study will provide further insight into how sexual selection shapes coloration in barn swallows. Furthermore, this work will add to our understanding of the decisions made by females when allocating paternity and offspring sex, specifically whether females assess their own attractiveness in addition to the attractiveness of their mate.}

\normaltalk{\capitalisewords{Direct Effects of Brown-Headed Cowbird (Molothrus Ater) Brood Parasitism on the Growth and Development of Grassland Songbirds}}{\underline{Sarah K Winnicki}, W A Boyle}{Patterns of animal growth and development vary widely yet the drivers of that variation are often unclear. In altricial birds, brood parasitism presents a unique set of selective pressures potentially influencing host development. We hypothesized that Brown-headed Cowbird (Molothrus ater) brood parasitism influences host nestling growth and development through direct sibling competition for parental care in ways that are dependent on the relative size of hosts and parasites. At the Konza Prairie in Northeast Kansas, a site where ~50\% of nests of grassland songbirds are parasitized, we located and monitored 148 nestling-stage nests of three grassland-obligate songbirds ranging in size from ~40\% to 270\% of the cowbirds' adult mass: Grasshopper Sparrows (Ammodramus savannarum), Dickcissels (Spiza americana) and Eastern Meadowlarks (Sturnella magna). We measured the growth of 316 nestlings' tarsi, wings, mass, bills, and feathers every two days. The effects of cowbird parasitism varied between species and was not obviously linked to host size as we had predicted. The presence of cowbirds was associated with trade-offs in development for meadowlark and sparrow nestlings; both had larger bones but suppressed feather growth when cowbirds were present. Dickcissels, the most commonly parasitized species, did not grow more slowly when cowbirds were present but did show delayed alertness. Our study identifies the development strategies that allow threatened grassland hosts to fledge alongside cowbirds and provides insight on the past and future evolution of cowbird hosts.}

\normaltalk{\capitalisewords{Environmental Drivers of Annual Survival and Site Fidelity of Lekking Great Snipe in Norway}}{\underline{Brett K Sandercock}, Stein A Saether, John A Kalas}{Great Snipe (Gallinago media) are a rare species of shorebird with a patchy distribution in the Scandinavian mountains.  Breeding birds feed primarily on earthworms to meet their energy demand for nocturnal lekking behavior, nesting, and long-distance migratory movements to sub-Saharan Africa.  We investigated the demography of Great Snipe with long-term monitoring of two lek complexes in central Norway over a 34 year period (1986-2018).  Males dispersed among leks within and between years but showed strong site fidelity once established as territorial birds.  In both sexes, time-since-marking models revealed a strong effect of transience with lower apparent survival immediately after first marking.  Apparent survival of returning birds was similar among sites but lower among males (phi = 0.529 to 0.563) than females (phi = 0.578 to 0.656).  The two strongest environmental drivers of annual variation in adult survival were spring temperatures during arrival and phase of the rodent cycle.  Spring temperatures affected access to food resources at alpine mires, whereas rodent cycles were correlated with predation pressure on lekking and nesting birds.  Overall, Great Snipe are a relatively short-lived species of shorebird with a life expectancy of 2-4 years.  Environmental conditions encountered during the breeding season play an important role in the annual cycle of this migratory species.}

\normaltalk{\capitalisewords{Using Molecular Techniques to Study the Diets and Gut Microbiota of Migratory Birds}}{\underline{Brian K Trevelline}}{The advent of next-generation sequencing techniques has opened up new research possibilities throughout the fields of ecology and evolution. For ornithologists, these tools provide the means for studying the intricate relationships between birds and a variety of both macro and microorganisms with unprecedented power and clarity. Using molecular data from several migratory species, I will discuss the advantages, limitations, and promise of these techniques in respect to fundamental questions in avian ecology and conservation. Specifically, I will show how DNA metabarcoding of fecal material can be applied to better understand diet specialization and niche partitioning in songbird communities. I will also illustrate how these methods can be used to study intraspecific differences in breeding and wintering diets, as well as diet shifts in response to habitat degradation and reduced food availability. Lastly, I will demonstrate how these techniques can be applied to study the effects of diet and habitat quality on the gut microbiome, which can profoundly influence aspects of avian physiology. To do this, I will highlight new research on the potential for vertical microbial transmission from mother to offspring, horizontal microbial transmission between brood parasites and their nestmates, intraspecific differences in gut microbiota across wintering habitats, and the effect of captive rearing on the gut microbiome of an endangered migrant.}

\normaltalk{\capitalisewords{Effects of Non-Breeding Foraging on Mixed-Species Flocking in Florida}}{Harrison H Jones, Mitchell J Walters, \underline{Scott K Robinson}}{Mixed-species flocks are ubiquitous across non-breeding forest bird communities, yet the extent to which positive or negative interactions structure these associations has been a subject of debate. Here we describe foraging ecology and flocking interactions of an insectivorous bird community in hardwood forests of North-central Florida. Our goal was to determine if similarly foraging species are more or less likely to associate in flocks, and if foraging ecology correlates with intraspecific abundance patterns within flocks. We quantified attack foraging maneuvers, substrate, and microhabitat of all 17 common insectivorous species and characterized the composition of 92 flocks. Most species had different, well-defined foraging niches during the non-breeding season, and some species shifted substrate use to exploit novel resources (dead leaf clusters, epiphytes) they rarely use during the breeding season. Flocking dominated community structure; 14 of 17 species joined more than 5\% of flocks, and 10 species had flocking propensities of over 0.80. Using network analysis, we found that similarly foraging species were significantly more likely to associate in flocks, a result driven by lower association strengths in large-bodied woodpeckers. We found no evidence of assortment by foraging behavior, possibly because each species had a distinct foraging niche. Intraspecific abundance patterns were correlated with foraging substrate use; species searching live leaves were more abundant than those foraging in more specialized substrates such as dead leaves, which usually only occurred singly. Our results highlight the importance of foraging substrate use and mixed-species flocks in structuring the non-breeding ecology of migratory birds.}

\normaltalk{\capitalisewords{Urbanization and Elevated Cholesterol in American Crows}}{\underline{Andrea K Townsend}, Hannah Staab, Christopher M Barker}{Although urban areas can be sources of abundant food for wildlife, anthropogenic foods may be lower in quality than natural food sources.  Overall, the consequences of anthropogenic food consumption for wildlife are poorly understood. Here, we examined how urbanization and anthropogenic food were linked to cholesterol, condition, and survival of American crows (Corvus brachyrhynchos). We collected cholesterol and landscape data from 166 crow nestlings along urban to rural gradient in Davis, CA, USA. We ran a supplementation experiment with high-cholesterol fast food (McDonald's cheeseburgers) on 86 nestlings in a relatively rural population in Clinton, NY, USA. In both populations, we also examined the links between cholesterol level, condition, and survival of the crows. We found that plasma cholesterol increased with percentage of impervious surface along the urban to rural gradient and that cholesterol levels were sensitive to anthropogenic foods: crows supplemented with fast food cheeseburgers had higher cholesterol levels than unsupplemented crows. Elevated cholesterol levels had no detectable effects on survival and were associated with higher indices of body condition, although urbanization itself was linked to lower survival. Elevated cholesterol levels could indicate access to high-calorie, high-fat anthropogenic foods, which might, in some contexts, improve body condition, offsetting other negative effects of urbanization.}

\normaltalk{\capitalisewords{The Effects of Recreational Activities on Avian Occupancy in a High Latitude Tundra Ecosystem}}{\underline{Avery L Meeker}, John M Marzluff}{Tourism is increasing in tundra ecosystems across the world, yet its  influence on bird communities and its interaction with other drivers of change is poorly known. To help fill this gap, we interviewed 11 people with traditional ecological knowledge of Denali National Park and Preserve, and measured occupancy rates of 13 bird species in relation to road proximity, traffic volume, and amount of hiking. Interviewees noted declines in American Golden-plover, Arctic Tern, Long-tailed Jaeger, and Northern Wheatear over the past five decades. Our occupancy study confirmed these reports as we did not detect a single Arctic Tern, few Northern Wheatears, and found both plovers and jaegers to be sensitive to hiking disturbance. We found that occupancy of tundra by Lapland Longspur and Horned Lark was reduced by road disturbance while that of five other generalist shrub-tolerant species increased. Hiking negatively affected occupancy of five species, but positively affected Fox Sparrow occupancy. Detection of species varied based on survey length, noise, start time and Julian date. The knowledge gained from this study reveals a dramatically different park today than a few decades past. Park managers should seek to balance human recreation with the needs of sensitive tundra-breeding birds to further protect species of conservation concern. This may be done by siting new trails in shrub-lands or forest, limiting access to tundra hiking areas during the early breeding season, creating better trails to reduce social trails, educating tourists about nesting birds, and closing especially important nesting areas to the public.}

\normaltalk{\capitalisewords{Effects of the Social Environment on Cloacal Microbiome Diversity}}{\underline{Jennifer L Houtz}, Conor C Taff, Cedric Zimmer, Maren N Vitousek}{Reciprocal interactions exist between host-associated microbiota and the degree of host sociality, as social organization can impact individual microbiome composition and the microbiome can influence host social behavior. Previous work across vertebrates has demonstrated that increased social connectedness leads to increased microbial diversity, which is often associated with better health. The semi-colonial, highly gregarious tree swallow (Tachycineta bicolor) naturally differs in its degree of social interactivity, measured as the number of visitors to the nest box and visits to other nest boxes. As expected, preliminary findings suggest that the frequency of social interactions is correlated with microbial diversity in our study population. Few studies have directly manipulated the social environment and assessed how this change might influence host microbial diversity. Here, we experimentally manipulated a social signal by dulling the white breast plumage of breeding female tree swallows in order to alter social interactions. We recorded their social interactions through a network of RFID readers installed at each nest box and characterized their cloacal microbiomes through 16S rRNA amplicon sequencing. Color-manipulated females received more female visitors at the nest box, and increased their cloacal bacterial diversity relative to sham controls. Our findings highlight how social interactions may propagate host microbial diversity, and potentially its health-related impacts within populations.}

\normaltalk{\capitalisewords{Stinky is Sexy: The Major Histocompatibility Complex, Mate Choice and Personal Odor in Leach's Storm-Petrels}}{\underline{Sarah L Jennings}, Brian A Hoover, Yung Wa Sin, Gabrielle A Nevitt, Scott V Edwards, Susan E Ebeler}{The Major Histocompatibility Complex (MHC) is a suite of highly polymorphic genes associated with adaptive immune response in vertebrates. In a number of species, MHC diversity is maintained in part through disassortative mate choice. Personal odor is thought to chemically signal MHC genotype allowing for assessment of this marker during courtship, but evidence supporting this link is limited to a few species. In this study, we investigated the interplay between MHC, mate choice, and personal odor in Leach's storm-petrels (Oceanodroma leucorhoa). This long-lived seabird has a remarkable sense of smell and a strong, musky scent. Mated pairs form long-term bonds with both sexes working to fledge a single chick each year. From 2011 to 2016, we monitored 550 nests on Bon Portage Island, Nova Scotia, Canada documenting mating patterns and reproductive success. We genotyped over 1000 birds for the MHC Class IIB molecule. Using this large dataset, we found a correlation between female MHC heterozygosity and reproductive success. We also discovered that males make disassortative mating choices to avoid homozygous females. To explore the link between personal odor and MHC genotype, we collected feathers from 80 individuals in 2016. Using gas-chromatography mass spectroscopy, we quantified the volatile chemical components of each individual's feather odor. While our efforts to examine whether the chemical composition of feather odor is correlated with MHC genotype are ongoing, we believe this to be the first study in a wild species to investigate all aspects of the relationship between MHC genotype, mate choice and personal odor.}

\normaltalk{\capitalisewords{Measuring Impacts of Non-Native Species on Cavity-Nesting Birds at the Continental Scale}}{\underline{Robyn L Bailey}, Holly A Faulkner, Vicki Y Martin, Tina B Phillips, David N Bonter}{Invasive species are a threat to global biodiversity. In North America, nest site competition from invasive cavity-nesting birds can limit the opportunities for successful nesting by native birds. We sought to quantify the relative impacts of European Starling (Sturnus vulgaris) and House Sparrow (Passer domesticus) on reproduction by native cavity-nesting birds. We administered a survey to citizen scientists who maintain nest boxes for birds in the United States and Canada. Respondents (N = 874) monitored an average of 56.87 (± 1.31 SE, range 1–494) nest boxes in 2018. Nearly one in three observers documented a non-native species usurping the nest of a native species (30.5\%). For people reporting any nest usurpation activity, rates of nest usurpation varied by host species (range: 0–28\%). House Sparrows were more likely than European Starlings to usurp the nests of bluebirds (Sialia sp.), Tree Swallows (Tachycineta bicolor), House Wrens (Troglodytes aedon), and chickadees (Poecile sp.), whereas European Starlings were more likely to usurp the nests of American Kestrels (Falco sparverius; paired t-tests comparing usurpation rates by House Sparrows and European Starlings, all P $<$ 0.05). Purple Martins (Progne subis) were equally likely to have nests usurped by starlings and sparrows. Our results suggest that House Sparrows and European Starlings are likely impacting reproduction at the population level for several native, cavity-nesting species.}

\normaltalk{\capitalisewords{Variation in Nest Survival of the Federally Threatened Streaked Horned Lark on Grasslands in the South Puget Sound Lowlands}}{\underline{Gary L Slater}, Adrian L Wolf, James Lynch}{Understanding demographic rates of threatened species is an important component in developing conservation strategies aimed at species recovery because survival and reproduction directly affect population growth rates. The Streaked Horned Lark (Eremophila alpestris strigata) is a federally listed subspecies that occurs in three regional populations of the Pacific Northwest.  Previous modeling indicated populations are declining and information is needed on factors that influence reproduction to inform management and restoration activities. In the South Puget Lowland region, the population numbers ~250 individuals, with $>$ 55\% of the population occurring on military training lands at Joint Base Lewis-McChord Military Base (JBLM).  We investigated reproduction in larks from 2013 to 2017 at three grassland sites (2 airfields and 1 native prairie) on JBLM.  At one of the airfields, predator removals were conducted as part of efforts to minimize wildlife-aircraft collision risk. We located and monitored 419 nests. The primary cause of nest failure was predation.  Preliminary results from the nest survival analysis indicated that site, date, and year were the biggest factors influencing nest survival.  However, nest survival was influenced by predator removal or distance to roads. At the native prairie site, variables associated with fire management were not associated with nest survival, whereas the distance to edge was positively associated with nest survival.  We will discuss how results from this study will help advance conservation strategies for the lark.}

\normaltalk{\capitalisewords{Mesoscale Winds Shape Spring Stopover Distributions of Migrating Birds along the Gulf of Mexico}}{\underline{Hannah L Clipp}, Emily B Cohen, Jaclyn A Smolinsky, Andrew Farnsworth, Kyle G Horton, Jeffrey J Buler}{Avian migration systems are inextricably linked to wind patterns aloft. At fine scales, winds affect migrants' departure decisions, as well as orientation, airspeed, and altitude during flight. At broad scales, winds can shape migration routes and flyways. Yet spatiotemporal relationships between wind patterns over large water bodies and proximate bird stopover distributions remain under-explored. We used data from the U.S. network of weather surveillance radars to measure the influence of winds over the Gulf of Mexico (GOM), Atlantic Ocean, and Caribbean Sea on the distributions of birds departing from terrestrial stopover habitats along the northern coast of the GOM. Specifically, we modeled the relative influence and interactions of 20 variables, including wind, geography, and landscape, on stopover densities of nocturnally-migrating landbirds during spring migration (March–May) in 2008–2015. After controlling for departure weather conditions and landscape composition, the total influence of winds aloft during migration on monthly mean bird stopover density was consistently high. Strong southerly winds over the GOM and easterly winds over the Caribbean Sea tended to increase overall stopover densities along the coast, while strong northerly winds over the Caribbean Sea tended to decrease stopover densities. In addition, strong easterly winds over the Atlantic Ocean increased stopover densities within the eastern GOM coastal region. This quantitative analysis suggests that winds aloft during migration influence when and where birds stop along the northern GOM coast. Furthermore, we conclude that easterly winds over the Atlantic Ocean and Caribbean Sea steer migrating landbirds toward the GOM coast.}

\normaltalk{\capitalisewords{Using Big Data to Assess the Vulnerability of North American Birds to Climate Change}}{\underline{Brooke L Bateman}, Chad Wilsey, Lotem Taylor, Joanna Wu, Geoff L , Gary Langham}{Biodiversity is being lost globally at an alarming rate with extinction rates up to 100x greater than historically. Climate change exacerbates the global biodiversity crisis, with 24–50\% birds highly vulnerable to climate change alone. In this work, we rely on the use of big data to develop an assessment of climate change vulnerability of over 600 North American Birds at a continental scale in both winter and summer. We built species distribution models based on 140+ million records from avian point count datasets with coverage including Mexico, the United States, and Canada. This effort uses the latest IPCC climate data, high (1km) spatial resolution, advanced modeling methods, an extensive occurrence dataset, and bird-guild specific habitat and land-use covariates. Our results indicate that sixty percent of North American birds are moderately or highly vulnerable to climate change. Of these climate vulnerable species, 75\% would have reduced vulnerability with a 1.5 °C versus a 3.0 °C warming scenario. We also assess bird range-shift models in conjunction with regional threats to develop a US-focused global change threat analysis for birds. This analysis gives a more holistic picture of the threat of global change stressors on birds, and continues to support conservation and advocacy work for birds facing climate change. This work revisits and expands upon our 2014 Birds and Climate Change Report with an updated and urgent message for continued policy actions to reduce global greenhouse gas emissions.}

\normaltalk{\capitalisewords{Population Genetics of a Beringian Endemic, Rock Sandpiper}}{\underline{Christin L Pruett}, Kevin Winker}{Recent research has suggested that worldwide declines of shorebird populations are linked to climate change due to increasing rates of nest predation and habitat alterations. To understand the long-term effects these changes might have, in the context of the adaptive potential of Arctic shorebirds, population-level evaluations of gene flow and genetic diversity are needed. Rock Sandpipers (Calidris ptilocnemis) are unique high-latitude shorebirds in that their distribution is restricted to Beringia and they have subspecies-level phenotypic and genetic differences among populations. In addition, Rock Sandpiper populations are declining and are listed as being of high conservation concern in the US Shorebird Conservation Plan. Rock Sandpipers were sampled from 8 locations in the Bering Sea region, including populations from all described subspecies. Eight microsatellite loci were genotyped and analyses of this data showed that island populations are isolated from one another and from mainland populations due to very limited gene flow. Prior work suggests that the uniqueness of island populations could be due to isolation during Pleistocene glacial cycles. Although Rock Sandpiper populations exhibit minimal gene flow, they have similar levels of genetic diversity suggesting that effective population sizes have been large enough to limit the effects of genetic drift. Based on this research, subspecies should continue to be conserved as separate units.}

\normaltalk{\capitalisewords{Saving the Kiwikiu (Pseudonestor Xanthophrys): Recovery Efforts in Maui, Hawaii}}{\underline{Hanna L Mounce}, Chris Farmer, Lainie Berry, Bryce Masuda, Chris Warren, Laura Berthold, Fern Duvall}{Kiwikiu (Maui Parrotbill; Pseudonestor xanthophrys), one of the most endangered Hawaiian passerines, continues to be under unmanageable threat from invasive mammalian predators and non-native disease. Recent assessments of this critically endangered honeycreeper estimate $>$ 312 individuals across ~30 km2. The US Fish \& Wildlife Service has recommended establishing a second population on the leeward slopes of Haleakal\={a} volcano to increase total population size and provide protection from severe weather events or other catastrophic loss within the species' small current range. Given the severity of the decline in the current population, recovery actions must be implemented now despite imperfect information in regards to current or future risks. Nakula Natural Area Reserve, the site selected for the first experimental releases, exists in a deteriorated state as a result of a century of browsing and grazing damage from non-native ungulates. Following fencing and eradication of ungulates, the forest in this area has begun to recover through natural regeneration and conservation restoration efforts guided by results of five years of experimental restoration trials in the release area. Although the forest is far from recovered, the first releases of captive and wild Kiwikiu into Nakula are proposed to begin in October 2019. The short term goal of this project is to create a disjunct population that survives multiple years. This is the first step of a multi-year effort to implement actions explicitly identified in the USFWS species recovery plan to re-establish a population on southern Haleakal\={a} and eventually to have a second self-sustaining population.}

\normaltalk{\capitalisewords{What Factors Influence Nest Survival of the Endangered Yellow-Headed Amazon?}}{\underline{Fabio L Tarazona-Tubens}, Charles R Britt, Fitsum Abadi, Martha J Desmond}{Several factors have been identified as driving forces behind the decline of Psittacid populations worldwide. The endangered Yellow-headed parrot (Amazona oratrix) is not an exception to this declining population trend, suffering considerable reduction in abundance and distribution across its historic range. Identifying natural and anthropogenic causes of nest failure is a critical step towards improving our understanding of this species to ultimately develop conservation plans that will increase recruitment of individuals into the population. Our objective was to model nest survival in relation to temporal and environmental factors, and physical properties of nest cavities. We examined survival of Yellow-headed parrot nests across protected areas in Belize during the 2017 and 2018 nesting seasons (n=102). In addition, we used a subset of our data to model natural (i.e. predation, abandonment) and anthropogenic (i.e. poaching), causes of nest failure independently. Preliminary results indicated that mean cumulative nest survival estimate was 0.355 (SE = 0.003), and was positively influenced by properties of the cavity (i.e. cavity depth) but negatively affected by local-scale environmental factors (i.e. frontal visibility and canopy cover). These factors were also ranked as important predictors of survival when only natural causes of failure were modeled independently. However, when anthropogenic causes of failure were modeled independently, cumulative nest survival estimate was 0.259 (SE = 0.006) and was negatively influenced by temporal factors (i.e. nestling stage). Our results suggest that this species has low nest survival and that low annual recruitment may contribute to future population declines of Yellow-headed parrots in Belize.}

\normaltalk{\capitalisewords{Experimental Forest Fragmentation Alters Interaction Networks of Amazonian Mixed-Species Flocks}}{\underline{Cameron L Rutt}, Karl Mokross, Philip C Stouffer}{Habitat fragmentation has been associated with myriad negative effects for forest-dependent birds in the Neotropics. However, the vast majority of these studies inferred changes by comparing pre-existing fragments with separate control sites. This confounds area loss with isolation and ignores effects of patchy distributions and local habitat heterogeneity. To directly test the effects of fragmentation on Amazonian mixed-species flocks – a complex and diverse species interaction network – we accompanied birds before and after resiolation of three 10-ha fragments at the Biological Dynamics of Forest Fragments Project, north of Manaus, Brazil. Following initial isolation in the 1980s, these fragments have been surrounded by a matrix of developing second growth forest that was cut in 2013-2014. We quantified flock space use, species richness and attendance, and the social network structure of both experimental and control flocks pre- and post-treatment. Mixed-species flocks decayed following reisolation, including one flock that disappeared entirely by the third year post-manipulation. Home-ranges of fragment flocks shrunk and movements were confined by newly created hard edges. Flock core areas also shifted to a greater extent in experimental flocks, and species richness declined. These results provide direct experimental evidence of rapid ecological decay in forest fragments, affecting both direct metrics (e.g. species richness) and emergent properties of a complex social network. It also provides retrospective insight into the value of adjacent second growth habitat for species participating in mixed-species flocks.}

\normaltalk{\capitalisewords{Are Wetlands Hotspots for Bioaccumulation of Mercury in Songbirds?}}{\underline{Rebecka L Brasso}, Katie A Rittenhouse, Virginia L Winder}{Environmental conditions in wetlands facilitate favorable biogeochemical conditions for the conversion of inorganic mercury into methylmercury. For this reason, wetlands are increasingly classified as mercury hotspots, places where biota exhibit elevated mercury concentrations. While it is clear that wetlands play an important role in methylmercury production, factors such as geographic variation in mercury deposition, wetland type, and trophic dynamics can cause variation in mercury dynamics and bioaccumulation in biota occupying wetlands or connected to wetland trophic systems. Here, we use songbirds as bioindicators in a two-pronged approach aimed at evaluating the state of our understanding of mercury bioaccumulation by songbirds in wetland ecosystems. First, we use a case study in southeast Missouri to compare blood mercury concentrations in tree swallows (Tachycineta bicolor) and eastern bluebirds (Sialia sialis) occupying wetland and non-wetland habitats to test the hypothesis that wetlands are mercury hotspots for songbirds. Adult tree swallows in wetlands had significantly higher blood mercury concentrations than those in non-wetlands; however, no difference between ecosystems was detected in eastern bluebirds. Second, we present a review of the current literature on mercury in songbirds in wetland ecosystems across North America. Mercury concentrations in songbirds varied among wetland types and with geographic location; songbird blood varied 3–10 fold at locations separated only by ~10 to several hundred kilometers. This magnitude of difference in blood mercury concentrations among wetlands exceeds documented differences between wetland and non-wetland ecosystems. Therefore, we caution against the unequivocal assumption that wetlands are hotspots for mercury bioaccumulation in songbirds.}

\normaltalk{\capitalisewords{Do High-Energy Lifestyles in Birds Promote High Maximal Metabolic Capacities?}}{\underline{David L Swanson}, Nathan E Thomas, Yufeng Zhang}{Life-history traits related to pace of life may influence daily energy expenditure in birds and can produce fitness effects under varying environmental or ecological contexts. Energetically expensive lifestyles are hypothesized to be positively correlated with high metabolic demands and elevated basal metabolic rates. Because basal and maximal metabolic capacities are generally positively correlated across bird species, it might be expected that high-energy lifestyles would also result in elevated maximal metabolic capacities. We tested whether an energetically expensive lifestyle is associated with elevated metabolic rates in swallows and shorebirds, two bird groups with energetically expensive lifestyles. We measured basal (BMR) and summit (Msum = maximum cold-induced metabolic rate) metabolic rates in six species of swallows and two species of shorebirds and compared these data with literature data for other birds to address the hypothesis that swallows and shorebirds have both higher BMR and Msum than other birds. BMR in swallows was higher than that in other birds after controlling for phylogeny, but Msum for swallows did not differ significantly from that for other birds. Similarly, shorebirds in this study also showed high BMR relative to other birds, but not higher Msum. These data suggest that a high BMR may be required to support high-energy lifestyles in birds, but that basal and maximum metabolic capacities are modulated independently such that a high metabolic capacity is not a trait coupled to a high-energy lifestyle.}

\normaltalk{\capitalisewords{Gyrfalcon Dietary Plasticity in a Changing Tundra Ecosystem}}{\underline{Devin L Johnson}, Michael Henderson, David L Anderson, Travis Booms, Bryce Robinson, Cory T Williams}{Phenological shifts occur at different rates across taxa in response to climate change, leading to mismatch between linked trophic levels. The degree to which specialist predators can adjust their foraging strategy to mitigate the negative impact of phenological mismatch remains unclear. We studied a population of sub-Arctic breeding Gyrfalcons –traditionally considered a ptarmigan specialist– to quantify patterns in reproductive success and dietary strategy as a function of reproductive phenology. We installed motion-activated cameras during the brood-rearing period on Alaska's Seward Peninsula, collecting dietary, morphometric, and phenological data for 48 breeding attempts from 2014 – 2018. Gyrfalcon productivity positively correlated with earlier reproduction. Diet shifted from primarily ptarmigan and shorebirds during the early breeding season to Arctic ground squirrels and passerines later in the breeding season, consistent with seasonal patterns in prey availability. Nestlings at late-breeding nests consumed higher percentages of atypical prey (e.g. jaegers, gulls, and seabirds). Our results suggest improved nestling body condition for both early-breeding dietary specialists and late-breeding dietary generalists. Thus, earlier breeding individuals are better timed with their peak optimal prey availability (on which they may specialize), whereas late breeding individuals compensate for mismatch between hatch date and prey availability by increasing diet breadth at little detriment to nestling body condition. This behavioral adjustment to a more generalist diet may mitigate the adverse effects of phenological mismatch, and dietary plasticity may thereby serve as a mechanism of resilience to climate change.}

\normaltalk{\capitalisewords{A Genomic Investigation of the Potential Despeciation of Deeply Divergent Lineages in a Malagasy Passerine}}{Nicholas L Perry, \underline{Nicholas L Block}, Jane L Younger, Marie Jeanne Raherilalao, Steven M Goodman, Sushma Reddy}{Despeciation, sometimes referred to as reverse speciation, is the process of previously distinct evolutionary lineages merging into one due to an increase in gene flow. On small evolutionary time scales, this process is likely common, but it has rarely been documented in lineages of appreciable age. Using data from mitochondrial DNA, microsatellites, and host-specific parasite mitochondrial DNA, Block et al. (2015) reported a potential case of despeciation among four mitochondrial lineages of a Malagasy passerine, Xanthomixis zosterops, that were up to ~3.6 million years old. In this study, our goal was to augment the limited microsatellite data with genomic-level sequencing to assess whether a much larger nuclear DNA dataset would support the microsatellite conclusion of panmixia among these deeply divergent mitochondrial lineages. To accomplish this, we sequenced over 4000 ultraconserved element (UCE) loci for 47 samples of X. zosterops that spanned the geographic and phylogenetic range of the species. To determine if any genetic structure existed among the samples, we analyzed the resulting dataset using a variety of statistical methods, including principal component analysis and Bayesian clustering approaches. Analyses supported the conclusion that X. zosterops is best treated as a single population (K=1), despite the high level of divergence among mitochondrial lineages.}

\normaltalk{\capitalisewords{Cascading Effects of Widespread Disease-Induced Mortality of a Foundation Tree Species on Hawaiian Forest Bird Communities}}{\underline{Kristina L Paxton}, Patrick J Hart}{Few studies have quantitatively documented how ecological communities in terrestrial ecosystems respond to large-scale loss of a foundation species that creates stable conditions for other species and modulates ecosystem processes. The recent widespread changes in forest structure across the Hawaiian landscape as a result of disease-induced mortality of '\={O}hi'a lehua (Metrosideros polymorpha), a foundation tree species in Hawaiian forests, poses a serious threat to the structure and function of Hawai'i's remaining native forests and reliant bird communities. We used advances in recording technology and soundscape analysis tools to assess changes in bird communities in forests with different levels of '\={O}hi'a mortality. In addition, we evaluated whether the diversity and composition of understory plant species moderates how bird communities respond to structural changes in the forest canopy. In forests with high levels of '\={O}hi'a mortality, we predicted a community collapse of both native and non-native birds, given the foundational role of '\={O}hi'a in stabilizing forest conditions and providing nesting substrate and food resources for birds. However, high understory plant diversity may help to buffer the negative effects of '\={O}hi'a mortality, and result in only a loss of specific species most reliant on '\={O}hi'a. Alternatively, we predicted within forests with high '\={O}hi'a mortality and a low diversity of understory plants a total turnover of bird communities as the forest becomes more favorable for non-native open habitat bird species. Quantifying how dependent bird communities respond to '\={O}hi'a mortality will allow us to understand what factors are the most influential in stabilizing reliant communities.}

\normaltalk{\capitalisewords{Wedge-Tailed Shearwater Persistence in Human-Dominated Areas}}{\underline{Jessica L Idle}, Brooke M Friswold, Kristen C Harmon, Melissa R Price}{Management actions on O'ahu's offshore islets have resulted in an increase of Wedge-tailed Shearwater (Ardenna pacifica, WTSH) populations to the point that new WTSH colonies have formed on coastal areas of O'ahu that primarily support human activities. WTSH colonies residing in human-dominated areas may have lower fledging success than colonies occurring in restricted access areas. The objectives of this study were to: (1) compare fledging success between WTSH colonies at Kailua Beach Park (KBP), a human-dominated area, and Marine Corps Base Hawai'i - K\={a}ne'ohe Bay (MCBH-KB), a restricted access area; (2) determine the carrying capacity of the KBP colony; and (3) determine the population viability of the KBP colony. Nest monitoring and human presence surveys were conducted weekly at both sites. Carrying capacity was estimated for the KBP colony using offshore islet WTSH density estimates, and Vortex was used to perform a population viability analysis (PVA). The number of human detections was significantly different between sites (p$<$0.001) but no significant difference in fledging success was observed (p=0.38). Depending on the level of predator control and habitat restoration, it is predicted that the KBP colony will reach a population size of between 94 and 750 individuals.  Under current conditions, the KBP colony was predicted to likely persist for the next 100 years. Existing colonies on O'ahu will likely expand over time until carrying capacity is reached, resulting in a need for management strategies of native seabirds in coastal regions with a high degree of human presence.}

\normaltalk{\capitalisewords{If You Build It, They Will Come: Avian Utilization and Predation at a Large-Scale Photovoltaic Solar Facility}}{\underline{Jessica L Griffiths}, Daniel E Meade, Jason D Dart}{Rapid expansion of renewable energy infrastructure in the western United States has raised concerns about the effect of these facilities on bird populations, but there is currently little available information about avian utilization of and fatalities at large-scale solar facilities.  Topaz Solar Farms (TSF) is a 3,500-acre photovoltaic solar energy facility (550-megawatt) located in the Carrizo Plain in San Luis Obispo County, California.  We conducted point counts and fatality searches at TSF during construction (2012-2014) and for three years post-construction (2015-2017) in onsite areas (solar panel arrays, powerlines, and substation) and adjacent offsite grassland areas.  We analyzed avian utilization, species richness, and fatalities (average number of bird carcasses detected per search plot).  Avian utilization did not differ significantly between on and offsite areas overall but was significantly higher onsite than offsite in dry summer months, while species richness was significantly higher onsite.  These results could be due to the presence of structural components onsite that provide shade, shelter, and perches which were lacking in the area prior to facility development.  Avian fatalities were highest around overhead powerlines, though very few fatalities (1\%) could be attributed to project-related causes (e.g. collision).  Instead, predation was the most common cause of mortality.  Raptor utilization was highest in powerline areas, indicating that powerlines may be causing indirect avian mortality by providing perches for raptors and increasing raptor presence onsite.   The design of future large-scale solar facilities should take into consideration the potential effects of attracting raptors and consider potential solutions.}

\normaltalk{\capitalisewords{Year-Round Monitoring Reveals a Unique Seasonal Pattern of Bird-Window Collisions along the Pacific Coast of North America}}{\underline{Krista L De Groot}, Alison N Porter, Andrea R Norris, Andrew C Huang}{Collisions with windows are one of the leading causes of direct anthropogenic mortality in birds.  Published studies on factors affecting bird-window collision rates, seasonal differences, and species vulnerability rely heavily on data collected in eastern North America.  This is problematic because there are considerable differences in land cover, topography and bird movement patterns in western North America compared to the east, and there are limited data on vulnerability of western North American bird species and populations to the threat of window collisions.  We monitored a stratified random sample of 8 buildings at a Pacific coastal university campus in British Columbia, Canada.  Collision monitoring was conducted across each of the four seasons, for periods of 45 days. We examined the effect of season, building features, and local vegetation characteristics on frequency of bird-window collisions.  The variables that explained the greatest amount of variation in the number of collisions were: season, facade width, and the height of reflections of vegetation. Our results also revealed a unique seasonal pattern of collision rates and high frequency of bird-window collisions for the Varied Thrush; a western North American landbird currently experiencing steep population declines across most of its range.}

\normaltalk{\capitalisewords{Evaluating Bird Feeders as Tools to Connect People with Nature, and Challenges Associated with Middle School Interventions}}{\underline{Ruby L Hammond}, Tad C Theimer}{Urban living creates disconnect between people and nature, and there is growing concern that this disconnect leads to apathy toward wildlife conservation. Because "connection to nature" (CTN; measured using scientifically-validated questionnaires) is positively correlated with pro-environmental behavior, we tested the efficacy of bird feeders as tools to increase CTN in 6th graders and their parents at two schools in AZ and TN. We visited seven classrooms that did not get feeders (control), and twelve that received feeders and a one-month supply of seed (treatment). We administered CTN questionnaires (9-pt Likert) during classroom visits, and another questionnaire one month later. Students and parents who previously owned a feeder had significantly greater CTN than those who did not (t75=-3.8, p$<$0.001, student=7.0±0.2SE vs. 6.1±0.1SE, respectively; t96=-2.0, p=0.05, parent=7.0±0.2SE vs. 6.6±0.1SE, respectively). For participants not previously owning a feeder, results of mixed effects repeated measures models suggested CTN in all students did not change (n=149, p=0.21), while treatment parents' CTN increased significantly by ~0.3 Likert points more than for control parents (n=81, p=0.02). Self-assessments revealed that all students did not believe they increased CTN (F2=1.7, p=0.20) while treatment parents believed they (F2=5.9, p=0.004;  control vs. treatment=4.5±0.4SE vs. 6.2±0.3SE) and their child (F2=6.2, p=0.003;  control vs. treatment=4.6±0.4SE vs. 6.5±0.4SE) increased CTN. Only 30\% of enrolled adults completed the entire study, limiting the power of our analyses. We describe challenges of this type of research, and conclude that teacher enthusiasm and the ability to distribute surveys to parents electronically would improve retention in similar studies.}

\normaltalk{\capitalisewords{Long-Term Data Show Shifts in Urban Bird Communities by Land Use Type in Arizona}}{Daniel C Allen, \underline{Heather L Bateman}, Paige S Warren}{Long-term studies are needed to assess the potential for species persistence in the face of expanding urbanization. We investigated trends in bird data from an urban LTER (long-term ecology research) project to explore how communities of urban birds changed over time at the regional, landscape, and local spatial scales. At 43 sites during winter and spring from 2001-2002 and 2004-2016 birds were surveyed using 10-minute point counts. Sites occurred in five land use categories (agriculture, desert, desert-riparian, urban, urban-riparian). We examined changes in abundance related to functional group or guild. We used ordinations to assess differences in communities and linear regressions to investigate guild abundance change over time. Winter bird diversity (gamma) was consistent over time; however, spring diversity decreased. Trends of bird abundances showed interesting patterns of winners and losers by land use and traits. For example, winter nonbreeding migratory species decreased in desert, urban, and urban riparian areas. Summer migratory breeders were stable and increased in urban areas. Birds feeding on plants generally increased in urban and urban-riparian areas; whereas, predators decreased in desert-riparian and urban-riparian areas. Nest parasites (cowbirds) decreased in deserts and increased in urban-riparian areas. Overall, desert-riparian areas seems to have the fewest changes in bird communities, and urban-riparian areas and urban areas experienced the most shifts (seeing both increases and decreases in specific bird guilds). Bird species needing water and tree habitats appear to be losing ground in desert and urban areas and increasing in agriculture areas.}

\normaltalk{\capitalisewords{Connecting the Dots: Identifying Melanistic Pathways in a Polymorphic Raptor, Buteo Jamaicensis Harlani}}{\underline{Jacqueline L Barry}}{The Red-tailed Hawk (Buteo jamaicensis) is one of the most widespread raptors, breeding throughout North and Central America and exhibiting extensive phenotypic variation across a range of geographic landscapes. One subspecies in particular, Harlan's Hawk (B.j. harlani), displays the most extreme plumage variation, despite documented interbreeding with other subspecies. Harlan's Hawk occurs in light, intermediate, and dark morphs, with the dark morph is far more common, and has a multi-colored, variably marked tail, a trait unique to Harlan's Hawk alone. Genetic analyses have attempted to resolve the evolutionary relationships of Red-tailed Hawk subspecies, including Harlan's Hawk, but the genetic mechanism(s) driving melanistic plumage in this bird remains unknown. Whereas the melanocortin-1 receptor (MC1R) has been associated with plumage variation in other birds, no such correlation has been observed in Harlan's Hawk. Here, I test the associations of several candidate genes known to regulate melanin with the high phenotypic variation observed in Red-tailed Hawk plumage. MC1R, CORIN, and TYRP1 were not associated with melanistic variation in B.j. harlani, B.j. calurus and B.j. borealis. ASIP, POMC, and SLC24A5 are currently under study.}

\normaltalk{\capitalisewords{Synthesis of Migration Ecology in the Gulf of Mexico Guides Species Conservation}}{\underline{Jill L Deppe}, Thomas J Benson, Rachel Bolus, Jeffrey Buler, Antonio Celis Murillo, Robert Diehl, Richard Feldman, Erika Knight, Gary Langham, David Mehlman, Nicole Michel, Frank Moore}{The Gulf of Mexico is a geographic region of high risk for migratory songbirds due to elevated energetic demands, potential for unfavorable weather over water, and degraded habitat quality. Nine years of capture and automated radio telemetry data for Swainson's and Wood Thrushes during fall migration in coastal Alabama and the northern Yucatan Peninsula documented the critical effects of fat and weather on migratory route, stopover behavior, and survival. Birds with at least moderate fat reserves and tailwinds crossed the Gulf with high survival, and successfully compensated for unfavorable winds by departing with more fat. In contrast, birds with little fat reoriented inland toward higher quality bottomland forests within one day of capture. Short stopovers, low fruit abundance, mass loss, and reorientation all signal poor coastal habitat quality. Swainson's Thrushes captured on the Yucatan Peninsula coast had little fat. At sites with high fruit abundance, they stayed for an average of four days, gained mass, and departed southward, indicating favorable habitat quality. These results strongly support the importance of high quality habitats for successful migration through the Gulf region, and they underscore the value of native plant management and/or restoration for frugivorous species in coastal landscapes, including nearby inland habitats. We integrated thrush occurrence and tracking data with information on threats to prioritize sites for targeted conservation campaigns. Using information on existing conservation efforts and protected land status, we identified specific actions, e.g., planting native species in yards and restoring public lands, to advance conservation of migratory birds.}

\normaltalk{\capitalisewords{Population Genetics and Phylogeography of Merlin (Falco Columbarius) in North America}}{\underline{Breanna L Martinico}, George K Sage, Meg C Gravley, Sandra L Talbot, Bruce A Haak, Joshua M Hull}{Understanding the genetic consequences of contemporary and historical processes is important in light of anthropogenic environmental change. In the mid-20th century, many raptor species experienced population declines due to the widespread use of the pesticide DDT, an organochlorine compound with long-lived metabolites. Currently, little is known about the genetic impacts on many species that faced population bottlenecks in the DDT-era but received little conservation attention. We assessed the genetic impacts of the DDT-era population bottleneck in North American Merlins (Falco columbarius) and the impacts on population structure associated with historic Pleistocene glaciation. We sampled migrating Merlins across North America (n=272) and genotyped individuals at 23 polymorphic microsatellite loci and along a 569-base pair segment of the control region of the mitochondrial genome. We used hierarchical analysis of molecular variance, pairwise Fst/phist comparisons, and Baysean clustering to detect genetic differentiation between eastern and western groups, individuals that use distinct migratory flyways, and among three recognized subspecies. Across all analyses, we found strong support for one panmictic population of Merlin in North America, as there was little to no population differentiation detected; $>$98\% of variation was due to differences within populations rather than among populations or groups. Furthermore, we did not detect a genetic signal from the DDT-era population bottleneck. We found a pattern of two common control region haplotypes present across all regions, suggesting a complex demographic history of Merlin in North America, unique among other cosmopolitan raptor species.}

\normaltalk{\capitalisewords{Simultaneous Rectrix Molt and Comparative Molt Ecology of Eastern Wood-Warblers}}{\underline{Ronald L Mumme}, Robert S Mulvihill}{Previous work has shown than Hooded Warblers (Setophaga citrina) molt all 12 rectrices simultaneously during primary molt, and that most adults initiating rectrix molt before the end of parental care desert their late-season nestlings and fledglings, leaving their mate to provide all remaining parental care. The purpose of our current study was to determine if simultaneous rectrix molt is unique to Hooded Warblers or if it is widespread in other North American parulids. We examined molt data collected from 1287 individuals representing 13 warbler species and 7 genera captured at Powdermill Avian Research Center in western Pennsylvania during late summer and early fall, 1986-2000. We found that all 13 species replace their rectrices more-or-less simultaneously, and that the synchronicity of rectrix loss in Hooded Warblers is comparable to that of other parulids. Simultaneous rectrix molt begins at the time the 5th primary is dropped in Hooded Warblers and other Setophaga, and when the 6th primary is dropped in Seiurus, Parkesia, Vermivora, Mniotilta, Geothlypis, and Cardellina. The 13 species show small differences in timing and speed of molt, with single-brooded species generally molting earlier and more quickly than double-brooded species. Across all species, males generally molt earlier than females. We conclude that (1) simultaneous rectrix replacement during primary molt is an evolutionarily conserved molt strategy that is widespread in migratory parulids that breed in North America, and (2) tradeoffs between parental care and molt are likely to occur whenever temporal overlap occurs between these two demanding phases of the annual cycle.}

\normaltalk{\capitalisewords{Seasonal and Directional Dispersal Behavior in an Ongoing Dove Invasion}}{\underline{David L Slager}}{The dispersal behavior that underlies range expansions can be difficult to study. Eurasian Collared-Doves have staged independent northwestward invasions across both Europe and North America, reaching carrying capacity in Europe but continuing to increase exponentially in the Americas, where their dispersal behavior remains enigmatic. I used aggregated behavioral observations from eBird to investigate Eurasian Collared-Dove dispersal behavior along the North American Pacific coast, a natural barrier to landbird dispersal. Using coastal and pelagic observations, I assessed the seasonality and directionality of dispersal and its prevalence across years. Dispersal flights peaked in spring, like in Europe, and were strongly north-biased, consistent with northwestward initial colonization of North America and northward deflection of northwest-bound birds at the Pacific Ocean. A non-significant trend of increasing dispersal across years may reflect evolution of dispersal via spatial sorting and selection. These results inform full-life-cycle modeling and management of this invasive dove, raise new questions about evolutionary mechanisms behind the invasion, and exemplify a novel approach to using citizen science to study dispersal behavior, a longstanding challenge in ecology.}

\normaltalk{\capitalisewords{A Burning Question: Does Personality Predict How Pre-Breeding Florida Scrub-Jays Explore Habitat in a Fire History Mosaic?}}{\underline{David L Sherer}, Patrick J Bohlen, J R Bowman}{Florida Scrub-Jays (FLSJ; federally Threatened) are cooperatively breeding habitat specialists and short-distance dispersers endemic to fire-maintained Florida scrub. Lifetime reproductive success is highest among individuals breeding in early-successional habitat. However, such habitat is limited and competition for it as breeding space is likely intense. Although survival and fecundity decline in later-successional habitats, some birds settle there, perhaps to avoid competition.  We investigated how individual behavioral phenotype (personality; bold-shy), natal habitat quality, relative body condition, and relative territory location (core vs. periphery) influence extra-territorial foray behavior across a mosaic of habitat in various time-since-fire (TSF) classes. Our study system was a population of individually-marked FLSJs at Archbold Biological Station in Highlands County, Florida. We hypothesized that exploration by pre-breeding FLSJs in habitat with heterogenous fire history is assortative to reduce competition, and to potentially match habitat to their suite of behavioral attributes or natal territory characteristics. We recorded identities of individual FLSJs engaged in pre-dispersal forays across a sampling grid during three breeding seasons. We compared TSF of explored vs. unexplored habitat to classify preference and conducted neophobia assays in the field to quantify personality in yearling birds. We paired these data with fire history and morphometric records to create and compare models of exploratory behavior.  Our results indicate significant variation in habitat preference by individual FLSJs during forays. We will discuss mechanisms – including personality – driving this variation, which potentially reduce intraspecific competition and improve individual fitness among FLSJs.}

\normaltalk{\capitalisewords{Genomic and Phenotypic Divergence in Red Warblers (Cardellina Rubra)}}{\underline{Whitney L Tsai}, Emma White, James M Maley, John E McCormack}{The montane, pine-oak forests of the Mexican highlands are both highly biodiverse and highly threatened by deforestation. High rates of diversification and endemism in this region suggest the potential of previously undetected genetic lineages. Red Warblers (Cardellina rubra) are one of the many species endemic to the Mexican Highlands. There are three subspecies that inhabit disjunct montane regions: C. r. melanauris in the Sierra Madre Occidental, C. r. rubra in the Transvolcanic Belt, and C. r rowleyi in the Sierra Madre del Sur. In addition to geographic separation, variation in plumage coloration is documented across the subspecies. C. r. melanauris exhibits a gray auricular patch whereas the southern subspecies both exhibit a white auricular patch. Furthermore, C. r. rowleyi has brighter red plumage than the other two subspecies. We assessed phenotypic and genomic differentiation in C. rubra by analyzing morphology, coloration, and genomic markers. We found significant differences in morphology and plumage color between C. r. melanauris and the two southern subspecies. We also found strong evidence for genomic structure between subspecies. Our work provides insights into the evolutionary history of birds in the Mexican highlands.}

\normaltalk{\capitalisewords{Resolving Difficult Nodes in Manakin Phylogeny using Low-Coverage Whole Genome Sequencing}}{\underline{Edward L Braun}, Noor D White, Rebecca T Kimball, Andre E Moncrieff, Min Zhao, Sarah M Kurtis, Robb T Brumfield}{Phylogenetic relationships among many groups of birds have proven surprisingly difficult to resolve in a convincing manner, even with large datasets. There are clades that are challenging to resolve in many groups of birds, including the manakins (Pipridae), a family of Neotropical suboscines. We targeted two clades within manakins, the genus Lepidothrix and the clade comprising the genera Chiroxiphia and Antilophia, for whole genome sequencing. We used an inexpensive approach that combined sequencing a single Illumina library for each taxon to a depth of 5-20x followed by read mapping. We achieved $>$95\% coverage of typical single copy regions by mapping reads to those regions as long as we had at least 10x coverage of the genome for each taxon. This makes it possible to extract and analyze a broad set of phylogenetic markers, including introns, exons, ultra-conserved elements or other regions. We assessed the amount of discordance among gene trees to understand why these nodes have remained challenging and we explored the impact of data type on estimates of phylogeny. This allowed us to establish the best-corroborated phylogenetic trees for these clades.}

\normaltalk{\capitalisewords{Comparative Genomics Reveal Modes of Differentiation in North American Warm Desert Birds}}{\underline{Kaiya L Provost}, Stephanie Shue, Meghan Forcellati, Brian T Smith}{The avifaunas of the Sonoran and Chihuahuan deserts harbor many of the same species, but geographic variation within species suggests that populations have started to differentiate among the deserts. The mechanisms causing this early phase of diversification range from local adaptation to genetic drift in allopatry. Here, we investigate whether local adaptation linked to variation in temperature and precipitation regimes may have impacted differentiation by integrating genomic and morphological data for 10 bird species. Under this framework, we predict that the degree of morphological variation among deserts is correlated with the frequency of genomic regions under selection across species. Using re-sequenced genomes and phenotypic measurements from museum specimens, we characterized genetic variation, identified highly differentiated portions of the genome (i.e., genomic islands of speciation), and quantified extremity size, a trait important in thermoregulation in desert environments. We find that variation in genomic structure among deserts ranges from no structure to isolation with gene flow, and that genomic islands are frequently observed across species. We also find that in a subset of the species, there are significant differences in sizes of the extremities among desert populations. Species with size differences were often found to have numerous loci under selection. By leveraging integrative data from whole genomes and morphology, we show that the birds in the North American warm deserts display a myriad of genetic and phenotypic patterns across the environmental gradient, and that these patterns likely resulted from both neutral and adaptive modes of diversification.}

\normaltalk{\capitalisewords{Soaring to New Heights: Analytical Approaches for Cumulative Effects Assessment for Migratory Birds}}{Shawna A Pelech, \underline{Carmen L Mahon}, Nathalie Lowry}{Landscapes throughout much of Canada and North America have been subject to an increasing amount and combination of human-induced disturbances (stressors) over the past decades. Understanding and mitigating the effects of multiple stressors requires approaches that examine the combined or cumulative effect of multiple stressors or sectors on ecosystem components [i.e. cumulative effect assessment (CEA)]. Most literature and guidance documents for CEA focus on project-based CEA and specific scientific guidance within such documents (e.g. on analytical approaches) is often lacking or incomplete. Our purpose was to review and provide scientific guidance on the range of analytical approaches to CEA particularly where migratory birds have been selected as a valued component. We first describe the four general categories of analytical approaches to CEA: (1) qualitative review; (2) habitat supply; (3) empirical species-stressor modeling; and (4) expert-based modeling (Bayesian Belief Networks and fuzzy modeling). We then outline a scientific framework for an empirical species-stressor modeling approach to CEA for migratory birds as part of a larger regional or effects-based CEA.}

\normaltalk{\capitalisewords{Additive and Interactive Cumulative Effects on Boreal Landbirds: Winners and Losers in a Multi-Stressor Landscape}}{\underline{Carmen L Mahon}, Gillian L Holloway, Erin M Bayne, Judith D Toms}{Stressors created by multiple resource industries can result in cumulative effects over time and space. We evaluated whether the effects of multiple stressors in the boreal forest (linear features, energy, forestry) combine additively or interactively by testing a candidate model set of 12 cumulative effects models of abundance for 27 landbird species. We fitted paired additive and interactive Generalized Additive Models and examined model predictions in the actual current and a theoretical no-disturbance study area. Of the 27 species examined, an additive disturbance model was the best for nine species, while an interactive disturbance model was the best for 11 species. In the current study area, disturbance models predicted strong increases in abundance for species associated with deciduous forest and open habitats (winning species) and moderate decreases for species associated with conifer forest (losing species). We found a 15\% change in landbird community composition between the current study area, with 8.4\% disturbance, and the theoretical no-disturbance study area. Complex synergistic and antagonistic interactions among stressors were observed for 39\% of landbird species, with the majority of interactions observed being synergistic. Stressors with relatively small disturbance areas, such as narrow linear disturbances, frequently interacted with other stressors to affect species' responses, and energy sector stressors often had additive or interactive effects with forestry stressors. Interactive cumulative effects from multiple sectors will make it increasingly difficult to manage impacts unless interactions among stressors are incorporated into cumulative effects assessments and regional land use planning processes.}

\normaltalk{\capitalisewords{Declines in Eastern Canada Piping Plovers: Are Overwintering Grounds the Key to Conservation?}}{\underline{Cheri L Gratto-Trevor}, Jen Rock, François Shaffer}{Overall, management activities on the breeding grounds aimed at improving productivity have increased population numbers of a subspecies at risk, Atlantic Piping Plovers (Charadrius melodus melodus). However, numbers in Eastern Canada (NS, NB, PE, NL, QC) have generally decreased, even though productivity there has normally been greater than in areas where populations have increased in Atlantic US. The Eastern Canada population is currently less than 200 pairs. A 2013 analysis suggests that, for Eastern Canada Piping Plovers, conditions in the nonbreeding season may be more important than productivity in affecting population change. Therefore, we initiated a study in 2014 to determine what factors most affect population change across the annual cycle, and to identify migratory staging sites and wintering areas, so that key sites and conservation issues at these sites can be identified and ultimately managed. We have marked 324 adults and 327 chicks with unique alphanumeric coded flags. For all adults marked before 2018, 65\% (157/242) have been resighted in the nonbreeding season, some in several seasons (29\% spring migration, 43\% fall, 54\% all migration, 31\% winter). Of those seen in winter, 25\% were in Atlantic US (NC to FL), 20\% in FL Gulf coast, and 55\% in the Caribbean (primarily the Bahamas and Turks and Caicos). Results are preliminary, but annual return rates from the different wintering areas have been quite variable, and we need to examine possible reasons for these differences.}

\normaltalk{\capitalisewords{Linking Demographic Rates to Population Trends in Six Species of Arctic-Breeding Shorebirds}}{\underline{Emily L Weiser}, Brett K Sandercock, Richard B Lanctot, Stephen C Brown,  Arctic Shorebird Demographics Network}{Like other long-distance migrants, shorebirds are difficult to track across the annual cycle, but a full portrayal of the factors affecting population status and trend is crucial for successful management. In 2010-2014, the Arctic Shorebird Demographics Network (ASDN) coordinated data collection at 16 field sites to develop estimates of annual adult survival rates and seasonal reproductive success. We used these and other demographic parameter estimates from the literature to develop the first range-wide population models for six species. We then used the models to identify the vital rates with the strongest influence on population trend, highlighting the rates that are not well known based on previous studies. The models also provided estimates of population trends that we compared with trends estimated from previous studies. Understanding the factors that affect the most influential vital rates will be crucial for predicting how Arctic-breeding shorebirds will respond to changes in environmental conditions and for determining where to focus conservation actions for declining species.}

\normaltalk{\capitalisewords{Vertebrate Control from Raptors Motivates Farmers to Install Nest Boxes, but is Secondary Poisoning a Concern?}}{\underline{Breanna L Martinico}, Sara M Kross, Josh M Hull, Ryan P Bourbour, Emily M Phillips, Elizabeth Hiroyasu}{In an effort to boost natural rodent pest control services provided by birds of prey, farmers across Western North America have recently begun installing artificial nest boxes for Barn Owls and Kestrels and artificial perches for hunting raptors. We conducted a survey of California farmers and found that 95\% of respondents believed that raptors were beneficial for vertebrate pest control, and 56\% of respondents believed that raptors were beneficial for crop yields. Our diet study indicated that Barn Owl diet in the region is comprised of over 99\% of agricultural pests, and our models reveal that in theory, Barn Owls are capable of keeping rodent populations below an environmental capacity, and in some circumstances, could drive pest populations to near-zero. However, because many farmers also utilize rodenticides to control rodent pests, raptors can suffer from both lethal and sub-lethal secondary poisoning. We have little understanding of how often raptors are exposed to rodenticides in an agricultural setting, especially at a sub-lethal level, and what effect this exposure has on their behavior and reproductive success. We recently started a 3-year study to quantify the frequency and effects of secondary poisoning in Barn Owls breeding on farms by monitoring chick growth, diet, provisioning, and foraging ecology in relation to rodenticide exposure. Preliminary results show little to no rodenticide exposure in Barn Owls in winter at sites where rodenticides are applied near buildings (but not at the field-scale) and adds to existing diet data showing consumption of exclusively agricultural pests.}

\normaltalk{\capitalisewords{Female Preferences Shape Elaborate Male Displays and Social Skills on the Lek}}{\underline{Gail L Patricelli}}{Over evolutionary time, female mating preferences may select for elaborate display traits in males, especially in lekking species.  Over a shorter timescale, during courtship, female preferences may also select for males who adjust the presentation and performance of their displays in response to the subtle signals and cues females produce.  In this way, sexual selection by female choice may favor males with both the ability to produce elaborate displays and the "social skills" required to adjust them appropriately during courtship. I will present examples from studies on satin bowerbirds (Ptilonorhynchus violaceus) and greater sage-grouse (Centrocercus urophasianus) where my collaborators and I have used robotic females to study courtship interactions in the wild, measuring the degree to which males respond to female signals, and how these responses affect male mating success. Using these examples, I will discuss the role of sexual selection in shaping social skills, and the importance of examining female behaviors during courtship, even in species where the male displays are much, much flashier.}

\normaltalk{\capitalisewords{Model-Based Vulnerability Assessment of Boreal Birds in a Changing Climate}}{\underline{Brooke L Bateman}, Joanna Wu, Chad Wilsey, Lotem Taylor, Geoff LeBaron, Gary Langham}{Climate change threatens wildlife and habitats globally, with some regions, such as the boreal forests of North America, expected to see dramatic changes in the coming decades. We developed species distribution models at the North American continental level for 115 boreal forest bird species in summer and winter, incorporating both climate and habitat change at a 1-km scale. We then assessed the threat of climate change for each species across its range using a model-based vulnerability assessment based on projected range loss and gain. Overall, boreal forest bird species are highly vulnerable to climate change, with 90\% of species assigned to the moderate or high vulnerability classifications in a 3.0 C warming scenario, and 70\% under a 1.5 C scenario. For most species, projected changes in climate and expansion of grasslands and deciduous forests into the region radically reduce the amount of suitable habitat in the southern part of their current range. Particularly in summer, climate change will likely push boreal forest birds beyond the northern limit of their range, which will be constrained by the availability and pace of expansion of boreal forest as it transitions to tundra. Given the high vulnerability of this group to climate and habitat changes, we need a combination of both climate and on the ground management and monitoring actions. We will highlight a community science-monitoring program, Climate Watch, as a way to understand bird range shifts in the boreal forest, and to inform climate change focused bird management.}

\normaltalk{\capitalisewords{Estimating Avian Abundance from Unmarked Populations: Requirements, Assumptions, and Finding the Best Method for your Research}}{\underline{Courtney L Amundson}, Quresh Latif, Jonathan Valente}{For over 50 years, point counts have been a primary method used for estimating avian abundance and population trends for unmarked bird populations.  Nevertheless, interpreting raw counts is problematic.  Most importantly, not all birds in an area are detected during surveys; thus, counts underestimate true population size. Second, study designs for passive surveys frequently ignore critical aspects of avian behavior and ecology leading to biased inference.  Third, unadjusted counts may not adequately address variability in counts through space and time.   These issues have led to an explosion of analytical approaches to improve count-based abundance estimates. Furthermore, methods have expanded from using discrete point or transect-based counts for primarily single-species abundance estimates to quantifying population vital rates, avian community dynamics, and as one data component of integrated population models.   Recent statistical approaches are dominated by hierarchical models that provide avian abundance estimates adjusted for imperfect detection.  While powerful, these methods vary in detection probability components and population attributes estimated (e.g., effective sampling area and superpopulation size) depending on the study design, field techniques, and models employed. Moreover, emerging techniques each have unique assumptions that are too often overlooked or misapplied.  We provide an overview of critical considerations for detection and enumeration of birds from count-based surveys and describe currently available methods for addressing problems inherent to estimating population size using count data, providing context for specific approaches presented throughout the day.}

\normaltalk{\capitalisewords{Optimizing Aerial Seabird Survey Design: Pitfalls and Progress}}{\underline{Kayla L Davis}, Matthew T Farr, Randy R Wilson, Emily D Silverman, Allison Sussman, James E Lyons, Elise F Zipkin}{Characterizing seabird species' composition, distribution, and abundance is necessary for assessing the impacts of changing environmental conditions and for guiding management practices. However, estimating seabird trends is particularly challenging because it can be difficult to locate and count flocks in the vast marine environment. The typical study design consists of aerial surveys in which observers count individuals within flocks and attempt to identify seabird species from planes moving at 150–200 km/h at an altitude of 50–100 m. Inherent to this study design are issues not considered explicitly in subsequent analyses including imperfect observer detection, flock size estimation error, species misidentification, and constraints of the survey unit. We investigate these issues and the potential for biasing estimates with a case study examining seabirds in the Gulf of Mexico. As part of the ongoing Gulf of Mexico Marine Assessment Program for Protected Species (GoMMAPPS), seabird data are collected during aerial surveys with a double observer protocol. This protocol allowed us to compare records of species identification and flock detection and estimation from two, independent observers. We evaluated the extent and magnitude of observation errors and the potential for proliferation into subsequent analyses. Our results indicated that observers detect birds relatively consistently. However, important inconsistencies exist in flock counts (especially for flock sizes greater than 100 individuals) and species-level identification (particularly in mixed flocks). We propose suggestions for handling these issues, including modeling species groups instead of individual species and using an ordinal modeling approach to address potential discrepancies in flock sizes.}

\normaltalk{\capitalisewords{Scaling Up by Integrating Standardized and Unstandardized, Community Science Count Data: a Case Study in the Upper Mississippi River}}{\underline{Nicole L Michel}, Timothy D Meehan, Sarah P Saunders}{Bird monitoring designs have long faced a tradeoff between collecting rigorous standardized data at small spatiotemporal scales and collecting potentially less-informative nonstandardized data across large spatial and long temporal scales. Recent years have seen the development of a plethora of statistical methods for increasing accuracy and precision of occupancy/abundance estimates by correcting for imperfect detection. However, these methods require substantially greater effort either through collection of auxiliary data (e.g., distance, time of observation), use of multiple observers, or repeated surveys, and often rely on trained observers. As a result, such methods are rarely implemented at sufficiently large scales for inference on population-level trends and drivers. Alternatively, large-scale nonstandardized datasets, such as eBird, lack these ancillary data, making it difficult to correct for imperfect detection and robustly estimate occupancy or abundance. One potential solution stems from the recent development of integrated models that share information between standardized and nonstandardized data, leveraging the strengths of each dataset to improve inference across large spatial scales. We present a case study from bottomland forests of the Upper Mississippi River in which we integrate data collected through a standardized protocol (i.e., distance and time estimation) with nonstandardized data from eBird to estimate habitat suitability at a regional scale. We compare two methods of data integration, treating the nonstandardized data as either coequal or as a covariate, and evaluate differences in precision of parameter estimates. Data integration represents a powerful approach for improving the precision and applicability of information used to inform large-scale conservation and management.}

\normaltalk{\capitalisewords{Mechanisms Underlying the Maintenance of Species Boundaries: Lessons from Avian Hybrid Zones}}{\underline{Jennifer L Walsh}, Stepfanie M Aguillon, Gemma V Clucas}{Hybrid zones have played an important role in our understanding of evolution, speciation, and reproductive isolation, and have been the focus of ornithological research for decades. In fact, research in avian hybrid zones has long served as a baseline for formulating general theories about hybridization in non-avian taxa. In addition to this strong history in the scientific literature, new techniques are increasingly being applied to avian hybrid zones—from genomics to demographic modeling—to continue to advance our understanding of both old and new questions about avian speciation. In this symposium, we highlight ongoing work by scientists at the forefront of avian hybrid zone research who will present empirical examples about the mechanisms influencing the maintenance of species boundaries in birds. Our speakers will discuss several mechanisms that can drive hybridization, including genomic and behavioral processes, as well as habitat associations. In this introductory talk, we will give a summary of the processes that can facilitate or inhibit hybridization in birds as well as an overview of new technologies and methodological approaches that are integral to our expanding view and understanding of avian hybridization.}

\normaltalk{\capitalisewords{Population Increases Despite Habitat Saturation: What can We Learn from Five Decades of Research?}}{\underline{Eric L Walters}, Natasha DG Hagemeyer, Sahas Barve, Mario B Pesendorfer, Walter D Koenig}{The population of cooperatively breeding acorn woodpeckers (Melanerpes formicivorus) at Hastings Reserve, Carmel Valley, California has been under continuous study since 1968. Methodological advances have allowed us to address new research questions and reassess some of our previous findings. One of our most interesting findings is the observed population increase despite a reduction in the availability of preferred breeding habitat over the past three decades. The increase in population abundance is largely determined by the formation of new social groups, driven by an overall increase in canopy cover. Population growth appears to be determined by the addition of new groups rather than an increase in mean group size. The role of social behavior and our assessment of the relative contributions of breeders vs helpers has also changed with time. We originally viewed all helpers as having an overall positive effect on group reproduction. By coupling acorn production with demographic information, we determined that helper males, and not females, had an increasingly positive effect on reproductive output with increasing acorn crop. Teasing apart the mechanism by which only helper males increase productivity in good years is being addressed by detailed monitoring of individual movement patterns. We developed solar-powered nanotags that have been attached to the majority of the population to determine that many birds, including breeders, spend inordinate amounts of time off of their territories and have incredibly complex social networks. These findings would not have been possible without long-term monitoring, coupled with carefully designed experiments.}

\normaltalk{\capitalisewords{One State, Global Connections: Insights from 100 Years of Bird Banding in Alaska}}{\underline{Amy L Scarpignato}, Autumn-Lynn Harrison, Peter P Marra}{As a part of the Atlas of Migratory Connectivity of the Birds of North America, we synthesized 100 years of bird banding records of Alaskan birds from the USGS Bird Banding Laboratory Banding and Encounter database. Since 1921, over 200 banding permittees have banded 1.3 million individual birds of 280 species in Alaska. Of these, 54,000 were re-encountered outside of Alaska and four thousand birds banded outside of Alaska were re-encountered in Alaska, totaling over 180 species in 12 orders. Waterfowl comprised 98\% of encounter records. Records spanning breeding and wintering locations were available for 72 species, allowing synthesis of migratory connectivity patterns across species and flyways that we will present via density maps and connectivity networks. Alaskan banded birds were encountered in 23 countries, 41 U.S. states, and 7 Canadian provinces. The highest proportion of recoveries (51\%) were located along the west coast of the United States, followed by 14\% of recoveries from Mexico. Less than 1\% of all breeding to winter recoveries were from outside the Americas. Information on human impacts available in the encounter records indicated that the majority of birds recovered by non-natural means in wintering areas were shot or found entangled. Within Alaska, birds shot, entangled, or captured in traps or snares dominated recoveries. The long history of banding of Alaska's birds provides enormous value to our knowledge of migratory connectivity and species conservation in both the western and eastern hemispheres, demonstrating Alaska's unique position as a single U.S. state with global connections.}

\normaltalk{\capitalisewords{Connectivity of Pre-Adult, Non-Territorial Migratory Golden Eagles During the Nesting Season in Alaska}}{\underline{Carol L McIntyre}, Stephen B Lewis, Todd E Katzner, Tricia A Miller, Michael Lanzone, Michael W Collopy, David C Douglas}{Conservation of long-lived species requires protecting the areas they use throughout their life cycle. Golden Eagles (Aquila chrysaetos) are a long-lived species that exhibit deferred breeding, resulting in a long (i.e., $>$ three years) natal dispersal period. Natal dispersal of Golden Eagles is poorly documented. We know little about the areas they inhabit and the resources they use during this period. We have been quantifying the year-round movements of migratory Golden Eagles raised in Denali National Park and Preserve, Alaska since 1997 using satellite and GSM telemetry. Our study results are providing new information on movements during their natal dispersal period, including during the nesting season. The nesting season, from courtship through dispersal of young, for migratory Golden Eagles in interior and northern Alaska extends from mid- February through early October. During this time, territorial eagles remain on and near their territories. In contrast, pre-adult non-territorial eagles often exhibit nomadic behavior for part of the nesting season, moving thousands of kilometers across Alaska. However, many also settle temporarily (several weeks to several months) in specific areas, including in areas on the Arctic Coastal Plain in northern Alaska. The repeated use of the Arctic Coastal Plain by multiple telemetered eagles over years, suggests that portions of it support important, and perhaps critical, habitats and resources for Alaska's migratory Golden Eagles before they enter a breeding population. Conservation efforts need to consider the role of these settlement areas in the persistence Alaska's Golden Eagles.}

\normaltalk{\capitalisewords{Contrasting Patterns of Migratory Connectivity in Two Species of Alaska-Breeding Shorebirds}}{\underline{T L Tibbitts}, Robert Gill, Jr, Daniel Ruthrauff, Nils Warnock, Christopher Harwood, Autumn-Lynn Harrison, Michael Hallworth, David Douglas}{Describing migration systems of long-distance migratory shorebirds, including the degree of mixing and separation of breeding populations throughout the annual cycle (i.e., migratory connectivity), is important for understanding differential impacts of environmental change on populations. We used satellite telemetry to document the year-round movements of two species of shorebirds and gain insights into their movement ecology and migratory connectivity. We tracked 31 individual Whimbrels (Numenius phaeopus) from two Alaskan breeding areas (North Slope, Kanuti) in 2009 to 2013 for a total of 42 south- and 20 north-bound migrations. We tracked 20 individual Black-bellied Plovers (Pluvialis squatarola) from two breeding areas (North Slope, Seward Peninsula) in 2014 to 2018 for a total of 25 south- and 14 north-bound migrations. All Whimbrels migrated via the Pacific Flyway where the two populations mixed at staging areas in California and Mexico and overlapped on the nonbreeding range from Mexico to Chile. In contrast, plovers from the two breeding populations migrated separately. North Slope plovers took the Central Flyway to stage in prairie Canada and winter along the Gulf Coast and western Mexico whereas the Seward Peninsula plovers staged in western Alaska and traveled the Pacific Flyway to winter in The Californias, sometimes near North Slope birds. For both species, movement patterns of individuals persisted across years although migration timing and choice of stopover sites varied. We discuss how knowledge about migratory connectivity could inform conservation strategies for these breeding populations.}

\normaltalk{\capitalisewords{Is There an Association Between Communities of Wood Decay Fungi and Black-Capped Chickadee Nest Cavities?}}{\underline{Connor L Gable}, Alec R Lindsay, Michelle A Jusino}{Wood decay fungi soften dead hardwood and coniferous trees by degrading the main components of wood: lignin and cellulose.  Wood decay fungi thereby may diminish the amount of time and energy required for a cavity-nesting species to excavate a cavity, which is generally an energetically costly investment.  Here, we examine whether there is an association between communities of fungi, including putative decay fungi, and the nest cavities of black-capped chickadees (Poecile atricapillus).  We compared fungal communities in trees with chickadee nest cavities to nearby similar-quality trees without active chickadee nests.  We identified taxa of fungi present in our samples using high-throughput amplicon sequencing (HTS) of the internal transcribed spacer 2 (ITS2) region.  HTS allowed us to characterize more than 90 putative decay fungi in our samples, with both nest and non-nest trees containing highly diverse fungal communities.  We recorded physical characteristics of nest and non-nest trees to determine how other factors might influence nest-site selection.  We found that wood hardness, measured as specific gravity, was lower in trees containing nest cavities compared to control trees (0.25 vs 0.37, respectively; p$<$ 0.05).  The probability that a tree contained a nest cavity was significantly related to specific gravity (p$<$ 0.005).   We further describe the differences observed in wood decay fungi found in the nest trees.}

\normaltalk{\capitalisewords{Aspects of Municipalities Associated with Occupancy and Abundance of Chimney Swifts in Illinois}}{\underline{Maureen L Hurd}, Thomas J Benson, Michael P Ward}{Aerial insectivorous birds such as the Chimney Swift (Chaetura pelagica) are experiencing population declines across North America.  While causes of these declines are unknown, habitat availability is likely a contributing factor.  Chimney Swifts historically nested in tree cavities but switched to nesting in masonry chimneys as North American settlements expanded.  Available chimneys are disappearing with the growing trends of chimney caps and gas/electric heating, and the loss of available nest sites may be driving Chimney Swift declines.  We examined the influence of various habitat and landscape factors on Chimney Swift occupancy and abundance.  We conducted surveys in municipalities of varying size throughout Illinois and recorded the number of uncapped chimneys at each survey point.  We used these data along with landscape-level data such as composition of land cover surrounding points, areal extent of municipality, and age distribution of buildings.  Chimney Swifts occupied 97\% of municipalities (n = 126), 20\% of natural areas (n = 10), and 72.5\% of all sampled points (n = 476), but abundance varied considerably.  Swift abundance was greatest at the center of small, isolated municipalities where uncapped chimneys tended to be most prevalent.  We detected swifts at all points with at least five uncapped chimneys (n = 60), but only 21\% of points with no uncapped chimneys (n = 270).  Our results suggest that although Chimney Swifts are still widespread, contemporary building practices will continue to drive population declines and management should focus on approaches for providing and preserving suitable nesting sites.}

\normaltalk{\capitalisewords{Space-Use of Nesting and Non-Nesting Brachyramphus Murrelets in Icy Bay, Alaska, 2007--2012}}{\underline{Michelle L Kissling}, Paul M Lukacs, Stephen B Lewis, Scott M Gende}{Among seabirds, Brachyramphus murrelets are unique in their space use requirements largely because of their dispersed nesting behavior and inland nest locations that can be considerable distances from their marine foraging habitat. We studied space-use of radio-tagged Kittlitz's murrelets (B. brevirostris; KIMU), a species typically associated with glacial habitats, in Icy Bay, Alaska, 2007–2012. For comparison, we radio-tagged marbled murrelets (B. marmoratus; MAMU), the congeneric species typically associated with forested habitats, in the same study area in 2011 and 2012. We found that variation in marine home range (95\% UD) of both species was not explained by sex, reproductive status, or year, and that marine core use areas (50\% UD) of KIMU doubled that of MAMU in years when both species were tagged. Both breeding propensity and nesting success of KIMU was consistently lower than MAMU by at least half. Nesting habitat of the two species overlapped more than expected, though MAMU nested closer to the water than KIMU. Similarly, KIMU nests located closer to the water were more likely to be successful. Although we are uncertain as to why KIMU utilize larger marine areas compared to more successful MAMU, we postulate that aspects of foraging efficiency and commuting distance work synergistically to lower reproductive performance for KIMU. With ongoing loss of glacial habitats and subsequent vegetation succession, we predict that future habitat and resource conditions will benefit MAMU but be to the detriment of KIMU, at least in glacial systems like Icy Bay.}

\normaltalk{\capitalisewords{Post-Breeding Migration of Kittlitz's Murrelets from the Gulf of Alaska to the Bering Sea and Beyond}}{David C Douglas, John F Piatt, Mayumi L Arimitsu, Erica N Madison, \underline{Michelle L Kissling}, Sarah K Schoen}{The movements of Kittlitz's Murrelets (Brachyramphus brevirostris) during the non-breeding season are poorly known. We captured Kittlitz's Murrelets primarily in glacial-marine habitats adjacent to post-glacial terrestrial breeding habitats in the Gulf of Alaska during May–July 2009–2015. We attached 5-g solar-powered satellite transmitters (PTTs) dorsally with sutures to 47 murrelets. Twenty-seven PTTs provided tracking data after birds departed their breeding areas. Mean tracking duration among those 27 birds was 58.1 d (41.1 SD). All murrelets migrated toward the Bering Sea with short periods of movement (median 2 d, 1–3 IQR) separated by short stopovers (median 1 d, 1–4 IQR). During movement periods, travel averaged 79.4 km/d (83.5 SD, 449.1 maximum). All five Kittlitz's Murrelets tagged in May 2011 in Prince William Sound were tracked to the Bering Sea by August and four continued north to the Chukchi and Beaufort seas, accumulating tracks upwards of 2500–4000 km. Most birds appeared to molt along coasts of the Alaska Peninsula or eastern Bering Sea, as evidenced by 2–3-week stopovers during August–October that often ended with PTT failure, which we surmise was caused by a molt-induced detachment of the tag.  Using short-term PTT temperature changes as a proxy of diving activity, we found that markedly less diving was recorded at night compared to daylight hours.  After PTTs were shed in late-autumn, ship-based visual surveys have indicated that Kittlitz's Murrelets remain in the Bering Sea during winter, where they associate with sea-ice habitats before returning to glacial-marine habitats during the pre-breeding season.}

\normaltalk{\capitalisewords{Avian Functional Traits and the Impacts of Climate Change on Ecosystem Function}}{\underline{Alex L Pigot}, Chris Trisos, Cory Merow, Joseph Tobias}{Climate change is set to become a major driver of biodiversity loss this century but the potential impacts on the functioning and stability of ecosystems remains poorly understood. Birds support a multitude of fundamental ecological processes, from seed dispersal, to nutrient recycling and top-down trophic regulation. Furthermore, the unparalleled availability of data on the distributions and key functional traits of birds make them a model group for understanding how climate change is altering Earth's ecosystems globally. Using data on the ecology and morphology of all birds, I will present recent findings examining the projected changes in the functional diversity and composition of avian communities expected under multiple future climate change scenarios. Our results provide critical information identifying those ecological functions and ecosystems most at risk and provide the foundation for developing strategies to mitigate the ecological impacts of climate change.}

\normaltalk{\capitalisewords{Synthesizing Full Annual Cycle Science and Threats to Drive Conservation and Policy Actions for North American Migratory Birds}}{\underline{Jill L Deppe}, Tom Auer, Erika Knight, Gary Langham, Peter Marra, Stuart Mackenzie, Kenneth Rosenberg, Melanie Smith, Lotem Taylor}{Effective conservation of migratory birds requires a full annual cycle approach. Recent technological and analytical advances combined with the rapid accumulation of community science observations have revealed valuable insight into species distributions, habitat use and movements at broad spatial scales as well as connectivity among periods of the annual cycle. Synthesizing this information, derived from disparate data sources, across multiple species and integrating it with knowledge of spatio-temporal threats is needed to inform where and how we should focus our investments to most effectively and efficiently conserve migratory birds. Conservation depends on an array of actions, such as public and private land protection and management, restoration, public policy and community engagement. However, for any of these actions to be successful, scientists, stakeholders and decision makers need to be engaged throughout the process, from defining tangible conservation outputs to evaluating and implementing them. We will describe efforts to apply a translational ecology framework to migratory bird conservation at scale across the Western Hemisphere by building on existing science. Through examples targeting different conservation and policy actions, we will illustrate the process and value of bringing together diverse teams of scientists, conservation practitioners, policy advocates, and outreach and communication specialists to activate migration science for conservation.}

\normaltalk{\capitalisewords{Modeling Persistent Effects of West Nile Virus on Avian Population Dynamics in the Northeastern United States}}{\underline{Gaetan L Dupont}, David N Bonter, Orin J Robinson}{Soon after its introduction to North America in 1999, West Nile virus spread rapidly across the continent and negatively impacted the populations of a wide variety of avian host species. However, recent and long-term analyses of these avian population dynamics are lacking. Using statistical and machine learning methods, I analyzed citizen science data from Project FeederWatch from the past 22 years in the northeastern United States to generate population trends with a dynamic, spatiotemporal modeling approach. These trends were generated for a suite of six species, chosen a priori, including three species previously documented as significantly affected and three that were considered not significantly affected. I confirmed the primary period of population decline in the three affected species as documented in previous studies, while also identifying a secondary period of population decline. Both periods of decline were synchronized across the three affected species and were loosely associated with disease prevalence as reported in humans. Population trends from the three unaffected species showed independent and insignificant temporal dynamics. Spatial trend modeling revealed geographic hotspots of population decline that were similar for the three affected species. These results suggest that West Nile virus continues to be a driver of population dynamics for susceptible avian species.}

\normaltalk{\capitalisewords{State of Population Trend Monitoring for North American Birds}}{\underline{Nicole L Michel}, Adam C Smith, Marie-Anne R Hudson, John Sauer}{In North America, our ability to track long-term bird population changes is only possible because of the tremendous effort that government agencies, conservation organizations, and skilled volunteers have put towards systematic monitoring. Through the combined efforts of the Christmas Bird Count, Waterfowl Breeding Population and Habitat Survey, North American Breeding Bird Survey, cooperative surveys of migrating shorebirds in Canada and the US, and numerous other large-scale surveys, we have reasonably reliable information on regional, national, and continental population trends of almost 550 species of birds. Yet critical gaps remain. We will highlight ongoing efforts to improve standardized continental monitoring, including: 1) monitoring regions with limited access and few trained surveyors (e.g., Boreal, Mexico); 2) collaborative tools to monitor non-breeding pelagic seabirds; 3) adaptive monitoring for nocturnal or cryptic species, e.g., owls; 4) full annual cycle monitoring; and 5) incorporating new technologies (e.g., autonomous recording units, drones) to augment human survey efforts. The recent growth and expansion of eBird leads us to a crossroads in monitoring science. Many recent papers call for rigorous field methods that enable explicit estimation of detection probability. Yet the ubiquity of eBird and quantitative advances suggest that less-standardized field methods may contribute to reliable monitoring – particularly when combined with standardized data in integrated models. Reconciling these contradictory themes will be among the next great scientific challenges for North American bird conservation. Regardless, unless we take appropriate and informed conservation action, the future of bird monitoring will hold little more than the documentation of decline and collapse.}

\normaltalk{\capitalisewords{Relationships Between Local- And Landscape-Scale Management and Urban Bird Communities Across Six Major Metropolitan Areas: Do Yards Matter?}}{\underline{Desiree L Narango}, Susannah B Lerman, Sharon J Hall, Sarah E Hobbie, Christopher Neill, Tara L Trammell, Peter Groffman}{Urban landscapes have bird assemblages consisting of fewer habitat specialists compared to natural areas. However, development within urban landscapes is highly heterogeneous and may harbor species of conservation concern in both shared (i.e. residential yards) or spared (i.e. protected parks) land uses. Studies that investigate how urban management decisions affect biodiversity at multiple scales are required for effective conservation. Here, we used a hierarchical multi-species occupancy model to determine how avian species richness, community coincidence, and species of conservation concern differ with local-scale land management, and landscape-scale tree canopy cover and impervious surface across six major metropolitan areas. Shared-use land typologies included low-management, high-management, water-conservation \& wildlife-friendly yards, and spared-use included large peripheral \& smaller interstitial parks. Our results suggest that urban bird communities are generally structured by landscape-scale characteristics rather than parcel-scale management, however, relationships were unique to each region. Our spatially explicit, multi-region design revealed findings similar to other landscape-scale research that shows avian species richness is positively related to canopy cover, and negatively related to impervious surface. Across cities, community richness did not differ between land uses, however parks supported communities with the most insectivores and species of conservation concern. Moreover, within metropolitan areas, yards with high management had the highest community similarity, while other yard types had similar turnover to parkland. These results demonstrate that residential landscapes have the potential to benefit local avian richness and reduce homogenization when residents prioritize management regimes that favor mature tree canopies, retain parks, and promote residential areas with natural features.}

\normaltalk{\capitalisewords{Population Trends and Connectivity of Snowy Plovers on the Southern Great Plains}}{\underline{Kristen M Heath}, Warren C Conway, Clint W Boal, Dan P Collins, Jeffrey S Beauchamp, William P Johnson, Glen Hensley, Sarah T Saalfeld, Blake A Grisham}{Previous research from 2007-2009 estimated that breeding snowy plovers (Charadrius nivosus) at saline lakes in Texas experienced a ~75\% population decline in the preceding decade. However, there is a paucity of data concerning population connectivity of snowy plovers on the Southern Great Plains (SGP), where snowy plover breeding habitat can be 10-600 km apart. Such discrete distribution of breeding habitat, coupled with weather stochasticity and low snowy plover abundance at sites in Texas and New Mexico, increases risk of regional extirpation and population persistence probability may be quite low. We used the Motus Wildlife Tracking Network to track snowy plover movements across 6 sites on the SGP in 2017 and 2018.  Additionally, we performed surveys in Texas and used 20-year survey data from 2 National Wildlife Refuges to assess current population trends, relative abundances and detection probability across the SGP landscape. Trend data from Salt Plains National Wildlife Refuge (SPNWR) in Oklahoma indicate population stability during the past decade (P $>$ 0.05), with detection probabilities ~1.0, mirroring previous detection estimates in Texas.  However, this stability at SPWNR contrasts with long term declines of breeding snowy plovers in Texas ($>$40\%) and New Mexico Bitter Lake NWR ($>$60\%). We detected within breeding season movements between isolated plover habitats 12 to $>$ 600 km apart, indicating that SGP population connectivity exists. This connectivity, combined with stable trends in Oklahoma, indicates that regional persistence may be maintained by plovers occurring at SPNWR, further emphasizing the hemispheric importance of SPNWR for interior snowy plovers.}

\normaltalk{\capitalisewords{Factors Related to Occupancy Dynamics and Breeding Propensity of Northern Spotted Owls in a Protected Area: Habitat, Weather and Barred Owls}}{\underline{Katie M Dugger}, Anna O Mangan, Tara Chestnut, Jody C Vogeler, Ian K Breckheimer, Wendy M King, Keith E Bagnall}{The federally threatened Northern Spotted Owl (Strix occidentalis caurina) occurs on public lands throughout the Pacific Northwest, including Mount Rainier National Park (MRNP), Washington, USA.  MRNP provides an ideal place to evaluate potential impacts of climate change and the invasive Barred Owl (Strix varia) on Northern Spotted Owl demographics because it has virtually no history of timber harvest or large forest disturbance within its boundaries since the park's creation in 1899. We used a multi-state, multi-season occupancy model to investigate the effects of Barred Owl presence, local and regional weather, and habitat characteristics on Northern Spotted Owl occupancy dynamics and breeding propensity at MRNP from 1997-2016.  Occupancy of spotted owl territories in MRNP has declined by 50\% in the last 20 years and rates of occupancy by breeding spotted owls decreased to a low of 25\% in 2016. Occupancy rates were higher on territories with steeper terrain. Breeding propensity was lower when Barred Owls were detected, but higher when early nesting season temperatures during March and April were higher. Detection probabilities for breeding Spotted Owls decreased when Barred Owls were present in the territory. Other habitat characteristics were not associated with Spotted Owl occupancy dynamics, which likely reflected the long history of conservation in the park, with old-growth forest predominating in most areas. This study illustrates the strong relationship between the presence of Barred Owls and Spotted Owl demographics and breeding site selection on protected lands where habitat loss through timber harvest and wildfire has not occurred.}

\normaltalk{\capitalisewords{Do Drones Bug Songbirds?}}{\underline{Andy M Wilson}, Kenneth S Boyle, Jennifer L Gilmore, Cody Kiefer}{Audio recorders attached to drones provide a potential new and highly efficient way to survey vocal bird species in less accessible habitats. However, it unknown to what extent drone noise could affect bird behavior, specifically song output, which has both methodological and ethical implications. We conducted 27 replicates of an experiment where we hovered a quadcopter drone 50 m above the center of a 50 m x 50 m array of four autonomous recordings units (ARUs). The ARUs recorded ambient birdsong for 11 minutes: 4 mins before, 3 mins during, and 4 mins after the drone flight. Our experimental set-up allowed us to use sound localization to track songbird output in space and time in the area under the hovering drone. The experiment was conducted in Adams Co. Pennsylvania, in June 2017. In all we recorded 10,095 song bouts by approximately 544 individual birds of seven target songbird species. We used distance sampling with time-removal models to estimate the density of singing birds before, during, and after the drone flight. We found no evidence that the drone affected singing behavior of six species, but for the Yellow Warbler (Setophaga petechia) we found that 38\% of birds within 40 m of the center of the ARU array curtailed singing during the drone flight, and that song output remained depressed four minutes after the drone's departure.}

\normaltalk{\capitalisewords{A Tale of Two Todies: Understanding How Vocalizations Mediate Behaviors in a Sympatric Zone}}{\underline{Holly M Garrod}, Robert Curry}{The evolution of species resulting from the geographic barriers between populations is a common rule in biogeography; however, climate change has been shown to change former ecological barriers, reestablishing contact between populations. In birds, vocal recognition is known to play an important role in mediating gene flow and competition. Most studies of interactions between closely related taxa to date have focused on song-learning birds, where by contrast, little is known about how vocalizations mediate interactions between birds who do not learn their songs.  We chose Hispaniola as a study system, where two birds with innate vocalizations, the Broad-billed Tody and Narrow-billed Tody, are separated along an elevational gradient. The former inhabits lowland forests and the latter inhabits montane forests, with recent reports suggesting greater range overlap than previously known. In order to understand the behavioral mechanisms separating these two similar species, we conducted standardized paired playback experiments in sympatric and allopatric zones to assess responses to conspecific and heterospecific vocalizations. Narrow-billed Todies responded more aggressively to heterospecific vocalizations in sympatric zones. However, in allopatric zones, only conspecific vocalizations elicited a response for both species. This pattern suggests Narrow-billed Todies perceive Broad-billed Todies as a threat, this could be from competition of space, resources, or even potential mates. This may allow us to understand how other similar species would interact and persist as ranges continue to alter due to climate change.}

\normaltalk{\capitalisewords{Plasticity of Aggressive Behavior in Bachman's Sparrows}}{\underline{Joseph M Niederhauser}, Sabah Ali, Maria P Ziadi, Rindy C Anderson}{In changing environments, behavioral plasticity should be favored by selection when the cost of plasticity is low and individuals can match their behavior to current environmental conditions. However, some behaviors may be fixed due to phylogenetic, physiological, or developmental constraints, and may covary in consistent and predictable ways, known as behavioral syndromes or animal "personalities." Some personalities are sensitive to environmental changes with shyer, less aggressive individuals having better fitness in rapidly changing habitats than bolder, more aggressive individuals. However, personality traits have been found to be plastic in some cases, with variation depending on context or experience. This raises the question: are personality traits plastic in frequently disturbed environments? To explore this question, we studied aggressive behavior in the Bachman's Sparrow (Peucaea aestivalis), a Near Threatened songbird that resides in habitats that experience frequent fire every one to three years. We measured aggressiveness over four breeding seasons using simulated territorial intrusions. We found that aggressive behavior varied across years and had a low repeatability. Aggressiveness was related to fire history: sparrows were very aggressive immediately after a fire and became less aggressive as more time passed. These data suggest that Bachman's Sparrows are flexible in their aggressiveness, but what remains to be tested is why they adjust their behavior relative to fire. This is one of the first studies to show plasticity in aggressive behavior across years.}

\normaltalk{\capitalisewords{Exploratory Behavior in Brown-Headed Nuthatches may Indicate Successful Population Reintroduction Strategy}}{\underline{Mary Mack M Gray}}{The reintroduction of species to historical ranges from which they have been extirpated is widely used to counteract increasing rates of fragmentation and biodiversity loss. One poorly-studied aspect of reintroductions is the role of animal personality, which is hereditary and can influence dispersal tendency, habitat selection, and population persistence. In particular, the rate at which an animal explores a novel environment (termed 'exploratory behavior'), is a personality type that can predict survival in a new environment. There is growing evidence that slow explorers experience increased survivorship and reproductive success in a novel environment when compared with fast explorers. In order to test the hypothesis that slow exploration is favored in a novel environment, I compared exploratory behavior in three Brown-headed Nuthatch (Sitta pusilla) populations: a reintroduced population, the population from which the reintroduced birds originated (original donor), and a distant control population. I placed birds in an exploratory chamber and compared the number of flights/hops, scanning events, thoroughness of exploration, and individual behaviors in each population. I found that birds from each population behave differently when placed in a novel environment, but that birds from the reintroduced population were the slowest explorers in terms of fewest flights/hops, fewest scanning events, and highest proportion of birds that remained frozen in place. These results reveal a shift in personality type in the years since reintroduction that adds to the growing consensus that slower explorers fare better in novel environments, which has important implications for future reintroduction efforts.}

\normaltalk{\capitalisewords{Increased Predator Density Changes Nest Attendance Behavior of Yellow-Headed Blackbirds (Xanthocephalus Xanthocephalus)}}{\underline{Tyler M Harms}, Colby Behrens, Zachary J Ruff, Stephen J Dinsmore}{Nest predation is a major factor limiting reproductive success in birds and activity near the nest has been shown to increase predation risk.  Parents must carefully balance this risk with their needs as well as those of the eggs and nestlings by altering nest attendance behavior (e.g., off-bouts), particularly in areas with increased density of predators.  We tested this hypothesis by evaluating the influence of the density of the Marsh Wren (Cistothorus palustris), a common nest predator in freshwater marshes throughout the Midwest United States, on the off-bout frequency of Yellow-headed Blackbirds (Xanthocephalus xanthocephalus).  We placed temperature data loggers in 52 nests at Big Wall Lake, Iowa, USA in 2015 to monitor internal nest temperatures (°C) during the incubation stage and then compared these temperatures to ambient temperatures during morning (05:00-10:00 h), day (10:00-16:00 h), and evening (16:00-21:00 h) hours to determine when parents left and returned to the nest.  Yellow-headed Blackbird off-bout frequency was negatively correlated with local Marsh Wren density during both the morning (β = -0.19, 95\% CI: -0.33 to -0.04) and evening (β = -0.21, 95\% CI: -0.37 to -0.06) hours but was not correlated with local Marsh Wren density during the day.  Additionally, there was no relationship between off-bout frequency and local density of Yellow-headed Blackbirds nor with ambient temperature (°C), precipitation (cm) or Julian day during any time period.  Our study, therefore, suggests that Yellow-headed Blackbirds are sensitive to nest predation risk and alter their nest attendance behavior accordingly to increase overall fitness.}

\normaltalk{\capitalisewords{Parental Behaviors Mediate the Effects of Bird Blow Fly Ectoparasitism on Nestling Wood Thrushes in Suburban Forests}}{\underline{Katherine M Straley}, Paige S Warren, David I King}{Suburban ecosystems provide unique opportunities to study how an animal's behavior responds to novel habitat changes. A major outstanding question is whether birds have the behavioral flexibility to adaptively respond to the conditions they experience in developed landscapes. Urbanization influences species interactions, including host-parasite interactions. Using the Wood Thrush (Hylocichla mustelina), a declining songbird, we examined a) whether suburban birds experience different ectoparasite loads, and b) if parental behavioral decisions will mediate the impacts of ectoparasites on nestling health. In particular we compared: the abundance of bird blow flies (Protocalliphora) from the nest, measurements of nestling condition, and parental behaviors monitored via nest cameras from suburban forest fragments and a larger forest. Preliminary analyses indicate that suburban nestling body condition was not affected by the presence of bird blow flies, while rural nestlings experienced a decrease in body condition when parasitized. Parents nesting in suburban sites provisioned nestlings at higher rates across multiple age classes, allowing the suburban nestlings to maintain body condition while parasitized. Our results suggest that suburban birds are indeed flexible in their parental behavior, and this may be due to greater abundance or higher quality of food resources or predator relaxation in suburban forest fragments.}

\normaltalk{\capitalisewords{How Do Dispersal and Migration Influence Range Expansion of Birds?}}{\underline{Teresa M Pegan}, Brian Weeks, Benjamin Winger}{Migratory birds have extremely high dispersal ability, with some species flying between hemispheres twice annually during their seasonal migrations. This vagility suggests that dispersal limitation should not constrain geographic range sizes in migratory birds. However, there is also evidence that migratory behavior can hamper colonization across continents, perhaps because dispersal into novel locations can lead to inefficient migration if individuals attempt to follow their ancestral migratory patterns. It remains unclear whether migration generally constrains or facilitates geographic breeding range expansion. Here, I test whether migration constrains geographic range within a continent by comparing migration distance with range-filling (the occupied proportion of abiotically-suitable area, calculated with species distribution models) in 460 North American land bird species. Contrary to my prediction, migration showed a weakly positive relationship with range-filling. Because migration distance correlates with dispersal ability, I tested whether dispersal ability (measured with hand-wing index) could explain variation in range-filling in the same set of species. I found no relationship between hand-wing index and range-filling, suggesting that species' geographic ranges within continental North America are not influenced by variation in dispersal ability. The association between long-distance migration and high range-filling may indicate that long-distance migration facilitates colonization of or persistence in most of the abiotically-suitable habitat available to a species, but this relationship is not related to the dispersal ability of migrants. Instead, long-distance migration may allow species to occupy areas where few competitors are present to exclude them from suitable habitat.}

\normaltalk{\capitalisewords{Patterns and Mechanisms of Heterogeneous Breeding Distribution Shifts in North American Migratory Birds}}{\underline{Hanna M McCaslin}, Julie A Heath}{Species distributions are shaped by complex interactions between physiology, environment, biotic interactions, and movement capacity. Climate change is affecting distributions by affecting physiological tolerances directly and by altering the timing and distribution of seasonality in the temperate region. Warming temperature is the most commonly hypothesized driver of shifts in breeding distributions, but multidirectionality of shifts suggests that tracking climate niches is not the only underlying factor. We quantified shifts in breeding distribution centroid for 73 species of North American migratory birds shifts from 1994-2017 across western, central, and eastern regions and tested whether life history characteristics were correlated with these shifts to identify potential mechanisms. We found that 44\% of regional shifts were towards the equator, 55\% were poleward, and a species did not necessarily shift in the same direction across regions. Further, shifts in centroids were not explained by trends in regional abundance, suggesting that centroid shifts were not attributable to population declines or increases at distribution margins. We did not find any life history traits that explained southward shifts, but diet, migratory strategy, and tolerance to humans predicted northward shifts. Our results clearly indicate the prevalence of multidirectional breeding distribution shifts, and suggest that life history is one component in a likely complex set of interacting mechanisms acting at many scales to drive shifts.}

\normaltalk{\capitalisewords{Can Nest Concealment in Wood Thrushes Predict Reproductive Success?}}{\underline{Alexandra M Israel}, Bridget J Stutchbury}{Predation is an important selective pressure that influences the defense strategies of many animals. The Wood Thrush (Hylocichla mustelina) is a long-distance migrant that has declined severely in the last 50 years and is currently listed as a Species at Risk in Canada. Declines in this species are attributed to nest predation, which is a major source of nest failure in many species of birds, accounting for nearly 80\% of nest failures on average. Declines in Wood Thrushes are also a result of a brood parasite known as the Brown-headed Cowbird (Molothrus ater), which can partially depredate nests before laying its own eggs. Concealment of nests with vegetation is thought to be an important factor in reducing the chance of detection by visually oriented predators, including brood parasites. This study investigated whether nest concealment had an influence on the success of Wood Thrush nests in the context of a fragmented landscape and as a Species at Risk. Through visual estimates of concealment in 161 nests over 2 years, results show that Wood Thrushes experience high nest predation (51.6\%) and moderate cowbird parasitism (23.6\%) in southwestern Ontario. However, nest concealment does not predict probability of having a successful nest or evading parasitism. We are also testing if nest concealment impacts an incubating female's flight initiation distance and blood corticosterone levels.}

\normaltalk{\capitalisewords{To Kill a Mockingbird: The Combined Effects of Two Radically Different Parasites}}{\underline{Jordan M Herman}, Vanina D Fiorini, Ignacio Crudele, Juan C Reboreda, Shawn A Pladas, Sara E Bush, Dale H Clayton}{Co-infection, or simultaneous parasitism by multiple parasites, can have significant consequences for host fitness. Still, research examining the effects of multiple parasites under natural conditions remains scarce, and many parasites that co-occur regularly in nature have been largely overlooked. Nest flies and cowbirds (genus Molothrus) are common parasites of breeding birds, and co-occur frequently in host nests throughout their ranges in the Americas. Independent studies of these parasites have reported significant reductions in host fitness due to parasitism, yet their combined effects on host fitness is unknown. In Argentina, chalk-browed mockingbirds (Mimus saturninus) are heavily parasitized by shiny cowbirds (Molothrus bonariensis) and the nest fly Philornis seguyi. During two field seasons in 2017 and 2018, we investigated how simultaneous parasitism by these radically difference parasites influences the survival of mockingbird nestlings. In 2018, we used action cameras to conduct constant daylight surveillance of nests to determine the exact cause and time of death. Using a two-factor design, cowbirds and Philornis were experimentally eliminated from mockingbird nests, resulting in four unique treatments. We found that mockingbirds in nests with neither parasite had the highest survival of all treatments, and the presence of either cowbirds or nest flies significantly reduced mockingbird survival. While we did not detect an interaction between the parasites, nests that were simultaneously parasitized by cowbirds and nest flies had the lowest overall fledging success. Our study shows that parasitism by cowbirds and Philornis has devastating effects on the survival of host nestlings.}

\normaltalk{\capitalisewords{Eggshell Characteristics as Predictors of Heavy Metal Concentration in House Sparrow Eggs}}{\underline{Suzanne M Hartley}, Caren B Cooper}{Persistent environmental contaminants pose serious health risks to humans and wildlife alike. However, the high costs of monitoring for contaminants means the presence of contaminants in the environment may be unknown.  Environmental contaminants can alter the chemical pathways involved in avian eggshell formation and pigmentation. Determining whether these contaminants alter eggshell characteristics (color and speckling) in predictable patterns has important applications for eggs as indicators of environmental contaminants. We used House Sparrows (Passer domesticus) as a model organism for determining patterns between eggshell characteristics and heavy metal concentrations.  House Sparrows are an ubiquitous non-native species, that are commensal with humans, potentially exposing them to similar environmental contaminants as humans. Volunteer songbird nestbox monitors from across the United States collected House Sparrow eggs (n= 2,200) as part of the citizen science project, Sparrow Swap.  We examined how eggshell characteristics vary across a geographic scale and throughout the breeding season. We investigated if heavy metals concentrations are correlated with eggshell color, speckling, thickness, and calcium concentrations. We measured metal and calcium concentrations in 100 eggshells of different clutches using mass spectrometry and used the software SpotEgg to classify eggs based on color and speckling. Our results indicate that heavy metals, including (Ar, Se, Cd, Cu, Pb), are present in detectable levels in House Sparrow eggs found across the country. Our works provides a basis for exploring the use of this invasive pest species for environmental monitoring.}

\normaltalk{\capitalisewords{Creating Diverse and Quality Habitat for Birds Through Application of Management Guidelines for Land Managers in Northern Hardwood Forests}}{\underline{Suzanne M Treyger}, Michael F Burger}{Northern hardwood forests provide important breeding, migratory stop-over, and wintering habitat for more than one hundred species of birds. One of their most important ecological functions is to provide breeding habitat for several dozen bird species, many of which are experiencing population declines due to a number of factors, including habitat fragmentation and loss of quality habitat. Quality forest habitat for birds and other wildlife means intact, healthy, resilient, regenerating, and diverse forested landscapes.  The application of sustainable forest management can greatly improve forest habitat. Silvicultural prescriptions, under the guidance of a professional forester, can create favorable conditions for birds while achieving other ecosystem services objectives, such as timber management and improving water quality. Audubon New York created Forest Management for New York Birds: A Forester's Guide as a resource for foresters and other land managers to integrate important habitat components into forest management planning. This new resource provides guidance on how to manage forested landscapes to provide a balance of forest age classes to meet the habitat needs of a suite of forest birds, as well as stand-level features that increase structural complexity.  Audubon disseminates the information in the guide through various outreach efforts and technical assistance to our target audience of foresters, land managers, and forest owners. In addition, Audubon applies the management recommended in the guide to habitat improvement projects throughout the state on public and private land.}

\normaltalk{\capitalisewords{Understanding the Effects of Nest Site Temperature on Songbird Reproductive Fitness and Parental Care}}{\underline{Tayler M Scherr}, Anna D Chalfoun}{Weather patterns are changing at an unprecedented pace and may affect the fitness and behavior of birds individually, additively, or synergistically with other forms of environmental change. Passerines are particularly vulnerable to temperature at the nest site, as there is a tight range of temperatures under which optimal development of young occurs. Sagebrush-obligate songbirds are an ideal study species to examine how temperature and physical habitat loss may combine to influence fitness and behavior, because they inhabit arid and relatively exposed environments. Additionally, the sagebrush-steppe has been highly altered by human activities such as energy extraction. We investigated how temperature at the nest site affects reproductive fitness metrics and parental behavior of three sagebrush-obligate songbirds along a gradient of habitat loss associated with energy development in western Wyoming, USA. We examined how mean temperature and temperature variation affected nestling growth. We predicted that we would see a difference in responses across the three species, given that physiological responses to temperature can vary with body size. Further, we asked how adults may be able to modulate the effects of temperature via parental care at the nest site. Previous work in this system has shown adults select nest sites in relation to temperature, therefore adults may also be able to alter parental care to modulate the effects of temperature at the nest site. A better understanding, however, of the extent to which this occurs is needed.}

\normaltalk{\capitalisewords{Behavioral Niche Partitioning Reexamined: Do Behavioral Foraging Differences Predict Dietary Differences?}}{\underline{Cody M Kent}, Thomas W Sherry}{Behavioral niche partitioning is an important and widely assumed explanation for the coexistence of ecologically similar species. Here we looked critically at this explanation by testing its core assumption, namely that evolved differences in foraging behavior correspond with comparable differences in actual resource use. Our study combined foraging behavior, available prey, and observed diets of five widely coexisting wintering (non-breeding) Parulid warblers that are food-limited and compete both intra- and interspecifically for food resources. Consistent with past work on our focal species, we found that birds differ markedly in their foraging behavior, enough that some species pairs essentially did not overlap in foraging substrate. In contrast to behavioral results, these birds overlapped strongly in arthropods consumed, despite small but diagnosable dietary species differences. To test the prediction that major behavioral differences could explain the small dietary differences, we generated expected diets based on available prey and foraging microhabitat use. Consistent with behavioral niche partitioning as a coexistence mechanism, we found that the small dietary differences were explained by a combination of foraging substrate and available prey. Thus, our results indicate that differences in foraging behavior among these bird species help explain subtle differences in diet, potentially facilitating coexistence. Moreover, these large behavioral differences, despite only minor dietary specializations, have important conceptual and practical consequences: Foraging behavior and other genetic bases for species resource-use differences may need to be greater than previously recognized, bringing into question how these differences arise.}

\normaltalk{\capitalisewords{Down in the 'weeds' Searching for Answers to Wood Thrush Fledgling and Juvenile Survival}}{\underline{Sue M Hayes}, Brendan Boyd, Bridget Stutchbury}{The survival of juvenile songbirds has been an under-studied area in population dynamics of songbirds because of the difficulty in tracking their movements once they leave the natal territory and no longer require parental care. Our research is focused on the question of how forest fragmentation affects survivorship, dispersal and the on-set of fall migration in juvenile Wood Thrush (Hylocichla mustelina). We are testing if there are negative carry-over effects for Wood Thrush originating from small forest fragments that put them at a disadvantage compared to offspring originating from larger fragments. This study is unique as it has only recently been made possible to execute through the innovation of the automated long-distance radio telemetry collaborative Motus Wildlife Tracking System. This system is built on an array of $>$300 receiver towers that allows researchers to track tagged wildlife movements over larger spatial scales. Using long-life radio-tags, we are able to detect and track tagged juvenile Wood Thrush dispersal movements at a spatial and temporal scale that has never been done before. Preliminary results indicate that there was 58\% juvenile survival for those individuals that survived to depart on autumn migration, however only 37\% survived to the following spring. We are testing if territory quality impacts fledgling and juvenile survival.}

\normaltalk{\capitalisewords{Landscape Matrix and Species Traits Mediate Avian Responses to Forest Fragmentation}}{\underline{Christina M Kennedy}, Peter P Marra}{Species in fragmented landscapes depend upon, not only on native habitats, but also the surrounding anthropogenic "matrix" land uses. Yet, the ways in which different human-modified matrices hinder or support species is still not well known, nor are the dominant mechanisms that underlie their responses. Here, we synthesize findings from a series of studies on bird responses within fragmented landscapes that span the gradient of low-to-high intensity land uses in central Jamaica. Based on surveys across intact forest, habitat remnants, and human-dominated areas, we find that matrix conditions are a main driver of community composition, local colonization and extinction, movement patterns, and resource use. Coupling response patterns with biological traits support an emergence of generalities about which species are most vulnerable to landscape alteration and dominant mechanisms likely at play. We find that the landscape matrix impact species largely through its mediation of resource availability, and that matrix heterogeneity largely benefit generalist species rather than specialist species. These findings underscore that effective management of human-modified landscapes require an in-depth understanding of how species dynamically use native habitats together with different matrix types and the trade-offs between the benefits available in the matrix and the impacts on disturbance of native habitats.}

\normaltalk{\capitalisewords{Demographic Consequences of Road Mortality to Snowy Plovers at Gulf Islands National Seashore, FL}}{\underline{Maureen M Durkin}, Jonathan B Cohen}{Gulf Islands National Seashore holds up to 20\% of Florida's state-threatened Snowy Plovers (Charadrius nivosus). Despite this substantial population, little is known about productivity and survival of Snowy Plovers in this area. Further, 14 miles of public roadway bisect the site- directly through plover nesting and foraging habitat. While shorebird road mortality has been an issue of concern for many years, data has been non-systematically collected in the past, and the effects of detection rate and scavenging on counts are unknown. To accurately estimate the number of plovers killed, we did systematic carcass surveys in 2013-2016. We also monitored nests and broods of breeding Snowy Plovers to determine productivity for the site. Preliminary results indicate that upwards of 10\% of adults, 15\% of chicks, and 25\% of fledglings are killed at some sites in a single breeding season. In addition to this direct mortality, road mortality of nesting adults has also resulted in subsequent nest abandonment by the surviving mate. We used these mortality rates from carcass surveys in a population viability analysis (PVA), and modeled the probability of extinction of both site and regional populations under different road mortality and productivity scenarios. Our results indicate that the Florida Panhandle Snowy Plovers exhibit source-sink dynamics, and our study area is currently a regional sink. Road mortality is a significant contributor to the situation, but is exacerbated by recent low reproductive rates. Potential management actions aimed at sinks for this population should address both road mortality and low productivity issues.}

\normaltalk{\capitalisewords{Fear of Predators Influences Nestling Condition Across a Suburbanization Gradient}}{\underline{Aaron M Grade}, Paige S Warren, Susannah B Lerman}{Fear of predators is a powerful driver of behavior in birds and can alter parental behaviors, such as provisioning rates, in the face of predation risk. Evolutionary theory posits that fear responses should maximize lifetime reproductive fitness. Response to fear represents a tradeoff between self-maintenance and parental investment. Birds in suburban areas are often exposed to high densities of potential predators, but lower per capita predation, a phenomenon called the "predation paradox." As human development expands, conservation of species across their range hinges on a deeper mechanistic understanding of the effects of suburbanization on reproductive outcomes. Fear effects may represent a mechanism for the predation paradox, and the resulting pattern of reduced nestling condition in areas of increased human-development. We investigated the interaction between parental behavior, fear effects, and suburbanization using House Wren (Troglodytes aedon) as a model species. We predicted that increased predation risk as well as increased levels of suburbanization would result in decreased nestling condition. Through playbacks, we experimentally altered abundances of predators that consume adult wrens across a gradient of suburbanization. Nestlings exhibited slightly decreased body mass in more suburbanized areas and a large reduction in body mass when exposed to predator playbacks. This may represent an evolutionary tradeoff to prioritize self-maintenance over current reproduction with increased risk of adult mortality. Manipulative studies in situ are essential for developing a mechanistic understanding of bird parental behavior in human-dominated landscapes.}

\normaltalk{\capitalisewords{Regional Variation in the Abundance of House Finches Created the Context that Determined the Impact of an Emerging Pathogen}}{\underline{Wesley M Hochachka}, Andr\'{e} A Dhondt}{The bacterium Mycoplasma gallisepticum emerged as a pathogen of House Finches in the eastern United States in the early 1990s, causing dramatic declines in abundance of the finches across much of their eastern North American range. M. gallisepeticum is now present in House Finch populations across much, if not all, of the finches' range. Both indirect and anecdotal evidence suggest that the impact of this pathogen on House Finch populations has varied across North America, but to date the affects of the emerging disease have not been systematically examined across western North America as a whole. We use data from Project FeederWatch to provide indices of relative abundance of this common feeder-attending bird species. We found variable changes in the abundance of House Finches coincident with the likely time of arrival of M. gallisepticum, ranging from no detectable changes in finch abundance through to moderate declines that were smaller than those seen for eastern North American populations of House Finches. We discuss how the varying impact of the pathogen appears likely to be the result of both variation in pre-disease density of House Finches (with resultant variation in the rate of disease transmission), and the evolutionary context that this varying density presented during the evolution of virulence of M. gallisepticum.}

\normaltalk{\capitalisewords{The Effects of Climate, Habitat, and Diet on Methylmercury Bioavailability for New York Songbirds}}{\underline{Evan M Adams}, Amy K Sauer, Oksana Lane, Kevin Regan, David C Evers}{Mercury (Hg) is a global pollutant that affects songbird populations across a variety of ecosystems after it is converted into methylmercury (MeHg)—a more toxic and environmentally persistent form of Hg. Wetland habitats are particularly effective at methylating Hg, but this process can also occur in terrestrial habitats. Using songbirds as an indicator of MeHg bioavailability in terrestrial ecosystems, we want to understand the combined roles that habitat, climate, and food web length play in dictating MeHg exposure risk across a variety of ecosystems. To address this objective, 2243 blood Hg samples were collected from 81 passerines and near-passerines species in New York state, USA, across 10 different sampling regions spanning Long Island to western New York from 2013-2017. Using a general linear mixed modeling framework, we found that wetland habitat area, 50-year averages of summer maximum temperatures, and trophic position were all correlated with songbird blood Hg concentrations. Statewide, sites with high amounts of wetland habitat, higher average summer maximum temperatures and individuals at higher trophic positions had higher blood Hg concentrations. These patterns had a large degree of spatial variability for habitat, climate, and food web related variables suggesting that the drivers of MeHg bioavailability changed significantly across the state. Mercury deposition, land cover, and climate are all expected to change throughout the northeastern United States in the coming decades; terrestrial MeHg bioavailability will respond to these changes and focused research and monitoring efforts will be critical to understand how exposure risk changes across the landscape.}

\normaltalk{\capitalisewords{Sialic Acid Quantity and Glycosidic Linkage on Avian Erythrocytes}}{\underline{Jeanne M Fair}, Mark Jankowski, Scott Glaberman, David Kimball, Kirsten McCabe}{Variation in physiological traits is often aligned with variation in susceptibility to pathogens. We investigated this concept in the context of influenza A virus and avian Plasmodium, two important pathogens of birds. If it can be determined why the same avian influenza strain causes mortality in one species while producing minimal signs of disease in another, predictive models may be developed that improve targeted disease surveillance strategies. The cell membrane sugar sialic acid (Sia) is a major receptor for influenza A virus (IAV) that varies across tissue type and species in its modifications and how it is bound to galactose and the underlying glycan. The primary form of Sia important for IAV binding is N-acetylneuraminic acid that is bound to galactose via a2-3 or a2-6 glycosydic linkage. Measuring Neu5Ac and linkage type in red blood cells (RBC) across numerous avian groups (76 avian species, 29 families, and 16 orders), we hypothesized that Neu5Ac quantity would not exhibit phylogenetic signal while glycosidic linkage would, as the quantity of RBC bound Sia is likely affected by many contemporary factors whereas glycosidic linkage on RBCs has no known ligands that would cause it to vary outside of phylogenetic patterns. Indeed, we found no phylogenetic signal in quantity, but detectable signal in the a2-6, but not a2-3 linkage. The lack of signal in a2-3 may be related to an uneven exposure to a2-3 binding IAVs whereas a2-6 binding viruses are primarily not observed in birds.}

\normaltalk{\capitalisewords{A Tale of Two Transects: Sexual Selection Across a Hummingbird Hybrid Zone}}{\underline{Brian M Myers}, David T Rankin, Alan Brelsford, Kevin J Burns, Christopher J Clark}{Hybrid zones have received significant attention in biology, although fewer studies have assessed how selection acts upon behavioral traits across these regions of interaction in comparison to other types of characters. Areas of contact serve as natural laboratories that offer insight into reproductive isolation and the speciation processthrough analysis of characters that diagnose interbreeding parent species. In a hybrid zone in southern Oregon and northern California between the Allen's (Selasphorus sasin) and Rufous (Selasphorus rufus) Hummingbirds, we found that hybrids varied in genetics, morphology, and behavior. Using Linear Discriminant Function Analysis and cline analysis, we characterized variation across a coastal and inland transect of the hybrid zone to assess how evolutionary patterns compared across each transect. Here, we emphasized sexually-selected courtship behaviors that are stereotyped in each parent species. The courtship displays of both species involved a male hummingbird performing a J-shaped dive, during which the male produced a species-specific sound with his tail feathers. These displays were broken into distinct components, which were rearranged to comprise the displays found within Allen's and Rufous Hummingbird. Hybrids performed courtship displays that incorporated different combinations of the same underlying components of parental species. We used these components to study the discrete courtship behaviors they formed and the variation that resulted across the hybrid zone as a result of recombination. We then compared behavioral clines to morphological and genomic clines to investigate the role of sexually-selected traits as a reproductive barrier between Allen's and Rufous Hummingbird.}

\normaltalk{\capitalisewords{The Relative Roles of Natural Selection and Sexual Selection in Speciation in the Tanagers}}{\underline{Rosalyn M Price-Waldman}, Amelia J Demery, Nicholas A Mason, Allison J Shultz, Pascal O Title, Kevin J Burns}{A central goal of evolutionary biology is to understand why diversification rates vary so widely among clades. Previous studies have suggested that variation in evolutionary rates of traits primarily under natural selection (for example, morphology, dispersal ability, or climatic niche) may drive speciation. Other studies have suggested that variation in evolutionary rates of traits primarily under sexual selection (for example, plumage, song, or behavior) may drive speciation. However, few studies have compared the effects of multiple traits on speciation or have quantified the relative importance of natural and sexual selection in shaping patterns of diversification. We compiled four datasets of traits under natural and sexual selection (plumage, niche, song, morphology) that have previously been linked to speciation rates in the largest family of songbirds, the tanagers. We used several phylogenetic comparative methods (phylogenetic multiple regression, phylogenetic structural equation modeling) to explore the relative contributions of plumage, niche, song, and morphology to speciation in the tanagers while also modeling evolutionary correlations between traits (for example, beak size constrains song frequency, plumage is shaped by habitat) and their effects on speciation rates. We show that the most variation in speciation rates can be explained by rates of song evolution, and that more variation is explained by the combined effects of signaling traits (song and plumage) than by morphology and niche. However, as a whole the best-fit model included all four trait types. Our results suggest that speciation at macroevolutionary scales involves a complex interplay between traits under natural and sexual selection.}

\normaltalk{\capitalisewords{Signal Evolution and Morphological Complexity in Hummingbirds (Aves: Trochilidae)}}{\underline{Chad M Eliason}, Rafael Maia, Juan L Parra, Matthew D Shawkey}{The origins of morphological and functional novelty remain central questions in evolutionary biology, and rapidly-evolving sexually selected traits are particularly understudied in this regard. Hummingbirds have some of the brightest and most iridescent colors in nature. Their feathers contain optically complex stacks of hollow, platelet-like, melanin-filled organelles called melanosomes, but neither how these produce colors nor their evolution has been systematically studied. We first used nanoscale morphological measurements and optical modeling to identify the physical basis of color production in 34 hummingbird species. We found that in general the melanosome stacks function as multilayer reflectors, with platelet thickness and air space size explaining variation in hue (color) and saturation (color purity). Additionally, the outer keratin cortex both affects saturation (through modulating primary peak width) and, in conjunction with small, previously-undescribed melanosomes that lie below it, produces secondary reflectance peaks. We then compared evolutionary rates of both the morphological components and the colors (at a wavelength-specific level) they produce. The outer keratin cortex evolved at a faster rate than the other morphological traits, likely because of its evolutionary independence. Intriguingly, shorter wavelength colors evolved faster than long wavelength colors, perhaps due to sensitivity biases in bird vision or because of a developmental process that enables greater lability of smaller melanosomes. Together, these data demonstrate that morphological complexity enables greater evolutionary lability and potential for new functionality.}

\normaltalk{\capitalisewords{Speciation on an Island: Genetic Differentiation in Penguins and Shags}}{\underline{Theresa M Burg}, Juan Carlos Guevara-L'{o}pez, Naomi Perkins, Charles A Bost}{Morphological and behavioural differences exist in many species.  In rare instances these differences correspond to genetic differences at neutral molecular markers.  On Kerguelen Island in the southern Indian Ocean, two species of seabirds, Kerguelen shag (Leucocarbo verrucosus) and gentoo penguin (Pygoscelis papua), exhibit similar divergent behavioural and morphological patterns.  Both species appear to have two main ecotypes based on foraging strategy.  Colonies are either comprised of individuals who forage shallow waters or offshore in deeper waters.  Differences in foraging strategy appear to be stable within a colony over time and correspond to morphological differences.  On Kerguelen the differences between the ecotypes are so extreme that shags from 'inshore' colonies may be 60\% lighter than individuals from 'offshore' colonies.  To determine if these two foraging strategies have led to genetic differentiation, we used neutral microsatellite markers to examine individuals from colonies containing the different ecotypes.}

\normaltalk{\capitalisewords{Assessing the Influences of Habitat Structure on Bird Song Propagation}}{\underline{Braelei M Hardt}, Lauryn Benedict}{Many avian taxa produce long-range acoustics signals, in the form of songs, for the purpose of communication within and across territories. These signals play an important role in mate attraction and territory defense, which are key behaviors for reproductive success. The Acoustic Adaptation Hypothesis (AAH) posits that long-range acoustic signals are adapted for minimized signal degradation in the environment in which they are produced, leading to optimized communication within a species' preferred habitat and therefore higher reproductive fitness. While previous playback experiments have found some evidence for the AAH in 'open' vs. 'closed' habitats, this study focuses on testing signal degradation in a more diverse group of spatial and temporal habitats across a set of songbird species which exhibit large variability in both song structure and habitat preference. We used PCA on 22 habitat characteristics in four distinct habitats, both during and after the breeding season, and determined that habitats are most differentiated by their density and reflective/topographical properties. We then quantified three measures of signal degradation and LME models to determine how they relate to these environmental descriptors. We found that our three measures of signal degradation, which quantify changes in signal strength and shape over distance, are associated with separate and sometimes conflicting environmental aspects. This indicates that birds may not be able to optimize all three measures of signal quality for one specific environment, and may face trade-offs between producing songs that are either loud, clear, or separate from noise in their preferred habitat.}

\normaltalk{\capitalisewords{Using DNA Metabarcoding to Determine Seaside Sparrow Diet Following Large-Scale Disturbances}}{\underline{Allison M Snider}, Andrea Bonisoli Alquati, Stefan Woltmann, Philip C Stouffer, Sabrina S Taylor}{Seaside Sparrows (Ammospiza maritima) in the northern Gulf of Mexico rely on coastal saltmarshes that face a variety of threats. Disturbances like oil spills and hurricanes directly affect sparrows and may also change the composition of their invertebrate prey base. As upper-level predators in the saltmarsh, sparrow response to these disturbances is a unique indicator of how changes at the base of the food web can extend to higher trophic levels. To identify plasticity in sparrow feeding ecology and infer sparrow response to potential food web disturbances, we studied species of prey consumed by sparrows from 2011-2017, allowing us to identify impacts of the Deepwater Horizon (DWH) oil spill (2010) and Hurricane Isaac (2012). We used DNA metabarcoding to determine species consumed by sparrows from over 280 diet samples taken from oiled, unoiled, and control sites. Diet varied among individuals, but the majority of prey DNA sequences belonged to crabs, spiders, and Lepidoptera, suggesting these are the most important prey taxa. Despite documented changes to invertebrate communities following the DWH spill, preliminary analyses suggest diet composition did not differ between oiled and unoiled sites—corroborating findings from previous stable isotope and fatty acid analyses. While oiling alone does not appear to be a significant predictor of diet, temporal variation was significant. Diet composition in 2011, 2012, and 2013 was different from all other years, with apparent stabilization in 2014. No differences occur among 2014-2017, suggesting resources stabilized or all sites reached a new baseline following oil redistribution by Hurricane Isaac.}

\normaltalk{\capitalisewords{The Common Yellowthroat Genoscape- a Genomic Approach to Delineating Conservation Units}}{\underline{Christen M Bossu}, Jasmine Rajbhandary, Thomas B Smith, Kristen Ruegg}{The use of genomic data has power to combine information from neutral and adaptive loci to delineate conservation units and detect adaptive differentiation within a conservation genetics framework. Here we utilize bird genoscapes - innovative and cost effective genomic and GIS-based tool- to identify spatial and temporal structure of conservation units in a migratory bird species that is numerous but showing a patchy and often marked population declines, the Common Yellowthroat. We used low coverage whole genome re-sequencing methods to map genetic variation across geographic space and define conservation units at finer spatial scales than previously possible. We then used the genomic information to design high-resolution genetic tags which allowed us to screen the DNA contained within a single feather of a living bird or carcass collected away from the breeding location to identify the breeding population of origin. As a consequence, we have a snapshot of the entire annual cycle of Common Yellowthroats. Ultimately, we have a map of population-specific migratory routes and thus a better understanding of where the population declines are occurring, a task that has proven difficult in migratory birds that move vast distances between breeding and wintering grounds.}

\normaltalk{\capitalisewords{Habitat Modeling for Desert Thrashers}}{\underline{Dawn M Fletcher}, Lauren B Harter, Elisabeth Ammon}{Avian communities in the Mojave and Sonoran Deserts have experienced severe population declines in recent decades, and Bendire's and LeConte's thrashers are two of the fastest-declining species in these aridlands. LeConte's and Bendire's thrashers are recognized by Partners in Flight as species at risk, with LeConte's thrashers experiencing a 67\% decline and Bendire's thrashers a 90\% decline since 1970. These thrashers are traditionally under-sampled in bird monitoring programs due to low population density, early breeding phenology, and relative inaccessibility to large expanses of roadless areas in the desert. The Desert Thrasher Working Group (DTWG) was formed in 2011 to raise awareness, conduct habitat suitability modeling, and enhance monitoring strategies for these taxa. Nineteen government agencies and partners that span the species' ranges collaborated to develop best thrasher monitoring and management practices. The DTWG has established a standardized field protocol for inventory and monitoring, developed species distribution models to guide random sampling, conducted initial surveys, and generated regional habitat suitability models and accurate species distributions on a fine scale. Long-term goals include determining population sizes, monitoring trends, and refining existing habitat models that can lead to best habitat management practices. Partners of DTWG have conducted approximately 2,700 surveys on 911 plots across five states with 132 plots occupied by LeConte's thrasher and 59 occupied by Bendire's thrasher. In this presentation, we will discuss sampling design, survey results, and habitat models for these species, and illustrate how our initiative is effective for managing rare and elusive species.}

\normaltalk{\capitalisewords{Spanning the Habitat Gradient: Red-Headed Woodpecker Nest-Site Selection in Three Distinct Cover Types}}{\underline{Benjamin M Nickley}, Lesley P Bulluck}{Understanding species habitat relationships is fundamental to ecology and underpins conservation-based management.  Species with broad habitat preferences and wide geographic ranges are capable of fulfilling breeding requirements in different habitat contexts.  Habitat selection in these species may vary in a context-specific way, so determining the factors driving habitat selection requires spanning the habitat gradient.  Complex landscapes offer a unique opportunity to compare habitat selection across habitat gradients within a single study site.  Here, we used a comparative approach to model scale-dependent nest-site selection of the red-headed woodpecker in three distinct cover types at Fort A.P. Hill military installation, Virginia.  We found that nest preferences varied among cover types for several patch-scale habitat features. In closed forests, red-headed woodpeckers selected patches with an open canopy characteristic that lacked holly in the understory and contained large pines.  In open forests, patches with intermediate canopy cover were selected.  In wetlands, patches with low percent stem cover were preferred.  We also found a single variable, medium/large snag density, was highly influential in models for all cover types at the patch scale, indicating the importance of suitable snags for nesting, irrespective of cover type.  Models at the tree scale showed similar results for all cover types: red-headed woodpeckers consistently preferred large snags with less bark.  Our results demonstrate the value of comparing habitat selection across a wide habitat gradient.  Managers charged with promoting red-headed woodpecker populations would do well to manage for large, partially decayed snags while also considering context-specific habitat needs.}

\normaltalk{\capitalisewords{Survival, Habitat Selection, and Behavior of Post-Fledging Golden-Cheeked Warblers}}{\underline{Evalynn M Trumbo}, Michael P Brawn, Jeffrey Brawn}{Understanding associations between habitat and the demography of endangered wildlife is essential for effective management. Understanding this relationship is further complicated by the fact that different habitat features may have different impacts across various life-stages of an individual. Specific habitat features, such as tree composition and canopy cover, may influence survival in that different habitat provides cover and foraging substrate. This study was investigating the endangered golden-cheeked warblers (Setophaga chrysoparia) in Texas. To do this, we monitored nests until fledging and deployed a VHF transmitter per nest (n=8 and n=15, for 2017 and 2018, respectively). We tracked fledglings ~4 weeks after fledging. Fifteen of 23 (65\%) fledglings survived at least 32 days (the life of the transmitter's battery). Using Program MARK and linear mixed models we estimated nest and post-fledgling survival and investigated fledglings' relationships with differences in habitat, behavior, and an individual's age. Nest period survival was 0.24 (95\% CI 0.10-0.41) and fledgling period survival was 0.75 (95\% CI 0.49-0.89). Nest survival decreased over the breeding season, while fledgling survival was influenced by vegetation density. Fledglings also behaved differently as they age. As fledglings mature, they move to habitat with more understory density and canopy cover and areas with more oak and less juniper. Fledglings also moved greater distances and foraged more as they age. This study has identified certain habitat features (i.e., understory density) that managers could use to potentially create high-quality habitat for golden-cheeked warblers.}

\normaltalk{\capitalisewords{Patterns of Home Range Resource Use by the Volcano Junco, a Costa Rican Highland Endemic}}{\underline{Elizabeth Besozzi M Besozzi}, Michael Patten}{A chief conservation concern in a changing world is how fragile bird communities will respond to anthropogenic disturbance. Highland bird species are particularly at risk of dwindling habitat: as low elevation vegetation creeps higher, mountaintop habitats will shrink, and the species that depend on them may disappear. Given that the tipping point is unclear, understanding how the avifauna of montane habitats use and move through their environments takes on new importance. The Cordillera de Talamanca of Costa Rica hosts high numbers of endemic bird species, many of which occupy páramo habitat threatened by global temperature increase. Though relatively common above 2,700 m, little is known about the Volcano Junco (Junco vulcani), one such páramo endemic; records of its breeding behavior, resource use, and movement are, at best, anecdotal. Clearcutting within the Talamancan páramo has facilitated the downslope expansion of J. vulcani, suggesting that rather than selecting for páramo habitat, J. vulcani may be restricted to páramo by the structural complexity of the next highest vegetation strata, notoriously impenetrable oak grove. I spent three months delineating the breeding territories and cataloguing the resource use of J. vulcani at Cerro Buenavista and Cerro las Vueltas, an anthropogenically disturbed and undisturbed site, respectively, to provide insights into human-mediated differences in range expansion and habitat selection. Though the prognosis for J. vulcani is hopeful, this research highlights the species-specific idiosyncrasies of páramo bird ecology, and emphasizes the need for conservation protocols that address a variety of unique requirements.}

\normaltalk{\capitalisewords{Greener Pastures: Idiosyncratic Avian Responses to Early Forest Regeneration in the Maya Biosphere Reserve}}{\underline{Anna M Lello-Smith}, Amanda D Rodewald, Viviana Ruiz-Gutierrez}{The contribution of pastureland to tropical deforestation has prompted investment in a wide variety of restoration programs. Unfortunately, few restoration efforts have been rigorously evaluated for outcomes, including short- and long-term biodiversity values of restored lands. This gap in our understanding is critical because natural and facilitated recovery rates of forests and associated avian communities vary substantially across geographic regions, climatic conditions, and disturbance gradients. We surveyed resident bird communities in Guatemala's Maya Biosphere Reserve across a chronosequence of regenerating pastures and adjacent mature forest to evaluate species-specific responses to early forest regeneration. We combined point count surveys and historical land-use data to model avian occupancy as a function of pasture age (2-10 years old) and in comparison to mature forest. In relation to pasture age, species occupancy followed one of four patterns: (1) steep decline in pastures over 10 years of regeneration; (2) a quadratic response, with a peak at intermediate ages; (3) rapid increase with pasture age and forest regeneration toward the mature forest baseline; and (4) a weak response, with occupancy in older pastures remaining below mature forest occupancy. Contrary to expectations, response to pasture age did not vary consistently among foraging guilds - even for understory- and terrestrial-foraging insectivores that are generally considered sensitive to habitat disturbance. These results underscore the wide range of responses species exhibit to restoration and suggest that foraging guild, a categorization commonly applied to diverse tropical communities, may be insufficient to predict which species will benefit from early-stage restoration efforts.}

\normaltalk{\capitalisewords{Shrub Encroachment Alters Distributions and Demography of Breeding Birds in Semiarid Grasslands}}{\underline{Erik M Andersen}, Robert J Steidl}{Grasslands worldwide are transitioning from being dominated by perennial grasses to being dominated by woody plants.  Changes in the composition and structure of grassland plant communities in response to shrub encroachment can affect the quantity and quality of habitat available to grassland birds, which have declined more rapidly than other groups across North America.  In semiarid grasslands of southeastern Arizona, we established 140 10-ha plots that spanned a gradient of shrub cover on which we evaluated how shrub encroachment affected species richness, distributions, densities, and nesting success of breeding birds.  Shrub encroachment restructured the avian community by altering the distributions of constituent species markedly.  As encroachment progressed, overall density of birds and species richness increased until shrub cover reached ~25\%.  Grassland obligates, many of which have declined range-wide, were affected negatively however, with distributions and densities of most species decreasing markedly as shrub cover increased.  Increasing cover of woody plants on plots or near nests did not affect nesting success of the most common grassland obligates, shrub specialists, or groups of species based on nesting substrate (ground- and shrub-nesting species).  Woody cover influenced nesting success for only one species, mourning dove, for which nesting success increased as cover of woody plants increased at the plot scale.  Understanding how populations and communities of grassland birds are affected by shrub encroachment can inform conservation and restoration efforts by identifying achievable targets of shrub control that would benefit imperiled grassland species.}

\normaltalk{\capitalisewords{Engaging Undergraduate Students in an Authentic Course-Based Research Experience using Archived Nest Video Footage}}{\underline{Christine M Stracey}}{Calls for reforming undergraduate STEM education include an emphasis on authentic research where students experience the process from hypothesis generation, to data analysis, to conclusions. In this presentation I give an example of a course-based research experience where students in BIOL 334: Animal Behavior simultaneously generate data for my research while developing hypotheses for class projects using video footage of bird nests. As video monitoring technologies have advanced, the amount of data collected has increased exponentially. This rapid increase in data allows researchers to observe previously unknown behaviors but it can also be an obstacle to collection of relevant data.  One way to increase the amount of videos screened, is to involve undergraduate students in the processing of videos while engaging them in authentic research. To begin the semester-long experience, students choose a volume of video data to work with, read relevant primary papers for background information, and then develop research questions and data collection protocols. Students then focus on data collection and analysis based on their hypotheses. In addition to collecting their data, students also collect data for my continuing research projects. At the end of the semester, students present their cumulative work in a poster session with other biology classes. Quality control is maintained by agreement between student research pairs that is further verified by a research assistant. Qualitative student feedback on course evaluations has been overwhelmingly positive and I have the added benefit of being able to make progress on data mining for my own research.}

\normaltalk{\capitalisewords{Engaging with Veterinarians to Reduce Cat Overpopulation and Cat Impacts on Wildlife Populations}}{Lauren J Sherwood, Amy G Wilson, Cluny S South, Steven M Roche, \underline{Tanya M Luszcz}}{Outdoor-roaming cats present a multidisciplinary challenge for wildlife conservation and feline welfare.  Veterinary professionals are critical stakeholders in the outdoor-roaming cat issue, making it imperative that their attitudes and perspectives are well understood.  We conducted a survey of veterinarians across British Columbia, Canada to evaluate their perceptions of cat overpopulation, the impacts of cats on wildlife, and cat management strategies.  Fifty-six percent of respondents agreed that cat overpopulation was a local problem, and 91\% agreed that veterinarians play a role in reducing cat overpopulation.  Veterinarians were less cohesive in their perceptions of the impact of outdoor-roaming cats on wildlife. Fifty-three percent believed that after habitat loss, cats are one of the most significant direct causes of bird and wildlife mortality, whereas 30\% were neutral, and 17\% disagreed.  Fifty-six percent of veterinarians felt that keeping cats indoors would be one of the most effective strategies to reduce cat impacts on wildlife, but they also felt that public education programs and low-cost spay-neuter programs would be effective and more popular with cat owners.  Most veterinarians (79\%) stated that they recommend against unsupervised outdoor access, citing safety and disease risks.  Overall, our survey results suggest that veterinarians are committed participants in the alleviation of cat overpopulation, and that there are opportunities for conservationists to better engage veterinarians on the issue of cat impacts on wildlife.}

\normaltalk{\capitalisewords{Application of a Novel Nest Density Estimator: An Example using Sagebrush-Steppe Songbirds}}{\underline{Kaitlyn M Reintsma}, Lorelle I Berkeley, Victoria J Dreitz}{Studies show nest density is an important demographic rate for shorter-lived species, such as sagebrush-steppe songbirds. A recent analytical approach, temporal nest density estimator (TNDE), was developed to estimate the nest density of waterfowl species using data routinely collected to assess nest success while accounting for detection and availability of nests. To understand the general applicability of TNDE to species in other avian orders, we evaluated the performance of TNDE on a songbird species, Brewer's sparrow (Spizella breweri). We assessed the TNDE by comparing estimates of nest detection rate and nest density from TNDE to distance sampling methods for 43 Brewer's sparrow nests monitored in 2015. The TNDE method produced similar but more precise nest detection and density estimates than the distance sampling method. Now that TNDE has been validated, we plan on converting it to a Bayesian framework and using it in conjunction with fine resolution remote sensing data to determine nest-site selection for three sagebrush-steppe songbirds: Brewer's sparrows, McCown's longspurs (Rhynchophanes mccownii), and vesper sparrows (Pooecetes gramineus). These three species are representative of sagebrush specialists, grassland specialists, and generalists that use both, respectively. This study will allow us to explore the capability of TNDE and remote sensing data to assist in habitat selection studies. These methods may allow more accurate nest density estimates on broader scales with less effort, which will aid in identifying priority areas for conservation and management.}

\normaltalk{\capitalisewords{Nesting Sites and Declining Snowpack Co-Limit Habitat of North Americas Highest-Elevation Breeding Bird}}{\underline{Tim M Brown}, Karim Hanna, Erika Zavaleta}{Diminishing snowpack in mountains throughout the world is expected to negatively impact many alpine organisms. In western North America snow pack is predicted to decline as much as 80\% by 2100. The Sierra Nevada Grey-crowned Rosy-finch (Leucosticte tephrocotis dawsoni) breeds at the highest elevations within its narrow geographic range of the Sierra Nevada and White Mountains.  The Rosy-finch is an omnivore, foraging on aeolian invertebrates and plant seeds found on and around snowfields in the alpine ecosystem. Based on predicted habitat suitability we chose 15 sites above tree line throughout the Sierra Nevada and White Mountain ranges of California.  We conducted point count surveys at 250m intervals along 2km transects to assess Rosy-finch occupancy and collected data on environmental variables including invertebrate abundance, distance to snowpack and cliffs along each transect.  Within 200m of cliffs, we observed an increased probability ($>$30\% detection) and 100m of snow (~40\% detection) of detecting a Rosy-finch. We found that the probability of detecting Rosy Finches decreased with distance away from both cliffs (logistic regression, p=$<$0.0001) and snow (p =$<$0.0001), indicating that the Rosy-finch could depend on sites where snowpack, observed to be a primary foraging habitat, and cliffs, used for nest sites, co-occur at high elevations.  These findings inform development of a framework to understand Rosy-finch response to rapid climate change.}

\normaltalk{\capitalisewords{Differential Effects of Landscape Composition and Patch Size on Avian Habitat Use of Restored Fields in Agriculturally Fragmented Landscapes}}{\underline{Bryan M Reiley}, Thomas J Benson}{Existing private-land conservation programs play an important role in restoring wildlife habitat in agriculturally fragmented landscapes. These conservation programs are generally implemented for a variety of environmental purposes in addition to providing wildlife habitat and as a result private-land programs are often implemented without identifying target species. Conservation benefits of private-land conservation programs could be improved if the needs of declining species were used to guide future enrollments and management of habitat created through these programs. To improve implementation and management of these programs, we examined what within-field, patch, and landscape features affected habitat selection by 6 conservation priority bird species. Specifically, we were interested in understanding how species responded to patch size, surrounding privateland conservation habitat, and the degree to which the amount of surrounding cropland moderated the response to the amount of conservation habitat. To do this, we conducted bird surveys at 172 private-land fields during 2012–2015. While most species were positively associated with patch size, only the Dickcissel (Spiza americana) was strongly positively associated with this variable. Dickcissel (+) and the Field Sparrow (Spizella pusilla) (-) had strong relationships with the amount of surrounding private-land conservation habitat, although we found no evidence that benefits were greater in landscapes with higher amounts of row-crop agriculture. Contrary to previous studies, our results suggest that for most of our focal species, targeting fields in areas with existing private-land conservation habitat and creating larger patches does not improve the benefit of these programs, although maintaining and placing them in grass-dominated landscapes will.}

\normaltalk{\capitalisewords{Nocturnal Flight Calls Influence Vulnerability to Artificial Light and Building Collision}}{\underline{Benjamin M Winger}, Brian Weeks, Andrew Farnsworth, Andrew Jones, Mary Hennen, David Willard}{Understanding interactions between biota and the built environment is increasingly important as human modification of the landscape expands in extent and intensity. For migratory birds, collisions with lighted structures are a major cause of mortality, but the mechanisms behind these collisions are poorly understood. Using 40 years of collision records of passerine birds, we investigated the importance of species' behavioral ecologies in predicting rates of building collisions during nocturnal migration through Chicago, IL and Cleveland, OH, USA. We found that use of nocturnal flight calls is an important determinant of collision risk in nocturnally migrating passerine birds. Species that use flight calls during nocturnal migration collided with buildings more than expected given their local abundance, whereas those that do not use such communication collided much less frequently. Our results suggest that a stronger attraction response to artificial light at night in species that produce flight calls may mediate these differences in collision rates. Nocturnal flight calls likely evolved to facilitate collective decision-making during navigation, but this same social behavior now exacerbates species' vulnerability to a widespread anthropogenic disturbance. Our results also suggest that social behavior during migration may reflect poorly-understood differences in navigational mechanisms across lineages of birds.}

\normaltalk{\capitalisewords{Colorful and Calm or Angry and Ornamented: Correlated Nest Defense and Uv and Melanin Ornaments in Eastern Bluebirds}}{\underline{Lauren M Gillespie}, Lynn Siefferman}{In times of changing climate and shifting environmental conditions, understanding individual variability and flexibility of relationships between physiology and certain behaviors crucial to survival and reproduction are more important than ever. In bluebirds, ultra-violet (UV) blue plumage brightness and chroma are sexually selected ornamental traits indicating male competitive ability among conspecifics. Defense against predators may be just as important, and research here investigates ornament signaling in relation to predator defense behavior. Here, we present data from a Mississippi population of bluebirds inhabiting two sites where mated pairs were exposed to simulated predator intrusions in 2013 and 2014 using an American crow decoy during the nestling phase, and aggressive and vigilance behaviors were recorded. Plumage samples were collected and analyzed for UV reflectance and melanin ornament intensity. We found darker melanic colored and more UV ornamented males are less aggressive at both sites, and relationships are stronger at one site. More aggressive females at another site are more UV ornamented and more melanin pigmented, while less aggressive females at the other site are more UV ornamented and less melanin pigmented. Plumage does not correlate with habitat features of either site while defense behaviors do. Results here prompt discussion of how differing environmental or sexual selection pressures may act more strongly on females presenting unique reproductive issues regarding functional hormonal mechanisms in having or maintaining flexibility in sexually selected plumage traits, especially when correlating with aggression.}

\normaltalk{\capitalisewords{Hybridization Between Red-Crowned and Lilac-Crowned Parrots in Southern California}}{\underline{James M Maley}, Margaret E Schedl, Brooke Durham, Whitney LE Tsai, Ryan S Terrill, Kimball L Garrett, John E McCormack}{Southern California is home to a large population of naturalized parrots, all former pets or descendants of pets. There are at least six species of Amazona parrots that are known or suspected of breeding in Los Angeles County. The most abundant species is Red-crowned Parrot, numbering in the thousands, followed by Lilac-crowned Parrots. These two close relatives are allopatric in their native ranges, but breed in the same areas of Los Angeles. Lilac-crowned Parrots are often found at higher elevations in their native range, and have been found breeding in the mountains above Los Angeles. Phenotypic hybrids have been suspected to occur but we wanted to determine if the two species are hybridizing based on genomic data. To determine the level of genetic admixture we sequenced ultraconserved elements for 4 Lilac-crowned and 4 Red-crowned parrots collected between 1934 and 1955 from their native ranges. We also sequenced UCEs for 22 Amazona parrots salvaged in Los Angeles and San Diego. Based on 25,647 single nucleotide polymorphisms we found evidence of several hybrids between the two species. There appears to be limited introgression from Lilac-crowned Parrots into Red-crowned Parrots, the extent of which requires further study to determine. It remains to be seen if species boundaries will remain, or if the parrot flock of Southern California will become a huge hybrid swarm.}

\normaltalk{\capitalisewords{Phylogenomic of Euphoniinae Subfamily (Aves:Fringillidae): a Taxonomic and Biogeographic Review}}{\underline{Alma M Vazquez Lopez}, Blanca E Hernandez Ba\~{n}os}{Euphoniinae is a subdivision of the Fringillidae family; it is formed by the genera Euphonia and Chlorophonia. Chlorophonia genus has five species and with a green pattern coloration. In contrast, Euphonia genus is a large group with different coloration patterns and with 27 species. The relationship between Euphoniinae members is not determinate. A Phylogeny of Fringillidae showed that the blue-crowned Euphonias are closer related with Chlorophonia genus than with the Euphonia species. Based on that the IOC proposed to reorder the subfamily in Chlorophonia, Euphonia and Cyanophonia (for blue-crowned Euphonias) genera. The goals of this work are to obtain a phylogenetic of Euphoniinae and propose a new taxonomic arrangement. We obtained large sampling tissues for Euphoniinae thanks to many Scientific Collections. The DNA was extracted and sent to SNPsaurus Company to obtain a nextRAD genotyping. The genotyping analysis was done with BBMaptools (IT 0.88) and the genotyping calling with Samtools. The partitions and GTR models were obtained with PartitionFinder2. The phylogeny was done by Maximum Likelihood with RaxML. The phylogeny has two principal clades Chlorophonia-Cyanophonias and "true" Euphonias. These clades are divided into them. The first one is divided into a Chlorophonia and Cyanophonias clade. The second clade is divided also for two clades, into these clades is not possible to match a pattern coloration, but personal specimens reviews and colorimetry measurements suggest an apomorphic melanic orange in one of them. This preview results partially agree with IOC taxonomic and suggested a fourth genus into the Euphoniinae.}

\normaltalk{\capitalisewords{Genetic, Morphological and Color Variation in Piranga Bidentata (Aves:Cardinalidae)}}{\underline{Sahid M Robles Bello}, Blanca E Hernandez Ba\~{n}os}{Describing and explaining the origins of biodiversity are two of the chief goals of Systematics. As the biodiversity crisis worsens, establishing species limits accurately can help us make better-informed conservation decisions.   In this ongoing study we aim to establish species limits and explain the causes of color differences in Piranga bidentata, a polytypic songbird that inhabits both Mexican Sierras as well as Central America up to west Panama, with 4 described subspecies. While females through its entire distribution are yellow, the males on West Mexico are bright orange, and the males on East Mexico and Central America are blood red. Using genetic, morphometric, colorimetric and environmental data from museum specimens, we quantified variation and divergence between subspecies. Our prediction is that if these differences in color are caused primarily by random variation accumulated in geographic isolation, phylogenetic relations should explain most of the color variation. If these differences in color better are explained by climatic variables, this is better interpreted as the result ecological or adaptive processes.  Our preliminary results show that P. bidentata populations sort into at least two well-supported lineages based on mitochondrial DNA data. This corresponds to a West-East split in Mexico. We found that there's significant color and morphometric divergence between subspecies, and that this might be linked to environmental differences.}

\normaltalk{\capitalisewords{The {NSF} is Growing Pathways to {STEM} for Rural Community College Students via Community Partnerships and Intensive Mentoring}}{\underline{Lauren M Gillespie}, Steve Heinisch, Neil Grandgenett}{Best known for degrees and certificates in engineering, technology, computer science, or manufacturing areas, community colleges play integral roles in education pipelines of clinical, biomedical, and public health occupations, many of which require a baccalaureate and/or other advanced degree. Community colleges are necessary for the continued evolution of future generations of scientifically literate citizens and help prepare undergraduate STEM students for entrance into globally-competitive workforces. Production of individuals valuing and utilizing quality education in STEM areas in rural states is imperative at a time when scientific literacy is critical to global and public health.   Here, we describe a National Science Foundation-funded scholarship program investigating outcomes of relationships between 1st and 2nd-year student research cohorts, mentors, and local industries. Central Community College (Columbus, NE) is one of select community colleges awarded in the NSF S-STEM funding initiative, largely due to college-industry partnerships, and established internal funding. These opportunities bridge established gaps to low-income, high-achieving student success and retention in STEM. 'Project GPS' (Growing Pathways to STEM) provides student development of applied, critical thinking skills, fostering resilience, and, utilizing teamwork and creativity through 1) exposure to empirical avian research, 2) multi-institution, interdisciplinary research collaboration, 3) unique exposure to professional development activities and paid internship opportunities, and 4) a wholistic-mentoring perspective tailored to realities community college students face. Establishment of similar programs in rural areas can increase access to science education while examining barriers to STEM success in at-risk student populations relatively independent of financial burdens and academic prowess.}

\normaltalk{\capitalisewords{The Interactive Effects of Fire and Recreation on Golden Eagles}}{\underline{Caitlin M Davis}, Julie A Heath}{Sagebrush-steppe ecosystems, and species within them, are vulnerable to multiple threats, though few studies have examined how threats may interact to affect ecosystem dynamics. Wildfire and outdoor recreation negatively affect golden eagle (Aquila chrysaetos) territory occupancy and reproduction through habitat loss and disturbance, respectively. The focus of our research is to determine whether these threats may have synergistic impacts on eagle nesting ecology. We used a before-after-control-impact design to answer research questions at 22 historical eagle territories in southwestern Idaho. In 2015, wildfire burned 14 territories and 8 remained unburned. We collected data on recreation and eagle territory occupancy, reproduction and diet in 2017 and 2018 and compared these data to pre-fire levels gathered in 2013 and 2014. Our results show that motorized and non-motorized recreation shifted after the fire and recreation volumes were generally higher in unburned areas than burned areas. Regardless of whether an area had burned, off-road vehicle recreation negatively impacted eagle territory occupancy. However, the effect of pedestrian recreation depended on fire. Before fire, pedestrian use negatively impacted the probability that eagles laid eggs. After fire, pedestrian use decreased in burned areas which alleviated the negative impact on egg-laying.  Finally, in areas with low recreation, eagles were more likely to feed their young pigeons, which are a vector for the disease trichomoniasis that kills nestling eagles. Overall, our results suggest that the effects of recreation disturbance overshadows other impacts to eagle ecology, and that there is a mosaic of stressors that threaten eagles across the landscape.}

\normaltalk{\capitalisewords{Annual Variation in Use of a Spring Stopover Site by Three Migratory Shorebirds}}{\underline{Anna M Tucker}, Conor P McGowan}{Understanding how stopover site conditions influence both demographic rates and migratory behaviors is critical for interpreting changes in passage population sizes at stopover sites and predicting population responses to future changes. We used an open robust design model to estimate within-year and between-year parameters related to survival and stopover site use, including temporary emigration and transience, for migratory birds during stopover. We analyzed mark-resight observations of red knot (Calidris canutus rufa), ruddy turnstone (Arenaria interpres), and sanderling (Calidris alba) using Delaware Bay during spring stopover from 2005-2018. For ruddy turnstone, the probability of returning to this site if present in the previous year was positively associated with the availability of a key prey item in the previous year (β = 0.26, f = 0.98), but annual survival probability was not associated with prey availability (β = 0.07, f = 0.55). For red knot, however, both apparent annual survival probability (β = 0.36, f = 0.95) and the probability of returning to the site (β = -0.36, f = 0.94) were associated with prey availability. Shorebird use of this site typically peaked during May 26-28, but the proportion of the population present during this period varied dramatically for red knot (range: 0.07-0.59) but less so for ruddy turnstone (0.24-0.43) and sanderling (0.18-0.54). This demonstrates that conditions at this site have species-specific effects on demographics and migratory behavior, both of which should be considered in analyzing and interpreting monitoring data.}

\normaltalk{\capitalisewords{Migratory Bird Use of Stopover Sites in Midwestern Forests and Urban Habitats During Spring and Fall}}{\underline{Jessica M Outcalt}, Jeffrey J Buler, Jaclyn A Smolinsky, John B Dunning}{Populations of migratory birds are in decline, and their conservation often depends on understanding the plethora of habitats used throughout their annual cycles. The migratory period is an especially risk-prone and dangerous time for landbirds. Stopover sites along the migratory route are important conservation targets, especially within heavily developed landscapes. Our goal in this project was to identify important stopover habitats in Indiana, a highly developed and fragmented region. We used two NEXRAD weather surveillance radars in northern and central Indiana to observe stopover sites from which birds were taking off during both spring and fall migration. We also monitored changes in stopover habitat associations over a recent twelve-year period, thus tracking both spatial and temporal variation in migratory bird activity and habitat use during spring and fall. As expected, high densities of migratory birds utilized hardwood forests and riparian areas as stopover habitat. Somewhat unexpectedly, we also found consistently high densities of migratory birds in urban areas. Anthropogenic light at night is a relatively novel hazard for nocturnally migrating birds, since it disrupts their navigational systems. It is possible that nocturnal migrants passing through Indiana are disoriented by anthropogenic light at night and drawn into lower-quality urban stopover habitat. However, an alternate explanation could be that urban forests in Indiana actually provide adequate refueling potential for migrants and should be considered conservation and management targets.}

\normaltalk{\capitalisewords{Conservation in the Context of Migratory Connectivity: A Golden-Winged Warbler Case Study}}{\underline{Amber M Roth}, Anna Buckardt Thomas}{Understanding migratory connectivity is a difficult challenge, yet, it is central to effective full-annual-cycle conservation of migratory species. Golden-winged Warbler (Vermivora chrysoptera) is a Neotropical migrant with two distinct breeding populations in the Great Lakes and Appalachian Mountains, and is a species of conservation concern. There is a growing body of research suggesting strong migratory connectivity structure in the species, with Great Lakes breeders wintering primarily in Central America and Appalachian breeders wintering almost exclusively in northern South America. Our research fills geographic gaps in past studies by deploying archival light-level geolocators on male Golden-winged Warblers at four previously unmonitored breeding areas near the edge of the breeding range. We tagged birds in northern Wisconsin, western Upper Peninsula of Michigan, northwestern Vermont, and southeastern New York. We expected to either confirm strong migratory connectivity structure found in other studies or uncover new intricacies of Golden-winged Warbler migratory connectivity. Analysis of light-level data from 17 male Golden-winged Warblers revealed some exceptions to the well-accepted "rules" of Golden-winged Warbler connectivity. Notably, a lack of distinct migratory connectivity structure, with birds breeding in the Upper Peninsula of Michigan wintering in Panama (n=1) and Colombia (n=1), and Vermont breeders wintering in Costa Rica (n=2), Panama (n=1) and Colombia (n=2). Despite these exceptions, we propose re-examining the way we define Golden-winged Warbler populations and associated conservation regions by including migratory connectivity structure as a criterion.}

\normaltalk{\capitalisewords{Full Annual Cycle Ecology of Migratory Birds in the Lower Great Lakes}}{\underline{Christopher M Tonra}}{Increasingly, the bias in research towards the stationary portion of the breeding season in animal ecology is being recognized as a barrier to effective conservation. While breeding seasons limit productivity, survival can be most limited outside of the breeding season, and determining limiting factors during these stages can be critical to understanding population dynamics and habitat requirements. Furthermore, we lack a basic understanding of the ecology of most species outside of breeding, including relevant scales at which conservation must be enacted.  This is the case for many populations of migratory birds in the either breed, stage/stopover during migration, or overwinter in the lower Great Lakes. I will present an overview of full annual cycle research on migratory birds in the western basin of Lake Erie. This will include results from applications for modern tracking technology to bridge current information gaps in species such as Rusty Blackbird (Euphagus carolinus), Black-crowned Night-Heron (Nycticorax nycticorax), and Eastern Whip-Poor-Will (Antrostomus vociferous).  I will demonstrate how the scales at which nonbreeding habitat utilization is examined is often too limited, that there can be substantial variation within populations in nonbreeding behavior, and that we do not yet fully understand the extent to which stages of the annual cycle can interact.  It is critical that these types of studies continue in this region and beyond better understand avian ecology and optimize conservation efforts.}

\normaltalk{\capitalisewords{Bill Size Evolution in Response to Human-Mediated Ecological Change in a Tidal Marsh Songbird}}{\underline{Phred M Benham}, Rauri CK Bowie}{Tidal marsh birds have experienced a recent history of dramatic habitat loss coupled with increasing temperature and salinity, yet little is known about population responses to these impacts. Bill size increases in tidal marsh sparrow populations experiencing higher temperatures. This is thought to be an adaptation to increase heat dissipation from the bill surface, reduce reliance on evaporative cooling, and conserve water in these freshwater-limited habitats. We predict that warming temperatures in tidal marsh environments will select for larger bills. To test this, 290 Savannah sparrows collected from the California coast between 1889-1952 and post-2000 were measured. Modern birds showed a negative correlation between latitude and bill size compared to birds collected pre-1952, which have smaller bills and a positive relationship between bill size and latitude (time-period: F=21.9, p$<$0.0001; latitude: F=13.9, p=0.0003; time*latitude: F=20, p$<$0.0001). We found a significant correlation between bill size and temperature across the whole dataset (p=0.0005; R2=0.044). However, temporal variation in bill size was found to vary significantly with temperature change at some localities (San Francisco Bay: p=0.008; R2=0.062), but not in others (Humboldt Bay: p=0.83; R2= -0.02). Moreover, bills in northern populations became narrower over time with no change in length; whereas, southern birds evolved longer bills with no width change. These results suggest that warming temperatures along the Pacific coast of California have contributed to temporal changes in bill morphology; however, idiosyncratic population responses may reflect variation in other selective pressures (e.g., diet), population demography, or extent of gene flow from interior Savannah sparrows.}

\normaltalk{\capitalisewords{A Phylogenomic Tree of Screech-Owls: Elucidating the Evolution of Color Polymorphism and Its Ecological Correlates}}{\underline{Sarah M Kurtis}, Hannah L Owens, Jessica A Oswald, Peter A Hosner, Robert Guralnick, Julie Allen, Minji Ku, Edward L Braun, Rebecca T Kimball}{The existence of polymorphism in traits such as pigmentation may enable variable forms to exploit different niches via several mechanisms, such as crypsis in different light environments or energy homeostasis at varied temperatures and elevations. Screech-owls (genus Megascops) are highly polymorphic, demonstrating differential compositions of intraspecific color variation in approximately 75\% of species. While members of this genus inhabit a great diversity of habitats, the extent to which niche correlates with plumage pigmentation is unknown. To better understand the evolution of coloration and its ecological correlates, we sequenced thousands of ultraconserved elements (UCEs) in 32 individuals across virtually all species to construct a well-resolved phylogeny of Megascops. Increased subspecies sampling allowed for finer-scale resolution of widespread species. We explored whether conflicts between previous studies in the placement of the allopatric Western and Eastern screech-owl might be due to hybridization or other processes. Trait mapping suggested multiple losses of the red morph and several instances of the reemergence of the intermediate brown morph across evolutionary lineages. Niche modeling and correlation analysis demonstrated that species potentially having lost the red morph occur in habitats with significantly higher altitude and slightly higher variance in maximum temperature than those maintaining the red morph. However, no correlation between color morphology and precipitation was found. These findings may imply that physiological limitations linked with pigmentation rather than direct selection on coloration via crypsis are shaping the evolution of color polymorphism in owls.}

\normaltalk{\capitalisewords{Contrasting Drivers of Diversity and Turnover in Andean Birds and their Parasites}}{\underline{Sabrina M McNew}, Lisa N Barrow, John M Bates, Shane G DuBay, Spencer C Galen, Shannon J Hackett, Andrew B Johnson, Heather Skeen, Tomas Valqui, Weckstein D Jason, Jessie L Williamson, Christopher C Witt}{Changes in community composition from place to place, or turnover, can be used to identify mechanisms that underlie the origins and maintenance of diversity. Turnover can be influenced by abiotic factors, such as climate, or biotic factors, such as antagonistic interactions with competitors or symbionts. Here we investigate geographic variation and covariation between communities of birds and their haemosporidian blood parasites along elevational and latitudinal gradients in the Peruvian Andes. We screened ~4000 host individuals from 18 localities. We found ~1700 infections representing ~400 haemosporidian mtDNA lineages. We use phylogenetic generalized dissimilarity modelling (GDM) to test whether the parasite community composition was best predicted by geographic distance, aspects of host community composition, or environmental and climatic variation. We used similar models to test the predictors of host community composition. We found that bird community composition changed in response to elevation, climate, and geographic distance. In contrast, parasite community composition was best predicted by precipitation and host species turnover and was not strongly affected by geographic distance. Models of phylogenetic community structure suggested that fundamentally different processes affect parasite community composition at deep and shallow phylogenetic scales. Haemosporidian parasite communities generally tracked geographic variation in host community composition and diversity, but also appear to have been subject to evolutionary constraints related to environmental tolerances, vector availability, and phylogenetic limits to host compatibility.}

\normaltalk{\capitalisewords{The Genetics of Adaptation and Diversification of the "Great Speciators" - The Wallacean Zosterops}}{\underline{Luke M Bloch}, Mohammad Irham, Tri Haryoko, Cynthia Claypool-Wang, Jimmy A McGuire, Karen Rowe, Rauri CK Bowie}{This study on the genetics of adaptation and diversification in the avian genus Zosterops (White-eyes or Silver-eyes) represents among the first attempts at investigating the genomic landscape of a vertebrate group with the most rapid rate of diversification among vertebrates ( lineage diversification rate of 1.95-2.63 species per million years). Within the genus Zosterops there are ~100 described species that are thought to have arisen within the past 2 million years. We examine a remarkable avian diversification in the Indonesian Archipelago, and in particular the geologically and biologically unique island of Sulawesi. Members of the genus Zosterops are the first bird genus to be thoroughly genotyped from Wallacea and provide a contrast to recent studies of the genomes of the Galapagos finches and those of the New Guinea munias, two other important island radiations of birds. Using genomes to unravel the role of inter-specific hybridization and intra-specific adaptation among the members of the Sulawesi Zosterops species complex is revealing the evolutionarily and ecologically important loci underlying the processes that have allowed this group to radiate so successfully across the Old World. As such our results provide insight into avian diversification at large. Further insight into the importance of locally increased levels of differentiation within the genome (i.e. "islands of diversification"), and the role of genomic introgression on avian diversification has also been further elucidated.}

\normaltalk{\capitalisewords{Partitioning of Signal Space in a Breeding Warbler Community}}{\underline{Joanna M Sblendorio}, Maarten J Vonhof, Sharon A Gill}{The communication channel of vocalizing animals is called signal space, a multidimensional area defined by temporal and spectral signal features. The availability of signal space depends upon ambient noise, the number of competitors for the space, and the extent of song overlap among species within an acoustic community. To minimize acoustic interference, sympatric species should inhabit a well-defined acoustic niche. Partitioning of signal space results from signal evolution in favor of species-specific song, but also occurs through behavioral adjustments to avoid overlap. Additionally, songs of closely-related species may be less similar than songs of phylogenetically distant species, suggesting divergence of songs to prevent hybrid matings. We hypothesize that warbler species partition signal space, and test whether phylogenetic distance between species predicts overlap within that space. We recorded songs of 10-15 focal males per species from a community of warblers that co-occur at Fort Drum, New York. To generate signal space axes, we ran a principal components analysis of song characteristics, yielding three significant PCs which collectively explained 88\% of variation among species and reflected heavy loading of frequency traits (PC1), temporal traits (PC2), and bandwidth (PC3). Most species inhabited clearly partitioned acoustic space, with interspecific variation in area occupied and little overlap among species. Preliminary analysis of distances between acoustic niches suggests that songs of phylogenetically closer species are more similar than those of more distant species. Our study is a critical step in understanding partitioning of signal space in communities, and how signal overlap can drive signal evolution.}

\normaltalk{\capitalisewords{Morphologically, Genetically and Vocally Divergent Warbling Vireo Subspecies Exhibit Indiscriminate Aggression Toward Intruders in a New Contact Zone}}{\underline{Garth M Spellman}, Aja Hammond, Tammy Zhang}{Geographic variation within the Warbling Vireo (Vireo gilvus, Vieillot, 1808) has long been recognized by ornithologists. Two subspecies groups are often recognized: the swainsoni subspecies group, which includes the two western North American subspecies swainsoni and brewsteri, the gilvus subspecies group, which includes all the eastern North American birds. Historically these two groups have been separated by the Great Plains with some evidence of sympatry in southern Canada. The eastern subspecies, gilvus, has been expanding their range in the southern plains coincident with agricultural and urban expansion forming new areas of sympatry and potential hybrid zones along the Front Range of Colorado. In this study, we first examine morphological (measurements from museum specimens) and vocal differentiation (song data from xeno-canto) between two subspecies brewsteri and gilvus. Next, we perform aggression trials of territorial males along a transect in the new zone of sympatry in Colorado. We found strong, statistically significant evidence of morphological and vocal divergence between the two subspecies, which is consistent with previously published evidence of genetic divergence. However, the behavioral trials indicate that territorial males respond with equal levels of aggression to either subspecies song.}

\normaltalk{\capitalisewords{Circadian Rhythms, Melatonin and the Daily and Seasonal Control of Birdsong}}{\underline{Vincent M Cassone}, Clifford E Harpole}{Circadian rhythms are fundamental properties of most eukaryotes as well as some prokaryotes. These rhythms and the clocks that drive them synchronize overt behavior, physiology and biochemistry to temporal cues in the environment as well as coordinate complex circadian rhythms among tissues and cell-types so as to orchestrate efficient physiological processes on a daily and seasonal basis. In birds, these rhythms are particularly noticeable due to the predominance of diurnality of birds and to their vocal presence in our lives. Circadian organization in passerine birds is coordinated by at least 3 circadian pacemakers in the pineal gland, eyes and hypothalamic suprachiasmatic nuclei, whose mutual interactions are critical for stable rhythmicity and which are synchronized by ambient light: dark cycles through photoreceptors in the eyes, pineal gland and in several sites within the brain.  Recent work in our laboratory has been focused on mechanisms by which bird song and vocalizations are regulated by the clock via the circadian secretion of pineal melatonin. Bird song control structures express melatonin receptors as well as genes that have been associated with molecular clockworks such as period, cryptochromes and bmal1.  Pinealectomy abolishes circadian patterns of bird song, and daily administration of melatonin synchronizes patterns of vocalization. Administration of seasonally appropriate durations of melatonin induce changes in bird song control structures as well as seasonally appropriate vocalization.}

\normaltalk{\capitalisewords{Strategies for Navigating Trade-Offs Between Avian-Mediated Services and Disservices Affecting Food Production and Safety}}{\underline{Christina M Kennedy}, Christopher E Latimer, Olivia M Smith, Joseph M Taylor, Jeb P Owen, Erin Rankin, William E Snyder}{Wild birds can impart billions of dollars in services (e.g. pollination and natural pest control), and disservices (e.g. crop damage and pathogen transmission) to farmers each year.  A growing concern is that agricultural simplification and intensification can "tip the scales" in favor of species most likely to pose significant risks to food safety and production. Although trade-offs between avian-mediated services and disservices resulting from agricultural practices are increasingly recognized at local-scales, they have yet to be quantitatively evaluated across broad spatial extents to inform farm management and policy recommendations. Given the diversity of farming practices throughout the U.S., farmers need targeted yet generalizable recommendations that will enhance the long-term benefits provided by wild birds while minimizing their risks. Here, we will discuss a collaborative effort by The Nature Conservancy, Washington State University, and the Cornell Lab of Ornithology to 1) evaluate the ecological roles of wild birds in agroecosystems based on sampling across $>$50 diversified organic farms across the U.S. west coast; 2) quantify the impacts of local farming practices and the landscape context on multiple aspects of bird communities; and 3) scale up the local patterns observed on farms to the continental scale of the U.S. by harnessing the power of large-scale citizen science datasets and mapping tools. We will discuss how our findings shed light on the impacts of environmental changes and farming practices to prioritize actions that improve avian-mediated ecosystem service trade-offs and offer regionally-tailored bird-management advice.}

\normaltalk{\capitalisewords{Understanding Food Safety Risks Associated with Wild Birds: a Meta-Analysis and Conceptual Framework}}{\underline{Olivia M Smith}, Jeb P Owen}{Food borne illnesses caused by enteric bacteria threaten human health worldwide, and birds are suspected reservoirs for these pathogens on farms. Birds represent a special case of concern because they cannot be excluded from production areas by fencing, leading to recommendations of habitat removal on farms. This creates conflicts between food safety, avian conservation, and natural pest control objectives. Here, we conduct a comprehensive meta-analysis of prevalence data for Campylobacter spp., E. coli, and Salmonella spp., in 431 North American breeding bird species. We propose a comprehensive framework for understanding avian-mediated food safety risk that includes pathogen acquisition, reservoir competence, avian contact with food production, and bacterial survival and transmission. We place the literature into this framework, to identify important knowledge gaps that affect our understanding of the relationship between birds and foodborne illness. We found significant bias in which bird species have been tested, with most studies focusing on a small number of taxa that are common near people (e.g., European starling) or are in contact with human waste (e.g., gulls). We conclude that current data do not provide sufficient information to determine the effects of birds on enteric pathogen transmission and foodborne illness in people. We propose that future research focuses on species in contact with food production, and demonstrate shedding of pathogenic bacteria into the food production environment. Addressing these essential knowledge gaps will support sound biosecurity planning and enhance bird conservation efforts that are currently undermined by unsupported fears of pathogen transmission by birds.}

\normaltalk{\capitalisewords{Designing Coastal Conservation to Achieve both Ecological and Human Well-Being Goals}}{\underline{Gust M Annis}, Doug R Pearsall, Chris A May, Dave N Ewert}{It is increasingly recognized that conservation plans providing multiple benefits to people and nature gain broader acceptance with increased chance of success over plans that focus exclusively on ecological endpoints.  We define a process for implementing multiple-benefit conservation where components of human well-being were identified and used not as costs or threats to conservation, but as features that can be enhanced through improved ecological conditions.  Working within nearshore and coastal areas of the western Lake Erie basin (WLEB) we developed a conservation plan that incorporates regionally important human values identified through stakeholder surveys along with ecological priorities.  Our optimization incorporated a total of 26 features representing ecological and human well-being priorities (13 of each) and five costs.  The WLEB is an important area for migrating birds while concurrently providing regionally important economic revenues to the region through bird-watching opportunities, including birding festivals.  We incorporated bird habitat as an ecological priority while birding visits – identified using eBird data – served as one of our human well-being priorities.  The resulting plan identifies the most efficient locations for implementing conservation actions that meet ecological goals while sustaining or enhancing human values.  Results indicate that the areas contributing most to multiple goals are clustered along the coast and encompass coastal wetlands, bird stopover habitat, and important recreational areas. The plan has informed coastal restoration investments in Ohio and Michigan, and we continue to promote its adoption to partners in the region.}

\normaltalk{\capitalisewords{Forecasting Avian Responses to Climate-Mediated Increases in Fire Activity Across the Northwestern Boreal Forest}}{\underline{Steven M Matsuoka}, Peter S'{o}lymos, Amy L Breen, Colleen M Handel, T S Rupp, C L Mahon, Thomas A Kurkowski}{In the western boreal forest, wildfire activity has doubled in recent decades and is forecast to continue to increase as climate warming lengthens the fire season. Fires are the dominant disturbance of boreal forests and climate-mediated increases in fire activity will have widespread impacts on boreal fauna. We simulated the effects of future increases in fire activity (2010–2100) on the distribution and abundance of forest birds across the northwestern boreal forest region spanning Alaska, Yukon, and northern British Columbia. We developed species distribution models (SDM) that spatially predict breeding density across this study area for 25 boreal bird species using a large dataset of point-count surveys compiled across the continental boreal. We used these SDMs to map avian density relative to future landscape conditions simulated by ALFRESCO, which we used to model wildfire spread in future years as functions of climate and vegetation. Our results forecast substantial changes in potential population size for several boreal bird species.}

\normaltalk{\capitalisewords{Differences in Introgression Across Two Hybrid Zones in the Great Plains}}{\underline{Shawn M Billerman}, Jennifer Walsh, Bronwyn G Butcher, Irby J Lovette}{Hybrid zones – places where two divergent taxa come into secondary contact and interbreed – are useful to study the evolution of reproductive isolation. Many factors can contribute to the evolution of reproductive isolation in hybrid systems, including intrinsic and extrinsic mechanisms. By investigating genetic introgression across a hybrid zone, we can begin to understand those factors that contribute most to reproductive isolation in a system. Suture zones, where multiple hybrid zones between a variety of taxa cluster geographically, are well suited to test if shared mechanisms are important for broad scale patterns of species divergence. We studied two hybrid zones in one such suture zone in the Great Plains of North America. We sampled both Eastern (Pipilo erythrophthalmus) and Spotted Towhees (Pipilo maculatus), and Baltimore (Icterus galbula) and Bullock's Orioles (I. bullockii), across the same transect in the Great Plains, and tested how patterns of genetic introgression vary between species, and what that may tell us about how hybridization is maintained in the Great Plains. We used a RAD-seq approach to generate SNPs from across the genome of each hybrid zone and used these data to assess introgression. Across the towhee hybrid zone, we found extremely low levels of divergence, extensive introgression, and wide, shallow clines. Preliminary data from the oriole hybrid zone, however, show steep clines and much less introgression. These differences suggest that there are different selection pressures between these systems, and may help us better understand how and why these hybrid zones are maintained across the Great Plains.}

\normaltalk{\capitalisewords{Hybridization in Northern Flickers Reveals Loci Associated with Feather Coloration}}{\underline{Stepfanie M Aguillon}, Irby J Lovette}{Hybrid zones are useful laboratories for studying the relationship between genotype and phenotype in natural settings. The hybridization dynamics of the Northern Flicker complex has received considerable attention by ornithologists—primarily due to the conspicuous plumage differences and the geographically extensive hybrid zone between red-shafted (Colaptes auratus cafer) and yellow-shafted (C. a. auratus) flickers. Despite this extensive study and the distinct differences across multiple plumage traits, the flickers are extremely genetically similar and to date no fixed genetic differences have been identified. Here, we use low coverage whole genome re-sequencing of 20 allopatric and 48 hybrid flickers to explore the functional genomic bases underlying the phenotypic differences between the taxa. Using a dataset of 2+ million SNPs, we identify an extremely low baseline of differentiation across the entire genome punctuated by a few regions of elevated differentiation ("peaks"). Genome-wide associations using the hybrid individuals show that loci significantly associated with phenotypic differences are largely restricted in the genome to these peaks of elevated differentiation. Moreover, many of the identified loci are located near genes known to be involved in feather pigmentation or with functions related to feather development. These findings demonstrate the utility of using genomics in historically well-studied systems and show that natural hybrid zones can help to identify new candidates associated with phenotypic differences.}

\normaltalk{\capitalisewords{Sex Chromosome Inversions may Enforce Reproductive Isolation Across an Avian Hybrid Zone}}{\underline{Daniel M Hooper}, Simon C Griffith, Trevor D Price}{Across hybrid zones, the sex chromosomes are often more strongly differentiated than the autosomes. This is regularly attributed to the greater frequency of reproductive incompatibilities accumulating on sex chromosomes and their exposure in the heterogametic sex. Working within an avian hybrid zone, we explore the possibility that chromosome inversions differentially accumulate on the Z chromosome compared to the autosomes and thereby contribute to Z chromosome differentiation. We analyze the northern Australian hybrid zone between two subspecies of the long-tailed finch (Poephila acuticauda), using reduced representation genomic sequencing for 300 individuals over a 1500 km transect spanning the entire species range and use a SNP-chip assay to score for Z chromosome type for an additional 300 individuals from within the hybrid zone we identified. Autosomal differentiation between subspecies is minimal. In contrast, 75\% of the Z chromosome is highly differentiated and shows a steep genomic cline which is displaced 350 km to the west of the center of bill color admixture.  Differentiation is associated with two or more putative chromosomal inversions, each predominating in one subspecies. If inversions reduce recombination between hybrid incompatibilities, they are selectively favored and should therefore accumulate in hybrid zones. We argue that this predisposes inversions to differentially accumulate on the Z chromosome.}

\normaltalk{\capitalisewords{America's Wildlife Values: How Values Lead to Challenges and Opportunities for Bird Conservation}}{\underline{Alia M Dietsch}}{Research suggests that values (i.e., enduring, fundamental beliefs) held by Americans have shifted due to changes in everyday life associated with urbanization and societal-level increases in wealth and education (i.e., 'modernization').  In the context of wildlife conservation, a shift of values would ultimately lead to new public priorities regarding the treatment and management of wildlife, including birds. Our body of research suggests that a societal shift in values would be congruent with the rise of specific recreation behaviors (e.g., birdwatching) and conservation-relevant attitudes, such as support of restricting human activities to protect migratory birds. The focus of these attitudes and behaviors is steeped in mutualism values focused on the perceived needs of others, including non-human animals. The America's Wildlife Values study explored the distribution of wildlife values across the United States, and how those values lead to challenges and opportunities for wildlife management. Using data collected from 43,949 respondents representing all 50 states, this talk specifically focuses on results pertinent to bird conservation. The study highlights a system in which the process of modernization following World War II changed the social-ecological context that, in turn, affected people's awareness of and their experiences with wildlife. Our work underscores the importance of understanding what shapes values, as well as how values influence thought and behavior in the context of bird conservation.}

\normaltalk{\capitalisewords{Human Dimensions Applications in Shorebird Subsistence Harvest and Indigenous Knowledge in Alaska}}{\underline{Jacqueline M Keating}, Liliana C Naves, T L Tibbitts, Daniel R Ruthrauff}{Shorebird population declines worldwide have increased the need for knowledge and collaboration among stakeholders across migratory routes. Subsistence harvest of shorebirds in Alaska has occurred for millennia and is relatively small, but it includes species of conservation concern. The need for accurate harvest estimates and for developing communication with subsistence users on topics related to shorebird conservation required a human dimensions approach to understand local ethnotaxonomy, harvest methods and uses, local knowledge, and the cultural value of shorebirds in Alaska Native communities. Harvest estimates were based on surveys conducted in 1990–2015 (n=775 community-years). The estimated harvest was 2,783 shorebirds and 4,676 shorebird eggs per year. Data on ethnography and ethnotaxonomy were collected in key respondent interviews conducted in communities in the Yukon-Kuskokwim Delta region (n=72). We identified 24 Yup'ik shorebird ethnotaxonomic categories. This talk will highlight the study's human dimensions components, including ethnotaxonomic implications for harvest assessment, demographic and cultural influences on local interactions with shorebirds, opportunities for community-based conservation, and potential for collaboration among stakeholders throughout the East Asia-Australasia Flyway.}

\normaltalk{\capitalisewords{Interspecific Aggression and the Evolution of Avian Mimicry in Woodpeckers}}{\underline{Gavin M Leighton}, Eliot T Miller, Benjamin G Freeman, Alexander C Lees, Russell A Ligon}{The evolution of mimetic phenotypes represents an ideal scenario to understand evolution. While many cases of mimicry represent anti-predator adaptions via crypsis or warning coloration, many putative cases of mimicry in birds are not due to predation or crypsis. Instead, mimicry in birds may be due to subordinate species evolving plumage that resembles larger, more dominant species. We present evidence that overlap of ranges, as a proxy of social interactions, does predict the evolution of mimetic phenotypes among several complexes of woodpeckers, even after controlling for evolutionary relatedness and several ecological variables. Given the presence of social mimicry, we then focus on a single pair of species, Hairy Woodpeckers and Downy Woodpeckers to determine the target of mimicry. Using a massive behavioral dataset that was generated by citizen scientists we find that Hairy Woodpeckers target Downy Woodpeckers for aggression more than expected after controlling for abundance. However, we find that Downy Woodpeckers are more dominant than expected after controlling for phylogenetic position and body size, suggesting that Downy Woodpeckers are winning contests with species other than Hairy Woodpeckers more than one would expect by chance. The benefit of mimetic plumage in Downy Woodpeckers is therefore likely due to the effect of mimetic plumage on third party species; and we suggest that social interactions among third-party species are an underappreciated yet important factor for understanding the evolution of mimicry due to social interactions in avian species.}

\normaltalk{\capitalisewords{Revisionist History - Towards Comparable Species Across Avian Taxonomy}}{\underline{John M Bates}, Sushma Reddy}{Species are important units in biology and are critical for biodiversity science. Taxonomic classifications attempt to create lists of comparable basal units of evolution, including newly "discovered" species.  However, species descriptions have increasingly become undervalued in the modern scientific process. Evolution-focused journals tend to publish phylogenetic and phylogeographic analyses, but are loath to include species descriptions [to avoid potential controversies? For lack of space?, or because they are not exciting?]. Conversely, taxonomic authorities can be skeptical about genetic data and generally favor phenotypic information when it comes to distinguishing species. Much of the controversy, from our perspective, is not just on which species concept is favored, but rather how different types of information are applied. We propose that avian taxonomy needs to re-focus on evolutionary theory and approach species limits as hypotheses. Integrative taxonomy promotes the use of multiple types of data to illuminate on reproductive isolation. We currently have more data and data types than ever before. As we strive to understand the connections between phenotypic traits, ecological specifications, and their underlying genomic mutations, we need to integrate data types, which allows for more comparable units of evolution at the species level.}

\normaltalk{\capitalisewords{What Information Do We Need to Anticipate the Interaction Pattern of a Hummingbird?}}{\underline{Ana M Martin Gonzalez}}{Hummingbirds show an extremely high metabolic rate and almost entire dependency on floral nectar as food source, forcing them to be highly efficient foragers. Furthermore, strong differences in bill morphology, body size and foraging patterns, including several cases of extreme morphological match between the bill and the corolla shapes, suggest a high specialization in hummingbird-plant interactions. Examining a large dataset of hummingbird-plant interaction networks spanning across the Americas we found higher levels of interaction specialization and feeding niche partitioning in localities with a higher Late Quaternary climatic stability, warmer temperatures, higher precipitation and a higher species richness and hummingbird phylogenetic relatedness. That is, areas with sufficient species richness and stability to develop specialized interactions, and where insect pollination is hampered. Integrating interaction network data with ecological and morphological traits, we have been able to discern that this environment-network specialization relationship is highly context-dependent, and mediated – at least partly – by the filtering of hummingbird functional traits. Communities composed of functionally diverse hummingbirds show significantly higher levels of interaction specialization and niche partitioning. Moreover, hummingbirds with large distributions, migratory habits, and short-medium sizes tend to be more generalistic and flexible in their behavior compared to smaller range species or species with extreme morphologies, which tend to be more specialized. Integrating interaction data with ecological and morphological traits provides with valuable information that will help us anticipate species role within communities, putative future interactions and the success of invasive species.}

\normaltalk{\capitalisewords{Using Structured Decision Making to Connect Bird Monitoring to Restoration Decision-Making}}{\underline{Evan M Adams}, Auriel M Fournier, James E Lyons, Randy Wilson, Jeffrey S Gleason, Janell M Brush, Robert J Cooper, Stephen J DeMaso, Melanie J Driscoll, Peter C Frederick, Patrick G Jodice, David B Reeves}{In 2010 the Deepwater Horizon oil spill impacted habitat many bird populations across the Gulf of Mexico. At the time, there was little baseline information on current bird population status or an understanding how to effectively manage many species. To address these issues, scientists and managers across many institutions and agencies collaborated to form the Gulf of Mexico Avian Monitoring Network (GoMAMN). This organization focused on maximizing the relevance of bird monitoring data in the Gulf for population status assessments, informing management action effectiveness, and understanding how ecological processes affect birds. Translational ecology is at the core of this effort: the goal is to ensure bird monitoring projects inform actionable restoration decisions in the Gulf. The end-users of the products are bird monitoring practitioners, restoration funders, and the broader conservation community. Structured Decision Making (SDM) was used to incorporate a broad range of stakeholder perspectives into the decision process and create a decision framework to best achieve GoMAMN's objectives. After building the SDM decision-making framework with a core group, we expanded the process to include a broader range of stakeholders that we refer to as the Community of Practice (COP). Feedback was directly integrated into the SDM framework and used to weigh the importance of each of the fundamental objectives, thus creating a tool for designing a Gulfwide monitoring plan that representative of a diverse COP. This framework can be used to build ecological monitoring programs with broad support across the community that have clear connections to ongoing management.}

\normaltalk{\capitalisewords{A Strategic Plan for Monitoring Birds in the Gulf of Mexico: Translating Stakeholder Values into Priorities}}{\underline{Auriel M Fournier}, James E Lyons, R Randy Wilson, Jeffrey S Gleason, Evan M Adams, Janell M Brush, Robert J Cooper, Stephen J DeMaso, Melanie J Driscoll, Peter C Frederick, Patrick G Jodice, David B Reeves}{Conservation planning for large, dynamic coastal and marine ecosystems has multiple benefits, but is often challenging to implement. In addition, decision making can be hampered because many frameworks do not have a way of addressing and incorporating the multiple values and concerns of stakeholders. Prior to the Deepwater Horizon oil spill, avian monitoring projects frequently used study designs which were inconsistent with understanding species trends, response to management or understanding ecological processes at the Gulf of Mexico scale. In response to this, the Gulf of Mexico Avian Monitoring Network was created and subsequently, used the principles of structured decision making to identify stakeholder objectives and values, which in turn led to the identification of bird monitoring priorities.  This is all articulated in the Gulf of Mexico Strategic Bird Monitoring Plan through a series of taxa-based chapters, each led by a regional taxa subject matter expert. By using stakeholder objectives and values to identify bird monitoring priorities, practitioners and decision makers have: (1) a set of agreed upon objectives and core-values; (2) a transparent means of setting priorities across political and jurisdictional boundaries; and (3) a framework to facilitate communication and collaboration of data needs.}

\normaltalk{\capitalisewords{Development, not Bird Feeders, is Associated with Increased Occupancy of Northern Cardinals in Northern Michigan}}{\underline{Sarah M Toner}, David N Bonter}{Anthropogenic changes to ecosystems can influence species range distributions. Over the past decades, northern cardinals (Cardinalis cardinalis) have expanded their range dramatically into the Upper Midwest and Northeast. The mechanisms driving this expansion are unclear, but are commonly attributed to the provisioning of supplemental food through bird feeders and changes in habitat due to development. In this study, I examined whether urban development and/or the availability of supplemental food are linked to the range expansion of cardinals into northern Michigan. During the summer of 2018, I performed a series of point counts in six cities on the periphery of cardinal range in northern Michigan. I surveyed 132 points along twelve transects across urban-rural gradients and used song playback to enhance cardinal detection probabilities. I developed occupancy models to compare cardinal occupancy with level of development, the presence of feeders, average January nocturnal temperature, and latitude. Top-ranked models included development and latitude as predictors of cardinal occupancy, but the presence of feeders did not. Cardinal occupancy in general was higher at lower latitudes and showed a positive relationship with development at all latitudes across the edge of the species' range. I found no support for the hypothesis that the presence of feeders facilitates the range expansion of northern cardinals. Based on these results, I conclude that anthropogenic land use, and not expansion of bird feeding, is likely facilitating the cardinal range expansion.}

\normaltalk{\capitalisewords{Decline and Survival of North America's Migratory Avifauna Determined by a Weather Radar Network}}{\underline{Adriaan M Dokter}}{Global anthropogenic changes pose serious threats to many species of migratory birds. Yet, a comprehensive long-term and large-scale understanding of the population dynamics of migratory birds is currently lacking. Here we report observations of continental-wide biomass flow of avian migrants across the contiguous U.S. by a network of 143 weather radars. These radar networks can measure migratory fluxes of birds into and out of geographic areas in a standardized and continuous way, providing trends of abundance, productivity and mortality for the full migratory avifauna.  We found a steep decrease in the migratory biomass passing through the eastern US over the last 12 years, suggesting alarming population declines across the migratory avifauna. Similar declines were estimated from breeding bird survey (BBS) data, indicating population losses in many species, including still common species.  We also compared the spring versus autumn passage of biomass crossing coast-to-coast transects at different latitudes over the US. Such seasonal ratios can provide information on recruitment success and winter mortality, and their year-to-year variation. We find that the assemblage of migrants wintering south of the U.S. achieved higher survival during the combined period of migration and overwintering between autumn and spring transect passage, despite an average 3-4-fold longer migration distance compared to temperate wintering migrants. The expected higher mortality associated with longer migration was likely balanced by high survival in the (sub)-tropical non-breeding grounds.   Our results add impetuous to conservation efforts targeting the full north American bird community, including common birds.}

\normaltalk{\capitalisewords{American Seabirds: A Review of Status, Threats, and Conservation on Land and at Sea}}{\underline{Hannahrose M Nevins}, Holly Goyert, Brad Keitt, David Wiedenfeld}{Seabirds face a unique set of threats on land and at sea and are among the most imperiled groups of birds, with 28\% of 361 species IUCN-listed as Vulnerable, Endangered or Critically Endangered. We reviewed seabird population trends based on published studies from six regions in North and South America: Arctic (Bering Sea), NE Atlantic (Gulf of Maine), North Pacific (Gulf of Alaska, California Current), Caribbean and Hawaiian archipelagos, and Humboldt Current. The majority (70\%) of monitored seabirds are in decline, although some species are showing increases (e.g., Roseate Tern, Common Murre). Prior to the 1900s, population declines were caused by massive over-exploitation (e.g., egging, guano mining, and feather exploitation) and, more recently, unmitigated damages (e.g. oil pollution; bycatch in gillnets and high seas driftnets, and longlines; and overharvest of forage fish). Current major drivers of seabird population declines are from pervasive invasive mammalian predators on islands, and climate and fisheries impacts at sea. This tends to vary by region: climate-driven changes are apparent in productive high latitude systems, whereas on tropical islands, a legacy of introduction of non-native mammals continue to drive threatened species to the brink (e.g. Hawaiian and Black-capped petrels). Conservation initiatives have been successful to restore breeding colonies through island restoration, with the best examples of recovery of extirpated populations from Mexican Islands. Mitigation of fisheries bycatch has also proven successful, especially in industrial longline fisheries in the North Pacific and some Southern Ocean waters.}

\normaltalk{\capitalisewords{Wild Birds and One Health: Dynamic Boundaries Affect the Health of Birds, Livestock, Humans, and the Environment}}{\underline{Andrew M Ramey}}{The One Health initiative strives to forge collaborations between medical professionals, veterinarians, and environmental scientists to find solutions to the most pressing societal issues threatening the health of people, animals, and our environment.  Ornithologists may play an important role in the One Health movement as wild birds may serve as sources of zoonotic infections, reservoirs for domestic animal pathogens, and indicators of environmental change.  Previous studies provide evidence that the boundaries among wild bird, domestic animal, human, and environmental health are dynamic.  Examples include the intermittent spillover of disease agents across the wild bird-poultry interface, sporadic detection of common pathogens in birds and humans, and periodic avian mortality events associated with environmental perturbations.  This symposium will explore these dynamic boundaries by bringing together professionals from diverse clinical, governmental, and academic organizations to share their experiences conducting research on avian health and disease within the One Health framework.  Symposium sessions will be formatted in such a way to foster cross-pollination of ideas across disciplines and to encourage new collaborations.}

\normaltalk{\capitalisewords{Prevalence, Diversity, and Transmission of Blood Parasites in Alaskan Avifauna}}{\underline{Matthew M Smith}, Caroline Van Hemert, Brandt Meixell, Colleen Handel, Andrew Ramey}{Avian haemosporidian parasites are known to cause disease and mortality among certain bird species, particularly among naïve populations previously isolated from parasite exposure. Although a variety of factors can affect the prevalence and transmission of these parasites, temperature has been identified as a key parameter. Environmental changes related to climate warming are expected to influence the distribution and abundance of blood parasites and their associated vectors, particularly in northern regions where these changes are more pronounced. Until recently, little work had been done to evaluate the presence and potential impacts of Leucocytozoon, Haemoproteus, or Plasmodium blood parasites on Alaskan bird populations. Over the past decade, however, information has steadily grown, thus providing important baseline data on the prevalence and diversity of blood parasites in both migratory and resident Alaskan bird species. Here we summarize historical and contemporary findings on the prevalence, diversity, and transmission of haemosporidian parasites in Alaskan avifauna. Recent research on both landbirds and waterbirds has shown that parasite prevalence varies by region, host characteristics, and select environmental variables. Phylogenetic analysis of parasite DNA has also provided insights about parasite dispersal, species diversity, and host specificity of different parasite genera. We conclude by addressing predictions of haemosporidian parasite distribution related to projected environmental changes, possibilities for demographic effects on Alaskan bird populations, and the directions and perspectives for future research in the region.}

\normaltalk{\capitalisewords{It's not Just Flowerpots and Tires: Defining the Impacts of West Nile Virus on a Woodland Bird, Ruffed Grouse}}{\underline{Lisa M Williams}, Justin D Brown, Nicole M Nemeth, Michael L Hutchinson, Matt W Helwig, Andrew D Kyle}{West Nile virus (WNV) has increasingly been identified as negatively impacting a diversity of wild avian taxa.  Historically, little was known about WNV ecology in woodland habitats as nearly all public health WNV surveillance occurs in urban and suburban environments to monitor risks to human health. Since 2001, ruffed grouse (Bonasa umbellus) declines in Pennsylvania have been temporally-associated with WNV.  Ruffed grouse populations declined precipitously across Pennsylvania from 2002-2005 after initial introduction of WNV and have not recovered.  Pennsylvania Game Commission research has shown that ruffed grouse declines are associated with WNV activity in Pennsylvania, characterized the susceptibility of ruffed grouse to WNV, identified potential woodland vectors of WNV, and evaluated whether existing public health surveillance adequately represents WNV activity in woodlands. Research is now focused on determining if there are landscape barriers to disease prevalence and characterizing the woodland water sources that produce WNV vectors. The goal of this 4-year research effort is to define and mitigate the impacts of WNV on ruffed grouse. Up-to-date research findings will be presented, along with a discussion of steps that land managers can take to minimize WNV risks to woodland wildlife.}

\normaltalk{\capitalisewords{Lessons Learned from Research and Surveillance Directed at Highly Pathogenic Influenza Viruses in Wild Birds of North America}}{\underline{Andrew M Ramey}, Thomas J DeLiberto, Yohannes Berhane, David E Swayne, David E Stallknecht}{Following detections of highly pathogenic (HP) influenza A viruses (IAVs) in wild birds inhabiting East Asia after the turn of the millennium, the intensity of sampling of wild birds for IAVs increased throughout much of North America and the objectives for surveillance and research efforts were directed towards detecting Eurasian origin HP IAVs and understanding the potential of such viruses to be maintained and dispersed by wild birds, respectively.  In this review, we highlight five important lessons learned from research and surveillance directed at HP IAVs in wild birds inhabiting North America: (1) Wild birds disperse IAVs between North America and adjacent regions via migration, (2) HP IAVs may be introduced to wild birds in North America, (3) HP IAVs can cross the wild bird-poultry interface in North America, (4) The probability of encountering and detecting a specific virus may be low, and (5) Population immunity of wild birds may influence HP IAV outbreaks in North America.  We review empirical support derived from research and surveillance efforts for each lesson learned and, furthermore, identify implications for future surveillance efforts, biosecurity, and population health.  We conclude by identifying five additional areas in which we think future mechanistic research relative to IAVs in wild birds in North America are likely lead to other important lessons learned in the years ahead.}

\normaltalk{\capitalisewords{Long-Term Dynamics of Breeding Dispersal Distances by Adult Northern Spotted Owls}}{\underline{Julianna MA Jenkins}, Damon Lesmeister, Eric Forsman, Katie Dugger}{Dispersal in animals is driven by numerous selection pressures, including competition, and spatiotemporal variation in habitat quality. The magnitude of dispersal movements and trends in movement distance over time may signal changing conditions within the population or on the landscape. We examined breeding dispersal distances within a marked population of Northern Spotted Owls (Strix occidentalis) from 1990 to 2017 to determine the relative importance of social and temporal variables, including the expansion of Barred Owls (S.varia), on dispersal distances. We analyzed 2,078 breeding dispersal events in Oregon and Washington. The median breeding dispersal distance for all owls was 3.13 km with 99\% measuring less than 35 km. Social factors were most closely associated with breeding dispersal distance. Owls that were single prior to the dispersal event tended to move farther than those that were paired. Dispersal distance also declined as the amount of time spent on the previous territory increased.  We found support for a small increase in breeding dispersal distances over the time series, 2.2 km over 28 years, concurrent with increased Barred Owl densities and declines in Spotted Owl populations.}

\normaltalk{\capitalisewords{Gradual Evolution Towards Flightlessness in Steamer-Ducks}}{\underline{Leonardo n Campagna}, Kevin McCracken, Irby J Lovette}{Flightlessness in birds is the product of changes in suites of characters--including increased body size and reduced anterior limbs--that have evolved repeatedly and independently under similar ecological conditions (generally insularity). It remains unknown whether this phenotypic convergence extends to the genomic level, partially because many losses of flight occurred long ago (such as in penguins or ratites), thus complicating the study of the genetic pathways to flightlessness. Here we use genome sequencing to study the evolution of flightlessness in a group of ducks that are current and dynamic exemplars of this major functional transition. These recently diverged Tachyeres steamer-ducks differ in their ability to fly: one species is predominantly flighted and three are mainly flightless. Through a genome-wide association analysis we identify two narrow candidate genomic regions implicated in the morphological changes that led to flightlessness, and reconstruct the number of times flightnesses has evolved in Tachyeres. The strongest association is with DYRK1A, a gene that when knocked-out in mice leads to alterations in growth and bone morphogenesis. These findings, together with phylogenetic and demographic analyses, imply that the genomic changes leading to flightlessness in Tachyeres may have evolved once, and that this trait remains functionally polymorphic in two species.}

\normaltalk{\capitalisewords{Differences in Foraging Strategies among Common Loons Overwintering on a Large Freshwater Reservoir in Northwest South Carolina}}{\underline{John N Mager}, James D Paruk, Brooks Wade}{Most Common Loons that breed on freshwater lakes overwinter on marine waters along the coasts of North America.  We examined the activity budgets of loons that alternatively overwinter on Lake Jocassee, a large, 7500-acre reservoir in northwest South Carolina that was created in 1973 and currently serves as both a staging lake for migrants and an overwintering lake for a small, but notable (~150-200 individuals) loon population.  Loons overwintering here exhibit high site fidelity, and from initial investigations of the time-activity budgets of focal individuals (N=23), we found that individuals adopt one of two alternative lifestyles/strategies: some spend most ($>$75\%) daylight hours within small groups (2-7 individuals), usually within one of three major reservoir arms.  Others live solitarily, and move transiently throughout the reservoir, spending relatively little or no time with conspecifics. Group-living individuals dedicated a lower percent of daily activities (41.7\%) towards foraging than did solitary individuals (65.2\%; t17 = 3.03, P $<$ 0.01), and at times group-living individuals engaged in pronounced foraging bouts at the water surface.  For both solitary and group-living loons, individuals had shorter dives when foraging with another loon (53.6 s.) than when foraging alone (72.2 s.; paired t11 = 2.64, P = 0.02).  To our knowledge this is the first examination of activity budgets of loons overwintering on a freshwater lake and the second to note potential differences in foraging strategies among overwintering loons, necessitating further examination of the associated benefits and costs of adopting either strategy among the various microhabitats within the reservoir.}

\normaltalk{\capitalisewords{Shifts in Avian Community Composition Across a Human-Modified Landscape in Borneo, Malaysia}}{\underline{Jessica N Hightower}, Kalsum M Yusah, Robert J Fletcher}{Deforestation over the last century has led to increasingly heterogeneous human-dominated landscapes, with novel habitats that often differ significantly from an organism's preferred habitat. Yet many species continue to persist in these novel habitats. However, it remains unclear what mechanisms facilitate the persistence of species outside of their preferred habitat. We investigated the patterns of avian species occurrence and abundance as a function of distance from the edge of a forest patch and the type of land cover surrounding the patch (i.e. the matrix). We addressed these issues at the Stability of Altered Forest Ecosystems (SAFE), a large-scale experimental forest fragmentation project located in Sabah, Malaysia. We selected 7 study plots in two different matrix types: 1) forest fragments surrounded by forest that was recently logged, characterized by remnant trees, a dense secondary vegetation, and lacking a continuous canopy (n=4); and 2) forest fragments embedded within oil palm plantations (n=3). At each forest fragment we surveyed the breeding bird community using point counts and 100-m mist-net transects located in the interior of the forest fragment, at the edge of the forest fragment, 150 meters from the forest fragment edge, and 300 meters from the forest fragment edge. Our results indicated a difference between bird communities at each sampling location (interior, edge, 150 m, and 300 m) and between matrix types. Understanding how birds continue or fail to persist in new habitats is essential to future conservation planning that may lead to improved management of human-dominated landscapes.}

\normaltalk{\capitalisewords{Effects of Translocation on Burrowing Owl (Athene Cunicularia) Survival, Fidelity, and Nesting Rates in Arizona}}{\underline{Dejeanne N Doublet}, Martha J Desmond, David H Johnson, Fitsum Abadi}{The burrowing owl (Athene cunicularia) is a declining species that has been the subject of translocation efforts. Conflicts between the owls and development led to a program in Arizona that relocates owls from construction sites to artificial burrows. We aimed to evaluate this program's effectiveness by assessing survival, fidelity, and nest survival of translocated and non-translocated owls using VHF radio-telemetry. We used Burnham's joint live-encounter dead-recovery models to estimate adult survival and fidelity. Nest survival models were used to estimate nest survival. From 2017-2019, we tracked 85 adult owls. Annual survival (S) and fidelity (F) estimates were lower for translocated owls (S=0.35±0.08, F=0.13±0.06) compared to non-translocated owls (S=0.80±0.07, F=0.56±0.09). Owls were soft-released in groups of 10 owls/cage. The best predictor of survival and fidelity rates of translocated owls was the number of males in each release cage. Annual survival, for instance, decreased from 0.70±0.11 for groups with the fewest males to 0.02±0.02 for groups with the most males. We monitored 129 nests and found that translocated owls had a lower cumulative nest survival estimate (CNS=0.21±0.06) compared to non-translocated owls (CNS=0.76±0.06, P=2.2±0.2). However, owls translocated in prior years had a similar nest survival estimate (CNS=0.65±0.10) compared to non-translocated owls, suggesting that owls that survive the first year after release can thereafter join the breeding population. We recommend releasing owls individually or as pairs at high-quality release sites before the nesting season begins to allow them time to establish territories. These critical management recommendations will be applicable to other translocation efforts nationwide.}

\normaltalk{\capitalisewords{The Indirect Effect of Flight Demands on Egg Shape in Migratory and Sedentary Fork-Tailed Flycatchers (Tyrannus Savana)}}{\underline{Valentina N G'{o}mez-Baham'{o}n}, Elizabeth R Chen, Diego T Tuero, Marcelo Assis, Miguel  Marini, John Bates}{Bird eggs have evolved a wide variety of shapes including spherical, elliptical, and asymmetrical. Two mayor types of hypotheses have been proposed to explain such variation: 1) adaptive hypotheses, including life history traits and breeding ecology as major drivers of egg shape, and 2) correlational evolution of egg shape with skeletal and muscular features resulting from different flight demands. The former has recently been supported by evidence showing that at broad taxonomic scales, egg shape is correlated with flight capacity. Within species variation often spans the spherical, asymmetrical and elliptical shapes found at the macroevolutionary level, yet studies that test the flight hypothesis at the intraspecific level are lacking. Migratory and sedentary birds differ on morphological features such as wing length and shape, muscle fiber composition and skeletal structure. The hypothesis of indirect effect of flight on egg shape would predict that egg shape is constrained in migratory birds, whereas in sedentary eggs it would have a wider variation resulting from relaxed selection on aerodynamics. In this study we present a novel mathematical method to quantify egg shape allowing comparisons across all egg shapes and test whether migratory and sedentary populations of Fork tailed Flycatchers differ as predicted. We aim to analyze shape parameters using cluster algorithms to determine which account for the variation in our sample and conduct a normal mixture analysis to determine if migratory and sedentary eggs differ. To test whether migratory egg shape is constrained we aim to conduct analyses of variance comparing with sedentary eggs.}

\normaltalk{\capitalisewords{Canada Jay (Perisoreus Canadensis) Foraging Ecology in Denali National Park and Preserve, Alaska}}{\underline{Kaeli N Swift}, John M Marzluff, Emily Williams}{In Denali National Park and Preserve, resident Canada Jays are thought to survive the harsh winters and begin reproductive efforts by subsisting on perishable food items stored during the autumn.  Given that a changing climate may affect the integrity and availability of food stores, a robust understanding of Canada Jay foraging ecology is needed for conservation planning. During the winter and autumn of 2018, we followed and video-recorded the foraging behaviors of 10 marked pairs of territorial adults.  During these observations we documented the food type, location and microhabitat features of food acquisition and cache site, and monitored where jays searched for food. We documented 547 food acquisitions and 309 caches.  Of these events, we identified the food item in 41\% of cases and identified 18 items including berries, animal flesh, mushrooms, insects and various anthropogenic items.  Animals flesh and berries accounted for the largest sources of identified foods. On average, jays cached 11m from the food acquisition site and did so most frequently (32\% of caches) under the bark of exterior spruce tree branches.  Jays also cached on trunks (14\%), interior foliage (13\%), exterior foliage (19\%) and interior bare branches (17\%). When foraging, jays primary searched on the ground during the autumn, and on interior tree bark in winter. Together these data lay a foundation for future studies addressing how variations in foraging and caching affect reproductive success.}

\normaltalk{\capitalisewords{Immigration may Decouple Heterozygosity and Inbreeding in Declining Populations of Florida Scrub-Jays}}{\underline{Tram N Nguyen}, Nancy Chen, Reed Bowman, John W Fitzpatrick, Andrew G Clark}{Understanding the genomic consequences of declining populations – even while they still appear to be robust – is important for conservation management and for predicting species' vulnerability to intensifying global change. In particular, habitat loss and fragmentation may induce extirpations and reduce connectivity between populations, leading to shrinking populations and concomitant declines in genetic diversity. Although ample theoretical predictions can be made for the genetic consequences of shrinking, isolated populations, empirical evidence still remains scarce, especially in natural populations that still experience immigration. We used 7,834 autosomal single-nucleotide polymorphisms and demographic data for 288 Florida scrub-jays (FSJ, Aphelocoma coerulescens) sampled eight years apart to compare changes in genetic diversity, level of inbreeding, and degree of relatedness of individuals across two populations having contrasting trajectories. At Archbold Biological Station (ABS), the FSJ population was stable owing to intensive management, while another nearby population at Placid Lakes Estates (PLE) was declining precipitously. We characterized in detail the genetic changes accompanying the beginning of this rapid population decline. Surprisingly, we found that at both ends of the sampling period, birds in PLE were more heterozygous than birds in ABS, likely owing to greater immigration rates into PLE during and prior to the sampled interval. Paradoxically, the PLE birds were also more inbred and related. We found no significant changes in heterozygosity, inbreeding, or relatedness across our sampled interval in either population, presumably because insufficient time had elapsed to yield a detectable genetic signature of the decline at PLE. Our results show how population structure and immigration can decouple the expected relationship between genetic diversity and inbreeding during the early stages of a severe population decline.}

\normaltalk{\capitalisewords{Post-Fire Nest Site Selection and Survival of Black-Backed Woodpeckers}}{\underline{Andrew N Stillman}, Rodney B Siegel, Robert L Wilkerson, Matthew Johnson, Christine A Howell, Morgan W Tingley}{Black-backed Woodpeckers (Picoides arcticus) are highly associated with recently burned forests in western North America where they specialize on resources present in dead and dying trees. We evaluated the influence of habitat and nest tree characteristics on Black-backed Woodpecker nest site selection in post-fire forests and tested the effects of these same covariates on nest survival. Between 2011 and 2018, we monitored 118 Black-backed Woodpecker nests in northern California and used hierarchical Bayesian models to examine the influence of habitat characteristics on nest site selection and nest survival. Black-backed Woodpeckers showed strong nest site selection relationships for all the biotic variables that we measured: woodpeckers selected moderately-sized trees in areas of high snag density burned at high severity, but also in areas relatively close to low severity or unburned edges. However, only nest initiation date affected nest survival, with decreased survival in late-season nests. While our results demonstrate that habitat characteristics guide nest site selection in Black-backed Woodpeckers, these same habitat characteristics did not appear to impact nest survival in post-fire forests. In addition, our results suggest that management actions aimed at maintaining populations of Black-backed Woodpeckers should prioritize pyrodiverse habitats that include dense stands of snags within ~500 m of forest that burned at low severity or remained unburned. }

\normaltalk{\capitalisewords{Evolution of Polygyny in Weaverbirds (Family Ploceidae)}}{\underline{Thilina N De Silva}, Sumudu W Fernando, A Townsend Peterson}{The interesting question of how polygyny, a phenomenon where a male is mated simultaneously with multiple females, evolved intrigues ecologists as well as evolutionary biologists to date. $>$90\% of the bird species are considered socially monogamous; polygyny is assumed as an evolutionarily derived state, and polygynous birds are suggested as having monogamous ancestors. Weaverbirds (family Ploceidae) consist of over 115 species, of which close to a half of the taxa are polygynous. To test the suggested hypotheses that: (1) ancestral weaverbirds were monogamous forest dwellers, and (2) polygyny coevolved with colonization of open habitats such as savanna and reed habitat, we accompanied phylogenetic comparative analyses and multilocus phylogenetic hypotheses derived via maximum likelihood and Bayesian methods. Our results support the idea that the ancestral weaverbirds were monogamous savanna dwellers, and that polygyny coevolved subsequently, alongside colonial behavior. In weaverbirds, mating systems showed high correlation to breeding habitat, diet, gregariousness, and parental investment. The outcome of this study is consistent with the popular hypothesis that polygyny evolves as a consequence of male birds being able to monopolize resources. This study comprehensively documents the evolution of polygyny within a bird family for the first time, and discusses the evolutionary pathways that may have led to polygyny in weaverbirds.}

\normaltalk{\capitalisewords{Age-, Sex-, and Size-Related Differences in the Distribution of Migrating Landbirds on the Northern Gulf Coast}}{\underline{Liliana N Calderon}, Theodore J Zenzal, Wylie Barrow, Barry Wilson, Jeffrey J Buler}{Migrating birds can be constrained by factors extrinsic to habitats (e.g., demographic and physiological traits) when stopping over near large water bodies like the Gulf of Mexico (GOM). We investigate variability in the age, sex, and body size (based on wing chord length) of migrating landbirds in relation to the geographic location of stopover sites along the northern GOM (i.e., distance from the coast and longitude). We mist-netted birds at six stopover sites located within Alabama and Louisiana during spring and autumn 2018 that range from 1 to 75 km from the coast. We found differences in the age, sex, and size distributions of migrants related to distance from the coast and longitude. For example, during spring, we found a positive linear relationship between body size and distance from the coast within species with mean wing chord increasing by 0.015 mm per km away from the coast. This novel intraspecific pattern with respect to body size corroborates existing evidence of a similar interspecific pattern that found smaller species stopping over at relatively higher densities nearer to the GOM than larger species. We also found higher ratios of older birds at inland sites compared to coastal sites. Our study provides new insights into the stopover distributions of birds in relation to the GOM and emphasizes the conservation value of coastal stopover habitats.}

\normaltalk{\capitalisewords{Pintail Populations Became more Vulnerable to Climate Change under Agricultural Intensification}}{\underline{Qing N Zhao}, Todd Arnold, Jim Devries, David Howerter, Bob Clark, Mitch Weegman}{It is essential to understand the effects of land use patterns in altering climatic vulnerability of migratory birds, which rely on different landscapes throughout the year. During the last decades northern pintails (Anas acuta; hereafter pintails) in North America did not respond to favourable environmental conditions as most other ducks did. The mechanisms and drivers involved in this divergence remain poorly understood. While pintails are similar to other ducks in their dependence on wetlands throughout their annual cycle, their extensive use of croplands for nesting differentiates them and makes them particularly vulnerable to changes in agricultural land use on prairie breeding grounds. We aimed to quantify the interactive effects of land use and wetland habitat on pintail demography by developing an integrated population model for long-term (1961–2014) band-recovery, breeding population survey, land use and pond count data covering prairie and parkland habitats. We found that a positive interactive effect of pond count and agricultural intensification was insufficient to overcome the strong negative effect of agricultural intensification on pintail productivity, but the interaction indicated that that pintail populations were more negatively impacted by drought conditions associated with climate change under higher agricultural intensification. Our results indicate that pintail populations have become more vulnerable to climate change under intensified land use, which suggests that future conservation strategies must adapt to these altered relationships. The interacting effects of land use and climate on demography should be considered more frequently in avian ecology with integrated population modelling approaches.}

\normaltalk{\capitalisewords{On Point Count Sampling for Understanding Recruitment, Survival and Movement}}{\underline{Qing N Zhao}}{Traditionally the estimation of demographic parameters such as survival, reproduction, and movement requires intensive data from marked animals that can be difficult and expensive to collect. The development of dynamic N-mixture models (Dail \& Madsen 2011) allows the estimation of demographic parameters using count data of unmarked animals only. The original dynamic N-mixture model does not separate immigration and emigration from reproduction and survival, however, limiting its ability to explain movement among habitat patches. The original dynamic N-mixture model also assumes constant detection probability over time, an assumption that is unlikely to be met in long-term studies. We developed spatially explicit dynamic N-mixture models that separate immigration and emigration from reproduction and survival. We also considered temporal variability in detection probability in these models. We used simulation studies to evaluation the inferential performance of these models. We found that these models provide valid estimates for reproduction, survival, movement, as well as local population size and detection probability in most cases, except when both local population size and detection probability are low. The models are robust to the violation of the assumption about movement distance. The models are sensitive to the assumption of constant detection probability, and correctly accounting for temporal variability in detection probability is essential for these models. The robust design leads to improved estimation of time-varying detection probability. Overall, our models are useful to understand metapopulation dynamics based on count data that are relatively easy to collect in in long-term studies.}

\normaltalk{\capitalisewords{Connecting Environment, Stress Coping Capacity, and Fitness: Insights from a Nest-Box Breeding Population of Tree Swallows (Tachycineta Bicolor)}}{\underline{Maren N Vitousek}}{Many fitness-relevant traits show high within-individual variation, which makes their function and evolution challenging to study in the wild. Nest box breeding birds provide three major advantages in studying such traits. First, many birds that utilize nest boxes can be readily captured at narrowly defined periods of time within specific life history stages, and accessed rapidly enough to measure many physiological traits at "pre-disturbance" levels. Second, nest boxes facilitate the continuous monitoring of many social and parental behaviors, and some physiological traits, providing rich data sets on parental behaviors and social interactions. Third, nest box-breeding populations are particularly amenable to longitudinal studies that connect environmental and trait variation with fitness consequences. Here I highlight some of the insights gained from a long-term study of Tree Swallows (Tachycineta bicolor) that breed in and around Ithaca, New York. Life history data, collected continuously for over 30 years in this population, have provided insight into how temperature and resource availability influence fitness. More recently, standardized physiological measurements coupled with large-scale behavioral data collection have begun to yield insights into the causes and consequences of variation in stress coping strategies. Experimental stressor exposures have shown that the regulation of glucocorticoid hormones is a key predictor of the capacity to cope with stress. At the same time, accumulating data on the lifetime trajectories of individuals has begun to reveal how variation in stress coping strategies affects lifetime fitness across different environments.}

\normaltalk{\capitalisewords{Identifying Functional Roles in Mixed-Species Groups}}{\underline{Laura N Vander Meiden}, Daizaburo Shizuka}{In mixed-species groups of animals, species have been traditionally categorized as either nuclear, facilitating the formation and cohesion of the group, or satellite. However, this dichotomous approach is limited by the many, distinct attributes that are ascribed to nuclear species, resulting in species that play different roles being lumped together under a broad, functionally ambiguous term. Here, we propose new categories of species roles based on their function within mixed-species groups. We conducted an exhaustive review of mixed-species group dynamics across taxa and identified four distinct functional roles found in mixed-species groups 1) facilitating group formation, 2) maintaining the cohesion of the group, 3) leading the group through space and 4) providing information, such as when predators are present, to the group. In some groups one species may play multiple roles, while in others different roles may be played by different species. Likewise, not all groups will contain all roles. We propose experimental and analytical approaches to test the roles of species within these groups. By utilizing this framework to characterize species found in mixed-species groups, we can begin to draw comparisons between mixed-species groups of different locations, compositions, and taxa.}

\normaltalk{\capitalisewords{Urban White Ibises Facing Trade-Offs in South Florida}}{\underline{Catharine N Welch}, Sonia M Hernandez, Maureen Murray, Shannon E Curry, Anjelika Kidd, Jeffrey Hepinstall-Cymerman, Emily Lankau, Michael Yabsley, Kristen Navarra, Caroline Cummings, Erin Lipp, Sonia Altizer}{Within the last two decades, American white ibises, nomadic birds with dynamic boundaries, have been increasingly observed in urban areas of South Florida foraging on lawns, golf courses, and landfills, and hand-fed in urban parks. Many studies have explored trade-offs that urban wildlife face when they take advantage of novel anthropogenic food resources. Because of the recent shift from wetland specialists to urban generalists, ibises make a great model species to study urban adaptation. We have been studying this system since 2009, hypothesizing that ibises are spending more time in urban areas, showing higher site fidelity to urban areas where food handouts are abundant, and questioning the effects of this shift on their health where they consume carbohydrate-rich and protein-poor diets. From 2015 to 2017, we captured ibises along a gradient of urban-to-wildlands, collected biological samples and attached GPS transmitters to a subset of adults. Our results suggest that urban ibises show higher site fidelity throughout the non-breeding season and are heavily provisioned with anthropogenic food, which appears to offer a trade-off by providing low-quality, but easily accessible calories that may not support high mass but may increase time available for anti-ectoparasitic behaviors, such as preening. In contrast to natural ibises, have higher Salmonella prevalence with a high diversity of serotypes, 58\% of which associated with human cases. Understanding the trade-offs ibis face in human-modified habitats is important, not only for ibis conservation, but for public and environmental health.}

\normaltalk{\capitalisewords{Dynamic Replenishment: Implementing a Multi-Benefit Approach to Creating Bird Habitat and Recharging Groundwater}}{\underline{Julia nmn Barfield}, Melissa Rohde, Greg Golet, Mark Reynolds, Katie Andrews}{Increasing water demands are threatening global water security, freshwater biodiversity and ecosystems. Groundwater accounts for as much as 33\% of global water use, but it often serves as the primary water resource for communities and ecosystems in arid climates. In 2014, California passed the Sustainable Groundwater Management Act (SGMA) -- California's most significant water reform in a century. As SGMA is implemented, there is evidence that flexible and dynamic approaches can increase water security, while providing habitat for wildlife and improving water supplies for communities. Yet there are few examples of these types of projects being implemented. We recently implemented one such program. In partnership with local farmers, we piloted an on-farm groundwater recharge project that delivered "pop-up" habitat for migratory waterbirds along the Pacific Flyway. The pilot was timed to provide fall stopover sites for migratory shorebirds. This method of working at numerous sites at various times of year represents a flexible, cost-effective strategy that can be adapted to changes in farming production schedules and water availability. We estimated an infiltration rate of ninety percent on enrolled fields. Monitoring results suggest we provided habitat for a diversity of waterbirds including waterfowl, waders, shorebirds, raptors, and species of management concern such as sandhill crane, long-billed curlew and bald eagle. This dynamic approach provides a solution to California's need to maximize water storage while also meeting the diverse needs of an important wildlife community, without requiring investment in costly permanent infrastructure that is inflexible to changes in water availability and climate.}

\normaltalk{\capitalisewords{Female Bird Songs -- The Other Half of the Conversation}}{\underline{Lauryn nmn Benedict}, Tj Hathcock, Nadje Najar, Liam Mitchell, Jakica Cavar, David Logue}{Bird song serves as a classic model in the field of animal communication. Studies of song have revealed much about the genetic, neural, and physiological mechanisms that underlie signaling, as well as the evolutionary factors that shape signals. The vast majority of this work, however, is focused on males. With growing recognition that female birds regularly sing, and that song is an ancestral trait in female passerines, recent research is making great strides towards investigating the whole avian vocal repertoire. Here, I will present results on the form, function, and evolution of female song in multiple North American bird species. Studies of New world Warblers reveal a rapid pace of discovery within this field. Studies of canyon wrens offer evidence that female song can be functional, even when it is rarely used. Collectively, research on female song has advanced our understanding of how sex interacts with animal signal production, signal evolution, and sexual selection.}

\normaltalk{\capitalisewords{Grassland Bird Declines: How can We Stop the Downward Spiral?}}{\underline{Arvind O Panjabi}, Maureen Correll, Erin H Strasser, Jacy Bernath-Plaisted, Adam Beh, Tammy L VerCauteren}{Grassland bird populations have been declining for decades. Despite recent conservation attention, many species continue to decline, especially those that migrate between the western Great Plains and Chihuahuan Desert. As with many migratory birds, drivers of population declines for most grassland species are poorly understood due to incomplete knowledge of their lifecycle and potential limiting factors they may encounter throughout the year. Therefore, conservation strategies and programs have generally not addressed known limiting factors and instead have focused on locally relevant habitat conservation and management, primarily on the breeding grounds. However, recent advances in non-breeding ecology, including wintering distribution, abundance and habitat requirements, land cover trends and migratory patterns, coupled with seasonal demographic rates from across the annual cycle, now permit a more holistic view of grassland bird conservation needs. This information is being synthesized in a full annual cycle Integrated Population Model for the Baird's Sparrow, a species typical of the Great Plains-Chihuahuan Desert grassland bird guild. Bird Conservancy of the Rockies, along with partners, will host a multi-stakeholder workshop for the Central Flyway in 2020 to share results of the IPM and seek to align existing and new conservation efforts to address limiting factors across the life cycle. By engaging and leveraging resources across all stakeholders, including NGOs, landowners, industry, foundations and government agencies, in a common strategy to address identified limiting factors, we aim to reverse the decline in Baird's Sparrow and other grassland birds through a science-based, full annual cycle, coordinated approach.}

\normaltalk{\capitalisewords{Risk Assessment of Great Lakes Piping Plovers using eABM: An R Package to Run Environment Agent-Based Models}}{\underline{Brandon P Edwards}, Shoshanah Jacobs, Daniel Gillis}{Environment agent-based models (eABMs) are an efficient, spatially-explicit model used to simulate one or more communities of animals in an environment. This type of model has been shown to be a useful risk assessment tool across different taxa. It has been used to simulate the effects of impingement and entrainment on a population of lake whitefish (Coregonus clupeformis), and to simulate effects of anthropogenic disturbance on a breeding population of Great Lakes piping plover (Charadrius melodus circumcinctus). eABMs are typically developed on a case-by-case basis; that is, a researcher may develop one eABM for one animal species independently from another researcher developing an eABM for another species. At their core, eABMs rely only on a habitat matrix as well as rules regarding the life cycle of the animal of interest. The R package eABM seeks to provide a generalized framework to researchers looking to use eABMs in their own risk assessment research. The package allows users to provide a habitat matrix and a set of rules associated with their study species, rather than develop an eABM from scratch. Using a breeding population of Great Lakes piping plovers as a motivating example, we demonstrate the typical workflow of a researcher using the eABM package to develop risk assessment scenarios to simulate, and we provide examples of scenarios that can be simulated using this package.}

\normaltalk{\capitalisewords{Carry-over Effects of Breeding Season Habitat on a Long-Distance Migrant}}{\underline{Brendan P Boyd}, Bridget Stutchbury, Sue M Hayes}{The Wood Thrush is an iconic North American long-distance migrant that has been steadily declining for decades. While habitat loss and fragmentation on the breeding grounds has been shown to have a negative effect on immediate breeding success, long-term impacts on breeding adults, or carry-over effects, have not been studied. The Motus Wildlife Tracking System is an automated radio telemetry array that, for the first time, can link breeding season habitat to fall migration and annual survival. Wood Thrushes occupying small forest fragments are expected to experience high rates of brood parasitism and nest predation, which could directly delay fall migration due to timing constraints from late re-nesting or indirectly delay migration if adults are in poorer condition. Blood samples were collected during 2018 to analyze the level of corticosterone, a stress response hormone that can be indicative of habitat quality. During the 2016-2018 breeding seasons, I captured 87 adult Wood Thrushes in a variety of forest fragments and fitted them with coded radio transmitters in order to track their movements using the Motus Wildlife Tracking System. Wood Thrushes are large enough to carry radio-tags with a one year battery life, allowing detection of adults who return to the study area in southwestern Ontario.  A relationship between habitat and fall migration has not yet been identified; however, Wood Thrushes tagged in small forest fragments were less likely to return the following year than those tagged in large fragments.}

\normaltalk{\capitalisewords{Neighbor-Stranger Discrimination in a Neotropical Suboscine Bird: Embracing Behavioral Complexity to Uncover Patterns}}{\underline{Jonathan P Kelley}, Corey E Tarwater}{In many socially monogamous bird species in the tropics, territorial disputes between pairs are complex interactions comprising of multiple individuals: multiple territorial pairs in which both males and females defend territories, juveniles that may be on the natal territory for extended periods, as well as neighbors (i.e. territorial pairs not in the adjacent regions. Further complications arise from endogenous contexts that are often unknown in birds, such as the age of intruder or pairs. Using a framework of behavioral complexity, we examined the ability of T. atrinucha to distinguish familiar neighbors from unfamiliar strangers, a pattern that is yet untested and unknown for most Suboscine species. The analysis of neighborhood response networks—specifically the incorporation of the response of territorial neighbors to playback and the explicit modeling of time lags—improved model performance and allowed us to detect neighbor-stranger discrimination in T. atrinucha. This highlights the importance of behavioral complexity theory—rather than behavioral reductionism—in (1) detecting new aspects of the behavioral ecology of poorly-studied birds, and (2) amending existing knowledge about avian behavioral ecology.}

\normaltalk{\capitalisewords{Modeling the Implications of Future Bioenergy Scenarios on Bird Diversity and Abundance in the Northern Great Plains}}{\underline{Andrew P Baltensperger}, Reza Goljani-Goljani-Amirkhiz, Mark Dixon, David Swanson}{Over the past 30 years, scientists have proactively identified numerous threats posed by climate change to ecosystems, wildlife, and people. More recently, efforts have shifted towards coping with these challenges by adapting ecological, agricultural, and infrastructural systems. One means of addressing the problem of climate change is to limit atmospheric carbon concentrations by increasing production of second-generation (biomass) biofuels. However, the agricultural, economic, and ecological impacts of such an approach remain largely unknown. Predictions of how different future scenarios (varying second-generation biofuel cultivation intensities) would affect the abundance and biodiversity patterns of birds in agricultural regions are necessary to pro-actively inform conservation planning. We used sets of georeferenced abundance records, 41 environmental predictors, and the RandomForests algorithm to create ecological niche models for two dozen bird species across the Northern Great Plains of the U.S. Models formed the basis for sets of species-specific distribution and relative abundance maps for the present day. Substituting six alternative biofuel land cover projections into the models, we also spatially predicted the effects of land use change on birds in the year 2060. Our quantitative species distribution and biodiversity change projections should help managers balance approaches for carbon sequestration with the ecological requirements of birds as we work to develop adaptive strategies for conserving biodiversity while acting to mitigate climate change.}

\normaltalk{\capitalisewords{Validating Climate Suitability Projections: Community Science Reveals that Birds Track their Ecological Niche in Summer and Winter}}{\underline{Sarah P Saunders}, Nicole L Michel, Brooke L Bateman, Chad B Wilsey, Zachary Slavin, Kathy Dale, Geoff LeBaron, Gary M Langham}{Climate change poses an intensifying threat to many bird species, and projections of future climate suitability provide insight into how species may shift their distributions in response. Climate suitability (CS) is characterized using ecological niche models (ENMs), which correlate species occurrence with current environmental covariates and predict future distributions using the modeled relationships together with climate projections. Despite their widespread adoption, ENMs rely on several assumptions (e.g. species can and will move to track shifting CS) that are rarely validated in situ, precluding their reliability in conservation decision-making. Using data from a novel, large-scale community science program, we developed dynamic occupancy models to evaluate whether bluebirds and nuthatches are tracking CS projections in summer and winter. We estimated occupancy and colonization/extinction dynamics in relation to changes in CS between the 2000s (historic CS) and 2020s (near-future CS). We included a strata-level random effect to reduce covariation between climate and land-use changes. Across species and seasons, initial occupancy was positively associated with historic CS, representing an independent validation of ENMs. Extinction rates for several species were negatively related to the difference in historic and near-future CS in winter only, thus birds tended to leave sites declining in CS during winter rather than summer. While summer colonization rates were positively related to near-future CS for bluebirds and nuthatches, winter colonization rates were related to near-future CS for nuthatches only. Our findings elucidate the dynamic processes underlying these species' distributions, and support the use of CS projections to predict responses to climate change.}

\normaltalk{\capitalisewords{Cutting Out without Losing Out: Subsampling Soundscapes for Long-Term Avian Biodiversity Monitoring}}{\underline{Jonathan P Eiseman}, Maarten J Vonhof, Sharon A Gill}{Soundscapes emerge from aggregations of acoustic signals generated by biological, technological and geophysical processes and are increasingly used to monitor avian biodiversity. Developments in recording technologies facilitate the long-term collection of high-resolution data; however, complications arise in making decisions about appropriate sampling designs to characterize variation over space and time. We ask whether temporal subsampling leads to significant loss of information on daily and seasonal patterns using acoustic indices. We tested the accuracy of hourly means generated by intermittent recording schedules compared to the "gold standard" of hourly means generated by continuous sampling. Grassland and forest soundscapes were continuously collected from ten nature preserves in southwest Michigan during two independent 24-hour periods in spring and summer using 30 Wildlife Acoustics SM2+'s. We compared recording schedules (continuous vs. 1 every 5, 10, 15, 20, 30, and 60 minutes) by calculating six acoustic indices and performing spearman's rho correlations to measure relationships between continuous and subsampled hourly means. We found that sampling 1 every 5 minutes was highly correlated with continuous recording, which suggests information loss was minimal, while indices calculated based on sampling 1 every 60 minutes were poorly correlated with continuous recording. Additionally, lowered sampling schedules in forest locations were regularly less correlated with continuous sampling as opposed to grassland locations, providing evidence that optimal sampling may differ based on habitat. Our findings demonstrate that characterizing avian biodiversity using soundscapes can be conducted under reduced sampling approaches, allowing researchers and managers to realize significant savings in storage and processing costs.}

\normaltalk{\capitalisewords{A Bayesian Network Approach for Improved Seasonal Distribution Models of Long-Distance Migratory Passerines using Tyrannus Flycatchers}}{\underline{Maggie P MacPherson}}{Widely used examples of species distribution modeling that are important for implementing a variety of conservation strategies include Gap Analysis Program (GAP) models mapping land cover and bioclimatic envelopes, Habitat Suitability Indices (HSIs), maximum entropy models (MAXENT), and genetic algorithm for rule-set prediction (GARP). However, deeper looks at mechanistic underpinnings of abiotic factors correlated with distributions of long-distance migratory passerines have been linked to either proximate physiological underpinnings (e.g., tracking the phenology of food resources), or to evolutionary precursors (e.g., predispositions for ephemeral resources or the retainment of ancestral migration patterns). Here, I present a Bayesian belief network (BBN) modeling framework integrating a compilation of research of evolutionary, ecological and physiological trends that describe the current distributions of members of the Tyrannus (Kingbird) genus. BBNs are an underutilized form of species distribution modelling that offer a unique approach that aims to exhaustively identify explicit causal relationships among organisms and their habitats, as well as incorporating measures of uncertainty for each node contributing to site occupancy. BBNs go beyond species-habitat correlations because they explicitly consider discrete processes that influence occupancy across space and time. Using this framework, I discuss relative contributions of evolutionary, ecological and physiological mechanisms predicting seasonal distributions of migratory Tyrannus savana (Fork-tailed Flycatcher) and credible intervals for hypotheses within each mechanistic driver. BBNs integrating multiple sets of hypotheses improve our ability to predict distributions of long-distance migratory passerines and offer advancement to our current approaches to coarse species-habitat modeling employed in conservation plans.}

\normaltalk{\capitalisewords{Identifying and Characterizing Transposable Element Polymorphisms in the Northern Flicker (Colaptes Auratus) Species Complex}}{\underline{Jack P Hruska}, Joe D Manthey}{Transposable elements (TEs) comprise a substantial portion of many vertebrate genomes. Conversely, bird genomes have a paucity of TEs ($<$10\% genomic content). However, recent work has demonstrated that the Downy Woodpecker is a notable exception, containing a greater than two-fold increase in TEs relative to other birds.  This is largely due to the expansion of chicken-repeat 1 (CR1), a non-long terminal repeat (LTR) retrotransposon that has been recently active in this species. In spite of genomic CR1 proliferation, polymorphic CR1s in Downy Woodpeckers and closely-related species are at relatively-low allelic, suggesting that negative selection has been counteracting CR1 expansions. Here, we tested this hypothesis in five closely-related lineages of the Northern Flicker (Colaptes auratus) complex by identifying and assessing polymorphisms in four subfamilies (J2\_pass, J3\_pass, E\_pass and Y4) of CR1. Using a de novo long-read genome assembly of Colaptes auratus as a reference, we identified several thousand CR1 polymorphisms that were segregating across lineages with the Mobile Element Locator Tool (MELT) and assessed their frequencies relative to single nucleotide polymorphisms (SNPs). Consistent with a previous assessment of TE activity in woodpeckers, we present evidence that CR1s have been amplifying in the recent past across lineages of Colaptes auratus but have been largely suppressed by natural selection.}

\normaltalk{\capitalisewords{Guild-Specific Effects of Intensive Forest Management on Avian Abundance}}{\underline{Jake P Verschuyl}, Andrew J Kroll, Jay Jones, Matthew G Betts}{Intensively managed tree plantations can supply wood products to an expanding human population while reducing pressure on natural forests. Herbicides are used to accelerate growth of crop trees by suppressing competing vegetation but early-seral communities may be negatively affected by reduced broadleaf vegetation abundance and richness. We used a large-scale randomized complete block experiment to test avian population responses to variation in stand management intensity post-harvest in the Pacific Northwest, USA, 2011-2018. We evaluated how abundance changed for 55 species in response to three levels of plant cover reduction (Light, Moderate, and Intensive herbicide applications) in relation to a control without herbicide. By 2015, we found no evidence of differences in total abundance of non-leaf-gleaning species on any of the treatments compared to the control, but three leaf gleaning species still showed reduced abundance. By 2018, there was no evidence of on-going reductions for leaf-gleaning species in treated stands. Substantial block-specific variation suggests that other factors may mediate treatment effects on individual stands. Our results indicate trade-offs between avian abundance and wood production may be less severe than previously recognized. For example, breeding populations of the 55 species occurred on all treatments, and evidence of abundance differences between treatment and control stands did not exist in the final year of study for leaf-gleaning species. We emphasize demographic information is required to compare relative contributions of treatment and control stands to stands originating from natural disturbance, and to determine how all stand types contribute to the maintenance of regional bird populations.}

\normaltalk{\capitalisewords{Messy Eaters: DNA on Dirty Talons and Beaks Reveal What a Migrating Raptor Eats}}{\underline{Ryan P Bourbour}, Joshua M Hull}{Some evidence suggests that accipiters and falcons may be tracking migrating avian prey along migration corridors to fuel their journey, however, the extent migrating raptors rely on ephemeral resources to power migration is poorly understood. Currently, most of our knowledge on the foraging ecology of raptors has been documented on the breeding and non-breeding grounds where individuals may be tied to known territories or resources. Because migration represents an energetically demanding and vulnerable time for raptors, understanding critical resources along the migratory route is important. Systematically studying the diet of migrating raptors has been logistically challenging in the past because conventional diet study methods (e.g. observations, pellet, and prey remains analyses, remote-sensing cameras) often fall short with enigmatic or highly vagile predators. The goal of our study was to develop a broadly applicable method to document prey selection when conventional methods could not be used. We swabbed the exterior of migrant raptor beaks and talons to collect prey DNA from two raptor species, Sharp-shinned Hawks (Accipiter striatus) and Merlins (Falco columbarius), over two migration seasons. We extracted trace DNA from swab tips and amplified COI gene sequences using previously published primers which were then referenced to an open-access barcode database to identify prey to species. In our preliminary results, all prey DNA sequences matched probable resident and migratory avian prey with high accuracy. Using this novel dietary study technique, we can study and describe the foraging ecology of migrant raptors, which ultimately furthers our understanding of migration ecology.}

\normaltalk{\capitalisewords{Phylogenomics of Pigeons and Doves from Whole Genome Sequencing}}{\underline{Kevin P Johnson}, Bret M Boyd, Nam-Phuong Nguyen, Julie M Allen, David W Steadman, Robert M Waterhouse, Andrew D Sweet, Dale H Clayton, Sarah E Bush, Michael D Shapiro}{Pigeons and doves (Columbiformes) comprise over 300 species and are distributed around the world.  Prior molecular studies of the higher level phylogeny of pigeons and doves (Columbiformes) have produced conflicting and unstable results.  However, these studies have been based on a relatively small number of gene sequences ($<$10).  We used Illumina technology to sequence entire genomes of 61 species of Columbiformes, using 250bp paired-end reads.  We mapped these reads to full coding sequences of 6,363 ortholog genes from Columba livia to obtain a consensus sequence for each species.  We analyzed the phylogenomic data matrix from these gene sequences using both concatenated and coalescent approaches.  In the concatenated analysis, all but one branch was supported by 100\% of bootstrap replicates.  The tree resulting from the Astral coalescent analysis had only one branch rearrangement compared to the concatenated tree, and all but four branches were supported by 100\% local posterior probability.  We used six fossil calibrations within Columbiformes to date the phylogeny for this group.  Results of these analysis suggested modern pigeons and doves began radiating about 50-60 mya, which is similar to a previously published estimate.  The phylogenomic tree supports a monophyletic group of mostly South East Asian and Australasian genera as sister to a monophyletic group containing all the New World taxa as well as two more recently derived Old World lineages.  This strongly supported backbone tree for Columbiformes highlights the utility of genome sequences for phylogenetic reconstruction in birds.}

\normaltalk{\capitalisewords{Ancient and Contemporary Hybridization Amongst Wood Warblers}}{\underline{David P Toews}, Scott A Taylor, Henry Streby, Gunnar Kramer, Courtney Brennan, Andy Jones, Irby J Lovette}{Understanding the extent and pace of gene flow between diverging lineages can illustrate the drivers of reproductive isolation, as well as how that isolation can vary through time or across different regions of the genome. Hybrid zones are geographic areas where reproductive isolation between contemporary individuals is incomplete, and heterotypic individuals actively hybridize. Rarer instances of hybridization—not necessarily restricted to contemporary hybrid zones—may also facilitate gene flow between divergent lineages, although these can be more difficult to detect, particularly if the hybrid zone is no longer active. Here we use two approaches to try and quantify rare and/or ancient gene flow amongst Parulid warblers. First, we use whole-genome analysis of rare, intergeneric hybrids between Vermivora and Setophaga warblers—combined with samples from the putative parental taxa—to quantify the effects of hybridization at this deep level of divergence. Second, we use whole-genome analysis from every species of Setophaga warbler (n = 154 individuals / 34 species) to quantify whether certain regions of Parulid genomes have been shared via ancient gene flow. We use these data to draw a more general understanding of how a history of gene flow and hybridization has shaped wood warbler evolution.}

\normaltalk{\capitalisewords{The Impact of Species Interactions on Latitudinal Gradients in Trait Evolution in Birds}}{\underline{Jonathan P Drury}}{Whether competitive interactions between species are more important evolutionary forces in tropical regions, where species richness is higher, or in temperate regions, where fewer niches are available, remains an open question. Interspecific competition could impact trait evolution in several ways: competition between closely related, sympatric lineages could drive divergence via character displacement, or competition between species with similar life histories might restrict divergence if niche-filling during diversification constrains the trajectory of trait evolution. To test whether the strength of competition of one form or another varies with latitude, we fitted several phylogenetic models of trait evolution to a global avian dataset containing several ecomorphological traits. The set of models we fitted includes models that are neutral with respect to species interactions as well as several newly-developed models that incorporate the impact of competition on trait evolution. At the level of families, we found very little support for the hypothesis that competition drives divergence via character displacement at a higher rate in tropical regions. However, in larger scale analyses, we found abundant support for models of diversity dependent trait evolution, supporting the hypothesis that the rapid accumulation of species richness in tropical regions has driven overall lower rates of trait evolution in the tropics. This study highlights the enormous utility of global trait datasets and their use in addressing long-standing evolutionary hypotheses.}

\normaltalk{\capitalisewords{Sorting Through the Ashes: a Synthesis, Roadmap and Discussion for the Conservation of the Birds of North America}}{\underline{Peter P Marra}, Ken Rosenberg}{Slowing the loss of biodiversity across terrestrial and marine biomes is perhaps the greatest conservation challenge we face as environmentalists in the 21st century. Over the past 150 years, vertebrate extinctions have been driven largely by habitat loss, overharvesting and invasive species, but the pervasiveness of current avian declines suggests multiple and interacting causes ranging from habitat loss to overharvest to cat depredation, and identifying which of these factors drives population dynamics is complex and challenging. Our ability to halt future extinctions depends on our effectiveness at monitoring population trajectories, identifying and then eliminating the factors contributing to declines, to allow populations to recover.  Targeted research to understand migratory connectivity for declining species must be coupled with efforts to identify population limiting factors and then implementation of effective policies and societal change that emphasize reducing threats to habitats and minimizing avoidable anthropogenic mortality.  History shows that conservation action and legislation can work. The series of talks in this symposium signal an urgent need and lay out a roadmap to address the ongoing threats that birds face to avert potential collapse of the continental avifauna.}

\normaltalk{\capitalisewords{Sea Level Rise Impacts on Waterbird Habitat Across the Hawaiian Islands}}{\underline{Melissa R Price}, Kristen Harmon}{Native waterbirds in Hawaii have declined since colonial settlement of the islands due to sport hunting, loss of habitat, and the introduction of invasive predators, resulting in listing under the Endangered Species Act for most remaining species. Following the banning of sport hunting of waterbirds, recovery is thought to be hindered by availability of habitat, a problem which will be potentially increased with sea level rise. In this study we determined the availability of wetland habitat across the Hawaiian Islands for all remaining native waterbird species by the year 2100 under sea level rise projections derived from existing models of passive flooding, annual high wave flooding, coastal erosion, and wetland accretion in Hawai'i. Sea level rise projection layers were obtained from the National Oceanic and Atmospheric Administration and the University of Hawai'i Coastal Geology Group. ArcGIS Pro was used to spatially and quantitatively map losses and gains of wetland habitat under sea level rise projections by the year 2100. We found that waterbirds will lose some habitats, particularly estuarine environments, due to increased depths following sea level rise, and gains in potential habitat are limited by existing urban and suburban development. However, restoration of Hawaiian flooded agro-ecosystems in lowland areas may offset these losses. Further models will be strengthened by the incorporation of freshwater springs, streams, and other hydrological inputs into the models to determine which areas will be most suitable for waterbirds which differ in hydrological preferences.}

\normaltalk{\capitalisewords{Combining BBS and PIF Population Data to Estimate Recent Changes in Illinois' Total Bird Population}}{\underline{James R Herkert}}{To try and estimate the total scale of recent (2005-2015) bird population changes in Illinois.  I combine data on population trends from the Breeding Bird Survey (BBS) with the Partners In Flight (PIF) Population Estimates Database (version 2.0). Both data sets represent the state of the art for their respective purposes – BBS for trends and PIF for population size – and thus combining them provides the best available estimate for total population changes.  The results suggest that Illinois has lost an average of approximately one million birds/year over the last ten years. They also suggest that annual variation in the size of the breeding population in Illinois may be on the order of 1 to 5 million birds annually.  The data also reveal surprising insights regarding on-going changes in breeding bird populations within the state; such as the fact that the Dickcissel, a grassland bird, appears to currently be the second fastest increasing bird in the state based on changes in the total statewide breeding population.  There remains uncertainty in both the estimates of trend and population size, but nonetheless the data can provide a useful and informative lens by which to assess changes in breeding bird populations.}

\normaltalk{\capitalisewords{Alternate Migration Strategies Mediate the Effects of Stochastic and Predictable Climatic Changes on Migratory Birds}}{\underline{Nathan R Senner}, Jennifer A Linscott, Eldar Rakhimberdiev, David C Douglas}{Alternative behavioral strategies can exist at stable frequencies within a population. Periods of environmental change can shift the balance between such strategies, though, potentially destabilizing population dynamics. Two separate migration strategies exist within Hudsonian Godwits (Limosa haemastica) breeding in southcentral Alaska: some individuals stop only once for 3 – 4 weeks during northward migration, while others stop multiple times for short periods at each stop. Climate change is projected to lead to two changes that may affect the balance between these strategies. First, the number of storm fronts passing through mid-continental North America will increase, which can delay the progression of migration. Simultaneously, the onset of spring in southcentral Alaska is occurring increasingly early, potentially strengthening selection on migratory timing. To explore how these two pressures affect godwits employing different migration strategies, we examined the migratory timing and patterns of 29 godwits over the course of three consecutive migrations from 2009 – 2012. We found that storms disrupted migration by causing godwits to make unexpected stops and, ultimately, to arrive up to 5 days later than normal at the breeding grounds. Migratory delays, however, did not result in adverse fitness consequences for late-arriving individuals. Nonetheless, future increases in the incidence of storms may have detrimental consequences, especially for those godwits typically making only a single stop. When faced with storms, these individuals continued on to their normal stopover site — for their normal stopover duration — after making a storm-induced stop. Such inflexible behavior may prove costly if spring continues to occur earlier.}

\normaltalk{\capitalisewords{Hierarchical Fear: Predation Risk Has Direct and Indirect Effects on Nestling Growth and Stress}}{\underline{Devin R de Zwaan}, Kathy Martin}{Nestling development is a critical life-stage that represents a particularly vulnerable period for altricial songbirds. Rapid growth and early fledging decreases predation risk by advancing development of mobility traits useful in predator evasion. However, predation risk may reduce development rate indirectly by disrupting parental provisioning behaviour, or directly by a physiological response (i.e., corticosterone) to the presence of a predator. We experimentally increased perceived predation risk to investigate its influence on nestling growth and feather corticosterone, as well as downstream effects on fledging age, for an alpine population of Horned Lark (Eremophila alpestris) in British Columbia, Canada. Increased predation risk reduced parental provisioning rate which in turn decreased wing length and increased feather corticosterone. However, nestlings responded directly to predator presence by increasing wing growth. The overall effect of predation risk on nestling growth depended on the mediating influence of the parents, as parents that responded more strongly to predation risk (greater reduction in provisioning rate) produced smaller nestlings with greater corticosterone levels compared to parents with a limited response. Smaller nestlings with increased corticosterone subsequently fledged at an older age, remaining vulnerable in the nest for a longer period. Thus, parental behaviour influences the extent of predator effects on nestling development. Since rapid growth decreases predation risk, parents capable of compensating after disruptions in provisioning rate to maintain nestling development have a distinct fitness advantage.}

\normaltalk{\capitalisewords{Climate Delineates the Northern Range Boundary of the Northern Cardinal}}{\underline{Colleen R Miller}, Warren P Porter, Megan J Fitzpatrick, Benjamin Zuckerberg}{Over the past century, Northern Cardinals (Cardinalis cardinalis) have expanded their range north along with many other species. Seminal work on the northern range boundaries of species has suggested that climate, such as winter minimum temperatures, is an important constraint on range limitations, and that the warming climate is likely relaxing the constraint of minimum temperature and facilitating northward range expansions. A proposed alternative explanation for these ranges shifts is that species are tracking non-climatic anthropogenic changes, and that many species are increasingly synanthropic and taking advantage of supplemental food. As an example, wild bird feeding has become increasingly popular in North America, and researchers have questioned whether winter bird species are following the appearance of human-provided resources into northern latitudes. To understand the influence of climate on the northern range boundary of this wintering bird, I developed and compared mechanistic and correlational models of the Northern Cardinal across its range in the United States. In doing so, I found that the northern range boundary of cardinals aligned with energetic limitations based on the mechanistic model and provided strong support for the role of minimum temperature in defining their northern range boundary.  Furthermore, this study demonstrates that climate and metabolic constraints shapes the fundamental niche and northern range boundary of a wintering bird. This study marks a step forward in understanding the underlying relationships of shifting distributions of species during the Anthropocene and how human impacts on the climate and landscape may alter species' relationships with their environment.}

\normaltalk{\capitalisewords{Characterizing a Century of Species Losses in an Isolated Tropical Forest Fragment}}{\underline{Jenna R Curtis}, W D Robinson, Ghislain Rompr\'{e}, Bruce McCune}{Ninety years of ornithological observation on Barro Colorado Island (BCI) in the Panama Canal provide a unique opportunity to understand long-term consequences of insularization for birds in tropical forest fragments. In the century following its isolation from the mainland, BCI has lost 25\% of its resident bird species. Here we use historical species lists and recent canal-zone bird censuses to evaluate the relative strength of different explanations for bird extinctions on BCI. We also present evidence for the "drying hypothesis", which posits that fragmentation-associated edge effects produce dryer environmental conditions within isolated forest patches. Using a combination of logistic regression, hierarchical cluster analysis, and non-parametric ordination, we found that bird community structure along the canal is associated with a natural precipitation gradient, as well as forest age and anthropogenic disturbance. The species assemblage on BCI has significantly shifted over time and now more closely resembles bird communities in nearby dry forests that receive lower annual rainfall than BCI. Abundance, diet, and intolerance of dry conditions are the factors that best explain bird extinctions. A species' occurrence in drier forests has greater explanatory power than other variables previously hypothesized to influence extinction risk such as nest depredation, body mass, and resource specialization. Long-term microclimatic changes following fragmentation may create unsuitable habitat for some birds, especially insectivores with small populations or limited distribution that rely on moist forest refugia. Our results suggest the drying hypothesis provides a competitive explanation for some long-term species losses from this tropical forest fragment.}

\normaltalk{\capitalisewords{The Effects of Habitat Management and Landscape Context on Northern Bobwhite Winter Survival in Southwest Missouri}}{\underline{Alisha R Mosloff}, Mitch D Weegman, Frank R Thompson III, Tom Thompson}{Long-term declines in northern bobwhite quail and more recent questions regarding patch-burn grazing and extensive grassland management suggest additional information on habitat and management effects on bobwhite demographics is needed. Our objectives were to estimate winter survival and proximate drivers in the context of traditional (i.e., intensive) bobwhite habitat management versus extensive grassland management. We tracked 220 bobwhite 3 times per week from 1 November through 31 January in 2017-2018 and 2018-2019 on 5 conservation areas in southwest Missouri. Two sites were traditionally managed with landscape features comprising hedgerows, strip crops, and food plots, while 3 conservation areas were extensively managed with prescribed burning and grazing. We classified land cover as native prairie, mixed grassland, cool-season pasture or cropland based on aerial photography, Lidar and on-the-ground vegetation sampling, and recorded if a bird was in a shrubby thicket or mature forest edge when located. We used logistic exposure known-fate survival models in an information-theoretic framework to evaluate drivers of survival and estimated survival with the most supported models. Preliminary analyses based on the first year's data did not support land-cover effects but survival was greater for birds in shrubby thickets (66\%; confidence interval (CI) 45-80\%) than those not in shrubby thickets (38\%; CI 10-67\%). Future analyses will evaluate more quantitative measures of land cover composition and vegetation structure, and their impact on survival. The relationships between survival and habitat and management will inform conservation and management that promotes bobwhite survival during winter.}

\normaltalk{\capitalisewords{Cumulative Effects of Forest Harvesting, Wildfires, and Roads on Olive-Sided Flycatcher Habitat in British Columbia: Implications for Environmental Assessment Processes}}{\underline{Andrea R Norris}, Krista De Groot, Andrew Robinson, Kimberly M Dohms, Jeffrey Thomas, Adam Lee, Kathleen Moore, Kathy Martin}{Policy-makers require broad-scale land-use planning tools that fully integrate habitat needs, climate change projections and biodiversity goals, in order to address terrestrial bird population declines. We examined how forest stand characteristics, forest cutting, wildfires, insect outbreaks, current oil and gas development, and roads directly and indirectly influenced population densities of an aerial insectivorous Species at Risk, Olive-sided Flycatcher (OSFL). We applied generalized linear mixed-effects models to land cover and long-term avian datasets for two distinct ecoregions in British Columbia (BC), Canada; central BC (1995-2011), and northeastern BC (1993-2015). We applied habitat models to state-and-transition landscape simulation models to project future habitat conditions under various land-use scenarios. We found similarities in general habitat preferences between the two regions: OSFL populations were strongly associated with mature, open forest stands that were in close proximity to wetlands, and roads had a negative effect on OSFL densities. Forest cutting had a positive effect on OSFL densities in central BC, but a negative effect that was compounded by roads in landscapes heavily impacted by oil and gas development in northeastern BC. Under the current management scenario, we projected that the cumulative effects of oil and gas development, forest harvesting, and disturbances from fires and spruce and pine beetle outbreaks, would reduce the amount of habitat suitable to OSFL by approximately 60\% in northeastern BC.  Further work should examine the effects of loss of mature forest on other mature-forest dependent migratory birds, and explore the threats posed by emerging fire management practices across British Columbia.}

\normaltalk{\capitalisewords{Conservation of Tidal Marsh Sparrows in an Urban Setting Referencing Range Wide Nesting Occurrence}}{\underline{Alison R Kocek}, Chris S Elphick, Thomas P Hodgman, Adrienne I Kovach, Brian J Olsen, Katharine J Ruskin, Gregory Shriver, Jonathan B Cohen}{Understanding habitat selection and its fitness consequences in remnant populations of birds in fragmented and urbanized habitat may provide guidance to land managers for imperiled species conservation. We studied Saltmarsh Sparrow (Ammospiza caudacuta) and Seaside Sparrow (A. maritima) nest site selection and nest survival at five sites in New York City (NYC) and western Long Island in 2012 and 2013. Populations of Saltmarsh Sparrows in NYC selected nest sites with greater percent cover of tall forms of low marsh grasses than is typical for this species and Seaside Sparrows selected an upland shrub species as a possible substitute for low densities of tall low elevation marsh grasses at one site.  Daily nest survival increased with nest height above the ground for both species, contrary to previous studies for Saltmarsh Sparrows in intact habitat and nest heights were found to be greater in New York than at other study locations.  The difference in nest site selection we observed in NYC populations from published studies did not appear to have a short-term fitness cost. Although high marsh is the optimal habitat restoration target for Saltmarsh Sparrows, it may be difficult to create in small, urbanized marshes and our results suggest that other approaches may be successful. A mix of substrates and vegetation heights may allow sparrows to exhibit flexibility in nest site selection and promote nest survival in the face of changing limiting factors.}

\normaltalk{\capitalisewords{Chronic, Sublethal Effects of High Temperatures will Cause Severe Declines in Arid-Zone Birds During the 21st Century}}{\underline{Shannon R Conradie}, Stephan M Woodborne, Cunningham J Susan, Andrew E McKechnie}{Earth's climate is warming at an unprecedented rate, with climate change affecting virtually every habitat on the planet. Birds inhabiting hot, arid regions may be among the terrestrial organisms most vulnerable to climate change. The potential for increasingly frequent and intense heat waves to cause lethal dehydration and hyperthermia is well documented, but the consequences of sub-lethal fitness costs associated with chronic exposure to sustained hot weather remain unclear. We mapped exposure to acute lethal risks and chronic sub-lethal fitness costs using data for southern African species occurring within the Kalahari Desert, under past, present and future climates. The risks of lethal dehydration or hyperthermia will remain low during the 21st Century. However, exposure to conditions associated with chronic, sublethal costs related to body mass loss, reduced nestling growth rates and increased breeding failure will expand dramatically by the end of the century. For instance, by the 2080s the region will experience 10-20 consecutive days per year where southern pied babblers (Turdoides bicolor) will lose ~ 4 \% of body mass per day, conditions under which persistence for this species will be unlikely. Similarly, exposure to air temperature maxima associated with delayed fledging, reduced fledgling size and breeding failure will increase several-fold in Southern Yellow-billed Hornbills (Tockus leucomelas) and Southern Fiscals (Lanius collaris). Our analysis reveals that sublethal costs of chronic heat exposure are likely to drive large declines in avian diversity in the southern African arid zone by the end of the century.}

\normaltalk{\capitalisewords{Will Low Oxygen Slow Range Shifts? Response to a Novel High Elevation Environment from Populations Across a Hummingbird Range}}{\underline{Austin R Spence}, Morgan W Tingley}{Many montane species are shifting upslope to track thermal regimes in response to global climate change, yet species moving upslope experience additional abiotic challenges, particularly lower oxygen availability. To better predict how species will respond to warming temperatures, we need to understand if other abiotic factors will hamper their ability to track their thermal niche. To do this, we tested the thermal and respiratory physiology of Anna's Hummingbirds (Calypte anna) along an elevational gradient within their current elevational range and above their elevational range where, based on temperature alone, they are expected to be in the future. We collected birds from three elevations across their range and tested their aerobic performance and thermal tolerance at 1200 meters above sea level. We performed these tests again above their range limit, at 3600 meters above sea level, to test their response to novel temperature and oxygen conditions. Birds sourced from higher elevations exhibited better aerobic performance when tested within the range, but these differences did not confer differential performance above the elevational range limit. Alternatively, thermal tolerance showed little intra-specific variation, and exhibited similar responses at the high elevation site. This study shows physiological performance within the range may not be the best indicator for how the species will do in novel abiotic conditions. Additionally, it will become imperative to include multiple physiological responses to novel environments, as opposed to temperature alone, to help us better predict how species and their ranges will respond to future warming temperatures.}

\normaltalk{\capitalisewords{Skeletal Distribution of Medullary Bone in Neornithes - Implications for the Identification of Reproductive Tissues in Non-Avian Dinosaurs}}{\underline{Aurore R Canoville}, Mary H Schweitzer, Lindsay E Zanno}{To date, the skeletal distribution of medullary bone (MB), a sex-specific tissue formed by female birds during lay, has been poorly documented in wild birds. Studies investigating MB structure, composition, and metabolism, have focused almost exclusively on long limb bones, spurring the common misconception that its deposition is concentrated within these elements.   The present work constitutes the first comprehensive investigation of MB skeletal distribution across Neonithes using micro-computed tomography and histochemical data, revisits previous hypotheses pertaining to MB distribution patterns, and provides new criteria for identifying purported MB-like tissues in non-avian dinosaurs. We document that the skeletal distribution of MB varies interspecifically, but does not differ between captive and wild-caught individuals. We find MB is a systemic tissue that can be deposited within virtually all skeletal regions, and note that it is uniformly present in the tibiotarsus of studied specimens. Our results confirm previous hypotheses that skeletal distribution of MB is directly related to the distribution of red bone marrow, and inversely correlated to the combined skeletal distributions of pneumaticity and yellow bone marrow. The extent of pneumaticity skeletal distribution is linked to the body size and lifestyle habits of bird species. Hence we find that small-bodied and diving birds possess widespread MB deposition, whereas MB distribution is highly restricted in large-bodied or efficient flyers.   The homology of the pulmonary system between living birds and some non-avian dinosaurs permits us to derive a series of location-based predictions that can be used to critically evaluate purported MB-like tissues in fossil specimens.}

\normaltalk{\capitalisewords{Replicated Invasions Reveal Hotspots of Natural Selection in the European Starling}}{\underline{Natalie R Hofmeister}, Lee A Rollins, Katarina Stuart, Scott J Werner, Irby J Lovette, David F Clayton}{During invasion, a population's success hinges on a combination of demographic luck and localized selection. Here we compare whole-genome sequences from one native and two introduced starling populations to discriminate among demographic and selective drivers of the starling's success. We also investigate patterns of genomic diversity and differentiation using reduced-representation genome sequencing across North American (n=17) and Australian (n=24) populations. Since colonizing Australia and North America in the 19th century, starlings have diverged in parallel expansions. We hypothesize that rapid adaptive evolution will shape observed genetic patterns among in replicated invasions. We find that more variants than expected are highly differentiated but low in nucleotide diversity, and we show directional shifts in allele frequency under novel selective pressures. We verify candidates for selection using models of divergence that are conditioned on site frequency spectra calculated for candidate genomic regions. Despite rapid expansion in evolutionarily young invasions, we find that precipitation and temperature drive incipient local adaptation in both North American and Australians invasions. However, demographic history and population structure differ among invasions: geographic distance explains much of the genetic variation in Australia, whereas genetic variation in North American starlings is best explained by bioclimatic variation. Candidate variants supported by multiple selection inferences are in coding regions of genes that are involved in energy metabolism and neurotransmission. Parallel invasions illustrate how selective pressures and demographic booms can shape an infamous invader's genome to support its spread world-wide.}

\normaltalk{\capitalisewords{Genomics of Parallel Divergence and Local Adaptation in Two North American Woodpeckers}}{\underline{Lucas R Moreira}, Brian T Smith}{Hairy (Picoides villosus) and Downy (P. pubescens) Woodpeckers inhabit a wide variety of habitats in North America, are syntopic, and exhibit remarkable parallel patterns of geographic variation in plumage and body size. Despite the striking similarity in phenotypic variation, these woodpeckers are not sister taxa, suggesting their exceptional resemblance evolved independently. Moreover, these two species share a common population history – during the glacial cycles of the Pleistocene, a large portion of their present-day distribution was glaciated, and populations of both species were restricted to southern refugia. Considering the ecological convergences between the Downy and Hairy Woodpeckers and the variety of biotic and abiotic selective pressures they have shared throughout their evolutionary history, we investigated (1) whether populations have responded similarly to past changes in climate, and (2) whether comparable genetic mechanisms have been targeted by natural selection to promote post-expansion local adaptation. We used whole-genome resequencing of 140 individuals of Downy and Hairy Woodpecker from seven populations across North America to test for shared signatures of demographic and adaptive processes. Our results reveal spatially and temporally concordant changes in population size in Hairy and Downy Woodpeckers. However, despite similar demographic histories, natural selection has recruited a different set of genes to promote local adaptation in these two species. A closer look at the loci under selection reveals a range of molecular functions, including metabolism, immunity, and thermoregulation. These findings suggest that despite shared population history, local adaptation was predominantly governed by historical contingency rather than evolutionary determinism.}

\normaltalk{\capitalisewords{Timing is of the Essence: Later Breeding Predicts Lower Survival in American Kestrels (Falco Sparverius)}}{\underline{Kathleen R Callery}, John A Smallwood, Amber L Eschenbauch, Emilie R Luttmann, Julie A Heath}{Organisms have evolved annual cycles of life history stages so that energetically expensive events -  like breeding - are timed to coincide with peak food abundance. Breeding asynchronously from this optimal time could have negative fitness consequences, especially if resource scarcity forces breeders to make trade-offs between sustaining their offspring and maintaining their own health. We used breeding season mark-recapture data from long-term monitoring sites of American kestrels (Falco sparverius) to test the hypothesis that the synchrony of nesting and the onset of spring at the nest site would predict adult survival. In addition, we also tested whether the timing of when a bird fledged relative to the onset of spring predicted hatch-year mortality. We modelled apparent survival and capture rates using Cormack-Jolly-Seber mark-recapture survival models. All of the models were run in MARK through the RMARK interface. Preliminary results show that timing did predict apparent survival rates of adult American kestrels, with birds breeding later than the onset of spring having lower survival rates than birds that bred earlier. Climate change has the potential to shift the timing of peak food availability faster than some species are able to change their breeding phenology. Understanding the fitness consequences of the existing variation in breeding phenology for a species will help us predict how that species could be affected by a changing climate.}

\normaltalk{\capitalisewords{Cross --Species Responsiveness to Auditory Stimuli:  Do Birds and Humans React to Similar Sounds in Similar Ways?}}{\underline{Janice R Crook-Hill}}{Research comparing acoustic communication in distantly related vertebrates is limited. Our study is motivated by the question, "Do avian vocalizations and human music have common evolutionary origins?" If so, then we suggest there may be similar elements in avian vocalizations and human music which elicit similar responses across species. To explore this, we used recorded segments of alarm calls and contact calls of wild Tufted Titmice and "alarming" and "calming" human music. We played all four segments to both wild birds and human volunteers and compared responses. Avian behavioral responses were observed through playback at feeders. Human participants' reactions were assessed through a survey given after hearing the recordings. Our data indicate similar responses from birds and humans in that alarm calls and "alarming music" elicited avoidance behavior in birds and were interpreted by humans as frightening, whereas contact calls and "calming" music elicited no change of behavior in birds and were interpreted as calm by humans. Next, we plan to isolate similar acoustic parameters within the "alarming" and "calming" segments of each species' recordings, playback the isolated segments, and again document responses of both birds and humans, in an attempt to find commonalities in music and avian vocalizations that elicit similar responses across species.}

\normaltalk{\capitalisewords{Variation in the Microbiome of Kirtland's Warblers Between their Wintering and Breeding Areas}}{\underline{Heather R Skeen}, Nathan W Cooper, Shannon J Hackett, John M Bates, Peter P Marra}{Each individual animal is host to a complex microbial community composed of millions of microorganisms, collectively known as the microbiome.  The microbiome is quickly becoming recognized as an influential aspect of the host.  In this study, the fecal microbiome of an endangered species, Kirtland's Warbler (Setophaga kirtlandii), is quantified and compared using individual birds tracked with coded radio tags and Motus towers from their wintering grounds on Cat Island, Bahamas through to their breeding grounds in lower Michigan, thereby significantly advancing our understanding of birds throughout their annual cycle. Sampling was conducted at both locations for two consecutive years.  Using metagenomics sequencing and multi-level comparisons of microbial diversity, the changes, similarities, and differences between individual warblers and populations were identified and analyzed.  These microbiome data join a rich matrix of information on these birds, including fecundity, age, diet, weight, and migration dates, as well as breeding and wintering territories.  The composition of the microbiome is relatively understudied in wild animals, especially in migratory birds.  Identifying changes that occur during different stages of migration will lead to a more integrative understanding of the biological changes that occur when a bird travels long distances between dramatically different habitats.  Preliminary results show distinct partitioning of microbial communities between geographic locations and migratory points as well as shifts in microbiome composition within individual birds.}

\normaltalk{\capitalisewords{The Common Loon Genoscape}}{\underline{Alec R Lindsay}, Brenda Larison, Christen Bossu, Jasmine Rajbhandary, Michael D Sorenson, Jeffrey M DaCosta, Joseph D Kaplan, David C Evers, James Paruk, Thomas B Smith, Kristen Ruegg}{The common loon (Gavia immer) is an iconic migratory species of North America. A comprehensive understanding of the conservation concerns surrounding this long-lived and ecologically sensitive species is limited by a lack of information on loon population genetic structure. Our common loon genetic archive comprised 585 samples collected from: breeding birds from Alaska to Iceland, wintering birds from California to Scotland, and migratory birds from the Great Lakes and the American Southwest. We used RADseq techniques on a subset of individuals to build a panel of 158 loci that were subsequently used in a Fluidigm SNP genotyping assay for all birds in the archive. Population genetic analyses showed modest subdivision of breeding populations across North America, probable trans-Rockies migration events, and continent-wide migratory significance of the Great Lakes. An isolated Wyoming population of loons groups with the geographically distant Alberta/Saskatchewan birds, rather than birds in nearby Montana or Washington State. Samples from birds wintering in the Shetland Islands (Scotland) cluster with birds breeding on the East Coast of North America and Iceland. The Fluidigm assay we designed will allow more samples to be added to this dataset at low cost, making future collections from breeding, migrating, and wintering birds more valuable for ongoing conservation efforts for this species.}

\normaltalk{\capitalisewords{Evolution of Migration in the Nightingale-Thrushes (Turdidae: Catharus) Inferred via Morphometric and Phylogenetic Analysis}}{\underline{Matthew R Halley}, John Klicka, Therese A Catanach, Jason D Weckstein}{The genus Catharus has been hailed as a model system for investigating the evolution of migration, though a majority of named taxa have yet to be studied via phylogenetic analysis. Previous studies have reconstructed the migratory taxa as a paraphyletic group and determined that the common ancestor of the genus was likely a migratory species. However, the evolution of migration is still largely equivocal because two scenarios are equally parsimonious: (1) migration was lost once, in the common ancestor of all Catharus except C. ustulatus, and subsequently regained twice, in C. guttatus and in the common ancestor of the clade containing C. fuscescens, C. minimus and C. bicknelli; or (2) migration was lost three times: in C. occidentalis, in the common ancestor of C. frantzii and C. gracilirostris, and in the common ancestor of the clade comprised of C. mexicanus and its allies. We used DNA sequences from the flanking regions of ultraconserved elements (UCEs) to reconstruct a phylogeny of Catharus, with broad geographic breadth and nearly comprehensive sampling of named taxa. As a quantitative proxy for scoring migration and potential migratory distance traveled we calculated hand-wing index (HWI) for a taxonomically comprehensive sample of adults (n $>$ 1000 specimens). We mapped those quantitative data to the resulting phylogeny to reconstruct ancestral values of wing morphology and test hypotheses about the evolutionary history of seasonal migration.}

\normaltalk{\capitalisewords{Unintended Consequences of Translocation: Reproductive Skew and Reduced Effective Population Size in the Red-Cockaded Woodpecker}}{\underline{J R Bowman}, Greg T Thompson, Emily Angell}{Translocation is an important conservation tool to rescue small, isolated populations. The infusion of new individuals can increase genetic diversity and lead to rapid population growth. However, if outbred pairs have greater fitness, this can potentially lead to high reproductive skew and unintended consequences for genetic diversity. Over nearly 20 years (1998-2016) we translocated 54 red-cockaded woodpeckers (RCW) to Avon Park Air Force Range, in central Florida. From 2004 to 2018, the population nearly doubled (22 to 42 groups), but one translocated male was especially successful, breeding for 14 years and producing 11 young that eventually became breeders – more than any other bird over 27 years of monitoring. One grandson was nearly as productive. The result of this reproductive skew is that by 2018, 33\% of all birds in our population were direct descendants of this one male, reducing the effective population size. We illustrate how mean kinship and genetic diversity changed before and throughout our period of translocations and the impact of this one lineage on the genetic health of this population. We emphasize that inbreeding resulting from high reproductive skew of outbred pairs might be especially severe in small, inbred populations, even as they grow. At present, translocation of RCWs is restricted to very small populations ($<$ 30 groups); however, translocation for the primary goal of genetic rescue can counteract skew, reduce inbreeding and increase genetic diversity even in medium or large-sized populations.}

\normaltalk{\capitalisewords{Male Red Carotenoid Coloration Indicates Mitochondrial Performance in the House Finch}}{\underline{Wendy R Hood}, Geoffrey E Hill, Andreas N Kavazis, Yufeng Zhang}{The coloration displayed by many birds conveys important social signals that are commonly assessed during sexual selection. Most yellow, orange, and red colors observed are derived from dietary carotenoids. These pigments may either be shunted directly to growing feathers and bare-parts, or they may be biochemically modified. Those species that convert dietary yellow to red carotenoids have been shown to reveal condition-dependent signals of their relative quality. Birds with redder ornaments commonly outperform birds with less red ornaments, including provisioning young, resisting and recovering from parasites, and managing oxidative stress, among others. It has been proposed that the unique redox conditions required for the ketolation of yellow to red carotenoids must occur within the mitochondrion. Following this prediction, we hypothesized that the conversion of yellow dietary carotenoid pigments to red in the House Finch, and thus the relative expression of red feathers, would be inexorably linked to mitochondrial function. We collected molting hatch-year House Finches in Auburn, AL. Each male was photographed to quantify feather hue and the liver removed to evaluate mitochondrial bioenergetic capacity. Among the significant patterns observed includes a positive correlation between the hue of growing feathers and the respiratory control ratio. The results of this investigation suggest that the reason so many seemingly disparate variables link to male colorations in red birds, and house finch in particular, is that the conditions required for the production of red coloration are linked to the most fundamental of physiological processes, oxidative phosphorylation}

\normaltalk{\capitalisewords{Infanticide and Adoption in the Green-Rumped Parrotlet: Killing Cavities, Remorseless Rogues and Betrayed Saints}}{\underline{Steven R Beissinger}, Karl S Berg}{We examine infanticide and adoption in a socially monogamous parrot during a 25-year study in Venezuela. A total of 595 nestlings and 244 eggs were killed from infanticide attacks by parrotlets at 256 nests. Attacks occurred throughout the nesting cycle but were most frequent during hatching and post-brooding, resulting in complete nest failure at 47\% of nests. Both sexes killed offspring, but males were three-times as likely to be attackers as females. Infanticide occurred in two contexts: (1) at nests with intact breeding pairs (68\%); and (2) at nests where a mate had died (32\%). At nests with intact pairs, infanticide rate was positively related to population size, and male-female pairs competing for nest sites accounted for 75\% of the attacks. These attacks occurred more often at high quality nest sites, which caused pairs to abandon them if nests failed. Mate loss was usually accompanied by conspicuous contests attracting up to 15 competitors vying to become step-parents. Of 128 widows, 88\% secured replacement mates and half the Step-Dads assisted their widows (adopted). Less than half of 43 widowers attracted Step-Moms and only 2 adopted offspring. At nests with widows, Step-Dads accounted for 63\% of infanticide attacks, while only 30\% of attacks at nests with widowers were attributed to Step-Moms. Step-Dads at nests where infanticide occurred subsequently nested significantly sooner and produced more offspring with their females than Step-Dads that adopted. Infanticide in parrotlets is driven primarily by resource competition and secondarily by sexual selection, while adoption appears maladaptive.}

\normaltalk{\capitalisewords{Within-Pair Annual and Lifetime Reproductive Success Drives the Opportunity for Sexual Selection in Black-Throated Blue Warblers}}{\underline{Ryan R Germain}, Michael T Hallworth, T S Sillett, Michael S Webster}{For many socially-monogamous species, male reproductive success stems from both 'within-pair' (WP) offspring produced by their socially-paired mate and additional, 'extra-pair' (EP) offspring sired throughout the population. Both WP and EP pathways offer distinct opportunities for sexual selection to operate in wild populations, as each are the product of sub-components of mate attraction, female fecundity, and paternity allocation. Determining where the greatest sources of population-wide variance and covariance exist among these components, and hence where the greatest opportunities for sexual selection to operate may occur, can shed key light on the evolutionary dynamics of animal mating systems in nature. We use 20 years of complete pairing, nesting, and genetic paternity data from a population of Black-throated Blue Warblers (Setophaga caerulescens) in New Hampshire to decompose population-wide variance in male annual and lifetime reproductive success (ARS and LRS, respectively) to their WP and EP components, and thereby quantify the opportunity for sexual selection to operate in this system. The majority of variance in male ARS and LRS over the study was attributable to WP success, specifically the number of broods each social female produced, and the proportion of WP offspring that the male sired. Previous evidence suggests that high-quality territories facilitate female double brooding, and that males on such territories prioritize mate-guarding rather than seeking EP matings. These results indicate that habitat-based variance in WP success drives the opportunity for sexual selection in this system, and that males should prioritize securing high-quality territories over seeking EP mates to maximize individual fitness.}

\normaltalk{\capitalisewords{Within-Individual Advancement in Bar-Tailed Godwit Departure from New Zealand Does not Lead to Earlier Arrival in Alaska}}{\underline{Jesse R Conklin}, Simeon Lisovski, Phil F Battley}{Long-distance migratory birds generally show weaker responses to climate-related advancement in breeding phenology than do short-distance migrants, and this is routinely attributed to greater behavioural inflexibility in the initiation of migration. However, this assumption has never been tested with individual-level data at the point of migratory departure. In Bar-tailed Godwits (Limosa lapponica baueri), we show that northward departure on a 10,000-km non-stop flight from New Zealand advanced by six days during 2008–2018, and that this population-level change occurred almost entirely through within-individual advancement (mean = –0.47 d/yr). Advancement in departure was greatest for individuals breeding in northern Alaska, but geolocator-tracking in 2008–2009 and 2013–2014 revealed that arrival in northern Alaska did not advance, due to a longer stopover in the threatened coastal areas of the Yellow Sea. These results challenge the characterization of timing of long-distance migration as endogenously controlled and inflexible, and clearly demonstrate that the magnitude of responses observed at breeding grounds may misrepresent the inherent flexibility in migratory systems. Further, within-population variation in advancement of New Zealand departure did not match expectations from geographic variation in advancement of snowmelt in Alaska (which was greater for southern breeders), calling into question climate-related changes in breeding phenology as the main driver for changes in migration timing. We propose that natural or human-induced changes at staging sites can disrupt adaptation to climate-related advancement in breeding phenology, but may also independently drive phenological changes in migration.}

\normaltalk{\capitalisewords{Detecting Shifts in the Phenology and Distribution of Bird Assemblages on the Kenai Peninsula, Alaska in a Changing Climate}}{\underline{Dawn R Magness}, Todd Eskelin}{The Kenai Peninsula is experiencing longer growing seasons that are warmer and subject to drought because of increased evapotranspiration and reduced snow pack. Winters are milder. Treeline is rising into alpine habitats, epidemic bark beetle outbreaks have triggered regional tree mortality with corresponding changes to forest structure, and woody plants are colonizing peatlands. Climate velocity forecasts to 2050 suggest that species on the Kenai will need to move 4-11 km/year to track suitable climate conditions and species could move 12-19 km/yr to colonize emerging climate conditions on the Kenai. Bird may have high adaptive capacity due to their dispersal abilities and behavioral plasticity to respond to this rapid change. We use historical observations from the 1960s and ebird data to identify species that have responded with lengthened breeding season, overwintering, and colonization. We use acoustic data recorded at 12 locations from April – August in 2017 and 2018 to establish baseline phenology measured by arrival date, peak singing, and detectability across season. Bird responses vary by species. Short distance migrants have lengthened breeding seasons, bi-modal peak singing distributions, and have begun overwintering. Long-distance migrants have less documented behavioral plasticity.  We hope to leverage acoustic data to document phenological shifts in detail and to detect colonizing species on the Kenai Peninsula.}

\normaltalk{\capitalisewords{Minding the Gap: An Energetics-Based Perspective on not Migrating}}{\underline{Dan R Ruthrauff}, Anne Dekinga, Robert Gill, Jr, Theunis Piersma}{The Pacific Ocean represents the largest natural barrier to movements of terrestrial animals, but there is an increasing realization that this seemingly impassible expanse is regularly transited by a diverse avian assemblage. Many of these migrants are shorebird species, and trans-Pacific passages by shorebirds require dramatic physiological changes and energetic expenditures that define the limits of vertebrates. Despite these perceived costs, extreme migration enables species to avail themselves of seasonally productive and predictably safe environments across the globe. The Pacific Basin is rimmed with highly productive wetlands and estuaries that support many of these migratory species. Cook Inlet, Alaska, is one such site, but instead serves as the setting for an ecophysiological feat that rivals extreme migrations. The Rock Sandpiper forgoes long migrations in favor of exploiting resources during winter in Cook Inlet, the world's coldest regularly inhabited shorebird site. To survive winters in Cook Inlet, Rock Sandpipers likewise retool their physiology and incur energetic debts approaching those of shorebirds undergoing active migration, but do so across a long, dark winter season rather than a punctuated migratory period. On average, Rock Sandpipers expend energy over the October–April winter season at rates that exceed basal metabolic rates by a factor of nearly 5, requiring the hourly consumption of about 125 10-mm individuals of their primary prey, the bivalve Limecola balthica. This unique life history raises questions about energetic constraints in the evolution of migratory systems, questions underscored by the puzzling absence of other shorebird species in Cook Inlet during winter.}

\normaltalk{\capitalisewords{Rapid Global Differentiation of Migratory Phenotypes in Bar-Tailed Godwits}}{\underline{Jesse R Conklin}, Yvonne I Verkuil, Theunis Piersma, Michaël C Fontaine}{Migratory phenotypes are generally considered heritable and under strong selection, and therefore relatively inflexible in changing conditions. However, divergent patterns observed within species attest to recent evolution of migratory phenotypes, and colonization of flyways, since the Last Glacial Maximum (LGM; $<$20,000 years). Bar-tailed Godwits (Limosa lapponica) include five recognized subspecies distinguished by plumage, body size, annual-cycle phenology, and distance and direction of migration. Using genome-wide information ($>$20,000 single-nucleotide polymorphisms) across all major breeding populations, we assessed the fine-scale genetic structure and evolutionary history of the species. Our analyses support a primary divergence of two major clades (Beringia and western Palaearctic), which later admixed to form the NE Russian subspecies menzbieri. We found clear geographic structure within the Alaska breeding subspecies baueri, which reflects latitudinal clines in morphology and migration timing. In contrast, we found no genetic structure within taymyrensis, despite the use of two completely distinct migration routes in that subspecies. In general, magnitude of phenotypic differences did not match global genetic population structure, suggesting that migratory differences either do not impede gene flow, or are too recent to leave a genome-wide signal. We propose that population structure in Bar-tailed Godwits is mainly driven by phylogeographic processes (e.g. post-LGM changes in habitat and sea level) rather than isolation by phenotypic differences. Ongoing work will further reconstruct the temporal and spatial patterns of diversification, and explicitly examine the impact of phenotypic differences on gene flow.}

\normaltalk{\capitalisewords{Long-Term Dynamics of Tree-Cavity-Nesting Communities in Temperate and Subtropical Forests}}{Kristina L Cockle, \underline{Andrea R Norris}, Karen L Wiebe, M K Trzcinski, Amanda B Edworthy, Kathy Martin}{Communities that rely on tree cavities can be viewed as hierarchical networks, or nest webs, whereby cavities flow upward from trees to cavity producers (e.g., woodpeckers, decay organisms) to a diverse assemblage of non-excavators (e.g., parrots, songbirds). We studied over 5000 tree-cavity nests in subtropical Argentina (2006-2019) and temperate British Columbia, Canada (1995-2016) to understand community dynamics and response to disturbance. Across both regions, short-term nest-site selection studies failed to indicate the long-lasting cavities that supported most nests and species over the long term. Species turnover among (vs. within) cavities was the primary driver of cavity-nester diversity at the landscape scale, such that biodiversity of cavity nesters was maintained by a few long-lasting, highly productive cavities and many single-use, fast decaying cavities across forest landscapes. During a bark beetle outbreak in Canada, Red-breasted Nuthatches (Sitta canadensis) moved into new habitat with high food availability, and increased their excavation rates, whereas non-excavators shifted from dependence on cavities excavated by Northern Flickers (Colaptes auratus) to those created by a broader suite of beetle-eating excavators. After the beetle outbreak, wildfires destroyed 40\% of cavities and flickers increased their excavation rate from 18\% to 31\%. In Argentina, use of excavated (vs. decay-formed) cavities increased from 7\% in old-growth forest to 57\% on farms. We propose that multiple species of excavators act as agents of resilience in nest webs, allowing non-excavators to exploit otherwise cavity-poor habitats, and increasing the ability of nest webs to respond dynamically to disturbance.}

\normaltalk{\capitalisewords{Banding Together to Learn and Preserve: 25 Years of Research in Cross-Cultural Western Alaska}}{\underline{Craig R Ely}}{Humans representing diverse user groups are often implicated in natural resource issues, but these same user groups are often ignored as cooperators in efforts to solve issues they contribute to.  In this talk I describe ornithological research conducted in 1986−2010 on the Yukon-Kuskokwim Delta, Alaska, in cooperation with people from the community of Chevak.  The research was initiated in response to declines of geese populations in the 1980s, which also led to the first flyway-wide agreement among native groups, state, federal, and private conservation organizations (The Hooper Bay Agreement).  Our study was one of many which led to an unprecedented influx of outside researchers into remote areas which for millennia have been the homeland of indigenous people in western Alaska.  Despite a strong public relations campaign by the U.S. Fish and Wildlife Service espousing the conservation needs and research benefits, there was a pervasive mistrust of outsiders spending summers in traditional use areas.  Therefore, we enlisted youth from Chevak to help capturing Cackling Geese (Branta hutchinsii minima) as part of research to monitor their distribution and survival in the Pacific Flyway, and to explain our work to local people.  We worked and lived together at a remote site at the nexus of Cup'ik, western, and scientific cultures. By sharing our mishaps and good fortune I highlight benefits of cooperative research.}

\normaltalk{\capitalisewords{Can Oceanic Effects on Growth and Time to Fledging Mediate Terrestrial Predator Limitation of an at-Risk Seabird?}}{Timothy W Knudson, \underline{James R Lovvorn}, M J Lawonn, Robin M Corcoran, Daniel D Roby, John F Piatt, William H Pyle}{Kittlitz's Murrelet (Brachyramphus brevirostris) nests solitarily on the ground in barren, often alpine areas where exposure to predators is generally low. Nestling growth rates are exceptionally high and nestling periods very short compared to those of other Alcidae. This strategy reduces duration of exposure of nests to predators, but demands adequate deliveries of high-energy prey. We investigated whether variations in energy content of prey fed to chicks could alter growth rates and resulting duration of predator exposure, and whether prolonged predator exposure appreciably reduced nest success. From 2009‒2016, we monitored 139 nests; 49\% were depredated (almost all by foxes) and 25\% fledged. Prey fed to nestlings were mainly Pacific sand lance (80\%) and capelin (19\%), with capelin having 2.3× higher energy content per fish. In a year of slow chick growth, increased sand lance energy density of 31\% from 4.29 to 5.64 kJ/g (within published values), or increased proportion of capelin in the diet from 5.6\% to 27.2\%, were needed to achieve maximum chick growth. Maximum growth rates were attainable by delivering only 1.9 capelin/day versus 5.5 sand lance/day. Slow growth increased time to fledging by up to 5 days, decreasing survival by 7.7\% (0.142 to 0.131). Breeding propensity of Kittlitz's Murrelet averages only 20\%, so even small effects on nest success could affect populations. Although nest success was limited mainly by predation, oceanic effects on prey quantity and quality had overriding effects in one year, and small but substantive effects in other years by mediating exposure to predation.}

\normaltalk{\capitalisewords{Social Dominance Hierarchies among Species and their Consequences for Behaviour, Ecology and Evolution}}{\underline{Paul R Martin}}{Every day, species fight, displace, signal, and avoid each other, often in the context of competition for shared resources. What are the consequences of these interactions for the behaviour and ecology of species? Studying ecologically-similar, closely related species of birds, we found the outcomes of aggressive contests for resources are usually asymmetric, with a consistent social hierarchy of dominant and subordinate species within a species pair. Dominant species are usually larger, but the importance of body size declines with evolutionary distance among species. Subordinate species tend to arrive and breed later than sympatric dominant species, have lower survival and greater investment in annual reproduction, and migrate further, all after controlling for differences in body size. Subordinate species are also more likely to diverge in ecologically important traits, such as leg morphologies, when they co-occur with dominants. Subordinate species are often excluded from preferred habitats by dominant species; comparative tests with urban-adapted birds suggest that these effects are phylogenetically widespread and occur worldwide. Whether subordinates persist with dominants, however, depends on the nature of shared habitats; in the case of urban-breeding birds, subordinate species persist in cities in economically developing countries, but are reduced or excluded from cities in developed nations. Overall, mounting evidence suggests that the position of a species within social dominance hierarchies has broad implications for species' behaviour, ecology, and trait evolution, but the extent and importance of these interactions are not yet fully understood.}

\normaltalk{\capitalisewords{Global Biogeographic Gradients in Avian Colouration}}{\underline{Christopher R Cooney}, Yichen He, Gavin H Thomas}{The idea that life is generally more colourful at tropical latitudes has enthralled biologists for centuries, yet attempts to decisively test this and other ecogeographical 'rules' of animal colouration (e.g. Gloger's rule) have been hampered by the lack of quantitative, high-resolution colouration data for large numbers of species. However, as technology improves (and prices fall) such data is becoming increasingly easy to collect, opening up exciting opportunities—and new challenges—for testing these enduring hypotheses at global scales. In this talk I will present the initial results of a project whose ultimate aim is to capture and quantify plumage colouration for all $>$10,000 extant bird species. Focusing on a dataset composed of $>$100,000 calibrated UV/Vis images of museum specimens covering almost all passerine species, I will describe how crowd-sourced data collection combined with 'Deep Learning' computer algorithms can assist in the extraction of accurate colour measurements for very large image-based datasets. I will then show how this unique dataset can be used to first document and then test several longstanding hypotheses for the relationship between avian colouration and geographical variation in climatic conditions, habitat, and ecology, thus providing new insight into the biogeography of animal colouration at global scales.}

\normaltalk{\capitalisewords{Golden-Winged Warbler Conservation: Using Science to Take Action}}{\underline{Margaret R Fowle}, Mark S LaBarr, Sara E Barker}{Early successional and young forest habitat and the birds that use it are declining in eastern North America. Improving and expanding this habitat on both private and public lands requires active management, but landowners need technical assistance and, in many cases, financial incentives, to do so. The Champlain Valley of Vermont is host to the only known breeding population of Golden-winged Warblers in Vermont, and Audubon Vermont has been working with landowners in the Champlain Valley to improve habitat for this bird since 2009. In 2015, Audubon began collaborating with the Cornell Lab of Ornithology to prioritize its Golden-winged Warbler habitat management efforts in cooperation with local land trusts in the southern Champlain Valley. With the support of federal cost-share incentive programs, private foundations, and the Cornell land trust small grant program, work has included several projects managing both private and public lands for golden-wings and other shrubland obligates. On-the-ground management activities followed research-based habitat recommendations from the Golden-winged Warbler Working Group's Conservation Plan and Management Guides as well as best practices developed by Audubon Vermont and New York. These actions have proved successful in terms of golden-wing and other shrubland bird species response, and outreach and collaborative efforts have effectively engaged private landowners, land trusts, and land managers alike.}

\normaltalk{\capitalisewords{Contrasting Regional and Range-Wide Trends Inform Causes of Decline: Case Study with the Red-Cockaded Woodpecker and Associated Pineland Birds}}{\underline{Jeffrey R Walters}, Kenneth V Rosenberg, John R Sauer}{Efforts to recover the endangered Red-cockaded Woodpecker (RCW) over the past 25 years have included restoration of millions of hectares of pine habitats in the southeastern United States, particularly the longleaf pine habitats that are the core of the species' range. In particular, the return of frequent fire to these systems through prescribed burning has transformed many areas of longleaf pine from a degraded state characterized by dense hardwood mid-story and sparse groundcover to their historic savanna-like condition. A recent Species Status Assessment documented that RCW numbers have increased considerably in response to this management, and that most populations are increasing currently. Work at Marine Corps Base Camp Lejeune in coastal North Carolina indicates that the RCW may be serving as an effective umbrella species: not only are site occupancy and density of RCWs strongly correlated with an index of habitat quality for RCWs, but so are site occupancy and density of many other avian species. We contrasted regional and range-wide population trends of species found in southeastern pine habitats, many of which have benefited from habitat improvements for RCWs and some of which have not, to look for a signal of an impact of longleaf pine restoration. We found that several species experiencing range-wide declines (e.g. Red-headed Woodpecker, Brown-headed Nuthatch) show positive population responses in regions with most intensive RCW management. The patterns observed provide insights into causes of declines of North American bird species, including the impact of habitat change in the breeding grounds on Neotropical migrants.}

\normaltalk{\capitalisewords{Are Declines in Migratory Bird Species more Closely Associated with Breeding or Wintering Locations?}}{\underline{John R Sauer}, Kenneth V Rosenberg, Peter P Marra, William A Link}{Documenting and modeling the dynamics of large-scale patterns of population change among species provides essential information for development of regional and national biodiversity conservation strategies.  Observations of consistent population declines in species that share life history traits (e.g., migration status or foraging strategy) or common breeding biomes (e.g., grassland breeding or wintering birds) have been used to infer common causality of declines among these species, leading to conservation actions designed to address these causes.  Trend information from continental scale surveys such as the North American Breeding Bird Survey (BBS) has also been used to address fundamental ecological questions about the relative importance of breeding season and wintering season locations in determining population regulation in migratory birds.  We use recently updated categorizations of breeding and wintering biomes for migratory bird species to evaluate the relative variation in population trends in breeding and wintering biomes for species monitored by the BBS.   We categorized 325 species into 9 breeding biomes and 15 wintering biomes, and used a hierarchical model to summarize patterns in trends.  Analysis of these results highlight clear patterns of species declines among breeding and wintering groups.  For species in most breeding biomes wintering location influence population trends.  Noteworthy declines occur among species in certain wintering biomes (e.g., Chihuahuan Grasslands and South American Lowlands).  Overall these results reaffirm the importance of non-breeding season factors for explaining declines in North American migratory birds.}

\normaltalk{\capitalisewords{Determining the Role of Human-Caused Bird Mortality in the Collapse of the North American Avifauna}}{\underline{Scott R Loss}}{In addition to important indirect human-caused threats, like climate change and habitat loss, bird populations are affected by numerous direct sources of human-caused mortality, such as predation by domestic cats, chemical poisoning, and collisions with vehicles and structures (e.g., power lines, buildings, wind turbines, and communication towers). These direct mortality sources collectively cause billions of annual bird fatalities in North America, but the degree to which they contribute to species population declines and the overall collapse of the North American avifauna remains poorly understood. I will review studies indicating that several sources of direct human-caused mortality appear capable of causing biologically important impacts at the population and/or species level. I will also describe factors that currently prevent a comprehensive understanding of how direct mortality sources contribute to population declines and avifaunal collapse. These factors include: data limitations (e.g., lack of spatiotemporal replication of mortality studies and demographic research); biological and ecological complexity (e.g., age-, stage-, and season- specific impacts and complicated avian life histories especially for migratory species); and quantitative challenges (e.g., clarifying the degree to which mortality sources are additive vs. compensatory and linking small-scale mortality effects to population-level processes across a species' entire annual cycle). Despite these challenges, the strong likelihood that direct human-caused mortality sources are contributing to the collapse of the North American avifauna means that further research, management, and policy should be aimed at ameliorating these threats.}

\normaltalk{\capitalisewords{Harmful Algal Blooms in Northern Waters: An Emerging Issue for Alaskan Seabirds?}}{\underline{Caroline R Van Hemert}, Matthew M Smith, Sarah K Schoen, Robert J Dusek, John F Piatt, Mayumi L Arimitsu, R W Litaker, John M Pearce}{Harmful algal blooms (HABs) produce biotoxins that can injure or kill shellfish, fish, wildlife, and humans. They occur naturally but have been increasing in frequency and severity worldwide due to recent climatic changes, including warming ocean temperatures. HABs pose a growing threat to northern marine ecosystems and associated wildlife and human inhabitants. During 2015–2016, an unprecedented die-off of Common Murres (Uria aalge) in Alaska was observed in association with anomalously warm ocean waters in the NE Pacific. Subsequent die-offs of seabirds, including Northern Fulmars (Fulmarus glacialis), kittiwakes, puffins, and shearwaters, occurred in several locations throughout Alaskan waters between 2016 and 2018. Tissues of seabirds, both dead and apparently healthy, and forage taxa were tested for saxitoxin (STX) and domoic acid (DA). STX was common in both die-off and "healthy" birds across different seabird species, locations, seasons, and years. The occurrence and concentration of STX differed by species, and concentrations varied by tissue type. In contrast, DA was found rarely in seabirds and only in trace concentrations. Detectable levels of STX and DA were also observed frequently in forage taxa. Although toxic thresholds have not been established for birds, our research has demonstrated widespread occurrence of STX in the Alaskan marine environment across multiple trophic levels. Additionally, STX has been implicated as a possible factor in at least one recent seabird die-off event. These findings suggest that HABs present a hazard to seabirds and other marine consumers and warrant additional research.}

\normaltalk{\capitalisewords{Estimating Hawaiian Songbird Population Density using Machine-Learning Produced Annotations of Automated Acoustic Recordings}}{\underline{Grady S Weyenberg}, Madori Rumpungworn, Lindsey Howells, Kristina L Paxton, Lisa H Crampton, Patrick J Hart}{Timely and accurate estimates of species density and distribution provide valuable information for scientists and wildlife managers, and when the goal is conservation of threatened populations or taxa, these estimates are an important part of the design and evaluation of management plans. Passive acoustic monitoring systems provide an attractive alternative to field surveys when targeting species that communicate with vocalizations. While these systems greatly reduce the amount of specialized fieldwork required for conducting the survey, by replacing trained observers with an autonomous recording device, they merely displace the problem of identification in the field with similarly time-intensive annotation of audio recordings.  In this study, we use deep neural network machine learning tools to assist in annotation of recordings from the Alakai Plateau, Kauai, Hawaii, USA, for the calls of Puaiohi (M. palmeri), a critically endangered endemic thrush. These annotations are then used to produce density estimates for the target species, incorporating information from the annotated samples such as cue rate (time between vocalizations) and sound power (to estimate distance from bird to microphone). This method can be used to implement a near real-time monitoring program targeting taxa with vocalizations amenable to automatic detection.}

\normaltalk{\capitalisewords{Ecological Correlates to Habitat Use in the Coastal Cactus Wren}}{\underline{Clark S Winchell}, Paul F Doherty, John M Taylor}{Fragmentation of natural habitats is a major threat to wildlife conservation.  In southern California, urbanization and agriculture have reduced coastal sage scrub to 70 and 90\% of its original extent.  Coastal sage scrub, where cactus is prominent, is the principal habitat for a subspecies of cactus wren (Campylorhynchus brunneicapillus  sandiegensis).  We established the degree and predictors of habitat use of coastal sage scrub by wrens.  Preserve lands within San Diego County facing south and below 460 m in elevations were mapped for cactus.  Cactus patches equal to or greater than 0.1 ha in size were assigned plots. We surveyed 6,710 ha, mapping 265 ha of cactus patches which comprised 695 plots. Wrens were detected on 157 plots.  The probability of detection of wrens on a single visit was 0.703 and 0.974 for three visits.  The area of cactus and the distribution of cactus patches across the landscape appear to be important factors influencing habitat use by cactus wrens.  We found that cactus patches aggregated in large patches have a greater than 60 percent chance of being used by wrens.  Plot use was higher on plots where neighboring plots were within 200 m and also supported cactus.  These results suggest that managing stands of cactus that are not fragmented will be important to the recovery of this subspecies of cactus wren.}

\normaltalk{\capitalisewords{AI Seeking Birds at Night: An Innovative Method to Monitor the Position of Nocturnal Species using Microphone Arrays and Robot Audition HARK}}{\underline{Shiho S Matsubayashi}, Fumiyuki Saito, Reiji Suzuki, Kazuhiro Nakadai, Hiroshi G Okuno}{Field biologists can greatly gain benefit from recent advances in bioacoustic technologies, particularly microphone arrays that estimate the position of the free-living birds based on the sound they produce. Sound information derived from microphone arrays have both temporal and spatial contexts, so that it can be related to ecology and habitat of the target of interest. Microphone arrays are also useful to passively monitor undisturbed behaviors, while minimizing the stress to the target species. Despite all these potential benefits, microphone arrays have not been widely used in field surveys mainly due to difficulties in implementing the electronic devices and analyzing the data, both of which could go beyond the expertise of many field ecologists. To overcome these challenges, we have developed a portable recording and analysis system, HARKBird. HARKBird consists of a laptop PC with commercially available low-cost microphone arrays and open-source robot audition HARK (Honda research institute, Audition for Robot with Kyoto university) that automatically recognizes, localizes, and separates sound events. We introduce here field observations of nocturnal owl (Strix uralensis) and ruddy crake (Porzana fusca), both of which are of conservation concern in Japan. Using multiple 8-channel microphone arrays, we captured courtship songs, feeding scenes, and territorial behaviors in time and space. Preliminary analyses show how microphone arrays and robot audition could increase efficiency both in data collection and analysis when monitoring nocturnal birds in the field.}

\normaltalk{\capitalisewords{Extraordinary Plumage Polymorphism in the Variable Antshrike (Thamnophilus Caerulescens, Thamnophilidae) is Consistent with Gloger's Rule}}{\underline{Rafael S Marcondes}, Robb T Brumfield}{Repeated associations between phenotype and environment (ecogeographic rules) suggest that responses to natural selection may be shared and predictable across species. For example, many birds and mammals tend to be darker in wetter climates and paler in drier climates. This pattern, known as Gloger's rule, appears widespread but has been reported more often for temperate than for tropical species. Here, we investigate whether plumage color variation in the Variable Antshrike, a Neotropical passerine whose range encompasses expansive environmental variation, is consistent with Gloger's Rule. Based on spectrophotometric data from 186 specimens, we found that the Variable Antshrike occupies 40\% of the color space delimited by all other 213 species in the Thamnophilidae. Plumage color variation in the Variable Antshrike is significantly correlated with several environmental variables. Precipitation variables in particular are the best predictors of plumage brightness, and browner plumage is correlated with greater seasonality. In two transects in Bolivia where mtDNA allele frequencies have previously been shown to vary clinally, we found that plumage brightness is correlated with transect distance, precipitation and vegetation cover. In contrast, the correlation between environment and the amount of red reflectance (redness of the plumage) was not as pronounced. Overall, patterns of plumage variation in T. caerulescens are consistent with the main Gloger's rule prediction of darker color in wetter areas, but ambiguous in regard to a secondary prediction whereby animals are expected to display reddish color in warmer climates.}

\normaltalk{\capitalisewords{Diversification and Biogeography in an Indo-Pacific Bird Radiation (Pachycephalidae)}}{\underline{Serina S Brady}, Leo Joseph, Robert G Moyle, Michael J Andersen}{The utility of islands as natural laboratories of evolution is exemplified in the patterns of differentiation in widespread, phenotypically variable lineages. Pachycephalidae is one of the most complex avian radiations spanning the vast archipelagos of the Indo-Pacific, making it an ideal group to study the patterns and processes of diversification on islands. Here, we present a robust phylogenetic hypothesis for all five genera within Pachycephalidae, based on thousands of ultraconserved elements (UCEs) that we generated with a target-capture approach and high- throughput sequencing. Our dataset comprises 104 individuals and includes 50 species in the family. We sampled more densely within taxonomically recalcitrant clades, such as the Pachycephala pectoralis complex. We estimated a species tree for all whistlers within a multispecies coalescent framework and explored questions pertaining to the groups' systematics and biogeographical origins at multiple taxonomic levels within this clade (e.g., from the entire family to within species-complexes). This work further refines our understanding of one of the regions' most enigmatic bird lineages and adds to our growing knowledge about the patterns and processes of diversification on island systems.}

\normaltalk{\capitalisewords{GPS-Tracking Reveals Selection for Prairie Ponds by Tree Swallows in Cropland-Dominated Landscapes}}{\underline{Andrew S Elgin}, Christy A Morrissey, Robert G Clark}{Patches of non-crop habitat in agroecosystems may benefit bird and insect populations by serving as shelter, breeding areas, or foraging habitat. Prairie ponds may function as such critical habitat on landscapes dominated by cropland. Aerial insectivorous birds, including Tree Swallows (Tachycineta bicolor), often forage over open water to capture emergent aquatic insect prey. However, it was unclear to what degree prairie ponds might be selected by foraging swallows relative to other habitat types. We hypothesized that prairie ponds act as insect reservoirs, harboring greater insect abundance than cropped areas, and consequently, foraging Tree Swallows would select for ponds, especially relative to cropland. Using miniature GPS tags, we tracked 24 adult breeding female Tree Swallows to evaluate whether ponds were selected as foraging habitat at sites dominated by either tamed grass, croplands, or mixed habitats in southcentral Saskatchewan. We used sweep-net transects to sample flying insects in different habitat types on each site. We confirmed that cropped uplands have the lowest insect abundance, while ponds and grassy field margins had greater abundance. Likewise, ponds harbored a greater diversity of flying insect taxa than uplands or field margins. Consistent with prediction, GPS-tracked Tree Swallows primarily selected for ponds as foraging areas, especially relative to cropland, and increasingly so when travelling farther from nest boxes, because ponds were more profitable foraging locations. This study underscores the importance of protecting and restoring prairie ponds in agricultural landscapes to mitigate aerial insectivore declines.}

\normaltalk{\capitalisewords{A Holistic Approach to Demographic Monitoring of Declining Grassland Songbirds}}{\underline{Jacy S Bernath-Plaisted}, Maureen D Correll, Arvind O Panjabi}{Steep declines in North America's grassland songbirds over the past half-century have spurred increased interest in the ecology and demographics of these species. Although many studies have examined nesting success in grassland songbirds, relatively few have explored other key demographic parameters such as adult and juvenile survival. Fewer still have quantified survival across all three phases within a single population. We monitored adult survival, nesting success, and juvenile survival of Baird's sparrow (Centronyx bairdii) and grasshopper sparrow (Ammodramus savannarum) breeding at sites in western North Dakota (2015-2018) and eastern Montana (2016-2018) with the objective of understanding environmental factors that influence different demographic rates and identifying life-history phases that are most limiting on the breeding grounds. We estimated nesting success, as well as adult and juvenile survival, with logistic exposure models. We also explored how climate and vegetation explained variation in survival. Our results suggested adult survival is relatively high and invariant, and thus unlikely to drive population declines of Baird's sparrow and grasshopper sparrow in the Northern Great Plains. Nesting success and juvenile survival were lower and more variable, however. In particular, juvenile survival in Baird's sparrow averaged well below 40\%, a theoretical threshold generally proposed for population viability in avian species. Juvenile survival related to both fledgling age and vegetation height. These findings suggest that further research on juvenile survival in this species could help identify targets for rangeland management.}

\normaltalk{\capitalisewords{Seasonal Variation in Effects of Competing Predators on Avian Nest Success}}{\underline{Kristen S Ellis}, Randy T Larsen, David N Koons}{The relationship between the rate of predation and prey abundance is an important component of predator-prey dynamics. However, these "functional responses" are less straightforward when multiple predators compete for shared prey. Interactions among competing predators can reduce or enhance effects of predation on prey mortality. Because many avian populations experience high rates of nest predation, understanding the role of specific predators on nest mortality will lead to more informed conservation and management strategies which attempt to increase productivity by removing certain predators or managing habitat to limit their impact. We monitored Snowy Plover Charadrius nivosus nests across 7 years at two study areas in Utah, USA with remote cameras. We used a multinomial extension of the logistic nest exposure model to evaluate effects of specific predators and the influence of nest abundance on nest mortality. We found that probabilities of nest mortality by gulls Larus spp. decreased with increasing nest abundance, whereas nest mortality by foxes and Common Ravens Corvus corax initially increased, indicating that dietary switching may occur when nests become more abundant. Nest mortalities by specific predators ranged between compensatory (e.g., mammalian meso-predators), additive, and over-additive (e.g., avian predators) across the breeding season. The non-independence between nest mortalities suggests that reductions in some predators may not translate to additive increases in overall nest success. Analyses of cause-specific mortality are rarely applied to avian nests, but examination of interacting impacts among competing predators on nest survival may provide insight into specific drivers of avian population dynamics.}

\normaltalk{\capitalisewords{Site Fidelity and Pairing Behavior in a Wintering Population of American Kestrels in North Texas}}{\underline{Kelsey S Biles}, James C Bednarz}{The American Kestrel (Falco sparverius) is North America's most common falcon, however data from the Breeding Bird Survey, raptor migration counts, and occupancy of nest boxes indicate that breeding populations have been steadily declining for decades. Researchers of this species have suggested that this decline may be related to survivorship or changes influencing their wintering ecology, which has received limited study.  We began investigating the wintering ecology of American Kestrels in North Texas in December 2016.  Here, we report our findings related to annual return rates, the degree of site fidelity, and observations of wintering pair behavior.  We captured kestrels using bal-chatri traps and marked birds with anodized bands with a unique two-character code. Throughout the winters of 2016-17 and 2017-18, we visited locations where we previously marked kestrels and recorded their presence and made general observations. Of the 19 marked during our first winter, we documented that at least 11 birds (58\%) returned to our study area the following winter.  The distance from the original capture location to the re-encounter location the next winter averaged 269 m and eight of the marked kestrels were re-sighted less than 239 m from their original capture location, indicating strong winter site fidelity.  Of 50 marked kestrels observed during 2017-18, at least 7 birds exhibited obvious pairing behavior, including copulations and courtship feedings. These observations may suggest that some American Kestrels participate in temporary "winter romances" or that they may re-unite both on their breeding and wintering territories.}

\normaltalk{\capitalisewords{Captivity and the Gut Microbiota of 'Akikiki: Implications for Conservation of a Critically Endangered Hawaiian Honeycreeper}}{\underline{Maria S Costantini}, Cali Crampton, Bryce Masuda, Floyd Reed}{The insectivorous Hawaiian honeycreeper, 'Akikiki (Oreomystis bairdi), a critically endangered forest bird on the island of Kaua'i, is subjected to a suite of introduced threats and currently only occupies a small (54km2) region of its previous range. Captive breeding attempts are currently underway, and agencies are contemplating the potential establishment of translocated "insurance" populations. This project explores an overlooked, but potentially critical factor to the success of translocation attempts by analyzing the effect of captivity and novel environments on host microbiota. Symbiotic microbiomes provide key health benefits to the host through such processes as dietary supplementation, boosts to host immune system, increased pathogen resistance, or tolerance to environmental perturbations. Because gut microbiota are heavily influenced by diet and environmental conditions, differences in the microbiota of captive-raised individuals may hinder their ability to adjust upon reintroduction. This project seeks to determine if captivity 1) alters natural gut microbiomes, and 2) negatively impacts microbe-related fitness benefits for 'Akikiki.  Fecal samples were collected from wild and captively reared 'Akikiki since spring of 2017, by mist-netting and collecting directly from the bird while in hand. These samples will be analyzed through Illumina MiSeq metabarcoding techniques of the 16S rRNA gene to determine bacterial community composition and diversity. The results from this analysis will better inform management strategies of this endangered bird, such as by developing an inoculation protocol for captive individuals, as well as will likely be informative for understanding similar species within and outside of Hawai'i.}

\normaltalk{\capitalisewords{Spatial Models to Facilitate Broad-Scale Conservation of Breeding Habitat for Secretive Marsh Birds}}{\underline{Bryan S Stevens}, Courtney J Conway}{Spatial models are commonly used to differentiate optimal habitat from sub-optimal habitat, and thus play a pivotal role in conservation planning and management for many avian species. Secretive marsh birds are a widely distributed group of birds that include multiple threatened and endangered species, many of which have experienced range contractions over broad spatial and temporal scales. Identification of optimal breeding habitat is thus imperative for effective conservation of marsh birds. We developed spatial models to predict the location of optimal breeding habitat for 14 species of marsh birds within the continental United States (U.S.). We used recently developed multi-scale occupancy models to build spatial habitat models for each of 14 marsh bird species across their U.S. breeding ranges. We modeled habitat as a function of wetland, land cover, and human disturbance attributes measured over multiple spatial scales, where covariates included were scale optimized and selected for optimal prediction. We used raster regression to translate statistical models into maps of predicted habitat suitability for each species, using covariate data measured at a 30-m resolution over the breeding range for each species. Our models enable identification of breeding habitat for each individual species, as well as areas that provide habitat for multiple species that could serve as hotspots for strategic conservation efforts. This work thus provides a vital first step towards conserving the most valuable remaining habitat for marsh birds across the continental U.S., and provides baseline predictions that can be updated and adapted locally as additional data are collected.}

\normaltalk{\capitalisewords{Global Trends in Avian Partial Migration: a Meta-Analysis}}{\underline{Erika S Zavaleta}}{Rapid environmental changes disproportionately affect migratory animals worldwide. Partial migration, where some individuals migrate and others remain resident, is widespread and relatively well-studied in birds. Partial migration rates depend on conditions that shape the relative advantages of migratory and resident strategies. Environmental changes are both increasing migration hazards and reducing barriers to overwintering, through trends including milder winters and increasing food sources. I conducted a global meta-analysis to address whether more partial migrants are remaining resident over time since the mid-20th century. A systematic Web of Science search yielded 1,764 possibly relevant published articles, from which I found 31 cases of long-term trend data (range 19-37 years) in proportions of avian populations that pursued resident vs. migratory strategies. The unweighted mean decline in migrating fraction was 1.09\% year-1 (95\% CI [0.15, 2.03]). Sixteen additional studies reported long-term population trends of winter residents, without data on breeding population sizes available to compute proportions. The mean observed increase in overwintering populations of these partial migrants was 13.8\% year-1 (95\% CI [7.4,20.2]). Authors' attribution of causes included changes such as climate, increased food availability, reduced ice cover on water bodies, and new water bodies created by impoundment. Nearly all studies took place in Europe, with the remainder in the US. These results and the paucity of published findings pinpoint the need for more comprehensive study. Global environmental changes appear to be reducing migratory bird numbers through declines not only in populations, but also in partial migration.}

\normaltalk{\capitalisewords{Molt-Migration in the Mexican Monsoon: 90 Years of Specimens Reveal a Phenological Mismatch Caused by Anthropogenic Climate Change}}{\underline{Ryan S Terrill}, Whitney L Tsai, James M Maley, Sievert Rohwer, Graham Pimm, John E McCormack}{Understanding the full annual cycle of migratory birds is essential for their conservation, especially events directly tied to investment in adult survival such as feather replacement. Birds that undergo long-distance migrations have annual cycles that are tightly linked to the predictable availability of resources along their migratory path. Studies have found that global climate change has begun to cause mismatches in the migratory timing of birds and availability of food along their paths. The timing of arrival on breeding grounds has received significant attention in this respect, but little attention has been paid to the potential effects of these phenological mismatches where birds undergo their annual molt. Many species of migratory birds take advantage of heavy late summer rains and the associated increase in primary productivity in the Mexican Monsoon to undergo their molt before continuing southward for the winter. This molt is energetically expensive, and many of these species travel hundreds of miles to molt their feathers during this resource-rich time. This monsoonal region receives $>$90\% of its annual rainfall during the monsoons, and is an arid desert during the rest of the year. Over the past 90 years, anthropogenic climate change has induced a shift in the onset of the monsoon so that the onset and peak of the monsoon is a month later than in pre-industrial times. We examined the onset and duration of molt in 340 specimens collected over 90 years from 15 species of migratory birds that undergo a stopover molt in the monsoon and found no support for a phenological shift in their molt timing. Experimental studies have shown that molting with insufficient food causes mortality instead of suspension of molt, and these birds may arrive on their molting ground to find a dry and foodless desert. This finding is concerning for the conservation of these species, because these birds may become stuck starting their molt before the food they rely on to undergo molt is available.}

\normaltalk{\capitalisewords{Testing the Hotspot Hypothesis: Lesser Prairie-Chicken Lek Formation and Female Space Use}}{\underline{Carly S Aulicky}, David Haukos}{Lek breeding species are characterized by grouped male displays where few males successfully reproduce with visiting females and contribute only gametes to their offspring. Due to the differences in paternal care, lek breeding birds, sexes differ in habitat use. For the lesser prairie-chicken (Tympanuchus pallidicinctus), a grassland obligate, males form leks in short grass that increases their visibility whereas females prioritize vegetation structure that provides cover for nests and broods. Because short grass is not limiting, how do males select where to form leks? Current lesser prairie-chicken conservation strategies assume lek formation occurs according to the hotspot hypothesis, where leks form in areas where female home ranges touch or overlap in high quality breeding habitat. However, the hotspot hypothesis has not been tested with empirical data to validate this assumption. We test the hotspot hypothesis using 165 estimated female lesser prairie-chicken home ranges and 53 lek locations across three Kansas field sites. We estimated hotspot locations at intersecting hen home ranges and created a point density raster to determine the probability of overlapping female space use at the 95\% isopleth. We plotted lek locations onto the point density surface and weighted leks by average annual attendance to determine if predicted hotspots explained lek locations more than expected at random. We examined shifts in female space use from GPS telemetry point locations with an optimized hotspot analysis by study year. Our results indicate that leks form in locations with high female home concentration consistent with the hotspot hypothesis.}

\normaltalk{\capitalisewords{Migratory Connectivity of a Declining Boreal Shorebird, the Lesser Yellowlegs}}{\underline{Katie S Christie}, Laura McDuffie, James A Johnson, Audrey Taylor}{The Lesser Yellowlegs is a neotropical migrant shorebird that nests in boreal wetlands across North America. This species has declined 5.3\% to 9.2\% per year in Alaska since 2003 and 5.3\% per year across North America since 1966. Causes for declines are not well understood, but unsustainable hunting at stopover sites in the Caribbean and wintering grounds in northern South America is thought to be the most immediate threat. The annual harvest of Lesser Yellowlegs on Barbados ranges from 5700 to 19900 birds and harvest on Guadeloupe likely exceeds 8000 birds. Until now, the relative vulnerability of different breeding populations to over-harvest and other threats during the non-breeding season has been unclear. With the recent development of lightweight tracking technology, we can now investigate questions related to the migratory ecology of this species. In 2019, we deployed 27 PinPoint Argos-GPS transmitters on adult Lesser Yellowlegs in Anchorage, Alaska, Yellowknife, Northwest Territories, and James Bay, Ontario. The majority of birds tagged in Anchorage and Yellowknife used the Central Flyway to migrate to nonbreeding locations across Central and South America and the Caribbean, with the exception of one bird that used the Atlantic Flyway. All birds nesting near James Bay used the Atlantic Flyway to migrate to locations in northern South America including Venezuela, Suriname, and the Dominican Republic. With additional tagging efforts across the breeding range in 2019, we expect to learn a great deal about threats facing Lesser Yellowlegs throughout their life cycle, facilitating targeted conservation actions for this species.}

\normaltalk{\capitalisewords{Managing Wild Birds for Strawberry Production, Pest Control, and Food Safety Outcomes in the California Central Coast}}{Elissa M Olimpi, Karina Garcia, David Gonthier, \underline{Daniel S Karp}}{Birds are increasingly viewed as food-safety hazards and pests in farmlands. Yet birds also benefit growers by consuming pests. Despite limited research into bird-induced crop damage and their inferred role as disease vectors, berry growers are often pressured to prevent birds from accessing their farms. We quantified the net effect of birds on strawberry production across 20 organic strawberry farms in California's Central Coast. Specifically, we used high-throughput sequencing of bird feces to build diet profiles, surveyed strawberry farms for berry damage and bird droppings, and implemented an exclosure experiment to quantify the net effect of birds on strawberry production. Diet analyses indicated that birds play complex roles on strawberry farms, consuming insect pests, predatory (beneficial) arthropods, and strawberries. Correspondingly, we found that the benefits and costs imposed by birds balanced such that birds had an overall neutral impact on strawberry yield. Birds were most likely to consume insect pests and reduce berry damage in simple landscapes, far from field edges. Near crop edges, especially in complex landscapes surrounded by more natural habitat, birds consumed more predatory arthropods and thus indirectly increased berry damage. Direct bird damage to strawberries declined on farms with low crop diversity and more natural habitat in the surrounding landscapes. In contrast, both bird damage and fecal contamination increased on farms with higher densities of fencing and wires, where birds often perch. Our findings suggest that the overall neutral impact of birds on strawberry production masks complex, shifting impacts that depend on farming context.}

\normaltalk{\capitalisewords{Habitat Saturation Drives the Frequency of a Rare Behavior in the Females of a Social Bird}}{\underline{Sahas S Barve}, Walter D Koenig, Joseph Haydock, Eric L Walters}{Joint nesting by females and cooperative polyandry—cooperatively breeding groups with a male-biased breeder sex ratio—are little-understood, rare breeding systems. We tested alternative hypotheses of factors potentially driving these phenomena in a population of joint-nesting acorn woodpeckers (Melanerpes formicivorus). During periods of high population density and, thus low independent breeding opportunities, acorn woodpecker females formed joint-nesting coalitions with close kin. Coalitions were typically associated with groups with a male bias. We found strong evidence for both inter- and intra-sexual conflict, as joint nesting conferred a fitness benefit to some males, a significant fitness cost to females, and no gain in per capita reproductive output for either sex. Such conflict, particularly the cost to females, may be an important reason why joint nesting is rare among cooperatively breeding taxa.}

\normaltalk{\capitalisewords{What About Those Rare Species? Study Design Implications for Multi-Species Avian Monitoring Programs}}{\underline{Jamie S Sanderlin}, Joseph L Ganey, William M Block, Michael K Schwartz, Kevin S McKelvey}{Multi-species avian monitoring programs are often designed well for capturing trends with common species, but tend to lack sufficient information to make inference on trends with rare species. These rare species, however, are frequently the species of most concern with large-scale monitoring programs. Bayesian hierarchical multi-species occupancy modeling is used to obtain parameter estimates with rare species using assumptions that individual species are governed by community-level parameters, but inference is limited with few detections. This is challenging since most bird monitoring programs have multiple objectives of determining trends of these rare species from changes in local extinction and colonization, in addition to the overall community trends with changes in species richness or diversity. Our previous work with optimal designs of multi-species community occupancy models indicated rare species occupancy was more influenced by spatial than temporal replicates, while species richness was more driven by temporal replicates. We use simulated data based on case studies of large-scale monitoring programs within Arizona National Forests to further explore optimal designs with consideration of rare species and limited budgets.}

\normaltalk{\capitalisewords{Nested Sampling of Forest Restoration Treatments Reveals Different Species Occupancy and Richness Patterns Depending on Spatial Scale}}{\underline{Quresh S Latif}, David C Pavlacky, Rob A Sparks, Richard L Truex}{Ornithologists increasingly monitor occupancy as a cost-effective approach to estimate population patterns and trends, and multi-species monitoring can additionally inform species richness. Practitioners typically focus on incorporating temporal replicates into their surveys to account for detectability, but many fail to appreciate how sampling scale affects inference. We leveraged nested bird data from an extension of the Integrated Monitoring in Bird Conservation Regions program to demonstrate how occupancy and richness patterns estimated at different scales reflect different processes. Sampling points were nested within 4×4 1kmtextasciicircum 2 grids (max 16 points each), and the Collaborative Forest Landscape Restoration Program (U.S. Forest Service) funded surveys and analysis. As with many studies that examine bird relationships with forest disturbance or management treatments, we found statistical support for both positive and negative species occupancy relationships with restoration treatments among survey points. We only found support for positive occupancy relationships with percent landscape treated for grids, however. Consequently, although species richness differed by $<$ 1 species between treated and untreated points, mean richness increased by ~5 species from grids in untreated landscapes to landscapes with ~60\% area treated. These results provide evidence for the often touted but less commonly documented importance of landscape heterogeneity for supporting biodiversity in lower elevation conifer forests of western North America. Additionally, we demonstrate how occupancy relationships measured at different scales can reflect different processes – coarse-scale relationships quantify species distribution whereas fine-scale relationships reflect habitat use or local abundance. We suggest ornithologists strongly consider sampling scale when planning occupancy-based surveys.}

\normaltalk{\capitalisewords{A General Framework to Integrate Repeated Count and Mark-Resight Data}}{\underline{Clark S Rushing}, Nathan J Hostetter, Ruth Boettcher, James Lyons}{Spatially and temporally repeated count data are increasingly used to estimate abundance and detection across a variety of taxa. Achieving adequate replication to estimate parameters can be challenging as count data provide relatively little information to separate abundance and detection parameters relative to mark-recapture data, as highlighted by recent focus on the identifiability and reliability N-mixture models. Analysis of count data, however, is nearly ubiquitous in ecology and developing logistically feasible sampling protocols and integrated analytical approaches to improve inferences of N-mixture models is needed. We developed a general analytical framework to integrate spatially and temporally repeated count and mark-resight data. This framework allows for spatial variation in abundance, imperfect detection, unknown marking status, incomplete band reads, and an unknown number of marked individuals that may vary among sites. Simulations are used to evaluate model performance across a variety of sampling scenarios, including different proportions of marked individuals, varying levels of detection probability, and varying rates of partial band reads. We also evaluate the impacts of a misspecified model that assumes marks are perfectly identified conditional on detecting a marked individual. We demonstrate an application of the field protocols and analytical approaches using a case study of American Oystercatcher (Haematopus palliatus) surveys across Virginia, North Carolina, and Georgia, USA during 2015.}

\normaltalk{\capitalisewords{Breeding Ecology of Kittlitz's Murrelets in the Aleutian Islands, Alaska}}{\underline{Robb S Kaler}, Leah Kenney, Jeff Williams, John Piatt}{The Kittlitz's murrelet (Brachyramphus brevirostris) is one of the rarest breeding seabirds in the North Pacific. During 2008-2011, we monitored 75 nests at Agattu Island, located in the western Aleutians, Alaska. During 2012-2016, we monitoring 43 nests at Adak Island, located in the central Aleutians, Alaska. Time-lapse cameras were placed at nests to monitor adult attendance, chick diets, and nest survival. At Agattu, 84\% of nests failed, with the leading cause of mortality likely attributed to starvation/exposure. Agattu chicks fledged at approximately 50\% of adult body mass (range: 104-139 grams, n=12), considerably lower than values reported from other locations. Over the 4-year study at Agattu we documented nearly 2,400 chick feedings and found chick diets were composed of sand lance (29\%), Hexagrammids (kelp greenling, 25\%), and Pacific cod/juvenile rock fish (24\%). At Adak Island, with a much reduced monitoring effort (cameras only), 67\% of nest failed, with the leading cause of mortality due to predation of eggs and chicks. The continued study of murrelets breeding in the Aleutian Islands will provide further insight into the reproductive biology of this rare and elusive seabird and provide a unique opportunity to elucidate its life history in non-glaciated habitats, especially in light of warming ocean trends.}

\normaltalk{\capitalisewords{Loss of a Keystone Informant Disrupts Information Cascades among Ant-Following Birds in a Tropical Forest Fragment}}{\underline{Henry S Pollock}, Ari E Martinez, J P Kelley, Janeene M Touchton, Corey E Tarwater}{The loss of keystone species can have cascading impacts on interspecific interactions and community structure. Many mixed-species bird flocks may be particularly susceptible to such losses, particularly the loss of keystone informants – species that provide valuable social information to heterospecifics about the presence of predators and food resources. Here, we investigate the impacts of forest fragmentation and the subsequent loss of a keystone informant, the Ocellated Antbird (Phaenostictus mcleannani), on a prominent type of mixed-species animal group of the Neotropics – foraging flocks of army-ant following birds. We compared an island fragment site (antbird extirpated) to a contiguous mainland site (antbird present) to explore impacts of keystone informant loss on information transfer and flock formation. We used a factorial playback experiment to simulate flocks of ant-following birds vocalizing at ant swarms and quantified recruitment of eavesdropping birds to flocks of different sizes, species richness, and dependency guild. Recruitment to playbacks at the mainland site was based on a hierarchical information cascade, with information flowing from more obligate to less obligate guilds and characterized by complex interactions between flock size, richness and species dependencies. In contrast, we found disruption of the information cascade at the island fragment site, with information flow concentrated among obligate and facultative guilds and virtually no recruitment from less dependent guilds. Our results suggest that loss of a keystone informant can fundamentally alter information flow within a community, with important implications for eavesdropping behavior, resource use and ultimately, the formation and structure of mixed-species animal groups.}

\normaltalk{\capitalisewords{An Incomplete Overview of Determining Avian Species Limits}}{\underline{Kevin S Winker}}{A series of topics will be briefly addressed: Single-locus criteria are unlikely to be convincing. Genetic independence is not a species limits requirement, but the degree of independence should be when it is nonzero. Time-based species (calling something a species simply because it has been separate from its closest relative for a long time) are unsatisfactory. We need more disentanglement of the signals of neutral processes versus selection,  and prioritize the latter because it's the main driver of speciation. Assortative mating is not an adequate criterion. Comparative methods remain the gold standard but can be controversial. The principles of phylogenetic systematics have to be able to break down at the species level: reticulations happen, and not necessarily between sisters. Peripatric speciation can create new species divorced from species relationships among source(s). When are divergences just adaptation and not really speciation? How much independence should genomes have? We'll continue to struggle with many of these issues for a long time.}

\normaltalk{\capitalisewords{Developing a Unified Framework for Avian Monitoring in Pacific Northwest Estuaries}}{\underline{Trina S Bayard}, Gary Slater, Kyle Spragens, Marla Koberstein, Amanda Summers}{The loss and degradation of estuary habitat in Puget Sound, WA, has been severe, with over 80\% of Puget Sound river delta habitat lost. Although bird populations have likely suffered substantial impacts, investments in tidal habitat restoration are largely driven by recovery needs for federally-listed Chinook salmon (Oncorhynchus tshawytscha), with unknown benefits to birds. The Marine Birds Work Group associated with the Puget Sound Ecosystem Monitoring Program (PSEMP) has sought to understand: 1) the current state of knowledge involving avian response to tidal restoration and 2) whether coastal stakeholders would benefit from a unified monitoring framework that addresses information needs at multiple scales. Having completed an inventory and synthesis of existing monitoring efforts, we are currently engaged in extensive outreach to coastal land owners and managers, including tribes, state and federal agencies, and conservation NGOs. This outreach is aimed at soliciting avian information needs, building community, and generating discussion about a potential unified approach. Pre-workshop engagement with estuary stakeholders involved a multi-pronged approach to identify appropriate participants and foster their participation. Two regional workshops were held to solicit input on information and monitoring needs and build and connect the community of people working on bird conservation and management in estuaries. Participant input will be synthesized into recommendations for key elements of a unified approach. If successful, this framework will address the priority information needs of land owners and managers identified in our workshops and build the foundation for stronger coordination across the region.}

\normaltalk{\capitalisewords{Genomics for the Management of 'Alal\={a} (Corvus Hawaiiensis) in a Conservation Breeding Program}}{\underline{Jesse S Leavitt}, Geneviève Blanchet, Erin Datlof, Renee Bellinger, Scott Geib, Bryce Masuda, Jolene Sutton}{Genetic diversity is important to consider for the conservation of small populations that have experienced a bottleneck. The critically endangered Hawaiian Crow, 'Alal\={a} (Corvus hawaiiensis) endured a century-long bottleneck, whereby the entire species was reduced to less than 20 individuals in the early 1990s and was declared extinct in the wild in 2002. Prior to extinction in the wild, all remaining 'Alal\={a} populations were brought into captivity to aid in their recovery. At present, ~125 individuals are alive, all of which descend from just nine total genetic founders. Due to their small population size, inbreeding has been inevitable and has resulted in inbreeding depression. To assist with recovery efforts and help inform conservation breeding management, this study aims to characterize genomic diversity in the current breeding population, and to compare this diversity to two other snap-shots in time: the onset of the conservation-breeding program in the early 1990s, and museum samples collected circa 1900. By evaluating allelic diversity, heterozygosity, and relatedness, we aim to more fully understand the genomic consequences of inbreeding on this species, and to begin developing tools for using molecular relatedness to help guide the choice of breeding pairs.}

\normaltalk{\capitalisewords{Assessing Assortative Mating in Hybrid Zones}}{\underline{Elizabeth SC Scordato}}{Hybrid zones are geographic regions where reproductive barriers that have formed in allopatry are challenged by potential gene flow in sympatry. They are thus natural laboratories where it is possible to test the relative contributions of different selective pressures to the maintenance or erosion of species boundaries. Recent application of next-generation sequencing to studies of hybrid zones has made it possible to identify the genomic architecture of reproductive barriers and to link patterns of hybridization to phenotypic and ecological variables. Furthermore, studies of assortative mating in hybrid zones enable estimation of the strength of various prezygotic barriers to reproduction. However, variation in the availability of mates across hybrid zones is an important variable affecting observed patterns of reproductive isolation and estimation of the strength of reproductive barriers. I discuss methods for leveraging genomic and phenotypic data and controlling for mate availability to quantify assortative mating in hybrid zones.}

\normaltalk{\capitalisewords{Breeding Ground Habitat Loss is the Primary Driver of Population Declines for a Long-Distance Migratory Songbird}}{\underline{Michael T Hallworth}, Peter P Marra, Erin Bayne, Oliver P Love, Emily McKinnon, Junior A Tremblay, Steve Van Wilgenburg, Bruno Drolet, Jacques Ibarzabal}{Many migratory populations are declining worldwide and for most of these species the proximate causes of their decline remain unknown. For many long-distance Neotropical migratory songbirds, it is assumed that habitat loss, on breeding or non-breeding grounds is the primary driver of population declines. In this study we combined data collected from tracking technology to quantify migratory connectivity, citizen science data for population trends and remote sensing for forest loss to quantify how forest change throughout the annual cycle contributes to the on-going population decline of a long-distance migratory songbird, the Connecticut warbler (Oporornis agilis). Migratory connectivity (MC) was stronger between breeding and autumn migration routes (MC = 0.24 ± 0.23) than between breeding and non-breeding locations (MC = -0.19 ± 0.14). Different Connecticut warbler populations tend to have population specific fall migration routes but winter in similar areas within South America. We found that cumulative forest loss within 50km of breeding locations contributed more to population declines (observed decline: -2.35\% yr-1) than forest loss on migratory stop-over regions or their wintering locations in South America. Forest loss during the breeding season appears to be the driver of observed population declines for the Connecticut warbler. Further research is needed to help inform forestry practices that ameliorate the need for forest products and abate on-going population declines.}

\normaltalk{\capitalisewords{Individual Distinctiveness in Vocalizations of a Suboscine Songbird}}{\underline{Josephine T Tagestad}, Daniel J Albrecht-Mallinger, Corey E Tarwater}{Assessment of organisms' vocal traits can lead to insights regarding behavior, age, health, and identity. Studies of bird song and its functions focus almost exclusively on the learned songs of oscine birds, while the innate songs of suboscine birds have received less attention. Suboscine songbirds are assumed to be less vocally complex than their oscine counterparts, resulting in lack of individual distinctiveness in their vocalizations. Here we test whether vocalizations of the suboscine bird of Ceratopipra mentalis, the red-capped manakin, are individually distinctive. We examined recordings of 20 males at 10 leks across the Isthmus of Panama, extracting cadence and frequency data from whistle-peeps, a contact call males give throughout the breeding season. We used a flexible discriminant function analysis (FDFA) in Program R to test for individual identity coded in vocalizations. Based on this analysis, individual variation is present, with the FDFA assigning identity to the calls with 87\% accuracy. Furthermore, individual identity explains more variation than local dialect. As C. mentalis is an important seed disperser and is predicted to decline with continued changes in climate, the use of vocal recordings to estimate abundance in a rapid, non-invasive way would aid in studying them in a changing world. More generally, we argue that further studies of suboscines are needed to test long-held assumptions about individual distinctiveness.}

\normaltalk{\capitalisewords{Assessing Optimality of Migratory Landbirds Crossing the Gulf of Mexico using Individual-Based Modeling of Telemetry Data}}{\underline{Rachel T Bolus}, Robert H Diehl, Jill L Deppe, Michael P Ward, Frank R Moore, Antonio Celis-Murillo, Theodore J Zenzal Jr, Jaclyn A Smolinsky}{In unfavorable conditions, small landbirds flying across the Gulf of Mexico can deplete their energetic stores to the point of exhaustion or starvation. Therefore, behaviors that decrease the energetic costs of flight should be adaptive. These behaviors include decisions of when to fly, where in the air column to fly, and at what airspeed to fly. An individual's optimal decisions depend on current weather, physiology, morphology, and motivation. From 2009 until 2014, we radio-tagged Swainson's Thrushes, Wood Thrushes, Gray Catbirds, Indigo Buntings, and Red-eyed Vireos on the northern coast of the Gulf of Mexico, and using a network of automated radio telemetry towers recorded the departure from Ft. Morgan, Alabama, USA and subsequent arrival of 79 individuals at the Yucatan Peninsula, Mexico. We used individual-based modeling to assess the costs and benefits of differing altitude selection strategies and to estimate the potential range of actual airspeeds of each individual (constrained by departure and arrival data. We determined whether these airspeed ranges were consistent with optimal airspeeds modeled from measured wing areas, fat loads, and modeled altitudes. Lastly, we compared strategies among species, by energetic stores, and by sex, to explore the role of context in these decisions.}

\normaltalk{\capitalisewords{Admixture Mapping the Genetic Basis of Species Differences in Avian Hybrid Zones}}{\underline{Alan T Brelsford}, Emmanuel Nwankwo, Brian Myers, Christopher J Clark, Alexander Kirschel}{Linking genotype, phenotype, and fitness in natural populations is a major goal of evolutionary biology, but one that is accomplished infrequently. Tension zones, a class of hybrid zone, present an unusual opportunity to accomplish this goal by combining admixture mapping and cline theory. Admixture between two species in a hybrid zone increases linkage disequilibrium and variation in quantitative traits above the background values found in the parent species, both of which increase the power of genome-wide association analyses. Having identified genomic regions associated with traits that differ between the parent species, we can then compare the width and position of clines in these loci with clines in the rest of the genome to estimate the strength of selection against a heterospecific allele at these loci. Here, we present results of admixture mapping analyses of carotenoid-based plumage traits in a hybrid zone between Red-fronted and Yellow-fronted tinkerbirds, and of structural color and tail feather morphology in a hybrid zone between Rufous and Allen's hummingbirds.}

\normaltalk{\capitalisewords{Phylogenomics of the Parrots of the World}}{\underline{Brian T Smith}, Robb Brumfield, Mateus Ferreira, William Mauck, Jon Merwin, Rob Moyle , Timothy Wright, Leo Joseph}{Parrots (Order: Psittaciformes) are a diverse pan-tropical clade known for their colorful plumage, intelligence, and challenging systematics. Recent efforts, largely fueled by small fragments of DNA sequence, have provided resolution to higher-level relationships across the clade. However, the compound effects of limited taxon sampling and few genetic markers have yielded a tree with many poorly resolved and unclear relationships. Here, we present a strongly supported phylogenetic hypothesis for parrots estimated with genome-wide markers and nearly complete species-level sampling. We give a detailed overview of phylogenetic relationships throughout the clade, highlighting cases of non-monophyletic genera and the placement of previously unsampled taxa. We further discuss congruence and conflict among gene trees with respect to newly resolved clades. In sum, our phylogenetic hypothesis will serve as a framework for a taxonomic revision for the entire clade, and represents a powerful tool for comparative inquiries ranging from conservation of this highly endangered group to the evolution of intelligence.}

\normaltalk{\capitalisewords{Population Dynamics of a Long-Distance Migratory Passerine at the Edge of Its Range}}{\underline{Michael T Murphy}, Lucas J Redmond, Amy C Dolan, Nathan W Cooper, Christopher M Chutter, Sarah Cancellieri}{The relative importance of top-down (i.e., predation) and bottom-up (i.e., resource) control on the dynamics of migratory bird populations is poorly known. Bottom-up control may be more likely in resource poor environments such as deserts, but given that nest failure in most species is due mainly to predation, top-down control may predominate. We document a rapid decline of an Eastern Kingbird (Tyrannus tyrannus) population breeding in the Great Basin Desert of eastern Oregon and evaluate whether it was driven by limited food resources (food ~water availability), nest predation, and either first-year (SJ), or adult survival (SA). Most nests failed (~68\%) due mainly to nest predation ($>$90\% of failures); nestling rarely starved. Water-year precipitation (October-April), and both breeding season precipitation and river flow were all independent of year and none could account for annual variation in nest success (NS) or number of young fledged per successful nest (FSN). Neither SJ nor SA varied with year, any measure of water resources (i.e., food), or reproductive success. Annual growth rate (ln[Nt+1/Nt]) was independent of year, all measures of water availability, SJ and SA, but varied directly with NS; FSN also tended to be higher in years of high NS. Persistent high rates of nest predation thus appeared to drive the population decline, and replacement nests contributed little to population growth rate. Simulations revealed that site-specific vital rates produced an even lower estimate of growth rate (λ = 0.812) than observed (λ = 0.914), indicating that the decline would have been greater without immigration.}

\normaltalk{\capitalisewords{Movement Patterns of Arctic-Breeding Shorebirds During Post-Breeding and Southbound Migration}}{\underline{Sarah T Saalfeld}, Richard B Lanctot, Stephen Brown, Jean-François Lamarre, Rebecca McGuire, Daniel Ruthrauff, Kyle Elliot, Christopher Latty}{Shorebirds use coastal regions of the Arctic during post-breeding to acquire resources to migrate. Limited information is available on the inter-connectedness of breeding and stopover sites, as well as residency time and movements among stopover sites. Such information is needed to understand effects of climate change and allow stakeholders to make informed management decisions. We deployed tracking devices on four shorebird species breeding at four sites in Alaska (Utqiaġvik, Colville River, Prudhoe Bay, and Canning River) and one site in Canada (Cambridge Bay, Nunavut). In 2017, we deployed 1.2 g GPS archival tags on 57 Dunlin and 68 Semipalmated Sandpipers; 6 Dunlin and 1 Semipalmated Sandpiper tags were recovered a year later from three of the sites. In 2018, we deployed 4–5 g GPS satellite tags on 15 Pectoral Sandpipers and 13 American Golden-Plovers. All tracking devices were programmed to collect one GPS location every one to two days during southbound migration to wintering areas.  We will describe general patterns and new information on post-breeding movements and southbound migration.}

\normaltalk{\capitalisewords{Hybrid Zones and the Development of Reproductive Isolation in Flooded Versus Terra Firme Forest Birds of the Amazon}}{\underline{Jason T Weir}}{Despite having the highest species richness of any region of the planet, we still have limited understanding of how reproductive isolation evolved in the Amazon basin. Two key forest types of the Amazon – unflooded terra firme and seasonally flooded varzea – each possess their own distinctive endemic avifaunas. Here I use genome-wide datasets and analysis of geographic clines, hybrid index and heterozygosity to compare a series of Amazonian hybrid zones in flooded versus terra firme forest birds to better understand the development of reproductive isolation and species boundaries in each forest type. I ask whether better dispersal abilities associated with varzea birds retards the accumulation of reproductive isolation?}

\normaltalk{\capitalisewords{Fighting over Food Unites the Birds of North America in a Continental Dominance Hierarchy}}{\underline{Eliot T Miller}, David N Bonter, Charles Eldermire, Benjamin G Freeman, Emma I Greig, Luke J Harmon, Curtis Lisle, Wesley M Hochachka}{Members of different species often engage in aggressive contests over resources. This series of aggressive contests between species may result in an interspecific dominance hierarchy. Such hierarchies are of interest because they could be used to address a variety of research questions, for example, do similarly ranked species tend to avoid each other in time or space, and what will happen when such species come into contact as climates change? Here, we propose a method for creating a continental-scale hierarchy, and we make initial analyses based on this hierarchy. Leveraging the existing network of citizen scientists from Project FeederWatch, we collected the data with which to create a continent-spanning interspecific dominance hierarchy that included species that do not currently have overlapping geographic distributions. We quantified the extent of intransitivities (rock-paper-scissors relationships) in the hierarchy, as intransitivities can promote local species' coexistence. Overall, the hierarchy was nearly linear, and largely predicted by body mass, although there were clade-specific deviations from the average mass–dominance relationship. Warblers and orioles, for instance, were more dominant than expected based on their body mass, while buntings, grosbeaks, and doves were less dominant than expected. Intransitive relationships were rare. Few interactions were reported between close relatives and ecological competitors like Mountain and Black-capped Chickadees, as such species often have only marginally overlapping geographic distributions, restricting opportunity for observation. Yet, these species' ranks-—emergent properties of the network—-were often in agreement with targeted studies of dominance relationships between them.}

\normaltalk{\capitalisewords{Climate, Habitat, and Geographic Range Overlap Drive Plumage Evolution and Mimicry in Woodpeckers}}{\underline{Eliot T Miller}, Gavin M Leighton, Benjamin G Freeman, Alexander C Lees, Russell A Ligon}{Organismal appearances are shaped by selection from both abiotic and biotic drivers. For example, more pigmented populations are found in more humid areas, a pattern known as Gloger's rule. Social selection may favour appearances that stand out, whilst predation pressure can induce appearances that blend in. Species may also converge on nearly identical colours and patterns in sympatry, often to avoid predation by mimicking noxious species. Here we leverage a massive global citizen-science database to determine how these factors act in concert to shape the plumage of the world's 230 species of woodpeckers. We find that habitat and climate profoundly shape woodpecker plumage, including convincing support for the generality of Gloger's rule. Yet, many species exhibit remarkable convergence explained neither by these factors nor by shared ancestry. Instead, this convergence is associated with geographic overlap between species, suggesting occasional strong selection for interspecific mimicry.}

\normaltalk{\capitalisewords{Urban Habitat Use by the an Endangered Hawaiian Waterbird on O'ahu, Hawaii}}{\underline{Martha T Kawasaki}, Patrick J Hart, Eben H Paxton}{Loss of suitable wetland habitats due to anthropogenic development is a leading cause for decline of the Hawaiian stilt or Ae'o, as well as the introduction of non-native predators and invasive wetland plants. While other Hawaiian waterbirds are largely restricted to wetlands, Hawaiian stilts appear to be adapting to the urban environment, using heavily modified upland habitats. In our study, we fitted four Hawaiian stilts with GPS satellite tags to document their use of developed areas, undeveloped fields, sports fields and wetland habitats over a 6 month period. We found a high use of non-wetland habitat, with significant differences in habitat occupancy among the individual stilts and across different times of day. Wetlands were the dominant habitat occupied from morning to early afternoon, but non-wetland habitats were occupied in higher frequency in the evening and early morning hours. The use of habitats outside wetlands implies management strategies may need to be updated to encompass these additional habitats.}

\normaltalk{\capitalisewords{Effective Multi-Scale Governance for Migratory Shorebirds: Lessons from the EAAF and an Australian Wetland Site}}{\underline{Evan V Hamman}}{Migratory shorebirds are amongst some of the world's most endangered birds. Their effective conservation requires a life-cycle approach and collaboration across international boundaries and within sub-national jurisdictions. Although scientific understanding of these unique birds has increased in recent years, evaluation of our governance responses (laws, policies, institutions etc.) is not well-developed. Laws and policies are the nuts and bolts of implementing effective environmental governance. The standards and institutions we establish, and the ways in which we go about monitoring and enforcing those rules (i.e. restricting hunting, development, agriculture etc.) must be an integral part of dealing with the threats. This paper examines the legal, political and other institutional arrangements for the conservation of migratory shorebirds in the East Asian-Australasian Flyway (EAAF). The work focuses on a case study of Australia's Moreton Bay Ramsar Site, one of the most important over-wintering sites for shorebirds along the flyway. Several levels of governance are evaluated, from the international, to the national, to the local. The paper adopts an approach which seeks to tease out those aspects of the local arrangement which either align or conflict with higher levels of governance including national and state-based laws as well as regional and international agreements for the protection of migratory waterbirds. Though the paper is focused on the EAAF, the lessons for effective multi-scale governance of migratory shorebirds (and indeed other migratory birds) will be translatable to other contexts, including in the United States.}

\normaltalk{\capitalisewords{Examining the Effect of Helicopter Noise on Bird Assemblages in Hawai'i's Protected Natural Areas}}{\underline{Karen V Gallardo Cruz}, Kristina L Paxton, Patrick J Hart}{Anthropogenic noise has been shown to have adverse effects on birds, including decreased breeding success, increased flushing behavior, and changes in vocalization patterns. The avifauna in Hawai'i is among the most threatened in the world, and helicopter noise in Hawai'i's forests could be another stressor native birds face in addition to disease, habitat loss, and non-native species. The number of helicopter overflights in Hawai'i Volcanoes National Park (HAVO) is one of the highest in the National Park system, but the effect of helicopter noise on native birds within the park has not been assessed. Our primary objective was to determine if helicopter noise affects the acoustic behavior of native bird assemblages within protected natural areas. We placed automated acoustic recorders in three forested areas that are subjected to helicopter traffic from air tours, two in HAVO and one in the Upper Waiakea Forest Reserve on the Island of Hawai'i. We addressed the following questions: 1) Does helicopter noise affect biophony? 2) Is there a threshold at which helicopter noise affects vocalizing behavior of birds? And 3) does response to helicopter noise vary among bird species? We found that Japanese White-eye and Hawai'i 'Amakihi increase their vocalization time in response to helicopter noise. Our results may serve as the foundation of an air tour management plan that considers potential effects of air tours on native forest birds}

\normaltalk{\capitalisewords{Designing a Bird Survey for Latin America: The PROALAS Protocol in eBird}}{\underline{Viviana V Ruiz-Gutierrez}, Anna Lello-Smith, Ruben Ortega Alvarez}{Understanding patterns of bird abundance and distributions is one of the biggest challenges facing avian research and conservation efforts across the Americas. Large-scale monitoring efforts, such as the Breeding Bird Survey in the US, have traditionally provided such information for much of the US and Canada. However, similar efforts are lacking for much of the Neotropics, limiting our inferences for resident and migratory species beyond the US and Canada. Although avian research and monitoring efforts exist in Latin America, these present a common set of challenges for making scalable inferences: 1) data are often project specific and not readily available, 2) protocols lack a consistent set of variables or compatible methodologies, and 3) protocols vary widely in objectives and often based on suggestions for monitoring birds in temperate regions. Here, we present the design and application of PROALAS- an occupancy-based, bird-monitoring program for Latin America embedded into eBird. We provide details on the factors that informed specific aspects of the PROTOCOL, which is composed of 30-m radius point counts and 100-m transects. In addition, we discuss a couple case-studies in Mexico, Guatemala and Costa Rica of how PROALAS is being applied by different partners in Latin America.}

\normaltalk{\capitalisewords{Revealing Unknown Interactions: The Role of Patagonian Tapaculos (Rhinocryptidae) on the Dispersal of Endemic Truffles}}{\underline{Marcos V Caiafa}, Michelle A Jusino, Iv\'{a}n A D\'{i}az, Matthew E Smith}{Some fungi play key roles in forests by feeding nutrients to the roots of trees in exchange for sugars in an interaction called mycorrhizae. Some mycorrhizal fungi form underground mushrooms called truffles that rely on animals to disperse their spores. Most truffles produce strong odors to attract mammals that dig them up, eat them, and spread the spores in their feces. Temperate rainforests of southern Chile are dominated by Nothofagaceae trees that depend on truffles for nutrients, but mammals are relatively scarce and the truffles in this system lack strong odors. However, the truffles resemble fruits, suggesting that they are dispersed by animals that eat fruit and rely largely on sight rather than smell to find food. The objective of this study is to document the symbiosis between two understory birds: Chucao Tapaculo (Scelorchilus rubecula) and Black-throated Huet-huet (Pteroptochos tarnii) and several species of endemic truffles in the temperate rainforests of southern Chile. We documented the diet of the birds by microscopically and molecularly analyzing fecal samples. Standard microscopy methods were used for fungal spore identification and quantification, and we confirmed the presence of fungal spores in the samples. We used high-throughput amplicon sequencing (HTS) to identify fungi by sequencing fungal-specific markers. Our HTS results confirmed the presence of at least 29 ectomycorrhizal taxa, including nine truffle-like fungi. These results suggest that both chucaos and huet-huets are eating truffle-like fungi and putatively dispersing spores. This is one of only a few cases of true symbiotic avian mycophagy reported so far.}

\normaltalk{\capitalisewords{Evidence for Massive Decline in the North American Avifauna}}{\underline{Kenneth V Rosenberg}, Adriaan M Dokter, Peter J Blancher, John R Sauer, Adam C Smith, Paul A Smith, Jessica C Stanton, Arvind O Panjabi, Laura Helft, Mike Parr, P M }{Species extinctions have defined the global biodiversity crisis, but extinction begins with loss in abundance of organisms that can result in extreme compositional and functional changes of ecosystems. Using multiple and independent monitoring networks, we report major population losses across much of the North American avifauna, including in once common species and from nearly every biome. By integrating the best available long-term population trends with estimates of population size, we provide the first calculations of net population change, across 530 bird species. Our startling results indicate a net population loss approaching three billion birds, or 29\% of 1970 abundance. Loss of abundance has occurred in nearly every major breeding and non-breeding biome, and across taxonomic groups—90\% of loss is within 12 large bird families. Grassland birds, shorebirds, and migratory species wintering in South America and coastal/marine regions have suffered particularly dire population declines. Surprisingly, introduced species (e.g. House Sparrow, European Starling) show among the largest loss (63\%), paralleling similar declines in these same species within their native range. Population loss is thus not restricted to rare and threatened species, but includes many widespread and common species that may be disproportionately influential components of food webs and in ecosystem function. Furthermore, declining species are not being replaced by species that fare well in human-altered landscapes. The few increasing groups of birds, especially waterfowl managed for adaptive harvest and raptors recovering from DDT and persecution, offer successful conservation models that must be applied to reverse widespread declines across the continental avifauna.}

\normaltalk{\capitalisewords{Developmental Experience may Contribute to Adult Female Preferences for Male Mating Signals}}{\underline{Amy VH Strauss}, Jeffrey Podos}{Reproductive isolation between closely related animal species may be maintained by behavioral mechanisms including divergence in sexual signals and signal preferences. In order to understand the role such communication systems play at species boundaries, we must examine the factors that drive the evolution of both mating signals and mating signal preferences. While much research has explored signal evolution in males, less is known about the evolution of female preferences. The present study investigates the development of preferences in female songbirds by testing the relative contributions of innate versus learned factors in shaping adult responses to male song. Wild-caught female songbirds were reared in a laboratory and as juveniles were exposed equally to conspecific and heterospecific male songs (sister species, sympatric). As adults, behavioral assays were used to characterize female responses to songs in the following categories: conspecific, heterospecific (sister species, exposed early), and heterospecific (novel, control). Results indicate that genetics alone cannot explain the development of song preferences by female songbirds and suggest a role for early experience.}

\normaltalk{\capitalisewords{Provisioning Coordination Increases with Mate Familiarity in a Long-Lived Pelagic Seabird, the Manx Shearwater}}{\underline{Chris W Tyson}, Natasha Gillies, Tim Guilford, Josh Hull}{Seabirds as a group are notable for displaying high mate fidelity and maintaining long-term pair bonds. While this behavior is generally believed to be adaptive due to a commensurate increase in reproductive success with pair experience, the behavioral benefits of mate familiarity are largely unknown. To evaluate the impact of mate familiarity on breeding competence and potential fitness implications, we studied Manx shearwaters (Puffinus puffins) on Skomer Island, Wales. From 2009 to 2018, we observed over 1,500 breeding attempts from 332 distinct pairs with 40\% of pairs remaining together for more than one year.     During our study, the probability of inexperienced pairs fledging a chick was ~65\%, which increased by ~5\% for each additional year of mate familiarity. To evaluate behavioral mechanisms underlying this relationship, we used radio-frequency identification (RFID) readers to monitor nest-attendance patterns of 65 pairs during chick rearing in 2018. During chick rearing, Manx shearwater pairs undergo two types of foraging trips; short duration trips for chick provisioning and long trips for self-care. We found that more experienced pairs coordinated to a greater degree by alternating between foraging trip types, resulting in more consistent chick provisioning as well as more balanced self-care foraging time within pairs. These results provide compelling evidence for a positive mate familiarity effect in this species and provide some of the first documentation for a specific breeding behavior improving as the pair bond matures.}

\normaltalk{\capitalisewords{Decadal Shifts in Patterns and Benefits of Nesting Near Territorial Predators by Tundra-Nesting Sea Ducks}}{\underline{Micah W Miller}, James R Lovvorn, Nathan Graff, Neesha Stellrecht, Lori T Quakenbush, David E Safine}{Birds may nest near territorial predators, presumably because benefits of protection exceed risks of nest depredation by host predators. Tundra-nesting sea ducks may nest near territorial Pomarine Jaegers (POJA, Stercorarius pomarinus) or Snowy Owls (SNOW, Bubo scandiacus) for defense against Arctic foxes (Vulpes lagopus). Lemming-specialist POJA and SNOW apparently nest in higher densities in years of moderate to high lemming abundance while foxes shift from lemmings to bird- and egg-focused diets, thus impacting sea duck nest survival. Anthropogenic reduction of foxes may influence nesting behavior of both sea ducks and territorial avian predators. During the 1990s, threatened Steller's eiders (STEI, Polysticta stelleri) nested close to avian predators, but this relationship was less defined during the 2010s when foxes were controlled to increase nest survival. Three other sea duck species (King Eider, Somateria spectabilis; threatened Spectacled Eider, S. fischeri; and Long-tailed Duck, C. hyemalis) also showed tendencies to nest near POJA and SNOW during the 2010s. Sea ducks appeared to avoid nesting near Parasitic Jaegers (Stercorarius parasiticus), a common nest predator. No benefit to apparent nest success was measurable through these associations. In one subarea, apparent POJA nesting densities declined greatly from the 1950s to the 2010s, perhaps in concert with declines in lemming densities. Widespread declines in lemming abundance may diminish the relationships among sea ducks, territorial avian predators, and foxes. This issue is especially important for STEI, which typically have shown minimal breeding effort in low lemming years when POJA and SNOW were absent or at very low densities.}

\normaltalk{\capitalisewords{Climate Change Threatens a Songbird Population Through Its Impacts on Breeding}}{\underline{Thomas W Bonnot}, W A Cox, Frank R Thompson, Joshua J Millspaugh}{Understanding global change processes that threaten species viability is critical for assessing vulnerability and deciding on appropriate conservation actions. Here we combine individual-based and metapopulation models to estimate effects of climate change on annual breeding productivity and population viability of a common forest songbird (Acadian flycatcher [Empidonax virescens]) through 2100 across the Central Hardwoods ecoregion, a 39.5 million hectare area of temperate and broadleaf forests in the United States.  Our approach integrates local-scale, individual breeding productivity, estimated from empirically-derived demographic parameters that varied with landscape and climatic factors (such as forest cover, daily temperature) into a dynamic-landscape metapopulation model that projects growth of the regional population over time. We show that warming temperatures under worst case scenario with unabated climate change could reduce breeding productivity to an extent that this currently abundant species will suffer population declines substantial enough to pose a significant risk of local-extinction from the region in the 21st century. However, we also show that this risk is greatly reduced for scenarios where emissions and warming are curtailed.  These results highlight the importance of considering both direct and indirect effects of climate change when assessing vulnerability of species.}

\normaltalk{\capitalisewords{Cultural Evolution Promotes Rapid Character Displacement in Sympatric Crossbills}}{Cody K Porter, \underline{Craig W Benkman}}{Cultural evolution of learned traits may promote rapid evolutionary divergence and possibly speciation, but testing this hypothesis remains a monumental challenge. We document substantial divergence of learned vocalizations via cultural evolution in the Cassia crossbill (Loxia sinesciuris) from a closely related, sympatric ecotype of the red crossbill (L. curvirostra) over just two decades. The rapid population-level divergence in vocalizations appears to be driven by individual-level modifications to vocalization structure over the course of their lifetime, which are learned by offspring and thus transmitted to subsequent generations. Field experiments demonstrate that individuals of both the Cassia crossbill and the red crossbill ecotype differentially respond to such changes in call structure, reducing heterospecific flock formation and thereby promoting assortative mating because crossbill choose mates from within flocks. This process of character displacement via cultural evolution has likely been critical for the extremely rapid diversification of reproductively isolated and ecologically divergent crossbill lineages in the absence of geographical barriers to gene flow. More generally, our results suggest that cultural evolution of adaptive behaviors can be a powerful driver of evolutionary change.}

\normaltalk{\capitalisewords{Trophic Transfer of Energy via Nestling Wading Birds to a Scavenger Community in an Oligotrophic Wetland}}{\underline{Wray W Gabel}}{Effective energy transfer and availability are crucial to maintaining ecosystem dynamics and productivity. Large aggregations of colonially-breeding birds can concentrate significant quantities of easily available energy in the form of fallen nestlings. Nestling carcasses provide a valuable resource for scavengers who consume carcasses and recycle their nutrients, however, the actual fate of nutrients and mechanisms of energy transfer via nestlings in these breeding colonies often remains undocumented. We used trail cameras to monitor the fates of appropriately sized chicken baits to examine this question for wading bird colonies in the Everglades of Florida, USA. Overall, 83\% of all carcasses were consumed, and Tukey Vultures (Cathartes aura) (45\% of baits) and American Alligators (Alligator mississippiensis) (31\% of baits) were the primary consumers of nestlings. Distance from nest to water, local nesting density, and island type (small or large heron islands) were important predictors of scavenger identity. We estimate that, on average, fallen nestlings from Great Egrets (Ardea alba) and small (Egretta) herons could support 26 female alligators and 18 Turkey Vultures for a period of 60 days annually. This suggests that fallen nestlings can serve as a considerable source of energy for wetland scavenger communities, and this may be exacerbated in the highly oligotrophic Everglades marshes. Breeding bird colonies can be found globally, and there are many undescribed scavenger communities that stand to benefit from large scale carcass deposition from comparable aggregations of colonially nesting birds.}

\normaltalk{\capitalisewords{A Range-Wide Geolocator Project Fills Critical Knowledge Gaps of Our Understanding of the Full Annual Cycle of Cerulean Warblers}}{\underline{Douglas W Raybuck}, David A Buehler, Than J Boves, Larkin J L, Scott H Stoleson, Lesley P Bulluck, Gregory A George, Laura J Kearns, Kate G Slankard, Sharon Petzinger, John J Cox, Kristin A Mylecraine}{Cerulean Warblers (Setophaga cerulea) are a declining Nearctic-Neotropical migrant species. Implementing full annual cycle conservation strategies to facilitate recovery has been difficult because we know little about the migratory period and connectivity between North American breeding regions and South American non-breeding regions. Cerulean Warblers spend over half of their adult lives in the northern Andes Mountains, where much of their montane forest habitat has vanished over the past half century while the species has declined at nearly 3\% per year. From 2014-2017, we deployed 255 light-level geolocators on male Cerulean Warblers at 11 study sites strategically spread throughout the breeding range with the objectives of a) tracking the approximate migration routes of individual males; b) evaluating the strength of connectivity between breeding and non-breeding regions; c) identifying stopover regions and duration; and d) documenting migration chronology. We retrieved migration data from 28 birds and used the R packages SGAT and GeoLight to determine locations of stationary periods and timing of migration. All of the 28 birds used migratory routes and stopovers in Central America and the majority (20/21; 95\%) of Great Lakes and Appalachian birds overwintered in northern Colombia or northwestern Venezuela, while 5/7 (71\%) of birds tagged in the Ozarks overwintered in Peru/Ecuador. Identification and protection of habitat in critical stopover regions could benefit the species by improving annual survival and condition upon arrival on the breeding grounds. The observed connectivity pattern could allow for enhanced effectiveness of conservation planning through focused conservation efforts of target populations.}

\normaltalk{\capitalisewords{Comparative Phylogeography of Four Lowland Forest Birds Distributed Across the Dahomey Gap of West Africa}}{\underline{Jerry W Huntley}, Paul Sweet, Toby Hibbitts, Hamidou Arame, Alphonse Adite}{The Upper and Lower Guinean forests of western and central Africa, respectively, are divided by the Dahomey Gap, a roughly 200 km-wide arid break in continuous tropical rainforest, centered on the countries of Benin and Togo. For decades, the Dahomey Gap has been posited as a major barrier to gene flow for avian species between these forest blocks. Recent studies have implicated the gap as an explanation for major genetic differences observed between species distributed across the region. However, these studies have worked under the constraints of sparse geographic sampling that were unable to reject alternative mechanisms (e.g., riverine barriers or isolation by distance). To test alternative models proposed to explain speciation in the Afrotropical lowlands, we apply population genomics and dense geographic sampling of four lowland forest species that are distributed across the Dahomey Gap, representing the first ever current generation sequencing dataset (ultraconserved elements) for African tropical phylogeography. First, we evaluate the efficacy of the Dahomey Gap as an historical and modern barrier to gene flow for avian species distributed on either side. Second, we examine competing hypotheses including Pleistocene climatic change and the formations of the nearby Niger and Sanaga Rivers as barriers within the region.  Using genomic data and comparative analytical techniques such as demographic model selection, we seek to further untangle the complicated relationships within this region revolving around multiple potential barriers and changing historical landscape.}

\normaltalk{\capitalisewords{From Single-Day to Multi-Year Revisits: Matching Sampling Strategies to Inferential Goals}}{\underline{Morgan W Tingley}}{To correct for imperfect detection in point count data, hierarchical models (e.g., occupancy models) require repeat temporal samples of observed occurrence. However, there is a wide spectrum in the literature – and little guidance – on what is the optimal amount of time between repeat surveys. If you wait a long time between surveys, you risk violating the controversial "closure assumption," but if you minimize time between surveys, your surveys may lack independence. Here, I present three case studies of point counts that vary in the amount of time between surveys: (1) replicates spread across years; (2) replicates spread across a season; and (3) immediately adjoining replicates. For each, I present the strengths and weaknesses of the sampling design, and how the model met the inferential goals. A critical consideration is how the spatial scale of a survey interacts with the temporal scale of a replicate, and the role both play in defining what is "detectability" and what is "occupancy" in each study. Finally, I offer recommendations for designing point count studies from the perspective of the future analyst.}

\normaltalk{\capitalisewords{Broad-Scale Monitoring to Evaluate the Contribution of Local Conservation to Regional Bird Populations}}{\underline{Adam W Green}, David C Pavlacky}{Monitoring is crucial for effective and efficient management of wildlife populations. The objectives of large-scale monitoring programs are often limited to determining population status and trend, yet many programs lack a framework for linking management decisions to trends in population size. Long-term trends are useful in conservation planning to prioritize and assess the vulnerability of species, but most use indices of relative abundance and have limited usefulness for evaluating population responses to conservation. We describe analyses of point count data from the Integrated Monitoring in Bird Conservation Regions program to obtain multi-scale trends of absolute abundance. We then describe an application to quantify the effects of local habitat, landscape condition, and management practices on populations. The combination of a robust sampling design, distance sampling, and analysis using N-mixture models allows us to estimate abundance and trend at scales ranging from national grasslands and private lands to states and Bird Conservation Regions. We can also quantify the contribution of landscape management to regional populations and determine the effort needed to meet conservation goals. Comparisons across scales can help managers prioritize resource allocation for species conservation and provide insight into the most effective and efficient conservation practices to stabilize declines or meet population and management goals. A future research direction is to integrate the limited spatial and temporal extent of the IMBCR program with other large-scale bird monitoring programs, such as the Breeding Bird Survey, leveraging the benefits of each program to make better inference and management decisions in the future.}

\normaltalk{\capitalisewords{Long-Term Effects of Environmental Change on Two Cavity Nesting Passerines}}{\underline{Andrew W Bartlow}, Tyler E Wysner, Karen S Musgrave, Charles D Hathcock, Jeanne M Fair}{Long-term data on phenology and fitness is important for predicting how avian populations will be altered in an era of global change. A nestbox dataset of Western Bluebirds (Sialia mexicana) and Ash-throated Flycatchers (Myiarchus cinerascens) was used to study changes to their populations over two decades in northern New Mexico. We tested for shifts in breeding date due to environmental change. We also determined if parasitic blowflies and changing conditions affect fledging success. We found that hatch date did not change over time for either species, but nesting elevation increased significantly for bluebirds. Hatching later in the breeding season resulted in smaller clutch sizes for both species. Results suggest breeding advancement has not occurred, but increasing nesting elevation may allow bluebirds, but not flycatchers, to cope with changing environments. In terms of parasites, blowflies increased over time in bluebird nests, but not flycatcher nests. Blowflies were more abundant in nests of both species when there was less precipitation, and we found that more blowflies resulted in higher fledging success for only bluebirds. There were significant interactions between blowflies and high temperatures and between blowflies and drier conditions on flycatcher fledging success. Interactive effects resulted in blowflies having a more negative impact on flycatcher fledging during higher temperatures and drier conditions. Blowflies that parasitize flycatchers are subcutaneous, which may be more virulent under certain conditions. Both studies suggest flycatchers are less able to cope with changing climates than bluebirds, through the inability to shift their breeding and changes to host-parasite dynamics.}

\normaltalk{\capitalisewords{Factors Associated with Marbled Murrelet (Brachyramphus Marmoratus) Nesting Activity in Western Oregon}}{\underline{James W Rivers}, Lindsay Adrean, S K Nelson, Marie-Sophie Garcia-Heras, Daniel Roby, Matthew Betts}{Recovering populations of the threatened Marbled Murrelet (Brachyramphus marmoratus) requires information about the factors that influence reproductive success, yet the cryptic breeding behavior of this species poses a sizable challenge for obtaining nest data. For example, only 30 active murrelet nests were located in Oregon prior to 2018 which, in turn, has hindered conservation planning efforts. During two breeding seasons (2017–2018) we tagged 137 murrelets at sea with VHF radio tracking tags to monitor inland movements to nest sites. In contrast to 2017 when we detected no inland movements by tagged birds, we found that 12 of our tagged murrelets (16\%) moved inland in 2018 resulting in the discovery of 8 active nests. Of the 6 nests monitored by videocamera, 5 nests had eggs that successfully hatched of which 3 ultimately fledged a chick. Despite having a modest sample of nests, our study was able to provide the first record of a new nest predator (Red-tailed Hawk [Buteo jamaicensis]), establish a new tree species for nesting in the United States (bigleaf maple [Acer macrophyllum]), and increase the number of active nests with known fates in the state by $>$25\%. Subsequent years of this long-term study should allow for additional analysis of environmental factors affecting nest success and provide information needed to improve conservation planning for this species in Oregon's coastal forests.}

\normaltalk{\capitalisewords{Limitations to Predicting the Future of Climate-Driven Bird Declines}}{\underline{Morgan W Tingley}}{With accumulating evidence implicating climate change in bird population declines, increasing our understanding of the mechanisms underlying climate-driven declines is a critical current challenge. Causal pathways for climate change can be classified as having direct or indirect impacts on bird populations. With the help of a resurgence in ecophysiology, there is a developing universal theory for how climate change directly impacts birds through thermal limitations and water balance. Emerging lessons from this research on direct effects include the growing consensus that tropical birds are at greater risk than temperate species. Indirect mechanisms, however, underlie the vast majority of climate change impacts on birds, through processes such as changing habitat, resource limitation, and shifting interactions with parasites, predators, and competitors. Recent work highlights how these biotic knock-on effects of climate change frequently have stronger impacts on species than the direct physiological impact of the abiotic environment. In some cases, indirect impacts are already leading birds to extinction. Future conservation efforts will rely on our ability to accurately predict climate change impacts, yet the mechanistic pathways that pose the greatest threats to birds are also the pathways with the greatest current uncertainty and the lowest integration with predictive models.}

\normaltalk{\capitalisewords{Ecological Niche Modeling and Potential Implications for Alaskan Birds under 2 \textdegree C Warming}}{\underline{Joanna X Wu}, Brooke Bateman, Chad B Wilsey, Lotem Taylor, Gary Langham}{Birds are one of the first taxa to show a response to ongoing climate change. Compounded by drivers such as land-use change, adaptive capacity, and trophic interactions, climate change has been shown to have complex, multidirectional impacts on birds. Audubon's latest ecological niche models for 600 North American species project changes in climate suitability at a 1-km resolution across three countries. We extracted species projections at Alaskan National Wildlife Refuges for the 2050s under greenhouse gas concentration trajectory RCP8.5, equivalent to about a 2.0 °C warming scenario, and summarized the information for each refuge. Compared to the present, climate in Alaska is projected to become, overall, more favorable for birds under this scenario. In general, we may see range expansion of many species, particular birds of the western forests, into Alaska. At the Kenai National Wildlife Refuge, the number of species that may colonize the refuge is projected to exceed the number of potential extirpations in both summer and winter. For these potential colonizations to become realized requires a number of other factors including dispersal ability, food and habitat availability, phenotypic plasticity, etc. Regardless, these models predict drastic change in avian composition in Alaska. Refuge managers can use these informed projections as one of the tools in planning for an uncertain future. We recommend managers in Alaska to focus on actions that increase connectivity across land ownerships for birds to move across boundaries, and consider increasing the amount of potential habitat for birds in the future.}

\normaltalk{\capitalisewords{Plague Management on Prairie Dog Colonies Maintains Habitat for Grassland Passerines and Raptors}}{\underline{Reesa Y Conrey}, Daniel W Tripp, Erin N Youngberg, Arvind O Panjabi}{Prairie dogs (Cynomys sp.) are highly susceptible to plague, a disease caused by the non-native bacterium Yersinia pestis, introduced to the Great Plains of North America in the 1940s–50s. Plague epizootics can have cascading effects on non-susceptible species associated with prairie dog colonies. Colorado Parks and Wildlife has completed a study of plague management in prairie dogs, in which oral vaccine treatments were compared to insecticidal dusting of burrows and placebo baits. We continue to study the effects of plague and plague management on birds and mammalian carnivores associated with colonies of black-tailed (C. ludovicianus) and Gunnison's prairie dogs (C. gunnisoni) in shortgrass and shrub-steppe regions of Colorado. Vaccine and insecticide applications did not eliminate plague from our study areas but maintained pockets of prairie dogs and promoted population recovery. Grasses were taller and plant cover generally higher following epizootics. Species such as Horned Larks (Eremophila alpestris) and Vesper Sparrows (Pooecetes gramineus) responded negatively to plague, while Grasshopper Sparrows (Ammodramus savannarum) and other species requiring taller structural components responded positively. Raptors such as Ferruginous Hawks (Buteo regalis) and Golden Eagles (Aquila chrysaetos) foraged at much higher rates on active colonies, while Burrowing Owls (Athene cunicularia) were common post-plague in the small regions that retained live prairie dogs. Species that prey upon prairie dogs or preferentially nest or forage in short stature grasslands are the most likely to benefit from plague management.}

\normaltalk{\capitalisewords{Genetic Connectivity and Differentiation among Horned Larks (Eremophila Alpestris) along an Elevational Gradient in the White Mountains of California}}{\underline{Cynthia Y Wang-Claypool}, Nicholas A Mason, Carla Cicero, Rauri C Bowie}{Species' distributions that span elevational gradients present opportunities to study how abiotic factors affect the ecology and evolutionary history of wide-ranging species. The Horned Lark (Eremophila alpestris) is a common and widespread bird of open habitats in North America, occurring across a breadth of abiotic conditions and environments. In this study, we examined population connectivity and genetic differentiation along an elevational transect among Horned Larks in the western United States. We explored signatures of adaptation to environmental conditions (e.g., temperature, moisture) and characterized signals of isolation-by-distance and isolation-by-environment as well as their roles in shaping patterns of genetic variation across the elevational transect. We included samples from California along the SE Sierras, from 285 m to 3540 m above sea level, which spans an elevational gradient from the floor of Death Valley to the highest peak of the White Mountains. We sampled thousands of loci using double-digest Restriction-Associated DNA (ddRAD) sequencing and performed an array of population genetic analyses that characterized population structure, admixture, and signatures of natural selection. Our study sheds light on how large elevational distributions evolve and provides insight into the ways by which organisms adapt to high-elevation environments.}

\normaltalk{\capitalisewords{Understanding Underserved Audiences to Ensure Ongoing Long-Term Citizen Science Data Collection for Project FeederWatch}}{\underline{Victoria Y Martin}, Emma Greig, David Bonter}{While citizen science is rapidly developing, questions have been raised about who typically participates (often highly-educated people) and how to broaden audience types to ensure the long-term viability and success of programs. To achieve high levels of sustained engagement, it is essential for project managers to understand their current and potential participants. Project FeederWatch has been operating for 32 years and currently attracts approximately 24,000 fee-paying participants annually to identify and count birds visiting backyard feeders during winter in North America. The data collected have yielded numerous scientific papers describing feeder birds' abundance, distributions, behavior, and health. While the number of volunteers and the high retention rate (approximately 65\% return every year) are impressive, the average age of participants has increased annually (from 52.1 years in 1999 to 64.0 years in 2017). To address the need to attract younger audiences, we have conducted several studies with young adults (primarily 18-35 year-olds) to understand their needs, interests, and the social networks they use to discuss birds they see at their feeders. This presentation will discuss the barriers and drivers for this under-served group to participate in citizen science, and highlight efforts underway to build a community for young adults who feed wild birds in North America.}

\normaltalk{\capitalisewords{Social Polygyny and Genetic Confirmation using minION DNA Sequencing in the Eastern Bluebird}}{Katherine M King, \underline{Memuna Z Khan}}{Orian's polygyny threshold model states that polygyny occurs when a female enjoys greater reproductive success with a high quality male in a socially polygynous arrangement than a low quality male in monogamy.  Assumptions of this model state that polygynous females bear the cost of sharing resources on a territory.  Although Eastern Bluebirds (Sialia sialis) commonly exhibit genetic polygyny in the form of extra pair fertilizations, reports of social polygyny are rare.  We report a serendipitous observation of social polygyny by a known age male on separate territories.  This male was observed tending a nest with 14 day old nestlings and tending a nest box 200m away that housed a different female incubating eggs during the same time period.  After the older nestlings had fledged, the male continued to tend young at the second nest. The context of this these observations suggest that the polygyny threshold model is not an adequate explanation.  Furthermore, behavioral observations were later genetically tested using the minION Nanopore gene sequencing tool on blood samples from the male breeder, and offspring from each his two nests.}

