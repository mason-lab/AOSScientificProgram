\normaltalk{\capitalisewords{Pleistocene Glaciation Explains the Disjunct Distribution of the Chestnut-Vented Nuthatch (Aves, Sittidae)}}{\underline{Min Zhao}, Yongbin Chang, Rebecca T Kimball, Jian Zhao, Fumin Lei, Yanhua Qu}{Pleistocene climatic oscillations have played an important role in shaping many species' current distributions. In recent years, there has been increasing interest in studying the effects of glacial periods on East Asian birds. Integrated approaches allow us to study past distribution range changes due to Pleistocene glaciation, and how these changes have affected current population genetic structure, especially for species with unusual distribution patterns. The Wuyi disjunction is the disjunct distribution of birds between the Wuyi Mountains in southeastern China and southwestern China. Although several species exhibit the Wuyi disjunction, the process behind this unusual distribution pattern has remained relatively unstudied. Therefore, we used the Chestnut-vented Nuthatch Sitta nagaensis as a model species to investigate the possible causes of the Wuyi disjunction. Based on phylogenetic analyses with three mitochondrial and six nuclear regions, the Wuyi population of the Chestnut-vented Nuthatch was closely related to populations in mid-Sichuan, from which it diverged approximately 0.1 million years ago, despite the long geographical distance between them (over 1300 km). In contrast, geographically close populations in mid- and southern Sichuan were genetically divergent from each other (more than half a million years). Ecological niche modeling suggested that the Chestnut-vented Nuthatch has experienced dramatic range expansions from Last Inter-glacial period to Last Glacial Maximum, with some range retraction following the Last Glacial period. We propose that the Wuyi disjunction of the Chestnut-vented Nuthatch was most likely due to recent range expansion from southwestern China during the glacial period, followed by post-glacial range retraction.}

\normaltalk{\capitalisewords{Social Mimicry Hypotheses in Mixed-Species Bird Flocks Tested in Costa Rica´s Highland Bird Community}}{\underline{Pablo Mu\~{n}oz}, Luis Sandoval}{Mixed-species flocks have been widely studied focusing mostly on their composition, foraging efficiency, and protection against predators. But, the processes governing their cohesion are not totally understood yet. Plumage coloration, as a convergence among the mixed-species flocks, is one of the less studied factors that may affect flock cohesion. This plumage coloration convergence of mixed-species flocks fits Moynihan´s social mimicry hypothesis. This hypothesis states that birds in mixed-species flocks may present a convergence in coloration, which will make them less conspicuous compared to other bird species, as a way to compensate the flock acoustic conspicuousness. However, plumage coloration convergence in mixed-species flocks has not been properly addressed yet. Our objective was to test Moynihan's social mimicry hypothesis in mixed-species flocks from Costa Rica highlands. We measured plumage reflectance of 6 adult individuals per species (3 per sex) using reflectance spectrophotometry. We chose 20 species that form part of mixed-species flocks and 20 species that do not, but are genetically close to the species that form flocks. In each individual we measured reflectance from: crown, nape, mantle, rump, breast, sides, and dorsal side of external rectrices, and primaries. We used a tetrahedral colorspace visual model to compare plumage patches among species. We found partial support for the social mimicry hypothesis, because just some unrelated flocking species have similar coloration. This convergence may respond also to the type of habitat they use the most while in the flock, which means that the convergence we expect will match their habitat background.}

\normaltalk{\capitalisewords{Okinawa Rail Has Relatively Larger Wulst in the Brain: Relationship Between Anvil-using Behavior and Brain Morphology}}{\underline{Tatsuro Nakao}, Naomichi Ogihara, Takeshi Yamasaki, Masaki Shimada}{Okinawa Rail (Gallirallus okinawae) is a flightless bird. They are the only species among the Rallidae in which anvil-using behavior has been observed. Relative brain volume of animals who perform behaviors requiring high cognitive abilities tends to be larger. In birds, especially the brain area called Wulst in the Pallium is considered to be the center of cognitively advanced behavioral learning. Therefore, Okinawa Rail that performs anvil-using behavior may have a larger brain volume, and/or Wrust in a Pallium may be relatively larger than other Rallidae species. Our samples consisted of 29 skull specimens of extant birds from 5 species of Rallidae including Okinawa Rail. All skulls were digitally scanned using a CT scanner. We subsequently extracted the internal surface of the skull (endocast)end from acquired CT images. The approximate volume of Wurst from each specimen was estimated, as well as brain volume. The ratio of of Wurst to the brain volume was compared among species. We also compared relative brain volume of Okinawa Rail and those of other Rallidae species. Although there was no correlation between flightlessness nor anvil-using behavior and brain volume in Rallidae species, it was suggested that the ratio of Wulst to brain volume of Okinawa Rail was significantly larger than those of other Rallidae species. Our results suggest that the development of Wulst, not the larger brain, is related to the anvil-using behavior.}

\normaltalk{\capitalisewords{Extra-Pair Parentage in Field Sparrows (Spizella Pusilla)}}{\underline{Emma Conroy}, Nicole Nemeth, Joel Ralston}{Extra-pair parentage (EPP) occurs when socially monogamous male and female birds mate outside of their social pair, resulting in young that are not genetically related to one of the parents. EPP may occur either when females engage in copulations with males outside of their social pair, or because of intraspecific brood parasitism (IBP) when females lay their eggs in nests belonging to other females of the same species. Both extra-pair paternity and IBP may influence reproductive fitness by reducing the costs of parental care. In this study we investigate rates of EPP in Field Sparrows (Spizella pusilla). Breeding pairs of Field Sparrow were captured, bled, and color banded at St. Patrick's County Park in South Bend, IN in June and July 2018. Banded adults were observed to establish social pairs and locate nests. Blood samples were taken from 23 social parents and potential extra-pair mates. We found 8 nests and obtained genetic samples from 18 offspring. Each individual was genotyped at five microsatellite loci to determine parentage. Offspring not sharing alleles with the social father or mother were determined to be the product of extra-pair paternity or IBP, respectively. These results will contribute to ongoing research on EPP, IBP, and their effects on fitness in shrubland birds.}

\normaltalk{\capitalisewords{Living on Edge: Interactions Between Stress Hormones and Blood Parasites in Birds along Elevation Gradients}}{\underline{Viridiana Martinez}, Jacquelyn K Grace}{Vertebrate animals respond to changes in the environment with hormonal responses, such as glucocorticoids, to maintain homeostasis. Glucocorticoids are steroid hormones that influence metabolism and the immune system. Corticosterone (i.e., the primary avian glucocorticoid) is elevated in birds infected with parasites, although why this relationship exists is unclear. We still know very little about the interactions between parasite prevalence and the hormonal stress response in the wild, or the differing roles of glucocorticoid concentrations in mediating parasite colonization. Elevational zones are excellent models for investigating these interactions in nature because there are known parasite trends across differing elevations. This study investigates the interactions between glucocorticoids, parasite prevalence, and environmental elevation to improve our understanding of the mechanisms underpinning elevational refugia from disease, with implications for disease transmission and species survival under climate change scenarios. We present preliminary results from our first field season and discuss them within the context of the ongoing project.}

\normaltalk{\capitalisewords{Role of Age and Competition in Territorial Aggression in a Tropical Understory Bird}}{\underline{Laura Gomez Murillo}, Jonathan P Kelly, Camilo Alfonso Cuta, Megan Pendred, Corey Tarwater}{Territorial aggression is predicted to be an adaptive response to competitors as it can increase access to food resources, breeding space, and mates. Nevertheless, aggression may come at a cost in terms of risk of injury or increased probability of predation. Therefore, individuals must balance the costs and benefits of territorial aggression and this balance will depend upon intrinsic and extrinsic factors. Density, sex, the age of the territory holders, the relative ages of mated pairs, and the age of the territory intruder may alter the cost-benefit tradeoff. We examined territorial aggression in Black-crowned Antshrikes (Thamnophilus atrinucha), an understory tropical insectivore. We conducted playback experiments using simulated territorial intrusions of males of different ages in sites across the Isthmus of Panama which vary in density. Birds were more responsive to playback in higher density sites, younger territory holders were more responsive than older individuals, and males were generally more responsive, but this varied based on individual age and relative age of mated pairs. Although density is commonly examined in studies of aggression, our results indicate that age plays a dominant role in territorial aggression. Furthermore, most studies are conducted in species where females do not sing. Our work indicates that females not only participate in territorial defense, but can be more responsive than males, depending upon age. Understanding how the costs and benefits of territorial aggression vary in response to intrinsic and extrinsic factors is critical for evaluating mate choice and variation in individual fitness across contexts.}

\normaltalk{\capitalisewords{NMR-Based Metabolomics: a Novel Approach to Study Stress in Wild Birds}}{\underline{Asha Perera}, Karen Machin, Catherine Soos}{Human-induced environmental changes are occurring globally at unprecedented rates. These changes can act as stressors and may have significant impacts on wildlife health and populations. Metabolomics is increasingly used to study the effects of environmental stressors. The energy required for maintenance or other functions may be re-routed towards coping with stressors, ultimately resulting in fluctuations in metabolite levels associated with energy metabolism. While multiple studies have successfully applied metabolomics techniques in invertebrate and mammalian species, the use of this technique in wild birds is in its infancy. Our main objective was to validate the use of metabolomics in evaluating effects of stress on the metabolome of waterfowl. Captive Lesser Scaup (Aythya affinis) were implanted with either a biodegradable corticosterone pellet to mimic effects of chronic stress or a placebo pellet. 1H Nuclear Magnetic Resonance (NMR) spectroscopy was performed on serum samples collected during the active implant period. We hypothesized that metabolite profiles would differentiate ducks that received corticosterone pellets from placebo (control). A quantitative metabolite analysis showed that serum metabolite profiles could successfully differentiate ducks with higher circulatory corticosterone from control individuals. We further identified multiple key metabolites that varied between the two groups, all of which play important roles in energy metabolism. To our knowledge, this is the first study to investigate the use of NMR-based metabolomics techniques to study stress responses in wild birds. Metabolomics technique shows promise as a novel tool in identifying and characterizing physiological responses associated with large-scale environmental changes in wild birds.}

\normaltalk{\capitalisewords{The Effects of Louisiana's Shifting Coastline and Island Characteristics on Seabird Colony Presence}}{\underline{Juita Martinez}, Paul Leberg}{Many of the nation's seabird populations strictly rely on barrier and bay islands to complete their life cycles. Multiple factors are contributing to the rapid degradation of Louisiana's coastline and as a result valuable wildlife habitat are disappearing. Extensive restoration involving dredging and wave action barriers on selected islands have taken place over the last two decades. Although limited studies have been conducted on the effects of such large-scale projects on Louisiana's seabird nesting population throughout time. We utilized spatial mapping tools and historical aerial imagery to analyze island characteristics of 93 different barrier and bay islands spanning two decades in Terrebonne and Barataria Bay, Louisiana. With an extensive seabird colonial record dating back to the 1970s, we modeled the influence of year, area, perimeter, distance to mainland, elevation and restoration status on the presence and absences of seabird colonies on any given island. Our model for brown pelican (Pelicanus occidentalis) colony presence showed an increased on islands further from the mainland (P=0.01). We observed larger numbers of brown pelicans and laughing gulls (Leucophaeus atricilla) present on restored islands compared to unrestored (P$<$0.01). We also observed an increased likelihood of at least one seabird species nesting on restored islands than on unrestored islands (P$<$0.01). Further understanding of coastal fluctuations and restoration practices will aid in improving wildlife and environmental management efforts in the future.}

\normaltalk{\capitalisewords{Abundance and Occupancy of Mountain Plover in Northwestern México}}{\underline{Julio Merayo Garc\'{i}a}, Irene Ruvalcaba Ortega, Michael Wunder, Jos\'{e} I Gonzalez Rojas, Antonio Guzm\'{a}n Velasco}{The Mountain Plover (Charadrius montanus) is an upland shorebird associated to short-grass prairies, particularly to prairie dog (Cynomys spp.) colonies. It has declined during the last 50 years and is classified as "Threatened" in the Mexican List of Endangered Species. Its southernmost breeding population,  was discovered in Mexico barely 20 year ago and has been poorly studied, with two nesting records and under-sampled density estimations, .We therefore aimed to increase information on the distribution and abundance of the breeding population in the Grassland Priority Conservation Area "El Tokio", in northeastern Mexico, through occupancy (ψ) and N-Mixture models. We generated a grid of 30 ha plots, and selected 53 randomly in seven prairie dog towns in 2017, increasing to 77 plots and 15 sites in 2018. We sampled 3-4 times each plot between May-July, on consecutive days in 2017 and on the same day in 2018. Plovers were detected in five sites. Occupancy rates (ψ2017=0.4464, IC95\%=0.2053-0.7157, ψ2018=0.2319, IC95\%=0.1486 - 0.3432) and detection probabilities (p2017= 0.2870, IC95\%=0.1358-0.5077, p2018=0.6379, IC95\%=0.4916 - 0.7625) varied between years, showing an increase on the detection probability and  the accuracy of the estimation when repetitions where made during the same day. Mean plover abundance was less than 300 individuals. Our data suggest that the Mexican breeding population is relatively small; however a greater effort including other potential habitats in could increase such estimation.}

\normaltalk{\capitalisewords{Effects of Aviation Noise on Avian Communication along a Noise Pollution Gradient}}{Paola A Garrison-Tovar, Jazmine James, Denton Shepherd, \underline{Rachel Bolus}}{Songbirds sing in the dawn chorus to obtain mates and defend territories. However, dawn chorus is increasingly affected by noise pollution as humans encroach into their habitats. Noise pollution creates competition in the airspace, masking signals and decreasing communication quality. Avian populations can adapt their singing by adjusting the timing or frequency of their song to counteract the effects of this interference, even though these adjustments could have a negative effect on fitness. To determine the level at which noise affects the timing of song, we deployed acoustic recording units in six sites of varying noise. Three sites were near airports of varying size: Salt Lake City Airport and Cedar City Airport in Utah, USA and the Bar Ten Ranch Airstrip in Arizona, USA. Three other sites were in the same habitat and area of each of the airport sites, but farther away from airport noise. From these recordings, we quantified relative noise levels using Raven software and measured the timing of the dawn chorus in each of these sites. To determine the effects of variable noise on frequency of song, we measured the maximal, minimal, and peak frequencies of archived recordings of Tyrannidae and Icteridae and used the National Transportation Noise Map to quantify average noise levels at recording locations.}

\normaltalk{\capitalisewords{Marsh Bird Responses to Urbanization and Urban Wetland Habitat Availability at Multiple Spatial Scales}}{Stephanie J Beilke, Sarah Saunders, \underline{Anastasia Rahlin}}{Marsh birds are undergoing rapid declines in the United States, with habitat degradation and wetland loss considered primary causes of declines. Illinois and Indiana wetlands have suffered losses of 87-95\% since the early 1800s. Due to the cryptic natures of many wetland bird species, the factors and scales affecting wetland bird occupancy and habitat use remain unclear. To effectively conserve declining species, we need to confidently predict their occupancy and abundances. We conducted point counts and playback at 346 points across Illinois and Indiana, using the Standardized North American Marsh Bird Monitoring Protocol developed by Conway 2011. Points were located primarily in the Chicagoland region, in wetland patches surrounded by an urbanized matrix. We surveyed wetland patches for 17 focal species, including the American Bittern, Common Gallinule, Least Bittern, King Rail, Pied-billed Grebe, Sora, and Virginia Rail. We extracted remotely sensed wetland habitat, urban development, and invasive species (Phragmites) cover data from the National Wetlands Inventory, the Global Urban Footprint database, and USGS. We created occupancy and abundance models for all species of interest in order to determine how wetland birds respond to different wetland habitat types (emergent wetland, forested and shrub wetland, and open water), urbanization, and invasive Phragmites cover at different spatial scales (13 ha, 50 ha, and 1,257 ha). Our results underscore the importance of teasing apart the effects of multiple covariates at multiple spatial scales to highlight the best management and conservation recommendations across a suite of declining wetland bird species.}

\normaltalk{\capitalisewords{Invasive Plants and Bird Nesting Success in Missouri River Riparian Forests}}{\underline{Amanda Hegg}, Mark Dixon, David Swanson}{Nest predation has a major influence on the productivity of bird communities, and nest-, patch- and landscape-level characteristics may play important roles in mediating nest success. Introduction of invasive plants, however, may alter habitat features, predation, and nesting success. We examined whether invasive trees, Russian olive (Elaeagnus angustifolia) and eastern red cedar (Juniperus virginiana), in Missouri River riparian habitats influenced bird nesting success by monitoring nests and conducting vegetation surveys around nests from 2016-2018. At the nest-scale, we tested the impacts of nesting in invasive versus native substrates and the influence of vegetation cover on nesting success. At the patch-scale, we tested the effects of vegetation density, composition, and invasive tree prevalence on nesting success. Preliminary nest-scale data from the 2016-2017 field seasons suggest that birds used Russian olive for nesting more than any other nest substrate, and nesting success was generally lower in Russian olive than in native plants for individual bird species and all species combined. There was no uniform nest success trend for nests in eastern red cedar. Vegetation cover immediately surrounding nests best explained daily survival rate, and nesting success declined as percent cover increased. There was no support for the influence of patch-scale characters, such as the prevalence of Russian olive, on nest success for all bird species combined. These preliminary findings demonstrate the complexity of the relationship between birds and their nesting habitat and offer partial support for the idea that invasive tree species can negatively influence nesting success for riparian bird species.}

\normaltalk{\capitalisewords{Spatiotemporal Variation of Diurnal Raptors in a Dry Forest of Western Mexico}}{\underline{Marisela Mart\'{i}nez-Ruiz}, Miguel A De Labra-Hern\'{a}ndez, Katherine Renton}{Birds in seasonal environments cope with marked phenological changes over the year, and have to deal with additional challenges of extreme climatic events such as hurricanes. Little is known regarding the response of raptors to seasonality or to hurricane disturbance. We evaluated spatiotemporal variation in diversity of diurnal raptors in the tropical dry forest of Mexico. Over two years we conducted surveys for raptors in 29 transects representing 4 habitats of the dry forest. Transects were distributed within and outside the eyewall of maximum winds along the path of major Hurricane Patricia. Seasonality influenced raptor diversity, both within and outside the hurricane eyewall, with higher diversity during the dry season than the rainy season. The more temporally variable habitats of deciduous forest and agricultural fields showed significant temporal variation in raptor density, both in areas affected and unaffected by hurricane winds. Alternatively, evergreen habitats of wetlands and semi-deciduous forest only showed temporal variation in raptor density in areas affected by maximum hurricane winds, which may be a reflection of damage to larger trees and the low resilience of these habitats to hurricanes. Analysis of Similarities found high dissimilarity in species composition among habitats in areas unaffected by maximum hurricane winds, both in the dry and rainy season. However, in areas affected by hurricane winds, raptor communities were only significantly dissimilar among habitats during the dry season. Our results show that hurricane disturbance may alter dynamics of spatiotemporal variation in top-predator raptor communities with potential cascading effects in tropical dry forest systems.}

\normaltalk{\capitalisewords{What's New at the BBL? Preparing for the Next Century of Bird Banding}}{Jennifer S Malpass, Bruce Peterjohn, \underline{Antonio Celis-Murillo}}{The USGS Bird Banding Lab (BBL) has been a leader of the North American Bird Banding Program since the inception of the Migratory Bird Treaty Act in 1918. BBL operations have adapted over the past century to address changes in bird banding practices and availability of new data management processes. Today the BBL is initiating major revisions to better serve the needs of contemporary bird banders and users of bird banding data. The BBL is redesigning Reportband.gov, the website where members of the public can report encounters of banded birds. The BBL is also preparing to transition its banding data submission software (BANDIT) to a web-based platform that will allow banders to directly connect with the BBL database. The BBL is exploring the development of mobile data entry programs to allow bird banders to submit data from the field. In addition, the revised BBL website will allow bird watchers, bird banders, and data users access to a variety of self-serve features that were previously only available by contacting staff directly. These projects offer rewarding opportunities to leverage expertise from local universities and others through novel collaborations, streamline operations to reduce staff workload, and modernize practices to facilitate the next century of bird banding.}

\normaltalk{\capitalisewords{Seaside Sparrow Nest Survival Following the 2010 Deepwater Horizon Oil Spill}}{Megan E Hart, Anna Perez-Umphrey, Philip C Stouffer, Sabrina S Taylor, Christine Bergeon Burns, Andrea Bonisoli Alquati, \underline{Stefan Woltmann}}{In 2010, the Deepwater Horizon oil spill released an estimated 4.9 million barrels of oil into the Gulf of Mexico, damaging coastal ecosystems. The Seaside Sparrow (Ammospiza maritima) is a year-round resident of Louisiana saltmarshes, which were severely impacted by the spill. We monitored Seaside Sparrow nests in oiled and unoiled plots in Plaquemines Parish, Louisiana from 2012-2017 to assess possible impacts on Seaside Sparrow reproductive success. We found lower daily nest survival rates (DSR) and more-frequent nest failure on oiled sites across all years. The lowest DSR (±SE) was on oiled sites in 2012 (0.87 ± 0.02) and the highest was on unoiled sites in 2015 (0.94 ± 0.01). Significant annual variation in DSR within our study system was independent of oiling status. Study site characteristics, including southern exposure (southern exposed plots were more likely to be oiled), nest predator communities, and vegetation structure and composition also differed between oiled and unoiled plots, which likely influenced nest site selection and DSR. Although the DWH spill impacted other aspects of Seaside Sparrow ecology, we found no definitive effect of the spill on nest survival. Confounding effects of storms, site locations, and high annual variability in local nest survival rates all highlight the need for continued and long-term monitoring.}

\normaltalk{\capitalisewords{Tracing Nutrient Sources to Lipid Production in a Passerine using Stable Isotope (δ13C, δ2H) Tracers}}{\underline{Libesha / Anparasan}, Keith A Hobson}{Using stable isotope measurements of avian tissues to determine origin and migratory patterns is well established. However, isotopically determining nutritional origins of lipids, the primary fuel of migration, has not been attempted. We explored isotopic links between diet and stored lipids in captive White-throated Sparrows (Zonotrichia albicollis) using δ13C and δ2H measurements. Sparrows were randomly separated into four groups (n=6) and fed isotopically distinct diets, consisting of a combination of high δ13C (C4 starch), or low δ13C (C3 starch) diet, with a high δ2H (deuterium spiked), or low δ2H (tap) drinking water with a constant casein protein source. After 4 months of feeding, lipids were extracted for analysis and isotopic values of lipids and dietary components measured using CF-IRMS. The carbohydrate contributed to approximately 80\% of the δ13C of the lipid and the water contributed to approximately 18\% of the δ2H of the lipids. C4 carbohydrate fed bird lipids had higher δ13C values compared to C3 carbohydrate fed birds (-29.1 vs -16.6 ‰, P$<$0.01). Similarly lipids from birds provided with spiked water had higher δ2H values compared to birds given tap water (-187.3 vs -254.6 ‰, P$<$0.01). This indicates that lipids isotopically reflect dietary sources in a consistent manner. We derived isotopic discrimination factors linking environmental isotopic sources with body lipid for δ13C which can now be used to provide continental isoscapes for tracing sources of stored lipids. These findings act as the preliminary step to using lipid isotopes a tool for the conversation and management of migratory passerines.}

\normaltalk{\capitalisewords{Genomic and Phenotypic Variation in Russet Nightingale-Thrushes}}{\underline{David A Mosack}, James M Maley, Whitney LE Tsai, John E McCormack}{Taxonomic classification of the Russet Nightingale-Thrush (Catharus occidentalis) and close relatives has been historically prone to errors and revision due to phenotypic similarities between allopatric populations. Currently, there are four recognized subspecies of Russet Nightingale-Thrush, each of which with its own loosely defined range. In this study, we first applied modern statistical techniques to assess the distinctness of populations by subspecies using morphometrics, and then further investigated the question of distinctness using genomic analyses of Ultraconserved Elements and mitochondrial DNA. While morphometric analysis indicated no significant differences between specimens when grouped by subspecies, phylogenetic analyses of mtDNA showed two major clades: one primarily consisting of specimens from the Sierra Madre and Transvolcanic ranges, and one primarily of specimens from the states of Oaxaca and Guerrero (states respectively representing the ranges of C. o. occidentalis and C. o. fulvescens). This suggests the possibility of a population in the southeast Sierra Madre del Sur and Transvolcanic ranges that is genetically distinct from populations farther to the north, which does not reflect current subspecific taxonomy. Further study of Russet Nightingale-Thrush genomics, plumage coloration, and vocalizations is necessary to adequately address long-standing taxonomic difficulties in this group.}

\normaltalk{\capitalisewords{Adaptation to the Arctic: Community Genomics of Alaskan Galliforms}}{\underline{Sarah A Sonsthagen}, Robert E Wilson, Robert S Cornman, Sandra L Talbot}{Alaskan resident species are specialized for life in the cool summer and extremely cold winter environments characteristic of the Arctic. The ability of these species to thrive in Arctic landscapes likely requires novel adaptions at genes underling response to environmental signaling.  Given their year-round occupancy and specialization to sub-Arctic and Arctic ecosystems, resident species such as ptarmigan and grouse may be more (or differentially) impacted by the challenges faced with environmental change than are migratory species.  We assayed ptarmigan (rock, willow, and white-tailed) and grouse (spruce, ruff, and sharp-tailed) species sampled along a latitudinal gradient (60–70°N)  from temperate to sub-arctic to Arctic Alaska at genes (~3000 exons across 417 genes) associated with pathways (e.g., physiological, environmental signaling, behavioral, and immunological) that may be involved in species' responses to life in the Arctic. Our goal was to identify gene variants that may have arisen as a result of adaptation to the Arctic as well and to determine the strength of environmental selection and the ability of Arctic populations of gallinaceous birds to respond to a changing Arctic conditions.  This functional genomics approach targeting a suite of resident avian species will increase our understanding of species diversification and interaction, hybridization, population genetic dynamics, and functional adaptation within the context of accelerating environmental change.  As such, this approach provides a much-needed perspective on the ability of these species to respond to changing environmental conditions, and ultimately persistence and health.}

\normaltalk{\capitalisewords{The Role of Geographical Barriers in Shaping Amazonian Bird Distributions}}{\underline{Marco A Rego}, Glaucia C Del-Rio, Robb T Brumfield}{In the Amazon region, rivers function as barriers separating closely related terrestrial vertebrate species. Although this pattern has been described for a long time, few studies, most of them on birds, explored how river barriers differ in their efficiency in isolating species. The Amazon forest is still poorly studied, and the lack of taxonomic revisions affects the knowledge on bird species distributions. Here we gathered and georeferenced 450,000 museum specimen records to build a comprehensive bird database. We aimed to look for common distributional patterns and quantify species turnovers across the Amazon Basin Suboscines. Our goal was to describe different turnover patterns and which environmental features influence distributional breaks across the basin. We used concave hull polygons and point interpolation methods to generate distributional ranges and extracted distributional edges for each taxon. We overlapped edges of each species complex and stacked turnover regions across all species complexes to detected where turnovers clusters. We found that (1) overall, river headwaters show fewer species turnovers than the lower river portions; (2) rivers with higher numbers of turnovers have their headwaters outside evergreen Forest, in areas such as the high Andes and the Cerrado region; (3) the lower portion of the Amazon river shows fewer turnovers when compared with its middle portion. Our results suggest that Amazonian rivers differ considerably in their effectiveness in separating sister taxa, and that even the same river might have different levels of isolation potential.}

\normaltalk{\capitalisewords{Geographic Associations with Anthropogenic Noise Pollution for North American Breeding Birds}}{\underline{Christopher A Lepczyk}, Brian T Klingbeil, Frank A La Sorte, Daniel Fink, Curtis H Flather}{Anthropogenic noise pollution (ANP) is a globally invasive phenomenon, yet most research has occurred at local scales with few species. We investigated continental-scale breeding season associations with ANP for 323 bird species to test if small-scale predictions related to breeding habitat, migratory behavior, body mass, and vocal traits are consistent at broad spatial extents for an extensive group of species. We calculated each species' breeding season association with ANP using spatiotemporal exploratory models that estimate weekly probabilities of occurrence of bird species from eBird and spatially explicit estimates of ANP. We evaluated how species' breeding season associations with ANP were related to mean body mass and if it differed among species based on migratory behavior or breeding habitat affinity. For 64 species, we used vocal trait data to evaluate differences in song duration, pitch, and complexity based on body mass, ANP, and habitat association. Neotropical migrants were associated with significantly lower ANP than residents and species that breed in anthropogenic environments are associated with twice the level of ANP as species breeding in forested habitats. Songs were less complex among forest breeding species than nonforest breeding species and song pitch had a negative relationship with ANP whereas song complexity had a positive relationship with ANP. ANP is an important factor shaping breeding bird distributions. Vocal traits play an important role in understanding the factors that affect sensitivity to ANP for groups of species but differences between relationships observed at local and continental scales complicate predicting long-term consequences of ANP exposure.}

\normaltalk{\capitalisewords{Body Size and Telomere Dynamics in House Sparrows}}{\underline{Anuj A Ghimire}, Jennifer T Vangorder-Braid, Aubrey E Sirman, Rebecca C Young, David Westneat, Britt J Heidinger}{Within in species, growth rate and body size are often negatively correlated with longevity, but the underlying mechanisms are not well understood. Telomeres are one mechanism that may be important in underlying this link. Telomeres form protective caps at chromosome ends that enhance genome integrity, but shorten in response to cell division and stress exposure. Here we tested the hypothesis that individuals that grow more quickly experience greater telomere loss and that these effects are exacerbated under stressful conditions. To test this hypothesis we experimentally manipulated stress exposure in house sparrow chicks during development and examined the effects on growth rate, change in telomere length and body size. Two days after hatching, chicks within nests (n=33) were randomly assigned to either an experimental stress or control treatment. Between days 2 and 10 post-hatching, chicks in the experimental stress treatment were exposed to a standardized handling restraint stressors and chicks in the control treatment remained undisturbed. Chicks were blood sampled on days 2 and 10 post-hatching and measured on days 2, 6, and 10 post-hatching to examine the effects of treatment on growth, changes in telomere length, and body size. Results will be discussed within the context of life-history theory.}

\normaltalk{\capitalisewords{Space Use by Breeding and Non-Breeding Whooping Cranes During the Breeding Season}}{\underline{John A Conkin}, Mark T Bidwell}{Knowledge of space and habitat use by endangered species is required to protect areas to support their recovery. Critical habitat (CH) was partially identified under Canada's Species at Risk Act for the last remaining, wild and self-sustaining population of whooping cranes (Grus americana) which breeds in and near Wood Buffalo National Park, Canada. However, existing CH does not correspond well to the population's current breeding range which expanded ~300 times since 1966 with up to 35\% now occurring outside of CH. Moreover, previous home range estimates of breeders were based on resightings of unmarked individuals in different years, and space use by non-breeders has never been described or used to inform designation of CH. From 2012 to 2018, we monitored known breeding (N=10) vs. non-breeding (N=13) adult and subadult cranes fitted with GPS tracking devices to quantify movement and space use during the breeding season. We used continuous time movement models to determine if each crane had a defined home range during one or more breeding seasons, and to estimate home range size and configuration. Typically, confirmed breeders had well-defined home ranges and non-breeders displayed more nomadic behaviour, but some breeders shifted their home ranges during the breeding season, apparently corresponding to the periods of incubation and following hatching and fledging. Results will improve knowledge of space and habitat use by cranes during the breeding season, and may inform future designation of CH for whooping cranes, e.g., by accommodating seasonally-shifting home ranges or incorporating requirements of non-breeders.}

\normaltalk{\capitalisewords{Correlating Body Condition and Habitat Management of Wintering Duck Populations in Texas Wetlands}}{\underline{Allison A Guggenheimer}, Jacquelyn K Grace}{Texas is a critical stopover and wintering location for waterfowl. Poor body condition on wintering grounds can have long-term impacts on populations via survival and later breeding success. My study investigates the factors affecting body condition in green-winged teal (Anas carolinensis), blue-winged teal (Anas discors), northern pintail (Anas acuta), gadwall (Anas strepera), and northern shoveler (Anas clypeata). I hypothesize that (1) ducks at highly managed native wetland habitats will be in the highest body condition; (2) body condition trends throughout the winter (November-January) will be species, sex, and age-dependent, but will generally be stable until the spring equinox in March when mass will increase for migration; and (3) ducks collected after extreme weather events will be in lower body condition than those collected after a milder season. Using close to 2000 bodies salvaged from hunters from 2017 to 2019 I have analyzed morphometric measurements, mass, and gizzard fat to form a picture of body condition across time. I compared these measurements to land and water management techniques as well as habitat variables for bird collection sites. These results have the potential to impact land and water management practices on waterfowl wintering grounds, providing benefits to both land managers, scientists, and waterfowl populations.}

\normaltalk{\capitalisewords{Using GPS-Tracking to Fill Knowledge Gaps in the Full Annual Cycle of an Elusive Aerial Insectivore in Steep Decline}}{\underline{Aaron A Skinner}, Jay Wright, Stephen N Matthews, Christopher M Tonra}{Migratory insectivorous birds have seen some of the largest declines of any avian group in North America. The Eastern whip-poor-will (hereafter: whip-poor-will) is no exception: it has declined nearly 70\% over the last 50 years, but the reasons behind these declines are largely unknown. Whip-poor-wills spend roughly two-thirds of the annual cycle in migration and on the winter grounds, yet these periods have been little studied, limiting our ability to design effective conservation strategies. Whip-poor-wills winter primarily in Central America, a region that has experienced high rates of agricultural expansion and intensification in a time period concurrent with their declines. Agriculture adversely affects insect abundance and alters insect population composition, as well as causes habitat loss and landscape fragmentation, thus, implicating agricultural practices as a possible factor in their decline. In this study, we recovered high-resolution GPS tags from 11 Whip-poor-wills from two sites in Ohio. We identified important wintering areas and the migratory routes they took, illuminating a poorly known aspect of their annual cycle. In addition, we combined spatial data from the tags with available landcover data to examine habitat utilization, and determine how winter habitat characteristics affect Whip-poor-will winter home range size and overwinter trophic position. This work will fill important knowledge gaps regarding the Whip-poor-will annual cycle, informing future conservation plans for this once common bird as well as for other migratory insectivores. This data will supplement a similar study to be deployed in 2019 and 2020, more than quadrupling the number of tags deployed.}

\normaltalk{\capitalisewords{Consequences of Infection by Avian Malaria Parasites in the Urban Forests of Puerto Rico}}{\underline{Nicole A Gutierrez-Ramos}, Miguel A Acevedo}{Virulence – the negative effect caused to the host by a parasite – is viewed as an unavoidable cost of infection because the parasite reproduces at the expense of host resources.  Further analysis has shown that virulence depends on multiple traits such as host or parasite life history and environmental interactions. A major variable for health assessment is body condition which can decrease because of both infection and living in sub-optimal habitats such as urban areas.  Avian malaria is widespread in optimal and sub-optimal habitats and although fatal in many cases, there is no consensus of the overall effect on bird health.  The purpose of this study is to test the consequences of infection by avian malaria parasites through body condition in birds of urban forests of Puerto Rico. We analyzed avian malaria parasite infection status and morphological measurements of Bananaquits (Coereba flaveola) and Black-faced grassquits (Tiaris bicolor) – two common species in the urban forest areas of Puerto Rico.  After obtaining a body condition index through a principal component analysis (PCA) and using infection status as an independent variable, we found no statistically significant difference between infected and non-infected birds (p=0.09).  These results suggest tolerance to infection in C.flaveola and T.bicolor because overall health through body condition is similar between infected and non-infected birds. Our study provides evidence of a lack of negative consequences by avian malaria parasite infection in urban birds of Puerto Rico and supports the idea that virulence is variable depending on multiple host, parasite and environmental traits.}

\normaltalk{\capitalisewords{New Primers for PCR Detection and Differential DNA Amplification of Avian Haemosporidian Parasites Belonging to Different Genera}}{\underline{Maria A Pacheco}, Axl S Cepeda S Cepeda, Rasa Bernotienė, Nubia E Matta, Gediminas Valkiūnas, Ananias A Escalante}{Avian haemosporidians are a diverse group of vector-borne parasitic protozoa found in birds worldwide. There is an increasing interest in these parasites as their speciation relate with the diversification and ecology of birds. Furthermore, there are knowledge gaps on their potential impact on avian populations' dynamics with the exception being a few iconic cases where these parasites have diminished avian populations. Thus, the detection and identification of these parasites is an expanding area of research. Most studies rely on molecular methods targeting a small fragment of the parasite cytochrome b (cytb). However, the methods used often cannot separate different genera found in mixed infections that are common in wildlife. Using an alignment of 114 mitochondrial genome sequences belonging to several Haemosporidia genera, two different sets of primers targeting mtDNA genes were designed, one set for the detection all parasites and another for differential amplification of parasite DNA of different haemosporidian genera. All pairs of primers were tested in three laboratories. Overall, these primers exhibited high sensitivity regardless of the differences in laboratory practices, parasite species, and parasitemias. Primers designed to separate parasite genera showed high specificity as confirmed by sequencing.  A nested-multiplex (single-tube PCR) assay to differentially detect lineages of Plasmodium and Haemoproteus parasites was designed and successfully tested by yielding amplicons with different sizes detectable in a standard agarose gel. The designed assay is the first test for the detection and differentiation of species belonging to these two genera that can be used in avian parasites population studies.}

\normaltalk{\capitalisewords{Florivory as an Opportunistic but Essential Foraging Strategy for Avian Specialists}}{\underline{Emily A Webb}}{Opportunistic foraging is an underappreciated component of the avian specialist diet. There are many examples of non-standard foraging strategies that occur seasonally or in eruptions that are just as important to the diets of birds as their standard fare. An interesting ecophysiological question to consider is why birds do this and what implications this has for their physiology and ecology of the plants and animals on which they are opportunistically foraging. Perhaps the most understudied but ecologically important example of this is flower-eating behavior (i.e. florivory). The existence of florivory as a foraging strategy is taxonomically widespread and is supported with mostly anecdotal but some peer-reviewed evidence, and it has important implications for the dietary physiology of birds as well as the fitness of flowering plants. Here I present data on the trends of anecdotal and primary literature-based florivory studies across taxa as well as a few avian case studies. I also examine several hypotheses to explain these behaviors including macronutrient compensation, digestibility (e.g. sucrose, secondary compounds), and unintentional consumption.}

\normaltalk{\capitalisewords{Comparing Approaches for Modeling Bird Distributions from Imbalanced and Spatially-Biased Data}}{\underline{Valerie A Steen}, Morgan Tingley, Chris Elphick, Peter Paton, Charles Clarkson}{Spatially-biased species occurrence data are a feature of low-structure citizen science datasets. Spatial thinning can mitigate errors in species distribution models (SDMs) using these data. However, when one occurrence class is rare due to a low frequency of detections or non-detections, SDMs may suffer from class imbalance. This results in predictions with poor ability to discriminate between locations with presence and absence. We explored the alternate and combined effects of balancing classes and spatially thinning eBird citizen science data. We created four types of eBird training datasets for rare to common breeding birds in the Northeastern U.S.: classes balanced with only the majority class thinned, both classes balanced and thinned, both classes thinned without balancing, and all data without thinning or balancing. We tested SDMs created from these training datasets on an independent, systematically collected large and spatially comprehensive dataset combining on and off-road point counts from CT and RI Breeding Bird Atlas projects. We evaluated predictive performance of the different approaches with respect to species prevalence. This evaluation will help guide future predictive modeling efforts using spatially-biased occurrence data collections when species are rare or common.}

\normaltalk{\capitalisewords{The Risks and Benefits of an Emerging Aquatic Insect Diet: Methylmercury in Eastern Phoebe (Sayornis Phoebe) Chicks}}{\underline{Sarah A Dzielski}, Cornelia W Twining, Lisa B Cleckner, N R Razavi}{Mercury (Hg) bioaccumulation in birds is greatly influenced by their diets. Previous studies showed that individuals feeding at higher trophic levels have higher concentrations of methylmercury (MeHg) in their tissues than species feeding at lower trophic levels. Additionally, since the conversion of inorganic Hg to MeHg occurs in aquatic environments, organisms with aquatic diets are at a higher risk of MeHg contamination than terrestrial organisms. Despite this risk, emergent aquatic insects are a high quality food source for insectivorous birds due to their high omega-3 highly unsaturated fatty acid (HUFA) content and were linked to increased breeding success in riparian species. In this study, we examine MeHg concentrations in Eastern Phoebe (Sayornis Phoebe) chicks and their invertebrate prey to explore the trade-off between MeHg content and nutritional quality of terrestrial and aquatic food sources. We predict that 1. phoebe chicks whose diets contain more aquatic insects will have higher MeHg feather concentrations than phoebe chicks with more terrestrial diets and 2. aquatic insects contain more MeHg than terrestrial prey. As aquatic invertebrates were nearly ten times higher in HUFAs than terrestrial prey at our sampling locations, this study provides an assessment of the risks and benefits to wildlife of relying upon aquatic food sources.}

\normaltalk{\capitalisewords{Habitat Associations of Migrant Waterbirds using Restored Shallow Lakes in Iowa}}{\underline{Rachel A Vanausdall}, Stephen J Dinsmore}{Wetlands in the Prairie Pothole Region (PPR) have experienced dramatic declines in areal cover, and many wetlands that remain are degraded. Recently, restoration of wetlands has been a key management strategy for wildlife, particularly for waterbirds. In Iowa, the Department of Natural Resources and Ducks Unlimited, Inc. have restored 38 shallow lakes, which are large, mostly permanent wetlands. To assess the impact of habitat variables and years-since-restoration on migrant waterbird use, we conducted surveys at shallow lakes restored 1 to 12 years prior to this study and soon-to-be-restored shallow lakes in Iowa during the spring 2016-2018. We found that emergent cover influenced diving ducks and species richness, with the greatest numbers occurring around 40-50\% emergent cover. Water level change negatively influenced dabbling ducks (β = -0.23, SE = 0.10) and positively influenced diving (β = 0.27, SE = 0.13) ducks and total waterbird abundance (β = 0.69, SE = 0.22). Years-since-restoration influenced goose/swan abundance (β = 0.21, SE = 0.05) and secretive marsh bird abundance (β = 0.10, SE = 0.03), and total wetland area showed a positive effect on several groups. These findings emphasize how management strategies for restored shallow lakes should mimic the natural wet-dry cycle of prairie wetlands to provide benefits for all waterbirds.}

\normaltalk{\capitalisewords{Detection of Rain and Insects in Automatic Passive Recordings using Random Forest Tree Classification}}{\underline{Oscar A Ram\'{i}rez Al\'{a}n}, Victor Colino Rabanal, David Segura Sequeira, Monica Retamosa Izaguirre}{The use of passive automated recorders has allowed the generation of a variety of information about the soundscape of the environments, helping us identify the different acoustic patterns and processes in a landscape. This methodology acquires a significant importance since it is not invasive and the information collected can be used to obtain species' richness, diversity and soundscape indexes, and collections of sounds produced by animals, among others. In nature there are several factors that affect the sounds produced by the biological communities. One of the biggest problems, especially in tropical rainforests, has been the presence of insects and rains which mask the sounds of the landscape. Usually the presence of this factors becomes undesirable information for classification issues. This study focuses on generating a prediction model of rainfall and insects using random forest trees using a subset of the detected event as signals of interests (training model) on acoustic data. Additionally, we present an R package called "Sinax", containing 13 functions which can be used to implement a soundscape analysis such as ACI (Acoustic Complexity Index), ADI (Acoustic\_Diversity Index), BIO (Bioacoustic index), NDSI (Normalized Difference Soundscape Index) and others. The package is designed to handle and analyze a large amount of data with a single workflow."}

\normaltalk{\capitalisewords{How Landscape Structure and Habitat Change Affects the Abundance and Distribution of Wintering Neotropical Migratory Shorebirds}}{\underline{John A Herbert}, Grant Gunn, David Mizrahi, Caz Taylor}{Wildlife populations are reliant on suitable habitat availability over time, as such, habitat change continues to be one of the leading factors of global population declines. The Guiana coast in northern South America has extensive coastal mudflats, providing valuable habitat for shorebirds and other waterbirds, but is subject to dynamic changes among years. Aerial survey data suggests significant declines of winter shorebird populations in the Guiana coast from the 1980s to the 2010s. Here we are investigating whether long-term coastal habitat change in South America is having an effect on winter populations of neotropical migratory shorebirds. Using remotely sensed imagery and newly developed techniques in Google Earth Engine, we are measuring land cover change over 34 years along the Guiana coast. Then using repeated aerial surveys spanning 30-years, we are investigating how habitat change and land cover affects shorebird distributions. The way in which the coastal dynamics affect wildlife populations is unknown, so the historical quantification of coastal landscape structure and how it affects shorebird populations will provide valuable insight to how coastal species are affected by dynamic habitat change.}

\normaltalk{\capitalisewords{Influences of Multi-Scale Habitat Quality and Female Behavior on Nest Success of Eastern Wild Turkeys}}{\underline{Matthew B Gonnerman}, Stephanie Shea, Pauline Kamath, Kelsey Sullivan, Erik Blomberg}{Complete evaluation of habitat quality requires understanding both the characteristics of habitat that individuals select or avoid, and how that selection affects individual fitness. To better assess nesting habitat quality for eastern wild turkey (Meleagris gallopavo silvestris), I will create a multi-scale predictive model that integrates models of nest site selection with those of nest success. Nest success may also be affected by heterogeneity in reproductive behavior of individuals, e.g. timing of movements or distance traveled to nesting range. To identify sources of heterogeneity in nest success, I will create models of nest daily survival rate (DSR) that account for individual variation caused by differences in hen movement behavior prior to nesting. During 2018 I deployed 12 GPS and 42 VHF transmitters on hens across four study areas representing major land use types found in Maine; agricultural, forested, and suburban/urban. Estimates of DSR were produced using nest monitoring data in the RMark package in program R. Seasonal home ranges and movement between them were quantified using data from GPS marked hens within a dynamic Brownian bridge movement model framework. Over the first year of research, I observed 31 nests with a 0.268 (0.076–0.504, 95\% CI) probability of surviving to 38 days after initiation. Based on GPS-marked birds, the average seasonal movement distance between the wintering and nesting range was 4.26km, and individual female movements ranged from 1.65km to 9.08km. Following additional data collection, I will incorporate information on landcover and individual movements into estimates of selection and success.}

\normaltalk{\capitalisewords{Great Gray Owl Home Range and Habitat Selection During the Breeding Season}}{\underline{Katherine B Gura}, Bryan Bedrosian, Susan Patla, Anna Chalfoun}{Habitat change is increasing across landscapes, with largely unknown consequences for under-studied raptor species.  Identifying the resource requirements of such species during key stages such as breeding is therefore critical for effective management. Throughout the Rocky Mountains, older-aged montane and sub-alpine forests are changing rapidly due to wildfire, disease and beetle outbreaks, drought, climate change, logging and development.  Great Gray Owls (Strix nebulosa) are associated with older-aged, boreal forest habitats, and studies conducted outside of the Rocky Mountains suggest that they respond negatively to the loss of key habitat elements. We are quantifying breeding-season home-range attributes and habitat preferences of adult Great Gray Owls across multiple spatial (home-range and site-level) and temporal (nesting and post-fledging; day versus night) scales in western Wyoming, USA.  In 2018 we outfitted adult owls (n = 11) with GPS remote-download transmitters and collected location data throughout the breeding season (1 May – 15 September).  Using 50\% and 95\% kernel density estimates (KDE), mean core area was 4.21km2 (SE = 1.40) and mean home-range size was 22.72km2 (SE = 8.30).  Mean core and home-range areas using dynamic Brownian bridge analyses were 1.26km2 (SE = 0.28) and 8.02km2 (SE = 1.64), respectively.  Resource selection analyses will incorporate both remotely-sensed and ground-based habitat data.  We conducted on-the-ground habitat surveys at used and available points within 95\% KDE home ranges using a stratified random sample design (n=398).  In 2018, only two of our study animals successfully fledged young, so 2018 data reflect a year when recruitment was low.}

\normaltalk{\capitalisewords{The Relationship Between Morphological Change Across Ecological and Macroevolutionary Timescales in Migratory Birds}}{\underline{Brian C Weeks}, Benjamin M Winger}{Understanding how macroevolutionary patterns of morphological change relate to speciation is a fundamental goal of evolutionary biology. Often, to characterize morphological change within clades, traits are mapped onto phylogenies and ancestral trait states are estimated using models of trait evolution. These large-scale patterns of trait evolution are then interpreted as indicative of emergent micro-evolutionary processes; however, evidence connecting morphological change across ecological and macroevolutionary scales is limited, which precludes a mechanistic understanding of how morphological change may relate to evolutionary or ecological dynamics. Here, we use a dataset of 47,876 specimens of migratory birds that died following collisions with buildings in Chicago IL from 1981-2016 to estimate rates of change and rates of adaptation to local environmental conditions for 52 species. We explore the phylogenetic signal in these rates of change, and link decadal-scale morphological change with macroevolutionary estimates of trait disparification for the clades that contain these focal species. We hypothesized that short-term rates of change in morphology would be correlated with macroevolutionary rates of disparification among clades, and that these rates of change would in turn be associated with increased rates of speciation. Our results, however, reveal a complex relationship between morphological change on ecological scales (i.e., over decades) and macroevolutionary morphological dynamics. These findings speak broadly to the complexity of how microevolutionary processes scale to produce macroevolutionary patterns.}

\normaltalk{\capitalisewords{Intergenerational Stress in Parents and Nestling and Telomere Lengths in House Sparrows}}{\underline{Rebecca C Young}, Aubrey Sirman, Jen Vangorder-Braid, Jeffrey Kittilson, Anuj Ghimire, David F Westneat, Britt J Heidinger}{Theoretically, adaptive parental effects may result in offspring that are better prepared for specific environmental conditions or adapt to them more quickly.  Thus, parental stress may affect offspring negatively or make them more prepared for stressful conditions.  Yet we still know little about the underlying mechanisms that produce cross-generational effects.  Telomere lengths are a good candidate for cross-generational effects because they're a chromosomal DNA structure but also associated with fitness in a variety of avian systems.  Mechanistically, telomere length may degrade in response to hormonal and oxidative stress, thus accurately tracking individual quality.  These lengths or loss rates may then predict survival or reproductive output.   We stressed adult house sparrows during the pre-breeding period and measured telomere lengths in the nestlings at day 2.  Then nestlings were allocated to either a stress (handling) or control (not handled) treatment (fully blocked) and re-sampled at day 10.  Telomere lengths were measured with PCR.  Preliminary data indicate that nestlings with stressed parents may have had shorter telomeres at day 2, but further analysis will determine if parental stress treatment interacts with chick treatments.}

\normaltalk{\capitalisewords{Structural Perspectives on the Evolution of Personality}}{\underline{Kathryn C Chenard}, Renee Duckworth}{Personality traits are behaviors that are expressed differently among individuals in a population, and which are stable over time and across contexts. These axes of variation are found in species throughout the animal kingdom, from birds to insects to mollusks, despite separation by wide gulfs of evolutionary history and neurological organization. However, neurological tissue itself represents a common link that is limited in the variation it can produce. Brain structure is a relatively stable trait which has been found to covary with variation in personality among individuals. Similarities in the axes of modular variation of brain regions may be behind the ubiquity of personality traits. Covariation between brain regions may then become coupled and uncoupled as guided by selection and developmental processes, mirroring variation in behavioral traits in a population. This may be molded by 3 potential mechanisms: reinforcement of behavior over an individual's lifespan due to characterization and life-history, as a consequence of developmental environment, timing and allocation, and by selection for specific modules or clusters of modules with other aspects of phenotype. To test this, we compared variation in brain region volumes within different bird species, and behavior tested captive zebra finches to link brain structure to personality.}

\normaltalk{\capitalisewords{Migratory Ecology of American Woodcock (Scolopax Minor) in Eastern North America}}{\underline{Alexander C Fish}, Erik J Blomberg, Amber M Roth}{Migration can be the most dangerous time during the annual life cycle and may be responsible for declines in some species.  During migration birds must navigate an unfamiliar landscape, while simultaneously contending with anthropogenic structures, light pollution, changing land practices, and novel threats (e.g., feral cats).  The American woodcock (Scolopax minor) is a migratory forest bird experiencing long-term population decline throughout its breeding range, raising concern from numerous stakeholders.  We created the Eastern Woodcock Migratory Research Cooperative to deploy GPS satellite tags throughout the American woodcock's eastern range, with the ultimate goal to understand migratory phenology, survival, and stopover habitat selection.  We will deploy transmitters throughout both the breeding and wintering range to study fall and spring migration respectively.  As of February 2019 we deployed 80 tags in Maine, Maryland, New Jersey, New York, North Carolina, Ontario, Pennsylvania, Quebec, Rhode Island, South Carolina, and Virginia, monitoring woodcock during both spring and fall migration.  Woodcock initiated fall migration between October 17th and December 1st in 2018.  We observed single night flight distances in excess of 800 km and stopover durations as short as a single day, however both of these metrics were variable among individual birds.  We documented woodcock stopping over in forested, agricultural, industrial, suburban, and urban areas, highlighting the diverse landscapes use during migration.  Woodcock primarily overwintered in Gulf Coast and Southern Atlantic states.  We will continue to deploy tags and monitor migrating woodcock migration to building an integrated population model.}

\normaltalk{\capitalisewords{Stopover Biology of Grassland Birds at a Restored Tallgrass Prairie}}{\underline{Amy West C West}, David Swanson}{Migration is an energetically expensive but critical part of the annual cycle of migrant birds. Grassland birds are experiencing the greatest population declines of any bird guild, likely because of conversion of grassland habitats to agriculture. Consequently, grassland migrants may have limited options for migration stopover, but few studies have examined stopover biology for grassland birds. In addition, the simple occurrence of birds in grassland fragments during migration periods does not mean that the habitat is providing the food resources necessary for birds to replenish fat stores and continue migration. We captured birds by mist-net at Spirit Mound, a 130-ha restored tallgrass prairie in southeastern South Dakota, during spring and fall migration periods and conducted plasma metabolite profiling to determine triglyceride, an indicator of fat deposition, and beta-hydroxybutyrate, an indicator of fat depletion, levels. Preliminary results suggest that grassland, woodland/shrub, and habitat generalist bird species are all present on prairies during the migration seasons and that bird abundance is higher during fall than during spring. Moreover, birds appear to be successfully adding fat at this site, suggesting that restored tallgrass prairies can provide high-quality stopover habitat for grassland migrant birds. These findings have implications for creation and management of restored prairie habitats throughout grassland bird migratory pathways.}

\normaltalk{\capitalisewords{Testing for Population-Level Morphological Variation in Museum Collections of Clark's Nutcrackers}}{\underline{Caitlyn C Oliver Brown}, Benjamin E Carter}{Museum collections allow for studies of morphological evolution across both space and time. Museums provide easy and inexpensive access to large numbers of specimens from across the entire geographic range of a species and across considerable time periods. In this study, we took advantage of a unique population-level set of specimens to test whether morphological variation within a single population is representative across the entire geographic range of a species. The focal dataset is L.R. Mewaldt's collection of Clark's nutcrackers (Nucifraga columbiana), sampled exclusively from within a 43 km radius in western Montana over a four-year period, from 1946-1950. In this study, we collected morphological measurements from the 465 individuals in that set as well as an additional 300 specimens from throughout the range of N. columbiana for a total of 765 specimens. We tested whether the mean and variance of morphological factors differed between the focal population and the entire range, which would suggest population-level evolutionary changes, and also tested whether there were any temporal differences, which might suggest evolutionary change across the 120-year timespan represented in the collections.}

\normaltalk{\capitalisewords{Non-Breeding Time Budget and Diet of Loggerhead Shrikes in Northeastern Arkansas}}{Emily R Donahue, \underline{Lee C Bryant}, Jacob L Wessels, Joseph Youtz, Rhett Raibley, Than J Boves}{For wild organisms, energy is currency and must be spent efficiently to buy survival. For animals with demanding metabolic needs, such as birds, efficient energy use is especially important. Loggerhead Shrikes (hereinafter 'LOSH') are one avian species that must balance energy usage, particularly during the cold, non-growing season. Research suggests low over-winter survival may contribute to its decline. Thus, understanding LOSH daily energy use and diet may help pinpoint specific strains on populations. To improve our understanding of these important factors, we conducted behavioral surveys on non-breeding LOSH in northeastern Arkansas to characterize daily activities (including perching, foraging, caching, and evading); foraging attempt and success rates; perch, larder, and hunting substrates; and diet composition (from both surveys and larders). We determined how they partition their time (and energy) and identified prey items from which they obtain energy (diet). LOSH spent the majority of their time on a low energy activity, perching (80\%), with most of that time spent on utility wires (67\%). They spent only about 5\% of their time actively hunting (an energetically expensive activity) but had relatively high foraging success ($>$50\%). Despite the season, their diet primarily consisted of arthropods (85\%), with the addition of vertebrates such as anurans (13\%) and rodents (0.7\%), which suggests possible benefits LOSH may provide as natural pest control. This behavioral and diet analysis informs us not only of possible key habitat features and resources that could help LOSH survive the non-breeding season, but also elucidates the possible ecosystem services they may provide.}

\normaltalk{\capitalisewords{Characterizing Movements of Saltmarsh and Seaside Sparrows on Wintering Grounds using Radio Telemetry}}{\underline{Marae C Lindquist}, Evangelyn L Buckland, John P Carpenter, Dr Raymond M Danner}{There are large gaps in knowledge regarding the wintering populations of many marsh sparrows, including saltmarsh (Ammodramus caudacutus) and seaside (Ammodramus maritimus) sparrows. Both species are listed as Species of Greatest Conservation Need in the NC Wildlife Action Plan because they are losing essential marsh habitat due to climate change, sea level rise, development, and other anthropogenic forces. During the winters of 2019 and 2020 we are researching the winter population biology of saltmarsh and seaside sparrows on Masonboro Island National Estuarine Research Reserve in Southeastern North Carolina to estimate density, abundance, spaces use, survival, and migratory status using mark and recapture and radio telemetry. Our two hypotheses are: (1) saltmarsh and seaside sparrows will follow the tidal cycle, moving into low tide habitat to forage, and moving back into the upper marsh during high tide. (2) Saltmarsh and seaside sparrows will remain in the same home range throughout the non-breeding stationary period. We will be using radio telemetry to track the birds' daily movements. We are attaching radio tags to 20 birds each winter and tracking them using a null peak radio antenna system. We will combine our radio telemetry data with mark and recapture data to to estimate density and abundance at low tide and use known fate models to estimate survival. For the conclusion of this project we aim to provide pertinent information to aid with the conservation of both saltmarsh and seaside sparrows.}

\normaltalk{\capitalisewords{Post-Hurricane Assessment of Abundance and Diversity of Land Birds in Coastal Dry Forest at Jobos Bay}}{\underline{Mariangely C Col'{o}n-Cruz}, Ian J P\'{e}rez-Cruz, Fred C Schaffner}{On September 20, 2017, Hurricane Maria struck the island of Puerto Rico; just ten days after Hurricane Irma ravaged the Island's north coast. This storm had devastating effects on habitats throughout the island of Puerto Rico, including massive crown loss and defoliation of mangroves and coastal secondary dry forest at Jobos Bay National Estuarine Research Reserve (JBNERR). Through mist netting sampling of resident and migratory land birds in an interhabitat corridor connecting mangroves with coastal dry forest, and a second site a mesquite (Prosopis pallida)-dominated dry forest from 2017 to 2019 we present an assessment post hurricane recovery of land bird abundance and diversity at this site.  We found reduced total species abundance and diversity, a slow increase in small granivores such as Common Ground Doves (Columbina passerina) and Black-faced Grassquits (Tiaris bicolor); a nearly complete absence of hummingbirds, as well as a decrease in Northern Waterthrushes (Parkesia noveboracensis); but an apparent increase in birds of open habitats including Grey Kingbirds (Tyrannus dominicensis), Northern Mockingbirds (Mimus polyglottos), and Eurasian Collard Doves (Streptopelia decaocto).}

\normaltalk{\capitalisewords{Sex-Specific Demography of a Monomorphic Passerine in Southern Louisiana}}{\underline{Elizabeth C Heintz}, Eric J Tobin, Scott M Duke-Sylvester}{In many species of bird, sex can be determined using plumage dimorphism between males and females. This is not the case for species such as the Carolina Wren (Thryothorus ludovcianus), which appear monomorphic under visible light. As sex has been shown to affect survivorship in a number of other species of birds, accurate sex determination is an integral part of estimating survivorship in this species. Shorthand field metrics, such as wing length or tail length or the presence of a brood patch in the breeding season, are used to ascribe a sex to an individual. However, these methods may not always be reliable due to temporal variability or overlap in the distribution of measurements. This can result in inaccurate sex determination or no sex determination. In this study, we use molecular techniques to determine the sex of individual Carolina Wrens, a resident bird species of Louisiana. We then incorporate sex as a covariate in a mark-recapture model to explore differences in survivorship between males and females. We also combine our molecular data with field measurements to refine the size range definitions for male and female Carolina Wrens in Louisiana.}

\normaltalk{\capitalisewords{Breeding Songbird Habitat Utilization in Urban Riparian Forests}}{\underline{Lana C Milbern}, Stephen N Matthews}{Urbanization can have profound influences shaping patterns of songbird diversity. For many species, urbanization poses considerable challenges yet for songbirds  urban areas provide important habitat, most notably in riparian urban forests. With renewed interest to maintain urban forest habitat, understanding the interactions between songbirds and their urban environments is critical to making informed land management decisions.  To explore the link between forests and birds, our study focuses on foraging behaviors of songbirds in restored and unrestored sections of urban riparian forests in Columbus, Ohio USA. The goals of our study are to understand how breeding urban songbirds utilize their floristic environment for foraging, identify foraging-related factors, such as food availability, and identify how forest restoration efforts will affect foraging ecology. We also aim to understand how vegetative complexity and urbanization may affect breeding songbird territory density. To answer these questions, we are conducting spot maps, foraging surveys, vegetation surveys, and arthropod counts at urban and rural sites from May to August in 2018 and 2019. Preliminary data show that insectivorous urban songbirds utilize black walnut for foraging approximately twice as much as expected and Amur honeysuckle about half as much as expected based on tree species density. Our data also show that Northern Cardinal and Blue Jay territory density increases with degree of urbanization, whereas Eastern Wood-Pewee and Acadian Flycatcher territory density decreases with urbanization. We hope that this project will help inform urban restoration work in respect to avian conservation.}

\normaltalk{\capitalisewords{Relationships Between Eastern Whip-Poor-will and Chuck-Will's-Widow Abundance and Landscape Composition and Management}}{\underline{Melissa C Roach}, Frank Thompson}{Eastern Whip-poor-will (Antrostomus vociferous) and Chuck-will's-widow (Antrostomus carolinensis) are nocturnal aerial insectivores that have experienced steep declines in abundance over the last 50 years across their ranges. There is little information on the effects of forest management practices on these species, in part, due to their secretive nature. However, increased efforts to restore savanna and woodland in the Eastern U.S. could potentially benefit these species. Our objective was to relate abundance of Eastern Whip-poor-will and Chuck-will's-widow to land cover, forest structure, and woodland restoration practices in the Missouri Ozarks. We conducted 385 nocturnal roadside point counts in the Ozark Highlands, Missouri during the 2014 and 2015 breeding seasons in areas with and without pine woodland restoration. We detected 375 Eastern Whip-poor-wills and 111 Chuck-will's-widows with a range of 0-5 and 0-2 individuals per point, respectively. We used Bayesian time-removal models in an information theoretic approach to evaluate the effects of management treatment, canopy cover, basal area, and forest cover. Abundances of Eastern Whip-poor-will and Chuck-will's-widow were positively related to the proportion of the landscape forested but negatively related to canopy cover in the forest. Abundance was also positively related to the proportion of the landscape treated by forest thinning. Pine woodland restoration appears to be benefiting these species of conservation concern in the Missouri Ozarks.}

\normaltalk{\capitalisewords{Engaging Incarcerated People in Avian Research and Conservation: The Sustainability in Prisons Project (SPP)}}{\underline{Philip C Fischer}, Jessica Brown, Kelli Bush, Teresa Lorenz}{Incarcerated people have been overlooked for their interest in avian science and a desire to make positive contributions in conservation. The Sustainability in Prisons Project (SPP) is a program designed to address these issues in Washington State.  It is a partnership founded by The Evergreen State College and Washington Department of Corrections (WADOC) that links incarcerated people, agency biologists, and academics. We will present two SPP partnership examples. The first example is partnership with the U.S. Forest Service (USFS) to identify threats to cavity nesting birds. For this project, incarcerated SPP technicians are trained in bird identification and behavior. They review $>$500 hours of camera-trap footage annually from video-monitored nests, which represents a threefold increase over undergraduate students engaged in this same project.  Results are used to provide feedback on USFS forest management projects in the northwestern U.S.  In the second example we will describe a partnership with the Washington Department of Fish and Wildlife where incarcerated technicians aid in the recovery of western pond turtle (Actinemys marmorata), an endangered species in Washington. This project has direct applicability to recovery of rare bird species. In this project, highly trained technicians provide intensive care to diseased turtles in a specialized, on-site facility. The level of care that turtles receive is not economically or logistically feasible elsewhere because holding facilities require continual cleanings, and turtles require intensive, long-term care to make a successful recovery.  Our primary objective is to raise awareness of this valuable, mutually-beneficial resource with avian researchers and managers.}

\normaltalk{\capitalisewords{Effect of Male Desertion on Provisioning, Growth, and Survival of Hooded Warbler Nestlings}}{\underline{William D Harrod}, Ronald L Mumme}{Parental care and molt are two taxing phases of the avian life cycle and trade-offs between the two occur frequently. The Hooded Warbler (Setophaga citrina) experiences these trade-offs when late-season parental care and molt overlap in time. Males regularly desert nestlings and fledglings during molt. However, it is unknown how females and young react to male desertion. We examined how male desertion affects nestling provisioning rate, growth rate, and survival in a color-banded population of Hooded Warblers in northwestern Pennsylvania. Our hypothesis was that females would be able to increase provisioning rate and compensate for the lack of a male. Provisioning data were obtained from 19 late-season nests that were monitored either from a blind or with a video camera. Nestling growth data were obtained from 14 late-season nests where nestlings were weighed every 2-3 days until they reached day 5. Nestling survival data were obtained from 21 late-season nests. In nests where males deserted, females significantly increased the rate at which they provisioned young. Nestling growth rate and survival were both similar between deserted and non-deserted nests. Our hypothesis was supported; male desertion had little or no effect on the growth and survival of nestlings because females were able to compensate and effectively provision young.}

\normaltalk{\capitalisewords{The Function and Mechanisms of Female Ornamentation in a Lekking Bird}}{\underline{Alexis D Earl}, Abby A Kimmitt, Ellen D Ketterson, Richard K Simpson, Jessica L Yorzinski}{The study of male ornamentation has been fundamental to progressing our understanding of evolution, yet we are only now beginning to consider the fascinating evolved traits of female animals. Although female ornamentation was once thought to be nonadaptive, existing simply due to residual expression of genes associated with traits of male conspecifics, recent studies have provided evidence for female ornamentation functioning in intrasexual competition and male mate choice in some monogamous species. In lekking birds, little is known about why females of some species are ornamented, while most are not. We investigated the function and mechanisms of female ornamentation in a lekking species in which females have ornamentation. Female peafowl, (peahens, pavo cristatus) have iridescent neck plumage. We captured 25 peahens and quantified luminance of ornamentation in the front, back and sides of their necks. To test whether ornamentation may signal position in the female social hierarchy, we scored dominance behaviors and calculated rank using David's Score to account for nonlinear relationships. We then compared rank to luminance. Our results suggest a significant relationship between rank and luminance of the back and sides of the neck, but not front of the neck. This is likely explained by female body orientation during aggressive intrasexual interactions, as well as intersexual mating interactions, both of which involve pecking at the side and back neck feathers. We also found positive relationships between luminance and baseline corticosterone, but not estradiol. This study provides insight into the evolution and significance of conspicuous female traits.}

\normaltalk{\capitalisewords{RFID Technology: Does Tagging Method Matter for Survivorship of Backyard Birds?}}{\underline{Rebecca D Garlinger}, Robert L Curry}{Radio frequency identification (RFID) technology aids in tracking avian behavior; however, little is known about how RFID tags affect survival. We investigated two RFID-tagging methods on chickadees: the first method (MT1) involved gluing passive integrated transponder (PIT) tags to color bands in the field, while the second (MT2) involved smaller PIT tags embedded in color bands. In one population (GM) we used MT1 to tag nearly all chickadees in 2012-2013. We did not PIT-tag GM birds from 1998-2011 or 2014-2015 and began PIT-tagging with MT2 in 2016-2019. For this population, we analyzed survivorship between PIT-tag and non-PIT-tag years. In a second chickadee population (NF) we used MT1 to PIT-tag ~ ½ of breeders in 2012 and 2013. We compared survival for birds with and without PIT tags across those two years. In a third population (HM), we began to PIT-tag birds in 2016 and have continued to tag birds using MT2 through 2019. Analysis for MT1 across GM and NF populations demonstrated that survival (10\%) was significantly lower in PIT-tag years than in years without tags (38\%). Among NF birds with tags, 54\% and 23\% of birds survived in 2012 and 2013 respectively; rates for birds without PIT-tags were 62\% and 44\%. We will analyze and compare the affect of MT2 on survivorship in GM and HM populations from 2016-2019 following the 2019 season, prior to the 2019 AOS meeting. This will allow us to determine if our newer tagging method, MT2, proves more promising for survivorship than MT1.}

\normaltalk{\capitalisewords{The Shorebirds may Disappear from the Around of the Largest City in South America}}{\underline{Fabio dnh Schunck}}{Shorebirds and their migratory routes are globally threatened, and one such region in South America is situated near the center of the city of São Paulo, one of the largest urban centers in the world, with more than 21 million inhabitants. Characterized as "Várzea of the Enbu-Mirim River", this wetland is in the basin of the Guarapiranga River, with aquatic habitats partially protected within a municipal reserve. Between 2007 and 2010, this region was impacted by a public works Project called "Rodoanel", a thoroughfare around the city of São Paulo. With the aim of identifying possible environmental impacts of the Rodoanel affecting migratory shorebird populations, an ongoing, voluntary monitoring project started 15 years ago, thus covering the periods before, during, and after the Rodoanel. Monthly censuses were conducted from 2007 to the present, totaling 143 over 150 field-hours. The study registered Bartramia longicauda, Actitis macularius, Tringa solitaria, T. flavipes, T. melanoleuca, Calidris fuscicollis and C. melanotos, that use the region between August and April. Before the Rodoanel project was begun, groups of up to 400 T. flavipes and 300 T. melanoleuca were registered; during the public works activities, these numbers fell drastically. After the completion of the Rodoanel, T. flavipes returned in good numbers, but T. melanoleuca stopped using the area, probably due to the grounding that affected their preferred feeding areas. B. longicauda also seems to have disappeared. These data reveal the impact that the Rodoanel has had on this group of birds, and show that the Guarapiranga region needs urgent, formal protection in the face of disorganized urban sprawl and other anthropogenic alterations of relictual urban wetlands.}

\normaltalk{\capitalisewords{Shorebirds Use the Surroundings of the Largest Urban Area of South America}}{\underline{Fabio dnh Schunck}}{Brazil possesses 32 species of migratory shorebirds. The country has several reserves recognized internationally as priority areas for conservation of these migratory birds, but there exist many other regions with few available data, which limits their priority for conservation attention. One of these areas lies near the center of the metropolis of São Paulo, with 21 million inhabitants ranking among the most heavily populated cities in the world. São Paulo is situated about 50 km from the Atlantic coast (750 m a.s.l), between two migratory corridors: the Atlantic littoral and the Brazilian interior.  In light of the lack of data from this region, an ongoing, voluntary research project has been conducted over the past 19 years in wetlands in the city, with the aim of identifying which species and how many of them are using these urban oases during migration and at other times of the year.  Up to the year 2000, six species had been registered: Pluvialis dominica, Bartramia longicauda, Tringa solitaria, T. flavipes, Calidris alba and C. fuscicollis. The list grew to 16 between 2000 and 2019, with the addition of Pluvialis squatarola, Charadrius semipalmatus, C. collaris, Limosa haemastica, Actitis macularius, Tringa melanoleuca, Calidris canutus, C. melanotos, C himantopus and Phalaropus tricolor. The region occasionally receives visitors from the coastal region, especial in association with the passage of strong storms, making it a form of safe-haven for these birds. These relictual wetland habitats in and around the city of São Paulo need to be protected as formal reserves, where monitoring of Nearctic migrants, should be continued.}

\normaltalk{\capitalisewords{The People of the Largest City in South America are Unaware of their Shorebirds}}{\underline{Fabio dnh Schunck}}{Migratory shorebirds occur in a wide variety of habitats, including urban centers.  In Brazil, which boasts one of the most species-rich avifaunas in the world, including 32 species of migratory shorebirds, awareness of these birds is, like many areas of the world, generally poor and restricted to a small subset of the population. The level of familiarity with migratory shorebirds is lesser yet when one considers modern population centers like the megalopolis of São Paulo, with 21 million inhabitants, where a large percentage of the people have little or no contact with natural environments. These people are always surprised to learn that these amazing little birds perform vast migratory movements out of the boreal regions of North America to spend the winter months in Brazil, sometimes passing through the very city in which they reside, spending a few days resting and feeding before continuing their flights. With the aim of popularizing local knowledge of Nearctic migrant shorebirds and the importance of their conservation, especially in the metropolitan region of São Paulo, data accumulated over the past 19 years is being shared with the public through illustrated presentations in universities, parks, and schools, field trips, and lay articles in widely read magazines. Thus, contact is achieved with a good cross-section of the population ranging from biology students and general residents to retired people and school children. These efforts have been entirely voluntary, with the much-appreciated support of the American Birding Association, particularly the Birder's Exchange Program, which has donated optical equipment.}

\normaltalk{\capitalisewords{Comparative Morphospace Filling in Three Passerine Flycatcher Clades}}{\underline{Clay E Corbin}}{Recently interests in evolutionary morphology are growing. This is especially true for analysis of association between morphological diversity and species diversification. One extension of this analysis is niche partitioning and hence, morphospace filling.  As a clade diversified, do newer lineages or subclades with smaller membership occupy the extremes of the morphological space?  Also, does niche filling occur along a particular ecological axis?  Nine morphological measures on species in Tyrannidae, Muscicapidae,and Monarchidae were used to test 1) small-clade peripheralization hypothesis of Ricklefs, 2) chronological axis diversification hypothesis of Foote. Species were grouped into genera and then subsequently into groups with large numbers of species and relatively depauperate genera. Statistical analyses included PCA, quadratic regression, and jackknife distance analysis. Outliers were small clades but tended to be distributed throughout the morphological space. Large clades were centralized and had reduced confidence ellipse s compared to small clades. Morphological characterization of small relative to large clades was family-level dependent. Hence while superficial morphological similarities exist among flycatcher radiations, morphospace filling seems to be subject to large scale phylogenetic constraint.}

\normaltalk{\capitalisewords{The Effects of Urbanization on Alarm Call  Variation of Carolina Wrens (Thryothorus Ludovicianus) and Heterospecific Response}}{\underline{Stephen E Scribner}, Diane L Neudorf}{Urbanization has caused many songbird populations to decrease. Habitat loss due to increased urban land cover, decreases in biodiversity, reduced food resources, and light pollution have all been documented as side effects of urbanization. More recently, anthropogenic noise caused by urbanization has been looked at with more scrutiny. Birds rely heavily on vocal communication to define territory, attract and locate mates, and warn conspecifics of potential threats. Urban landscapes possess an ambient background noise frequency lower than that of a natural rural environment. Several bird species raise their minimum song frequency and amplitude to combat the urban background noise but little has been studied on the effects of urban noise levels on the alarm calls of birds. We studied how the alarm calls of Carolina Wrens (Thryothorus ludovicianus) are influenced by urbanization and habitat density. We measured the effect of vegetation density in both urban and rural habitats on wren alarm call propagation and selection.  We also documented responses by heterospecific species to different alarm calls of the wren in both habitats to determine if population isolation influences alarm call effectiveness.  We will present initial results from this research.}

\normaltalk{\capitalisewords{Investigating Poecivirus as a Cause of Global Beak Deformities}}{\underline{Danielle E Gerik}, Caroline Van Hemert, Maxine Zylberberg, Colleen M Handel, Joseph L DeRisi}{Avian keratin disorder (AKD), an emerging avian disease characterized by debilitating beak overgrowth, was first documented in Black-capped Chickadees (Poecile atricapillus) in Alaska in the late 1990s. Since then, AKD has been reported in more than 30 bird species worldwide. Among the recent observations are reports of AKD-like symptoms in Austral Thrushes in Argentina and endangered Red-cockaded Woodpeckers in South Carolina. Previous work has shown that a newly discovered virus, called Poecivirus, is strongly associated with clinical signs of AKD in Black-capped Chickadees in Alaska, and is suspected to be the etiological agent of this disease. Poecivirus has since been detected in deformed individuals of 8 different avian species (Black-capped Chickadee, Red-breasted Nuthatch, Northwestern Crow, Blackpoll Warbler, Mew Gull, Black-billed Magpie, Hairy Woodpecker, Red-tailed Hawk). However, the geographic scope and range of potential host species for this virus remains unknown. In this study, we used genetic techniques to test for the presence of Poecivirus in 17 species exhibiting AKD-like symptoms. Samples for this analysis were collected opportunistically from various locations in North America, South America, and Europe. Resultant viral sequences from these samples were used to assess host and geographical extent. Here, we present preliminary results of our efforts to sequence the full Poecivirus genome in avian host species from across the world.}

\normaltalk{\capitalisewords{An Urban-Rural Comparison of Song Repertoire Size and Song Characteristics of Song Sparrows}}{\underline{Dustin E Brewer}, Samantha G Alcantara, Adam M Fudickar}{Characteristics of birdsong vary for some species between urban and rural populations, though correlations with repertoire size are less known. We recorded the complete song repertoires of eight Song Sparrows (Melospiza melodia) on the campus of Indiana University and in the rural countryside outside of Bloomington (n = 8 urban, n = 8 rural). For every song type from each individual's song repertoire (range = 6 to 11), we measured low frequency, high frequency, duration, delta frequency, peak frequency, notes per song, and number of several particular note types and configurations. We averaged these values for each song type to compute a 'repertoire mean' for each measurement, for all 16 birds. Between our urban and rural sites, number of 'complex trills' (medians = urban: 0.42/song, rural: 0.73/song; W = 48; p = 0.1) and number of 'buzzes' (medians = urban: 2.2/song, rural: 1.5/song; W = 13.5; p = 0.06) may vary. Peak frequency (medians = urban: 4803.6, rural: 5309.0; W = 54; p = 0.02) and notes per second (medians = urban: 9.1 notes/song, rural: 11.9 notes/song; W = 54; p = 0.02) likely vary. We found no compelling evidence that repertoire size or other variables measured vary between our urban and rural site (all p $>$ 0.1). We plan to increase our sample sizes in 2019, to better compare the populations. Some of the differences between the songs of birds at our urban and rural site could be adaptive.}

\normaltalk{\capitalisewords{Factors Limiting Reproductive Success of American Oystercatchers (Haematopus Palliatus) in Florida's Big Bend}}{\underline{Nick E vitale}, Janell Brush, Abby Powell}{Florida's Big Bend region of the Gulf Coast provides important habitat for both wintering and breeding oystercatchers. However, breeding Oystercatchers in the region consistently have low reproductive success and face a number of potential threats, including habitat loss, climate change, human disturbances, and a host of potential predators. We evaluated survival of nests and chicks with respect to six covariates for two nesting areas within the Big Bend (Cedar Key and Barge) in 2017 and 2018. Additionally, we measured habitat loss in the region over a 40-year period. Nest survival in Cedar Key was largely limited by failures from overwash (46\% of nest attempts), whereas Barge had a high hatch probability for the species during the two years of our study (mean = 0.45, CI = 0.31 – 0.58). Model results show chick survival was largely driven by a negative relationship with raccoon presence and nest lay date. Additionally, we detected 12 other potential predators using the area, which likely influence survival rates. Our study suggests limitations in food resources on nesting islands are playing a potentially large role in low chick survival rates, though additional research is needed to fully understand this relationship.  In addition to factors influencing reproductive success in a given season, a 41\% loss in area was documented for islands used by nesting Oystercatchers of the past 40 years. In Cedar Key, recent declines in island area were largely attributed to losses in oyster reefs, which provide a critical role in shoreline protection.}

\normaltalk{\capitalisewords{Consistent Fall Nesting of Neotropic Cormorants (Phalacrocorax Brasilianus) in Coastal Texas}}{\underline{Richard E Gibbons}, Peter Deichmann}{The Neotropic Cormorant (Phalacrocorax brasilianus) is a resident and migratory bird of the Texas coast with a prolonged breeding season with egg-laying onset from March to October in both estuarine and inland colonies. Unlike the Double-crested Cormorant (Phalacrocorax auritus), Neotropic Cormorant nesting ecology is poorly understood. While spring nesting is generally synchronous, little is known of later season nesting attempts. To better understand the patterns of nesting and roosting, we surveyed a long-term waterbird colony weekly for several years. A distinctly bimodal pattern emerged with fall egg-laying onset consistently occurring in mid-October after a three-month period without nesting. Additional fall nesting data from the Gulf Coast region were gathered and supported the bimodal nesting pattern.}

\normaltalk{\capitalisewords{Regional Variation in the Relationship Between Breeding Phenology and the Onset of Spring}}{\underline{Brooke E Goodnow}, Lesley P Bulluck}{Extreme and unpredictable global temperatures have led to shifts in the timing of ecological events (phenology) such as tree leaf-out, insect emergence, and egg laying in birds. Previous long-term studies assessing the effects of climate change on the timing of avian breeding have focused on single populations. This study employs citizen science data to assess the relationship between spring phenology clutch initiation dates across eight populations of eastern bluebirds (Sialia sialis) breeding in eastern North America. Specifically, we use data from NestWatch and the National Phenology Network to assess regional variation in the relationship between the onset of spring and clutch initiation date. Eight distinct spatial clusters of nesting records were identified in seven eastern US states based on consistent, long-term sample sizes (at least 10 years with $>$40 nest records/year). We found that three northern breeding populations of eastern bluebirds experience greater variation in the timing of spring onset and showed plasticity in their clutch initiation such that they initiate their first clutches earlier in warmer years, but this was not the case for the southern breeding populations that experience less annual variation in spring onset. In southern populations, we found less synchrony and longer times between early and late clutches for warmer years than the northern populations. Future analyses will include modeling West Nile virus (WNV) transmission as a function of avian breeding phenology; preliminary models indicate that timing of avian reproduction is an important predictor of WNV transmission.}

\normaltalk{\capitalisewords{An Analysis of Translocation Regimes for the Endangered Puaiohi}}{\underline{Jean E Fantle-Lepczyk}, David Duffy, Lisa Crampton, Andrew Taylor, Sheila Conant}{The ongoing and often synergistic effects of habitat loss, invasive species, and climate change pose challenges for conservation and management as widespread species become greatly reduced, sometimes to a single small population.  To address this problem, conservation biologists must consider using approaches like translocation to create new populations, reducing the probability of extinction by splitting a population into two or more populations in geographically distinct locales. The puaiohi (Myadestes palmeri), an endangered Hawaiian forest bird, has a small population size (494; 95\% CI 414–580) and restricted range (~40 km2). One recovery plan objective involves translocating birds to higher elevation Hawaiian Islands.   To evaluate translocation scenarios, we built upon previously developed population viability analysis models and considered how translocation regimes (initial population, number supplemented per event, supplementation interval, and supplementation length) would affect both original and new populations.  Furthermore, we modeled the puaiohi release population under three different conditions:  a stable population, a predator-controlled environment, and a habitat improved in terms of resource availability.  Our results indicated that while translocation offers hope of increasing puaiohi population size and decreasing extinction risk, success will depend on conditions at the release site.  Furthermore, harvest and rearing of eggs to the juvenile stage or re-establishment of a captive breeding program may be necessary to provide enough birds to translocate, as the current wild population may not be productive enough to sustain levels of harvest necessary to successfully establish a new population.}

\normaltalk{\capitalisewords{The Fairywren Project: Testing the Causes of Intraspecific Variation in Social Behavior and Signaling Traits Across an Ecological Gradient}}{\underline{Allison E Johnson}, Joseph Welklin}{Ecology is at the heart of trait evolution—animals are under constant pressure to adapt to the circumstances they are living in. Much of our understanding of the role of ecology in the evolution of behaviors and traits associated with animal sociality comes from case studies of single populations or comparative studies performed across species (meta-analyses). However, neither of these approaches captures the response within species to changes in ecology, which will ultimately drive the patterns seen across species. Ecological gradients provide an opportunity to capture such changes in selection within species. The fairywrens (genus Malurus) are a socially diverse group of passerines native to Australia and New Guinea. In addition to extreme variation in social breeding behavior observed both within and across species, fairywrens also exhibit variation in annual timing and age of first molt into nuptial plumage—a trait tied strongly to mating success. Through the Fairywren Project, a citizen science program that utilizes eBird sightings of fairywrens by birders across Australia, we are addressing how patterns of molt and social group size relate to the dramatic ecological gradients these species encounter throughout their ranges. In addition to this broad, multispecies approach, we will be performing detailed experiments in purple-backed fairywrens (Malurus assimilis) along a rainfall gradient to address how ecology drives changes in social group structure, drivers of delayed dispersal, and interindividual aggression mediated by plumage color across populations of the same species. Here we present on preliminary data and future directions for the Fairywren Project.}

\normaltalk{\capitalisewords{Phenology and Productivity in a Montane Bird Assemblage: Trends and Responses to Elevation and Climate Variation}}{\underline{James F Saracco}, Rodney B Siegel, Lauren Helton, Sarah L Stock, David F DeSante}{Climate variation has been linked to historical and predicted future distributions and dynamics of wildlife populations. However, demographic mechanisms underlying these changes remain poorly understood. Here, we assessed variation and trends in climate and avian demographic variables from mist‐netting data at six sites along an elevation gradient spanning the montane zone of Yosemite National Park between 1993 and 2017. We implemented multi‐species hierarchical models to relate demographic responses to elevation and climate covariates. Annual variation in climate and avian demographic variables was high. Snowfall declined, while spring temperature increased over the study period. Breeding phenology advanced by 5 days, and productivity increased by 0.8\% each year. Breeding phenology was 12 days earlier at the lowest compared to highest site, 18 days earlier in years with lowest compared to highest snowfall anomalies, and 6 days earlier in relatively warm springs (after controlling for snowfall effects). Productivity was positively related to elevation. However, elevation–productivity responses varied among species. Species with higher productivity at higher compared to lower elevations tended to be species with documented range retractions during the past century. Productivity tended to be negatively related to snowfall and was positively related to spring temperature. Our results suggest that birds have tracked the variable climatic conditions in this system and have benefited from a trend toward warmer, drier springs. We caution that continued warming and multi‐year drought or extreme weather years may alter these relationships in the future.}

\normaltalk{\capitalisewords{Evolution and Taxonomy Study of Herons in East Africa}}{\underline{Golya G Shahrokhi}, Michael Patten}{The herons (Aves: Ardeidae) constitute one of the taxonomic families with many unresolved phylogenetic relationships, both within and among clades. Although some morphological and molecular studies have been done on different species and resolved some basic relationships, several species of this family have not been studied at all. The Little Egret (Egretta garzetta), Western Reef-Heron (Egretta gularis), and Dimorphic Egret (Egretta dimorpha) are three members of Ardeidae with many ambiguities in their taxonomic relationship. Their similar morphology, overlapping ranges, and reports of possible inbreeding has pushed taxonomists and ornithologists into controversial conflicts about how to classify them. The primary objective of this research is to use mitochondrial and nuclear markers to help clarify the phylogenetic relationships of these three taxa. I also will test for the prevalence and origin of hybridization among them using both molecular markers and spatial analyses. I plan to use tissue and blood samples from both preserved (from museums) and wild specimens. After collecting samples, the extracted DNA will be amplified for both mitochondrial (mtDNA) and nuclear markers. The mtDNA markers are cytochrome b (cytb) and cytochrome c oxidase subunit I (COI) and the nuclear markers are the color morph gene (MC1R) and two additional markers. For hybridization, I would screen single nucleotide polymorphisms (SNPs) to obtain abundant data on the extent of interbreeding as well as obtain a picture of dispersal trends. The analyzed data will be joined with geographical and spatial analyses to clarify the history and evolutionary pattern of their speciation and divergence.}

\normaltalk{\capitalisewords{Visualizing Bird Distributions Across the Annual Cycle: New Opportunities and Strategies}}{\underline{Justin G Schuetz}, Tom Auer}{As observational data sets grow and enable description of bird distributions at broader spatial-temporal extents and finer spatial-temporal resolutions, new opportunities arise for visualizing when and where birds occur throughout the year. Movements of populations across the landscape are often visualized with animated maps or with multiple maps that reveal bird distributions at different points in time. While both of these communication strategies have benefits, in some cases they challenge our ability to perceive, distill, or integrate information about species distributions across the entire annual cycle. Here we use occurrence and abundance estimates from eBird Status and Trends model results to characterize bird distributions in static space-time-bird maps. We vary hue, chroma, and transparency values within an HCL color space to reveal three features of biological interest at an annual time scale on a pixel by pixel basis: relative abundance, timing of peak abundance, and degree of residency throughout the year. By summarizing temporal features of bird distributions in static maps, we expand the set of tools available for exploring bird biology across the annual cycle and for communicating research results.}

\normaltalk{\capitalisewords{Impact of Irruptive Black-Capped Chickadees on Social Structure of Resident Carolina Chickadees}}{\underline{Michael G Rowley}, Christie Eldredge, Matthew Dula, Robert L Curry}{Species ranges often overlap with one another, resulting in unfamiliar social interactions that may affect how information or disease spreads through a population. During 'irruption' years, Black-capped Chickadees penetrate the range of Carolina Chickadees, increasing the total number of interactions between individual chickadees of the two species. Our study compared levels of network density (the number of observed connections divided by the number of possible connections) in Carolina Chickadees in southeastern Pennsylvania between irruption and non-irruption years. We hypothesized that social network density would be higher in non-irruption years than in irruption years, as they will be more drawn to the relatively unfamiliar irruption Black-capped Chickadees, and this will detract from their number of realized conspecific social interactions. We used social network analysis to compare Carolina Chickadee network density in social networks constructed from feeder visits detected using radio-frequency identification (RFID) methods during fall-winter of 2012-2013 (irruption; 135,000 points from 91 Carolina Chickadees) and 2017-2018 (non-irruption; 109,000 points from 44 Carolina Chickadees). Network density among Carolina chickadees during irruption years (0.24) was 21\% less than during non-irruption years (0.30), supporting our hypothesis. These results suggest that the introduction of other species can displace conspecific social interactions, potentially impeding the spread of information or disease within populations.}

\normaltalk{\capitalisewords{Using High-Resolution sUAS Imagery to Create 3D Models of Greater Sage-Grouse Habitat}}{\underline{Ryan G Howell}, Randy T Larsen, Steven L Petersen, Ryan R Jensen}{Greater sage-grouse (Centrocercus urophasianus, hereafter sage-grouse) are dependent on sagebrush (Artemisia tridentata) in all life stages. Habitat use is influenced by sagebrush height and percent cover during nesting, brooding, and winter. Measuring sagebrush is traditionally expensive and time consuming if measured in situ, and limited when using traditional remote sensing methods without ancillary data. Improvements in Small Unmanned Aerial Systems (sUAS) technology have greatly facilitated acquisition of high resolution imagery on western rangelands. The emergence of photogrammetry algorithms to generate 3D point clouds from true color imagery provides potential for more efficiently measuring sage-grouse nesting, brooding, and winter habitat. Our objective is to test the accuracy of using high resolution sUAS imagery and photogrammetry to create a three dimensional model of sage-grouse habitat, comprised of sagebrush, grasses, and forbs. We acquired imagery near Strawberry Reservoir, Utah using a DJI Inspire 2 multi-rotor UAV equipped with an RGB camera, flown at 100, 150, 250, and 390 feet. We mosaicked the imagery and generated a Digital Surface Model (DSM) using Pix4D software. We estimated plant height by subtracting fine scale elevation data calculated from the sUAS imagery from the DSM, then perform an accuracy assessment using on the ground measurements taken at the time of flight. We found that by incorporating high resolution sUAS imagery and 3D sagebrush models we can monitor habitat for greater sage-grouse and other sagebrush obligate species in a more cost and time efficient manner.}

\normaltalk{\capitalisewords{Using Tree Swallows to Evaluate the Effectiveness of Artificial Wetlands to Support Aerial Insectivores}}{\underline{Victoria G Roper}, Lauri Green}{Over the last 100 years, 50\% of natural wetlands in the United States have been lost due to urbanization or agriculture. As natural wetlands have been lost, artificial wetlands are constructed to restore water quality and valuable ecological habitat for bird assemblages. The conservation value of artificial wetlands needs to be explored further. In this study, we investigated how wetland habitat type (natural versus artificial) and quality (water quality, insect availability, and foraging rate) affect populations of breeding Tree Swallows (Tachycineta bicolor). We evaluated reproductive success (nest box occupancy rate, clutch size, and hatchling mass) with respect to nutrient enrichment, prey availability, and foraging rate in wetlands. Alkaline persulfate digestion was performed for total nitrogen and phosphorus in water and sediment. Surprisingly, a pooled t-test showed that total nitrogen in water was significantly greater in a natural site than the artificial wetland (n=8, t=5.04, p=0.001). There was significantly greater total phosphorus in natural wetland sediment than artificial (n=8, t=-2.99, p=0.017). These results may be attributed to the presence of cattail (Typha latifolia), the presence of wool-grass (Scirpus cyperinus), and closer distance to agriculture. Foraging rate did not differ between site type (natural versus artificial). Additionally, insect densities were greater at the artificial site due to higher dissolved oxygen (mg/L), and lower turbidity (NTU). Further, nest occupancy rates and clutch sizes were greater in wetlands with lower nitrogen and higher prey density showing that artificial wetlands can support avian aerial insectivore populations assuming disturbances resulting from distance from agriculture are minimized.}

\normaltalk{\capitalisewords{Dispersal on the Non-Breeding Grounds: a Neglected Process at the Heart of Range Dynamics in Migratory Birds}}{\underline{Stephen H Vickers}, Aldina M Franco, Nicole L Michel}{A striking majority of literature discussing shifts in species ranges is focussed solely on the breeding range, and how new breeding areas are colonised through dispersal processes. For migratory species, of which $>$40\% of bird species can be defined, range shifts can also take place around the areas occupied during the non-breeding period. Just as breeding and natal dispersal are the processes that underpin breeding range shifts, equivalent movement processes occur on the non-breeding grounds, and may be pivotal in allowing new non-breeding areas to be colonised. However, these processes have not yet been reviewed, or even clearly defined, in the literature on migratory species. Here we present a conceptual model of the mechanisms underpinning non-breeding range-shifts, highlighting the critical role of inter-annual changes in the non-breeding sites occupied by individuals – a process analogous to dispersal for breeding sites. Non-breeding site changes can occur either within individuals (analogous to breeding dispersal), or between parents and offspring (analogous to natal dispersal). We propose a new term, 'non-breeding range dispersal', to describe the distance between consecutive non-breeding locations, a metric analogous to dispersal distance for breeding ranges. We show how non-breeding range dispersal (individual-scale) is a fundamental driver of migratory connectivity (population scale) as well as a limiting factor for non-breeding range shifts. Finally, we explore links between non-breeding range dispersal and the heritability of the migratory programme (genetic, epigenetic or cultural). Using a range of real-world examples, we illustrate how these concepts are essential for understanding range dynamics in migratory species.}

\normaltalk{\capitalisewords{Niche Segregation among Three Sympatric Species of Swallows in Southern Ontario}}{\underline{Kaelyn H Bumelis}, Michael D Cadman, Gregory W Mitchell, Keith A Hobson}{Barn Swallows (Hirundo rustica), Cliff Swallows (Petrochelidon pyrrhonota) and Tree Swallows (Tachycineta bicolor) breed sympatrically in Southern Ontario but it is not clear how these species differ ecologically, and their coexistence implies niche segregation. I investigated potential differences in nestling diet and post-fledging movements among these species. To assess diet, DNA barcoding of feces was used to identify presence/absence of prey species. Similarly, stable isotope analysis (δ13C, δ15N) of nestling feathers allowed for the determination of dietary sources for provisioned young and their respective trophic levels. To determine post-fledging movements, VHF radio transmitters were used with the Motus Wildlife Tracking System to track fledglings of the three species. Results are discussed in terms of relative overlap and ecological segregation among species. This information may be useful for conservation and management of these species by helping to identify potential factors contributing to differential declines operating on the breeding grounds.}

\normaltalk{\capitalisewords{Range Shifts, Environmental Change and Tipping Points: Predicting Impacts on Long-Distance Migratory Herbivores}}{\underline{David H Ward}, Richard A Stillman, Ellie M River, Whelan Gilkerson, Kevin A Woods, J D Golicher, Bart A Nolet, Preben Clausen}{Climate change is leading to worldwide shifts in the distribution of biodiversity, and fundamental changes to global animal migrations. Some arctic-nesting species may shorten their migration distance as warmer temperatures allow them to winter closer to their high-latitude breeding grounds. However, such decisions are not without risks, including the increased energy demands of remaining in colder regions and the associated increase in vulnerability to natural or anthropogenic environmental change.  In this study, we used an individual-based model to predict the effects of environmental change on Black Brant that forage on eelgrass at the Izembek Lagoon complex in southwest Alaska.  Brant use this site during fall and spring migration and increasingly in winter. Simulations indicated that a $>$50\% reduction from the current decadal mean biomass of eelgrass, which has occurred in some years, could reduce the proportion of birds migrating successfully to their breeding grounds.  The model also predicted that access to eelgrass in lagoons other than Izembek was critical for overwinter survival and spring migration of brant in years of reduced eelgrass biomass and 100\% ice coverage in Izembek Lagoon.  Brant were predicted to still successfully emigrate in fall with up to a doubling of population size; however, increases of over 20\% above the current winter population in years of minimum eelgrass biomass could reduce the number of potential breeding birds.  We also evaluate the response of goose populations to increase in disturbances within each season and discuss the consequences of these predictions in the face of natural and human sources of change.}

\normaltalk{\capitalisewords{Birds Across Borders: Building an International Community of Ornithology}}{\underline{Lauren H Daigler}, Farren Grande, Sylvia Hanson, Adam Parol, Jennifer Campbell-Smith}{Around the world, birds influence our lives, cultures, and connections to one another. While humanity divides itself by political and cultural borders, birds don't see these human divisions, but instead see a unified and connected natural world. Why shouldn't humanity do the same? Birds Across Borders is an initiative to bring together students, teachers, and professionals from around the world for a singular purpose: to build a better, more connected world through our passion of modern-day dinosaurs. The program is a grass-roots initiative working with schools, community centers, local governments, creative studios, independent researchers, and other organizations to empower students around the world to connect with one another, professionals, and their environment. Birds Across Borders will raise awareness about endangered avian species, foster kinder relationships and understanding between human communities and other species living alongside them, and encourage people to explore the field of ornithology. Using a website we have developed as a hub, we will keep our community updated on project initiatives, ornithological research, creative projects by our members, and other cross-cultural and cross-curricular studies on different avian species and topics. We envision a place where a student from one country can connect with students from other places and find common ground thanks to a shared interest and experience with different bird species.}

\normaltalk{\capitalisewords{The Long and Short of It: Are Telomeres Related to Lifetime Reproductive Success in Free-Living House Sparrows?}}{\underline{Britt J Heidinger}, Rebecca C Young, Aurelia C Kucera, Jeff D Kittilson, David F Westneat}{Individuals with longer telomeres and slower telomere erosion often live longer. In some species there is also evidence that greater investment in reproduction increases telomere loss. Thus, telomere length and loss are emerging as important mechanisms and/or biomarkers of aging and may underlie the commonly observed trade-off between investment in reproduction and survival. However, little is known about the relationship between telomere length and loss and lifetime fitness. Here we examined the relationship between telomere length and loss in historically collected red blood cell samples and reproductive performance measures and longevity in free-living house sparrows (Passer domesticus). All individuals were originally sampled at the end of the post-natal development period and again in adulthood. Telomere length was measured using qPCR and results will be discussed within the context of life-history theory.}

\normaltalk{\capitalisewords{Behavioral Responses of an Eastern Population of Red-Winged Blackbirds to Experimental Brown-Headed Cowbird Parasitism}}{\underline{Justin J Reel}, Todd J Underwood}{Red-winged blackbirds (Agelaius phoeniceus) are parasitized by Brown-headed Cowbirds (Molothrus ater) less frequently in eastern than in central North America. Rarely, geographic variation in parasitism frequencies reflects a different behavioral response to parasitism by hosts where only some populations reject parasitism. We tested whether rejection behavior explains the low parasitism frequencies observed in an eastern Red-winged Blackbird population. Red-winged Blackbirds are known to accept experimental parasitism but most of this research was conducted in central North America and did not test responses by nest stage. We experimentally parasitized Red-winged Blackbird nests with model cowbird eggs in Pennsylvania to determine their response in the east. Nests were tested across three nest stages plus a control. We also surveyed other songbird nests to identify community-wide parasitism frequencies. Red-winged Blackbirds accepted 69\% of cowbird eggs overall. However, rejection differed significantly by nest stage with significantly more rejections during nest building before host eggs were laid. Those rejections were likely a nest sanitation response rather than true rejection of cowbird eggs. Excluding rejections during building, this eastern population of Red-winged Blackbirds accepted similar proportions of cowbird eggs to central populations. Thus, geographic variation in rejection behavior likely does not contribute to the low parasitism frequencies on eastern Red-winged Blackbirds. Overall parasitism frequencies on 11 songbird species surveyed was low at 7.4\%. We suggest that the low abundance of cowbirds in the east and the preference for other more suitable hosts may explain the low parasitism frequencies on eastern red-winged blackbirds.}

\normaltalk{\capitalisewords{Social Context Influences the Expression of Aggression in Zebra Finches}}{\underline{Anne-Laure J Blanche}, Kathryn Chenard, Renee A Duckworth}{By definition, social behaviors are always expressed in the context of other individuals. However, this makes measuring them in a standardized fashion difficult because measurements outside the normal social context may not act as a true measure of behavior in natural settings. Here, we assess the importance of social context in the expression of aggression in zebra finches, a social species in which aggression is important in resource acquisition and the establishment and perpetuation of dominance hierarchies. Aggression was measured repeatedly in twenty-two individuals in a flock context and was measured once for each individual outside their normal flock in a mirror test. Birds in a flock were recorded accessing a treat feeder, and the number of aggressive interactions as well as their dominance position were assessed. Preliminary results indicate that aggression and dominance are highly repeatable in the flock context and are positively correlated with one another. During the mirror test, aggressive response to the individuals' own reflection in a mirror was scored to assess an individual's reaction to an unfamiliar but equally-matched bird. The flock measures will be compared to measures of aggression from a mirror test. Given that many studies measure aggression by using the solo mirror test, it is important to determine how it relates to the variation in aggression expressed in more naturally occurring flock contexts.}

\normaltalk{\capitalisewords{'Smart' Nest Boxes: Expediently and Remotely Collecting Avian Data Utilizing Unique Advantages Within Community Colleges}}{\underline{Dylan J Smith}, Michael Bates, Landon Sokol, Tychique Kutalu, Elizabeth Ewing, Jenessa Grooms, Andres Espino, Alejandro Espino, Kayla N Kreizel, Steve Heinisch, Neil Grandgenett, Lauren M Gillespie}{Community colleges systems significantly contribute to STEM major areas at baccalaureate institutions. Students planning to forge academic paths to obtain advanced degrees in life or engineering sciences can benefit from program areas in mechatronics and computer science as coding functionality becomes an essential skill across disciplines. We are a community college student-cohort funded by a National Science Foundation S-STEM scholarship program and we propose a project bridging gaps across disciplines while providing accessible technology to avian (or mammalian) cavity-nesting researchers or enthusiasts. We propose an evolving "smart" nest box iteration intended: 1) to obtain data (e.g.  plumage reflectance or specific song-recordings) usually requiring costly equipment, 2) to reduce nest disturbance while monitoring breeding to more efficiently collect data, and, 3) to create a cost-efficient, accessible model. We will utilize 3-D printing and design technology to assist housing and powering box technology. We will integrate sensing technologies utilized in quality assurance industrial manufacturing. We plan to apply either a design of systems used in industry or integrate systems utilizing a Raspberry Pi and python with industry guidance. We look to build on troubleshooting suggestions of other smart boxes models, such as creating specific algorithms to vary minimum frame rates of cameras during set time periods, and, to increase them as necessary based on bird activity levels to reduce power consumption. We plan to build and test this system in the upcoming 2019 breeding season and report specific methods and results of data collected.}

\normaltalk{\capitalisewords{Temperature and Precipitation Experienced Prior to Capture in Northeastern Pennsylvania is Associated with Timing and Body Condition of Gray Catbirds and Common Yellowthroats During Fall Migration}}{\underline{Robert J Smith}, Margret I Hatch}{Even as the fall migratory period encompasses a significant portion of the avian annual cycle, most avian climate or weather studies have focused on the phenology of spring migration or onset of breeding. Further, few studies have evaluated how body condition is influenced by climate, or components of climate, especially during the fall migratory period. We used nine years of capture data (2006 – 2008, 2012, 2014 – 2018) from northeastern Pennsylvania combined with data collected from weather stations north of our study area (Global Surface Summary of the Day, United States National Oceanic and Atmospheric Administration National Climatic Data Center) to examine relationships between temperature and precipitation experienced during the breeding and fall migratory periods north of our study area and passage timing and condition in Gray Catbirds (Dumatella carolinensis) and Common Yellowthroats (Geothlypis trichas) using our study site during fall migration. We found that climatic elements experienced prior to capture at our site were associated with arrival timing and body condition, and that these elements may have differential effects dependent upon age. Identifying influential climate conditions, including where, when and how climate influences fitness, is important to understanding population dynamics, especially in light of climate change.}

\normaltalk{\capitalisewords{Tree Foraging Preferences of Warblers During Fall Migration along the Sacony Creek Trail}}{\underline{Daniel J Stoner}, Todd Underwood}{During spring migration, warblers (Parulidae) show strong foraging preferences for trees with more abundant food resources and native trees at their stopover sites. However, foraging preferences during fall migration have not been studied. In fall 2018, we observed 12 species of warblers foraging during migration along the Sacony Creek in Kutztown, PA and compared these observations to expected numbers based on a tree population survey. Foraging preferences were examined across all tree species, native compared to non-native tree species, and among native tree species based on their Lepidopteran food value. Warblers showed no significant foraging preference across all trees, and no significant preference to forage on native compared to non-native tree species. Finally, warblers showed no significant preference to forage on native tree species based on their Lepidopteran food values. A lack of preferences may be due to the decline in leaf quality in fall, which causes caterpillar abundance to decline. Thus, foraging during fall migration might be influenced by other types of arthropod prey. Furthermore, leaf drop may have lead us to underestimate foraging preferences because some abundant native trees species dropped most of their leaves half-way through our study. A tendency of warblers to forage on the non-native and abundant White Mulberry (Morus alba) likely influenced the lack of a preference for natives because all other non-native trees were completely avoided. Future research should consider the influence of non-Lepidopteran prey available and how leaf drop alters foraging habitat during fall migration.}

\normaltalk{\capitalisewords{Modeling Population Dynamics of the Northern Bobwhite (Colinus Virginianus) in Ohio}}{\underline{Connor J Rosenblatt}, Robert J Gates, Stephen N Matthews}{Northern Bobwhite (Colinus virginianus) populations have been declining across much of the species' range. Previous work in Ohio showed that overwinter survival was the primary factor influencing population growth rate during a 3-year period with prolonged winter snow cover. In contrast, a recent analysis in the more optimal core range of bobwhite in the southeastern U.S. demonstrated that recruitment was a driving limiting factor of population growth. Additional demographic data from the Ohio population prompted us to use matrix population modeling to estimate population growth rate over a longer time period that included a broader range of winter severity. We used demographic data obtained from radio-marked bobwhites during 2009-2014 to further evaluate the sensitivity of population growth rates to demographic parameters in Ohio. Winter survival rate was highly variable; survival rate ranged from 0.075-0.393, and was associated with snow cover. Population growth rates exhibited highest sensitivity to this parameter compared to other parameters. The matrix population model is a component of an integrated population model we are developing to evaluate the putative effort of chick survival rates and to project population growth through land cover change, weather, and climate change scenarios. We are ultimately concerned about the viability of northern bobwhites at the northern edge of the species range. The integrated population model will inform managers about how to best focus conservation efforts to sustain bobwhites in the face of environmental change, and estimate future population growth rates with reduced uncertainty.}

\normaltalk{\capitalisewords{What Factors are Related to Feather Mite Abundance Within, and Among, Species of Parulid Warblers?}}{\underline{Kevin J Krajcir}, Alix E Matthews, Alexander J Worm, Than J Boves}{Feather mites are ubiquitous, diverse, and poorly studied obligate avian ectosymbionts that have been described as parasites, commensals, and mutualists of their hosts. In correlative studies that have attempted to assess this relationship, mite abundance is often used as the main explanatory variable. However, we still lack information about the factors that influence mite abundance (both within and among species), in part because most studies either use only a single species or focus on a relatively narrow geographic region. To overcome these shortcomings, we aimed to describe how feather mite abundance varies, within and among species, across geography, climate, and host ecology. We captured individuals of seven different species of New World warblers (Parulidae) from across their breeding ranges and, using a macro lens, photographed wing and tail feathers to census mites. In total, we captured 129 individuals across six states. Within species, feather mite abundance and latitude were negatively correlated (p = 0.004), but mite abundance did not differ by host age (all species p $>$ 0.1; SY vs ASY). Among species, mite abundance varied by species identity and ecological stratum (both p $<$ 0.001). In total, mite abundance was related to factors of species ecology, and within species, geographic location (and the climate/weather at that location). It is still unclear how these factors may also influence the functional relationship between feather mites and their hosts. Next, we plan to evaluate how these same factors may mediate the relationship between mite abundance and host condition, survival, and physiology.}

\normaltalk{\capitalisewords{Rapid Detection of Avian Blood Parasites and West Nile Virus in the Common Loon Utilizing Loop-Mediated Isothermal Amplification}}{\underline{Carly J Paget}, Josh S Sharp, Alec R Lindsay}{The common loon (Gavia immer) is widely recognized as a key indicator of aquatic health to monitor environmental conditions within a variety of habitats. Incubating and foraging loons are frequently harassed by biting insects like blackflies and mosquitos, but there is comparably little information on the blood parasites transmitted to loons by these insect pests. Additionally, the occurrence of multiple blood pathogens has been difficult to assess using traditional PCR-based methods that are inherently expensive and time-consuming. For this reason, the consequences of pathogenic infection rates on common loon survival and reproductive success are poorly understood. We present a novel genetic assay of blood parasite infections that uses loop-mediated isothermal amplification (LAMP) of DNA isolated from Leucocytozoon sp. and Plasmodium sp., and RNA isolated from West Nile Virus. This assay was designed for the common loon and tested on a population of loons breeding in the eastern Upper Peninsula of Michigan. Utilizing pathogen-specific primer sets, LAMP assays offer a cheap, fast, and reliable protocol for detecting DNAs of interest using a visual assay that can be performed in a single reaction tube at one temperature. As such, the application of this molecular technique should be meaningful for other field researchers interested in assaying the presence of infectious pathogens within wild loons.}

\normaltalk{\capitalisewords{Demography and Infectious Disease Ecology of a Resident Passerine in Southern Louisiana}}{\underline{Eric J Tobin}, Scott M Duke-Sylvester}{Infectious diseases can shorten a bird's life span. In this study we use molecular methods to diagnose over 600 Northern Cardinals (Cardinalis cardinalis) individuals as either positive or negative for avian malaria. We use this information in a multi-strata mark-recapture models, that include disease status along with fat, breeding status, and age to determine survival and population growth rates. Malaria serostatus and sex are used as grouping variables in our models, along with other body metrics such as size and molt-cycle as individual variables. Preliminary results suggest an increase in survivorship with increasing size for C. cardinalis, differential sex-based survivorship, and a reduction in survival for seropositive individuals.}

\normaltalk{\capitalisewords{Methylation in the GR Promoter in the Brain and Blood of the European Starling}}{\underline{Stefanie J Siller}}{The hypothalamic-pituitary-adrenal (HPA) axis mediates responses to stress. It is regulated by a negative-feedback mechanism controlled by the level of glucocorticoid receptors (GR) in the hippocampus and the hypothalamus. GR level in turn depends upon GR gene expression, which can be altered by epigenetic mechanisms, specifically methylation. Our understanding of long-term effects of GR methylation on stress response and fitness are hindered by current studies' focus on laboratory mammals and the early environment, and the need to euthanize individuals to study methylation in the brain. In this study, we aim to tackle these obstacles by considering the viability of blood as a biomarker for GR methylation in the brain. Using pyrosequencing, we looked at percent methylation of CpG sites across the GR promoter in the blood, hippocampus, and hypothalamus of wild adult European starlings (Sturnus vulgaris). We found that total percent methylation of CpG sites differed significantly between tissue types: it was highest in the hippocampus, followed by the hypothalamus, and was lowest in the blood. We found a significant correlation between methylation of the blood and hippocampus at a single CpG site, but found no evidence for correlation between total promoter methylation in these tissues. This lack of correlation may suggest that blood is not a viable biomarker for methylation changes in the GR within the brain for future long-term analyses. However, future studies should consider how early life stress may play a critical role in altering GR methylation, producing similar patterns of methylation across tissues.}

\normaltalk{\capitalisewords{Territorial Responses According to Female Reproductive Status in a Species with Long Term Pair Bond and Year Round-Territory Defense}}{\underline{Ignacio J Guti\'{e}rrez Vargas}, Luis A Sandoval}{Changes in the territorial behavior of birds occur through the year, for example between breeding and non-breeding season, or within breeding season according to the nesting cycle. But how the territorial behavior responses vary in species that keep the same pair several years and maintain a territory year-round is poorly understood. The objective of this study was to investigate the territorial behavior of white-eared ground-sparrows (Melozone leucotis) and the differences in responses between territorial defenses correlated with the reproductive status of the pairs. We conducted a playback experiment where simulated territory intrusions by another pair of the same population but non-neighbor and a control duet in 38 territorial pairs of three populations located in the Central Valley of Costa Rica. We measured five behavioral responses: latency of the first vocalization and approaching within 3 m of the speaker, duration close the speaker, number of individuals approaching, and number of vocalizations. The five behavioral responses were combined through principal component analysis into one multivariate response, finding that this species shows the highest level of aggression when the pairs were non-breeding or nest building compared to pairs with juveniles and incubating. The white-eared ground-sparrows showed higher aggressiveness when the female is able to change the pair or have extra-pair copulations, therefore, is most likely that the behavior observed is a mechanism of defense to avoid losing the reproduction. The previous information agrees as aggressiveness decreases when the females are no longer available for reproduction.}

\normaltalk{\capitalisewords{Consequences of Severe Drought on Grassland Songbird Reproduction}}{\underline{Dylan J Smith}, Austin J Roe, Sarah K Winnicki, Emily J Williams, Andrew G Hope, W A Boyle}{Extreme climatic events can be major sources of selection, and demographic consequences of these events can cascade across trophic levels. However, extreme events are notoriously difficult to study due to their spatial and temporal unpredictability. We took advantage of a regional drought in 2018 affecting the Flint Hills of Kansas which occurred during ongoing population-level studies of three species of declining grassland songbirds: Grasshopper Sparrow (Ammodramus savannarum), Dickcissel (Spiza americana), and Eastern Meadowlark (Sturnella magna). We compared data from up to 4 years of normal-to-wet conditions with the 2018 drought year, documenting inter-annual differences in reproductive behavior and success. Unlike previous years, nest initiation dates in 2018 were concentrated early in the season, suggesting that birds did not re-nest or attempt second broods. Two species with domed nests shifted the orientation of their nests away from early morning sun and toward the direction of prevailing winds. Nest survival and productivity were similar across years. However, the identity of nest predators changed dramatically, dropping from 100\% snake predators to 50\% from 2017 to 2018. We explored potential trophic drivers of these responses finding evidence consistent with prey-switching for nest predators but not food-driven responses in our focal species. Behavioral shifts during extreme events thus may reflect both direct effects of weather and complex trophic interactions. Understanding the trophic links and demographic outcomes of drought in climatically-variable systems such as tallgrass prairie can provide insight into how populations might be affected by future environmental change.}

\normaltalk{\capitalisewords{Acoustic Monitoring of American Woodcock (Scolopax Minor) in Tennessee}}{\underline{Evan J Buck}, David A Buehler}{American Woodcock (Scolopax minor) populations have been monitored with the Singing Ground Survey by human observers since 1968 while populations have steadily declined. In January-March, 2017 and 2018, we tested the accuracy and feasibility of using a remote acoustic monitoring approach to achieve similar population monitoring goals.  We conducted paired counts (n = 32) by human observers and SongMeter SMII acoustic recorders at three strategically located point locations on each of three different publicly-owned management areas in eastern Tennessee to compare the effectiveness of the two methods. We developed a linear regression that related the number of calls noted by human observers to the number of individuals that were present calling.  Through this regression we estimated the number of individuals which were present on the acoustic recordings. We analyzed both "peent" calls and the aerial display "twitters".  The recording device had a similar detection range to the distance over which a human observer detected a woodcock (x = 200 m). Through this approach, recordings from the entire breeding season can be run through automatic detection analysis to obtain a population estimate from the number of "peent" calls detected per daily monitoring period. We maintain that an acoustic monitoring approach eliminates many of the shortcomings of traditional point count surveys for monitoring woodcock and provides an avenue for more effective and time efficient monitoring of the species.}

\normaltalk{\capitalisewords{Putting the Right Habitat, in the Right Places, at the Right Time: Integrated Waterbird Management and Monitoring}}{\underline{Patricia J Heglund}, Jana Newman, John Stanton}{Wetland managers typically count waterbirds and share the numbers locally with the public but the effort any one manager puts towards consistent surveys is subject to changing budgets, staff, and competing priorities. In addition, managers have not had a straightforward means for comparing results from one management area to another nor have they had the ability to clearly link waterbird abundance to their management actions. Understanding where along a flyway more habitat is needed, and when it is needed, as well as what management actions are successful has been hindered by the lack of consistency, the inability to compile and analyze data from many areas, and to broadly share results. The vision of the Integrated Waterbird Management and Monitoring (IWMM) initiative is to support a landscape where non-breeding waterbirds have the right habitat in the right places at the right time. Therefore, the IWMM initiative has developed standard monitoring protocols that track management actions, habitat conditions, and bird abundance and are supported by a publicly available database.  The IWMM initiative provides consistency in monitoring, the compiling of results of management, and provides information needed for managers to learn and make better decisions.}

\normaltalk{\capitalisewords{Distribution, Abundance, and Habitat Selection of Breeding Mississippi Kites (Ictinia Mississippiensis) in Southern Illinois}}{\underline{Benton J Hendrickson}, Michael W Eichholz}{Mississippi kites (Ictinia mississippiensis) are small neotropical migrant raptors that breed throughout the American southeast and Great Plains.  Within the Mississippi Alluvial Valley, they underwent a dramatic population decline in the early 1900's attributed to the loss of nesting habitat due to changing agricultural practices, but have experienced an increase in abundance and reoccupation of historic range since the 1950's.  Despite this, the Illinois Endangered Species Protection Board lists Mississippi kites as Threatened.   Extensive anecdotal evidence, including reports from State and Federal personnel, social media, and eBird suggests that the abundance of kites has continued to increase in southern Illinois, potentially warranting de-listing the species.  To determine the status of Mississippi kites in Illinois and identify factors influencing potential population growth and range expansion, we are conducting foot, automobile, and ATV surveys for nesting kites in 10 southern Illinois counties. In 2019, trapping, banding, and radio-tracking will begin to determine home-range size and locate new nesting locations.  Behavioral data, nest locations, and nest-site vegetation data will be used to build a Species Distribution Model estimating the amount of suitable kite habitat in the region.  Using home-range estimates and results from distribution modelling, we will establish a current minimum baseline of Mississippi Kite abundance in southern Illinois.  Additional pursuits, including molecular sexing and fecal dietary analysis will also be explored.  The results of this study will inform decisions regarding the conservation status of Mississippi kites in Illinois, as well as determine the factors influencing kite population dynamics in the region.}

\normaltalk{\capitalisewords{Variables Explaining Early-Successional Habitat Associations of Six Mature-Forest Bird Species During the Breeding Season}}{\underline{Patrick J Ruhl}, Kenneth F Kellner, John B Dunning Jr}{Some bird species that require mature forest habitat for breeding use early-successional habitat to a great extent during the latter part of the breeding season. However, for many mature-forest birds, the specific drivers behind this early-successional habitat association remain poorly understood. From 2015 – 2017, we used constant-effort mist-netting to sample breeding birds in six 7- to 9-year-old regenerating forest clearcuts in southern Indiana, and characterized early-successional habitat associations of six mature-forest bird species. We collected detailed environmental covariate data to account for variations in food availability and habitat structure variables corresponding to each banding session at the clearcut-site scale. We fit separate generalized linear mixed models, using a negative binomial error structure, for the six focal species captured within clearcuts. Both habitat structure and food availability variables were important predictors of mature-forest bird captures in clearcuts. Vegetation density was included in best fit models for five of the six mature-forest species, corroborating results from previous studies that attributed vegetation structure as the driving factor behind early-successional habitat associations. Fruit availability was included in best fit models explaining Scarlet Tanager and Wood Thrush captures, and invertebrate dry mass was included in best fit models for Worm-eating Warbler, Ovenbird, and Black-and-white Warbler. Differences in the relative importance of habitat structure and food availability variables for multiple mature-forest species in our models suggests that the driving factors influencing early-successional habitat use by mature-forest birds during the breeding season may vary among species.}

\normaltalk{\capitalisewords{Private-Public Partnerships for Intensive Management Skill Building}}{\underline{Constance J Woodman}, Donald J Brightsmith}{The skills needed for conservation science at the interface between wild and captive are a complex mix of field science and captive management.  They are the skills needed for translocation, fostering, captive release, reintroduction, wild nest management, conservation breeding, and other techniques.  What was in the past considered "technician work" such as zookeeper, bird breeder, veterinary technician, or wildlife rehabilitation work, can be core component skill sets for scientific conservation. This presentation demonstrates how a cross-boundary partnership between academia and commercial industry is integrating these skill sets into early career training. Undergraduate, graduate, and veterinary professional students from Texas A\&M participate in a structured multi-month training program. They work with the ~100 species held Hill Country Aviaries, then have the opportunity to follow up their ex situ skills building with in situ skills at the Tambopata Macaw Project in the Peruvian Amazon. Specialized, short-term opportunities are created for mid-career scientists who need to gain specific skills. These skills are needed for species that require intensive management for survival and ecosystem recovery. As deficits in critical skill areas can cause project failure, strategies to gain skills in a way that is complementary to academic education are important.}

\normaltalk{\capitalisewords{Grassland Community Responses to Habitat Structure and Management Actions}}{\underline{Alex J Glass}, Michael W Eichholz}{Grassland songbirds have declined at a faster rate than almost any other group of birds in North America, a trend driven largely by extensive habitat loss. Thus, effectively managing remaining grassland patches for songbird productivity is essential for their conservation. In order to better inform management practices, this study seeks to identify the mechanisms through which habitat management may affect grassland bird productivity (measured by nest success and nest density). We will investigate the responses in grassland bird productivity to management actions and habitat structure on grassland patches in southern Illinois. Additionally, we will estimate how management actions and the resulting changes in vegetation structure and composition affect other trophic levels in the grassland community, so we may relate those changes to changes in bird productivity. Our aim is to identify the post-management changes in the grassland community that affect grassland bird nest success and nest density.}

\normaltalk{\capitalisewords{Capitalizing on a Mass Mortality Event: Archiving Seabird Genetic Samples, Skins, and Skeletons from the M/v Selendang Ayu Oil Spill}}{\underline{Jack J Withrow}, Kevin Winker}{In December 2004 the M/V Selendang Ayu ran aground on Unalaska Island in the Aleutian Islands, spilling over 350,000 gallons of fuel oil and resulting in the salvage of over 1,500 seabird carcasses. With funding from the North Pacific Research Board we were able to archive genetic samples, skeletons, and skins (or some combination thereof) from a majority of these remains. The specimen material is available through loans or visits by researchers, and the associated data are available through our online database (Arctos). This material substantially increases existing Alaska sample sizes of alcids, cormorants, gulls, etc. and provides a unique opportunity to retroactively sample a time (15 years ago) that can't be revisited. Archived specimens of seabirds provide a biological record not only of the birds, but also of the ecosystem and factors influencing them—they are excellent marine bioindicators. They can be used to investigate food web dynamics (using stable isotopes), foraging success (though hormone levels deposited in feathers), contaminants levels (e.g., heavy metals), population demography (age and sex structure), genetic diversity and population genetics (germane to issues in conservation), genomics, morphology, impacts of oil spills on species and age/sex groups, and human impacts on Bering Sea nonbreeding bird communities. Preserving and archiving multiple sample types from these animals caters to an increasingly broad variety of researchers. Having the samples to address retrospective and geographic comparative analyses is important for studying the rates and characteristics of natural and anthropogenic changes.}

\normaltalk{\capitalisewords{Effects of Colony Size, Disturbance, and Defensive Behavior on Least Tern Productivity on the Mississippi Gulf Coast}}{\underline{Abigail J Darrah}}{The man-made beaches of Harrison County host up to 2500 breeding pairs of Least Terns, making this region a critical area for this declining species. Harrison County beaches are heavily managed and used for recreation, thus conservation efforts are currently directed toward minimizing human disturbance at breeding colonies. However, few studies have quantified how disturbance affects productivity of Least Tern colonies. During the breeding seasons of 2017 and 2018, observers surveyed all mainland colonies bi-weekly to count the number of active nests, downy chicks, and fledglings present. Eleven colonies across both years were selected for intensive study, in which observers followed the fates of individually-marked nests and conducted disturbance observations from outside the colony. Disturbance observations were conducted for 2-6 hours per day, and the observer recorded any source of disturbance, duration of disturbance in minutes, and estimated the proportion of the colony that reacted by flushing or engaging in defensive behavior. Nest survival probability followed a bell-shaped pattern, with lower survival probability early and late in the season, and was greater in large colonies. Nest and chick survival probability showed little relationship with the extent of disturbance recorded, but both productivity metrics were negatively related the quantity of time the terns spent flushed or engaged in defense throughout the season. Continued analyses may reveal differential responses to particular sources of disturbance, which may have implications for improving protective measures or using social attraction to move colony sites away from disturbance sources that cause the terns to flush.}

\normaltalk{\capitalisewords{Birds Eye View: 3D Digitization of the Moore Lab of Zoology's Holotype Specimens}}{\underline{Josh J Medina}, James M Maley, Haelan J Nunn, Kevin D Grundy, Edward Valdez, John E McCormack}{Access to specimens in natural history specimens can be a bottleneck in biodiversity research, especially when specimens are rare or fragile and cannot be loaned. Virtual Reality (VR) and photogrammetry allow for cutting-edge specimen digitization that allows viewers to access rare and fragile specimens from afar, and create comprehensive 3D archives for preserving valuable collections. Here, we introduce a virtual environment called Birds Eye View, which is an interactive digital archive containing type and rare specimens from the world's largest Mexican bird collection, the Moore Laboratory of Zoology. The specimens are hosted in a digital archive that can be accessed in VR and shared online. Guests of the archive can manipulate and measure birds in virtual space, access collection data unique to each specimen, and create digital copies scaled all the way up to 500\% of their actual size. We outline a photogrammetry protocol, which utilizes photographs and software to construct 3D scans. With all the affordability and non- invasiveness of traditional photography, photogrammetry accounts for a new dimension for collections digitization. The archive is both a database and an interface, allowing a collection to be shared and showcased nearly anywhere, at any time, by anyone. It also serves as a prototype for creating future 3D digital archives and collections. Using photogrammetry and VR, Birds Eye View seeks to push the limits of collections access, demonstrating an intuitive new way to access, showcase, and study what is for many the inaccessible.}

\normaltalk{\capitalisewords{The Ornithology Collection of the Natural History Museum of Los Angeles County (LACM)}}{\underline{Allison J Shultz}}{The Ornithology Collection of the Natural History Museum of Los Angeles County has over 120,000 bird specimens with global representation. The collection has over 97,000 skin specimens, 14,000 skeletons, 2,500 fluid specimens, and 4,000 tissue samples. I analyze the collection holdings, including growth through time, geographic strengths, major acquisitions, and taxonomic strengths. The collection holds many opportunities for ornithological studies, including a biannual grant for students to visit the collection for research.}

\normaltalk{\capitalisewords{Genetic Variation on Populations of Diglossa Baritula (Aves: Thraupidae)}}{\underline{Alondra K Terrones Ram\'{i}rez}, Blanca E Hern\'{a}ndez Ba\~{n}os}{The cinnamon-bellied flowerpiercer, Diglossa baritula, is a species complex endemic to the montane forests of Mesoamerica. There are currently three subspecies described based on patterns of coloration and distribution. D. baritula, represents a good model to search a pattern of genetic variation because it presents a disjunct distribution and plumage color differentiation. We expect that each disjunct populations have a defined genetic structure. We investigated genetic variation of populations of D. baritula based on sequence data of 870 pb of mitochondrial gene (mtDNA), ND2, from 54 tissue samples representing the three subespecies. Our results revealed the presence of two mtDNA genetic groups corresponding to subspecies D. b. baritula distributed from Jalisco to Oaxaca in Mexico and D. b. montana/D. b. parva distributed from Chiapas in Mexico to Nicaragua. The lineage split can be explained as a consequence of relative isolation of the populations separated by the Isthmus of Tehuantepec. The Isthmus of Tehuantepec is a barrier that has played an important role in shaping the avian diversity of Mesoamerica. D. baritula is made up of two phylogeographically structured clades that probably diverged roughly 785000 years ago. Therefore, the recent history of D. baritula was likely shaped by historical habitat fragmentation due to fluctuating environmental conditions in the Pleistocene. These results highlight the importance of the Isthmus of Tehuantepec and Pleistocene climatic events in driving isolation and population divergence.}

\normaltalk{\capitalisewords{Additional Evidence Suggests Elevating Numerous Philippine Bird Subspecies to Species Status}}{\underline{Kyle K Campbell}, Kevin S Winker}{Current taxonomy of birds in the Philippines grossly underestimates species-level avian diversity. Recently proposed splits suggest that many endemic island populations of Philippine birds, described as subspecies, merit elevation to full species. In a previous study (Campbell et al. 2016, PloSone e0159325), we identified highly divergent populations within more than half of the 48 Philippine bird species sampled. Fully 29 endemic populations representing 25 species surpassed conservative conceptual thresholds either genetically, phenotypically, or both. Extrapolating from this, current estimates of avian diversity and endemism may be underestimated in the Philippines by more than 37\%. Here we examine additional evidence for the elevation of these populations of Philippine birds to full species, recommending taxonomic revisions to the Philippines' unique and diverse avifauna.}

\normaltalk{\capitalisewords{Golden Eagle Use of Water Features in the West Desert of Utah}}{\underline{Danielle K Finlayson}, Randy T Larsen, Robert N Knight, Samantha E Phillips, Steven J Slater}{Golden Eagles (Aquila chrysaetos) are a species for which there is some conservation concern. Previous research has covered many aspects of Golden Eagle ecology, but little is known regarding their use of water. It has been assumed that raptors get their water exclusively from prey and therefore do not need to drink, but anecdotal observations suggest otherwise. Our objective is to quantify the use of water by Golden Eagles and document their behavior at these water sources. We predict water features to be most used in the summer months when temperatures are high. We also predict a greater tendency for young birds to use these features, due to inexperienced hunting skills. Finally, we predict drinking to be the primary reason for use, with bathing and other needs as secondary uses. We used remote cameras beginning in 2010 to photograph wildlife at water sources in Utah's west desert. We collected over 3.5 million photos from 90 natural springs and 26 artificial water developments. Photos of eagles are now being sorted by visitation events recording behavior, age class, and number of individuals in a given photo. In our analysis we will look at how usage rate varies by age and season, as well as recording behavior at water sources (drinking, bathing, or other). Use of natural springs and wildlife developments by Golden Eagles could have important implications for how both eagles and water features are managed in arid environments – particularly as availability of water is reduced in coming decades.}

\normaltalk{\capitalisewords{On the Basis of Sex and Age: Differences in Autumn Migration Phenology of Western Migrants}}{\underline{Aya K Pickett}, Robert A Miller, Heidi W Carlisle, Jay D Carlisle}{Though changes in avian migration timing have been well documented in the spring, there is relatively little information known about autumn migration. Recent publications have called for further exploration of migration timing according to differences in phenological traits. Here, we look at autumn migration of passerines in the Intermountain West to examine differences in timing between age and sex classes and learn how these trends may have changed over time. We used banding data from Intermountain Bird Observatory in southwestern Idaho between July 16 and October 15 from 1997 to 2017. We broadly examine the songbird community as a whole (32 species with an average of at least 10 captures in each year) and individually examine specific species based on abundance and relative ease of determining age and/or sex – some examples include Ruby-crowned Kinglet, Yellow Warbler, Wilson's Warbler, and White-crowned Sparrow. For many of the species we analyzed, migration timing differed significantly by both age and sex. For the songbird community as a whole, we found that migration trended progressively earlier over time. Due to a changing global climate, trends and changes in autumn migration are important to document as uncertainties surrounding availability of resources intensify. Understanding the full life history of birds is necessary to properly document change and make efforts towards conservation.}

\normaltalk{\capitalisewords{Genomic Data Suggest East Asian Mew Gulls and North American Mew Gulls Belong to Separate Biological Species}}{\underline{Symcha L Gillette}, Jack J Withrow, Travis C Glenn, Brant C Faircloth, Kevin Winker}{The Mew Gull (Larus canus) complex is comprised of Eurasian (L. c. canus and L. c. heinei), east Asian (L. c. camtschatchensis), and North American (L. c. brachyrhynchus) populations. The taxonomic classification of the species varies, with L. c. brachyrhynchus sometimes considered a separate species. Using ultraconserved elements, loci that are highly conserved across avian genomes, we studied the population genomics of camtschatchensis and brachyrhynchus to assess whether subspecies or species status is more appropriate for brachyrhynchus. We sequenced DNA from five North American and five Asian individuals and obtained a dataset consisting of 2,579 variable loci, with both alleles called. Using one single nucleotide polymorphism (SNP) per locus, we found significant differences between the populations (FST = 0.14, P = 0.013). The best fitting demographic model (using δaδi), was one of secondary contact and skewed bidirectional gene flow. Split time was estimated to be about 82 Kyr ago, with secondary contact beginning about 20 Kyr ago, at the end of the last glacial maximum. This speciation event thus appears to be the product of the most recent glacial cycles in Beringia. Gene flow occurs at ~1.4 individuals per generation from Asia to North America, and ~0 per generation in the other direction. Based on our results, we propose that L. brachyrhynchus be considered a full biological species.}

\normaltalk{\capitalisewords{Evaluating the Effectiveness of Dispersal Barriers in the Negro River Basin for White Sand Associated Passerines}}{\underline{Rachel L Wadleigh}, Camila C Ribas, L L Knowles}{The propensity to disperse across barriers varies broadly by taxa and barrier type. All birds face trade-offs between the potential costs of dispersal (e.g. higher predation risk, greater energy expense) and possible rewards of reaching better habitat, but the relative cost and reward varies (Bonte et al 2012). While genetic structure varies significantly across different barrier types, mixed support often exists across taxa for any given barrier indicating a need to better understand their relative strengths for different taxa (Naka and Brumfield 2018, Boubli et al 2014, Nazareno et al 2017). Better understanding these dispersal drivers and barriers will help to predict large scale evolutionary and ecological patterns. Furthermore, knowing how different species respond to shared environmental change allows for informed trait-based hypotheses about expected concordance and historical inference of environmental events, such as climate change (Papadopoulou and Knowles 2016, Capurucho et al 2013). To understand the role of habitat heterogeneity and barriers in structuring avian populations and constraining movement, multiple barriers need to be evaluated and compared across taxa employing high resolution genomic markers. Here we evaluate the relative effectiveness of barriers in the Negro River Basin for four passerines associated with Amazonian white sand ecosystems by examining the genetic structure of populations separated by different rivers and habitat types. A reduced representation genomic library was constructed using double digest RADseq to evaluate the strength of river and habitat barriers, and whether dispersal probabilities differ as a function of species and barrier type.}

\normaltalk{\capitalisewords{Why is Amazonia a 'source' of Biodiversity? Climate-Mediated Dispersal and Synchronous Speciation Across the Andes in an Avian Group (Tityrinae)}}{Lukas J Musher, Mateus Ferreira, \underline{Anya L Auerbach}, Jessica McKay, Joel Cracraft}{Amazonia is a 'source' of biodiversity for other Neotropical ecosystems, but which conditions trigger in situ speciation and emigration is contentious. Three hypotheses for how communities have assembled include (1) a stochastic model wherein chance dispersal events lead to gradual emigration and species accumulation, (2) diversity-dependence wherein successful dispersal events decline through time due to ecological limits, and (3) barrier displacement wherein environmental change facilitates dispersal to other biomes via transient habitat corridors. We sequenced thousands of molecular markers for the Neotropical Tityrinae (Aves) and applied a novel filtering protocol to identify loci with high utility for dated phylogenomics. We used these loci to estimate divergence times and model Tityrinae's evolutionary history. We detected a prominent role for speciation driven by barriers including synchronous speciation across the Andes, and found that dispersal increased toward the present. Because diversification increased but dispersal was non-random over time, we show that barrier displacement better explains Tityrinae's history than stochasticity or diversity-dependence. We propose that Amazonia is a source of biodiversity because (1) it is a relic of a biome that was once more extensive, (2) environmentally-mediated corridors facilitated emigration, and (3) constant diversification is attributed to a spatially heterogeneous landscape that is perpetually dynamic through time.}

\normaltalk{\capitalisewords{Exploration of the Structure and Function of the Casque in Cassowaries}}{\underline{Andrew L Mack}, Joshua Jones}{Both male and female cassowaries (Casuariidae) have distinctive casques that are superficially similar to casques found in many fossil taxa (e.g., Corythoraptor spp.).  The function of the casque in cassowaries and dinosaurs has received widespread speculation.  We are studying the possible functions of the casque through field observations, analysis of vocalizations, anatomical imagery (e.g., CT Scans, MRI), and dissection.  Several hypothesized functions (e.g., a foraging tool, a protective "helmet") can be dismissed based on observations.  Other functions (e.g., a sounding chamber for vocalizations, an olfactory organ) are unlikely based on anatomy.  The most widely accepted hypothesis has been as a secondary sexual characteristic.  But the complex internal structure of the casque suggests more is involved (although a visual signal is not excluded).  We are exploring possible roles in sound reception (cassowary vocalizations extend into very low frequencies: 20 Hz) and thermal regulation.  Our thermal imagery of cassowaries in different ambient conditions, and at different times in the same ambient temperatures, indicate blood flow into the casque varies depending on ambient temperature and possibly with behavioral circumstances.  This suggests blood flow to the casque varies and the casque is involved in thermoregulation.  But anatomical evidence suggests thermoregulation is not the sole, or even primary function of the casque.}

\normaltalk{\capitalisewords{Avian Feeding Behavior in Response to Alpha-Pinene Olfactory Cues}}{\underline{Marissa L Trombley}, Kyra A Brazell, Alec R Lindsay}{Many trees emit volatile organic compounds in response to harm caused by the foraging of herbivorous insects. Some species of insectivorous birds are able to detect specific tree volatiles and use them as an olfactory cue for foraging. Few studies have focused on passerine olfaction and coniferous tree volatiles in a natural setting, so we aimed to test whether songbirds native to the Upper Peninsula of Michigan change their feeding behavior when presented with feeders treated with typical volatile chemicals emitted by conifers in their natural habitat. To test this, we first acclimated winter mixed foraging flocks to custom feeders that were filled with black oil sunflower seed. On experimental days we presented birds with two feeders, one with alpha-pinene volatiles and the other with a control volatile (1-hexanol). Avian behavior was measured with the use of video cameras and RFID trackers, and subsequently analyzed for visitation preference. Our data provide an initial examination of the potential role coniferous volatile chemicals play in influencing avian foraging behavior.}

\normaltalk{\capitalisewords{Winter Occurrence and Roosting Behavior of Northern Saw-Whet Owl in Northwestern Arkansas (USA)}}{\underline{Mitchell L Pruitt}, Kimberly G Smith}{The Northern Saw-whet Owl (Aegolius acadicus) is a well-documented autumn migrant, especially in eastern and central North America. Despite large numbers captured each year at autumn banding stations, relatively little is known about where the species spends winter. The currently accepted extent of saw-whet owl's winter range in the central U.S. spans from central Kansas and central Missouri to eastern Tennessee. However, the species is thought to winter much further south than was historically believed. Northern Saw-whet Owls have recently been documented migrating through the southwestern Ozark Highlands. These observations and sporadic historic records during winter, led us to examine winter occurrence of the species in northwestern Arkansas. During autumn migrations of 2016 and 2017, 27 radio transmitters were deployed on saw-whet owls captured at our banding site in Madison County, Arkansas. Detections were obtained from 17 tagged individuals during winter following each migration season, ranging from 1 to 112 days after release, suggesting some saw-whet owls winter in northwestern Arkansas. Locations of 19 roost sites, most in shortleaf pine (Pinus echinata), were obtained from 7 individuals. All roost sites were in coniferous trees, surrounded by open understory. Landscape-scale habitat characteristics were obtained through GIS analysis of roost sites. Results suggested saw-whet owls selected roost sites in conifers in areas with more southwesterly mean aspect values. Further study is required to determine the full extent of Northern Saw-whet Owl's winter range; however, the abundance of pine and cedar could provide optimal wintering habitat in the Ozark Highlands.}

\normaltalk{\capitalisewords{Spring 2016-2018 Migration Counts at North America's Northernmost Raptor Migration Site: Gunsight Mountain, Alaska}}{\underline{Jesse L Watson}, Neil Paprocki, Dave Oleyar, Travis Booms}{Raptor migration counts are a common approach to assess regional populations for taxa that are difficult to monitor via other survey methods and widely dispersed on breeding and wintering grounds.  Spring migration counts number fewer than fall counts, but represent the survivors returning from wintering grounds, and potential recruits into breeding populations. In 2016, we conducted the first full season raptor migration count at Gunsight Mountain, Alaska; a known raptor migration corridor since the 1980s and the northernmost count site in North America,  Our goals were to fill knowledge gaps for raptor population size and trends in the state of Alaska.  Furthermore, we were interested in increasing the understanding of the size and status of Alaska's Golden Eagle (Aquila chrysaetos) population, a species known to migrate past Gunsight Mountain in large numbers.  From 2016-2018, annual counts averaged 508 hours of survey time, 67 days per season, and 2,872 total raptors.  The species composition at this site is unique with Golden Eagle and Harlan's Red-tailed Hawk (Buteo jamaicensis harlani) making up the majority of the count followed by Rough-legged Hawk (Buteo lagopus), Northern Harrier (Circus hudsonius), and Northern Goshawk (Accipiter gentilis).  Data from Gunsight Mountain provide valuable baseline estimates of raptor populations in the region and a solid foundation for future research conducted within the state.  We will present a comprehensive overview of the count data collected from 2016-2018 and will discuss plans for future raptor migration monitoring at northern latitudes.}

\normaltalk{\capitalisewords{Demographic Models to Support the Endangered Species Act Listing Decision for Golden-Winged Warblers}}{\underline{Emily L Weiser}, Wayne E Thogmartin, Tom C Will}{Golden-winged Warblers are declining and shifting their range in response to changing conditions and hybridization with Blue-winged Warblers. As a result, Golden-winged Warblers are now being considered for listing under the Endangered Species Act in the U.S. To support the U.S. Fish and Wildlife Service in evaluating information that can be used toward the listing decision, we are developing a population viability analysis for Golden-winged Warblers. As the first step to informing the analysis, we quantified true annual survival rates of adults from across the breeding range (Great Lakes and Appalachia). Previously, most estimates of annual survival in birds have been unable to distinguish between emigration and mortality, resulting in an apparent survival rate that is biased low. A recently developed spatially explicit Cormack-Jolly-Seber (sCJS) model corrects for emigration and provides an estimate of true survival. We applied the sCJS model to data from color-marked Golden-winged Warblers from several study areas to quantify annual survival rates in each population. These estimates will be used in a population viability analysis to help inform the ESA listing decision. The final author list for this presentation will include a number of researchers who generously contribute their data and expertise to this work; please visit the poster to see the full list.}

\normaltalk{\capitalisewords{Potential Environmental Predictors, Blood Meal Analysis, and Larvae Presence of an Important West Nile Virus Vector (Culex Restuans) in Ruffed Grouse Habitat}}{\underline{Samantha L Maywald}, Clay E Corbin, Lisa M Williams, John M Hrantiz, Karl W Henry}{Since 2001, Ruffed Grouse (Bonasa umbellus) population in Pennsylvania have been declining, this was correlated with the prevalence of West Nile Virus (WNV). While there are many studies of environmental predictors of mosquito abundance in urban habitats, little is known about WNV vectors in forests. We think the same environmental factors in urban habitats (water availability, topography, and bird hosts) will predict the abundance of Culex restuans, an important WNV vector, in forested Ruffed Grouse habitat. Mosquitos were surveyed mid-late summer at eight sites in Luzerne county using paired gravid mosquito traps. Samples (1/week x 13weeks) were sent to PADEP for identification and assay of WNV. Mosquito densities ranged from 46 to 624 per site and WNV was positive in five sites. We collected water samples in September at possible larval pools to test for total organic carbon and SUVA. Conducted blood meal analysis on mosquitoes to determine avian hosts. We used regression analysis to test if elevation, distance from human activity, temperature, species richness, and humidity were predictors of mosquito abundance. Across sites, humidity, humidity + temperature, and species richness were significant. Total organic carbon and SUVA predicted that when there is high organic carbon in water than there should be larval presence. Other local effects such as aspects of hydrography and vegetation may be more important in determining the dynamics of WNV vectors.  These predictors, a second season of data (summer 2019), and blood meal analysis, will help provide a habitat management plan for Ruffed Grouse in Pennsylvania.}

\normaltalk{\capitalisewords{The Macaulay Library: An Increasingly Powerful Resource for Research}}{\underline{Kathi L Borgmann}, Eliot Miller}{The Macaulay Library at the Cornell Lab of Ornithology is the world's largest curated collection of ornithological multimedia specimens. The Library has been in existence for 90 years and currently contains approximately 11 million specimens. More than 1,000 scientific publications have been published using multimedia assets from the Macaulay Library, but until fairly recently input (scientific accessioning) and output (access to resources) of the collection required the assistance of the Library's professional sound archivists. With the recent integration of the Library and the global citizen science project eBird, which permits the creation of multimedia specimens by members of the public, input of bird audio and photo specimens has grown exponentially. The research potential of the archive is enormous and is growing daily, and we encourage researchers to tap into this valuable resource; to date, photos have been particularly underutilized for research. Moreover, the Macaulay audio collection is growing at a rapid pace and provides research opportunities on a diversity of questions. This poster will highlight and discuss current best practices for obtaining access to Library resources, including high-quality sound recordings; strengths of the collection as exemplified by recent scientific papers; and ongoing efforts to further facilitate output of multimedia assets for researchers.}

\normaltalk{\capitalisewords{Won't you be My Neighbor? Distribution of Nesting Pairs Within a Hybrid-Zone Chickadee Population}}{\underline{Breanna L Bennett}, Emily S Burton, Robert J Driver, Robert L Curry}{Parapatric Black-capped Chickadees (Poecile atricapillus; BCCH) and Carolina Chickadees (P. carolinensis; CACH) hybridize, producing viable and nonsterile offspring (HYCH), in a narrow zone from Kansas to New Jersey. Our long-term field studies in southeastern Pennsylvania and genetic analyses have shown that Carolina Chickadees are expanding northward, in association with warming winter conditions, at the expense of range-contracting Black-capped Chickadees. Here, we examine the spatial distribution within one hybrid-zone population (Hawk Mountain), as numbers of Carolina and hybrid chickadees increased between 2001 and 2018, to test whether settlement facilitated or impaired hybridization. We focused on four different years for which both geospatial and genetic data existed for breeding pairs, 2005, 2010, 2015, and 2018 (n = 48). In 2005, 60\% of pairs were BCCH x BCCH, 15\% were BCCH x CACH, and 15\% were BCCH x HYCH. By 2010, pairs were 25\% BCCH x HYCH, 25\% BCCH x CACH, and 25\% CACH x HYCH, with other pairs mixed in. In 2015, 35\% of pairs were HYCH x HYCH, with small percentages of many other pair combinations mixed in, and by 2018 HYCH x HYCH pairs increased to 59\% of all pairs. Analysis in progress examined whether females or males of a given genotype create genetic "neighborhoods", i.e. geospatial clustering of certain same-genotype individuals. Available evidence suggests that Carolina and hybrid immigrants settled randomly within the Hawk Mountain population, in a pattern that neither promoted nor impeded hybridization.}

\normaltalk{\capitalisewords{Population-Level Effects of a Record-Setting Spring Snowstorm are Related to Diet and Migratory Strategy}}{\underline{Emily L Weiser}}{Extreme weather events can have measurable impacts on wild populations and are expected to continue increasing in frequency. In April 2018, Winter Storm Xanto brought unseasonable snowfall and cold temperatures to the northcentral U.S. and southcentral Canada. I used eBird data to evaluate potential effects of the storm on subsequent local occurrence of 59 species of birds. I tested for species-specific responses in occurrence to the local duration of snow cover and mean minimum temperature during the storm, as well as the interaction between the two variables. I then evaluated whether species-specific responses were related to species traits, such as diet and timing of migration.  This study provides new information about how wild bird populations respond to anomalous spring snowstorms and identifies which guilds might be most vulnerable as extreme weather events continue to increase in frequency.}

\normaltalk{\capitalisewords{A Novel Device to Best Detect Key Woodpeckers: An Experiment with the Magellanic Woodpecker}}{\underline{Amy L Wynia}, Virginie Rolland, Jaime E Jimenez}{Woodpeckers can be difficult to detect, as they are often cryptic, secretive, and wary of humans. Several methods exist to detect woodpeckers (e.g., callback surveys, passive point counts), yet no research has established which technique best detects these difficult picids. Thus, we designed an experiment to determine which of three methods best elicits a response (thus, a detection), from Magellanic Woodpeckers (Campephilus magellanicus). During austral summers 2015-2017, we used a novel device to simulate a double-knock (i.e., territorial acoustical signal), broadcasted a territorial call, or silently listened (control) for Magellanic Woodpeckers. We conducted our experiment on Navarino Island, Chile, where the Magellanic Woodpecker is the sole picid. The drumming device elicited a woodpecker response significantly more than the call and control. The odds of a woodpecker responding to a double-knock were 2.1 times more likely than responding to either a call or control. Significantly, wind negatively influenced the likelihood of a woodpecker responding; the odds of a woodpecker responding decreased by 42\% as wind category increased by one category. Further, woodpeckers were detected significantly less frequently later in the breeding season than earlier. For every additional month (i.e., Oct.-Mar.), the odds of detecting a woodpecker decreased by 16\%; this was expected because woodpeckers become less territorial as the breeding season progresses. In general, woodpeckers communicate via drums or double-knocks; therefore, using a drumming device will likely be an effective technique to detect woodpeckers not only within the Campephilus genus in Central and South America, but in other genera globally.}

\normaltalk{\capitalisewords{Nesting Success of Carolina Wrens in an Urban and a Rural Ecosystem}}{\underline{Diane L Neudorf}, Gabriel A Barragan}{More than 50\% of wildlife habitats have been altered or lost in Texas, and one of the leading causes of this loss is urbanization.  Urbanized habitats present birds with different challenges from those of natural areas.  Foraging areas may be limited due to less natural vegetation.  Human disturbances may impact foraging behavior and nestling feeding and there are differences in the predator community.  These differences may impact the reproductive success of birds in urbanized ecosystems.  Carolina Wrens (Thyrothorus ludovicianus) are resident, cavity-nesting songbirds of the eastern United States that inhabit both forest ecosystems and urban ecosystems with large trees.  Over an eight year period we erected nest boxes in both the City of Huntsville, TX and the Sam Houston State University field station to compare nesting success of wrens in the two areas.  We found differences in clutch size and nesting success between the two sites that varied with year.  We will discuss the implications of our findings for urban nesting birds.}

\normaltalk{\capitalisewords{A Meta-Analysis of Effects of Wind Energy Development on Grouse Survival and Behavior}}{\underline{Karl L Kosciuch}, Shay Howlin, Chad LeBeau}{Anthropogenic influences that fragment grouse (Tetraoninae spp.) habitat could affect population growth rates, persistence, or occupancy through negative behavioral responses effectively leading to habitat loss. In the U.S., there is concern that wind energy development will exacerbate declining sage-grouse (Centrocercus spp.) and prairie-chicken (Tympanuchus spp.) populations. We examined the literature to identify papers that quantified the effect of wind energy on grouse survival or displacement and conducted a meta-analysis to transform the correlation coefficient into Fisher's z values. We developed a hierarchical random-effects model to account for non-independence of calculated effect sizes from the same study with a random effect term. Eleven studies resulting in 23 study-result combinations were included in the hierarchical random-effects modelling to estimate the displacement effect size. We found little evidence of displacement due to the presence wind turbines on the studied grouse species as a whole (Zr = 0.087; 95\% CI = -0.015, 0.188; p-value = 0.094; k = 23).  Six studies resulting in 11 study-result combinations were included in the hierarchical random-effects modelling estimating the survival effect size. We did not detect a significant adverse effect of wind turbines on survival for the studied grouse species (Zr = -0.031; 95\% CI = -0.161, 0.100; p-value = 0.646; k = 11).  Overall, wind energy has limited effects on displacement and no measurable effect on survival.  Further, wind energy has a smaller effect on grouse displacement when compared to other anthropogenic disturbances such as oil and gas development and roads.}

\normaltalk{\capitalisewords{Title: Reaching Our Northern Limits: A Sampling Design for Comprehensive Avian Monitoring in Canada's Boreal Ecoregions}}{\underline{C L Mahon}, Logan McLeod, Margaret Campbell, Steven L Van Wilgenburg, Greg Campbell}{Monitoring is a key component of bird conservation and management. The effectiveness of long-term monitoring data is often limited by (1) coordination across regions, (2) relevant conservation objectives and questions, (3) rigorous sampling designs, and (4) adequate funds. Although long-term monitoring of breeding populations of terrestrial birds across boreal regions of Canada is a high priority, the design and implementation of avian monitoring programs is complicated by remote, rugged, and difficult to access landscapes. To address high priority data gaps across boreal ecoregions, we introduce the Boreal Monitoring Program (BMP). Our objectives are to outline the development of a statistically rigorous, cost effective, and feasible sampling design and to evaluate our proposed design against alternative designs (simple random sample, systematic, spatial balance). We outline the sampling and evaluation process for the BMP with a focus on three diverse jurisdictions in Canada's boreal region:  Newfoundland, Saskatchewan, and Yukon Territory. The collaboration across jurisdictions demonstrated use of common data layers, sampling approaches, and survey methods over large spatial scales with variable physiography, ecology, and bird species diversity. We designed the BMP as a flexible design that incorporates spatial balance, habitat representation, and cost with the option of incorporating existing legacy sites to reduce standard errors in mean estimates, increase sampling efficiency, and lower sampling costs. The BMP design provides a precedent for implementing a statistically defensible, efficient, and feasible (cost, operations, safety) design for large-scale monitoring of terrestrial birds across boreal ecoregions in Canada.}

\normaltalk{\capitalisewords{A Spatial Model of Streaked Horned Lark Breeding Habitat in the Columbia River, USA}}{\underline{Gary L Slater}, James R Hatten, Jerrmaine L Treadwell, Matt R Stevenson}{The Streaked Horned Lark (Eremophila alpestris strigata) is a federally listed subspecies of the Pacific Northwest. Across its breeding range, the lark relies primarily on human-modified habitats maintained in early successional states.  In the lower Columbia River, larks depend on deposition islands created via shipping channel maintenance projects.  The U.S. Army Corps of Engineers has proactively adopted a shifting-habitat-mosaic management approach when placing dredge materials to maintain lark habitat throughout the project area. To facilitate management, we created a geospatial tool capable of quantifying and tracking habitat suitability for larks throughout the project area.  We created spatial models of lark breeding habitat using lark survey data, satellite imagery, dredge-deposition maps, and surface-elevation models. We built candidate predictive models of lark breeding habitat after combining survey data (years 2016 and 2017), predictor variables, and logistic regression. In addition, we created a novel seral-stage predictor variable to identify young, mature, old, and unsuitable locations; critical information for lark conservation planning and dredge-disposal operational planning. We challenged probability models the year they were constructed (2016) and the following year (2017) with set-aside survey data. The best model contained terrain ruggedness, deposition age, vegetation greenness inside a 10-m cell, and heterogeneity in greenness within a 100-m radius. Verification accuracy ranged from 75-83\%. Importantly, the seral-stage variable allowed us to quantify and locate early, mature, and late habitats.  Model results will facilitate conservation planning by informing real-time management decisions. Furthermore, they provide a foundation for application in other human-modified habitats across the species' range.}

\normaltalk{\capitalisewords{Cerulean Warbler Habitat Selection, Breeding Biology, Survival, and Movements in the Ozark Region}}{\underline{Jacob L Wessels}, Than J Boves}{The Cerulean Warbler (Setophaga cerulea; 'ceruleans') is a declining species of Nearctic-Neotropical migratory songbird that breeds in deciduous forests of eastern North America. Because habitat loss and degradation are thought to be major causes of this decline, active habitat management has become an important part of the conservation strategy for ceruleans. However, ceruleans exhibit regional variation in habitat selection and reproductive success, which may complicate efforts to manage for them. Information on the ecology of local populations is therefore necessary to effectively manage for and conserve this species. Although there is a relatively large population of ceruleans in the Ozark region of Arkansas and Missouri, ceruleans have been understudied in that region. To address this important knowledge gap, we are studying their habitat selection, breeding biology, survival, and movements along the Buffalo National River in Arkansas and comparing the results with studies from other parts of their range. During our first field season, we captured and banded 28 males and estimated within-season apparent weekly survival at 0.962 (±0.0197 SE). We located 15 nests and estimated entire period nest success at 29\% (via logistic-exposure model). We estimated mean territory size at 2.149 ha (±0.281 SE). Ceruleans showed no significant preference for any tree species at territory, nest patch, or nest site scales. They mostly selected riparian habitat, often lacking tree species that ceruleans prefer elsewhere in their breeding distribution. This study will inform appropriate regional conservation for ceruleans and provide parameter estimates for use in integrative full-annual-cycle population models.}

\normaltalk{\capitalisewords{Is Geographical Prevalence of Beak Deformities Related to Patterns of Genetic Diversity in Northwestern Crows in Alaska?}}{\underline{Lisa M Pajot}, Colleen M Handel, Sarah A Sonsthagen, Caroline Van Hemert}{In Alaska, adult Northwestern Crows exhibit a high prevalence (16.9\%) of beak deformities associated with avian keratin disorder, the highest rate of gross deformities ever recorded in a wild bird population. Prevalence varies geographically, with sites in southeastern Alaska showing lower prevalence than sites in southcentral Alaska, suggestive of epizootic clusters. We investigated the population structure of Northwestern Crows in Alaska using mitochondrial DNA and microsatellite data to elucidate the relationship of dispersal of individuals relative to disease transmission. Specifically, we aimed to determine if the geographic prevalence of beak deformities was related to patterns of genetic differentiation, and if patterns of population structure along the North Pacific Coast could be explained by past glacial history or current physiographic barriers. We found that populations of Northwestern Crows in Alaska show a subdivision between southcentral and southeastern Alaska. Both nuclear and mitochondrial markers showed significant spatial genetic structure between southcentral and southeastern Alaska, and partitions in the genetic data appear to be influenced by topographic features (e.g., mountain ranges). The prevalence of beak deformities and the pattern of genetic diversity appear to be related, such that genetic diversity is lower in southcentral populations where a higher prevalence of beak deformities is observed. Future research examining the role of dispersal relative to disease transmission and the role of genes regulating beak growth and immune response could help clarify underlying processes producing the pattern of epizootic clusters associated with this disorder.}

\normaltalk{\capitalisewords{The Role of Climate Change in the Northward Range Expansion of Red-Bellied Woodpecker (Melanerpes Carolinus)}}{\underline{Erika M Mayette-Draper}, Joel Ralston}{Models of climate change predict increasing mean temperatures, and a more variable climate which will cause an increase in extreme weather. Red-bellied Woodpeckers (Melanerpes carolinus) have been identified as one of the most climate restricted species, because they are strongly limited by the average minimum temperature during a winter season. Additionally, this species has undergone a large range expansion in recent decades as climates have warmed. Ecological niche models (ENM) can show us how climate suitability is changing and how the species distributions are shifting in response. The objective for this study was to use an ENM to show spatial and temporal patterns in climate suitability across the expanding range of Red-bellied Woodpeckers over 115 years. An ENM was constructed using the program MAXENT and eight bioclimatic variables describing mean and extreme temperature and precipitation. We then used the ENM to model annual climate suitability from 1901-2015. A general decrease can be seen in suitability as latitude increases across the range. This decrease in suitability is marked by occasional periods of low suitability at high latitudes, likely representing extreme weather events, and potentially higher natural selection. However, this pattern weakened over time. Throughout the study period, suitability increased at higher latitudes, and inferred selective events became less extreme and less frequent.  These finding suggest that increased climate suitability at high latitudes as a result of climate change have permitted the northward expansion of this species.}

\normaltalk{\capitalisewords{What Drives Genetic and Phenotypic Divergence in the Red-Crowned Ant-Tanager (Habia Rubica, Aves: Cardinalidae)}}{\underline{Sandra M Ram\'{i}rez-Barrera}, Juli\'{a}n A Velasco, Tania M Orozco-T\'{e}llez, Alma M V\'{a}zquez-L'{o}pez, Blanca E Hern\'{a}ndez-Ba\~{n}os}{Within natural populations, genetic and phenotypic divergence may be influenced by factors such as sexual and natural selection, genetic drift and geographic isolation. So that, test the associations between genetic, phenotypic, geographic and environmental variation is the first step for understanding the relative contributions of the different drivers of genetic and phenotypic variation among populations within a species. In this work we evaluated the hypotheses of isolation by adaptation (IBA), isolation by environment (IBE) and isolation by distance (IBD), using a Multiple Matrix Regression with Randomization (MMRR) approach. The main target was to disentangle the relative influence of geographic and environmental distances on genetic (ND2 sequences, 1041 bp) and phenotypic (coloration and morphometric data) differentiation of the Red-crowned Ant-tanager (Habia rubica). This species is a good model for performing tests of IBD, IBA and IBE, because It is a highly polytypic species, within which up to seven phylogroups have been described, besides that it has a continental distribution that goes from central Mexico to north-eastern Argentina and southeastern Brazil, that encompasses a variety of suitable environments, and has an extensive geographically structured color variation, that is well documented. Our results provide strong evidence that geographic distance is the main driver of genetic variation in Habia rubica. We did not find evidence that climate variation is driving population differentiation in this species across a widespread geographical region. Furthermore, there was no substantial evidence that phenotypic variation (i.e., body size and plumage coloration) has influenced genetic differentiation between populations.}

\normaltalk{\capitalisewords{Raptor Evolution in North America Since the Pleistocene}}{\underline{Oona M Takano}, David W Steadman}{The Quaternary extinctions in North America marked the decline of large mammals approximately 50–10 Ka. The mammal extinctions likely caused changes in raptor diets, because of reduced competition pressure from larger carnivores and less opportunities for scavenging as a result of decreased availability of carrion. The extinction of larger predators may have allowed a shift of raptors in ecological niche space to occupy an apex predator position. The La Brea Tar Pits in California have yielded an unparalleled number of raptor fossils, primarily Golden Eagles (Aquila chrysaetos), spanning glacial periods in the Pleistocene to the post-glacial Holocene. We investigated changes in raptor morphology since the Pleistocene, and how morphological changes relate to dietary niche. For raptors, toe size and shape are important traits determining the type of prey captured. We compared morphology of 8 species of large birds from fossils found at the La Brea Tar Pits: two raptor species (Golden and Bald Eagles) that have persisted from the Pleistocene to the present, and three scavenger genera (vultures and caracaras) that each have an extinct Pleistocene and extant modern representative. All toe elements of each species were measured with digital calipers. Digital photographs of fossils were taken for geometric morphometric analyses, in order to calculate claw curvature and shape. Examining potential climate-caused shifts in raptor diet will allow a better understanding of changes in avian communities over time, and may have important conservation implications for species such as the endangered California Condor, in anticipation of future climate change.}

\normaltalk{\capitalisewords{Technology and Student Outcomes when Integrating Undergraduate Researchers in Egg Logger Development}}{\underline{Ashley M Ridlon}, Constance J Woodman}{Nesting and incubation are very important aspects of avian conservation. For endangered species, it is even more critical to understand these early life stages as maximizing hatch rate in wild and captive birds is important for propagation and relief from population bottlenecks. Increasingly simplified fabrication methods have allowed technologies relevant to incubation to be developed during the course of normal laboratory work. In our lab, undergraduate students have been successfully  integrated into egg logger technology development. The technology issues faced include  true 3D recording, methods to record temperature gradients, issues of humidity recording, clock accuracy, 3D printing in single or full color, and non-toxic gel or metal-based weighting. Logger development has direct implications for management as high-quality data informs decision making. Examples of egg logger use for Peruvian wild macaws and U.S. captive macaws are shared to show how field studies can be a part of technology testing and improvement. The integration of young scientists into technology development sets an expectation for future researchers that technology can be produced rapidly  and more widely utilized from within ornithological laboratories.}

\normaltalk{\capitalisewords{Inter- And Intra-Specific Variance in Boldness Behavior of Hybridizing Black-Capped and Carolina Chickadees}}{\underline{Taylor M Heuermann}, Robert L Curry}{Animal personality, behavior consistent within but variable among individuals, influences evolution in both predictable and changing environments. Our research program focuses on interactions between Black-capped (Poecile atricapillus) and Carolina (P. carolinensis) chickadees across their hybrid zone in southeastern Pennsylvania. As climate change drives the zone northward, understanding the species' response mechanisms to change becomes increasingly important. In this study, we investigated boldness in pure and mixed populations of Black-capped and Carolina chickadees. To measure boldness response, we used a motorized woodpecker decoy as a simulated predation threat at active nests. Most assays elicited a pair response, while video recordings provided individual-level behavioral measures. In terms of latency to make the first 'chick a dee' alarm call and total call rate, pairs in all populations behaved similarly; however, Black-capped chickadees (N = 16) gave fewer dee notes (less intense response) than chickadees from Carolina (N = 46) or hybrid-zone (N = 29) populations. Video analysis of spatial response remains in progress, though initial patterns from principal components analysis indicate that latency to approach and agitation levels in proximity to the decoy load onto PC1, while number of swoops and amount of time spent in proximity to the decoy load onto PC2. Genotypic analysis underway of the SNP830 polymorphism in DRD4, a personality-related gene, aims to determine whether a relationship exists between variations in boldness response and DRD4 SNP830 genotype.}

\normaltalk{\capitalisewords{The Effect of Predator Threat to Adults on Variation in Gray Catbird Nest Defense}}{\underline{Dominique M Varra}, Lucas J Redmond, Craig S Eroh, Jordan I Haq}{Reproductive success in birds is dependent on factors such as food availability, weather, and nest predation. Nest predation typically accounts for the bulk of nest failure and a variety of strategies have evolved that reduce the likelihood that nests are depredated. Nest defense is one strategy that could reduce the risk of nest predation. While this may lead to fitness gains by increasing reproductive success, there is a trade-off because adults risk injury or even death by defending their nests. Parental investment theory predicts, and empirical evidence supports, that adults would defend nests with either more and/or older offspring the most aggressively. Fewer studies, however, have explored how nest defense varies with the amount of risk specific predators pose to adults. We evaluated the relationship between threat level to adults and variation in nest defense in Gray Catbirds. We quantified catbird behavior during observations throughout the breeding season of 2018 with models of two different predators, a black rat snake which represented a high risk to adults and an eastern chipmunk which represented relatively little risk to adults. Preliminary results suggest that catbirds did not respond differently to the two predators. This implies they either lack the ability to differentiate between predators or that the predators used were perceived as equal threats. This study is ongoing as more data are needed to draw a firm conclusion.}

\normaltalk{\capitalisewords{The Behavior of Swainson's Hawks (Buteo Swainsoni) During Predation of Brazilian Free-Tailed Bats (Tadarida Brasiliensis) in Flight}}{\underline{Lillias M Zusi}, Laura Kloepper}{Predators that hunt in 3-dimensions and pursue prey that flee in 3-dimensions must use complex pursuit behaviors. While some research has been conducted examining the success rates of raptors predating bats, the types of pursuit behaviors that are most effective for catching bats has not been studied. In this study, we report a specific example of a predatory relationship taking place in 3-dimensions, a group of localized Swainson's Hawks (Buteo swainsoni) hunting a colony of Brazilian free-tailed bats (Tadarida brasiliensis), and characterize the hawks' pursuit behavior. We collected video and audio recordings of Swainson's hawk predation events on Brazilian free-tailed bats for 15 evenings in June 2018. We quantified the hawks' pursuit and prey capture behavior using criteria that broke down each attack sequence into approach type, angle of column entry, and final grab direction. We found significant associations between some of these behaviors and found that stoops and final grabs from above were associated with increased prey capture success. However, the most frequently observed behaviors were not associated with increased capture success. Our results suggest that Swainson's Hawks have not developed a specialized bat-hunting technique, which may be a result of the unreliability of these migratory bats as a food source, but more research is required to fully understand this predator-prey relationship.}

\normaltalk{\capitalisewords{What can a Feather Tell Us? a Study of Hydrogen Isotopes and the Migratory Patterns of Common Yellowthroats (Geothlypis Trichas)}}{Veronica T Schabert, Kelly A Roberts, Andrea Contina, \underline{Kristen M Covino}}{An understanding of the migratory patterns of songbirds is essential in making conservation decisions, understanding climate change implications, and can contribute to studies of species survival. Stable isotope analyses, specifically of feather hydrogen, have become a powerful tool for understanding the movement biology of Neotropical migrants who breed in North America and molt prior to their southward fall migration. This study analyzes feather hydrogen isotopes ratios of Common Yellowthroats (Geothlypis trichas), in order to better quantify their migratory patterns. Rectrices were collected from migrants at two stopover sites, Braddock Bay Bird Observatory in Greece, NY and Appledore Island Migration Station, ME, during the spring and fall migratory seasons of 2016 and 2017. Over 170 feather samples were cleaned and dried, the vanes were cut and packed into silver capsules, and isotopic determination was done by mass spectrometry. There were no within- or between-site differences in the feather isotope values between males and females or between age classes. Additionally, there were no seasonal differences in the feather isotope values at Braddock Bay and our lack of fall samples prevented this analysis of our Appledore Island site. Braddock Bay birds during spring had more depleted (more negative) feather isotope values compared to birds passing through Appledore Island, indicating that migrants passing through Braddock Bay are heading to more northerly breeding sites. Subsequent to these initial analyses, we will analyze these data both spatially (geographically) and temporally using probabilistic assignment methods to further reveal details of the migratory patterns of Common Yellowthroats.}

\normaltalk{\capitalisewords{Determining the Variation of Clock Allele Lengths Across Eleven Migratory and Resident Species}}{\underline{Linnea M Larson}, Joel Ralston}{Avian migration has been studied in depth, but the underlying genetic mechanisms remain largely unclear. One candidate gene, Clock, is believed to have an impact on migratory propensity and phenology. However, the relationship between Clock allele length and migratory behaviors appears to vary across species. Therefore, it may be informative to more systematically characterize variation in Clock across species with variable migratory strategies. In our research, we extracted DNA from 35 individuals across 11 species captured from March-May 2018 in northwestern Indiana. Nine of those species were migratory, while three were residents. We found variation in Clock allele lengths across species. The longest allele found was in Tufted Titmouse (Baeolophus bicolor) (200 bp) and the shortest was common in three species, Yellow-bellied Sapsucker (Sphyrapicus varius), Magnolia Warbler  Setophaga magnolia), and Yellow-rumped Warbler (Setophaga coronata) (186 bp). We found no difference in mean allele lengths or mean variance in allele lengths between migratory and resident species. However, sample sizes are still currently low, leading to low observed variation within species. These preliminary results support previous studies documenting variation in Clock alleles across species, but fail to explain differences in migratory strategy.}

\normaltalk{\capitalisewords{Molecular Differences of Warbling Vireos (Vireo Gilvus) Across a Contact Zone}}{\underline{Amanda M Carpenter}, Garth M Spellman, John Klicka, Theresa M Burg}{In North America, there are many east-west species pairs and for some of these, they meet at contact zones. An example of an east-west species pair is the Warbling Vireo (Vireo gilvus). Two western subspecies, Vireo gilvus swainsonii and Vireo gilvus brewsteri, and a single eastern subspecies, Vireo gilvus gilvus, have been described. To date there has been limited work on genetic differences of warbling vireo populations or their interactions along the contact zone. For example, V. g. brewsteri occupies a large portion of the range, yet no studies have examined genetic structure of this subspecies or compare genetic structure among all three subspecies. The contact zone is of interest because it extends from the southwest United States to northwest Canada, subspecies pairs meet at different areas along the contact zone, and all three subspecies meet at a central contact zone. Examining contact zones provides insight into evolutionary processes and their contributions to biodiversity. My main research questions include: 1) Are the eastern and western subspecies genetically different? 2) Are the two western subspecies genetically different? and 3) What is the distribution of the three subspecies in western North America? I have genotyped 503 warbling vireo samples for 11 variable microsatellite loci and sequenced a subset of samples from each subspecies for two mitochondrial DNA markers, cytochrome b and ATPase 6-8, to look at contemporary genetic structure and maternal lineages. Preliminary data show three genetic groups and four populations in the swainsonii range are grouping with brewsteri.}

\normaltalk{\capitalisewords{Estimating Age- And Sex-Specific Annual Survival of Wood Ducks (Aix Sponsa) in Iowa}}{\underline{Tyler M Harms}, Stephen J Dinsmore, Orrin E Jones III}{The Wood Duck (Aix sponsa) is one of the most recognized waterfowl and an important game species throughout eastern North America.  There has been no effort, however, to estimate annual survival of Wood Ducks across different age and sex classes anywhere in the Midwest, nor have potential changes to survival and recovery rates of Wood Ducks been evaluated as a function of regulations changes in Iowa, USA.  We estimated both age- and sex-specific survival and recovery rates of Wood Ducks in Iowa from 2006 to 2016 using band-recovery data.  Additionally, we investigated changes in these parameters in response to the implementation of an early hunting season for teal in Iowa and a change in the daily bag limit for Wood Ducks (2 to 3 birds).  Annual survival estimates (SE) ranged from 0.40 (±0.01) for hatch-year females to 0.57 (±0.01) for after-hatch-year males and were not affected by either regulation change.  Recovery rates (SE) ranged from 0.18 (±0.003) to 0.23 (±0.003) for hatch-year females and hatch-year males, respectively, and were not influenced by the increase in daily bag limit.  However, recovery rates for females of both age classes decreased after the implementation of the early teal season, suggesting females are less vulnerable to harvest during years with an early teal season.  These represent the first estimates of annual survival of Wood Ducks in the Midwest U.S. and will be useful for informing future decisions on changes to waterfowl hunting regulations in Iowa and in other midcontinent states.}

\normaltalk{\capitalisewords{Haemosporidians in Vireos: Insights on Coevolution from a Comparison of Elevational-Replacement Species}}{\underline{Selina M Bauernfeind}, Daniele L Wiley, Jessie L Williamson, Silas E Fischer, Henry M Streby, Christopher C Witt, Lisa N Barrow}{Avian haemosporidian parasites provide a lens through which to examine host-parasite interactions and evolutionary dynamics. The vireos (family Vireonidae) appear to be highly susceptible to haemosporidian infection and may be coevolving with a particular clade of haemosporidians in the genus Parahaemoproteus. Here we compared infection rates, parasitemia, and haemosporidian community composition between two highly infected Vireo species in the Western U.S. The Plumbeous Vireo (Vireo plumbeus) and Gray Vireo (Vireo vicinior) are elevational replacements along a rainfall gradient; V. vicinior occupies lower, drier, piñon-juniper habitats, while V. plumbeus occupies mid-elevation piñon-ponderosa forests. We screened V. vicinior (n=192) and V. plumbeus (n=39) samples from the southwestern U.S. using microscopy and PCR to amplify the mitochondrial 'barcode', cytb. We compared our results with the avian malaria database, MalAvi, and previous studies of vireo species. V. vicinior was ~71\% infected with mean parasitemia of 0.90\%. V. plumbeus was 82\% infected, with mean parasitemia of 0.71\%. We found some host-generalist lineages, along with a suite of closely related Parahaemoproteus lineages that appear to be specialized on vireos. Some haplotypes were shared between host species, but most were unique, and diversity was strikingly different between the two elevational replacement host species. In a broader context, our study confirms that vireos, even in arid habitats, appear to host a distinctive, prolific, and recently diversifying haemosporidian fauna.}

\normaltalk{\capitalisewords{Do Birds Alter Winter Call Structure and Use in Response to Anthropogenic Noise?}}{\underline{Kristina M Hannam}, Leeann Bruetsch}{Bird vocalizations serve many functions: songs communicate territorial and mate attraction information; calls communicate threats or location information. As humans increase the spread of roads and urban habitat, many birds are faced with masking noise that may interfere with vocal signals. Some birds respond to masking anthropogenic noise by altering the amplitude, frequency, length or timing of songs. There has been less focus on calls, known to be less variable than songs, used outside the breeding season and their response to anthropogenic noise. We tested the hypothesis that birds alter their winter call vocalizations to overcome masking anthropogenic noise. We recorded birds at winter feeding stations at 6 sites around Geneseo, NY, with differing anthropogenic traffic noise levels in 2016-17. We observed interactions at the feeders for an hour while recording vocalizations. In 2018 -19 we used passive acoustic recorders to assess call use relative to changing traffic noise levels over the course of the day. We examined whether birds in noise-impacted habitats altered the frequency, length or timing of their vocalizations and visits to the feeders. In 2016-17, Black-capped Chickadees (Poecile atricapillus) increased their call duration at noisy sites, and American Goldfinches (Spinus tristis) increased the high frequency of their calls at noisy sites. We expect less use of calls at times of high traffic in the 2018-19 recordings, and individuals at noisy sites to show altered calls throughout the day. The results of this study will help shed light on the plasticity of call responses to anthropogenic noise.}

\normaltalk{\capitalisewords{The Evolutionary History of Seaside Sparrows in Florida}}{\underline{Carolyn M Enloe}, Rebecca T Kimball, Akanksha Pandey, W A Cox}{Subspecies designations are often based on phenotypic differences in plumage color and/or behavior that can be contradicted by genetic structure revealed by molecular methods. Such taxonomic uncertainty can have major repercussions on conservation prioritization, listing status, and in turn, funding allocations. The increasing availability of molecular techniques, such as high throughput sequencing, allows for traditional subspecies relationships to be re-examined using modern techniques to determine if genetic evidence supports their status as discrete reproductive units. Relationships between the subspecies of Seaside Sparrow on the Gulf of Mexico coast of Florida are poorly understood and there is some disagreement as to whether or not there are separate subspecies. Currently, there are up to three recognized subspecies of Seaside Sparrow on the Florida's Gulf of Mexico coast, two of which (Ammospiza maritima peninsulae and A. m. juncicola) are endemic to Florida and are state-listed as Threatened. We evaluated the genetic relatedness of these three subspecies, as well as a fourth subspecies on the Atlantic coast, and assessed whether genetic clusters correspond geographically with current subspecific designations. Data generated using double digest restriction site associated DNA sequencing (ddRADseq) suggests that the Florida endemics, A. m peninsulae and A. m. juncicola should be merged, but that the third gulf coast subspecies, A. m. fisheri, as well as the Atlantic coast subspecies (A. m .macgillivraii) are genetically distinct units.}

\normaltalk{\capitalisewords{Natal Dispersal Distances of Florida Scrub-Jays in an Expanding Population}}{\underline{Lauren M Deaner}, Reed Bowman}{Natal dispersal is an important aspect of demography that influences survival, reproduction, and genetic viability. Thus, understanding when and how far individuals disperse is helpful in developing species recovery plans. Our understanding of Florida scrub-jay (Aphelocoma coerulescens) dispersal has focused on movements in saturated populations in either contiguous or fragmented habitats. However, efforts to grow populations via habitat restoration and translocation have created the opportunity to study dispersal in growing populations in unsaturated habitat within a fragmented midland-island metapopulation structure. These data may improve our ability to use translocation as a tool for species recovery. While tracking the results of a translocation experiment, we recorded natal dispersal from more than 80 descendants of translocated jays in unsaturated habitat. We quantified distances by age and sex, and recorded how each breeding position was obtained (i.e., replacement, budding, inheritance, or de novo). We compared our results to studies conducted in saturated contiguous and fragmented habitats. Unlike other studies, natal dispersal distances did not differ by sex, were longer than those documented in contiguous saturated habitats, but shorter than those in fragmented saturated habitats. Median age at dispersal was younger suggesting that when breeding space was not limited, scrub-jays became breeders rather than helping, consistent with habitat saturation models for the evolution of cooperative breeding. Frequency of establishing a de novo territory was greater in unsaturated habitat versus saturated habitats. Our data suggests that Florida scrub-jays may successfully disperse farther than previously thought and at an earlier age when colonizing suitable, unsaturated habitat.}

\normaltalk{\capitalisewords{Migration, Home Range, and Dispersal of Adult and Juvenile Northern Harriers (Circus Hudsonius) in Suisun Marsh, California}}{\underline{Shannon M Skalos}, Joshua M Hull, Michael L Casazza}{Northern Harriers are a ubiquitous raptor species found across wetland and grassland habitats in North America. Harriers are declining across much of their range and surprisingly little is known about their movement ecology across their annual cycle. Determining important migratory routes, stopover locations and dispersal across the annual cycle is crucial to informing harrier management and conservation. To understand harrier movement, we deployed GPS/GSM backpack transmitters on adult wintering females (n=10), adult breeding females (n=5) and fledgling females (n=9) in Suisun Marsh, California in 2018 (January – July). Wintering females departed Suisun in spring (February – April) and migrated to breeding areas across five western US states. One female migrated 20,953 km (round-trip) breeding north of the Arctic Circle in Alaska before returning to Suisun in November 2018, making this is the longest distance migration documented for a harrier. The remaining wintering females that returned to Suisun migrated an average of 1,798 km (round-trip, n=3). Females breeding in Suisun revealed an average home range size of 6.01 ha (95\% MCP, n = 4), and post-breeding season movements for both breeding adult females (n = 3) and fledgling females (n = 7) suggest residency in Suisun and the San Francisco Bay Delta area. These results reveal the presence of a resident and migratory population in Suisun, important migration corridors through the Sacramento Valley, wintering site fidelity in Suisun, and demonstrate targeted selection of remnant marsh, agricultural and grassland habitat for both breeding areas and stopover locations during migration.}

\normaltalk{\capitalisewords{Revisiting Cultural Evolution and Variation of Song in an Isolated Population of Mountain White-Crowned Sparrows}}{\underline{Caitlin M Sloof}, Scott F Lovell, M R Lein}{Songs are culturally inherited, and as a result geographically isolated populations may become culturally isolated over time, developing song variations which can produce a distinct population-wide dialect. This may be the case for the dialect of the Montane White-crowned sparrows (Zonotrichia leucophrys oriantha)(hereafter WCSP) isolated in Cypress Hills Provincial Park, whose next closest population is separated by 250km of unsuitable prairie habitat. Song dialects of WCSP have been a model system for investigations of the origin and function of intra-specific song variation since their description in the 1960s. Two previous studies found that the songs of WCSP in Cypress Hills had distinct changes between 1978 and 1999. We revisited this population to determine if the local song dialect has continued to change or if it has stabilized since 1978. We recorded songs of 21 males during the breeding season of 2018.  We measured temporal, frequency, and qualitative variables of songs and conducted univariate and multivariate statistical analyses to characterize variation within and among the individual males.  Additionally, measurements from songs of 2018 males were compared to song measurements of 41 males recorded in 1978 to test for song differences.  The continuous and relatively rapid change in song pattern suggest the Cypress Hills WCSP population is not only geographically isolated, but culturally isolated as well. The data collected are currently being analyzed.}

\normaltalk{\capitalisewords{Decadal Comparisons of the Nesting Ecology of McKay's Buntings Breeding on St. Matthew Island, Alaska}}{\underline{Rachel M Richardson}, Steven M Matsuoka, James A Johnson, Marc D Romano, Audrey R Taylor}{The McKay's Bunting (Plectrophenax hyperboreus) is one of the rarest and least studied passerines in North America. Endemic to the Bering Sea region in Alaska, the entire population of ca. 31,000 individuals breeds on two uninhabited and remote islands, St. Matthew and Hall, and winters along the western Alaska coast. The species' restricted range and small population size make it a high priority for research, monitoring, and conservation. In 2018, we examined the nesting ecology of McKay's Buntings breeding on St. Matthew Island with the goal of comparing current breeding chronology, nest survival rates, and fecundity levels to previous estimates from 1982 and 2003. We monitored 71 nests for initial clutch sizes, number of eggs hatched, number of young fledged, nest fates, and evidence for nest failures. Our preliminary results suggest high fledging rates in 2018 with low numbers of nest failures associated with predation by red fox (Vulpes vulpes) and singing vole (Microtus abbreviatus). In our poster, we will also examine trends in McKay's Bunting nesting chronology over time in the context of long-term ecological changes occurring in the Bering Sea region.}

\normaltalk{\capitalisewords{Adaptive Fat Regulation in the Mixed Arthropod-Fruit Diet of a Neotropical-Nearctic Migrant on the Wintering Grounds}}{\underline{Joseph M Wunderle}, Jr, Patricia K Lebow, Sarah M Rockwell, Adele Powell, Stuart Bearhop, Jennifer D White, Dave Currie, David N Ewert, Peter P Marra}{Size-corrected body mass and fat scores of Kirtland's Warblers decline with increasing midwinter (Jan. - Feb.) rainfall in a response consistent with adaptive fat regulation in which birds adjust their body condition to food resource predictability (e.g., Rogers 2005) as mediated by rainfall. Responses to uncertainty of availability of different food types in midwinter further support the theory.  For example, if predaceous arthropods are less predictable in time and space than fruits in midwinter, then birds foraging for predaceous arthropods will increase body condition in response to uncertainty of encountering arthropod prey.  Conversely, if fruit availability is more predictable (fruit displayed prominently and with sequential ripening) than predaceous arthropod availability, foragers will reduce fat and mass because of costs of fat-reserve maintenance. The hypothesis was assessed by correlating ranked residuals of body condition (corrected body mass or fat score) with ranked residuals of the proportion of the specified diet item (residuals for both adjusted for covariates).  As predicted, body mass and fat scores in midwinter (n = 23 KIWAs) showed positive associations with diet proportions of predaceous arthropods, but negative associations with Chiococca alba fruits, suggesting adaptive fat regulation in response to uncertainty of predaceous arthropod availability and predictability of C. alba fruit.  In late winter with same analyses of residuals, but larger sample (n = 32), neither body mass or fat scores were correlated with diet proportions of arthropods or fruits.  Thus, KIWAs maintain an arthropod-fruit diet before migration when a mixed diet may facilitate rapid body mass gain.}

\normaltalk{\capitalisewords{Miniature GPS Data Loggers Reveal Three Behavioral Strategies and Confirm the Habitat Use of Non-Migratory Veeries in South America}}{\underline{Christopher M Heckscher}, Devin Mendez, Armando A Aispuro, Alan H Kneidel, Melanie L Mancuso}{The habitat requirements of Veeries (Catharus fuscescens) during stationary periods of the non-breeding season is unknown because the species is often in transit between January and March rendering observations of habitat use by unmarked individuals untrustworthy. We used miniature global positioning system (GPS) data loggers (~1g) deployed at a Delaware, USA, site (2015 – 2018) to reveal the precise (~10m resolution) locations of male and female Veeries at first winter sites in South America. GPS fixes were obtained from within dense forest understory from remote and topographically diverse regions of the Brazilian Shield. All occupied sites were in apparent nutrient-poor forests of comparatively low to medium stature (e.g., campinarana) on the Brazilian Shield geologic formation. Among our subjects, three behaviors were evident: (1) A prolonged stationary period at a single site; (2) stationary periods at multiple sites separated by relocation events; (3) apparent continual unsettled movement. Corroborating our former research using geolocators, all Veeries undertook an intratropical migration to a second winter site prior to initiating Nearctic-Neotropical migration. Our results are the first to reveal the habitat associations of Veeries at known wintering sites and suggest that Veeries are largely dependent on uncommon stunted forests of the Brazilian Shield for a large portion of their non-breeding season. The use of naturally fragmented, uncommon, and threatened forest types in the southern Amazon Basin suggests the species is far more vulnerable than currently recognized. }

\normaltalk{\capitalisewords{Modifications to the Bownet Trapping Method to Increase Safety and Efficiency for Medium-Sized, Agile Birds}}{\underline{Luisa M Bergeron}, Corina Logan}{The goal of many ornithologists is to capture birds for research in a way that maximizes efficiency and safety of the target species.  Many trapping methods have trade-offs that aren't always well known.  Each researcher may endure significant trial-and-error, potentially causing injuries to their target species.  Researchers studying species that are notoriously hard to catch, or that live in more urban environments, may not be able to use well known and relatively safe trapping methods such as mist nets or walk-in traps.  The bownet trapping method has been used primarily for large-bodied species like raptors, or ducks on the nest.  However, there has been no previous documentation of the potential hazards of this method, or how to modify it for consistently safe deployment, especially for smaller birds. Here we will discuss bownet safety modifications used to trap wild Great-tailed Grackles - a medium-sized, agile passerine. We modified the bownet setup to prevent injuries to grackles by changing food placement within the trap, and exclusively using the remote trigger option.  Furthermore, we modified the moving part of the bow with foam to lessen the severity of injury if a bird does get hit after the trap is triggered.  We found it necessary to include these alterations to prevent injury to our bird subjects, and we believe it is important for other ornithologists to consider incorporating these or similar updates when using the bownet.  Broadly, we encourage more open publication of modifications to increase safety and efficiency of methods for trapping birds.}

\normaltalk{\capitalisewords{Predictability of the Predator, not Habituation to Humans, Drives Silver Gull Flight Initiation Distance}}{\underline{Stephen M Ferguson}, James I Barr, Philip W Bateman}{Animals must make escape decisions based on the perceived risk presented by potential predators. In populated environments flight initiation distances (FID) from approaching humans are often shorter, likely due to habituation to human presence.  We asked whether birds are able to discriminate between human approaches in different contexts and adjust their FID accordingly. Human access to Penguin Island (WA, Australia) is common and frequent, with multiple daily ferry trips; however, activity varies across three distinct environments: a boardwalk, beaches, and a native scrub wildlife sanctuary. We tested the FID of silver gulls (Chroicocephalus novaehollandiae) across each habitat, predicting the shortest FID on the boardwalk, where human approaches are frequent and follow a predictable path, intermediate FID on the beach, where human approaches are frequent but follow unpredictable paths, and the longest FID in the scrub, where human approaches are infrequent and follow unpredictable paths. We found that FID was shortest on the boardwalk, but did not differ between the beach and scrub. In addition, distance from the boardwalk did not affect FID in the scrub. We suggest that silver gulls use human path predictability in specific contexts, rather than general habituation to human presence, as a primary factor in making escape decisions.}

\normaltalk{\capitalisewords{Alaska Shorebird Conservation Plan, Version Iii}}{\underline{Christopher M Harwood}, H R Gates, James A Johnson, Richard B Lanctot, Daniel R Ruthrauff}{In recognition of declines among perhaps half of Alaska's breeding shorebirds, ongoing or emerging threats to shorebirds and their habitats, and considerable knowledge of Alaska's shorebirds acquired over the past decade, the Alaska Shorebird Group recently revised the Alaska Shorebird Conservation Plan. Similar to Version II (2008), we structured the plan in two parts: Part I describes Alaska's nearly 30 priority species, their conservation threats, and strategies to improve statewide conservation, and Part II considers these three elements for Alaska's five Bird Conservation Regions (Aleutian and Bering Sea Islands, Western Alaska, Arctic Plains and Mountains, Northwestern Interior Forest, Northern Pacific Rainforest). In addition to special recognition paid to species of greatest and high conservation concern, we included "Stewardship" species, defined as those species for which Alaska supports at least half of a population during the annual cycle. Climate change and severe weather, pollution, and energy- and mining-related activities ranked highest among conservation concerns in Alaska. Supplementing our tools for implementing conservation (research, inventory/monitoring, habitat management/protection, education/outreach, international collaboration), we introduced an evaluation of conservation progress to increase accountability. Based on considerable advances in tracking technologies largely unavailable prior to Version II, this plan stresses conservation approaches that recognize species' full annual cycles.}

\normaltalk{\capitalisewords{Incorporating Critical Bird Habitat and Climate Resilience in Argentina's Forest Use Plans}}{\underline{Anna M Pidgeon}, Natalia Politi, Guillermo Martinez-Pastur, Luis Rivera, Leonidas Lizarraga, Sebastian Martinuzzi, Volker C Radeloff}{Argentina encompasses 272,000 km2 of forest, and in 2007 established a national forest land use planning strategy (National Forest Law N 26331). However provincial and national governments initially incorporated biodiversity information minimally, if at all, when establishing zones of different allowable activities in their forests. We developed and implemented a straightforward approach to enhance the land-use plans for northwestern Argentina, based on the distributions of wilderness and species of conservation concern including the Tucuman Parrot (Amazona tucumana). The two northern provinces, Salta and Jujuy, incorporated these maps in their latest 5-year plan updates. Buoyed by this success, we are expanding efforts to all forested provinces, and expanding spatial analysis to characterize forest phenology and temperature variability from satellite imagery, as indicators of areas of high conservation value and that are thermal refugia. The expanded effort will incorporate occurrence and/or habitat quality data for up to 130 forest bird species, including five categorized by IUCN as Endangered, twenty as Near Threatened, and seven as Vulnerable. Integration of these ecosystem and biodiversity maps into outreach materials and presentations to national and regional land use planners will be our final and key step toward effective protection of critical habitat and sustainable forest management in Argentina.}

\normaltalk{\capitalisewords{Nest Site Characteristics of the Bahama Oriole: Habitat Requirements for a Critically Endangered Species}}{\underline{Briana M Yancy}, Cierra N McKoy, Janine M Antalffy, Kevin E Omland}{The Bahama Oriole (Icterus northopi) is a critically endangered species, restricted to Andros Island in the Bahamas.  This species was extirpated from Abaco Island in the 1990's for reasons unknown. Previous research suggested that the only tree species the Bahama Oriole used for nesting was the Coconut Palm (Cocos nucifera) in developed habitats. Recently, however, we documented the birds nesting in a native understory palm (Leucothrinax morrisii) in remote pine forests for the first time. We have documented many more nests in the pine forests. For each nest tree we measured the tree height, diameter and nest height. We also took measurements of habitat characteristics in a 10m-radius plot around the nest. These characteristics included: 1) number and height of thatch palms, 2) number and height of pines, 3) average height of the pine understory and 4) the burn history.  We then measured two randomly selected control plots 50 meters from the nest.  Our goal was to determine first if the birds are choosing habitats at random and if not, what specific habitat characteristics are important for nest sites in the Bahama Oriole.}

\normaltalk{\capitalisewords{Early Cottonwood-Willow Successional Forest Avian Diversity: What Do We Lose by Managing Sandbars for Plovers and Terns?}}{\underline{Stephanie M Nefas}, Mark Dixon, David Swanson}{In the Missouri National Recreational River (MNRR), riverine sandbars created by periodic floods are initially used by two federally listed bird species the Interior Least Tern (Stern antillarum athalassos) and the Piping Plover (Charadrius melodus). These sandbars then provide opportunities for recruitment and establishment of early successional cottonwood-willow forest, which supply habitat for a wide array of birds, as they become vegetated and unsuitable for Terns and Plovers to nest on. While nesting habitat for listed species can be maintained by vegetation removal, the biodiversity associated with an establishing riparian forest is relinquished.  An understanding of the current status trajectories and biological tradeoffs is necessary for a balanced ecosystem-based approach to sandbar management. We are assessing conditions and dynamics of riparian vegetation and land bird diversity during both breeding and migration seasons. My objective is to determine the degree that set-aside sandbars support diverse native plants, land birds, and cottonwood forest establishment; as they need to be actively contributing to the MNRR "Outstandingly Remarkable Values" as part of the Wild and Scenic rivers Act. In our first year of data collection we have begun to establish what species successfully nest, occupy, and use this habitat during migration. We have detected 82 different avian species within early successional cottonwood-willow forest habitat with an average nesting period survival of 26\% (0.94 ± 0.008 daily survival rate) for nesting species, appearing to be lower than expected.}

\normaltalk{\capitalisewords{Settling Differences: Factors Affecting Nest Size Variation in the Eastern Bluebird}}{\underline{Angelica N Reed}, Susan B McRae}{Within-population variation in nest size reflects differences in parental investment in building behavior. We investigated whether reproductive success or maternal condition were related to variation in nest size in a nest box breeding population of Eastern Bluebirds Sialia sialis. Breeding bluebirds were monitored through fledging, whereupon their nests were collected and weighed. Based on five years of data, we tested the prediction that nest weight was related to reproductive success (clutch size, hatching success and fledging success). Video observations revealed that nest building was exclusively conducted by mothers. Therefore, we also tested the prediction that nest weight is related to maternal condition (weight divided by wing length). After finding no significant relationships, we investigated alternative factors that may influence nest size. There was a seasonal effect: nest size was negatively correlated with lay date. In addition, nests built by immigrant females were significantly heavier than those built by females hatched on site. We will test the hypothesis that philopatric females have a 'natal memory' of nest size in relation to box dimensions by comparing their nest weights with that of their natal nest. In addition, we investigate whether there are thermal advantages to smaller nests due to the additional space above the nest enhancing box ventilation. We related mean maximum temperatures recorded in boxes with data loggers to nest size. Results of thermal analyses suggest that nest size significantly impacts maximum temperature in the box during incubation.}

\normaltalk{\capitalisewords{Grazing-Induced Trophic Relationships and Implications for Grassland Songbirds}}{\underline{Tracey N Johnson}, Sandra DeBano, Aaron Young, Patricia L Kennedy}{Grazing by cattle affects plant morphology, phenology, and species composition, which can result in changes within invertebrate communities at broad scales.  For grassland songbirds, altered invertebrate communities may influence prey availability or foraging preferences, and ultimately affect habitat use or demography.  Although grazing represents a complex disturbance, in managed systems stocking rate and associated vegetation utilization levels can determine vegetation responses to grazing, and potentially, habitat quality for other trophic levels.  To better understand the relationship between utilization and grassland bird food habits, we examined diet composition of nestlings for four songbird species breeding in paddocks with experimentally-manipulated stocking rates in a Pacific Northwest bunchgrass prairie in northeastern Oregon.  We compared nestling diet composition with invertebrate abundance in paddocks to examine the effect of utilization rates on dietary preferences of breeding birds.  Moths and butterflies comprised the largest proportion of prey in nestling diets (31\%) followed by short-horned grasshoppers (28\%) and spiders (20\%).  We observed variable preferences across invertebrate taxa, with strongest preference for moths and butterflies.  However, preferences in some cases were affected by vegetation utilization rate by cattle, and patterns suggest a prey-switching strategy depending on grazing management.  However, this compensation may still come at a cost to nestlings if more time is devoted to foraging for specific prey by adults, and we will explore whether nestling growth rates were affected.}

\normaltalk{\capitalisewords{Description of Possible Hybrid Barn X Cliff Swallows in East Central Nebraska Identified via Anomalous Plumage Variation}}{\underline{Kayla N Kreizel}, Elizabeth Ewing, Lazaro Gonzalez, Alejandro Espino, Alex Koch, Lauren M Gillespie}{Research proposed here documents 12 individuals that appear to be possible hybrid barn (Hirundo rustica) X cliff swallows (Petrochelidon pyrrhonota) nesting under a bridge. Birds were caught passively as by-catch in mist-nets (September 2018) in East-Central Nebraska during a routine banding demonstration for general biology labs. Upon attempting to identify the species, we noticed facial and breast plumage amongst individuals observed varied significantly from those in identification guides. There were differences in streamer length and asymmetry as well as size and color variation in tail spots. Two types of nests are present: typical, open-cup barn swallow nests and colonially-grouped, tube-shaped nests typical of cliff swallows. Behavioral observations revealed the birds were utilizing both nest-types. When students played cliff swallow calls, birds responded territorially by calling, swooping, and diving at the source of the sound. When birds return from spring migration to breed, they will be captured and blood sampled for both hormones and genetic testing and plumage samples will be taken. Streamers and facial plumage variation will be measured, photographed, and quantified. An alternative hypothesis to species hybridization is environmental contamination disrupting physiology or influencing mutations as the breeding population is located above an agricultural drainage creek. Facial plumage abnormalities and those of streamer/spot asymmetry are consistent with documented genetic-based plumage anomalies in barn swallow populations exposed to environmental radiation in Chernobyl.}

\normaltalk{\capitalisewords{Offspring Sex Ratio Bias in a Pennsylvania Population of Gray Catbirds}}{\underline{Eric N Thompson}, Lucas J Redmond, Duncan Noble, Jordan I Haq, Michael Fleming, Vraj Patel}{In species that are characterized by differences in the maximum reproductive success achievable between males and females, it may be expected that selection would favor production of that sex which can have the higher reproductive success. In birds, female gametes determine the sex of offspring because females are the heterogametic sex and, in some species, manipulation of offspring sex ratio may be possible. When sex ratio manipulation has been identified, there is some evidence that indicates that a number of factors are associated with biases in the production of one sex over another such as individual quality and timing of the breeding season.  We documented offspring sex ratio in broods of Gray Catbird nestlings in eastern Pennsylvania. Catbirds are multiple-brooded songbirds that nest in relatively high density on our study site.  We used an information theoretic approach to evaluate the influence of clutch initiation date and morphology of parents on brood sex ratio of offspring. Morphology can act as an indicator of individual quality. Our analyses suggested the only variable related to brood sex ratio bias was female tail length; females with shorter tails laid clutches that contained more males. Tail length in catbirds, as in other species, increases with age, thus our results suggest that female catbirds bias the sex ratio of their broods towards males early in life.  Future directions of this study will address a potential physiological mechanism driving brood sex ratio bias caused by variation in circulating levels of sex hormones.}

\normaltalk{\capitalisewords{Using Species Distribution Modeling to Estimate the Distribution of the West Indian Whistling Duck (Dendrocyna Arborea) in Puerto Rico}}{\underline{Sheylda N Diaz-Mendez}}{The West Indian Whistling Duck (WIWD hereafter) is listed as a vulnerable species by the International Union for Conservation of Nature (IUCN 2015). No estimates of the population in Puerto Rico (PR) are available since 2008, but it is theorized that this species may eventually reach a threatened status if habitat conditions are altered. In the face of climate change and other factors that affect bird populations in the Caribbean, it is important to relate the occurrence of WIWD to a set of habitats, topographic and climatic variables. Using a spatially balanced, probabilistic approach, I used presence records reported in eBird from 2002-2018, and a Species Distribution Models (SDM) to determine probability of WIWD occurrence in Puerto Rico. The hypothesis states that vegetation density is key for habitat use, additional covariates include landcover, elevation, Normalized Differential Vegetation Index (NDVI) temperature, precipitation and distance to buildings (anthropogenic). The analysis by SDM is visualized using Geographic Information Systems (GIS). The result is a distribution model that predicts habitat selection of the WIWD. At this point, it will be important to discuss preliminary results and whether this distribution map for WIWDs in Puerto Rico predicts habitat use in a changing climate. Additionally, the potential for networking should be explored to yield a similar product for the distribution of WIWD throughout its entire Caribbean range in order to synchronize conservation efforts regionally for this regional endemic species.}

\normaltalk{\capitalisewords{Multispecies Song Playback Attracts Songbirds to Hardwood Forest Habitats Near and Far from Roads}}{\underline{Leanna N DeJong}, Dr Darren Proppe}{Songbirds often underuse or avoid newly restored habitats despite their high vegetative quality. High quality habitats found near roads are similarly underpopulated. These aversions may be due to neophobia or inability to locate intended habitats. However, birds are known to use conspecific presence and song as an indicator for good habitat. Thus, multi-species song playback is a promising method for attracting songbird species to quality habitats that are underutilized. We investigated whether avian territory establishment in Northern hardwood forest far from and near to low or moderate use roads increased in response to acoustic playback of songs from six migratory songbird species. We found that playback increased overall territory establishment both far from and near roads. Species-specific responses to playback, however, were variable and differed between remote and near road locations. These results suggest that song playback is a valuable tool for reestablishing avian communities, and could be useful in anthropogenically degraded habitats, such as those found near roads. However, further studies are needed in order to better understand the survivorship and reproductive success of lured birds as well as the nuances of differential species responses.}

\normaltalk{\capitalisewords{Seabird Responses and Vegetative Succession During Island Recovery}}{\underline{Christy N Wails}, Todd Landers, Chris Gaskin, Dave Towns, Holly P Jones}{Islands are biodiversity hotspots but are highly sensitive to disturbances, including the introduction of invasive, predatory mammals. While there have been considerable advances to the technology used to remove invasive mammals, little is known about the changes island communities experience during a passive recovery period. New Zealand – the world's seabird biodiversity hotspot – has cleared mammalian predators from $>$100 islands over several decades, providing a unique opportunity to quantify island recovery. Here, we examine seabird and vegetative changes at Pokohinu (Burgess) Island, predator-free for almost 30 years, of the Mokohinau Islands. In 2009, we documented seabird distributions, plant diversity, and soil depth across the island and performed smaller surveys in 2014 and 2019. We found a 62.5\% increase in seabird burrow densities with a strong spatial overlap of species distributions. Diversity of plant species also significantly increased, possibly due to the increased abundance of seed-dispersing birds and the simultaneous expansion of invasive plant species. In addition, we performed burrow and vegetation surveys at four other Mokohinau Islands without the same extent of disturbance as Pokohinu. Seabird densities between islands were comparable but plant communities lacked many of the invasive species found at Pokohinu. Overall, our results suggest that vegetative cover might not be limiting nesting habitat availability within the Mokohinaus, as seabirds are able to utilize areas dominated by dense vegetation. We emphasize the importance of understanding vegetative succession during recovery periods which can help explain recolonization rates for more specialist seabird species.}

\normaltalk{\capitalisewords{Prey Selection by Migratory Birds of Prey using DNA Barcoding Technique}}{\underline{Anisha Na Pokharel}, John P DeLong}{Diet analysis of birds of prey have been an interesting field of study since long time ago. Various methods including prey remains, pellet analysis, video recording, gut content analysis has been done over the period of time . Although many factors influence prey selection, it's been logistically challenging to collect more information on prey identity. Most of the studies have been conducted during the breeding period. However, almost 60\% of the birds migrate between their breeding and wintering ground every year which is an essential part  of their life cycle. This migration allows birds to track the food resources and favorable conditions and yet not much studies have been done about the prey selection during migration. Hence, I am interested in using DNA barcoding technique to assess the prey selection by migratory hawks. There is high energy demands during migration, so would that affect the prey selection? In this technique, we compare the genetic sequences from prey feathers or remains  trapped in the talons or beaks of hawks banded at Cape May banding station to the reference samples at GenBank to determine prey eaten. I am focusing mainly on Sharp-shinned hawks(Accipiter striatus) and Cooper's hawk (Accipiter cooperii) mainly because of the high number of migration along east coast so more possible samples. As per the preliminary results, European starling and Rock pigeon have been identified as  the most common prey of Cooper's hawk while American robin seems numerous in the diet of Sharp-shinned hawks. In addition, I will be comparing data of prey availability in the region with the prey selected by the hawks during migration. Despite the challenges, this tool can be very valuable in gaining insights on feeding ecology of these migratory hawks.}

\normaltalk{\capitalisewords{Possible Hybrid Barn X Cliff Swallows and How Colonial vs Solitary Nesting Behavior May be Related to Plumage Anomalies}}{\underline{Lisset O Oropesa}, Elizabeth Ewing, Lazaro Gonzalez, Kayla N Kreizel, Andres Espino, Lauren M Gillespie}{Research proposed here documents and proposes expanded investigation into individuals that may compose a potential hybrid barn (Hirundo rustica) X cliff swallow (Petrochelidon pyrrhonota) population. Birds caught passively as by-catch in mist-nets (September 2018) in East-Central Nebraska during a routine banding demonstration for general biology labs were observed to have plumage amongst individuals that varied significantly from identification guides with differences in melanin and carotenoid pigmented and structurally colored ornaments. Two types of nests are present under the bridge in East Central Nebraska where birds were found, both the typical, open-cup barn swallow nests and the colonially-grouped, tube-shaped cliff swallows' nests. Behavioral observations revealed birds with anomalous plumage were utilizing both nest-types and birds responded territorially to cliff swallow vocalizations however all birds observed have the overall body-morphology of barn swallows. Birds will return from migration between April-May (depending on weather) and both courtship and nest building behavior will be documented and audio data will be collected to ascertain relationships to plumage aberrations. If results show evidence of a hybrid or hybridizing population, conservation of this population will be important to ensure further investigations into the genetic basis of nest building behaviors, especially if relationships between melanin plumage and behavior are observed.}

\normaltalk{\capitalisewords{A Software Tool for Producing Annotations of Prolonged Audio Recordings using Machine Learning and Deep Neural Networks}}{Grady S Weyenberg, \underline{Lindsey OA Howells}, Madori D Rumpungworn, Kristina L Paxton, Lisa H Crampton, Patrick J Hart}{Prolonged passive acoustic monitoring of animals has recently become possible due to the affordability of automated recording devices. The information that can be extracted from these audio recordings is valuable for supporting scientists with a variety of research goals. However, large-scale processing of these recordings to extract desired information presents challenges. Conventional methods of manually annotating bird calls are time-consuming and prone to error. Methods that use machines to assist with the annotation pipeline would greatly improve the utility of passive monitoring programs.   We have developed a software package that streamlines the process of training deep neural networks to annotate environmental audio recordings for target sounds. The supervised learning process uses Raven-compatible selection tables to identify target vocalizations in the training data, and produces similar tables when annotating new recordings. Built-in models are capable of reaching human-like annotation accuracy on simple vocalizations and can be trained on commodity hardware in minutes. Customized network models can be easily added by users, if desired. Our software is provided as a Python package, and is available under an open source license.}

\normaltalk{\capitalisewords{The Effect of Supplemental Food Availability on Avian Space Use}}{\underline{Rachael P Mady}, David N Bonter}{Many birds have access to supplemental food, but the consequences of feeding wild birds are poorly understood. Researchers studying the effect of supplemental feeding have mostly compared birds with access to abundant and inexhaustible food to those without. In reality, supplemental food availability varies due to inconsistent human behavior, with many people refilling feeders intermittently. This creates different "feeding regimes," altering the characteristics of the resource and potentially altering bird behavior, particularly space use. To determine if variation in the bird-feeding efforts of humans alters avian space use, I designed an experiment wherein birds were exposed to constant, pulsed, or no supplemental food. I conducted 50 m radius point counts for three common bird species near (10 m) and far (160-200 m) from experimental feeders. I found no significant difference (z=0.410, p=0.68) in the number of birds detected near feeders with pulsed or constant food availability. However, I did find a significant interaction between food availability and distance from the feeder (z=2.51, p=0.01). On average, there were about the same number of birds detected near feeders with pulsed (8.72 ± 1.52 SE) or constant (8.42 ± 0.87 SE) food availability, but more birds were detected far from a constant feeder (1.76 ±0.65 SE) than far from the pulsed feeders with food (0.60 ± 0.34 SE). Feeding regimes seem to influence space use such that birds with access to a pulsed resource are more locally "anchored" when food is available compared to birds with access to a constant resource.}

\normaltalk{\capitalisewords{Syntactical Organization Within the Singing of Hermit Thrush (Catharus Guttatus) Males}}{Morgan E Nesbitt, Luke McLean, \underline{Sean P Roach}}{Hermit thrush (Catharus guttatus) possesses a song that is well known for its beauty, yet little studied. Males possess repertoires of 6 to 14 song types that they deliver in a non-random order, with individuals exhibiting immediate variety (not repeating the same song type consecutively) and cycling between high- and low-frequency song types. They are an ideal species in which to study questions related to the communicative functions of different song types and components, as well as the biological relevance of song syntax (i.e., how song types are arranged). However, information regarding syntactical organization within hermit thrush song is lacking. To address this, we used an approach centered on Markov models to carry out an in-depth analysis of song type sequences that we recorded from spontaneously singing males in an eastern Canadian hermit thrush population (C. g. faxoni). Results revealed that hermit thrush song exhibits greater than first-order dependencies, meaning that the identity of a given song type within a bout depends not just on the immediately preceding song type but also on the song type(s) preceding that. In addition, comparison of transition patterns using Fisher's Exact Tests showed that some song types within a repertoire may be equivalent to each other with respect to the song types they precede or follow, suggesting a degree of redundancy within repertoires. This characterization of hermit thrush song syntax provides a basis for future studies including assessment of the biological significance of song type ordering via playback experiments.}

\normaltalk{\capitalisewords{Slate-Colored Juncos (Junco Hyemalis) in Western Massachusetts}}{\underline{Daniel P Shustack}, Hannah L Wait, Jaret Gonzalez}{The Dark-eyed Junco (Junco hyemalis) is widespread songbird in North America. In eastern North America there are two subspecies within the "slate-colored" junco group. The carolinensis subspecies, which resides in the southern Appalachian Mountains, is known to occur as far north as western Pennsylvania, but the northern boundary with the hyemalis subspecies is unknown. We initiated banding of breeding, migrating and overwintering juncos in the Berkshires of western Massachusetts in order to determine if the carolinensis subspecies, or intergrades of hyemalis and carolinenis are found among the hyemalis subspecies known to occur here. To date we have only found hyemalis juncos across all seasons, and morphometric valuables generally overlap among juncos captured in different seasons. Interestingly, our winter captures dropped ~80\% from 2017-2018 to 2018-2019, which led us to investigate patterns in junco populations over broader areas and timescales.}

\normaltalk{\capitalisewords{Using GIS to Identify Sites for Installation of Barn Owl Nest Boxes}}{\underline{R P Dorn}, John W McArthur, Sarah A Boyle, Michael D Collins}{The Barn Owl (Tyto alba) has one of the largest geographic ranges of any bird. Despite their widespread distribution, generalist diet, broad habitat affinity, and tolerance of humans, many Barn Owl populations have declined over the past five decades. In Tennessee, the species is listed as in need of management. Although Barn Owls have been studied elsewhere, not much is known about this secretive species in western Tennessee. Throughout much of their range, breeding densities appear to be limited by the availability of suitable nesting cavities. Barn Owls readily adopt nest boxes, and studies have demonstrated that nest boxes can increase Barn Owl populations. We used ArcGIS, land cover data, eBird, and satellite imagery to identify ideal locations to place Barn Owl boxes. We generated 1.5 km buffers around each recorded Barn Owl sighting to model the approximate home range, and we used overlapping buffer areas and satellite imagery to locate barn-like structures. We identified 33 promising sites and have begun to work with landowners and land managers to install nest boxes on their property. Setting up nest boxes is the first step in a long-term project to understand the ecology and demography of Barn Owls in western Tennessee. We have begun to install boxes in barns and on free-standing poles, and we will begin to monitor nest boxes for occupancy and reproduction in March 2019. This research will fill gaps in our knowledge and will provide information to better manage for this declining species.}

\normaltalk{\capitalisewords{Graceanna Lewis (1821--1912) and the First Phylogenetic Tree of Birds}}{\underline{Matthew R Halley}}{Graceanna Lewis (1821–1912), who studied systematics under John Cassin (1813–69) at the Academy of Natural Sciences of Philadelphia, is rarely mentioned among the great pioneers of ornithology. The primary reason is that her most significant contribution to science was never published. More than 70 years before Hennig's theoretical formalization of phylogenetic systematics, Lewis reconstructed an evolutionary tree of all known birds (including extinct forms) by grouping taxa by their shared derived characters. Her phylogeny of birds is displayed here for the first time in more than a century. To my knowledge, Lewis was the first to reconstruct a phylogeny of all birds and therefore deserves to be recognized as the "Mother of Avian Phylogenetics."}

\normaltalk{\capitalisewords{Phylogenetic Relationships of Loon Species Based on ddRAD-Seq Data}}{Quentin D Sprengelmeyer, Ashish Gurung, Jeffrey C DaCosta, Michael D Sorenson, \underline{Alec R Lindsay}}{Loons (Aves: Gaviiformes) are an iconic lineage of waterbirds with only five extant species and a holarctic distribution. Historical hypotheses of relationships between the five species have differed on the putative sister relationship between Gavia arctica and G. pacifica, depending on whether data were based on morphological or mtDNA characters. We used ddRAD-seq to sample and sequence thousands of independent nuclear loci from multiple individuals of all five species along with relevant outgroups for phylogenetic analysis. Various concatenation and species tree analyses all strongly supported a phylogeny that agrees with the historic hypothesis based on overall morphological similarity (G. stellata, ((G. arctica, G. pacifica), (G. adamsii, G. immer))), and differed from the largely mitochondrially-based hypothesis that placed G. pacifica sister to a (G. immer, G. adamsii) clade. Estimates of divergence times placed the most recent common ancestor of modern loons in the early Miocene, which is earlier than previously thought. We discuss how these results impact our understanding of the evolutionary history and conservation status of loons.}

\normaltalk{\capitalisewords{Population Genomics Infer Different Modes of Speciation with Gene Flow in the Green-Winged Teal Complex}}{\underline{Fern R Spaulding}, Jessica F McLaughlin, Kevin G McCracken, Travis C Glenn, Brant C Faircloth, Kevin Winker}{We examine divergence and gene flow in three subspecies of the holarctic Green-winged Teal (Anas crecca crecca, A. c. nimia, and A. c. carolinensis) and the closely related South American Yellow-billed Teal (Anas flavirostris). We used ultraconserved elements (UCEs, 886 loci, with both alleles called) for genetic markers, using sequence data to reconstruct relationships among these taxa and one single nucleotide polymorphism (SNP) per locus for population genomic analyses. Best-fit demographic models for pairwise relationships were found using the program δaδi. Relationships among these taxa can be summarized as (crecca, nimia)(carolinensis, flavirostris), supporting a relationship previously found with mtDNA. Pairwise FST values seemed concordant with current taxonomy (crecca-nimia: 0.036, carolinensis-nimia: 0.040, crecca-carolinensis: 0.027, carolinensis-flavirostris: 0.258). The best demographic model for key pairwise comparisons was split-bidirectional-migration, indicating divergence and speciation with gene flow in all three major contrasts (crecca-nimia, crecca-carolinensis, and carolinensis-flavirostris). Given prior work, gene flow was expected among the holarctic taxa, but gene flow between North American carolinensis and South American flavirostris (m ~0.07/generation), albeit low, was not. Three modes of divergence are likely involved: heteropatric, parapatric, and (mostly) allopatric speciation. Ultraconserved elements are a powerful tool for simultaneously studying systematics and population genomics in systems like this.}

\normaltalk{\capitalisewords{Geographic Variation and Speciation in Pipilo Maculatus}}{\underline{Daniel R Wait}, Carla Cicero, Ali Malik, Rauri C Bowie}{Studies of molecular evolution and speciation are greatly advanced by the quantity and quality of biological information found in natural history museums. In particular, extensive geographic sampling of specimens, tissues, and songs from the same individuals allow researchers to study traits that interact and lead to geographic variation, reproductive isolation, and speciation. We studied variation in the Spotted Towhee (Pipilo maculatus), a common passerine in western North America, to investigate patterns and processes of variation in phenotypic, genetic, and acoustic traits. Spotted Towhees have diverged vocally into two distinct groups that differ in song structure and trill rate. Pacific slope populations sing simple songs with relatively fast trills, while Great Basin and Rocky Mountain populations sing more complex songs with slower trills. The objectives of the current study were to assess whether genetic and morphological patterns of divergence are congruent with vocal patterns. Our analyses included 406 individuals from 74 sites and two putative contact zones centered in the Owen's Valley of eastern California and the Snake River Plane of Idaho. In addition to investigating congruence between different sets of traits on a broad geographic scale, we conducted clinal analyses across the contact zones as a first step in evaluating taxonomic boundaries. We discuss these data in the context of genetic introgression and isolation between vocal groups.}

\normaltalk{\capitalisewords{Sex-Specific Variation and Flexibility in Migration Behavior for Alpine Horned Larks}}{\underline{Devin R de Zwaan}, Scott Wilson, Elizabeth A Gow, Kathy Martin}{Migratory connectivity is critical to understanding how birds move through and utilize the landscape across the full annual cycle. Flexibility in migratory behaviours can dictate how individuals and populations respond to changing climates and land uses. Using archival geolocators (GLS), we tracked the migration of 17 Horned Larks (Eremophila alpestris; Male = 8, Female =9) from an alpine population in northern B.C., Canada, to identify their non-breeding habitat and address within-population variation in migratory behaviours (i.e., stopover duration, distance). All individuals had short-distance migrations (mean ± SD; 1233 ± 215 km) and overwintered in eastern Washington and Oregon. Females moved both south and north along the coastal mountains (67\%), while 75\% of males travelled north through the eastern Interior. Stopover regions (± 100 km) were repeatable between seasons, particularly for females (F = 87.5\%, M = 50.0\%). Spring stopover duration was surprisingly long (average 31d; range 9–70 d). Females exhibited weaker migratory connectivity and stopover dependency than males, with about twice the variation in wintering locations and stopover sites. Finally, male migration behaviour was unrelated to reproductive effort, while females that fledged a brood departed earlier. Females that initiated more nests departed the breeding site later, but then increased migration speed and arrived on the wintering grounds at the same time as early-departing females. Our results indicate greater migratory flexibility in females than males. Since females are the limiting sex in most avian populations, an ability to respond to conditions during the non-breeding season is critical for population maintenance.}

\normaltalk{\capitalisewords{Coordinated Misdirection: An Intricate Behavior Likely to Distract Potential Nest Predators in the Neotropics}}{\underline{Eric R Gulson-Castillo}, Benjamin G Freeman, Harold F Greeney}{We discuss a stereotyped behavior that has likely arisen independently multiple times to prevent nest depredation. Beginning with A. Skutch, naturalists have sporadically reported Neotropical birds approaching nests in synchrony before one veers off to a side and the other arrives discretely. Observers often infer this behavior might distract potential visual nest predators if their eyes follow the flying bird away from the nest, as humans tend to do. We surveyed the literature, supplemented by our observations from the field, and found that at least 28 species from 11 genera and 5 families of passerines perform this behavior. We discuss variation in how the display is performed (e.g. the frequency, stereotyped roles by sex) and factors that could favor the appearance of the displace (e.g. enclosed nests, mate guarding). We propose calling this behavior "coordinated misdirection" because its presumed function disguises the location of the nest and it requires synchrony between at least two birds. Finally, we encourage future research and observations to better elucidate the nature of coordinated misdirection and the breadth of birds that might perform it.}

\normaltalk{\capitalisewords{Nest Survival in a Cooperatively-Breeding Bird: Do Helpers Mediate the Effects of Poor Environmental Conditions?}}{\underline{Ian R Hoppe}, Allison E Johnson}{Nest success is a vital component of fitness for individuals, and a major driver of population dynamics for many species. Ecological factors can play an important proximate role shaping annual patterns of nest success among bird populations. Globally, cooperatively-breeding bird species are distributed in harsh environments in which conditions undergo drastic and unpredictable changes from year to year. This unpredictability is thought to favor delayed dispersal by young, sexually mature individuals, who remain in their natal territory and aid in provisioning nestlings. Thus, the presence of helpers at the nest may act as a buffer, mediating the negative impacts of environmental fluctuations on the reproductive success of a dominant breeding pair. We monitored breeding behavior in Purple-backed Fairywrens (Malurus assimilis) over six seasons between 2012 and 2018, capturing a high degree of variation in both social environment and physical environment. Previous studies of breeding behavior in this population have used fledging success as a proxy for reproductive success. Here, we estimate daily nest survival as a finer-scale metric by which to evaluate the relative importance of (and potential interacting effects of) social and climatological factors as predictors of reproductive output.}

\normaltalk{\capitalisewords{Population Response to Extreme Temperatures Across North American Bird Ranges}}{\underline{Austin R Spence}, Jacob B Socolar, Andrew N Stillman, Morgan W Tingley}{Extreme temperature events are becoming more frequent as climate change increases the mean and variability of global temperatures. Despite strong evidence that temperature and precipitation affect avian distributions, we know very little about the effects of extreme temperature anomalies on bird populations and how responses may vary across a species' geographic range. We used the North American Breeding Bird Survey to calculate yearly population changes of breeding passerine birds across their ranges from 1980–2016. We used daily climate surfaces to calculate local breeding-season temperature anomalies for each year. We combined these temperatures with the following year's population and modeled the effect of these abiotic extremes on bird population change. We hypothesized that hot temperature anomalies will be worse for birds at the southern portion of their breeding distribution, as extreme hot temperatures here may push birds close to their physiological limits. Similarly, we hypothesized that extreme cold temperatures will be worse for birds at the northern portion of their breeding distribution, as extreme cold temperatures may push birds close to their physiological limits or may reduce food availability. This work will provide a better understanding of how populations from across the range will respond differently to extreme temperature events. As extreme temperature anomalies become more common with continuing climate change, it is imperative to understand how species will react at both the population level and at larger spatial scales.}

\normaltalk{\capitalisewords{The Northern Waterthrush: Analyzing the Distribution and Abundance of a Secretive Songbird in Pennsylvania}}{\underline{Justin R Clarke}, Terry L Master}{Parkesia noveboracensis (Northern Waterthrush, NOWA) drastically declined in the ~20 years between the first and second Pennsylvania Breeding Bird Atlas (PBBA). This is one of the largest declines of any Pennsylvania breeding species despite higher sampling effort in the second atlas. PPBA data suggested a slight northward range contraction and detectable increase in elevation of occupied blocks, potentially caused by climate change. This study refines the second PBBA distribution and investigates factors that may be responsible for any detected changes in distribution in the core breeding range in northeastern Pennsylvania. In spring of 2017 and 2018, wetland surveys were conducted for singing males. At each of 54 sites, two point counts were conducted to characterize the avian community. Vegetative, physical, and hydrological characteristics, as well as natural history and reproductive behavior, were also recorded. Sites occupied by NOWA were compared to unoccupied but expected-to-be-occupied sites based on the second PPBA or habitat characteristics. We found nineteen occupied blocks, three of six confirmed (breeding) atlas survey blocks overlapped, nine of the nineteen possible atlas blocks overlapped, and seven entirely new blocks. Plant communities (particularly shrubs and rooted aquatic plants) were found to be different between occupied and unoccupied sites. Climate indices (average precipitation and average, minimum, and maximum temperatures) were compared between atlas years during which the northern margin of the NOWA range moved south about ten km and the southern margin moved north about nine km.}

\normaltalk{\capitalisewords{An Inventory of Katmai National Park's Seabird Colonies}}{\underline{Kelsey R Griffin}}{Katmai National Park and Preserve (Katmai) in Alaska includes nearly 500 miles of coastline along the west side of Shelikof Strait, which provides nesting and nearshore foraging habitat for seabirds. The potential for oil and gas development and increased shipping traffic pose risks for oil spills and groundings that could affect nesting seabirds and their habitat. Many of Katmai's seabird colonies have not been inventoried since the early 90's. A series of boat-based surveys were conducted in the summer months of 2016-18 to revisit known colonies and document new colony locations. Species and count data will be summarized and compared to historical colony data. Updated colony data will aid park management in determining future monitoring efforts.}

\normaltalk{\capitalisewords{Survival and Non-Breeding Habitat Selection of Loggerhead Shrikes in Agricultural Landscapes of Arkansas}}{\underline{Emily R Donahue}, Lee C Bryant, Jacob L Wessels, Joseph A Youtz, Rhett Raibley, Alexander J Worm, Kevin J Krajcir, Amy A Chabot, Keith A Hobson, Than J Boves}{As intensive agriculture has transformed much of the landscape across central North America, many grassland birds have either disappeared from or declined in these new, highly homogeneous habitats. Despite this conversion, some species, like the Loggerhead Shrike (hereinafter LOSH), still persist in agricultural landscapes. The Lower Mississippi Alluvial Valley (LMAV) is dominated by intensive agriculture but still provides non-breeding habitat for likely co-occurring subspecies (one sedentary, Lanius ludovicianus ludovicianus, and one migratory, Lanius ludovicianus migrans). However, we lack information on the sustainability of these populations, and LOSH non-breeding ecology is understudied range-wide. Specifically, little is known about survival, site fidelity, habitat selection, and if these metrics differ between subspecies during this period of co-existence. To fill these gaps, we banded and monitored 72 LOSH in the LMAV of Arkansas (across 3800 km2) from 2016–2018. We used isotopic (δ2H) and genetic analyses to assign individuals to subspecies and assessed apparent survival, movements, and habitat selection patterns (with a focus on agricultural features). For individuals from both subspecies, we estimated apparent annual survival at 42\% and, of those that returned once, 44\% survived (apparent) until the third year. Site fidelity was high for returnees; 84\% returned to territories within 1 km of their original non-breeding sites. Multiscale analysis of habitat selection revealed availability of perches, in particular utility wires, and presence of agricultural ditches to be preferred features. Our results will assist in the development of improved management and conservation strategies at both local and global scales.}

\normaltalk{\capitalisewords{Habitat Characteristics of Henslow's Sparrow and LeConte's Sparrow Flush Points in a Small Blackland Prairie in Southeastern Texas}}{\underline{Richard R Schaefer}, D C Rudolph, J H Williamson, Josh B Pierce, Clifford E Shackelford}{Many small, isolated blackland prairies occur on the Sam Houston National Forest (SHNF) in southeastern Texas. Much of the prairie vegetation has disappeared or been heavily altered by fire suppression, grazing, development, and agriculture. A more recent threat comes from the invasive King Ranch bluestem (KR, Bothriochloa ischaemum) of the Old World. Prairie restoration efforts initiated on the SHNF in 2004 reestablished prairie plant communities at many sites. Several species of grassland-dependent sparrows overwinter on these prairies. Notable among them are LeConte's Sparrow (LCSP, Ammospiza leconteii) and Henslow's Sparrow (HESP, Centronyx henslowii). During the winters of 2012, 2013, 2014, and 2017, we measured habitat variables at each point where individual sparrows were initially flushed at nine prairies on the SHNF, ranging in size from 0.8 ha to 6.4 ha. We measured the same variables at random points in each prairie. Only one prairie (4.5 ha) produced sufficient numbers of both species during all years for analyses. We determined differences in prairie use by LCSP and HESP, based on habitat variables. Points with greater litter depth, greater herbaceous plant height, and a lower percentage of KR within the grasses were more likely to have been HESP points than random points. Points with greater herbaceous height, less canopy height, and reduced canopy closure were more likely to have been LCSP points than random points.  Points with greater canopy height, greater vertical vegetative cover between ground level and 0.5 m, less litter depth, and reduced canopy closure were more likely to have been LCSP points than HESP points.  The mean distance between HESP points (28.9 m) and the nearest prairie edge was significantly less than that for LCSP points (40 m, P = 0.001).}

\normaltalk{\capitalisewords{Hornbill Conservation in Sulawesi, Indonesia}}{\underline{Eric R Gulson-Castillo}, Pantiati , Nurul L Winarni, Andriansyah }{In collaboration with Burung Indonesia (the Indonesian BirdLife International partner), I built and placed artificial cavities/nest boxes in degraded forest to attract hornbills and increase their reproductive opportunities. Two hornbills are endemic to the Indonesian island of Sulawesi: Knobbed Hornbill (Rhyticeros cassidix) and Sulawesi Hornbill (Rhabdotorrhinus exarhatus). These birds are familiar to people living nearby on Sulawesi and can be considered as keystone species due to their role in long-distance seed dispersal. This gives them potential for being good flagship species for the conservation of Sulawesi's forests, especially with further outreach efforts. These birds nest in natural tree cavities, especially in large emergent hardwoods with heart rot. However, these trees are scarce in degraded habitat, especially near human habitation. We made our cavities in the province of Gorontalo with the hope that, with new nest sites, hornbill abundance might increase and they could aid in dispersing seeds. I will discuss conservation concerns evident during my short stay in Sulawesi (7 months). Major concerns involve illegal logging and gold mining. Hunting was minimal although it occurred. Burung Indonesia's local office is trying to attain permission to manage habitat corridor with the ultimate aim of restoring it. They are also engaging in bottom-up conservation, with a 'village facilitator' system that tries to improve villager quality of live. Finally, I will also share my experiences working through the Indonesian permit system, with local collaborators, and while hiring people for activities that were beyond my skill level.}

\normaltalk{\capitalisewords{Science Outside the Classroom: Birds 'n' Bogs Citizen Science Program as a Teacher/Scientist Partnership}}{\underline{Audrey R Taylor}, Sierra Doherty, Peggy Kristich, Barbara Johnson}{Citizen science (engaging non-professionals in authentic scientific inquiry) has proved to be an indispensable means of combining scientific research with education and public outreach. Citizen science can "push the envelope" of what scientists can achieve with limited funding, personnel, and access, and also provides a powerful tool for increasing scientific literacy and active citizenship. Traditionally, citizen science has been used with an adult population or families led by a parent, with less emphasis on K-12 students. However, engaging children of all ages and backgrounds in the scientific process could enhance students' sense of "science identity" and ultimately lead to a stronger STEM workforce. This project represents an extension of an existing citizen science project that has been running for seven years in Anchorage, Alaska. The project's goal is to provide information on occupancy of twelve species of declining boreal wetland birds, but past volunteers have been mostly birders with little involvement of students or teachers. In 2018, we began working with a group of teachers at Machetanz Elementary in Wasilla, AK, to design a protocol for the program that would be suitable for use with elementary school students. Our objective is to have students work with the program from kindergarten through 5th grade, with the older students training younger students in bird identification and data collection skills. Our poster will report on our progress in developing this project for deployment to students in 2019 as well as some of the challenges and lessons we have learned along the way.}

\normaltalk{\capitalisewords{Habitat Selection and Foraging Ecology of the Hawaiian Short-Eared Owl}}{\underline{Laura R Luther}, Chad Wilhite, Alba Melero, Melissa R Price}{Fifty-two percent of raptor species have declining global populations, but for many species, a lack of baseline surveys hampers the ability to determine the expected population size and optimal foraging habitat for a given region. In this study, we determined the expected number of Short-eared owls (Asio flammeus sandwichensis or pueo) for the Hawaiian Islands, based on available prey biomass. Also, using estimates of plant cover and vegetation height, we compared the distribution of foraging efforts across land types on which pueo have been detected hunting or nesting in recent studies: native vegetation, wetlands, pastureland, shrubland and agriculture land. Although greater vegetative cover supported higher prey abundance, the Hawaiian Short-eared owl hunted more often in habitat with sparser vegetation that contained lower prey abundance. A negative correlation between estimates of plant cover and foraging suggested that accessibility of prey was of greater importance than biomass in the selection of hunting sites. Although high prey biomass across these systems indicate Hawai'i could support a large Short-eared owl population, environmental variables such as vegetation structure, perch sites and land use, as well as increased mortality due to anthropogenic causes, may constrain the raptor's population size.}

\normaltalk{\capitalisewords{Is Ecological Specialization Reflected in Metabolic Network Complexity? a Test with 250 Bird Species}}{\underline{Erin S Morrison}, Alexander V Badyaev}{To what extent does the complexity of a metabolic network reflect its current function vs. its function in the organism's past environments? Does specialization of a metabolic network to a particular environment trade-off with its ability to perform with a variety of inputs? We address whether a diet-dependent metabolic network co-evolves with ecological specialization by examining ecological correlates of carotenoid metabolic network diversity across 250 bird species. Production of carotenoid coloration in birds requires initial consumption of external carotenoids which are metabolically converted into plumage carotenoids. Thus, the starting points of carotenoid metabolism vary among ecological groups of species. We first categorized species based on diet and diet specialization. We then identified dietary carotenoids that were the starting points for each dietary classification and examined structural differences in metabolic networks between ecological groups. We specifically tested the relative contribution of dietary inputs, biochemical network connectivity, taxonomic relatedness, and ecological convergence to interspecific variability in the products of color-producing metabolic networks. The results of this study shed light on the role of metabolic network structure in facilitating adaptation to novel environments and resources.}

\normaltalk{\capitalisewords{Heteropatric Speciation in Birds}}{\underline{Kevin S Winker}}{Within migratory lineages, different populations often have allopatric breeding ranges but are sympatric during parts of the annual cycle. Despite the increased propensity for gene flow that these heteropatric distributions cause, many migratory populations diverge anyway, and without prolonged isolation. Divergence grows through disruptive selection from competition for cyclic resources that are heterogeneously distributed in space and time. Heteropatric speciation is a form of ecological speciation and speciation with gene flow in which reproductive isolation increases between populations as a byproduct of adaptation to different environments, enhancing breeding allopatry and allochrony despite seasonal sympatry. Theoretical and empirical studies suggest that it may be common among seasonally migratory organisms.}

\normaltalk{\capitalisewords{Predation of the Bahama Oriole (Icterus Northropi) by Rats (Rattus Rattus) on Andros}}{\underline{Matthew S Kane}, Michael Rowley, Briana Yancy, Janine Antalffy, Kevin Omland}{In May of 2018 the Omland Lab from UMBC traveled to Andros for the Bahama Oriole Project, a cooperative effort between UMBC and the BNT to study the critically endangered Bahama Oriole (Icterus northropi). While there, we surveyed the population of a potential nest predator, the Black Rat (Rattus rattus). The Black Rat is a globally invasive rodent that has played a role in the decline of island bird populations worldwide. By studying the rat population on Andros, we sought to gauge the potential threat Black Rats pose to the remaining Bahama Oriole Population.    For the survey, we utilized peanut butter Waxtags, a nonlethal pest monitoring tool. These tags contain a block of scented wax that rats bite, leaving behind distinct tooth marks. To determine the rat population density in developed areas versus Bahama Oriole habitat, we placed Waxtags in two sets of paired study sites. Each site pair consisted of a developed area next to an adjacent pine forest. After seven days of deployment, we detected no signs of rat activity. However, we did find several marks made by land crabs (Cardisoma guanhumi) native to our study site. Such results indicate that the rat population density in North Andros is likely very low. In several studies of rat populations, including populations on other islands, peanut butter WaxTags were an effective method of detection. Thus, our lack of detection indicates that Black Rats are unlikely to be majors predators of the Bahama Oriole.}

\normaltalk{\capitalisewords{Sex-Based Difference in Direct Benefits Gained by Immigrants in a Cooperative Breeding Bird with Low Group Relatedness}}{\underline{Shailee S Shah}, Dustin R Rubenstein}{The prevailing theory of kin selection cannot explain why 45\% of all cooperatively breeding birds form groups in which not all individuals are closely related. Direct benefits such as access to food or breeding opportunities must outweigh the costs of group living for these groups to remain stable. We investigated direct benefits gained by immigrants in social groups of superb starlings (Lamprotornis superbus). Superb starlings are plural cooperative breeders forming large groups (mean = 22.6) with low relatedness (mean = 0.12). Both sexes disperse, and both immigrant males and females, as well as natal males delaying dispersal, breed. Using 14 years of data from 9 groups, we examined factors governing the likelihood of immigrants breeding in their first year. We found that first-year immigrant females were more likely to breed with higher rainfall preceding the breeding season. Though a previous study showed that the proportion of first-time male breeders increased with increase in pre-breeding rainfall, we found that proportion of first-time immigrant or natal male breeders did not affect the likelihood of first-year female immigrants breeding. First-time immigrant male breeders, however, were more likely to pair with a more recent female immigrant. Accordingly, direct benefits – here, breeding opportunities – for females are likely governed by environmental factors affecting condition, whereas immigrant males are likely limited by competition from natal males and/or mate availability. Thus, while unrelated individuals of both sexes comprise superb starling social groups, the direct benefits they gain from group membership are governed by different environmental and social factors.}

\normaltalk{\capitalisewords{Exploring the Physiological Condition of Migratory Songbirds Near Lake Ontario: New Insights from the Gut Microbiota}}{\underline{Susan S Pagano}, Nicole Cavanaugh, Andre' O Hudson, Molly Border}{Annual migrations are energetically challenging and potentially stressful periods for birds, particularly near ecological barriers where birds require high-quality stopover sites for rest and energy replenishment. Information about the nutritional and health status of birds, and their refueling performance during stopovers, may be useful for identifying important habitats and resources to support migrants. Examination of the gut microbiota of birds during stopover may provide additional insight into the role of intrinsic and environmental factors that may influence stopover behavior and habitat use. This project focused on multiple species of migrating passerines during autumn stopover near the south shore of Lake Ontario. We took blood samples and cloacal swabs from birds captured at the Braddock Bay Bird Observatory, and we assessed physiological condition of birds using plasma indices (triglyceride, uric acid, total plasma protein, and glucose concentrations) and chronic stress levels via leukocyte counts (heterophils/lymphocytes), in addition to body condition index. Cloacal samples were cultured and individual colonies were isolated to amplify the V3/V4 region of the 16S rRNA gene, followed by nucleotide sequencing to identify bacterial genera. The gut microbiota were predominantly Proteobacteria and Firmicutes, with fewer Actinobacteria. The microbiota composition of thrush and sparrows overlapped in Proteobacteria (20\%), but were otherwise distinct. Overall chronic stress levels appeared to be low at this site and some thrush utilized the area for relatively long refueling periods. Follow-up study will incorporate information on diet and fruit quality.}

\normaltalk{\capitalisewords{The Influence of Habitat Characteristics on Northern Bobwhite Population and Spatial Ecology in Southern Illinois}}{\underline{Caleb S Crawford}, Michael W Eichholz}{Across North America, tallgrass prairie ecosystems have declined dramatically; consequently, many grassland nesting bird species have experienced rapid population declines in recent decades. Illinois was once dominated by native prairie, but now less than 0.01\% of tallgrass prairie habitat remains. Habitat loss, degradation, and fragmentation are the three main contributing factors causing declines in grassland bird species. The northern bobwhite (Colinus virginianus) has experienced population declines across much of its range and populations have continued to decline at a rate of 4.2\% per year. This study aims to understand how bobwhite populations are influenced by patch characteristics, habitat juxtaposition, and vegetation composition along with establishing reference data for a long-term study on the effects of restoration and hunting. Study sites include a focal area, Burning Star State Wildlife Management Area, and a private reference area. I will be attaching pendulum style radio transmitters and tracking bobwhites via radio telemetry to evaluate their movements and locate nests. I will monitor nests and measure vegetation characteristics to evaluate nest success and nest site selection. I will also conduct spring and fall call counts and habitat monitoring following NBCI protocol. Recently, emphasis has been placed on preserving and restoring bobwhite habitat and the need for broad scale management. Understanding how habitat characteristics influence bobwhite populations will be influential in recommending future restoration efforts and management practices aimed at increasing bobwhite populations.}

\normaltalk{\capitalisewords{Habitat Selection Parameters for the Coastal California Gnatcatcher: Finding the Sweet Spot}}{\underline{Clark S Winchell}, Paul F Doherty, Jr}{Management actions designed to protect viable populations of avian species often focus on vegetation restoration and are generally assumed to benefit the avian species of concern. Yet, restoration activities are often uninformed by avian survey data, an omission that can result in restoration of areas that do not maximize bird population viability.  Using the California gnatcatcher (Polioptila californica californica) as a case example, we demonstrate how animal survey data can help restoration activities achieve the goal of improving the viability of a species of concern by identifying habitat conditions with which the species most closely associates.  We found that gnatcatcher presence and colonization probabilities were maximized when coverage of California sagebrush (Artemesia californica) was near 40 percent and that California sagebrush is most closely associated with soil texture described as between 5\% to 20\% clay, 40\% to 70\% sand, and 20\% to 40\% silt, and was more likely to be present at lower elevations.  We also found that higher gnatcatcher extinction probabilities were associated with closed, dense, habitat, with optimal conditions between 30-40\% openness creating a slightly closed canopy.  Open habitat was associated with southern aspects, shallow slopes, and inland areas.  Understanding soil types and physical parameters of the environment affecting vegetation, especially the amount of sage needed for high gnatcatcher occupancy and colonization rates, will help define restoration target conditions.  Identifying desirable habitat conditions within an ecosystem allows ornithologists and land managers to design more powerful restoration prescriptions.}

\normaltalk{\capitalisewords{Woodpeckers Frequently Use Forest Habitat Damaged by Tornado for Foraging, but not Nesting}}{\underline{James S Kellam}}{An EF1 tornado bisected the forest at Powdermill Nature Reserve (Rector, Pa., USA) in 2012. In response, researchers began a long-term study of how the forest would recover, comparing sections where damaged timber was salvaged to sections that were left in the natural state. Five years after the event, I surveyed the unsalvaged locations to determine if woodpecker species were more likely to use the disturbed forest at higher or lower frequencies compared to intact stands unaffected by the tornado. Point counts were conducted along each of three 1 km tracks during June 2017. Combining data from all five species of woodpeckers at the study site, I found that woodpeckers were more often found near the open canopy of the disturbed forest than in intact forest adjacent to the tornado track. I also conducted extensive nest searches in the area and found that woodpeckers more commonly nested in the undisturbed sections of forest away from the tornado damage compared to sites along the tornado track. I conclude that catastrophic wind disturbance may subsequently attract woodpeckers as they forage on damaged timber as long as suitable nesting sites are still available in adjacent, undisturbed forest.}

\normaltalk{\capitalisewords{The Meaning of Song Structure: Exploring Behavioral Responses of Field Sparrows to Three Simple Song Types}}{\underline{Eliza S Foli}, Sharon A Gill, Maarten J Vonhof}{The frequency, rate, and overall quality of a male's song may indicate the singer's physical quality, either as a threat or as a potential mate. However, the degree to which a song must differ acoustically to communicate different information about the singer remains unclear. Analysis of male field sparrow simple song recorded in southwest Michigan revealed three structurally distinct clusters of simple song that differ in frequency characteristics. We hypothesize that songs in each cluster convey information about the singer and that receivers respond more strongly to songs more similar to their own song. To understand whether males distinguish among song clusters, we performed playback experiments on 32 males presenting simple song from the three clusters in a randomized order. We recorded behavior including number of songs, perch changes, and distance from the speaker before, during, and after song playback. We extracted two axes from a principal component analysis that reflected space use during trials (PC1) and song rate (PC2), and then ran linear mixed models to test whether males respond differently to songs from the different clusters and whether their own song influenced response. Males responded to songs from each cluster, as they changed behavior to playbacks, but males responded similarly to the song clusters. However, the male's own song appeared to influence his response, as males with narrow bandwidth songs tended to sing more during playback than males with broad bandwidth songs. This suggests that narrow bandwidth song may communicate male quality or aggression.}

\normaltalk{\capitalisewords{The Use of Trail Cameras to Identify Nest Predators of a Population of Gray Catbirds}}{\underline{Craig S Eroh}, Lucas J Redmond, Dominique M Varra}{In most bird populations, nest predation is the leading cause of nest failure, frequently resulting in failure rates that exceed 50\%. In many habitats, birds are faced with a diverse suite of avian, reptilian, and mammalian predators. Knowing the identity of nest predators could provide useful information, for example when making management decisions for species conservation. However, accurately identifying the type of predator that has depredated a nest after the event has occurred is fraught with difficulty. The use of remotely deployed camera systems solves issues related to predator identity and a growing body of literature has utilized this type of equipment. We used trail cameras placed near the nests of Gray Catbirds in an attempt to determine which nest predators were important contributors to nest failure on our eastern Pennsylvania student site. Beginning in the breeding season of 2018, we deployed trail cameras at twenty-two catbird nests and captured predation events by red squirrel, raccoon, and Cooper's and Sharp-shinned Hawks. Although our sample size was small, red squirrels were the most commonly recorded predator at catbird nests, suggesting that this species may be the most important nest predator on our study site. However, these results are only preliminary and more data will be needed to make definitive conclusions about which species/group of predators are primarily responsible for nest failure of Gray Catbirds at our study site.}

\normaltalk{\capitalisewords{Dietary Costs and Benefits of Lakeshore vs Aggregate Pit Breeding in Bank Swallows (Riparia Riparia)}}{\underline{Corrine S Genier}, Greg W Mitchell, Myles Falconer, Brian A Branfireun, Keith A Hobson}{Bank Swallows (Riparia riparia), a threatened species in Ontario, breed primarily in either banks at lakeshores or at exposed surfaces in inland man-made aggregate pits. Pits may be ecological traps for this species but the relative trade-offs in nesting at pits vs. natural sites are unknown. Availability of aquatic emergent insects is expected to be highest at lakeshore colonies with associated nutritional benefits including Omega-3 fatty acids. However, Bank Swallows may experience differential mercury exposure depending on habitat use, with higher expected levels at lakeshores and among adult birds. Potential differences in dietary quality among sites may directly influence juvenile body condition, with higher Omega-3 fatty acids being beneficial. This study seeks to compare these breeding habitats to evaluate dietary differences as revealed by fatty acid of blood plasma, stable isotope analyses in juvenile feathers, DNA barcoding of fecal matter, and mercury levels in feathers and blood. In 2017, we found lower feather δ2H at lakeshore sites, indicating more aquatic emergent diet. Despite the dietary difference of juveniles, there was no difference in mercury exposure at pits and lakeshores. Adults however experienced a 4 fold increase in mercury on wintering grounds compared to breeding grounds. This information will be important for management decisions related to the use of inland aggregate pits by this species and conservation of suitable nesting habitats.}

\normaltalk{\capitalisewords{Barred Owl Habitat Use Across an Urban Gradient}}{\underline{Sabrina S Taylor}, Vitek Jirinec, S T Williams, Alessandra M Bresnan, Patricia F Rodrigues, Anna A Perez Umphrey, Cameron L Rutt, Ashley Long, Philip C Stouffer}{Animals typically require food, water, shelter from weather, cover from predators, resting places, and places to raise their young; however, animals may meet these requirements in different areas of their home range. We examined habitat use in Barred Owls (n = 8) by deploying GPS tags (21-81 d for 63-243 locations) on individuals captured in urban areas ranging from contiguous forest to areas with high housing density. Preliminary results suggest that owls captured in contiguous forest seldom made forays into residential areas, but even those owls captured in residential areas often roosted in small, secluded clusters of trees in home range cores. Home ranges constructed using Brownian bridge kernels were usually clearly defined, averaging around 50 ha. Conversely, two owls had disjunct home ranges, occupying contiguous forest, but also repeatedly flying up to 11 km to occupy residential areas with little use of areas in between. These results suggest that owls in contiguous forest are usually able to meet all of their requirements, but owls in residential areas need at least small clusters of trees in secluded areas for roosting.}

\normaltalk{\capitalisewords{Integrating Biology and Chemistry in an Outdoor Ornithology Learning Environment}}{\underline{Susan S Pagano}, Annemarie D Ross, Todd Pagano}{For current undergraduate students pursuing research and management careers and graduate studies, it is imperative to gain hands-on experience in the field with professional research techniques. Traditional classroom-based instruction in organismal biology and taxonomy courses is useful in broadening students' background in core topics, but provides limited experience in applied research skills that cross disciplines and place the material in the context of the local environment. Further, the integration of applied field research techniques into biology courses can be logistically prohibitive due to lack of adequate dedicated field space with equipment that can support on-going student projects and integrate research with teaching practices. This project aimed to establish a permanent field station on the Rochester Institute of Technology (Rochester, NY) campus that can support avian-related/environmental teaching and research activities for undergraduate students. It also sought to engage students in other programs, including underrepresented students (and especially, Deaf/Hard-of-Hearing students), in cross-disciplinary learning activities. We provide examples of successful laboratory modules that incorporated bird banding and invasive species mapping in the ornithology coursework. We also demonstrate the integration of the station across disciplines in other biology and chemical technology courses, as well as student collaboration across courses on applied questions related to animal nutrition and fruit biochemistry. We found that integrated activities at the station centered on avian and habitat conservation allowed students to gain inspiration from the campus environment, especially related to key issues that impact our ecosystems and sustainable practices on campus.}

\normaltalk{\capitalisewords{Toward a Complete Phylogeny of the Procellariiformes}}{\underline{R T Chesser}, Helen F James, Vincent Bretagnolle, Andrea Estandia, Max A Levy, Andreanna J Welch}{The Procellariiformes are by far the largest group of oceanic birds, but no complete phylogeny of the order is available. Most previous molecular studies of the group have been based on small amounts of data, generally sequences of the mitochondrial gene cytochrome-b, and have suffered from a lack of resolution owing to the small amount of sequence data, especially at deeper phylogenetic levels. In contrast, recent large-scale genomic studies of avian phylogenetics have necessarily included only a few representatives of the Procellariiformes.   We are using genomic sequence capture of UCEs to study evolutionary relationships of all extant and recently extinct Procellariiformes, sampling 2-4 individuals of all ca. 140 extant and recently extinct species. Our overarching goal is to develop a phylogenomic framework for studying the evolutionary and ecological history of oceanic birds. A well-resolved species- and subspecies-level tree will enable us to perform comparative studies of diversification and trait evolution in the Procellariiformes, which vary tremendously in body size, wing morphology, ecological niche, foraging behavior, and extinction risk. Preliminary data from selected species produced a well-resolved and well-supported backbone phylogeny, enabling us to address relationships among families and the relationships of problematical taxa such as the fulmarine petrels and the Kerguelen Petrel Aphrodroma brevirostris.}

\normaltalk{\capitalisewords{A New Priority Area for Endemic and Threatened Birds in the Atlantic Forest of Southeastern Brazil}}{\underline{Fabio u Schunck}, Lu\'{i}s F Silveira, Valder S Nascimento}{The Atlantic Forest (AF) biome encompasses a significant portion of the species of birds in Brazil (849), including 216 endemics and 122 threatened with extinction.  The identification of priority areas for conservation of birds is fundamental, but it is hampered in the AF by the small number of localities that have been studied systematically, and with published data. With the aim of turning this situation around, we selected a sector of the Serra do Mar called "Curucutu" (23°56'S; 46°39'W), situated near the metropolis of São Paulo, to conduct a field inventory together with an assimilation of published material, museum specimens and gray literature. This inventory has been in continuous operation for the past 16 years, with the simultaneous use of mist-nets and point counts, totaling 405 field days. Between 1900 and 2007, 338 species were recorded in the region, but only 278 were published in publicly available works, many of which are difficult to find. Between 2007 and 2019, we recorded 373 species in the field, bringing the total for the region to 422, including 128 AF endemics, 29 threatened, and 45 migratory species. This high species richness is supported by the variety of habitats as well as its 800 meter elevational gradient, partially protected in a state reserve. The greatest threats to the avifauna continue to be deforestation, hunting, and illegal capture of some species. Our data show that the Curucutu region ranks among the highest-diversity sites for birds in the AF of southeastern Brazil, and should be considered a high-priority area for conservation action.}

\normaltalk{\capitalisewords{Coordination, Conflict, and Neglect in Biparental Storm-Petrel Incubation}}{\underline{Liam U Taylor}, Robert A Mauck}{Leach's Storm-petrels (Oceanodroma leucorhoa) are obligate biparental breeders, with parents of both sexes working to incubate a single egg in a burrow usually located hundreds of kilometers away from their foraging grounds. Using empirical energetic, behavioral, and long-term demographic data of Leach's Storm-petrels on Kent Island, New Brunswick, Canada, we parameterize a set of stochastic simulations to investigate the mechanisms by which two parents can simultaneously coordinate their energetic demands with the developmental demands of their egg. We find that a basic "no overlap" incubation condition (with either orderly or random swapping between parents) improves simulated hatching success rates from ~10\% to ~99\% (well above empirical levels). This simple rule also improves the developmental condition of the egg and the energetic outcomes of adults throughout the season. Further, we find that the energetic cost of the egg to females has minimal impact on incubation success, suggesting other constraints on clutch size and sex-biased reproductive effort in this species. Given that energetic parameters from the incubation period alone result in overestimates of hatching success, we extend the model to test the ways in which additional egg neglect can emerge from (1) coordination for other parts of the reproductive season (such as chick brooding, a period with additional energetic costs) and (2) parent-parent conflict strategies that maximize lifetime reproductive success at the expense of individual offspring.}

\normaltalk{\capitalisewords{Habitat Features and Similarity in Nest Defense Behavior in Mated Pairs of Eastern Bluebirds}}{\underline{Elizabeth V Ewing}, Kayla N Kreizel, Andres Espino, Jenessa Grooms, Andrew Hurley, Lauren M Gillespie}{Local environment may shape behavior, and understanding regional variability in behavioral flexibility is crucial to anticipating how songbirds can survive and adapt to changing climates. Nest defense is adaptive as protecting reproductive investments are crucial to offspring survival and larger scale population growth, and territorial songbirds allow for important examination of behavioral phenotype flexibility. Eastern bluebird-mated pairs in Oklahoma and North Carolina correlate interspecific nest defense increasing reproductive success. We present results of nest defense trials from data collected in Mississippi using a decoy predator and examine relationships to local habitat features that may influence defense or vigilance behaviors. Bluebirds inhabiting two sites were exposed to simulated predator intrusions during the nestling phase. Proximity to abiotic habitat features at one site positively correlates with female aggressive behavior and females nesting distal to abiotic structures at this same site are more vigilant. Mated pairs at both sites show similarity in defense. We are replicating these methods in Eastern bluebirds breeding along the Loup River in Columbus, NE in the upcoming 2019 breeding season to examine describe and compare how breeding latitude and/or habitat characteristics correlate to defense behaviors in a Nebraskan population of bluebirds.}

\normaltalk{\capitalisewords{Peregrine Falcon Subspecies Phylogenomics Using Whole-Genome Re-Sequencing}}{\underline{Garrett W Meeks}, Clifford M Anderson, Douglas A Bell, Andrew Dixon, Tom Maechtle, Keiya Nakajima, Yoshinori Kumazawa, Ben J Novak, Sandra Talbot, Clayton White, Xiangjiang Zhan, A J Department of Biological Sciences, University of North Texas, Denton, TX, USA}{The current taxonomic status of Peregrine Falcon (Falco peregrinus) subspecies is one that is widely debated due to discordance between morphological traits (e.g., plumage) and molecular data. Despite morphological differences among subspecies, previous genetic work has shown limited differentiation as a result of either insufficient resolution of molecular markers used, contemporary gene flow, ancestral polymorphism, or incomplete lineage sorting (ILS). With the advent of next-generation sequencing, the cost of sequencing non-model organisms has dropped dramatically, allowing high resolution whole-genome re-sequencing to be more accessible in studies that require a large number of samples. To date, we have generated multiple whole-genome sequences per subspecies for 18 of 20 Peregrine Falcon subspecies, including the likely extinct F. p. furuitii. Nuclear and mtDNA genomes have been analyzed by employing methods to accommodate contemporary gene flow, ancestral polymorphisms, and ILS. The resulting mtDNA phylogeny showed little congruence with current taxonomy, however preliminary nuclear DNA phylogenies largely supported subspecies designations. Some important exceptions include support for F. p. pealei in North America, but limited support for F. p. tundrius and F. p. anatum subspecies, which may represent a clinal distribution rather than distinct taxonomic units.  Additional samples have been collected for further study, including several samples from difficult to obtain subspecies (F. p. submelanogenys and F. p. babylonicus). Ultimately, this study will possess the most comprehensive sampling of subspecies and the first using more than one nuclear locus for Peregrine Falcon phylogenetic reconstruction.}

\normaltalk{\capitalisewords{Climatic Niche Reconstructions Give New Insights to the Evolution of the New World Jays}}{\underline{Sumudu W Fernando}, Thilina N De Silva, John McCormack, Andrew T Peterson}{Evolution of species' climatic niches through time has become an area of intensive study; the ultimate goals being (1) to understand the associations between climatic niche evolution and lineage diversification, and (2) to provide ecological explanations for current species distribution patterns and predict how distributional patterns will be impacted by future climate changes. The New World jays (NWJs) form a well-studied bird lineage within the family Corvidae, however, little understanding exists of the effects of climate on the diversification process in this clade, despite the availability of large amounts of occurrence data, and well resolved phylogenetic hypotheses. In this study, we attempt to reconstruct ancestral climate niches of New World Jays (NWJs) by mapping temperature and humidity niches on to a comprehensive and well-resolved phylogenetic framework. We employed an improved methodology that explicitly incorporates the uncertainty associated with knowledge of climate niches of species, by considering an accessible area for the species, to avoid overinterpretation of niche changes throughout the evolutionary history. Our ancestral reconstructions indicated that humidity niches are more dynamic compared to temperature niches, and that different clades exhibit different degrees of niche conservatism versus niche dynamism. Speciation events were often associated with humidity niche changes, thus hinting at an association between humidity niche evolution and NWJ diversification. We attempted to relate niche dynamism in lineages to species' ecology, geological history, and distributional patterns. Thus, our results provide a synthetic view of ecological niche and its evolution over time in the history of the NWJs.}

\normaltalk{\capitalisewords{Prevalence and Diversity of Avian Malaria Parasites in a Raptor Community}}{\underline{John W McArthur}, Michael D Collins}{Haemospordian blood parasites (Plasmodium and Haemoproteus) are common and widespread parasites of birds. These parasites are transmitted by dipteran insects and are less formally known as avian malaria parasites. Avian malaria is a model system for understanding host-parasite interactions. Most studies of avian malaria have examined passerine hosts, and studies of the factors that influence the diversity and composition of parasite assemblages largely identify host community composition as a driving force. Here, we examine the diversity and composition of haemosporidian parasites of a raptor community. We collected 151 blood samples from 13 diurnal and nocturnal raptor species from the Orders Accipitriformes, Falconiformes, Strigiformes, and Cathartiformes. We extracted DNA and used PCR to amplify a fragment of the 16S rRNA gene to screen samples for infection. For positive infections, we will use nested PCR to amplify a fragment of the haemosporidian cyt b gene and sequence the product to identify particular parasite lineages. We predict that the raptor community will harbor a diverse community of haemosporidian parasites. Because parasite assemblage composition is often strongly influenced by host community composition, we also predict that the parasite assemblage of this raptor community will differ from the parasite assemblage obtained from a nearby study of over 600 individuals from 35 passerine species.}

\normaltalk{\capitalisewords{Autonomous Recording Units as a Tool for Understanding Resilience in Avian Communities}}{\underline{Michael W McCloy}, Selma Glasscock, Perry Barboza, Jacquelyn K Grace}{The concept of resilience, or the ability of a system to return to its original state after a disruption, is a critical component of understanding ecosystem processes. Changes in populations over time are relatively well-studied in North American bird communities, but the resilience capacity of these communities is not. In this study, we use autonomous recording units (ARUs) to investigate avian resilience in the face of disturbance- namely fire and flooding. ARUs remotely and efficiently monitor bird activity and occupancy patterns in a way that allows for greater spatio-temporal coverage than traditional point counts. ARUs were installed at the Welder Wildlife Foundation about 60km north of Corpus Christi, Texas to monitor acoustic bird activity during the 2018 breeding season. They were programmed to record for two hours post-sunrise and two hours pre-sunset each day to coincide with periods of traditionally highest bird activity. Recorded vocalizations were classified into species electronically, of which a subset were manually checked for accuracy before being assigned final species identifiers. Here we present preliminary results of avian occupancy and diversity on the Welder Wildlife Refuge using ARUs and explore how ARUs can be an effective conservation tool in the monitoring and management of bird populations.}

\normaltalk{\capitalisewords{Bird Surveys in Small Geographical Areas: Biases and Tradeoffs}}{\underline{Lloyd W Morrison}, David G Peitz}{Point counts are the most commonly employed methodology in bird surveys.  In cases where inferences to small geographical areas are desired, the traditional spacing of survey points (≥250 m) results in a relatively small sample size, which may produce an underestimate of species richness.  Alternatively, locating survey points closer together may produce more complete species lists, but result in excessive double counting of individuals, yielding biased population estimates.  We investigated this trade-off in five small national park units in which survey points were spaced at 100 m intervals.  The Chao2, 2nd-order Jackknife, ICE and ACE statistical species richness estimators indicated that, at four of the five parks, sampling at 100 m intervals captured approximately 75–80\% of the estimated species richness.  Approximately half of the species observed were rare, occurring at ≤3 sites. Approximately 1/3 of all birds were observed far enough from the survey point to be counted at another point.  Based on the rarefaction curves, increasing the distance between survey points from 100 to 200 m would result in a reduction of observed species numbers by 27 to 48\%.  Conducting multi-species bird surveys using a single visit, point count methodology over relatively small geographical areas and employing traditionally spaced survey points will likely result in biased underestimates of species richness, even with the use of statistical species richness estimators, and lack of knowledge of rare species identities.  Decreasing the spacing of points and increasing sample size may still not result in detections of many rare or cryptic species.}

